
\documentclass[11pt,letter]{article}


\begin{document}

\title{Menexenus\thanks{Source: https://www.gutenberg.org/files/1682/1682-h/1682-h.htm. License: http://gutenberg.org/license ds}}
\date{\today}
\author{Plato, 427? BCE-347? BCE\\ Translated by Jowett, Benjamin, 1817-1893}
\maketitle

\setcounter{tocdepth}{1}
\tableofcontents
\renewcommand{\baselinestretch}{1.0}
\normalsize
\newpage

\section{
      APPENDIX I.
    }
\par  It seems impossible to separate by any exact line the genuine writings of Plato from the spurious. The only external evidence to them which is of much value is that of Aristotle; for the Alexandrian catalogues of a century later include manifest forgeries. Even the value of the Aristotelian authority is a good deal impaired by the uncertainty concerning the date and authorship of the writings which are ascribed to him. And several of the citations of Aristotle omit the name of Plato, and some of them omit the name of the dialogue from which they are taken. Prior, however, to the enquiry about the writings of a particular author, general considerations which equally affect all evidence to the genuineness of ancient writings are the following: Shorter works are more likely to have been forged, or to have received an erroneous designation, than longer ones; and some kinds of composition, such as epistles or panegyrical orations, are more liable to suspicion than others; those, again, which have a taste of sophistry in them, or the ring of a later age, or the slighter character of a rhetorical exercise, or in which a motive or some affinity to spurious writings can be detected, or which seem to have originated in a name or statement really occurring in some classical author, are also of doubtful credit; while there is no instance of any ancient writing proved to be a forgery, which combines excellence with length. A really great and original writer would have no object in fathering his works on Plato; and to the forger or imitator, the 'literary hack' of Alexandria and Athens, the Gods did not grant originality or genius. Further, in attempting to balance the evidence for and against a Platonic dialogue, we must not forget that the form of the Platonic writing was common to several of his contemporaries. Aeschines, Euclid, Phaedo, Antisthenes, and in the next generation Aristotle, are all said to have composed dialogues; and mistakes of names are very likely to have occurred. Greek literature in the third century before Christ was almost as voluminous as our own, and without the safeguards of regular publication, or printing, or binding, or even of distinct titles. An unknown writing was naturally attributed to a known writer whose works bore the same character; and the name once appended easily obtained authority. A tendency may also be observed to blend the works and opinions of the master with those of his scholars. To a later Platonist, the difference between Plato and his imitators was not so perceptible as to ourselves. The Memorabilia of Xenophon and the Dialogues of Plato are but a part of a considerable Socratic literature which has passed away. And we must consider how we should regard the question of the genuineness of a particular writing, if this lost literature had been preserved to us.

\par  These considerations lead us to adopt the following criteria of genuineness: (1) That is most certainly Plato's which Aristotle attributes to him by name, which (2) is of considerable length, of (3) great excellence, and also (4) in harmony with the general spirit of the Platonic writings. But the testimony of Aristotle cannot always be distinguished from that of a later age (see above); and has various degrees of importance. Those writings which he cites without mentioning Plato, under their own names, e.g. the Hippias, the Funeral Oration, the Phaedo, etc., have an inferior degree of evidence in their favour. They may have been supposed by him to be the writings of another, although in the case of really great works, e.g. the Phaedo, this is not credible; those again which are quoted but not named, are still more defective in their external credentials. There may be also a possibility that Aristotle was mistaken, or may have confused the master and his scholars in the case of a short writing; but this is inconceivable about a more important work, e.g. the Laws, especially when we remember that he was living at Athens, and a frequenter of the groves of the Academy, during the last twenty years of Plato's life. Nor must we forget that in all his numerous citations from the Platonic writings he never attributes any passage found in the extant dialogues to any one but Plato. And lastly, we may remark that one or two great writings, such as the Parmenides and the Politicus, which are wholly devoid of Aristotelian (1) credentials may be fairly attributed to Plato, on the ground of (2) length, (3) excellence, and (4) accordance with the general spirit of his writings. Indeed the greater part of the evidence for the genuineness of ancient Greek authors may be summed up under two heads only: (1) excellence; and (2) uniformity of tradition—a kind of evidence, which though in many cases sufficient, is of inferior value.

\par  Proceeding upon these principles we appear to arrive at the conclusion that nineteen-twentieths of all the writings which have ever been ascribed to Plato, are undoubtedly genuine. There is another portion of them, including the Epistles, the Epinomis, the dialogues rejected by the ancients themselves, namely, the Axiochus, De justo, De virtute, Demodocus, Sisyphus, Eryxias, which on grounds, both of internal and external evidence, we are able with equal certainty to reject. But there still remains a small portion of which we are unable to affirm either that they are genuine or spurious. They may have been written in youth, or possibly like the works of some painters, may be partly or wholly the compositions of pupils; or they may have been the writings of some contemporary transferred by accident to the more celebrated name of Plato, or of some Platonist in the next generation who aspired to imitate his master. Not that on grounds either of language or philosophy we should lightly reject them. Some difference of style, or inferiority of execution, or inconsistency of thought, can hardly be considered decisive of their spurious character. For who always does justice to himself, or who writes with equal care at all times? Certainly not Plato, who exhibits the greatest differences in dramatic power, in the formation of sentences, and in the use of words, if his earlier writings are compared with his later ones, say the Protagoras or Phaedrus with the Laws. Or who can be expected to think in the same manner during a period of authorship extending over above fifty years, in an age of great intellectual activity, as well as of political and literary transition? Certainly not Plato, whose earlier writings are separated from his later ones by as wide an interval of philosophical speculation as that which separates his later writings from Aristotle.

\par  The dialogues which have been translated in the first Appendix, and which appear to have the next claim to genuineness among the Platonic writings, are the Lesser Hippias, the Menexenus or Funeral Oration, the First Alcibiades. Of these, the Lesser Hippias and the Funeral Oration are cited by Aristotle; the first in the Metaphysics, the latter in the Rhetoric. Neither of them are expressly attributed to Plato, but in his citation of both of them he seems to be referring to passages in the extant dialogues. From the mention of 'Hippias' in the singular by Aristotle, we may perhaps infer that he was unacquainted with a second dialogue bearing the same name. Moreover, the mere existence of a Greater and Lesser Hippias, and of a First and Second Alcibiades, does to a certain extent throw a doubt upon both of them. Though a very clever and ingenious work, the Lesser Hippias does not appear to contain anything beyond the power of an imitator, who was also a careful student of the earlier Platonic writings, to invent. The motive or leading thought of the dialogue may be detected in Xen. Mem., and there is no similar instance of a 'motive' which is taken from Xenophon in an undoubted dialogue of Plato. On the other hand, the upholders of the genuineness of the dialogue will find in the Hippias a true Socratic spirit; they will compare the Ion as being akin both in subject and treatment; they will urge the authority of Aristotle; and they will detect in the treatment of the Sophist, in the satirical reasoning upon Homer, in the reductio ad absurdum of the doctrine that vice is ignorance, traces of a Platonic authorship. In reference to the last point we are doubtful, as in some of the other dialogues, whether the author is asserting or overthrowing the paradox of Socrates, or merely following the argument 'whither the wind blows.' That no conclusion is arrived at is also in accordance with the character of the earlier dialogues. The resemblances or imitations of the Gorgias, Protagoras, and Euthydemus, which have been observed in the Hippias, cannot with certainty be adduced on either side of the argument. On the whole, more may be said in favour of the genuineness of the Hippias than against it.

\par  The Menexenus or Funeral Oration is cited by Aristotle, and is interesting as supplying an example of the manner in which the orators praised 'the Athenians among the Athenians,' falsifying persons and dates, and casting a veil over the gloomier events of Athenian history. It exhibits an acquaintance with the funeral oration of Thucydides, and was, perhaps, intended to rival that great work. If genuine, the proper place of the Menexenus would be at the end of the Phaedrus. The satirical opening and the concluding words bear a great resemblance to the earlier dialogues; the oration itself is professedly a mimetic work, like the speeches in the Phaedrus, and cannot therefore be tested by a comparison of the other writings of Plato. The funeral oration of Pericles is expressly mentioned in the Phaedrus, and this may have suggested the subject, in the same manner that the Cleitophon appears to be suggested by the slight mention of Cleitophon and his attachment to Thrasymachus in the Republic; and the Theages by the mention of Theages in the Apology and Republic; or as the Second Alcibiades seems to be founded upon the text of Xenophon, Mem. A similar taste for parody appears not only in the Phaedrus, but in the Protagoras, in the Symposium, and to a certain extent in the Parmenides.

\par  To these two doubtful writings of Plato I have added the First Alcibiades, which, of all the disputed dialogues of Plato, has the greatest merit, and is somewhat longer than any of them, though not verified by the testimony of Aristotle, and in many respects at variance with the Symposium in the description of the relations of Socrates and Alcibiades. Like the Lesser Hippias and the Menexenus, it is to be compared to the earlier writings of Plato. The motive of the piece may, perhaps, be found in that passage of the Symposium in which Alcibiades describes himself as self-convicted by the words of Socrates. For the disparaging manner in which Schleiermacher has spoken of this dialogue there seems to be no sufficient foundation. At the same time, the lesson imparted is simple, and the irony more transparent than in the undoubted dialogues of Plato. We know, too, that Alcibiades was a favourite thesis, and that at least five or six dialogues bearing this name passed current in antiquity, and are attributed to contemporaries of Socrates and Plato. (1) In the entire absence of real external evidence (for the catalogues of the Alexandrian librarians cannot be regarded as trustworthy); and (2) in the absence of the highest marks either of poetical or philosophical excellence; and (3) considering that we have express testimony to the existence of contemporary writings bearing the name of Alcibiades, we are compelled to suspend our judgment on the genuineness of the extant dialogue.

\par  Neither at this point, nor at any other, do we propose to draw an absolute line of demarcation between genuine and spurious writings of Plato. They fade off imperceptibly from one class to another. There may have been degrees of genuineness in the dialogues themselves, as there are certainly degrees of evidence by which they are supported. The traditions of the oral discourses both of Socrates and Plato may have formed the basis of semi-Platonic writings; some of them may be of the same mixed character which is apparent in Aristotle and Hippocrates, although the form of them is different. But the writings of Plato, unlike the writings of Aristotle, seem never to have been confused with the writings of his disciples: this was probably due to their definite form, and to their inimitable excellence. The three dialogues which we have offered in the Appendix to the criticism of the reader may be partly spurious and partly genuine; they may be altogether spurious;—that is an alternative which must be frankly admitted. Nor can we maintain of some other dialogues, such as the Parmenides, and the Sophist, and Politicus, that no considerable objection can be urged against them, though greatly overbalanced by the weight (chiefly) of internal evidence in their favour. Nor, on the other hand, can we exclude a bare possibility that some dialogues which are usually rejected, such as the Greater Hippias and the Cleitophon, may be genuine. The nature and object of these semi-Platonic writings require more careful study and more comparison of them with one another, and with forged writings in general, than they have yet received, before we can finally decide on their character. We do not consider them all as genuine until they can be proved to be spurious, as is often maintained and still more often implied in this and similar discussions; but should say of some of them, that their genuineness is neither proven nor disproven until further evidence about them can be adduced. And we are as confident that the Epistles are spurious, as that the Republic, the Timaeus, and the Laws are genuine.

\par  On the whole, not a twentieth part of the writings which pass under the name of Plato, if we exclude the works rejected by the ancients themselves and two or three other plausible inventions, can be fairly doubted by those who are willing to allow that a considerable change and growth may have taken place in his philosophy (see above). That twentieth debatable portion scarcely in any degree affects our judgment of Plato, either as a thinker or a writer, and though suggesting some interesting questions to the scholar and critic, is of little importance to the general reader.

\par 
\section{
      MENEXENUS
    }
\par 
\section{
      INTRODUCTION.
    }
\par  The Menexenus has more the character of a rhetorical exercise than any other of the Platonic works. The writer seems to have wished to emulate Thucydides, and the far slighter work of Lysias. In his rivalry with the latter, to whom in the Phaedrus Plato shows a strong antipathy, he is entirely successful, but he is not equal to Thucydides. The Menexenus, though not without real Hellenic interest, falls very far short of the rugged grandeur and political insight of the great historian. The fiction of the speech having been invented by Aspasia is well sustained, and is in the manner of Plato, notwithstanding the anachronism which puts into her mouth an allusion to the peace of Antalcidas, an event occurring forty years after the date of the supposed oration. But Plato, like Shakespeare, is careless of such anachronisms, which are not supposed to strike the mind of the reader. The effect produced by these grandiloquent orations on Socrates, who does not recover after having heard one of them for three days and more, is truly Platonic.

\par  Such discourses, if we may form a judgment from the three which are extant (for the so-called Funeral Oration of Demosthenes is a bad and spurious imitation of Thucydides and Lysias), conformed to a regular type. They began with Gods and ancestors, and the legendary history of Athens, to which succeeded an almost equally fictitious account of later times. The Persian war usually formed the centre of the narrative; in the age of Isocrates and Demosthenes the Athenians were still living on the glories of Marathon and Salamis. The Menexenus veils in panegyric the weak places of Athenian history. The war of Athens and Boeotia is a war of liberation; the Athenians gave back the Spartans taken at Sphacteria out of kindness—indeed, the only fault of the city was too great kindness to their enemies, who were more honoured than the friends of others (compare Thucyd., which seems to contain the germ of the idea); we democrats are the aristocracy of virtue, and the like. These are the platitudes and falsehoods in which history is disguised. The taking of Athens is hardly mentioned.

\par  The author of the Menexenus, whether Plato or not, is evidently intending to ridicule the practice, and at the same time to show that he can beat the rhetoricians in their own line, as in the Phaedrus he may be supposed to offer an example of what Lysias might have said, and of how much better he might have written in his own style. The orators had recourse to their favourite loci communes, one of which, as we find in Lysias, was the shortness of the time allowed them for preparation. But Socrates points out that they had them always ready for delivery, and that there was no difficulty in improvising any number of such orations. To praise the Athenians among the Athenians was easy,—to praise them among the Lacedaemonians would have been a much more difficult task. Socrates himself has turned rhetorician, having learned of a woman, Aspasia, the mistress of Pericles; and any one whose teachers had been far inferior to his own—say, one who had learned from Antiphon the Rhamnusian—would be quite equal to the task of praising men to themselves. When we remember that Antiphon is described by Thucydides as the best pleader of his day, the satire on him and on the whole tribe of rhetoricians is transparent.

\par  The ironical assumption of Socrates, that he must be a good orator because he had learnt of Aspasia, is not coarse, as Schleiermacher supposes, but is rather to be regarded as fanciful. Nor can we say that the offer of Socrates to dance naked out of love for Menexenus, is any more un-Platonic than the threat of physical force which Phaedrus uses towards Socrates. Nor is there any real vulgarity in the fear which Socrates expresses that he will get a beating from his mistress, Aspasia: this is the natural exaggeration of what might be expected from an imperious woman. Socrates is not to be taken seriously in all that he says, and Plato, both in the Symposium and elsewhere, is not slow to admit a sort of Aristophanic humour. How a great original genius like Plato might or might not have written, what was his conception of humour, or what limits he would have prescribed to himself, if any, in drawing the picture of the Silenus Socrates, are problems which no critical instinct can determine.

\par  On the other hand, the dialogue has several Platonic traits, whether original or imitated may be uncertain. Socrates, when he departs from his character of a 'know nothing' and delivers a speech, generally pretends that what he is speaking is not his own composition. Thus in the Cratylus he is run away with; in the Phaedrus he has heard somebody say something—is inspired by the genius loci; in the Symposium he derives his wisdom from Diotima of Mantinea, and the like. But he does not impose on Menexenus by his dissimulation. Without violating the character of Socrates, Plato, who knows so well how to give a hint, or some one writing in his name, intimates clearly enough that the speech in the Menexenus like that in the Phaedrus is to be attributed to Socrates. The address of the dead to the living at the end of the oration may also be compared to the numerous addresses of the same kind which occur in Plato, in whom the dramatic element is always tending to prevail over the rhetorical. The remark has been often made, that in the Funeral Oration of Thucydides there is no allusion to the existence of the dead. But in the Menexenus a future state is clearly, although not strongly, asserted.

\par  Whether the Menexenus is a genuine writing of Plato, or an imitation only, remains uncertain. In either case, the thoughts are partly borrowed from the Funeral Oration of Thucydides; and the fact that they are so, is not in favour of the genuineness of the work. Internal evidence seems to leave the question of authorship in doubt. There are merits and there are defects which might lead to either conclusion. The form of the greater part of the work makes the enquiry difficult; the introduction and the finale certainly wear the look either of Plato or of an extremely skilful imitator. The excellence of the forgery may be fairly adduced as an argument that it is not a forgery at all. In this uncertainty the express testimony of Aristotle, who quotes, in the Rhetoric, the well-known words, 'It is easy to praise the Athenians among the Athenians,' from the Funeral Oration, may perhaps turn the balance in its favour. It must be remembered also that the work was famous in antiquity, and is included in the Alexandrian catalogues of Platonic writings.

\par 
\section{
      PERSONS OF THE DIALOGUE: Socrates and Menexenus.
    }
\par \textbf{SOCRATES}
\par   Whence come you, Menexenus? Are you from the Agora?

\par \textbf{MENEXENUS}
\par   Yes, Socrates; I have been at the Council.

\par \textbf{SOCRATES}
\par   And what might you be doing at the Council? And yet I need hardly ask, for I see that you, believing yourself to have arrived at the end of education and of philosophy, and to have had enough of them, are mounting upwards to things higher still, and, though rather young for the post, are intending to govern us elder men, like the rest of your family, which has always provided some one who kindly took care of us.

\par \textbf{MENEXENUS}
\par   Yes, Socrates, I shall be ready to hold office, if you allow and advise that I should, but not if you think otherwise. I went to the council chamber because I heard that the Council was about to choose some one who was to speak over the dead. For you know that there is to be a public funeral?

\par \textbf{SOCRATES}
\par   Yes, I know. And whom did they choose?

\par \textbf{MENEXENUS}
\par   No one; they delayed the election until tomorrow, but I believe that either Archinus or Dion will be chosen.

\par \textbf{SOCRATES}
\par   O Menexenus! Death in battle is certainly in many respects a noble thing. The dead man gets a fine and costly funeral, although he may have been poor, and an elaborate speech is made over him by a wise man who has long ago prepared what he has to say, although he who is praised may not have been good for much. The speakers praise him for what he has done and for what he has not done—that is the beauty of them—and they steal away our souls with their embellished words; in every conceivable form they praise the city; and they praise those who died in war, and all our ancestors who went before us; and they praise ourselves also who are still alive, until I feel quite elevated by their laudations, and I stand listening to their words, Menexenus, and become enchanted by them, and all in a moment I imagine myself to have become a greater and nobler and finer man than I was before. And if, as often happens, there are any foreigners who accompany me to the speech, I become suddenly conscious of having a sort of triumph over them, and they seem to experience a corresponding feeling of admiration at me, and at the greatness of the city, which appears to them, when they are under the influence of the speaker, more wonderful than ever. This consciousness of dignity lasts me more than three days, and not until the fourth or fifth day do I come to my senses and know where I am; in the meantime I have been living in the Islands of the Blest. Such is the art of our rhetoricians, and in such manner does the sound of their words keep ringing in my ears.

\par \textbf{MENEXENUS}
\par   You are always making fun of the rhetoricians, Socrates; this time, however, I am inclined to think that the speaker who is chosen will not have much to say, for he has been called upon to speak at a moment's notice, and he will be compelled almost to improvise.

\par \textbf{SOCRATES}
\par   But why, my friend, should he not have plenty to say? Every rhetorician has speeches ready made; nor is there any difficulty in improvising that sort of stuff. Had the orator to praise Athenians among Peloponnesians, or Peloponnesians among Athenians, he must be a good rhetorician who could succeed and gain credit. But there is no difficulty in a man's winning applause when he is contending for fame among the persons whom he is praising.

\par \textbf{MENEXENUS}
\par   Do you think not, Socrates?

\par \textbf{SOCRATES}
\par   Certainly 'not.'

\par \textbf{MENEXENUS}
\par   Do you think that you could speak yourself if there should be a necessity, and if the Council were to choose you?

\par \textbf{SOCRATES}
\par   That I should be able to speak is no great wonder, Menexenus, considering that I have an excellent mistress in the art of rhetoric,—she who has made so many good speakers, and one who was the best among all the Hellenes—Pericles, the son of Xanthippus.

\par \textbf{MENEXENUS}
\par   And who is she? I suppose that you mean Aspasia.

\par \textbf{SOCRATES}
\par   Yes, I do; and besides her I had Connus, the son of Metrobius, as a master, and he was my master in music, as she was in rhetoric. No wonder that a man who has received such an education should be a finished speaker; even the pupil of very inferior masters, say, for example, one who had learned music of Lamprus, and rhetoric of Antiphon the Rhamnusian, might make a figure if he were to praise the Athenians among the Athenians.

\par \textbf{MENEXENUS}
\par   And what would you be able to say if you had to speak?

\par \textbf{SOCRATES}
\par   Of my own wit, most likely nothing; but yesterday I heard Aspasia composing a funeral oration about these very dead. For she had been told, as you were saying, that the Athenians were going to choose a speaker, and she repeated to me the sort of speech which he should deliver, partly improvising and partly from previous thought, putting together fragments of the funeral oration which Pericles spoke, but which, as I believe, she composed.

\par \textbf{MENEXENUS}
\par   And can you remember what Aspasia said?

\par \textbf{SOCRATES}
\par   I ought to be able, for she taught me, and she was ready to strike me because I was always forgetting.

\par \textbf{MENEXENUS}
\par   Then why will you not rehearse what she said?

\par \textbf{SOCRATES}
\par   Because I am afraid that my mistress may be angry with me if I publish her speech.

\par \textbf{MENEXENUS}
\par   Nay, Socrates, let us have the speech, whether Aspasia's or any one else's, no matter. I hope that you will oblige me.

\par \textbf{SOCRATES}
\par   But I am afraid that you will laugh at me if I continue the games of youth in old age.

\par \textbf{MENEXENUS}
\par   Far otherwise, Socrates; let us by all means have the speech.

\par \textbf{SOCRATES}
\par   Truly I have such a disposition to oblige you, that if you bid me dance naked I should not like to refuse, since we are alone. Listen then:  If I remember rightly, she began as follows, with the mention of the dead: —(Thucyd.)

\par  There is a tribute of deeds and of words. The departed have already had the first, when going forth on their destined journey they were attended on their way by the state and by their friends; the tribute of words remains to be given to them, as is meet and by law ordained. For noble words are a memorial and a crown of noble actions, which are given to the doers of them by the hearers. A word is needed which will duly praise the dead and gently admonish the living, exhorting the brethren and descendants of the departed to imitate their virtue, and consoling their fathers and mothers and the survivors, if any, who may chance to be alive of the previous generation. What sort of a word will this be, and how shall we rightly begin the praises of these brave men? In their life they rejoiced their own friends with their valour, and their death they gave in exchange for the salvation of the living. And I think that we should praise them in the order in which nature made them good, for they were good because they were sprung from good fathers. Wherefore let us first of all praise the goodness of their birth; secondly, their nurture and education; and then let us set forth how noble their actions were, and how worthy of the education which they had received.

\par  And first as to their birth. Their ancestors were not strangers, nor are these their descendants sojourners only, whose fathers have come from another country; but they are the children of the soil, dwelling and living in their own land. And the country which brought them up is not like other countries, a stepmother to her children, but their own true mother; she bore them and nourished them and received them, and in her bosom they now repose. It is meet and right, therefore, that we should begin by praising the land which is their mother, and that will be a way of praising their noble birth.

\par  The country is worthy to be praised, not only by us, but by all mankind; first, and above all, as being dear to the Gods. This is proved by the strife and contention of the Gods respecting her. And ought not the country which the Gods praise to be praised by all mankind? The second praise which may be fairly claimed by her, is that at the time when the whole earth was sending forth and creating diverse animals, tame and wild, she our mother was free and pure from savage monsters, and out of all animals selected and brought forth man, who is superior to the rest in understanding, and alone has justice and religion. And a great proof that she brought forth the common ancestors of us and of the departed, is that she provided the means of support for her offspring. For as a woman proves her motherhood by giving milk to her young ones (and she who has no fountain of milk is not a mother), so did this our land prove that she was the mother of men, for in those days she alone and first of all brought forth wheat and barley for human food, which is the best and noblest sustenance for man, whom she regarded as her true offspring. And these are truer proofs of motherhood in a country than in a woman, for the woman in her conception and generation is but the imitation of the earth, and not the earth of the woman. And of the fruit of the earth she gave a plenteous supply, not only to her own, but to others also; and afterwards she made the olive to spring up to be a boon to her children, and to help them in their toils. And when she had herself nursed them and brought them up to manhood, she gave them Gods to be their rulers and teachers, whose names are well known, and need not now be repeated. They are the Gods who first ordered our lives, and instructed us in the arts for the supply of our daily needs, and taught us the acquisition and use of arms for the defence of the country.

\par  Thus born into the world and thus educated, the ancestors of the departed lived and made themselves a government, which I ought briefly to commemorate. For government is the nurture of man, and the government of good men is good, and of bad men bad. And I must show that our ancestors were trained under a good government, and for this reason they were good, and our contemporaries are also good, among whom our departed friends are to be reckoned. Then as now, and indeed always, from that time to this, speaking generally, our government was an aristocracy—a form of government which receives various names, according to the fancies of men, and is sometimes called democracy, but is really an aristocracy or government of the best which has the approval of the many. For kings we have always had, first hereditary and then elected, and authority is mostly in the hands of the people, who dispense offices and power to those who appear to be most deserving of them. Neither is a man rejected from weakness or poverty or obscurity of origin, nor honoured by reason of the opposite, as in other states, but there is one principle—he who appears to be wise and good is a governor and ruler. The basis of this our government is equality of birth; for other states are made up of all sorts and unequal conditions of men, and therefore their governments are unequal; there are tyrannies and there are oligarchies, in which the one party are slaves and the others masters. But we and our citizens are brethren, the children all of one mother, and we do not think it right to be one another's masters or servants; but the natural equality of birth compels us to seek for legal equality, and to recognize no superiority except in the reputation of virtue and wisdom.

\par  And so their and our fathers, and these, too, our brethren, being nobly born and having been brought up in all freedom, did both in their public and private capacity many noble deeds famous over the whole world. They were the deeds of men who thought that they ought to fight both against Hellenes for the sake of Hellenes on behalf of freedom, and against barbarians in the common interest of Hellas. Time would fail me to tell of their defence of their country against the invasion of Eumolpus and the Amazons, or of their defence of the Argives against the Cadmeians, or of the Heracleids against the Argives; besides, the poets have already declared in song to all mankind their glory, and therefore any commemoration of their deeds in prose which we might attempt would hold a second place. They already have their reward, and I say no more of them; but there are other worthy deeds of which no poet has worthily sung, and which are still wooing the poet's muse. Of these I am bound to make honourable mention, and shall invoke others to sing of them also in lyric and other strains, in a manner becoming the actors. And first I will tell how the Persians, lords of Asia, were enslaving Europe, and how the children of this land, who were our fathers, held them back. Of these I will speak first, and praise their valour, as is meet and fitting. He who would rightly estimate them should place himself in thought at that time, when the whole of Asia was subject to the third king of Persia. The first king, Cyrus, by his valour freed the Persians, who were his countrymen, and subjected the Medes, who were their lords, and he ruled over the rest of Asia, as far as Egypt; and after him came his son, who ruled all the accessible part of Egypt and Libya; the third king was Darius, who extended the land boundaries of the empire to Scythia, and with his fleet held the sea and the islands. None presumed to be his equal; the minds of all men were enthralled by him—so many and mighty and warlike nations had the power of Persia subdued. Now Darius had a quarrel against us and the Eretrians, because, as he said, we had conspired against Sardis, and he sent 500,000 men in transports and vessels of war, and 300 ships, and Datis as commander, telling him to bring the Eretrians and Athenians to the king, if he wished to keep his head on his shoulders. He sailed against the Eretrians, who were reputed to be amongst the noblest and most warlike of the Hellenes of that day, and they were numerous, but he conquered them all in three days; and when he had conquered them, in order that no one might escape, he searched the whole country after this manner: his soldiers, coming to the borders of Eretria and spreading from sea to sea, joined hands and passed through the whole country, in order that they might be able to tell the king that no one had escaped them. And from Eretria they went to Marathon with a like intention, expecting to bind the Athenians in the same yoke of necessity in which they had bound the Eretrians. Having effected one-half of their purpose, they were in the act of attempting the other, and none of the Hellenes dared to assist either the Eretrians or the Athenians, except the Lacedaemonians, and they arrived a day too late for the battle; but the rest were panic-stricken and kept quiet, too happy in having escaped for a time. He who has present to his mind that conflict will know what manner of men they were who received the onset of the barbarians at Marathon, and chastened the pride of the whole of Asia, and by the victory which they gained over the barbarians first taught other men that the power of the Persians was not invincible, but that hosts of men and the multitude of riches alike yield to valour. And I assert that those men are the fathers not only of ourselves, but of our liberties and of the liberties of all who are on the continent, for that was the action to which the Hellenes looked back when they ventured to fight for their own safety in the battles which ensued: they became disciples of the men of Marathon. To them, therefore, I assign in my speech the first place, and the second to those who fought and conquered in the sea fights at Salamis and Artemisium; for of them, too, one might have many things to say—of the assaults which they endured by sea and land, and how they repelled them. I will mention only that act of theirs which appears to me to be the noblest, and which followed that of Marathon and came nearest to it; for the men of Marathon only showed the Hellenes that it was possible to ward off the barbarians by land, the many by the few; but there was no proof that they could be defeated by ships, and at sea the Persians retained the reputation of being invincible in numbers and wealth and skill and strength. This is the glory of the men who fought at sea, that they dispelled the second terror which had hitherto possessed the Hellenes, and so made the fear of numbers, whether of ships or men, to cease among them. And so the soldiers of Marathon and the sailors of Salamis became the schoolmasters of Hellas; the one teaching and habituating the Hellenes not to fear the barbarians at sea, and the others not to fear them by land. Third in order, for the number and valour of the combatants, and third in the salvation of Hellas, I place the battle of Plataea. And now the Lacedaemonians as well as the Athenians took part in the struggle; they were all united in this greatest and most terrible conflict of all; wherefore their virtues will be celebrated in times to come, as they are now celebrated by us. But at a later period many Hellenic tribes were still on the side of the barbarians, and there was a report that the great king was going to make a new attempt upon the Hellenes, and therefore justice requires that we should also make mention of those who crowned the previous work of our salvation, and drove and purged away all barbarians from the sea. These were the men who fought by sea at the river Eurymedon, and who went on the expedition to Cyprus, and who sailed to Egypt and divers other places; and they should be gratefully remembered by us, because they compelled the king in fear for himself to look to his own safety instead of plotting the destruction of Hellas.

\par  And so the war against the barbarians was fought out to the end by the whole city on their own behalf, and on behalf of their countrymen. There was peace, and our city was held in honour; and then, as prosperity makes men jealous, there succeeded a jealousy of her, and jealousy begat envy, and so she became engaged against her will in a war with the Hellenes. On the breaking out of war, our citizens met the Lacedaemonians at Tanagra, and fought for the freedom of the Boeotians; the issue was doubtful, and was decided by the engagement which followed. For when the Lacedaemonians had gone on their way, leaving the Boeotians, whom they were aiding, on the third day after the battle of Tanagra, our countrymen conquered at Oenophyta, and righteously restored those who had been unrighteously exiled. And they were the first after the Persian war who fought on behalf of liberty in aid of Hellenes against Hellenes; they were brave men, and freed those whom they aided, and were the first too who were honourably interred in this sepulchre by the state. Afterwards there was a mighty war, in which all the Hellenes joined, and devastated our country, which was very ungrateful of them; and our countrymen, after defeating them in a naval engagement and taking their leaders, the Spartans, at Sphagia, when they might have destroyed them, spared their lives, and gave them back, and made peace, considering that they should war with the fellow-countrymen only until they gained a victory over them, and not because of the private anger of the state destroy the common interest of Hellas; but that with barbarians they should war to the death. Worthy of praise are they also who waged this war, and are here interred; for they proved, if any one doubted the superior prowess of the Athenians in the former war with the barbarians, that their doubts had no foundation—showing by their victory in the civil war with Hellas, in which they subdued the other chief state of the Hellenes, that they could conquer single-handed those with whom they had been allied in the war against the barbarians. After the peace there followed a third war, which was of a terrible and desperate nature, and in this many brave men who are here interred lost their lives—many of them had won victories in Sicily, whither they had gone over the seas to fight for the liberties of the Leontines, to whom they were bound by oaths; but, owing to the distance, the city was unable to help them, and they lost heart and came to misfortune, their very enemies and opponents winning more renown for valour and temperance than the friends of others. Many also fell in naval engagements at the Hellespont, after having in one day taken all the ships of the enemy, and defeated them in other naval engagements. And what I call the terrible and desperate nature of the war, is that the other Hellenes, in their extreme animosity towards the city, should have entered into negotiations with their bitterest enemy, the king of Persia, whom they, together with us, had expelled;—him, without us, they again brought back, barbarian against Hellenes, and all the hosts, both of Hellenes and barbarians, were united against Athens. And then shone forth the power and valour of our city. Her enemies had supposed that she was exhausted by the war, and our ships were blockaded at Mitylene. But the citizens themselves embarked, and came to the rescue with sixty other ships, and their valour was confessed of all men, for they conquered their enemies and delivered their friends. And yet by some evil fortune they were left to perish at sea, and therefore are not interred here. Ever to be remembered and honoured are they, for by their valour not only that sea-fight was won for us, but the entire war was decided by them, and through them the city gained the reputation of being invincible, even though attacked by all mankind. And that reputation was a true one, for the defeat which came upon us was our own doing. We were never conquered by others, and to this day we are still unconquered by them; but we were our own conquerors, and received defeat at our own hands. Afterwards there was quiet and peace abroad, but there sprang up war at home; and, if men are destined to have civil war, no one could have desired that his city should take the disorder in a milder form. How joyful and natural was the reconciliation of those who came from the Piraeus and those who came from the city; with what moderation did they order the war against the tyrants in Eleusis, and in a manner how unlike what the other Hellenes expected! And the reason of this gentleness was the veritable tie of blood, which created among them a friendship as of kinsmen, faithful not in word only, but in deed. And we ought also to remember those who then fell by one another's hands, and on such occasions as these to reconcile them with sacrifices and prayers, praying to those who have power over them, that they may be reconciled even as we are reconciled. For they did not attack one another out of malice or enmity, but they were unfortunate. And that such was the fact we ourselves are witnesses, who are of the same race with them, and have mutually received and granted forgiveness of what we have done and suffered. After this there was perfect peace, and the city had rest; and her feeling was that she forgave the barbarians, who had severely suffered at her hands and severely retaliated, but that she was indignant at the ingratitude of the Hellenes, when she remembered how they had received good from her and returned evil, having made common cause with the barbarians, depriving her of the ships which had once been their salvation, and dismantling our walls, which had preserved their own from falling. She thought that she would no longer defend the Hellenes, when enslaved either by one another or by the barbarians, and did accordingly. This was our feeling, while the Lacedaemonians were thinking that we who were the champions of liberty had fallen, and that their business was to subject the remaining Hellenes. And why should I say more? for the events of which I am speaking happened not long ago and we can all of us remember how the chief peoples of Hellas, Argives and Boeotians and Corinthians, came to feel the need of us, and, what is the greatest miracle of all, the Persian king himself was driven to such extremity as to come round to the opinion, that from this city, of which he was the destroyer, and from no other, his salvation would proceed.

\par  And if a person desired to bring a deserved accusation against our city, he would find only one charge which he could justly urge—that she was too compassionate and too favourable to the weaker side. And in this instance she was not able to hold out or keep her resolution of refusing aid to her injurers when they were being enslaved, but she was softened, and did in fact send out aid, and delivered the Hellenes from slavery, and they were free until they afterwards enslaved themselves. Whereas, to the great king she refused to give the assistance of the state, for she could not forget the trophies of Marathon and Salamis and Plataea; but she allowed exiles and volunteers to assist him, and they were his salvation. And she herself, when she was compelled, entered into the war, and built walls and ships, and fought with the Lacedaemonians on behalf of the Parians. Now the king fearing this city and wanting to stand aloof, when he saw the Lacedaemonians growing weary of the war at sea, asked of us, as the price of his alliance with us and the other allies, to give up the Hellenes in Asia, whom the Lacedaemonians had previously handed over to him, he thinking that we should refuse, and that then he might have a pretence for withdrawing from us. About the other allies he was mistaken, for the Corinthians and Argives and Boeotians, and the other states, were quite willing to let them go, and swore and covenanted, that, if he would pay them money, they would make over to him the Hellenes of the continent, and we alone refused to give them up and swear. Such was the natural nobility of this city, so sound and healthy was the spirit of freedom among us, and the instinctive dislike of the barbarian, because we are pure Hellenes, having no admixture of barbarism in us. For we are not like many others, descendants of Pelops or Cadmus or Egyptus or Danaus, who are by nature barbarians, and yet pass for Hellenes, and dwell in the midst of us; but we are pure Hellenes, uncontaminated by any foreign element, and therefore the hatred of the foreigner has passed unadulterated into the life-blood of the city. And so, notwithstanding our noble sentiments, we were again isolated, because we were unwilling to be guilty of the base and unholy act of giving up Hellenes to barbarians. And we were in the same case as when we were subdued before; but, by the favour of Heaven, we managed better, for we ended the war without the loss of our ships or walls or colonies; the enemy was only too glad to be quit of us. Yet in this war we lost many brave men, such as were those who fell owing to the ruggedness of the ground at the battle of Corinth, or by treason at Lechaeum. Brave men, too, were those who delivered the Persian king, and drove the Lacedaemonians from the sea. I remind you of them, and you must celebrate them together with me, and do honour to their memories.

\par  Such were the actions of the men who are here interred, and of others who have died on behalf of their country; many and glorious things I have spoken of them, and there are yet many more and more glorious things remaining to be told—many days and nights would not suffice to tell of them. Let them not be forgotten, and let every man remind their descendants that they also are soldiers who must not desert the ranks of their ancestors, or from cowardice fall behind. Even as I exhort you this day, and in all future time, whenever I meet with any of you, shall continue to remind and exhort you, O ye sons of heroes, that you strive to be the bravest of men. And I think that I ought now to repeat what your fathers desired to have said to you who are their survivors, when they went out to battle, in case anything happened to them. I will tell you what I heard them say, and what, if they had only speech, they would fain be saying, judging from what they then said. And you must imagine that you hear them saying what I now repeat to you:—

\par  'Sons, the event proves that your fathers were brave men; for we might have lived dishonourably, but have preferred to die honourably rather than bring you and your children into disgrace, and rather than dishonour our own fathers and forefathers; considering that life is not life to one who is a dishonour to his race, and that to such a one neither men nor Gods are friendly, either while he is on the earth or after death in the world below. Remember our words, then, and whatever is your aim let virtue be the condition of the attainment of your aim, and know that without this all possessions and pursuits are dishonourable and evil. For neither does wealth bring honour to the owner, if he be a coward; of such a one the wealth belongs to another, and not to himself. Nor does beauty and strength of body, when dwelling in a base and cowardly man, appear comely, but the reverse of comely, making the possessor more conspicuous, and manifesting forth his cowardice. And all knowledge, when separated from justice and virtue, is seen to be cunning and not wisdom; wherefore make this your first and last and constant and all-absorbing aim, to exceed, if possible, not only us but all your ancestors in virtue; and know that to excel you in virtue only brings us shame, but that to be excelled by you is a source of happiness to us. And we shall most likely be defeated, and you will most likely be victors in the contest, if you learn so to order your lives as not to abuse or waste the reputation of your ancestors, knowing that to a man who has any self-respect, nothing is more dishonourable than to be honoured, not for his own sake, but on account of the reputation of his ancestors. The honour of parents is a fair and noble treasure to their posterity, but to have the use of a treasure of wealth and honour, and to leave none to your successors, because you have neither money nor reputation of your own, is alike base and dishonourable. And if you follow our precepts you will be received by us as friends, when the hour of destiny brings you hither; but if you neglect our words and are disgraced in your lives, no one will welcome or receive you. This is the message which is to be delivered to our children.

\par  'Some of us have fathers and mothers still living, and we would urge them, if, as is likely, we shall die, to bear the calamity as lightly as possible, and not to condole with one another; for they have sorrows enough, and will not need any one to stir them up. While we gently heal their wounds, let us remind them that the Gods have heard the chief part of their prayers; for they prayed, not that their children might live for ever, but that they might be brave and renowned. And this, which is the greatest good, they have attained. A mortal man cannot expect to have everything in his own life turning out according to his will; and they, if they bear their misfortunes bravely, will be truly deemed brave fathers of the brave. But if they give way to their sorrows, either they will be suspected of not being our parents, or we of not being such as our panegyrists declare. Let not either of the two alternatives happen, but rather let them be our chief and true panegyrists, who show in their lives that they are true men, and had men for their sons. Of old the saying, "Nothing too much," appeared to be, and really was, well said. For he whose happiness rests with himself, if possible, wholly, and if not, as far as is possible,—who is not hanging in suspense on other men, or changing with the vicissitude of their fortune,—has his life ordered for the best. He is the temperate and valiant and wise; and when his riches come and go, when his children are given and taken away, he will remember the proverb—"Neither rejoicing overmuch nor grieving overmuch," for he relies upon himself. And such we would have our parents to be—that is our word and wish, and as such we now offer ourselves, neither lamenting overmuch, nor fearing overmuch, if we are to die at this time. And we entreat our fathers and mothers to retain these feelings throughout their future life, and to be assured that they will not please us by sorrowing and lamenting over us. But, if the dead have any knowledge of the living, they will displease us most by making themselves miserable and by taking their misfortunes too much to heart, and they will please us best if they bear their loss lightly and temperately. For our life will have the noblest end which is vouchsafed to man, and should be glorified rather than lamented. And if they will direct their minds to the care and nurture of our wives and children, they will soonest forget their misfortunes, and live in a better and nobler way, and be dearer to us.

\par  'This is all that we have to say to our families: and to the state we would say—Take care of our parents and of our sons: let her worthily cherish the old age of our parents, and bring up our sons in the right way. But we know that she will of her own accord take care of them, and does not need any exhortation of ours.'

\par  This, O ye children and parents of the dead, is the message which they bid us deliver to you, and which I do deliver with the utmost seriousness. And in their name I beseech you, the children, to imitate your fathers, and you, parents, to be of good cheer about yourselves; for we will nourish your age, and take care of you both publicly and privately in any place in which one of us may meet one of you who are the parents of the dead. And the care of you which the city shows, you know yourselves; for she has made provision by law concerning the parents and children of those who die in war; the highest authority is specially entrusted with the duty of watching over them above all other citizens, and they will see that your fathers and mothers have no wrong done to them. The city herself shares in the education of the children, desiring as far as it is possible that their orphanhood may not be felt by them; while they are children she is a parent to them, and when they have arrived at man's estate she sends them to their several duties, in full armour clad; and bringing freshly to their minds the ways of their fathers, she places in their hands the instruments of their fathers' virtues; for the sake of the omen, she would have them from the first begin to rule over their own houses arrayed in the strength and arms of their fathers. And as for the dead, she never ceases honouring them, celebrating in common for all rites which become the property of each; and in addition to this, holding gymnastic and equestrian contests, and musical festivals of every sort. She is to the dead in the place of a son and heir, and to their sons in the place of a father, and to their parents and elder kindred in the place of a guardian—ever and always caring for them. Considering this, you ought to bear your calamity the more gently; for thus you will be most endeared to the dead and to the living, and your sorrows will heal and be healed. And now do you and all, having lamented the dead in common according to the law, go your ways.

\par  You have heard, Menexenus, the oration of Aspasia the Milesian.

\par \textbf{MENEXENUS}
\par   Truly, Socrates, I marvel that Aspasia, who is only a woman, should be able to compose such a speech; she must be a rare one.

\par \textbf{SOCRATES}
\par   Well, if you are incredulous, you may come with me and hear her.

\par \textbf{MENEXENUS}
\par   I have often met Aspasia, Socrates, and know what she is like.

\par \textbf{SOCRATES}
\par   Well, and do you not admire her, and are you not grateful for her speech?

\par \textbf{MENEXENUS}
\par   Yes, Socrates, I am very grateful to her or to him who told you, and still more to you who have told me.

\par \textbf{SOCRATES}
\par   Very good. But you must take care not to tell of me, and then at some future time I will repeat to you many other excellent political speeches of hers.

\par \textbf{MENEXENUS}
\par   Fear not, only let me hear them, and I will keep the secret.

\par \textbf{SOCRATES}
\par   Then I will keep my promise.

\par 
 
\end{document}