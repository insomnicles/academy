
\documentclass[11pt,letter]{article}


\begin{document}

\title{Laws\thanks{Source: https://www.gutenberg.org/files/1750/1750-h/1750-h.htm. License: http://gutenberg.org/license ds}}
\date{\today}
\author{Plato, 427? BCE-347? BCE\\ Translated by Jowett, Benjamin, 1817-1893}
\maketitle

\setcounter{tocdepth}{1}
\tableofcontents
\renewcommand{\baselinestretch}{1.0}
\normalsize
\newpage

\section{
      INTRODUCTION AND ANALYSIS.
    }
\par  The genuineness of the Laws is sufficiently proved (1) by more than twenty citations of them in the writings of Aristotle, who was residing at Athens during the last twenty years of the life of Plato, and who, having left it after his death (B.C. 347), returned thither twelve years later (B.C. 335); (2) by the allusion of Isocrates

\par  (Oratio ad Philippum missa, p.84: To men tais paneguresin enochlein kai pros apantas legein tous sunprechontas en autais pros oudena legein estin, all omoios oi toioutoi ton logon (sc. speeches in the assembly) akuroi tugchanousin ontes tois nomois kai tais politeiais tais upo ton sophiston gegrammenais.) —writing 346 B.C., a year after the death of Plato, and probably not more than three or four years after the composition of the Laws—who speaks of the Laws and Republics written by philosophers (upo ton sophiston); (3) by the reference (Athen.) of the comic poet Alexis, a younger contemporary of Plato (fl. B.C 356-306), to the enactment about prices, which occurs in Laws xi., viz that the same goods should not be offered at two prices on the same day
 
\par  Meineke, Frag. Com. Graec. ); (4) by the unanimous voice of later antiquity and the absence of any suspicion among ancient writers worth speaking of to the contrary; for it is not said of Philippus of Opus that he composed any part of the Laws, but only that he copied them out of the waxen tablets, and was thought by some to have written the Epinomis (Diog. Laert.) That the longest and one of the best writings bearing the name of Plato should be a forgery, even if its genuineness were unsupported by external testimony, would be a singular phenomenon in ancient literature; and although the critical worth of the consensus of late writers is generally not to be compared with the express testimony of contemporaries, yet a somewhat greater value may be attributed to their consent in the present instance, because the admission of the Laws is combined with doubts about the Epinomis, a spurious writing, which is a kind of epilogue to the larger work probably of a much later date. This shows that the reception of the Laws was not altogether undiscriminating.

\par  The suspicion which has attached to the Laws of Plato in the judgment of some modern writers appears to rest partly (1) on differences in the style and form of the work, and (2) on differences of thought and opinion which they observe in them. Their suspicion is increased by the fact that these differences are accompanied by resemblances as striking to passages in other Platonic writings. They are sensible of a want of point in the dialogue and a general inferiority in the ideas, plan, manners, and style. They miss the poetical flow, the dramatic verisimilitude, the life and variety of the characters, the dialectic subtlety, the Attic purity, the luminous order, the exquisite urbanity; instead of which they find tautology, obscurity, self-sufficiency, sermonizing, rhetorical declamation, pedantry, egotism, uncouth forms of sentences, and peculiarities in the use of words and idioms. They are unable to discover any unity in the patched, irregular structure. The speculative element both in government and education is superseded by a narrow economical or religious vein. The grace and cheerfulness of Athenian life have disappeared; and a spirit of moroseness and religious intolerance has taken their place. The charm of youth is no longer there; the mannerism of age makes itself unpleasantly felt. The connection is often imperfect; and there is a want of arrangement, exhibited especially in the enumeration of the laws towards the end of the work. The Laws are full of flaws and repetitions. The Greek is in places very ungrammatical and intractable. A cynical levity is displayed in some passages, and a tone of disappointment and lamentation over human things in others. The critics seem also to observe in them bad imitations of thoughts which are better expressed in Plato's other writings. Lastly, they wonder how the mind which conceived the Republic could have left the Critias, Hermocrates, and Philosophus incomplete or unwritten, and have devoted the last years of life to the Laws.

\par  The questions which have been thus indirectly suggested may be considered by us under five or six heads: I, the characters; II, the plan; III, the style; IV, the imitations of other writings of Plato; V; the more general relation of the Laws to the Republic and the other dialogues; and VI, to the existing Athenian and Spartan states.

\par  I. Already in the Philebus the distinctive character of Socrates has disappeared; and in the Timaeus, Sophist, and Statesman his function of chief speaker is handed over to the Pythagorean philosopher Timaeus, and to the Eleatic Stranger, at whose feet he sits, and is silent. More and more Plato seems to have felt in his later writings that the character and method of Socrates were no longer suited to be the vehicle of his own philosophy. He is no longer interrogative but dogmatic; not 'a hesitating enquirer,' but one who speaks with the authority of a legislator. Even in the Republic we have seen that the argument which is carried on by Socrates in the old style with Thrasymachus in the first book, soon passes into the form of exposition. In the Laws he is nowhere mentioned. Yet so completely in the tradition of antiquity is Socrates identified with Plato, that in the criticism of the Laws which we find in the so-called Politics of Aristotle he is supposed by the writer still to be playing his part of the chief speaker (compare Pol. ).

\par  The Laws are discussed by three representatives of Athens, Crete, and Sparta. The Athenian, as might be expected, is the protagonist or chief speaker, while the second place is assigned to the Cretan, who, as one of the leaders of a new colony, has a special interest in the conversation. At least four-fifths of the answers are put into his mouth. The Spartan is every inch a soldier, a man of few words himself, better at deeds than words. The Athenian talks to the two others, although they are his equals in age, in the style of a master discoursing to his scholars; he frequently praises himself; he entertains a very poor opinion of the understanding of his companions. Certainly the boastfulness and rudeness of the Laws is the reverse of the refined irony and courtesy which characterize the earlier dialogues. We are no longer in such good company as in the Phaedrus and Symposium. Manners are lost sight of in the earnestness of the speakers, and dogmatic assertions take the place of poetical fancies.

\par  The scene is laid in Crete, and the conversation is held in the course of a walk from Cnosus to the cave and temple of Zeus, which takes place on one of the longest and hottest days of the year. The companions start at dawn, and arrive at the point in their conversation which terminates the fourth book, about noon. The God to whose temple they are going is the lawgiver of Crete, and this may be supposed to be the very cave at which he gave his oracles to Minos. But the externals of the scene, which are briefly and inartistically described, soon disappear, and we plunge abruptly into the subject of the dialogue. We are reminded by contrast of the higher art of the Phaedrus, in which the summer's day, and the cool stream, and the chirping of the grasshoppers, and the fragrance of the agnus castus, and the legends of the place are present to the imagination throughout the discourse.

\par  The typical Athenian apologizes for the tendency of his countrymen 'to spin a long discussion out of slender materials,' and in a similar spirit the Lacedaemonian Megillus apologizes for the Spartan brevity (compare Thucydid. ), acknowledging at the same time that there may be occasions when long discourses are necessary. The family of Megillus is the proxenus of Athens at Sparta; and he pays a beautiful compliment to the Athenian, significant of the character of the work, which, though borrowing many elements from Sparta, is also pervaded by an Athenian spirit. A good Athenian, he says, is more than ordinarily good, because he is inspired by nature and not manufactured by law. The love of listening which is attributed to the Timocrat in the Republic is also exhibited in him. The Athenian on his side has a pleasure in speaking to the Lacedaemonian of the struggle in which their ancestors were jointly engaged against the Persians. A connexion with Athens is likewise intimated by the Cretan Cleinias. He is the relative of Epimenides, whom, by an anachronism of a century,—perhaps arising as Zeller suggests (Plat. Stud.) out of a confusion of the visit of Epimenides and Diotima (Symp. ),—he describes as coming to Athens, not after the attempt of Cylon, but ten years before the Persian war. The Cretan and Lacedaemonian hardly contribute at all to the argument of which the Athenian is the expounder; they only supply information when asked about the institutions of their respective countries. A kind of simplicity or stupidity is ascribed to them. At first, they are dissatisfied with the free criticisms which the Athenian passes upon the laws of Minos and Lycurgus, but they acquiesce in his greater experience and knowledge of the world. They admit that there can be no objection to the enquiry; for in the spirit of the legislator himself, they are discussing his laws when there are no young men present to listen. They are unwilling to allow that the Spartan and Cretan lawgivers can have been mistaken in honouring courage as the first part of virtue, and are puzzled at hearing for the first time that 'Goods are only evil to the evil.' Several times they are on the point of quarrelling, and by an effort learn to restrain their natural feeling (compare Shakespeare, Henry V, act iii. sc. 2). In Book vii., the Lacedaemonian expresses a momentary irritation at the accusation which the Athenian brings against the Spartan institutions, of encouraging licentiousness in their women, but he is reminded by the Cretan that the permission to criticize them freely has been given, and cannot be retracted. His only criterion of truth is the authority of the Spartan lawgiver; he is 'interested,' in the novel speculations of the Athenian, but inclines to prefer the ordinances of Lycurgus.

\par  The three interlocutors all of them speak in the character of old men, which forms a pleasant bond of union between them. They have the feelings of old age about youth, about the state, about human things in general. Nothing in life seems to be of much importance to them; they are spectators rather than actors, and men in general appear to the Athenian speaker to be the playthings of the Gods and of circumstances. Still they have a fatherly care of the young, and are deeply impressed by sentiments of religion. They would give confidence to the aged by an increasing use of wine, which, as they get older, is to unloose their tongues and make them sing. The prospect of the existence of the soul after death is constantly present to them; though they can hardly be said to have the cheerful hope and resignation which animates Socrates in the Phaedo or Cephalus in the Republic. Plato appears to be expressing his own feelings in remarks of this sort. For at the time of writing the first book of the Laws he was at least seventy-four years of age, if we suppose him to allude to the victory of the Syracusans under Dionysius the Younger over the Locrians, which occurred in the year 356. Such a sadness was the natural effect of declining years and failing powers, which make men ask, 'After all, what profit is there in life?' They feel that their work is beginning to be over, and are ready to say, 'All the world is a stage;' or, in the actual words of Plato, 'Let us play as good plays as we can,' though 'we must be sometimes serious, which is not agreeable, but necessary.' These are feelings which have crossed the minds of reflective persons in all ages, and there is no reason to connect the Laws any more than other parts of Plato's writings with the very uncertain narrative of his life, or to imagine that this melancholy tone is attributable to disappointment at having failed to convert a Sicilian tyrant into a philosopher.

\par  II. The plan of the Laws is more irregular and has less connexion than any other of the writings of Plato. As Aristotle says in the Politics, 'The greater part consists of laws'; in Books v, vi, xi, xii the dialogue almost entirely disappears. Large portions of them are rather the materials for a work than a finished composition which may rank with the other Platonic dialogues. To use his own image, 'Some stones are regularly inserted in the building; others are lying on the ground ready for use.' There is probably truth in the tradition that the Laws were not published until after the death of Plato. We can easily believe that he has left imperfections, which would have been removed if he had lived a few years longer. The arrangement might have been improved; the connexion of the argument might have been made plainer, and the sentences more accurately framed. Something also may be attributed to the feebleness of old age. Even a rough sketch of the Phaedrus or Symposium would have had a very different look. There is, however, an interest in possessing one writing of Plato which is in the process of creation.

\par  We must endeavour to find a thread of order which will carry us through this comparative disorder. The first four books are described by Plato himself as the preface or preamble. Having arrived at the conclusion that each law should have a preamble, the lucky thought occurs to him at the end of the fourth book that the preceding discourse is the preamble of the whole. This preamble or introduction may be abridged as follows:—

\par  The institutions of Sparta and Crete are admitted by the Lacedaemonian and Cretan to have one aim only: they were intended by the legislator to inspire courage in war. To this the Athenian objects that the true lawgiver should frame his laws with a view to all the virtues and not to one only. Better is he who has temperance as well as courage, than he who has courage only; better is he who is faithful in civil broils, than he who is a good soldier only. Better, too, is peace than war; the reconciliation than the defeat of an enemy. And he who would attain all virtue should be trained amid pleasures as well as pains. Hence there should be convivial intercourse among the citizens, and a man's temperance should be tested in his cups, as we test his courage amid dangers. He should have a fear of the right sort, as well as a courage of the right sort.

\par  At the beginning of the second book the subject of pleasure leads to education, which in the early years of life is wholly a discipline imparted by the means of pleasure and pain. The discipline of pleasure is implanted chiefly by the practice of the song and the dance. Of these the forms should be fixed, and not allowed to depend on the fickle breath of the multitude. There will be choruses of boys, girls, and grown-up persons, and all will be heard repeating the same strain, that 'virtue is happiness.' One of them will give the law to the rest; this will be the chorus of aged minstrels, who will sing the most beautiful and the most useful of songs. They will require a little wine, to mellow the austerity of age, and make them amenable to the laws.

\par  After having laid down as the first principle of politics, that peace, and not war, is the true aim of the legislator, and briefly discussed music and festive intercourse, at the commencement of the third book Plato makes a digression, in which he speaks of the origin of society. He describes, first of all, the family; secondly, the patriarchal stage, which is an aggregation of families; thirdly, the founding of regular cities, like Ilium; fourthly, the establishment of a military and political system, like that of Sparta, with which he identifies Argos and Messene, dating from the return of the Heraclidae. But the aims of states should be good, or else, like the prayer of Theseus, they may be ruinous to themselves. This was the case in two out of three of the Heracleid kingdoms. They did not understand that the powers in a state should be balanced. The balance of powers saved Sparta, while the excess of tyranny in Persia and the excess of liberty at Athens have been the ruin of both...This discourse on politics is suddenly discovered to have an immediate practical use; for Cleinias the Cretan is about to give laws to a new colony.

\par  At the beginning of the fourth book, after enquiring into the circumstances and situation of the colony, the Athenian proceeds to make further reflections. Chance, and God, and the skill of the legislator, all co-operate in the formation of states. And the most favourable condition for the foundation of a new one is when the government is in the hands of a virtuous tyrant who has the good fortune to be the contemporary of a great legislator. But a virtuous tyrant is a contradiction in terms; we can at best only hope to have magistrates who are the servants of reason and the law. This leads to the enquiry, what is to be the polity of our new state. And the answer is, that we are to fear God, and honour our parents, and to cultivate virtue and justice; these are to be our first principles. Laws must be definite, and we should create in the citizens a predisposition to obey them. The legislator will teach as well as command; and with this view he will prefix preambles to his principal laws.

\par  The fifth book commences in a sort of dithyramb with another and higher preamble about the honour due to the soul, whence are deduced the duties of a man to his parents and his friends, to the suppliant and stranger. He should be true and just, free from envy and excess of all sorts, forgiving to crimes which are not incurable and are partly involuntary; and he should have a true taste. The noblest life has the greatest pleasures and the fewest pains...Having finished the preamble, and touched on some other preliminary considerations, we proceed to the Laws, beginning with the constitution of the state. This is not the best or ideal state, having all things common, but only the second-best, in which the land and houses are to be distributed among 5040 citizens divided into four classes. There is to be no gold or silver among them, and they are to have moderate wealth, and to respect number and numerical order in all things.

\par  In the first part of the sixth book, Plato completes his sketch of the constitution by the appointment of officers. He explains the manner in which guardians of the law, generals, priests, wardens of town and country, ministers of education, and other magistrates are to be appointed; and also in what way courts of appeal are to be constituted, and omissions in the law to be supplied. Next—and at this point the Laws strictly speaking begin—there follow enactments respecting marriage and the procreation of children, respecting property in slaves as well as of other kinds, respecting houses, married life, common tables for men and women. The question of age in marriage suggests the consideration of a similar question about the time for holding offices, and for military service, which had been previously omitted.

\par  Resuming the order of the discussion, which was indicated in the previous book, from marriage and birth we proceed to education in the seventh book. Education is to begin at or rather before birth; to be continued for a time by mothers and nurses under the supervision of the state; finally, to comprehend music and gymnastics. Under music is included reading, writing, playing on the lyre, arithmetic, geometry, and a knowledge of astronomy sufficient to preserve the minds of the citizens from impiety in after-life. Gymnastics are to be practised chiefly with a view to their use in war. The discussion of education, which was lightly touched upon in Book ii, is here completed.

\par  The eighth book contains regulations for civil life, beginning with festivals, games, and contests, military exercises and the like. On such occasions Plato seems to see young men and maidens meeting together, and hence he is led into discussing the relations of the sexes, the evil consequences which arise out of the indulgence of the passions, and the remedies for them. Then he proceeds to speak of agriculture, of arts and trades, of buying and selling, and of foreign commerce.

\par  The remaining books of the Laws, ix-xii, are chiefly concerned with criminal offences. In the first class are placed offences against the Gods, especially sacrilege or robbery of temples: next follow offences against the state,—conspiracy, treason, theft. The mention of thefts suggests a distinction between voluntary and involuntary, curable and incurable offences. Proceeding to the greater crime of homicide, Plato distinguishes between mere homicide, manslaughter, which is partly voluntary and partly involuntary, and murder, which arises from avarice, ambition, fear. He also enumerates murders by kindred, murders by slaves, wounds with or without intent to kill, wounds inflicted in anger, crimes of or against slaves, insults to parents. To these, various modes of purification or degrees of punishment are assigned, and the terrors of another world are also invoked against them.

\par  At the beginning of Book x, all acts of violence, including sacrilege, are summed up in a single law. The law is preceded by an admonition, in which the offenders are informed that no one ever did an unholy act or said an unlawful word while he retained his belief in the existence of the Gods; but either he denied their existence, or he believed that they took no care of man, or that they might be turned from their course by sacrifices and prayers. The remainder of the book is devoted to the refutation of these three classes of unbelievers, and concludes with the means to be taken for their reformation, and the announcement of their punishments if they continue obstinate and impenitent.

\par  The eleventh book is taken up with laws and with admonitions relating to individuals, which follow one another without any exact order. There are laws concerning deposits and the finding of treasure; concerning slaves and freedmen; concerning retail trade, bequests, divorces, enchantments, poisonings, magical arts, and the like. In the twelfth book the same subjects are continued. Laws are passed concerning violations of military discipline, concerning the high office of the examiners and their burial; concerning oaths and the violation of them, and the punishments of those who neglect their duties as citizens. Foreign travel is then discussed, and the permission to be accorded to citizens of journeying in foreign parts; the strangers who may come to visit the city are also spoken of, and the manner in which they are to be received. Laws are added respecting sureties, searches for property, right of possession by prescription, abduction of witnesses, theatrical competition, waging of private warfare, and bribery in offices. Rules are laid down respecting taxation, respecting economy in sacred rites, respecting judges, their duties and sentences, and respecting sepulchral places and ceremonies. Here the Laws end. Lastly, a Nocturnal Council is instituted for the preservation of the state, consisting of older and younger members, who are to exhibit in their lives that virtue which is the basis of the state, to know the one in many, and to be educated in divine and every other kind of knowledge which will enable them to fulfil their office.

\par  III. The style of the Laws differs in several important respects from that of the other dialogues of Plato: (1) in the want of character, power, and lively illustration; (2) in the frequency of mannerisms (compare Introduction to the Philebus); (3) in the form and rhythm of the sentences; (4) in the use of words. On the other hand, there are many passages (5) which are characterized by a sort of ethical grandeur; and (6) in which, perhaps, a greater insight into human nature, and a greater reach of practical wisdom is shown, than in any other of Plato's writings.

\par  1. The discourse of the three old men is described by themselves as an old man's game of play. Yet there is little of the liveliness of a game in their mode of treating the subject. They do not throw the ball to and fro, but two out of the three are listeners to the third, who is constantly asserting his superior wisdom and opportunities of knowledge, and apologizing (not without reason) for his own want of clearness of speech. He will 'carry them over the stream;' he will answer for them when the argument is beyond their comprehension; he is afraid of their ignorance of mathematics, and thinks that gymnastic is likely to be more intelligible to them;—he has repeated his words several times, and yet they cannot understand him. The subject did not properly take the form of dialogue, and also the literary vigour of Plato had passed away. The old men speak as they might be expected to speak, and in this there is a touch of dramatic truth. Plato has given the Laws that form or want of form which indicates the failure of natural power. There is no regular plan—none of that consciousness of what has preceded and what is to follow, which makes a perfect style,—but there are several attempts at a plan; the argument is 'pulled up,' and frequent explanations are offered why a particular topic was introduced.

\par  The fictions of the Laws have no longer the verisimilitude which is characteristic of the Phaedrus and the Timaeus, or even of the Statesman. We can hardly suppose that an educated Athenian would have placed the visit of Epimenides to Athens ten years before the Persian war, or have imagined that a war with Messene prevented the Lacedaemonians from coming to the rescue of Hellas. The narrative of the origin of the Dorian institutions, which are said to have been due to a fear of the growing power of the Assyrians, is a plausible invention, which may be compared with the tale of the island of Atlantis and the poem of Solon, but is not accredited by similar arts of deception. The other statement that the Dorians were Achaean exiles assembled by Dorieus, and the assertion that Troy was included in the Assyrian Empire, have some foundation (compare for the latter point, Diod. Sicul.). Nor is there anywhere in the Laws that lively enargeia, that vivid mise en scene, which is as characteristic of Plato as of some modern novelists.

\par  The old men are afraid of the ridicule which 'will fall on their heads more than enough,' and they do not often indulge in a joke. In one of the few which occur, the book of the Laws, if left incomplete, is compared to a monster wandering about without a head. But we no longer breathe the atmosphere of humour which pervades the Symposium and the Euthydemus, in which we pass within a few sentences from the broadest Aristophanic joke to the subtlest refinement of wit and fancy; instead of this, in the Laws an impression of baldness and feebleness is often left upon our minds. Some of the most amusing descriptions, as, for example, of children roaring for the first three years of life; or of the Athenians walking into the country with fighting-cocks under their arms; or of the slave doctor who knocks about his patients finely; and the gentleman doctor who courteously persuades them; or of the way of keeping order in the theatre, 'by a hint from a stick,' are narrated with a commonplace gravity; but where we find this sort of dry humour we shall not be far wrong in thinking that the writer intended to make us laugh. The seriousness of age takes the place of the jollity of youth. Life should have holidays and festivals; yet we rebuke ourselves when we laugh, and take our pleasures sadly. The irony of the earlier dialogues, of which some traces occur in the tenth book, is replaced by a severity which hardly condescends to regard human things. 'Let us say, if you please, that man is of some account, but I was speaking of him in comparison with God.'

\par  The imagery and illustrations are poor in themselves, and are not assisted by the surrounding phraseology. We have seen how in the Republic, and in the earlier dialogues, figures of speech such as 'the wave,' 'the drone,' 'the chase,' 'the bride,' appear and reappear at intervals. Notes are struck which are repeated from time to time, as in a strain of music. There is none of this subtle art in the Laws. The illustrations, such as the two kinds of doctors, 'the three kinds of funerals,' the fear potion, the puppet, the painter leaving a successor to restore his picture, the 'person stopping to consider where three ways meet,' the 'old laws about water of which he will not divert the course,' can hardly be said to do much credit to Plato's invention. The citations from the poets have lost that fanciful character which gave them their charm in the earlier dialogues. We are tired of images taken from the arts of navigation, or archery, or weaving, or painting, or medicine, or music. Yet the comparisons of life to a tragedy, or of the working of mind to the revolution of the self-moved, or of the aged parent to the image of a God dwelling in the house, or the reflection that 'man is made to be the plaything of God, and that this rightly considered is the best of him,' have great beauty.

\par  2. The clumsiness of the style is exhibited in frequent mannerisms and repetitions. The perfection of the Platonic dialogue consists in the accuracy with which the question and answer are fitted into one another, and the regularity with which the steps of the argument succeed one another. This finish of style is no longer discernible in the Laws. There is a want of variety in the answers; nothing can be drawn out of the respondents but 'Yes' or 'No,' 'True,' 'To be sure,' etc. ; the insipid forms, 'What do you mean?' 'To what are you referring?' are constantly returning. Again and again the speaker is charged, or charges himself, with obscurity; and he repeats again and again that he will explain his views more clearly. The process of thought which should be latent in the mind of the writer appears on the surface. In several passages the Athenian praises himself in the most unblushing manner, very unlike the irony of the earlier dialogues, as when he declares that 'the laws are a divine work given by some inspiration of the Gods,' and that 'youth should commit them to memory instead of the compositions of the poets.' The prosopopoeia which is adopted by Plato in the Protagoras and other dialogues is repeated until we grow weary of it. The legislator is always addressing the speakers or the youth of the state, and the speakers are constantly making addresses to the legislator. A tendency to a paradoxical manner of statement is also observable. 'We must have drinking,' 'we must have a virtuous tyrant'—this is too much for the duller wits of the Lacedaemonian and Cretan, who at first start back in surprise. More than in any other writing of Plato the tone is hortatory; the laws are sermons as well as laws; they are considered to have a religious sanction, and to rest upon a religious sentiment in the mind of the citizens. The words of the Athenian are attributed to the Lacedaemonian and Cretan, who are supposed to have made them their own, after the manner of the earlier dialogues. Resumptions of subjects which have been half disposed of in a previous passage constantly occur: the arrangement has neither the clearness of art nor the freedom of nature. Irrelevant remarks are made here and there, or illustrations used which are not properly fitted in. The dialogue is generally weak and laboured, and is in the later books fairly given up, apparently, because unsuited to the subject of the work. The long speeches or sermons of the Athenian, often extending over several pages, have never the grace and harmony which are exhibited in the earlier dialogues. For Plato is incapable of sustained composition; his genius is dramatic rather than oratorical; he can converse, but he cannot make a speech. Even the Timaeus, which is one of his most finished works, is full of abrupt transitions. There is the same kind of difference between the dialogue and the continuous discourse of Plato as between the narrative and speeches of Thucydides.

\par  3. The perfection of style is variety in unity, freedom, ease, clearness, the power of saying anything, and of striking any note in the scale of human feelings without impropriety; and such is the divine gift of language possessed by Plato in the Symposium and Phaedrus. From this there are many fallings-off in the Laws: first, in the structure of the sentences, which are rhythmical and monotonous,—the formal and sophistical manner of the age is superseding the natural genius of Plato: secondly, many of them are of enormous length, and the latter end often forgets the beginning of them,—they seem never to have received the second thoughts of the author; either the emphasis is wrongly placed, or there is a want of point in a clause; or an absolute case occurs which is not properly separated from the rest of the sentence; or words are aggregated in a manner which fails to show their relation to one another; or the connecting particles are omitted at the beginning of sentences; the uses of the relative and antecedent are more indistinct, the changes of person and number more frequent, examples of pleonasm, tautology, and periphrasis, antitheses of positive and negative, false emphasis, and other affectations, are more numerous than in the other writings of Plato; there is also a more common and sometimes unmeaning use of qualifying formulae, os epos eipein, kata dunamin, and of double expressions, pante pantos, oudame oudamos, opos kai ope—these are too numerous to be attributed to errors in the text; again, there is an over-curious adjustment of verb and participle, noun and epithet, and other artificial forms of cadence and expression take the place of natural variety: thirdly, the absence of metaphorical language is remarkable—the style is not devoid of ornament, but the ornament is of a debased rhetorical kind, patched on to instead of growing out of the subject; there is a great command of words, and a laboured use of them; forced attempts at metaphor occur in several passages,—e.g. parocheteuein logois; ta men os tithemena ta d os paratithemena; oinos kolazomenos upo nephontos eterou theou; the plays on the word nomos = nou dianome, ode etara: fourthly, there is a foolish extravagance of language in other passages,—'the swinish ignorance of arithmetic;' 'the justice and suitableness of the discourse on laws;' over-emphasis; 'best of Greeks,' said of all the Greeks, and the like: fifthly, poor and insipid illustrations are also common: sixthly, we may observe an excessive use of climax and hyperbole, aischron legein chre pros autous doulon te kai doulen kai paida kai ei pos oion te olen ten oikian: dokei touto to epitedeuma kata phusin tas peri ta aphrodisia edonas ou monon anthropon alla kai therion diephtharkenai.

\par  4. The peculiarities in the use of words which occur in the Laws have been collected by Zeller (Platonische Studien) and Stallbaum (Legg. ): first, in the use of nouns, such as allodemia, apeniautesis, glukuthumia, diatheter, thrasuxenia, koros, megalonoia, paidourgia: secondly, in the use of adjectives, such as aistor, biodotes, echthodopos, eitheos, chronios, and of adverbs, such as aniditi, anatei, nepoivei: thirdly, in the use of verbs, such as athurein, aissein (aixeien eipein), euthemoneisthai, parapodizesthai, sebein, temelein, tetan. These words however, as Stallbaum remarks, are formed according to analogy, and nearly all of them have the support of some poetical or other authority.

\par  Zeller and Stallbaum have also collected forms of words in the Laws, differing from the forms of the same words which occur in other places: e.g. blabos for blabe, abios for abiotos, acharistos for acharis, douleios for doulikos, paidelos for paidikos, exagrio for exagriaino, ileoumai for ilaskomai, and the Ionic word sophronistus, meaning 'correction.' Zeller has noted a fondness for substantives ending in -ma and -sis, such as georgema, diapauma, epithumema, zemioma, komodema, omilema; blapsis, loidoresis, paraggelsis, and others; also a use of substantives in the plural, which are commonly found only in the singular, maniai, atheotetes, phthonoi, phoboi, phuseis; also, a peculiar use of prepositions in composition, as in eneirgo, apoblapto, dianomotheteo, dieiretai, dieulabeisthai, and other words; also, a frequent occurrence of the Ionic datives plural in -aisi and -oisi, perhaps used for the sake of giving an ancient or archaic effect.

\par  To these peculiarities of words he has added a list of peculiar expressions and constructions. Among the most characteristic are the following: athuta pallakon spermata; amorphoi edrai; osa axiomata pros archontas; oi kata polin kairoi; muthos, used in several places of 'the discourse about laws;' and connected with this the frequent use of paramuthion and paramutheisthai in the general sense of 'address,' 'addressing'; aimulos eros; ataphoi praxeis; muthos akephalos; ethos euthuporon. He remarks also on the frequent employment of the abstract for the concrete; e.g. uperesia for uperetai, phugai for phugades, mechanai in the sense of 'contrivers,' douleia for douloi, basileiai for basileis, mainomena kedeumata for ganaika mainomenen; e chreia ton paidon in the sense of 'indigent children,' and paidon ikanotes; to ethos tes apeirias for e eiothuia apeiria; kuparitton upse te kai kalle thaumasia for kuparittoi mala upselai kai kalai. He further notes some curious uses of the genitive case, e.g. philias omologiai, maniai orges, laimargiai edones, cheimonon anupodesiai, anosioi plegon tolmai; and of the dative, omiliai echthrois, nomothesiai, anosioi plegon tolmai; and of the dative omiliai echthrois, nomothesiai epitropois; and also some rather uncommon periphrases, thremmata Neilou, xuggennetor teknon for alochos, Mouses lexis for poiesis, zographon paides, anthropon spermata and the like; the fondness for particles of limitation, especially tis and ge, sun tisi charisi, tois ge dunamenois and the like; the pleonastic use of tanun, of os, of os eros eipein, of ekastote; and the periphrastic use of the preposition peri. Lastly, he observes the tendency to hyperbata or transpositions of words, and to rhythmical uniformity as well as grammatical irregularity in the structure of the sentences.

\par  For nearly all the expressions which are adduced by Zeller as arguments against the genuineness of the Laws, Stallbaum finds some sort of authority. There is no real ground for doubting that the work was written by Plato, merely because several words occur in it which are not found in his other writings. An imitator may preserve the usual phraseology of a writer better than he would himself. But, on the other hand, the fact that authorities may be quoted in support of most of these uses of words, does not show that the diction is not peculiar. Several of them seem to be poetical or dialectical, and exhibit an attempt to enlarge the limits of Greek prose by the introduction of Homeric and tragic expressions. Most of them do not appear to have retained any hold on the later language of Greece. Like several experiments in language of the writers of the Elizabethan age, they were afterwards lost; and though occasionally found in Plutarch and imitators of Plato, they have not been accepted by Aristotle or passed into the common dialect of Greece.

\par  5. Unequal as the Laws are in style, they contain a few passages which are very grand and noble. For example, the address to the poets: 'Best of strangers, we also are poets of the best and noblest tragedy; for our whole state is an imitation of the best and noblest life, which we affirm to be indeed the very truth of tragedy.' Or again, the sight of young men and maidens in friendly intercourse with one another, suggesting the dangers to which youth is liable from the violence of passion; or the eloquent denunciation of unnatural lusts in the same passage; or the charming thought that the best legislator 'orders war for the sake of peace and not peace for the sake of war;' or the pleasant allusion, 'O Athenian—inhabitant of Attica, I will not say, for you seem to me worthy to be named after the Goddess Athene because you go back to first principles;' or the pithy saying, 'Many a victory has been and will be suicidal to the victors, but education is never suicidal;' or the fine expression that 'the walls of a city should be allowed to sleep in the earth, and that we should not attempt to disinter them;' or the remark that 'God is the measure of all things in a sense far higher than any man can be;' or that 'a man should be from the first a partaker of the truth, that he may live a true man as long as possible;' or the principle repeatedly laid down, that 'the sins of the fathers are not to be visited on the children;' or the description of the funeral rites of those priestly sages who depart in innocence; or the noble sentiment, that we should do more justice to slaves than to equals; or the curious observation, founded, perhaps, on his own experience, that there are a few 'divine men in every state however corrupt, whose conversation is of inestimable value;' or the acute remark, that public opinion is to be respected, because the judgments of mankind about virtue are better than their practice; or the deep religious and also modern feeling which pervades the tenth book (whatever may be thought of the arguments); the sense of the duty of living as a part of a whole, and in dependence on the will of God, who takes care of the least things as well as the greatest; and the picture of parents praying for their children—not as we may say, slightly altering the words of Plato, as if there were no truth or reality in the Gentile religions, but as if there were the greatest—are very striking to us. We must remember that the Laws, unlike the Republic, do not exhibit an ideal state, but are supposed to be on the level of human motives and feelings; they are also on the level of the popular religion, though elevated and purified: hence there is an attempt made to show that the pleasant is also just. But, on the other hand, the priority of the soul to the body, and of God to the soul, is always insisted upon as the true incentive to virtue; especially with great force and eloquence at the commencement of Book v. And the work of legislation is carried back to the first principles of morals.

\par  6. No other writing of Plato shows so profound an insight into the world and into human nature as the Laws. That 'cities will never cease from ill until they are better governed,' is the text of the Laws as well as of the Statesman and Republic. The principle that the balance of power preserves states; the reflection that no one ever passed his whole life in disbelief of the Gods; the remark that the characters of men are best seen in convivial intercourse; the observation that the people must be allowed to share not only in the government, but in the administration of justice; the desire to make laws, not with a view to courage only, but to all virtue; the clear perception that education begins with birth, or even, as he would say, before birth; the attempt to purify religion; the modern reflections, that punishment is not vindictive, and that limits must be set to the power of bequest; the impossibility of undeceiving the victims of quacks and jugglers; the provision for water, and for other requirements of health, and for concealing the bodies of the dead with as little hurt as possible to the living; above all, perhaps, the distinct consciousness that under the actual circumstances of mankind the ideal cannot be carried out, and yet may be a guiding principle—will appear to us, if we remember that we are still in the dawn of politics, to show a great depth of political wisdom.

\par  IV. The Laws of Plato contain numerous passages which closely resemble other passages in his writings. And at first sight a suspicion arises that the repetition shows the unequal hand of the imitator. For why should a writer say over again, in a more imperfect form, what he had already said in his most finished style and manner? And yet it may be urged on the other side that an author whose original powers are beginning to decay will be very liable to repeat himself, as in conversation, so in books. He may have forgotten what he had written before; he may be unconscious of the decline of his own powers. Hence arises a question of great interest, bearing on the genuineness of ancient writers. Is there any criterion by which we can distinguish the genuine resemblance from the spurious, or, in other words, the repetition of a thought or passage by an author himself from the appropriation of it by another? The question has, perhaps, never been fully discussed; and, though a real one, does not admit of a precise answer. A few general considerations on the subject may be offered:—

\par  (a) Is the difference such as might be expected to arise at different times of life or under different circumstances?—There would be nothing surprising in a writer, as he grew older, losing something of his own originality, and falling more and more under the spirit of his age. 'What a genius I had when I wrote that book!' was the pathetic exclamation of a famous English author, when in old age he chanced to take up one of his early works. There would be nothing surprising again in his losing somewhat of his powers of expression, and becoming less capable of framing language into a harmonious whole. There would also be a strong presumption that if the variation of style was uniform, it was attributable to some natural cause, and not to the arts of the imitator. The inferiority might be the result of feebleness and of want of activity of mind. But the natural weakness of a great author would commonly be different from the artificial weakness of an imitator; it would be continuous and uniform. The latter would be apt to fill his work with irregular patches, sometimes taken verbally from the writings of the author whom he personated, but rarely acquiring his spirit. His imitation would be obvious, irregular, superficial. The patches of purple would be easily detected among his threadbare and tattered garments. He would rarely take the pains to put the same thought into other words. There were many forgeries in English literature which attained a considerable degree of success 50 or 100 years ago; but it is doubtful whether attempts such as these could now escape detection, if there were any writings of the same author or of the same age to be compared with them. And ancient forgers were much less skilful than modern; they were far from being masters in the art of deception, and had rarely any motive for being so.

\par  (b) But, secondly, the imitator will commonly be least capable of understanding or imitating that part of a great writer which is most characteristic of him. In every man's writings there is something like himself and unlike others, which gives individuality. To appreciate this latent quality would require a kindred mind, and minute study and observation. There are a class of similarities which may be called undesigned coincidences, which are so remote as to be incapable of being borrowed from one another, and yet, when they are compared, find a natural explanation in their being the work of the same mind. The imitator might copy the turns of style—he might repeat images or illustrations, but he could not enter into the inner circle of Platonic philosophy. He would understand that part of it which became popular in the next generation, as for example, the doctrine of ideas or of numbers: he might approve of communism. But the higher flights of Plato about the science of dialectic, or the unity of virtue, or a person who is above the law, would be unintelligible to him.

\par  (c) The argument from imitation assumes a different character when the supposed imitations are associated with other passages having the impress of original genius. The strength of the argument from undesigned coincidences of style is much increased when they are found side by side with thoughts and expressions which can only have come from a great original writer. The great excellence, not only of the whole, but even of the parts of writings, is a strong proof of their genuineness—for although the great writer may fall below, the forger or imitator cannot rise much above himself. Whether we can attribute the worst parts of a work to a forger and the best to a great writer,—as for example, in the case of some of Shakespeare's plays,—depends upon the probability that they have been interpolated, or have been the joint work of two writers; and this can only be established either by express evidence or by a comparison of other writings of the same class. If the interpolation or double authorship of Greek writings in the time of Plato could be shown to be common, then a question, perhaps insoluble, would arise, not whether the whole, but whether parts of the Platonic dialogues are genuine, and, if parts only, which parts. Hebrew prophecies and Homeric poems and Laws of Manu may have grown together in early times, but there is no reason to think that any of the dialogues of Plato is the result of a similar process of accumulation. It is therefore rash to say with Oncken (Die Staatslehre des Aristoteles) that the form in which Aristotle knew the Laws of Plato must have been different from that in which they have come down to us.

\par  It must be admitted that these principles are difficult of application. Yet a criticism may be worth making which rests only on probabilities or impressions. Great disputes will arise about the merits of different passages, about what is truly characteristic and original or trivial and borrowed. Many have thought the Laws to be one of the greatest of Platonic writings, while in the judgment of Mr. Grote they hardly rise above the level of the forged epistles. The manner in which a writer would or would not have written at a particular time of life must be acknowledged to be a matter of conjecture. But enough has been said to show that similarities of a certain kind, whether criticism is able to detect them or not, may be such as must be attributed to an original writer, and not to a mere imitator.

\par  (d) Applying these principles to the case of the Laws, we have now to point out that they contain the class of refined or unconscious similarities which are indicative of genuineness. The parallelisms are like the repetitions of favourite thoughts into which every one is apt to fall unawares in conversation or in writing. They are found in a work which contains many beautiful and remarkable passages. We may therefore begin by claiming this presumption in their favour. Such undesigned coincidences, as we may venture to call them, are the following. The conception of justice as the union of temperance, wisdom, courage (Laws; Republic): the latent idea of dialectic implied in the notion of dividing laws after the kinds of virtue (Laws); the approval of the method of looking at one idea gathered from many things, 'than which a truer was never discovered by any man' (compare Republic): or again the description of the Laws as parents (Laws; Republic): the assumption that religion has been already settled by the oracle of Delphi (Laws; Republic), to which an appeal is also made in special cases (Laws): the notion of the battle with self, a paradox for which Plato in a manner apologizes both in the Laws and the Republic: the remark (Laws) that just men, even when they are deformed in body, may still be perfectly beautiful in respect of the excellent justice of their minds (compare Republic): the argument that ideals are none the worse because they cannot be carried out (Laws; Republic): the near approach to the idea of good in 'the principle which is common to all the four virtues,' a truth which the guardians must be compelled to recognize (Laws; compare Republic): or again the recognition by reason of the right pleasure and pain, which had previously been matter of habit (Laws; Republic): or the blasphemy of saying that the excellency of music is to give pleasure (Laws; Republic): again the story of the Sidonian Cadmus (Laws), which is a variation of the Phoenician tale of the earth-born men (Republic): the comparison of philosophy to a yelping she-dog, both in the Republic and in the Laws: the remark that no man can practise two trades (Laws; Republic): or the advantage of the middle condition (Laws; Republic): the tendency to speak of principles as moulds or forms; compare the ekmageia of song (Laws), and the tupoi of religion (Republic): or the remark (Laws) that 'the relaxation of justice makes many cities out of one,' which may be compared with the Republic: or the description of lawlessness 'creeping in little by little in the fashions of music and overturning all things,'—to us a paradox, but to Plato's mind a fixed idea, which is found in the Laws as well as in the Republic: or the figure of the parts of the human body under which the parts of the state are described (Laws; Republic): the apology for delay and diffuseness, which occurs not unfrequently in the Republic, is carried to an excess in the Laws (compare Theaet. ): the remarkable thought (Laws) that the soul of the sun is better than the sun, agrees with the relation in which the idea of good stands to the sun in the Republic, and with the substitution of mind for the idea of good in the Philebus: the passage about the tragic poets (Laws) agrees generally with the treatment of them in the Republic, but is more finely conceived, and worked out in a nobler spirit. Some lesser similarities of thought and manner should not be omitted, such as the mention of the thirty years' old students in the Republic, and the fifty years' old choristers in the Laws; or the making of the citizens out of wax (Laws) compared with the other image (Republic); or the number of the tyrant (729), which is NEARLY equal with the number of days and nights in the year (730), compared with the 'slight correction' of the sacred number 5040, which is divisible by all the numbers from 1 to 12 except 11, and divisible by 11, if two families be deducted; or once more, we may compare the ignorance of solid geometry of which he complains in the Republic and the puzzle about fractions with the difficulty in the Laws about commensurable and incommensurable quantities—and the malicious emphasis on the word gunaikeios (Laws) with the use of the same word (Republic). These and similar passages tend to show that the author of the Republic is also the author of the Laws. They are echoes of the same voice, expressions of the same mind, coincidences too subtle to have been invented by the ingenuity of any imitator. The force of the argument is increased, if we remember that no passage in the Laws is exactly copied,—nowhere do five or six words occur together which are found together elsewhere in Plato's writings.

\par  In other dialogues of Plato, as well as in the Republic, there are to be found parallels with the Laws. Such resemblances, as we might expect, occur chiefly (but not exclusively) in the dialogues which, on other grounds, we may suppose to be of later date. The punishment of evil is to be like evil men (Laws), as he says also in the Theaetetus. Compare again the dependence of tragedy and comedy on one another, of which he gives the reason in the Laws—'For serious things cannot be understood without laughable, nor opposites at all without opposites, if a man is really to have intelligence of either'; here he puts forward the principle which is the groundwork of the thesis of Socrates in the Symposium, 'that the genius of tragedy is the same as that of comedy, and that the writer of comedy ought to be a writer of tragedy also.' There is a truth and right which is above Law (Laws), as we learn also from the Statesman. That men are the possession of the Gods (Laws), is a reflection which likewise occurs in the Phaedo. The remark, whether serious or ironical (Laws), that 'the sons of the Gods naturally believed in the Gods, because they had the means of knowing about them,' is found in the Timaeus. The reign of Cronos, who is the divine ruler (Laws), is a reminiscence of the Statesman. It is remarkable that in the Sophist and Statesman (Soph. ), Plato, speaking in the character of the Eleatic Stranger, has already put on the old man. The madness of the poets, again, is a favourite notion of Plato's, which occurs also in the Laws, as well as in the Phaedrus, Ion, and elsewhere. There are traces in the Laws of the same desire to base speculation upon history which we find in the Critias. Once more, there is a striking parallel with the paradox of the Gorgias, that 'if you do evil, it is better to be punished than to be unpunished,' in the Laws: 'To live having all goods without justice and virtue is the greatest of evils if life be immortal, but not so great if the bad man lives but a short time.'

\par  The point to be considered is whether these are the kind of parallels which would be the work of an imitator. Would a forger have had the wit to select the most peculiar and characteristic thoughts of Plato; would he have caught the spirit of his philosophy; would he, instead of openly borrowing, have half concealed his favourite ideas; would he have formed them into a whole such as the Laws; would he have given another the credit which he might have obtained for himself; would he have remembered and made use of other passages of the Platonic writings and have never deviated into the phraseology of them? Without pressing such arguments as absolutely certain, we must acknowledge that such a comparison affords a new ground of real weight for believing the Laws to be a genuine writing of Plato.

\par  V. The relation of the Republic to the Laws is clearly set forth by Plato in the Laws. The Republic is the best state, the Laws is the best possible under the existing conditions of the Greek world. The Republic is the ideal, in which no man calls anything his own, which may or may not have existed in some remote clime, under the rule of some God, or son of a God (who can say? ), but is, at any rate, the pattern of all other states and the exemplar of human life. The Laws distinctly acknowledge what the Republic partly admits, that the ideal is inimitable by us, but that we should 'lift up our eyes to the heavens' and try to regulate our lives according to the divine image. The citizens are no longer to have wives and children in common, and are no longer to be under the government of philosophers. But the spirit of communism or communion is to continue among them, though reverence for the sacredness of the family, and respect of children for parents, not promiscuous hymeneals, are now the foundation of the state; the sexes are to be as nearly on an equality as possible; they are to meet at common tables, and to share warlike pursuits (if the women will consent), and to have a common education. The legislator has taken the place of the philosopher, but a council of elders is retained, who are to fulfil the duties of the legislator when he has passed out of life. The addition of younger persons to this council by co-optation is an improvement on the governing body of the Republic. The scheme of education in the Laws is of a far lower kind than that which Plato had conceived in the Republic. There he would have his rulers trained in all knowledge meeting in the idea of good, of which the different branches of mathematical science are but the hand-maidens or ministers; here he treats chiefly of popular education, stopping short with the preliminary sciences,—these are to be studied partly with a view to their practical usefulness, which in the Republic he holds cheap, and even more with a view to avoiding impiety, of which in the Republic he says nothing; he touches very lightly on dialectic, which is still to be retained for the rulers. Yet in the Laws there remain traces of the old educational ideas. He is still for banishing the poets; and as he finds the works of prose writers equally dangerous, he would substitute for them the study of his own laws. He insists strongly on the importance of mathematics as an educational instrument. He is no more reconciled to the Greek mythology than in the Republic, though he would rather say nothing about it out of a reverence for antiquity; and he is equally willing to have recourse to fictions, if they have a moral tendency. His thoughts recur to a golden age in which the sanctity of oaths was respected and in which men living nearer the Gods were more disposed to believe in them; but we must legislate for the world as it is, now that the old beliefs have passed away. Though he is no longer fired with dialectical enthusiasm, he would compel the guardians to 'look at one idea gathered from many things,' and to 'perceive the principle which is the same in all the four virtues.' He still recognizes the enormous influence of music, in which every youth is to be trained for three years; and he seems to attribute the existing degeneracy of the Athenian state and the laxity of morals partly to musical innovation, manifested in the unnatural divorce of the instrument and the voice, of the rhythm from the words, and partly to the influence of the mob who ruled at the theatres. He assimilates the education of the two sexes, as far as possible, both in music and gymnastic, and, as in the Republic, he would give to gymnastic a purely military character. In marriage, his object is still to produce the finest children for the state. As in the Statesman, he would unite in wedlock dissimilar natures—the passionate with the dull, the courageous with the gentle. And the virtuous tyrant of the Statesman, who has no place in the Republic, again appears. In this, as in all his writings, he has the strongest sense of the degeneracy and incapacity of the rulers of his own time.

\par  In the Laws, the philosophers, if not banished, like the poets, are at least ignored; and religion takes the place of philosophy in the regulation of human life. It must however be remembered that the religion of Plato is co-extensive with morality, and is that purified religion and mythology of which he speaks in the second book of the Republic. There is no real discrepancy in the two works. In a practical treatise, he speaks of religion rather than of philosophy; just as he appears to identify virtue with pleasure, and rather seeks to find the common element of the virtues than to maintain his old paradoxical theses that they are one, or that they are identical with knowledge. The dialectic and the idea of good, which even Glaucon in the Republic could not understand, would be out of place in a less ideal work. There may also be a change in his own mind, the purely intellectual aspect of philosophy having a diminishing interest to him in his old age.

\par  Some confusion occurs in the passage in which Plato speaks of the Republic, occasioned by his reference to a third state, which he proposes (D.V.) hereafter to expound. Like many other thoughts in the Laws, the allusion is obscure from not being worked out. Aristotle (Polit.) speaks of a state which is neither the best absolutely, nor the best under existing conditions, but an imaginary state, inferior to either, destitute, as he supposes, of the necessaries of life—apparently such a beginning of primitive society as is described in Laws iii. But it is not clear that by this the third state of Plato is intended. It is possible that Plato may have meant by his third state an historical sketch, bearing the same relation to the Laws which the unfinished Critias would have borne to the Republic; or he may, perhaps, have intended to describe a state more nearly approximating than the Laws to existing Greek states.

\par  The Statesman is a mere fragment when compared with the Laws, yet combining a second interest of dialectic as well as politics, which is wanting in the larger work. Several points of similarity and contrast may be observed between them. In some respects the Statesman is even more ideal than the Republic, looking back to a former state of paradisiacal life, in which the Gods ruled over mankind, as the Republic looks forward to a coming kingdom of philosophers. Of this kingdom of Cronos there is also mention in the Laws. Again, in the Statesman, the Eleatic Stranger rises above law to the conception of the living voice of the lawgiver, who is able to provide for individual cases. A similar thought is repeated in the Laws: 'If in the order of nature, and by divine destiny, a man were able to apprehend the truth about these things, he would have no need of laws to rule over him; for there is no law or order above knowledge, nor can mind without impiety be deemed the subject or slave of any, but rather the lord of all.' The union of opposite natures, who form the warp and the woof of the political web, is a favourite thought which occurs in both dialogues (Laws; Statesman).

\par  The Laws are confessedly a Second-best, an inferior Ideal, to which Plato has recourse, when he finds that the city of Philosophers is no longer 'within the horizon of practical politics.' But it is curious to observe that the higher Ideal is always returning (compare Arist. Polit. ), and that he is not much nearer the actual fact, nor more on the level of ordinary life in the Laws than in the Republic. It is also interesting to remark that the new Ideal is always falling away, and that he hardly supposes the one to be more capable of being realized than the other. Human beings are troublesome to manage; and the legislator cannot adapt his enactments to the infinite variety of circumstances; after all he must leave the administration of them to his successors; and though he would have liked to make them as permanent as they are in Egypt, he cannot escape from the necessity of change. At length Plato is obliged to institute a Nocturnal Council which is supposed to retain the mind of the legislator, and of which some of the members are even supposed to go abroad and inspect the institutions of foreign countries, as a foundation for changes in their own. The spirit of such changes, though avoiding the extravagance of a popular assembly, being only so much change as the conservative temper of old members is likely to allow, is nevertheless inconsistent with the fixedness of Egypt which Plato wishes to impress upon Hellenic institutions. He is inconsistent with himself as the truth begins to dawn upon him that 'in the execution things for the most part fall short of our conception of them' (Republic).

\par  And is not this true of ideals of government in general? We are always disappointed in them. Nothing great can be accomplished in the short space of human life; wherefore also we look forward to another (Republic). As we grow old, we are sensible that we have no power actively to pursue our ideals any longer. We have had our opportunity and do not aspire to be more than men: we have received our 'wages and are going home.' Neither do we despair of the future of mankind, because we have been able to do so little in comparison of the whole. We look in vain for consistency either in men or things. But we have seen enough of improvement in our own time to justify us in the belief that the world is worth working for and that a good man's life is not thrown away. Such reflections may help us to bring home to ourselves by inward sympathy the language of Plato in the Laws, and to combine into something like a whole his various and at first sight inconsistent utterances.

\par  VI. The Republic may be described as the Spartan constitution appended to a government of philosophers. But in the Laws an Athenian element is also introduced. Many enactments are taken from the Athenian; the four classes are borrowed from the constitution of Cleisthenes, which Plato regards as the best form of Athenian government, and the guardians of the law bear a certain resemblance to the archons. In the constitution of the Laws nearly all officers are elected by a vote more or less popular and by lot. But the assembly only exists for the purposes of election, and has no legislative or executive powers. The Nocturnal Council, which is the highest body in the state, has several of the functions of the ancient Athenian Areopagus, after which it appears to be modelled. Life is to wear, as at Athens, a joyous and festive look; there are to be Bacchic choruses, and men of mature age are encouraged in moderate potations. On the other hand, the common meals, the public education, the crypteia are borrowed from Sparta and not from Athens, and the superintendence of private life, which was to be practised by the governors, has also its prototype in Sparta. The extravagant dislike which Plato shows both to a naval power and to extreme democracy is the reverse of Athenian.

\par  The best-governed Hellenic states traced the origin of their laws to individual lawgivers. These were real persons, though we are uncertain how far they originated or only modified the institutions which are ascribed to them. But the lawgiver, though not a myth, was a fixed idea in the mind of the Greek,—as fixed as the Trojan war or the earth-born Cadmus. 'This was what Solon meant or said'—was the form in which the Athenian expressed his own conception of right and justice, or argued a disputed point of law. And the constant reference in the Laws of Plato to the lawgiver is altogether in accordance with Greek modes of thinking and speaking.

\par  There is also, as in the Republic, a Pythagorean element. The highest branch of education is arithmetic; to know the order of the heavenly bodies, and to reconcile the apparent contradiction of their movements, is an important part of religion; the lives of the citizens are to have a common measure, as also their vessels and coins; the great blessing of the state is the number 5040. Plato is deeply impressed by the antiquity of Egypt, and the unchangeableness of her ancient forms of song and dance. And he is also struck by the progress which the Egyptians had made in the mathematical sciences—in comparison of them the Greeks appeared to him to be little better than swine. Yet he censures the Egyptian meanness and inhospitality to strangers. He has traced the growth of states from their rude beginnings in a philosophical spirit; but of any life or growth of the Hellenic world in future ages he is silent. He has made the reflection that past time is the maker of states (Book iii. ); but he does not argue from the past to the future, that the process is always going on, or that the institutions of nations are relative to their stage of civilization. If he could have stamped indelibly upon Hellenic states the will of the legislator, he would have been satisfied. The utmost which he expects of future generations is that they should supply the omissions, or correct the errors which younger statesmen detect in his enactments. When institutions have been once subjected to this process of criticism, he would have them fixed for ever.

\par  THE PREAMBLE.

\par  BOOK I. Strangers, let me ask a question of you—Was a God or a man the author of your laws? 'A God, Stranger. In Crete, Zeus is said to have been the author of them; in Sparta, as Megillus will tell you, Apollo.' You Cretans believe, as Homer says, that Minos went every ninth year to converse with his Olympian sire, and gave you laws which he brought from him. 'Yes; and there was Rhadamanthus, his brother, who is reputed among us to have been a most righteous judge.' That is a reputation worthy of the son of Zeus. And as you and Megillus have been trained under these laws, I may ask you to give me an account of them. We can talk about them in our walk from Cnosus to the cave and temple of Zeus. I am told that the distance is considerable, but probably there are shady places under the trees, where, being no longer young, we may often rest and converse. 'Yes, Stranger, a little onward there are beautiful groves of cypresses, and green meadows in which we may repose.'

\par  My first question is, Why has the law ordained that you should have common meals, and practise gymnastics, and bear arms? 'My answer is, that all our institutions are of a military character. We lead the life of the camp even in time of peace, keeping up the organization of an army, and having meals in common; and as our country, owing to its ruggedness, is ill-suited for heavy-armed cavalry or infantry, our soldiers are archers, equipped with bows and arrows. The legislator was under the idea that war was the natural state of all mankind, and that peace is only a pretence; he thought that no possessions had any value which were not secured against enemies.' And do you think that superiority in war is the proper aim of government? 'Certainly I do, and my Spartan friend will agree with me.' And are there wars, not only of state against state, but of village against village, of family against family, of individual against individual? 'Yes.' And is a man his own enemy? 'There you come to first principles, like a true votary of the goddess Athene; and this is all the better, for you will the sooner recognize the truth of what I am saying—that all men everywhere are the enemies of all, and each individual of every other and of himself; and, further, that there is a victory and defeat—the best and the worst—which each man sustains, not at the hands of another, but of himself.' And does this extend to states and villages as well as to individuals? 'Certainly; there is a better in them which conquers or is conquered by the worse.' Whether the worse ever really conquers the better, is a question which may be left for the present; but your meaning is, that bad citizens do sometimes overcome the good, and that the state is then conquered by herself, and that when they are defeated the state is victorious over herself. Or, again, in a family there may be several brothers, and the bad may be a majority; and when the bad majority conquer the good minority, the family are worse than themselves. The use of the terms 'better or worse than himself or themselves' may be doubtful, but about the thing meant there can be no dispute. 'Very true.' Such a struggle might be determined by a judge. And which will be the better judge—he who destroys the worse and lets the better rule, or he who lets the better rule and makes the others voluntarily obey; or, thirdly, he who destroys no one, but reconciles the two parties? 'The last, clearly.' But the object of such a judge or legislator would not be war. 'True.' And as there are two kinds of war, one without and one within a state, of which the internal is by far the worse, will not the legislator chiefly direct his attention to this latter? He will reconcile the contending factions, and unite them against their external enemies. 'Certainly.' Every legislator will aim at the greatest good, and the greatest good is not victory in war, whether civil or external, but mutual peace and good-will, as in the body health is preferable to the purgation of disease. He who makes war his object instead of peace, or who pursues war except for the sake of peace, is not a true statesman. 'And yet, Stranger, the laws both of Crete and Sparta aim entirely at war.' Perhaps so; but do not let us quarrel about your legislators—let us be gentle; they were in earnest quite as much as we are, and we must try to discover their meaning. The poet Tyrtaeus (you know his poems in Crete, and my Lacedaemonian friend is only too familiar with them)—he was an Athenian by birth, and a Spartan citizen:—'Well,' he says, 'I sing not, I care not about any man, however rich or happy, unless he is brave in war.' Now I should like, in the name of us all, to ask the poet a question. Oh Tyrtaeus, I would say to him, we agree with you in praising those who excel in war, but which kind of war do you mean?—that dreadful war which is termed civil, or the milder sort which is waged against foreign enemies? You say that you abominate 'those who are not eager to taste their enemies' blood,' and you seem to mean chiefly their foreign enemies. 'Certainly he does.' But we contend that there are men better far than your heroes, Tyrtaeus, concerning whom another poet, Theognis the Sicilian, says that 'in a civil broil they are worth their weight in gold and silver.' For in a civil war, not only courage, but justice and temperance and wisdom are required, and all virtue is better than a part. The mercenary soldier is ready to die at his post; yet he is commonly a violent, senseless creature. And the legislator, whether inspired or uninspired, will make laws with a view to the highest virtue; and this is not brute courage, but loyalty in the hour of danger. The virtue of Tyrtaeus, although needful enough in his own time, is really of a fourth-rate description. 'You are degrading our legislator to a very low level.' Nay, we degrade not him, but ourselves, if we believe that the laws of Lycurgus and Minos had a view to war only. A divine lawgiver would have had regard to all the different kinds of virtue, and have arranged his laws in corresponding classes, and not in the modern fashion, which only makes them after the want of them is felt,—about inheritances and heiresses and assaults, and the like. As you truly said, virtue is the business of the legislator; but you went wrong when you referred all legislation to a part of virtue, and to an inferior part. For the object of laws, whether the Cretan or any other, is to make men happy. Now happiness or good is of two kinds—there are divine and there are human goods. He who has the divine has the human added to him; but he who has lost the greater is deprived of both. The lesser goods are health, beauty, strength, and, lastly, wealth; not the blind God, Pluto, but one who has eyes to see and follow wisdom. For mind or wisdom is the most divine of all goods; and next comes temperance, and justice springs from the union of wisdom and temperance with courage, which is the fourth or last. These four precede other goods, and the legislator will arrange all his ordinances accordingly, the human going back to the divine, and the divine to their leader mind. There will be enactments about marriage, about education, about all the states and feelings and experiences of men and women, at every age, in weal and woe, in war and peace; upon all the law will fix a stamp of praise and blame. There will also be regulations about property and expenditure, about contracts, about rewards and punishments, and finally about funeral rites and honours of the dead. The lawgiver will appoint guardians to preside over these things; and mind will harmonize his ordinances, and show them to be in agreement with temperance and justice. Now I want to know whether the same principles are observed in the laws of Lycurgus and Minos, or, as I should rather say, of Apollo and Zeus. We must go through the virtues, beginning with courage, and then we will show that what has preceded has relation to virtue.

\par  'I wish,' says the Lacedaemonian, 'that you, Stranger, would first criticize Cleinias and the Cretan laws.' Yes, is the reply, and I will criticize you and myself, as well as him. Tell me, Megillus, were not the common meals and gymnastic training instituted by your legislator with a view to war? 'Yes; and next in the order of importance comes hunting, and fourth the endurance of pain in boxing contests, and in the beatings which are the punishment of theft. There is, too, the so-called Crypteia or secret service, in which our youth wander about the country night and day unattended, and even in winter go unshod and have no beds to lie on. Moreover they wrestle and exercise under a blazing sun, and they have many similar customs.' Well, but is courage only a combat against fear and pain, and not against pleasure and flattery? 'Against both, I should say.' And which is worse,—to be overcome by pain, or by pleasure? 'The latter.' But did the lawgivers of Crete and Sparta legislate for a courage which is lame of one leg,—able to meet the attacks of pain but not those of pleasure, or for one which can meet both? 'For a courage which can meet both, I should say.' But if so, where are the institutions which train your citizens to be equally brave against pleasure and pain, and superior to enemies within as well as without? 'We confess that we have no institutions worth mentioning which are of this character.' I am not surprised, and will therefore only request forbearance on the part of us all, in case the love of truth should lead any of us to censure the laws of the others. Remember that I am more in the way of hearing criticisms of your laws than you can be; for in well-ordered states like Crete and Sparta, although an old man may sometimes speak of them in private to a ruler or elder, a similar liberty is not allowed to the young. But now being alone we shall not offend your legislator by a friendly examination of his laws. 'Take any freedom which you like.'

\par  My first observation is, that your lawgiver ordered you to endure hardships, because he thought that those who had not this discipline would run away from those who had. But he ought to have considered further, that those who had never learned to resist pleasure would be equally at the mercy of those who had, and these are often among the worst of mankind. Pleasure, like fear, would overcome them and take away their courage and freedom. 'Perhaps; but I must not be hasty in giving my assent.'

\par  Next as to temperance: what institutions have you which are adapted to promote temperance? 'There are the common meals and gymnastic exercises.' These are partly good and partly bad, and, as in medicine, what is good at one time and for one person, is bad at another time and for another person. Now although gymnastics and common meals do good, they are also a cause of evil in civil troubles, and they appear to encourage unnatural love, as has been shown at Miletus, in Boeotia, and at Thurii. And the Cretans are said to have invented the tale of Zeus and Ganymede in order to justify their evil practices by the example of the God who was their lawgiver. Leaving the story, we may observe that all law has to do with pleasure and pain; these are two fountains which are ever flowing in human nature, and he who drinks of them when and as much as he ought, is happy, and he who indulges in them to excess, is miserable. 'You may be right, but I still incline to think that the Lacedaemonian lawgiver did well in forbidding pleasure, if I may judge from the result. For there is no drunken revelry in Sparta, and any one found in a state of intoxication is severely punished; he is not excused as an Athenian would be at Athens on account of a festival. I myself have seen the Athenians drunk at the Dionysia—and at our colony, Tarentum, on a similar occasion, I have beheld the whole city in a state of intoxication.' I admit that these festivals should be properly regulated. Yet I might reply, 'Yes, Spartans, that is not your vice; but look at home and remember the licentiousness of your women.' And to all such accusations every one of us may reply in turn:—'Wonder not, Stranger; there are different customs in different countries.' Now this may be a sufficient answer; but we are speaking about the wisdom of lawgivers and not about the customs of men. To return to the question of drinking: shall we have total abstinence, as you have, or hard drinking, like the Scythians and Thracians, or moderate potations like the Persians? 'Give us arms, and we send all these nations flying before us.' My good friend, be modest; victories and defeats often arise from unknown causes, and afford no proof of the goodness or badness of institutions. The stronger overcomes the weaker, as the Athenians have overcome the Ceans, or the Syracusans the Locrians, who are, perhaps, the best governed state in that part of the world. People are apt to praise or censure practices without enquiring into the nature of them. This is the way with drink: one person brings many witnesses, who sing the praises of wine; another declares that sober men defeat drunkards in battle; and he again is refuted in turn. I should like to conduct the argument on some other method; for if you regard numbers, there are two cities on one side, and ten thousand on the other. 'I am ready to pursue any method which is likely to lead us to the truth.' Let me put the matter thus: Somebody praises the useful qualities of a goat; another has seen goats running about wild in a garden, and blames a goat or any other animal which happens to be without a keeper. 'How absurd!' Would a pilot who is sea-sick be a good pilot? 'No.' Or a general who is sick and drunk with fear and ignorant of war a good general? 'A general of old women he ought to be.' But can any one form an estimate of any society, which is intended to have a ruler, and which he only sees in an unruly and lawless state? 'No.' There is a convivial form of society—is there not? 'Yes.' And has this convivial society ever been rightly ordered? Of course you Spartans and Cretans have never seen anything of the kind, but I have had wide experience, and made many enquiries about such societies, and have hardly ever found anything right or good in them. 'We acknowledge our want of experience, and desire to learn of you.' Will you admit that in all societies there must be a leader? 'Yes.' And in time of war he must be a man of courage and absolutely devoid of fear, if this be possible? 'Certainly.' But we are talking now of a general who shall preside at meetings of friends—and as these have a tendency to be uproarious, they ought above all others to have a governor. 'Very good.' He should be a sober man and a man of the world, who will keep, make, and increase the peace of the society; a drunkard in charge of drunkards would be singularly fortunate if he avoided doing a serious mischief. 'Indeed he would.' Suppose a person to censure such meetings—he may be right, but also he may have known them only in their disorderly state, under a drunken master of the feast; and a drunken general or pilot cannot save his army or his ships. 'True; but although I see the advantage of an army having a good general, I do not equally see the good of a feast being well managed.' If you mean to ask what good accrues to the state from the right training of a single youth or a single chorus, I should reply, 'Not much'; but if you ask what is the good of education in general, I answer, that education makes good men, and that good men act nobly and overcome their enemies in battle. Victory is often suicidal to the victors, because it creates forgetfulness of education, but education itself is never suicidal. 'You imply that the regulation of convivial meetings is a part of education; how will you prove this?' I will tell you. But first let me offer a word of apology. We Athenians are always thought to be fond of talking, whereas the Lacedaemonian is celebrated for brevity, and the Cretan is considered to be sagacious and reserved. Now I fear that I may be charged with spinning a long discourse out of slender materials. For drinking cannot be rightly ordered without correct principles of music, and music runs up into education generally, and to discuss all these matters may be tedious; if you like, therefore, we will pass on to another part of our subject. 'Are you aware, Athenian, that our family is your proxenus at Sparta, and that from my boyhood I have regarded Athens as a second country, and having often fought your battles in my youth, I have become attached to you, and love the sound of the Attic dialect? The saying is true, that the best Athenians are more than ordinarily good, because they are good by nature; therefore, be assured that I shall be glad to hear you talk as much as you please.' 'I, too,' adds Cleinias, 'have a tie which binds me to you. You know that Epimenides, the Cretan prophet, came and offered sacrifices in your city by the command of an oracle ten years before the Persian war. He told the Athenians that the Persian host would not come for ten years, and would go away again, having suffered more harm than they had inflicted. Now Epimenides was of my family, and when he visited Athens he entered into friendship with your forefathers.' I see that you are willing to listen, and I have the will to speak, if I had only the ability. But, first, I must define the nature and power of education, and by this road we will travel on to the God Dionysus. The man who is to be good at anything must have early training;—the future builder must play at building, and the husbandman at digging; the soldier must learn to ride, and the carpenter to measure and use the rule,—all the thoughts and pleasures of children should bear on their after-profession.—Do you agree with me? 'Certainly.' And we must remember further that we are speaking of the education, not of a trainer, or of the captain of a ship, but of a perfect citizen who knows how to rule and how to obey; and such an education aims at virtue, and not at wealth or strength or mere cleverness. To the good man, education is of all things the most precious, and is also in constant need of renovation. 'We agree.' And we have before agreed that good men are those who are able to control themselves, and bad men are those who are not. Let me offer you an illustration which will assist our argument. Man is one; but in one and the same man are two foolish counsellors who contend within him—pleasure and pain, and of either he has expectations which we call hope and fear; and he is able to reason about good and evil, and reason, when affirmed by the state, becomes law. 'We cannot follow you.' Let me put the matter in another way: Every creature is a puppet of the Gods—whether he is a mere plaything or has any serious use we do not know; but this we do know, that he is drawn different ways by cords and strings. There is a soft golden cord which draws him towards virtue—this is the law of the state; and there are other cords made of iron and hard materials drawing him other ways. The golden reasoning influence has nothing of the nature of force, and therefore requires ministers in order to vanquish the other principles. This explains the doctrine that cities and citizens both conquer and are conquered by themselves. The individual follows reason, and the city law, which is embodied reason, either derived from the Gods or from the legislator. When virtue and vice are thus distinguished, education will be better understood, and in particular the relation of education to convivial intercourse. And now let us set wine before the puppet. You admit that wine stimulates the passions? 'Yes.' And does wine equally stimulate the reasoning faculties? 'No; it brings the soul back to a state of childhood.' In such a state a man has the least control over himself, and is, therefore, worst. 'Very true.' Then how can we believe that drinking should be encouraged? 'You seem to think that it ought to be.' And I am ready to maintain my position. 'We should like to hear you prove that a man ought to make a beast of himself.' You are speaking of the degradation of the soul: but how about the body? Would any man willingly degrade or weaken that? 'Certainly not.' And yet if he goes to a doctor or a gymnastic master, does he not make himself ill in the hope of getting well? for no one would like to be always taking medicine, or always to be in training. 'True.' And may not convivial meetings have a similar remedial use? And if so, are they not to be preferred to other modes of training because they are painless? 'But have they any such use?' Let us see: Are there not two kinds of fear—fear of evil and fear of an evil reputation? 'There are.' The latter kind of fear is opposed both to the fear of pain and to the love of pleasure. This is called by the legislator reverence, and is greatly honoured by him and by every good man; whereas confidence, which is the opposite quality, is the worst fault both of individuals and of states. This sort of fear or reverence is one of the two chief causes of victory in war, fearlessness of enemies being the other. 'True.' Then every one should be both fearful and fearless? 'Yes.' The right sort of fear is infused into a man when he comes face to face with shame, or cowardice, or the temptations of pleasure, and has to conquer them. He must learn by many trials to win the victory over himself, if he is ever to be made perfect. 'That is reasonable enough.' And now, suppose that the Gods had given mankind a drug, of which the effect was to exaggerate every sort of evil and danger, so that the bravest man entirely lost his presence of mind and became a coward for a time:—would such a drug have any value? 'But is there such a drug?' No; but suppose that there were; might not the legislator use such a mode of testing courage and cowardice? 'To be sure.' The legislator would induce fear in order to implant fearlessness; and would give rewards or punishments to those who behaved well or the reverse, under the influence of the drug? 'Certainly.' And this mode of training, whether practised in the case of one or many, whether in solitude or in the presence of a large company—if a man have sufficient confidence in himself to drink the potion amid his boon companions, leaving off in time and not taking too much,—would be an equally good test of temperance? 'Very true.' Let us return to the lawgiver and say to him, 'Well, lawgiver, no such fear-producing potion has been given by God or invented by man, but there is a potion which will make men fearless.' 'You mean wine.' Yes; has not wine an effect the contrary of that which I was just now describing,—first mellowing and humanizing a man, and then filling him with confidence, making him ready to say or do anything? 'Certainly.' Let us not forget that there are two qualities which should be cultivated in the soul—first, the greatest fearlessness, and, secondly, the greatest fear, which are both parts of reverence. Courage and fearlessness are trained amid dangers; but we have still to consider how fear is to be trained. We desire to attain fearlessness and confidence without the insolence and boldness which commonly attend them. For do not love, ignorance, avarice, wealth, beauty, strength, while they stimulate courage, also madden and intoxicate the soul? What better and more innocent test of character is there than festive intercourse? Would you make a bargain with a man in order to try whether he is honest? Or would you ascertain whether he is licentious by putting your wife or daughter into his hands? No one would deny that the test proposed is fairer, speedier, and safer than any other. And such a test will be particularly useful in the political science, which desires to know human natures and characters. 'Very true.'

\par  BOOK II. And are there any other uses of well-ordered potations? There are; but in order to explain them, I must repeat what I mean by right education; which, if I am not mistaken, depends on the due regulation of convivial intercourse. 'A high assumption.' I believe that virtue and vice are originally present to the mind of children in the form of pleasure and pain; reason and fixed principles come later, and happy is he who acquires them even in declining years; for he who possesses them is the perfect man. When pleasure and pain, and love and hate, are rightly implanted in the yet unconscious soul, and after the attainment of reason are discovered to be in harmony with her, this harmony of the soul is virtue, and the preparatory stage, anticipating reason, I call education. But the finer sense of pleasure and pain is apt to be impaired in the course of life; and therefore the Gods, pitying the toils and sorrows of mortals, have allowed them to have holidays, and given them the Muses and Apollo and Dionysus for leaders and playfellows. All young creatures love motion and frolic, and utter sounds of delight; but man only is capable of taking pleasure in rhythmical and harmonious movements. With these education begins; and the uneducated is he who has never known the discipline of the chorus, and the educated is he who has. The chorus is partly dance and partly song, and therefore the well-educated must sing and dance well. But when we say, 'He sings and dances well,' we mean that he sings and dances what is good. And if he thinks that to be good which is really good, he will have a much higher music and harmony in him, and be a far greater master of imitation in sound and gesture than he who is not of this opinion. 'True.' Then, if we know what is good and bad in song and dance, we shall know what education is? 'Very true.' Let us now consider the beauty of figure, melody, song, and dance. Will the same figures or sounds be equally well adapted to the manly and the cowardly when they are in trouble? 'How can they be, when the very colours of their faces are different?' Figures and melodies have a rhythm and harmony which are adapted to the expression of different feelings (I may remark, by the way, that the term 'colour,' which is a favourite word of music-masters, is not really applicable to music). And one class of harmonies is akin to courage and all virtue, the other to cowardice and all vice. 'We agree.' And do all men equally like all dances? 'Far otherwise.' Do some figures, then, appear to be beautiful which are not? For no one will admit that the forms of vice are more beautiful than the forms of virtue, or that he prefers the first kind to the second. And yet most persons say that the merit of music is to give pleasure. But this is impiety. There is, however, a more plausible account of the matter given by others, who make their likes or dislikes the criterion of excellence. Sometimes nature crosses habit, or conversely, and then they say that such and such fashions or gestures are pleasant, but they do not like to exhibit them before men of sense, although they enjoy them in private. 'Very true.' And do vicious measures and strains do any harm, or good measures any good to the lovers of them? 'Probably.' Say, rather 'Certainly': for the gentle indulgence which we often show to vicious men inevitably makes us become like them. And what can be worse than this? 'Nothing.' Then in a well-administered city, the poet will not be allowed to make the songs of the people just as he pleases, or to train his choruses without regard to virtue and vice. 'Certainly not.' And yet he may do this anywhere except in Egypt; for there ages ago they discovered the great truth which I am now asserting, that the young should be educated in forms and strains of virtue. These they fixed and consecrated in their temples; and no artist or musician is allowed to deviate from them. They are literally the same which they were ten thousand years ago. And this practice of theirs suggests the reflection that legislation about music is not an impossible thing. But the particular enactments must be the work of God or of some God-inspired man, as in Egypt their ancient chants are said to be the composition of the goddess Isis. The melodies which have a natural truth and correctness should be embodied in a law, and then the desire of novelty is not strong enough to change the old fashions. Is not the origin of music as follows? We rejoice when we think that we prosper, and we think that we prosper when we rejoice, and at such times we cannot rest, but our young men dance dances and sing songs, and our old men, who have lost the elasticity of youth, regale themselves with the memory of the past, while they contemplate the life and activity of the young. 'Most true.' People say that he who gives us most pleasure at such festivals is to win the palm: are they right? 'Possibly.' Let us not be hasty in deciding, but first imagine a festival at which the lord of the festival, having assembled the citizens, makes a proclamation that he shall be crowned victor who gives the most pleasure, from whatever source derived. We will further suppose that there are exhibitions of rhapsodists and musicians, tragic and comic poets, and even marionette-players—which of the pleasure-makers will win? Shall I answer for you?—the marionette-players will please the children; youths will decide for comedy; young men, educated women, and people in general will prefer tragedy; we old men are lovers of Homer and Hesiod. Now which of them is right? If you and I are asked, we shall certainly say that the old men's way of thinking ought to prevail. 'Very true.' So far I agree with the many that the excellence of music is to be measured by pleasure; but then the pleasure must be that of the good and educated, or better still, of one supremely virtuous and educated man. The true judge must have both wisdom and courage. For he must lead the multitude and not be led by them, and must not weakly yield to the uproar of the theatre, nor give false judgment out of that mouth which has just appealed to the Gods. The ancient custom of Hellas, which still prevails in Italy and Sicily, left the judgment to the spectators, but this custom has been the ruin of the poets, who seek only to please their patrons, and has degraded the audience by the representation of inferior characters. What is the inference? The same which we have often drawn, that education is the training of the young idea in what the law affirms and the elders approve. And as the soul of a child is too young to be trained in earnest, a kind of education has been invented which tempts him with plays and songs, as the sick are tempted by pleasant meats and drinks. And the wise legislator will compel the poet to express in his poems noble thoughts in fitting words and rhythms. 'But is this the practice elsewhere than in Crete and Lacedaemon? In other states, as far as I know, dances and music are constantly changed at the pleasure of the hearers.' I am afraid that I misled you; not liking to be always finding fault with mankind as they are, I described them as they ought to be. But let me understand: you say that such customs exist among the Cretans and Lacedaemonians, and that the rest of the world would be improved by adopting them? 'Much improved.' And you compel your poets to declare that the righteous are happy, and that the wicked man, even if he be as rich as Midas, is unhappy? Or, in the words of Tyrtaeus, 'I sing not, I care not about him' who is a great warrior not having justice; if he be unjust, 'I would not have him look calmly upon death or be swifter than the wind'; and may he be deprived of every good—that is, of every true good. For even if he have the goods which men regard, these are not really goods: first health; beauty next; thirdly wealth; and there are others. A man may have every sense purged and improved; he may be a tyrant, and do what he likes, and live for ever: but you and I will maintain that all these things are goods to the just, but to the unjust the greatest of evils, if life be immortal; not so great if he live for a short time only. If a man had health and wealth, and power, and was insolent and unjust, his life would still be miserable; he might be fair and rich, and do what he liked, but he would live basely, and if basely evilly, and if evilly painfully. 'There I cannot agree with you.' Then may heaven give us the spirit of agreement, for I am as convinced of the truth of what I say as that Crete is an island; and, if I were a lawgiver, I would exercise a censorship over the poets, and I would punish them if they said that the wicked are happy, or that injustice is profitable. And these are not the only matters in which I should make my citizens talk in a different way to the world in general. If I asked Zeus and Apollo, the divine legislators of Crete and Sparta,—'Are the just and pleasant life the same or not the same'?—and they replied,—'Not the same'; and I asked again—'Which is the happier'? And they said'—'The pleasant life,' this is an answer not fit for a God to utter, and therefore I ought rather to put the same question to some legislator. And if he replies 'The pleasant,' then I should say to him, 'O my father, did you not tell me that I should live as justly as possible'? and if to be just is to be happy, what is that principle of happiness or good which is superior to pleasure? Is the approval of gods and men to be deemed good and honourable, but unpleasant, and their disapproval the reverse? Or is the neither doing nor suffering evil good and honourable, although not pleasant? But you cannot make men like what is not pleasant, and therefore you must make them believe that the just is pleasant. The business of the legislator is to clear up this confusion. He will show that the just and the unjust are identical with the pleasurable and the painful, from the point of view of the just man, of the unjust the reverse. And which is the truer judgment? Surely that of the better soul. For if not the truth, it is the best and most moral of fictions; and the legislator who desires to propagate this useful lie, may be encouraged by remarking that mankind have believed the story of Cadmus and the dragon's teeth, and therefore he may be assured that he can make them believe anything, and need only consider what fiction will do the greatest good. That the happiest is also the holiest, this shall be our strain, which shall be sung by all three choruses alike. First will enter the choir of children, who will lift up their voices on high; and after them the young men, who will pray the God Paean to be gracious to the youth, and to testify to the truth of their words; then will come the chorus of elder men, between thirty and sixty; and, lastly, there will be the old men, and they will tell stories enforcing the same virtues, as with the voice of an oracle. 'Whom do you mean by the third chorus?' You remember how I spoke at first of the restless nature of young creatures, who jumped about and called out in a disorderly manner, and I said that no other animal attained any perception of rhythm; but that to us the Gods gave Apollo and the Muses and Dionysus to be our playfellows. Of the two first choruses I have already spoken, and I have now to speak of the third, or Dionysian chorus, which is composed of those who are between thirty and sixty years old. 'Let us hear.' We are agreed (are we not?) that men, women, and children should be always charming themselves with strains of virtue, and that there should be a variety in the strains, that they may not weary of them? Now the fairest and most useful of strains will be uttered by the elder men, and therefore we cannot let them off. But how can we make them sing? For a discreet elderly man is ashamed to hear the sound of his own voice in private, and still more in public. The only way is to give them drink; this will mellow the sourness of age. No one should be allowed to taste wine until they are eighteen; from eighteen to thirty they may take a little; but when they have reached forty years, they may be initiated into the mystery of drinking. Thus they will become softer and more impressible; and when a man's heart is warm within him, he will be more ready to charm himself and others with song. And what songs shall he sing? 'At Crete and Lacedaemon we only know choral songs.' Yes; that is because your way of life is military. Your young men are like wild colts feeding in a herd together; no one takes the individual colt and trains him apart, and tries to give him the qualities of a statesman as well as of a soldier. He who was thus trained would be a greater warrior than those of whom Tyrtaeus speaks, for he would be courageous, and yet he would know that courage was only fourth in the scale of virtue. 'Once more, I must say, Stranger, that you run down our lawgivers.' Not intentionally, my good friend, but whither the argument leads I follow; and I am trying to find some style of poetry suitable for those who dislike the common sort. 'Very good.' In all things which have a charm, either this charm is their good, or they have some accompanying truth or advantage. For example, in eating and drinking there is pleasure and also profit, that is to say, health; and in learning there is a pleasure and also truth. There is a pleasure or charm, too, in the imitative arts, as well as a law of proportion or equality; but the pleasure which they afford, however innocent, is not the criterion of their truth. The test of pleasure cannot be applied except to that which has no other good or evil, no truth or falsehood. But that which has truth must be judged of by the standard of truth, and therefore imitation and proportion are to be judged of by their truth alone. 'Certainly.' And as music is imitative, it is not to be judged by the criterion of pleasure, and the Muse whom we seek is the muse not of pleasure but of truth, for imitation has a truth. 'Doubtless.' And if so, the judge must know what is being imitated before he decides on the quality of the imitation, and he who does not know what is true will not know what is good. 'He will not.' Will any one be able to imitate the human body, if he does not know the number, proportion, colour, or figure of the limbs? 'How can he?' But suppose we know some picture or figure to be an exact resemblance of a man, should we not also require to know whether the picture is beautiful or not? 'Quite right.' The judge of the imitation is required to know, therefore, first the original, secondly the truth, and thirdly the merit of the execution? 'True.' Then let us not weary in the attempt to bring music to the standard of the Muses and of truth. The Muses are not like human poets; they never spoil or mix rhythms or scales, or mingle instruments and human voices, or confuse the manners and strains of men and women, or of freemen and slaves, or of rational beings and brute animals. They do not practise the baser sorts of musical arts, such as the 'matured judgments,' of whom Orpheus speaks, would ridicule. But modern poets separate metre from music, and melody and rhythm from words, and use the instrument alone without the voice. The consequence is, that the meaning of the rhythm and of the time are not understood. I am endeavouring to show how our fifty-year-old choristers are to be trained, and what they are to avoid. The opinion of the multitude about these matters is worthless; they who are only made to step in time by sheer force cannot be critics of music. 'Impossible.' Then our newly-appointed minstrels must be trained in music sufficiently to understand the nature of rhythms and systems; and they should select such as are suitable to men of their age, and will enable them to give and receive innocent pleasure. This is a knowledge which goes beyond that either of the poets or of their auditors in general. For although the poet must understand rhythm and music, he need not necessarily know whether the imitation is good or not, which was the third point required in a judge; but our chorus of elders must know all three, if they are to be the instructors of youth.

\par  And now we will resume the original argument, which may be summed up as follows: A convivial meeting is apt to grow tumultuous as the drinking proceeds; every man becomes light-headed, and fancies that he can rule the whole world. 'Doubtless.' And did we not say that the souls of the drinkers, when subdued by wine, are made softer and more malleable at the hand of the legislator? the docility of childhood returns to them. At times however they become too valiant and disorderly, drinking out of their turn, and interrupting one another. And the business of the legislator is to infuse into them that divine fear, which we call shame, in opposition to this disorderly boldness. But in order to discipline them there must be guardians of the law of drinking, and sober generals who shall take charge of the private soldiers; they are as necessary in drinking as in fighting, and he who disobeys these Dionysiac commanders will be equally disgraced. 'Very good.' If a drinking festival were well regulated, men would go away, not as they now do, greater enemies, but better friends. Of the greatest gift of Dionysus I hardly like to speak, lest I should be misunderstood. 'What is that?' According to tradition Dionysus was driven mad by his stepmother Here, and in order to revenge himself he inspired mankind with Bacchic madness. But these are stories which I would rather not repeat. However I do acknowledge that all men are born in an imperfect state, and are at first restless, irrational creatures: this, as you will remember, has been already said by us. 'I remember.' And that Apollo and the Muses and Dionysus gave us harmony and rhythm? 'Very true.' The other story implies that wine was given to punish us and make us mad; but we contend that wine is a balm and a cure; a spring of modesty in the soul, and of health and strength in the body. Again, the work of the chorus is co-extensive with the work of education; rhythm and melody answer to the voice, and the motions of the body correspond to all three, and the sound enters in and educates the soul in virtue. 'Yes.' And the movement which, when pursued as an amusement, is termed dancing, when studied with a view to the improvement of the body, becomes gymnastic. Shall we now proceed to speak of this? 'What Cretan or Lacedaemonian would approve of your omitting gymnastic?' Your question implies assent; and you will easily understand a subject which is familiar to you. Gymnastic is based on the natural tendency of every animal to rapid motion; and man adds a sense of rhythm, which is awakened by music; music and dancing together form the choral art. But before proceeding I must add a crowning word about drinking. Like other pleasures, it has a lawful use; but if a state or an individual is inclined to drink at will, I cannot allow them. I would go further than Crete or Lacedaemon and have the law of the Carthaginians, that no slave of either sex should drink wine at all, and no soldier while he is on a campaign, and no magistrate or officer while he is on duty, and that no one should drink by daylight or on a bridal night. And there are so many other occasions on which wine ought to be prohibited, that there will not be many vines grown or vineyards required in the state.

\par  BOOK III. If a man wants to know the origin of states and societies, he should behold them from the point of view of time. Thousands of cities have come into being and have passed away again in infinite ages, every one of them having had endless forms of government; and if we can ascertain the cause of these changes in states, that will probably explain their origin. What do you think of ancient traditions about deluges and destructions of mankind, and the preservation of a remnant? 'Every one believes in them.' Then let us suppose the world to have been destroyed by a deluge. The survivors would be hill-shepherds, small sparks of the human race, dwelling in isolation, and unacquainted with the arts and vices of civilization. We may further suppose that the cities on the plain and on the coast have been swept away, and that all inventions, and every sort of knowledge, have perished. 'Why, if all things were as they now are, nothing would have ever been invented. All our famous discoveries have been made within the last thousand years, and many of them are but of yesterday.' Yes, Cleinias, and you must not forget Epimenides, who was really of yesterday; he practised the lesson of moderation and abstinence which Hesiod only preached. 'True.' After the great destruction we may imagine that the earth was a desert, in which there were a herd or two of oxen and a few goats, hardly enough to support those who tended them; while of politics and governments the survivors would know nothing. And out of this state of things have arisen arts and laws, and a great deal of virtue and a great deal of vice; little by little the world has come to be what it is. At first, the few inhabitants would have had a natural fear of descending into the plains; although they would want to have intercourse with one another, they would have a difficulty in getting about, having lost the arts, and having no means of extracting metals from the earth, or of felling timber; for even if they had saved any tools, these would soon have been worn out, and they could get no more until the art of metallurgy had been again revived. Faction and war would be extinguished among them, for being solitary they would incline to be friendly; and having abundance of pasture and plenty of milk and flesh, they would have nothing to quarrel about. We may assume that they had also dwellings, clothes, pottery, for the weaving and plastic arts do not require the use of metals. In those days they were neither poor nor rich, and there was no insolence or injustice among them; for they were of noble natures, and lived up to their principles, and believed what they were told; knowing nothing of land or naval warfare, or of legal practices or party conflicts, they were simpler and more temperate, and also more just than the men of our day. 'Very true.' I am showing whence the need of lawgivers arises, for in primitive ages they neither had nor wanted them. Men lived according to the customs of their fathers, in a simple manner, under a patriarchal government, such as still exists both among Hellenes and barbarians, and is described in Homer as prevailing among the Cyclopes:—

\par  'They have no laws, and they dwell in rocks or on the tops of mountains, and every one is the judge of his wife and children, and they do not trouble themselves about one another.'

\par  'That is a charming poet of yours, though I know little of him, for in Crete foreign poets are not much read.' 'But he is well known in Sparta, though he describes Ionian rather than Dorian manners, and he seems to take your view of primitive society.' May we not suppose that government arose out of the union of single families who survived the destruction, and were under the rule of patriarchs, because they had originally descended from a single father and mother? 'That is very probable.' As time went on, men increased in number, and tilled the ground, living in a common habitation, which they protected by walls against wild beasts; but the several families retained the laws and customs which they separately received from their first parents. They would naturally like their own laws better than any others, and would be already formed by them when they met in a common society: thus legislation imperceptibly began among them. For in the next stage the associated families would appoint plenipotentiaries, who would select and present to the chiefs those of all their laws which they thought best. The chiefs in turn would make a further selection, and would thus become the lawgivers of the state, which they would form into an aristocracy or a monarchy. 'Probably.' In the third stage various other forms of government would arise. This state of society is described by Homer in speaking of the foundation of Dardania, which, he says,
 
\par  Here, as also in the account of the Cyclopes, the poet by some divine inspiration has attained truth. But to proceed with our tale. Ilium was built in a wide plain, on a low hill, which was surrounded by streams descending from Ida. This shows that many ages must have passed; for the men who remembered the deluge would never have placed their city at the mercy of the waters. When mankind began to multiply, many other cities were built in similar situations. These cities carried on a ten years' war against Troy, by sea as well as land, for men were ceasing to be afraid of the sea, and, in the meantime, while the chiefs of the army were at Troy, their homes fell into confusion. The youth revolted and refused to receive their own fathers; deaths, murders, exiles ensued. Under the new name of Dorians, which they received from their chief Dorieus, the exiles returned: the rest of the story is part of the history of Sparta.

\par  Thus, after digressing from the subject of laws into music and drinking, we return to the settlement of Sparta, which in laws and institutions is the sister of Crete. We have seen the rise of a first, second, and third state, during the lapse of ages; and now we arrive at a fourth state, and out of the comparison of all four we propose to gather the nature of laws and governments, and the changes which may be desirable in them. 'If,' replies the Spartan, 'our new discussion is likely to be as good as the last, I would think the longest day too short for such an employment.'

\par  Let us imagine the time when Lacedaemon, and Argos, and Messene were all subject, Megillus, to your ancestors. Afterwards, they distributed the army into three portions, and made three cities—Argos, Messene, Lacedaemon. 'Yes.' Temenus was the king of Argos, Cresphontes of Messene, Procles and Eurysthenes ruled at Lacedaemon. 'Just so.' And they all swore to assist any one of their number whose kingdom was subverted. 'Yes.' But did we not say that kingdoms or governments can only be subverted by themselves? 'That is true.' Yes, and the truth is now proved by facts: there were certain conditions upon which the three kingdoms were to assist one another; the government was to be mild and the people obedient, and the kings and people were to unite in assisting either of the two others when they were wronged. This latter condition was a great security. 'Clearly.' Such a provision is in opposition to the common notion that the lawgiver should make only such laws as the people like; but we say that he should rather be like a physician, prepared to effect a cure even at the cost of considerable suffering. 'Very true.' The early lawgivers had another great advantage—they were saved from the reproach which attends a division of land and the abolition of debts. No one could quarrel with the Dorians for dividing the territory, and they had no debts of long standing. 'They had not.' Then what was the reason why their legislation signally failed? For there were three kingdoms, two of them quickly lost their original constitution. That is a question which we cannot refuse to answer, if we mean to proceed with our old man's game of enquiring into laws and institutions. And the Dorian institutions are more worthy of consideration than any other, having been evidently intended to be a protection not only to the Peloponnese, but to all the Hellenes against the Barbarians. For the capture of Troy by the Achaeans had given great offence to the Assyrians, of whose empire it then formed part, and they were likely to retaliate. Accordingly the royal Heraclid brothers devised their military constitution, which was organised on a far better plan than the old Trojan expedition; and the Dorians themselves were far superior to the Achaeans, who had taken part in that expedition, and had been conquered by them. Such a scheme, undertaken by men who had shared with one another toils and dangers, sanctioned by the Delphian oracle, under the guidance of the Heraclidae, seemed to have a promise of permanence. 'Naturally.' Yet this has not proved to be the case. Instead of the three being one, they have always been at war; had they been united, in accordance with the original intention, they would have been invincible.

\par  And what caused their ruin? Did you ever observe that there are beautiful things of which men often say, 'What wonders they would have effected if rightly used?' and yet, after all, this may be a mistake. And so I say of the Heraclidae and their expedition, which I may perhaps have been justified in admiring, but which nevertheless suggests to me the general reflection,—'What wonders might not strength and military resources have accomplished, if the possessor had only known how to use them!' For consider: if the generals of the army had only known how to arrange their forces, might they not have given their subjects everlasting freedom, and the power of doing what they would in all the world? 'Very true.' Suppose a person to express his admiration of wealth or rank, does he not do so under the idea that by the help of these he can attain his desires? All men wish to obtain the control of all things, and they are always praying for what they desire. 'Certainly.' And we ask for our friends what they ask for themselves. 'Yes.' Dear is the son to the father, and yet the son, if he is young and foolish, will often pray to obtain what the father will pray that he may not obtain. 'True.' And when the father, in the heat of youth or the dotage of age, makes some rash prayer, the son, like Hippolytus, may have reason to pray that the word of his father may be ineffectual. 'You mean that a man should pray to have right desires, before he prays that his desires may be fulfilled; and that wisdom should be the first object of our prayers?' Yes; and you will remember my saying that wisdom should be the principal aim of the legislator; but you said that defence in war came first. And I replied, that there were four virtues, whereas you acknowledged one only—courage, and not wisdom which is the guide of all the rest. And I repeat—in jest if you like, but I am willing that you should receive my words in earnest—that 'the prayer of a fool is full of danger.' I will prove to you, if you will allow me, that the ruin of those states was not caused by cowardice or ignorance in war, but by ignorance of human affairs. 'Pray proceed: our attention will show better than compliments that we prize your words.' I maintain that ignorance is, and always has been, the ruin of states; wherefore the legislator should seek to banish it from the state; and the greatest ignorance is the love of what is known to be evil, and the hatred of what is known to be good; this is the last and greatest conflict of pleasure and reason in the soul. I say the greatest, because affecting the greater part of the soul; for the passions are in the individual what the people are in a state. And when they become opposed to reason or law, and instruction no longer avails—that is the last and greatest ignorance of states and men. 'I agree.' Let this, then, be our first principle:—That the citizen who does not know how to choose between good and evil must not have authority, although he possess great mental gifts, and many accomplishments; for he is really a fool. On the other hand, he who has this knowledge may be unable either to read or swim; nevertheless, he shall be counted wise and permitted to rule. For how can there be wisdom where there is no harmony?—the wise man is the saviour, and he who is devoid of wisdom is the destroyer of states and households. There are rulers and there are subjects in states. And the first claim to rule is that of parents to rule over their children; the second, that of the noble to rule over the ignoble; thirdly, the elder must govern the younger; in the fourth place, the slave must obey his master; fifthly, there is the power of the stronger, which the poet Pindar declares to be according to nature; sixthly, there is the rule of the wiser, which is also according to nature, as I must inform Pindar, if he does not know, and is the rule of law over obedient subjects. 'Most true.' And there is a seventh kind of rule which the Gods love,—in this the ruler is elected by lot.

\par  Then, now, we playfully say to him who fancies that it is easy to make laws:—You see, legislator, the many and inconsistent claims to authority; here is a spring of troubles which you must stay. And first of all you must help us to consider how the kings of Argos and Messene in olden days destroyed their famous empire—did they forget the saying of Hesiod, that 'the half is better than the whole'? And do we suppose that the ignorance of this truth is less fatal to kings than to peoples? 'Probably the evil is increased by their way of life.' The kings of those days transgressed the laws and violated their oaths. Their deeds were not in harmony with their words, and their folly, which seemed to them wisdom, was the ruin of the state. And how could the legislator have prevented this evil?—the remedy is easy to see now, but was not easy to foresee at the time. 'What is the remedy?' The institutions of Sparta may teach you, Megillus. Wherever there is excess, whether the vessel has too large a sail, or the body too much food, or the mind too much power, there destruction is certain. And similarly, a man who possesses arbitrary power is soon corrupted, and grows hateful to his dearest friends. In order to guard against this evil, the God who watched over Sparta gave you two kings instead of one, that they might balance one another; and further to lower the pulse of your body politic, some human wisdom, mingled with divine power, tempered the strength and self-sufficiency of youth with the moderation of age in the institution of your senate. A third saviour bridled your rising and swelling power by ephors, whom he assimilated to officers elected by lot: and thus the kingly power was preserved, and became the preserver of all the rest. Had the constitution been arranged by the original legislators, not even the portion of Aristodemus would have been saved; for they had no political experience, and imagined that a youthful spirit invested with power could be restrained by oaths. Now that God has instructed us in the arts of legislation, there is no merit in seeing all this, or in learning wisdom after the event. But if the coming danger could have been foreseen, and the union preserved, then no Persian or other enemy would have dared to attack Hellas; and indeed there was not so much credit to us in defeating the enemy, as discredit in our disloyalty to one another. For of the three cities one only fought on behalf of Hellas; and of the two others, Argos refused her aid; and Messenia was actually at war with Sparta: and if the Lacedaemonians and Athenians had not united, the Hellenes would have been absorbed in the Persian empire, and dispersed among the barbarians. We make these reflections upon past and present legislators because we desire to find out what other course could have been followed. We were saying just now, that a state can only be free and wise and harmonious when there is a balance of powers. There are many words by which we express the aims of the legislator,—temperance, wisdom, friendship; but we need not be disturbed by the variety of expression,—these words have all the same meaning. 'I should like to know at what in your opinion the legislator should aim.' Hear me, then. There are two mother forms of states—one monarchy, and the other democracy: the Persians have the first in the highest form, and the Athenians the second; and no government can be well administered which does not include both. There was a time when both the Persians and Athenians had more the character of a constitutional state than they now have. In the days of Cyrus the Persians were freemen as well as lords of others, and their soldiers were free and equal, and the kings used and honoured all the talent which they could find, and so the nation waxed great, because there was freedom and friendship and communion of soul. But Cyrus, though a wise general, never troubled himself about the education of his family. He was a soldier from his youth upward, and left his children who were born in the purple to be educated by women, who humoured and spoilt them. 'A rare education, truly!' Yes, such an education as princesses who had recently grown rich might be expected to give them in a country where the men were solely occupied with warlike pursuits. 'Likely enough.' Their father had possessions of men and animals, and never considered that the race to whom he was about to make them over had been educated in a very different school, not like the Persian shepherd, who was well able to take care of himself and his own. He did not see that his children had been brought up in the Median fashion, by women and eunuchs. The end was that one of the sons of Cyrus slew the other, and lost the kingdom by his own folly. Observe, again, that Darius, who restored the kingdom, had not received a royal education. He was one of the seven chiefs, and when he came to the throne he divided the empire into seven provinces; and he made equal laws, and implanted friendship among the people. Hence his subjects were greatly attached to him, and cheerfully helped him to extend his empire. Next followed Xerxes, who had received the same royal education as Cambyses, and met with a similar fate. The reflection naturally occurs to us—How could Darius, with all his experience, have made such a mistake! The ruin of Xerxes was not a mere accident, but the evil life which is generally led by the sons of very rich and royal persons; and this is what the legislator has seriously to consider. Justly may the Lacedaemonians be praised for not giving special honour to birth or wealth; for such advantages are not to be highly esteemed without virtue, and not even virtue is to be esteemed unless it be accompanied by temperance. 'Explain.' No one would like to live in the same house with a courageous man who had no control over himself, nor with a clever artist who was a rogue. Nor can justice and wisdom ever be separated from temperance. But considering these qualities with reference to the honour and dishonour which is to be assigned to them in states, would you say, on the other hand, that temperance, if existing without the other virtues in the soul, is worth anything or nothing? 'I cannot tell.' You have answered well. It would be absurd to speak of temperance as belonging to the class of honourable or of dishonourable qualities, because all other virtues in their various classes require temperance to be added to them; having the addition, they are honoured not in proportion to that, but to their own excellence. And ought not the legislator to determine these classes? 'Certainly.' Suppose then that, without going into details, we make three great classes of them. Most honourable are the goods of the soul, always assuming temperance as a condition of them; secondly, those of the body; thirdly, external possessions. The legislator who puts them in another order is doing an unholy and unpatriotic thing.

\par  These remarks were suggested by the history of the Persian kings; and to them I will now return. The ruin of their empire was caused by the loss of freedom and the growth of despotism; all community of feeling disappeared. Hatred and spoliation took the place of friendship; the people no longer fought heartily for their masters; the rulers, finding their myriads useless on the field of battle, resorted to mercenaries as their only salvation, and were thus compelled by their circumstances to proclaim the stupidest of falsehoods—that virtue is a trifle in comparison of money.

\par  But enough of the Persians: a different lesson is taught by the Athenians, whose example shows that a limited freedom is far better than an unlimited. Ancient Athens, at the time of the Persian invasion, had such a limited freedom. The people were divided into four classes, according to the amount of their property, and the universal love of order, as well as the fear of the approaching host, made them obedient and willing citizens. For Darius had sent Datis and Artaphernes, commanding them under pain of death to subjugate the Eretrians and Athenians. A report, whether true or not, came to Athens that all the Eretrians had been 'netted'; and the Athenians in terror sent all over Hellas for assistance. None came to their relief except the Lacedaemonians, and they arrived a day too late, when the battle of Marathon had been already fought. In process of time Xerxes came to the throne, and the Athenians heard of nothing but the bridge over the Hellespont, and the canal of Athos, and the innumerable host and fleet. They knew that these were intended to avenge the defeat of Marathon. Their case seemed desperate, for there was no Hellene likely to assist them by land, and at sea they were attacked by more than a thousand vessels;—their only hope, however slender, was in victory; so they relied upon themselves and upon the Gods. Their common danger, and the influence of their ancient constitution, greatly tended to promote harmony among them. Reverence and fear—that fear which the coward never knows—made them fight for their altars and their homes, and saved them from being dispersed all over the world. 'Your words, Athenian, are worthy of your country.' And you Megillus, who have inherited the virtues of your ancestors, are worthy to hear them. Let me ask you to take the moral of my tale. The Persians have lost their liberty in absolute slavery, and we in absolute freedom. In ancient times the Athenian people were not the masters, but the servants of the laws. 'Of what laws?' In the first place, there were laws about music, and the music was of various kinds: there was one kind which consisted of hymns, another of lamentations; there was also the paean and the dithyramb, and the so-called 'laws' (nomoi) or strains, which were played upon the harp. The regulation of such matters was not left to the whistling and clapping of the crowd; there was silence while the judges decided, and the boys, and the audience in general, were kept in order by raps of a stick. But after a while there arose a new race of poets, men of genius certainly, however careless of musical truth and propriety, who made pleasure the only criterion of excellence. That was a test which the spectators could apply for themselves; the whole audience, instead of being mute, became vociferous, and a theatrocracy took the place of an aristocracy. Could the judges have been free, there would have been no great harm done; a musical democracy would have been well enough—but conceit has been our ruin. Everybody knows everything, and is ready to say anything; the age of reverence is gone, and the age of irreverence and licentiousness has succeeded. 'Most true.' And with this freedom comes disobedience to rulers, parents, elders,—in the latter days to the law also; the end returns to the beginning, and the old Titanic nature reappears—men have no regard for the Gods or for oaths; and the evils of the human race seem as if they would never cease. Whither are we running away? Once more we must pull up the argument with bit and curb, lest, as the proverb says, we should fall off our ass. 'Good.' Our purpose in what we have been saying is to prove that the legislator ought to aim at securing for a state three things—freedom, friendship, wisdom. And we chose two states;—one was the type of freedom, and the other of despotism; and we showed that when in a mean they attained their highest perfection. In a similar spirit we spoke of the Dorian expedition, and of the settlement on the hills and in the plains of Troy; and of music, and the use of wine, and of all that preceded.

\par  And now, has our discussion been of any use? 'Yes, stranger; for by a singular coincidence the Cretans are about to send out a colony, of which the settlement has been confided to the Cnosians. Ten commissioners, of whom I am one, are to give laws to the colonists, and we may give any which we please—Cretan or foreign. And therefore let us make a selection from what has been said, and then proceed with the construction of the state.' Very good: I am quite at your service. 'And I too,' says Megillus.

\par  BOOK IV. And now, what is this city? I do not want to know what is to be the name of the place (for some accident,—a river or a local deity, will determine that), but what the situation is, whether maritime or inland. 'The city will be about eleven miles from the sea.' Are there harbours? 'Excellent.' And is the surrounding country self-supporting? 'Almost.' Any neighbouring states? 'No; and that is the reason for choosing the place, which has been deserted from time immemorial.' And is there a fair proportion of hill and plain and wood? 'Like Crete in general, more hill than plain.' Then there is some hope for your citizens; had the city been on the sea, and dependent for support on other countries, no human power could have preserved you from corruption. Even the distance of eleven miles is hardly enough. For the sea, although an agreeable, is a dangerous companion, and a highway of strange morals and manners as well as of commerce. But as the country is only moderately fertile there will be no great export trade and no great returns of gold and silver, which are the ruin of states. Is there timber for ship-building? 'There is no pine, nor much cypress; and very little stone-pine or plane wood for the interior of ships.' That is good. 'Why?' Because the city will not be able to imitate the bad ways of her enemies. 'What is the bearing of that remark?' To explain my meaning, I would ask you to remember what we said about the Cretan laws, that they had an eye to war only; whereas I maintained that they ought to have included all virtue. And I hope that you in your turn will retaliate upon me if I am false to my own principle. For I consider that the lawgiver should go straight to the mark of virtue and justice, and disregard wealth and every other good when separated from virtue. What further I mean, when I speak of the imitation of enemies, I will illustrate by the story of Minos, if our Cretan friend will allow me to mention it. Minos, who was a great sea-king, imposed upon the Athenians a cruel tribute, for in those days they were not a maritime power; they had no timber for ship-building, and therefore they could not 'imitate their enemies'; and better far, as I maintain, would it have been for them to have lost many times over the lives which they devoted to the tribute than to have turned soldiers into sailors. Naval warfare is not a very praiseworthy art; men should not be taught to leap on shore, and then again to hurry back to their ships, or to find specious excuses for throwing away their arms; bad customs ought not to be gilded with fine words. And retreat is always bad, as we are taught in Homer, when he introduces Odysseus, setting forth to Agamemnon the danger of ships being at hand when soldiers are disposed to fly. An army of lions trained in such ways would fly before a herd of deer. Further, a city which owes its preservation to a crowd of pilots and oarsmen and other undeserving persons, cannot bestow rewards of honour properly; and this is the ruin of states. 'Still, in Crete we say that the battle of Salamis was the salvation of Hellas.' Such is the prevailing opinion. But I and Megillus say that the battle of Marathon began the deliverance, and that the battle of Plataea completed it; for these battles made men better, whereas the battles of Salamis and Artemisium made them no better. And we further affirm that mere existence is not the great political good of individuals or states, but the continuance of the best existence. 'Certainly.' Let us then endeavour to follow this principle in colonization and legislation.

\par  And first, let me ask you who are to be the colonists? May any one come from any city of Crete? For you would surely not send a general invitation to all Hellas. Yet I observe that in Crete there are people who have come from Argos and Aegina and other places. 'Our recruits will be drawn from all Crete, and of other Hellenes we should prefer Peloponnesians. As you observe, there are Argives among the Cretans; moreover the Gortynians, who are the best of all Cretans, have come from Gortys in Peloponnesus.'

\par  Colonization is in some ways easier when the colony goes out in a swarm from one country, owing to the pressure of population, or revolution, or war. In this case there is the advantage that the new colonists have a community of race, language, and laws. But then again, they are less obedient to the legislator; and often they are anxious to keep the very laws and customs which caused their ruin at home. A mixed multitude, on the other hand, is more tractable, although there is a difficulty in making them pull together. There is nothing, however, which perfects men's virtue more than legislation and colonization. And yet I have a word to say which may seem to be depreciatory of legislators. 'What is that?'

\par  I was going to make the saddening reflection, that accidents of all sorts are the true legislators,—wars and pestilences and famines and the frequent recurrence of bad seasons. The observer will be inclined to say that almost all human things are chance; and this is certainly true about navigation and medicine, and the art of the general. But there is another thing which may equally be said. 'What is it?' That God governs all things, and that chance and opportunity co-operate with Him. And according to yet a third view, art has part with them, for surely in a storm it is well to have a pilot? And the same is true of legislation: even if circumstances are favourable, a skilful lawgiver is still necessary. 'Most true.' All artists would pray for certain conditions under which to exercise their art: and would not the legislator do the same? 'Certainly?' Come, legislator, let us say to him, and what are the conditions which you would have? He will answer, Grant me a city which is ruled by a tyrant; and let the tyrant be young, mindful, teachable, courageous, magnanimous; and let him have the inseparable condition of all virtue, which is temperance—not prudence, but that natural temperance which is the gift of children and animals, and is hardly reckoned among goods—with this he must be endowed, if the state is to acquire the form most conducive to happiness in the speediest manner. And I must add one other condition: the tyrant must be fortunate, and his good fortune must consist in his having the co-operation of a great legislator. When God has done all this, He has done the best which He can for a state; not so well if He has given them two legislators instead of one, and less and less well if He has given them a great many. An orderly tyranny most easily passes into the perfect state; in the second degree, a monarchy; in the third degree, a democracy; an oligarchy is worst of all. 'I do not understand.' I suppose that you have never seen a city which is subject to a tyranny? 'I have no desire to see one.' You would have seen what I am describing, if you ever had. The tyrant can speedily change the manners of a state, and affix the stamp of praise or blame on any action which he pleases; for the citizens readily follow the example which he sets. There is no quicker way of making changes; but there is a counterbalancing difficulty. It is hard to find the divine love of temperance and justice existing in any powerful form of government, whether in a monarchy or an oligarchy. In olden days there were chiefs like Nestor, who was the most eloquent and temperate of mankind, but there is no one his equal now. If such an one ever arises among us, blessed will he be, and blessed they who listen to his words. For where power and wisdom and temperance meet in one, there are the best laws and constitutions. I am endeavouring to show you how easy under the conditions supposed, and how difficult under any other, is the task of giving a city good laws. 'How do you mean?' Let us old men attempt to mould in words a constitution for your new state, as children make figures out of wax. 'Proceed. What constitution shall we give—democracy, oligarchy, or aristocracy?' To which of these classes, Megillus, do you refer your own state? 'The Spartan constitution seems to me to contain all these elements. Our state is a democracy and also an aristocracy; the power of the Ephors is tyrannical, and we have an ancient monarchy.' 'Much the same,' adds Cleinias, 'may be said of Cnosus.' The reason is that you have polities, but other states are mere aggregations of men dwelling together, which are named after their several ruling powers; whereas a state, if an 'ocracy' at all, should be called a theocracy. A tale of old will explain my meaning. There is a tradition of a golden age, in which all things were spontaneous and abundant. Cronos, then lord of the world, knew that no mortal nature could endure the temptations of power, and therefore he appointed demons or demi-gods, who are of a superior race, to have dominion over man, as man has dominion over the animals. They took care of us with great ease and pleasure to themselves, and no less to us; and the tradition says that only when God, and not man, is the ruler, can the human race cease from ill. This was the manner of life which prevailed under Cronos, and which we must strive to follow so far as the principle of immortality still abides in us and we live according to law and the dictates of right reason. But in an oligarchy or democracy, when the governing principle is athirst for pleasure, the laws are trampled under foot, and there is no possibility of salvation. Is it not often said that there are as many forms of laws as there are governments, and that they have no concern either with any one virtue or with all virtue, but are relative to the will of the government? Which is as much as to say that 'might makes right.' 'What do you mean?' I mean that governments enact their own laws, and that every government makes self-preservation its principal aim. He who transgresses the laws is regarded as an evil-doer, and punished accordingly. This was one of the unjust principles of government which we mentioned when speaking of the different claims to rule. We were agreed that parents should rule their children, the elder the younger, the noble the ignoble. But there were also several other principles, and among them Pindar's 'law of violence.' To whom then is our state to be entrusted? For many a government is only a victorious faction which has a monopoly of power, and refuses any share to the conquered, lest when they get into office they should remember their wrongs. Such governments are not polities, but parties; nor are any laws good which are made in the interest of particular classes only, and not of the whole. And in our state I mean to protest against making any man a ruler because he is rich, or strong, or noble. But those who are obedient to the laws, and who win the victory of obedience, shall be promoted to the service of the Gods according to the degree of their obedience. When I call the ruler the servant or minister of the law, this is not a mere paradox, but I mean to say that upon a willingness to obey the law the existence of the state depends. 'Truly, Stranger, you have a keen vision.' Why, yes; every man when he is old has his intellectual vision most keen. And now shall we call in our colonists and make a speech to them? Friends, we say to them, God holds in His hand the beginning, middle, and end of all things, and He moves in a straight line towards the accomplishment of His will. Justice always bears Him company, and punishes those who fall short of His laws. He who would be happy follows humbly in her train; but he who is lifted up with pride, or wealth, or honour, or beauty, is soon deserted by God, and, being deserted, he lives in confusion and disorder. To many he seems a great man; but in a short time he comes to utter destruction. Wherefore, seeing these things, what ought we to do or think? 'Every man ought to follow God.' What life, then, is pleasing to God? There is an old saying that 'like agrees with like, measure with measure,' and God ought to be our measure in all things. The temperate man is the friend of God because he is like Him, and the intemperate man is not His friend, because he is not like Him. And the conclusion is, that the best of all things for a good man is to pray and sacrifice to the Gods; but the bad man has a polluted soul; and therefore his service is wasted upon the Gods, while the good are accepted of them. I have told you the mark at which we ought to aim. You will say, How, and with what weapons? In the first place we affirm, that after the Olympian Gods and the Gods of the state, honour should be given to the Gods below, and to them should be offered everything in even numbers and of the second choice; the auspicious odd numbers and everything of the first choice are reserved for the Gods above. Next demi-gods or spirits must be honoured, and then heroes, and after them family gods, who will be worshipped at their local seats according to law. Further, the honour due to parents should not be forgotten; children owe all that they have to them, and the debt must be repaid by kindness and attention in old age. No unbecoming word must be uttered before them; for there is an avenging angel who hears them when they are angry, and the child should consider that the parent when he has been wronged has a right to be angry. After their death let them have a moderate funeral, such as their fathers have had before them; and there shall be an annual commemoration of them. Living on this wise, we shall be accepted of the Gods, and shall pass our days in good hope. The law will determine all our various duties towards relatives and friends and other citizens, and the whole state will be happy and prosperous. But if the legislator would persuade as well as command, he will add prefaces to his laws which will predispose the citizens to virtue. Even a little accomplished in the way of gaining the hearts of men is of great value. For most men are in no particular haste to become good. As Hesiod says:

\par  'Long and steep is the first half of the way to virtue, But when you have reached the top the rest is easy.'

\par  'Those are excellent words.' Yes; but may I tell you the effect which the preceding discourse has had upon me? I will express my meaning in an address to the lawgiver:—O lawgiver, if you know what we ought to do and say, you can surely tell us;—you are not like the poet, who, as you were just now saying, does not know the effect of his own words. And the poet may reply, that when he sits down on the tripod of the Muses he is not in his right mind, and that being a mere imitator he may be allowed to say all sorts of opposite things, and cannot tell which of them is true. But this licence cannot be allowed to the lawgiver. For example, there are three kinds of funerals; one of them is excessive, another mean, a third moderate, and you say that the last is right. Now if I had a rich wife, and she told me to bury her, and I were to sing of her burial, I should praise the extravagant kind; a poor man would commend a funeral of the meaner sort, and a man of moderate means would prefer a moderate funeral. But you, as legislator, would have to say exactly what you meant by 'moderate.' 'Very true.' And is our lawgiver to have no preamble or interpretation of his laws, never offering a word of advice to his subjects, after the manner of some doctors? For of doctors are there not two kinds? The one gentle and the other rough, doctors who are freemen and learn themselves and teach their pupils scientifically, and doctor's assistants who get their knowledge empirically by attending on their masters? 'Of course there are.' And did you ever observe that the gentlemen doctors practise upon freemen, and that slave doctors confine themselves to slaves? The latter go about the country or wait for the slaves at the dispensaries. They hold no parley with their patients about their diseases or the remedies of them; they practise by the rule of thumb, and give their decrees in the most arbitrary manner. When they have doctored one patient they run off to another, whom they treat with equal assurance, their duty being to relieve the master of the care of his sick slaves. But the other doctor, who practises on freemen, proceeds in quite a different way. He takes counsel with his patient and learns from him, and never does anything until he has persuaded him of what he is doing. He trusts to influence rather than force. Now is not the use of both methods far better than the use of either alone? And both together may be advantageously employed by us in legislation.

\par  We may illustrate our proposal by an example. The laws relating to marriage naturally come first, and therefore we may begin with them. The simple law would be as follows:—A man shall marry between the ages of thirty and thirty-five; if he do not, he shall be fined or deprived of certain privileges. The double law would add the reason why: Forasmuch as man desires immortality, which he attains by the procreation of children, no one should deprive himself of his share in this good. He who obeys the law is blameless, but he who disobeys must not be a gainer by his celibacy; and therefore he shall pay a yearly fine, and shall not be allowed to receive honour from the young. That is an example of what I call the double law, which may enable us to judge how far the addition of persuasion to threats is desirable. 'Lacedaemonians in general, Stranger, are in favour of brevity; in this case, however, I prefer length. But Cleinias is the real lawgiver, and he ought to be first consulted.' 'Thank you, Megillus.' Whether words are to be many or few, is a foolish question:—the best and not the shortest forms are always to be approved. And legislators have never thought of the advantages which they might gain by using persuasion as well as force, but trust to force only. And I have something else to say about the matter. Here have we been from early dawn until noon, discoursing about laws, and all that we have been saying is only the preamble of the laws which we are about to give. I tell you this, because I want you to observe that songs and strains have all of them preludes, but that laws, though called by the same name (nomoi), have never any prelude. Now I am disposed to give preludes to laws, dividing them into two parts—one containing the despotic command, which I described under the image of the slave doctor—the other the persuasive part, which I term the preamble. The legislator should give preludes or preambles to his laws. 'That shall be the way in my colony.' I am glad that you agree with me; this is a matter which it is important to remember. A preamble is not always necessary to a law: the lawgiver must determine when it is needed, as the musician determines when there is to be a prelude to a song. 'Most true: and now, having a preamble, let us recommence our discourse.' Enough has been said of Gods and parents, and we may proceed to consider what relates to the citizens—their souls, bodies, properties,—their occupations and amusements; and so arrive at the nature of education.

\par  The first word of the Laws somewhat abruptly introduces the thought which is present to the mind of Plato throughout the work, namely, that Law is of divine origin. In the words of a great English writer—'Her seat is the bosom of God, her voice the harmony of the world.' Though the particular laws of Sparta and Crete had a narrow and imperfect aim, this is not true of divine laws, which are based upon the principles of human nature, and not framed to meet the exigencies of the moment. They have their natural divisions, too, answering to the kinds of virtue; very unlike the discordant enactments of an Athenian assembly or of an English Parliament. Yet we may observe two inconsistencies in Plato's treatment of the subject: first, a lesser, inasmuch as he does not clearly distinguish the Cretan and Spartan laws, of which the exclusive aim is war, from those other laws of Zeus and Apollo which are said to be divine, and to comprehend all virtue. Secondly, we may retort on him his own complaint against Sparta and Crete, that he has himself given us a code of laws, which for the most part have a military character; and that we cannot point to 'obvious examples of similar institutions which are concerned with pleasure;' at least there is only one such, that which relates to the regulation of convivial intercourse. The military spirit which is condemned by him in the beginning of the Laws, reappears in the seventh and eighth books.

\par  The mention of Minos the great lawgiver, and of Rhadamanthus the righteous administrator of the law, suggests the two divisions of the laws into enactments and appointments of officers. The legislator and the judge stand side by side, and their functions cannot be wholly distinguished. For the judge is in some sort a legislator, at any rate in small matters; and his decisions growing into precedents, must determine the innumerable details which arise out of the conflict of circumstances. These Plato proposes to leave to a younger generation of legislators. The action of courts of law in making law seems to have escaped him, probably because the Athenian law-courts were popular assemblies; and, except in a mythical form, he can hardly be said to have had before his eyes the ideal of a judge. In reading the Laws of Plato, or any other ancient writing about Laws, we should consider how gradual the process is by which not only a legal system, but the administration of a court of law, becomes perfected.

\par  There are other subjects on which Plato breaks ground, as his manner is, early in the work. First, he gives a sketch of the subject of laws; they are to comprehend the whole of human life, from infancy to age, and from birth to death, although the proposed plan is far from being regularly executed in the books which follow, partly owing to the necessity of describing the constitution as well as the laws of his new colony. Secondly, he touches on the power of music, which may exercise so great an influence on the character of men for good or evil; he refers especially to the great offence—which he mentions again, and which he had condemned in the Republic—of varying the modes and rhythms, as well as to that of separating the words from the music. Thirdly, he reprobates the prevalence of unnatural loves in Sparta and Crete, which he attributes to the practice of syssitia and gymnastic exercises, and considers to be almost inseparable from them. To this subject he again returns in the eighth book. Fourthly, the virtues are affirmed to be inseparable from one another, even if not absolutely one; this, too, is a principle which he reasserts at the conclusion of the work. As in the beginnings of Plato's other writings, we have here several 'notes' struck, which form the preludes of longer discussions, although the hint is less ingeniously given, and the promise more imperfectly fulfilled than in the earlier dialogues.

\par  The distinction between ethics and politics has not yet dawned upon Plato's mind. To him, law is still floating in a region between the two. He would have desired that all the acts and laws of a state should have regard to all virtue. But he did not see that politics and law are subject to their own conditions, and are distinguished from ethics by natural differences. The actions of which politics take cognisance are necessarily collective or representative; and law is limited to external acts which affect others as well as the agents. Ethics, on the other hand, include the whole duty of man in relation both to himself and others. But Plato has never reflected on these differences. He fancies that the life of the state can be as easily fashioned as that of the individual. He is favourable to a balance of power, but never seems to have considered that power might be so balanced as to produce an absolute immobility in the state. Nor is he alive to the evils of confounding vice and crime; or to the necessity of governments abstaining from excessive interference with their subjects.

\par  Yet this confusion of ethics and politics has also a better and a truer side. If unable to grasp some important distinctions, Plato is at any rate seeking to elevate the lower to the higher; he does not pull down the principles of men to their practice, or narrow the conception of the state to the immediate necessities of politics. Political ideals of freedom and equality, of a divine government which has been or will be in some other age or country, have greatly tended to educate and ennoble the human race. And if not the first author of such ideals (for they are as old as Hesiod), Plato has done more than any other writer to impress them on the world. To those who censure his idealism we may reply in his own words—'He is not the worse painter who draws a perfectly beautiful figure, because no such figure of a man could ever have existed' (Republic).

\par  A new thought about education suddenly occurs to him, and for a time exercises a sort of fascination over his mind, though in the later books of the Laws it is forgotten or overlooked. As true courage is allied to temperance, so there must be an education which shall train mankind to resist pleasure as well as to endure pain. No one can be on his guard against that of which he has no experience. The perfectly trained citizen should have been accustomed to look his enemy in the face, and to measure his strength against her. This education in pleasure is to be given, partly by festive intercourse, but chiefly by the song and dance. Youth are to learn music and gymnastics; their elders are to be trained and tested at drinking parties. According to the old proverb, in vino veritas, they will then be open and visible to the world in their true characters; and also they will be more amenable to the laws, and more easily moulded by the hand of the legislator. The first reason is curious enough, though not important; the second can hardly be thought deserving of much attention. Yet if Plato means to say that society is one of the principal instruments of education in after-life, he has expressed in an obscure fashion a principle which is true, and to his contemporaries was also new. That at a banquet a degree of moral discipline might be exercised is an original thought, but Plato has not yet learnt to express his meaning in an abstract form. He is sensible that moderation is better than total abstinence, and that asceticism is but a one-sided training. He makes the sagacious remark, that 'those who are able to resist pleasure may often be among the worst of mankind.' He is as much aware as any modern utilitarian that the love of pleasure is the great motive of human action. This cannot be eradicated, and must therefore be regulated,—the pleasure must be of the right sort. Such reflections seem to be the real, though imperfectly expressed, groundwork of the discussion. As in the juxtaposition of the Bacchic madness and the great gift of Dionysus, or where he speaks of the different senses in which pleasure is and is not the object of imitative art, or in the illustration of the failure of the Dorian institutions from the prayer of Theseus, we have to gather his meaning as well as we can from the connexion.

\par  The feeling of old age is discernible in this as well as in several other passages of the Laws. Plato has arrived at the time when men sit still and look on at life; and he is willing to allow himself and others the few pleasures which remain to them. Wine is to cheer them now that their limbs are old and their blood runs cold. They are the best critics of dancing and music, but cannot be induced to join in song unless they have been enlivened by drinking. Youth has no need of the stimulus of wine, but age can only be made young again by its invigorating influence. Total abstinence for the young, moderate and increasing potations for the old, is Plato's principle. The fire, of which there is too much in the one, has to be brought to the other. Drunkenness, like madness, had a sacredness and mystery to the Greek; if, on the one hand, as in the case of the Tarentines, it degraded a whole population, it was also a mode of worshipping the god Dionysus, which was to be practised on certain occasions. Moreover, the intoxication produced by the fruit of the vine was very different from the grosser forms of drunkenness which prevail among some modern nations.

\par  The physician in modern times would restrict the old man's use of wine within narrow limits. He would tell us that you cannot restore strength by a stimulus. Wine may call back the vital powers in disease, but cannot reinvigorate old age. In his maxims of health and longevity, though aware of the importance of a simple diet, Plato has omitted to dwell on the perfect rule of moderation. His commendation of wine is probably a passing fancy, and may have arisen out of his own habits or tastes. If so, he is not the only philosopher whose theory has been based upon his practice.

\par  Plato's denial of wine to the young and his approval of it for their elders has some points of view which may be illustrated by the temperance controversy of our own times. Wine may be allowed to have a religious as well as a festive use; it is commended both in the Old and New Testament; it has been sung of by nearly all poets; and it may be truly said to have a healing influence both on body and mind. Yet it is also very liable to excess and abuse, and for this reason is prohibited by Mahometans, as well as of late years by many Christians, no less than by the ancient Spartans; and to sound its praises seriously seems to partake of the nature of a paradox. But we may rejoin with Plato that the abuse of a good thing does not take away the use of it. Total abstinence, as we often say, is not the best rule, but moderate indulgence; and it is probably true that a temperate use of wine may contribute some elements of character to social life which we can ill afford to lose. It draws men out of their reserve; it helps them to forget themselves and to appear as they by nature are when not on their guard, and therefore to make them more human and greater friends to their fellow-men. It gives them a new experience; it teaches them to combine self-control with a measure of indulgence; it may sometimes restore to them the simplicity of childhood. We entirely agree with Plato in forbidding the use of wine to the young; but when we are of mature age there are occasions on which we derive refreshment and strength from moderate potations. It is well to make abstinence the rule, but the rule may sometimes admit of an exception. We are in a higher, as well as in a lower sense, the better for the use of wine. The question runs up into wider ones—What is the general effect of asceticism on human nature? and, Must there not be a certain proportion between the aspirations of man and his powers?—questions which have been often discussed both by ancient and modern philosophers. So by comparing things old and new we may sometimes help to realize to ourselves the meaning of Plato in the altered circumstances of our own life.

\par  Like the importance which he attaches to festive entertainments, his depreciation of courage to the fourth place in the scale of virtue appears to be somewhat rhetorical and exaggerated. But he is speaking of courage in the lower sense of the term, not as including loyalty or temperance. He does not insist in this passage, as in the Protagoras, on the unity of the virtues; or, as in the Laches, on the identity of wisdom and courage. But he says that they all depend upon their leader mind, and that, out of the union of wisdom and temperance with courage, springs justice. Elsewhere he is disposed to regard temperance rather as a condition of all virtue than as a particular virtue. He generalizes temperance, as in the Republic he generalizes justice. The nature of the virtues is to run up into one another, and in many passages Plato makes but a faint effort to distinguish them. He still quotes the poets, somewhat enlarging, as his manner is, or playing with their meaning. The martial poet Tyrtaeus, and the oligarch Theognis, furnish him with happy illustrations of the two sorts of courage. The fear of fear, the division of goods into human and divine, the acknowledgment that peace and reconciliation are better than the appeal to the sword, the analysis of temperance into resistance of pleasure as well as endurance of pain, the distinction between the education which is suitable for a trade or profession, and for the whole of life, are important and probably new ethical conceptions. Nor has Plato forgotten his old paradox (Gorgias) that to be punished is better than to be unpunished, when he says, that to the bad man death is the only mitigation of his evil. He is not less ideal in many passages of the Laws than in the Gorgias or Republic. But his wings are heavy, and he is unequal to any sustained flight.

\par  There is more attempt at dramatic effect in the first book than in the later parts of the work. The outburst of martial spirit in the Lacedaemonian, 'O best of men'; the protest which the Cretan makes against the supposed insult to his lawgiver; the cordial acknowledgment on the part of both of them that laws should not be discussed publicly by those who live under their rule; the difficulty which they alike experience in following the speculations of the Athenian, are highly characteristic.

\par  In the second book, Plato pursues further his notion of educating by a right use of pleasure. He begins by conceiving an endless power of youthful life, which is to be reduced to rule and measure by harmony and rhythm. Men differ from the lower animals in that they are capable of musical discipline. But music, like all art, must be truly imitative, and imitative of what is true and good. Art and morality agree in rejecting pleasure as the criterion of good. True art is inseparable from the highest and most ennobling ideas. Plato only recognizes the identity of pleasure and good when the pleasure is of the higher kind. He is the enemy of 'songs without words,' which he supposes to have some confusing or enervating effect on the mind of the hearer; and he is also opposed to the modern degeneracy of the drama, which he would probably have illustrated, like Aristophanes, from Euripides and Agathon. From this passage may be gathered a more perfect conception of art than from any other of Plato's writings. He understands that art is at once imitative and ideal, an exact representation of truth, and also a representation of the highest truth. The same double view of art may be gathered from a comparison of the third and tenth books of the Republic, but is here more clearly and pointedly expressed.

\par  We are inclined to suspect that both here and in the Republic Plato exaggerates the influence really exercised by the song and the dance. But we must remember also the susceptible nature of the Greek, and the perfection to which these arts were carried by him. Further, the music had a sacred and Pythagorean character; the dance too was part of a religious festival. And only at such festivals the sexes mingled in public, and the youths passed under the eyes of their elders.

\par  At the beginning of the third book, Plato abruptly asks the question, What is the origin of states? The answer is, Infinite time. We have already seen—in the Theaetetus, where he supposes that in the course of ages every man has had numberless progenitors, kings and slaves, Greeks and barbarians; and in the Critias, where he says that nine thousand years have elapsed since the island of Atlantis fought with Athens—that Plato is no stranger to the conception of long periods of time. He imagines human society to have been interrupted by natural convulsions; and beginning from the last of these, he traces the steps by which the family has grown into the state, and the original scattered society, becoming more and more civilised, has finally passed into military organizations like those of Crete and Sparta. His conception of the origin of states is far truer in the Laws than in the Republic; but it must be remembered that here he is giving an historical, there an ideal picture of the growth of society.

\par  Modern enquirers, like Plato, have found in infinite ages the explanation not only of states, but of languages, men, animals, the world itself; like him, also, they have detected in later institutions the vestiges of a patriarchal state still surviving. Thus far Plato speaks as 'the spectator of all time and all existence,' who may be thought by some divine instinct to have guessed at truths which were hereafter to be revealed. He is far above the vulgar notion that Hellas is the civilized world (Statesman), or that civilization only began when the Hellenes appeared on the scene. But he has no special knowledge of 'the days before the flood'; and when he approaches more historical times, in preparing the way for his own theory of mixed government, he argues partially and erroneously. He is desirous of showing that unlimited power is ruinous to any state, and hence he is led to attribute a tyrannical spirit to the first Dorian kings. The decay of Argos and the destruction of Messene are adduced by him as a manifest proof of their failure; and Sparta, he thinks, was only preserved by the limitations which the wisdom of successive legislators introduced into the government. But there is no more reason to suppose that the Dorian rule of life which was followed at Sparta ever prevailed in Argos and Messene, than to assume that Dorian institutions were framed to protect the Greeks against the power of Assyria; or that the empire of Assyria was in any way affected by the Trojan war; or that the return of the Heraclidae was only the return of Achaean exiles, who received a new name from their leader Dorieus. Such fancies were chiefly based, as far as they had any foundation, on the use of analogy, which played a great part in the dawn of historical and geographical research. Because there was a Persian empire which was the natural enemy of the Greek, there must also have been an Assyrian empire, which had a similar hostility; and not only the fable of the island of Atlantis, but the Trojan war, in Plato's mind derived some features from the Persian struggle. So Herodotus makes the Nile answer to the Ister, and the valley of the Nile to the Red Sea. In the Republic, Plato is flying in the air regardless of fact and possibility—in the Laws, he is making history by analogy. In the former, he appears to be like some modern philosophers, absolutely devoid of historical sense; in the latter, he is on a level, not with Thucydides, or the critical historians of Greece, but with Herodotus, or even with Ctesias.

\par  The chief object of Plato in tracing the origin of society is to show the point at which regular government superseded the patriarchical authority, and the separate customs of different families were systematized by legislators, and took the form of laws consented to by them all. According to Plato, the only sound principle on which any government could be based was a mixture or balance of power. The balance of power saved Sparta, when the two other Heraclid states fell into disorder. Here is probably the first trace of a political idea, which has exercised a vast influence both in ancient and modern times. And yet we might fairly ask, a little parodying the language of Plato—O legislator, is unanimity only 'the struggle for existence'; or is the balance of powers in a state better than the harmony of them?

\par  In the fourth book we approach the realities of politics, and Plato begins to ascend to the height of his great argument. The reign of Cronos has passed away, and various forms of government have succeeded, which are all based on self-interest and self-preservation. Right and wrong, instead of being measured by the will of God, are created by the law of the state. The strongest assertions are made of the purely spiritual nature of religion—'Without holiness no man is accepted of God'; and of the duty of filial obedience,—'Honour thy parents.' The legislator must teach these precepts as well as command them. He is to be the educator as well as the lawgiver of future ages, and his laws are themselves to form a part of the education of the state. Unlike the poet, he must be definite and rational; he cannot be allowed to say one thing at one time, and another thing at another—he must know what he is about. And yet legislation has a poetical or rhetorical element, and must find words which will wing their way to the hearts of men. Laws must be promulgated before they are put in execution, and mankind must be reasoned with before they are punished. The legislator, when he promulgates a particular law, will courteously entreat those who are willing to hear his voice. Upon the rebellious only does the heavy blow descend. A sermon and a law in one, blending the secular punishment with the religious sanction, appeared to Plato a new idea which might have a great result in reforming the world. The experiment had never been tried of reasoning with mankind; the laws of others had never had any preambles, and Plato seems to have great pleasure in contemplating his discovery.

\par  In these quaint forms of thought and language, great principles of morals and legislation are enunciated by him for the first time. They all go back to mind and God, who holds the beginning, middle, and end of all things in His hand. The adjustment of the divine and human elements in the world is conceived in the spirit of modern popular philosophy, differing not much in the mode of expression. At first sight the legislator appears to be impotent, for all things are the sport of chance. But we admit also that God governs all things, and that chance and opportunity co-operate with Him (compare the saying, that chance is the name of the unknown cause). Lastly, while we acknowledge that God and chance govern mankind and provide the conditions of human action, experience will not allow us to deny a place to art. We know that there is a use in having a pilot, though the storm may overwhelm him; and a legislator is required to provide for the happiness of a state, although he will pray for favourable conditions under which he may exercise his art.

\par  BOOK V. Hear now, all ye who heard the laws about Gods and ancestors: Of all human possessions the soul is most divine, and most truly a man's own. For in every man there are two parts—a better which rules, and an inferior which serves; and the ruler is to be preferred to the servant. Wherefore I bid every one next after the Gods to honour his own soul, and he can only honour her by making her better. A man does not honour his soul by flattery, or gifts, or self-indulgence, or conceit of knowledge, nor when he blames others for his own errors; nor when he indulges in pleasure or refuses to bear pain; nor when he thinks that life at any price is a good, because he fears the world below, which, far from being an evil, may be the greatest good; nor when he prefers beauty to virtue—not reflecting that the soul, which came from heaven, is more honourable than the body, which is earth-born; nor when he covets dishonest gains, of which no amount is equal in value to virtue;—in a word, when he counts that which the legislator pronounces evil to be good, he degrades his soul, which is the divinest part of him. He does not consider that the real punishment of evil-doing is to grow like evil men, and to shun the conversation of the good: and that he who is joined to such men must do and suffer what they by nature do and say to one another, which suffering is not justice but retribution. For justice is noble, but retribution is only the companion of injustice. And whether a man escapes punishment or not, he is equally miserable; for in the one case he is not cured, and in the other case he perishes that the rest may be saved.

\par  The glory of man is to follow the better and improve the inferior. And the soul is that part of man which is most inclined to avoid the evil and dwell with the good. Wherefore also the soul is second only to the Gods in honour, and in the third place the body is to be esteemed, which often has a false honour. For honour is not to be given to the fair or the strong, or the swift or the tall, or to the healthy, any more than to their opposites, but to the mean states of all these habits; and so of property and external goods. No man should heap up riches that he may leave them to his children. The best condition for them as for the state is a middle one, in which there is a freedom without luxury. And the best inheritance of children is modesty. But modesty cannot be implanted by admonition only—the elders must set the example. He who would train the young must first train himself.

\par  He who honours his kindred and family may fairly expect that the Gods will give him children. He who would have friends must think much of their favours to him, and little of his to them. He who prefers to an Olympic, or any other victory, to win the palm of obedience to the laws, serves best both the state and his fellow-citizens. Engagements with strangers are to be deemed most sacred, because the stranger, having neither kindred nor friends, is immediately under the protection of Zeus, the God of strangers. A prudent man will not sin against the stranger; and still more carefully will he avoid sinning against the suppliant, which is an offence never passed over by the Gods.

\par  I will now speak of those particulars which are matters of praise and blame only, and which, although not enforced by the law, greatly affect the disposition to obey the law. Truth has the first place among the gifts of Gods and men, for truth begets trust; but he is not to be trusted who loves voluntary falsehood, and he who loves involuntary falsehood is a fool. Neither the ignorant nor the untrustworthy man is happy; for they have no friends in life, and die unlamented and untended. Good is he who does no injustice—better who prevents others from doing any—best of all who joins the rulers in punishing injustice. And this is true of goods and virtues in general; he who has and communicates them to others is the man of men; he who would, if he could, is second-best; he who has them and is jealous of imparting them to others is to be blamed, but the good or virtue which he has is to be valued still. Let every man contend in the race without envy; for the unenvious man increases the strength of the city; himself foremost in the race, he harms no one with calumny. Whereas the envious man is weak himself, and drives his rivals to despair with his slanders, thus depriving the whole city of incentives to the exercise of virtue, and tarnishing her glory. Every man should be gentle, but also passionate; for he must have the spirit to fight against incurable and malignant evil. But the evil which is remediable should be dealt with more in sorrow than anger. He who is unjust is to be pitied in any case; for no man voluntarily does evil or allows evil to exist in his soul. And therefore he who deals with the curable sort must be long-suffering and forbearing; but the incurable shall have the vials of our wrath poured out upon him. The greatest of all evils is self-love, which is thought to be natural and excusable, and is enforced as a duty, and yet is the cause of many errors. The lover is blinded about the beloved, and prefers his own interests to truth and right; but the truly great man seeks justice before all things. Self-love is the source of that ignorant conceit of knowledge which is always doing and never succeeding. Wherefore let every man avoid self-love, and follow the guidance of those who are better than himself. There are lesser matters which a man should recall to mind; for wisdom is like a stream, ever flowing in and out, and recollection flows in when knowledge is failing. Let no man either laugh or grieve overmuch; but let him control his feelings in the day of good- or ill-fortune, believing that the Gods will diminish the evils and increase the blessings of the righteous. These are thoughts which should ever occupy a good man's mind; he should remember them both in lighter and in more serious hours, and remind others of them.

\par  So much of divine matters and the relation of man to God. But man is man, and dependent on pleasure and pain; and therefore to acquire a true taste respecting either is a great matter. And what is a true taste? This can only be explained by a comparison of one life with another. Pleasure is an object of desire, pain of avoidance; and the absence of pain is to be preferred to pain, but not to pleasure. There are infinite kinds and degrees of both of them, and we choose the life which has more pleasure and avoid that which has less; but we do not choose that life in which the elements of pleasure are either feeble or equally balanced with pain. All the lives which we desire are pleasant; the choice of any others is due to inexperience.

\par  Now there are four lives—the temperate, the rational, the courageous, the healthful; and to these let us oppose four others—the intemperate, the foolish, the cowardly, the diseased. The temperate life has gentle pains and pleasures and placid desires, the intemperate life has violent delights, and still more violent desires. And the pleasures of the temperate exceed the pains, while the pains of the intemperate exceed the pleasures. But if this is true, none are voluntarily intemperate, but all who lack temperance are either ignorant or wanting in self-control: for men always choose the life which (as they think) exceeds in pleasure. The wise, the healthful, the courageous life have a similar advantage—they also exceed their opposites in pleasure. And, generally speaking, the life of virtue is far more pleasurable and honourable, fairer and happier far, than the life of vice. Let this be the preamble of our laws; the strain will follow.

\par  As in a web the warp is stronger than the woof, so should the rulers be stronger than their half-educated subjects. Let us suppose, then, that in the constitution of a state there are two parts, the appointment of the rulers, and the laws which they have to administer. But, before going further, there are some preliminary matters which have to be considered.

\par  As of animals, so also of men, a selection must be made; the bad breed must be got rid of, and the good retained. The legislator must purify them, and if he be not a despot he will find this task to be a difficult one. The severer kinds of purification are practised when great offenders are punished by death or exile, but there is a milder process which is necessary when the poor show a disposition to attack the property of the rich, for then the legislator will send them off to another land, under the name of a colony. In our case, however, we shall only need to purify the streams before they meet. This is often a troublesome business, but in theory we may suppose the operation performed, and the desired purity attained. Evil men we will hinder from coming, and receive the good as friends.

\par  Like the old Heraclid colony, we are fortunate in escaping the abolition of debts and the distribution of land, which are difficult and dangerous questions. But, perhaps, now that we are speaking of the subject, we ought to say how, if the danger existed, the legislator should try to avert it. He would have recourse to prayers, and trust to the healing influence of time. He would create a kindly spirit between creditors and debtors: those who have should give to those who have not, and poverty should be held to be rather the increase of a man's desires than the diminution of his property. Good-will is the only safe and enduring foundation of the political society; and upon this our city shall be built. The lawgiver, if he is wise, will not proceed with the arrangement of the state until all disputes about property are settled. And for him to introduce fresh grounds of quarrel would be madness.

\par  Let us now proceed to the distribution of our state, and determine the size of the territory and the number of the allotments. The territory should be sufficient to maintain the citizens in moderation, and the population should be numerous enough to defend themselves, and sometimes to aid their neighbours. We will fix the number of citizens at 5040, to which the number of houses and portions of land shall correspond. Let the number be divided into two parts and then into three; for it is very convenient for the purposes of distribution, and is capable of fifty-nine divisions, ten of which proceed without interval from one to ten. Here are numbers enough for war and peace, and for all contracts and dealings. These properties of numbers are true, and should be ascertained with a view to use.

\par  In carrying out the distribution of the land, a prudent legislator will be careful to respect any provision for religious worship which has been sanctioned by ancient tradition or by the oracles of Delphi, Dodona, or Ammon. All sacrifices, and altars, and temples, whatever may be their origin, should remain as they are. Every division should have a patron God or hero; to these a portion of the domain should be appropriated, and at their temples the inhabitants of the districts should meet together from time to time, for the sake of mutual help and friendship. All the citizens of a state should be known to one another; for where men are in the dark about each other's characters, there can be no justice or right administration. Every man should be true and single-minded, and should not allow himself to be deceived by others.

\par  And now the game opens, and we begin to move the pieces. At first sight, our constitution may appear singular and ill-adapted to a legislator who has not despotic power; but on second thoughts will be deemed to be, if not the very best, the second best. For there are three forms of government, a first, a second, and a third best, out of which Cleinias has now to choose. The first and highest form is that in which friends have all things in common, including wives and property,—in which they have common fears, hopes, desires, and do not even call their eyes or their hands their own. This is the ideal state; than which there never can be a truer or better—a state, whether inhabited by Gods or sons of Gods, which will make the dwellers therein blessed. Here is the pattern on which we must ever fix our eyes; but we are now concerned with another, which comes next to it, and we will afterwards proceed to a third.

\par  Inasmuch as our citizens are not fitted either by nature or education to receive the saying, Friends have all things in common, let them retain their houses and private property, but use them in the service of their country, who is their God and parent, and of the Gods and demigods of the land. Their first care should be to preserve the number of their lots. This may be secured in the following manner: when the possessor of a lot dies, he shall leave his lot to his best-beloved child, who will become the heir of all duties and interests, and will minister to the Gods and to the family, to the living and to the dead. Of the remaining children, the females must be given in marriage according to the law to be hereafter enacted; the males may be assigned to citizens who have no children of their own. How to equalize families and allotments will be one of the chief cares of the guardians of the laws. When parents have too many children they may give to those who have none, or couples may abstain from having children, or, if there is a want of offspring, special care may be taken to obtain them; or if the number of citizens becomes excessive, we may send away the surplus to found a colony. If, on the other hand, a war or plague diminishes the number of inhabitants, new citizens must be introduced; and these ought not, if possible, to be men of low birth or inferior training; but even God, it is said, cannot always fight against necessity.

\par  Wherefore we will thus address our citizens:—Good friends, honour order and equality, and above all the number 5040. Secondly, respect the original division of the lots, which must not be infringed by buying and selling, for the law says that the land which a man has is sacred and is given to him by God. And priests and priestesses will offer frequent sacrifices and pray that he who alienates either house or lot may receive the punishment which he deserves, and their prayers shall be inscribed on tablets of cypress-wood for the instruction of posterity. The guardians will keep a vigilant watch over the citizens, and they will punish those who disobey God and the law.

\par  To appreciate the benefit of such an institution a man requires to be well educated; for he certainly will not make a fortune in our state, in which all illiberal occupations are forbidden to freemen. The law also provides that no private person shall have gold or silver, except a little coin for daily use, which will not pass current in other countries. The state must also possess a common Hellenic currency, but this is only to be used in defraying the expenses of expeditions, or of embassies, or while a man is on foreign travels; but in the latter case he must deliver up what is over, when he comes back, to the treasury in return for an equal amount of local currency, on pain of losing the sum in question; and he who does not inform against an offender is to be mulcted in a like sum. No money is to be given or taken as a dowry, or to be lent on interest. The law will not protect a man in recovering either interest or principal. All these regulations imply that the aim of the legislator is not to make the city as rich or as mighty as possible, but the best and happiest. Now men can hardly be at the same time very virtuous and very rich. And why? Because he who makes twice as much and saves twice as much as he ought, receiving where he ought not and not spending where he ought, will be at least twice as rich as he who makes money where he ought, and spends where he ought. On the other hand, an utterly bad man is generally profligate and poor, while he who acquires honestly, and spends what he acquires on noble objects, can hardly be very rich. A very rich man is therefore not a good man, and therefore not a happy one. But the object of our laws is to make the citizens as friendly and happy as possible, which they cannot be if they are always at law and injuring each other in the pursuit of gain. And therefore we say that there is to be no silver or gold in the state, nor usury, nor the rearing of the meaner kinds of live-stock, but only agriculture, and only so much of this as will not lead men to neglect that for the sake of which money is made, first the soul and afterwards the body; neither of which are good for much without music and gymnastic. Money is to be held in honour last or third; the highest interests being those of the soul, and in the second class are to be ranked those of the body. This is the true order of legislation, which would be inverted by placing health before temperance, and wealth before health.

\par  It might be well if every man could come to the colony having equal property; but equality is impossible, and therefore we must avoid causes of offence by having property valued and by equalizing taxation. To this end, let us make four classes in which the citizens may be placed according to the measure of their original property, and the changes of their fortune. The greatest of evils is revolution; and this, as the law will say, is caused by extremes of poverty or wealth. The limit of poverty shall be the lot, which must not be diminished, and may be increased fivefold, but not more. He who exceeds the limit must give up the excess to the state; but if he does not, and is informed against, the surplus shall be divided between the informer and the Gods, and he shall pay a sum equal to the surplus out Of his own property. All property other than the lot must be inscribed in a register, so that any disputes which arise may be easily determined.

\par  The city shall be placed in a suitable situation, as nearly as possible in the centre of the country, and shall be divided into twelve wards. First, we will erect an acropolis, encircled by a wall, within which shall be placed the temples of Hestia, and Zeus, and Athene. From this shall be drawn lines dividing the city, and also the country, into twelve sections, and the country shall be subdivided into 5040 lots. Each lot shall contain two parts, one at a distance, the other near the city; and the distance of one part shall be compensated by the nearness of the other, the badness and goodness by the greater or less size. Twelve lots will be assigned to twelve Gods, and they will give their names to the tribes. The divisions of the city shall correspond to those of the country; and every man shall have two habitations, one near the centre of the country, the other at the extremity.

\par  The objection will naturally arise, that all the advantages of which we have been speaking will never concur. The citizens will not tolerate a settlement in which they are deprived of gold and silver, and have the number of their families regulated, and the sites of their houses fixed by law. It will be said that our city is a mere image of wax. And the legislator will answer: 'I know it, but I maintain that we ought to set forth an ideal which is as perfect as possible. If difficulties arise in the execution of the plan, we must avoid them and carry out the remainder. But the legislator must first be allowed to complete his idea without interruption.'

\par  The number twelve, which we have chosen for the number of division, must run through all parts of the state,—phratries, villages, ranks of soldiers, coins, and measures wet and dry, which are all to be made commensurable with one another. There is no meanness in requiring that the smallest vessels should have a common measure; for the divisions of number are useful in measuring height and depth, as well as sounds and motions, upwards or downwards, or round and round. The legislator should impress on his citizens the value of arithmetic. No instrument of education has so much power; nothing more tends to sharpen and inspire the dull intellect. But the legislator must be careful to instil a noble and generous spirit into the students, or they will tend to become cunning rather than wise. This may be proved by the example of the Egyptians and Phoenicians, who, notwithstanding their knowledge of arithmetic, are degraded in their general character; whether this defect in them is due to some natural cause or to a bad legislator. For it is clear that there are great differences in the power of regions to produce good men: heat and cold, and water and food, have great effects both on body and soul; and those spots are peculiarly fortunate in which the air is holy, and the Gods are pleased to dwell. To all this the legislator must attend, so far as in him lies.

\par  BOOK VI. And now we are about to consider (1) the appointment of magistrates; (2) the laws which they will have to administer must be determined. I may observe by the way that laws, however good, are useless and even injurious unless the magistrates are capable of executing them. And therefore (1) the intended rulers of our imaginary state should be tested from their youth upwards until the time of their election; and (2) those who are to elect them ought to be trained in habits of law, that they may form a right judgment of good and bad men. But uneducated colonists, who are unacquainted with each other, will not be likely to choose well. What, then, shall we do? I will tell you: The colony will have to be intrusted to the ten commissioners, of whom you are one, and I will help you and them, which is my reason for inventing this romance. And I cannot bear that the tale should go wandering about the world without a head,—it will be such an ugly monster. 'Very good.' Yes; and I will be as good as my word, if God be gracious and old age permit. But let us not forget what a courageously mad creation this our city is. 'What makes you say so?' Why, surely our courage is shown in imagining that the new colonists will quietly receive our laws? For no man likes to receive laws when they are first imposed: could we only wait until those who had been educated under them were grown up, and of an age to vote in the public elections, there would be far greater reason to expect permanence in our institutions. 'Very true.' The Cnosian founders should take the utmost pains in the matter of the colony, and in the election of the higher officers, particularly of the guardians of the law. The latter should be appointed in this way: The Cnosians, who take the lead in the colony, together with the colonists, will choose thirty-seven persons, of whom nineteen will be colonists, and the remaining eighteen Cnosians—you must be one of the eighteen yourself, and become a citizen of the new state. 'Why do not you and Megillus join us?' Athens is proud, and Sparta too; and they are both a long way off. But let me proceed with my scheme. When the state is permanently established, the mode of election will be as follows: All who are serving, or have served, in the army will be electors; and the election will be held in the most sacred of the temples. The voter will place on the altar a tablet, inscribing thereupon the name of the candidate whom he prefers, and of his father, tribe, and ward, writing at the side of them his own name in like manner; and he may take away any tablet which does not appear written to his mind, and place it in the Agora for thirty days. The 300 who obtain the greatest number of votes will be publicly announced, and out of them there will be a second election of 100; and out of the 100 a third and final election of thirty-seven, accompanied by the solemnity of the electors passing through victims. But then who is to arrange all this? There is a common saying, that the beginning is half the whole; and I should say a good deal more than half. 'Most true.' The only way of making a beginning is from the parent city; and though in after ages the tie may be broken, and quarrels may arise between them, yet in early days the child naturally looks to the mother for care and education. And, as I said before, the Cnosians ought to take an interest in the colony, and select 100 elders of their own citizens, to whom shall be added 100 of the colonists, to arrange and supervise the first elections and scrutinies; and when the colony has been started, the Cnosians may return home and leave the colonists to themselves.

\par  The thirty-seven magistrates who have been elected in the manner described, shall have the following duties: first, they shall be guardians of the law; secondly, of the registers of property in the four classes—not including the one, two, three, four minae, which are allowed as a surplus. He who is found to possess what is not entered in the registers, in addition to the confiscation of such property shall be proceeded against by law, and if he be cast he shall lose his share in the public property and in distributions of money; and his sentence shall be inscribed in some public place. The guardians are to continue in office twenty years only, and to commence holding office at fifty years, or if elected at sixty they are not to remain after seventy.

\par  Generals have now to be elected, and commanders of horse and brigadiers of foot. The generals shall be natives of the city, proposed by the guardians of the law, and elected by those who are or have been of the age for military service. Any one may challenge the person nominated and start another candidate, whom he affirms upon oath to be better qualified. The three who obtain the greatest number of votes shall be elected. The generals thus elected shall propose the taxiarchs or brigadiers, and the challenge may be made, and the voting shall take place, in the same manner as before. The elective assembly will be presided over in the first instance, and until the prytanes and council come into being, by the guardians of the law in some holy place; and they shall divide the citizens into three divisions,—hoplites, cavalry, and the rest of the army—placing each of them by itself. All are to vote for generals and cavalry officers. The brigadiers are to be voted for only by the hoplites. Next, the cavalry are to choose phylarchs for the generals; but captains of archers and other irregular troops are to be appointed by the generals themselves. The cavalry-officers shall be proposed and voted upon by the same persons who vote for the generals. The two who have the greatest number of votes shall be leaders of all the horse. Disputes about the voting may be raised once or twice, but, if a third time, the presiding officers shall decide.

\par  The council shall consist of 360, who may be conveniently divided into four sections, making ninety councillors of each class. In the first place, all the citizens shall select candidates from the first class; and they shall be compelled to vote under pain of a fine. This shall be the business of the first day. On the second day a similar selection shall be made from the second class under the same conditions. On the third day, candidates shall be selected from the third class; but the compulsion to vote shall only extend to the voters of the first three classes. On the fourth day, members of the council shall be selected from the fourth class; they shall be selected by all, but the compulsion to vote shall only extend to the second class, who, if they do not vote, shall pay a fine of triple the amount which was exacted at first, and to the first class, who shall pay a quadruple fine. On the fifth day, the names shall be exhibited, and out of them shall be chosen by all the citizens 180 of each class: these are severally to be reduced by lot to ninety, and 90 x 4 will form the council for the year.

\par  The mode of election which has been described is a mean between monarchy and democracy, and such a mean should ever be observed in the state. For servants and masters cannot be friends, and, although equality makes friendship, we must remember that there are two sorts of equality. One of them is the rule of number and measure; but there is also a higher equality, which is the judgment of Zeus. Of this he grants but little to mortal men; yet that little is the source of the greatest good to cities and individuals. It is proportioned to the nature of each man; it gives more to the better and less to the inferior, and is the true political justice; to this we in our state desire to look, as every legislator should, not to the interests either of tyrants or mobs. But justice cannot always be strictly enforced, and then equity and mercy have to be substituted: and for a similar reason, when true justice will not be endured, we must have recourse to the rougher justice of the lot, which God must be entreated to guide.

\par  These are the principal means of preserving the state, but perpetual care will also be required. When a ship is sailing on the sea, vigilance must not be relaxed night or day; and the vessel of state is tossing in a political sea, and therefore watch must continually succeed watch, and rulers must join hands with rulers. A small body will best perform this duty, and therefore the greater part of the 360 senators may be permitted to go and manage their own affairs, but a twelfth portion must be set aside in each month for the administration of the state. Their business will be to receive information and answer embassies; also they must endeavour to prevent or heal internal disorders; and with this object they must have the control of all assemblies of the citizens.

\par  Besides the council, there must be wardens of the city and of the agora, who will superintend houses, ways, harbours, markets, and fountains, in the city and the suburbs, and prevent any injury being done to them by man or beast. The temples, also, will require priests and priestesses. Those who hold the priestly office by hereditary tenure shall not be disturbed; but as there will probably be few or none such in a new colony, priests and priestesses shall be appointed for the Gods who have no servants. Some of these officers shall be elected by vote, some by lot; and all classes shall mingle in a friendly manner at the elections. The appointment of priests should be left to God,—that is, to the lot; but the person elected must prove that he is himself sound in body and of legitimate birth, and that his family has been free from homicide or any other stain of impurity. Priests and priestesses are to be not less than sixty years of age, and shall hold office for a year only. The laws which are to regulate matters of religion shall be brought from Delphi, and interpreters appointed to superintend their execution. These shall be elected in the following manner:—The twelve tribes shall be formed into three bodies of four, each of which shall select four candidates, and this shall be done three times: of each twelve thus selected the three who receive the largest number of votes, nine in all, after undergoing a scrutiny shall go to Delphi, in order that the God may elect one out of each triad. They shall be appointed for life; and when any of them dies, another shall be elected by the four tribes who made the original appointment. There shall also be treasurers of the temples; three for the greater temples, two for the lesser, and one for those of least importance.

\par  The defence of the city should be committed to the generals and other officers of the army, and to the wardens of the city and agora. The defence of the country shall be on this wise:—The twelve tribes shall allot among themselves annually the twelve divisions of the country, and each tribe shall appoint five wardens and commanders of the watch. The five wardens in each division shall choose out of their own tribe twelve guards, who are to be between twenty-five and thirty years of age. Both the wardens and the guards are to serve two years; and they shall make a round of the divisions, staying a month in each. They shall go from West to East during the first year, and back from East to West during the second. Thus they will gain a perfect knowledge of the country at every season of the year.

\par  While on service, their first duty will be to see that the country is well protected by means of fortifications and entrenchments; they will use the beasts of burden and the labourers whom they find on the spot, taking care however not to interfere with the regular course of agriculture. But while they thus render the country as inaccessible as possible to enemies, they will also make it as accessible as possible to friends by constructing and maintaining good roads. They will restrain and preserve the rain which comes down from heaven, making the barren places fertile, and the wet places dry. They will ornament the fountains with plantations and buildings, and provide water for irrigation at all seasons of the year. They will lead the streams to the temples and groves of the Gods; and in such spots the youth shall make gymnasia for themselves, and warm baths for the aged; there the rustic worn with toil will receive a kindly welcome, and be far better treated than at the hands of an unskilful doctor.

\par  These works will be both useful and ornamental; but the sixty wardens must not fail to give serious attention to other duties. For they must watch over the districts assigned to them, and also act as judges. In small matters the five commanders shall decide: in greater matters up to three minae, the five commanders and the twelve guards. Like all other judges, except those who have the final decision, they shall be liable to give an account. If the wardens impose unjust tasks on the villagers, or take by force their crops or implements, or yield to flattery or bribes in deciding suits, let them be publicly dishonoured. In regard to any other wrong-doing, if the question be of a mina, let the neighbours decide; but if the accused person will not submit, trusting that his monthly removals will enable him to escape payment, and also in suits about a larger amount, the injured party may have recourse to the common court; in the former case, if successful, he may exact a double penalty.

\par  The wardens and guards, while on their two years' service, shall live and eat together, and the guard who is absent from the daily meals without permission or sleeps out at night, shall be regarded as a deserter, and may be punished by any one who meets him. If any of the commanders is guilty of such an irregularity, the whole sixty shall have him punished; and he of them who screens him shall suffer a still heavier penalty than the offender himself. Now by service a man learns to rule; and he should pride himself upon serving well the laws and the Gods all his life, and upon having served ancient and honourable men in his youth. The twelve and the five should be their own servants, and use the labour of the villagers only for the good of the public. Let them search the country through, and acquire a perfect knowledge of every locality; with this view, hunting and field sports should be encouraged.

\par  Next we have to speak of the elections of the wardens of the agora and of the city. The wardens of the city shall be three in number, and they shall have the care of the streets, roads, buildings, and also of the water-supply. They shall be chosen out of the highest class, and when the number of candidates has been reduced to six who have the greatest number of votes, three out of the six shall be taken by lot, and, after a scrutiny, shall be admitted to their office. The wardens of the agora shall be five in number—ten are to be first elected, and every one shall vote for all the vacant places; the ten shall be afterwards reduced to five by lot, as in the former election. The first and second class shall be compelled to go to the assembly, but not the third and fourth, unless they are specially summoned. The wardens of the agora shall have the care of the temples and fountains which are in the agora, and shall punish those who injure them by stripes and bonds, if they be slaves or strangers; and by fines, if they be citizens. And the wardens of the city shall have a similar power of inflicting punishment and fines in their own department.

\par  In the next place, there must be directors of music and gymnastic; one class of them superintending gymnasia and schools, and the attendance and lodging of the boys and girls—the other having to do with contests of music and gymnastic. In musical contests there shall be one kind of judges of solo singing or playing, who will judge of rhapsodists, flute-players, harp-players and the like, and another of choruses. There shall be choruses of men and boys and maidens—one director will be enough to introduce them all, and he should not be less than forty years of age; secondly, of solos also there shall be one director, aged not less than thirty years; he will introduce the competitors and give judgment upon them. The director of the choruses is to be elected in an assembly at which all who take an interest in music are compelled to attend, and no one else. Candidates must only be proposed for their fitness, and opposed on the ground of unfitness. Ten are to be elected by vote, and the one of these on whom the lot falls shall be director for a year. Next shall be elected out of the second and third classes the judges of gymnastic contests, who are to be three in number, and are to be tested, after being chosen by lot out of twenty who have been elected by the three highest classes—these being compelled to attend at the election.

\par  One minister remains, who will have the general superintendence of education. He must be not less than fifty years old, and be himself the father of children born in wedlock. His office must be regarded by all as the highest in the state. For the right growth of the first shoot in plants and animals is the chief cause of matured perfection. Man is supposed to be a tame animal, but he becomes either the gentlest or the fiercest of creatures, accordingly as he is well or ill educated. Wherefore he who is elected to preside over education should be the best man possible. He shall hold office for five years, and shall be elected out of the guardians of the law, by the votes of the other magistrates with the exception of the senate and prytanes; and the election shall be held by ballot in the temple of Apollo.

\par  When a magistrate dies before his term of office has expired, another shall be elected in his place; and, if the guardian of an orphan dies, the relations shall appoint another within ten days, or be fined a drachma a day for neglect.

\par  The city which has no courts of law will soon cease to be a city; and a judge who sits in silence and leaves the enquiry to the litigants, as in arbitrations, is not a good judge. A few judges are better than many, but the few must be good. The matter in dispute should be clearly elicited; time and examination will find out the truth. Causes should first be tried before a court of neighbours: if the decision is unsatisfactory, let them be referred to a higher court; or, if necessary, to a higher still, of which the decision shall be final.

\par  Every magistrate is a judge, and every judge is a magistrate, on the day on which he is deciding the suit. This will therefore be an appropriate place to speak of judges and their functions. The supreme tribunal will be that on which the litigants agree; and let there be two other tribunals, one for public and the other for private causes. The high court of appeal shall be composed as follows:—All the officers of state shall meet on the last day but one of the year in some temple, and choose for a judge the best man out of every magistracy: and those who are elected, after they have undergone a scrutiny, shall be judges of appeal. They shall give their decisions openly, in the presence of the magistrates who have elected them; and the public may attend. If anybody charges one of them with having intentionally decided wrong, he shall lay his accusation before the guardians of the law, and if the judge be found guilty he shall pay damages to the extent of half the injury, unless the guardians of the law deem that he deserves a severer punishment, in which case the judges shall assess the penalty.

\par  As the whole people are injured by offences against the state, they should share in the trial of them. Such causes should originate with the people and be decided by them: the enquiry shall take place before any three of the highest magistrates upon whom the defendant and plaintiff can agree. Also in private suits all should judge as far as possible, and therefore there should be a court of law in every ward; for he who has no share in the administration of justice, believes that he has no share in the state. The judges in these courts shall be elected by lot and give their decision at once. The final judgment in all cases shall rest with the court of appeal. And so, having done with the appointment of courts and the election of officers, we will now make our laws.

\par  'Your way of proceeding, Stranger, is admirable.'

\par  Then so far our old man's game of play has gone off well.

\par  'Say, rather, our serious and noble pursuit.'

\par  Perhaps; but let me ask you whether you have ever observed the manner in which painters put in and rub out colour: yet their endless labour will last but a short time, unless they leave behind them some successor who will restore the picture and remove its defects. 'Certainly.' And have we not a similar object at the present moment? We are old ourselves, and therefore we must leave our work of legislation to be improved and perfected by the next generation; not only making laws for our guardians, but making them lawgivers. 'We must at least do our best.' Let us address them as follows. Beloved saviours of the laws, we give you an outline of legislation which you must fill up, according to a rule which we will prescribe for you. Megillus and Cleinias and I are agreed, and we hope that you will agree with us in thinking, that the whole energies of a man should be devoted to the attainment of manly virtue, whether this is to be gained by study, or habit, or desire, or opinion. And rather than accept institutions which tend to degrade and enslave him, he should fly his country and endure any hardship. These are our principles, and we would ask you to judge of our laws, and praise or blame them, accordingly as they are or are not capable of improving our citizens.

\par  And first of laws concerning religion. We have already said that the number 5040 has many convenient divisions: and we took a twelfth part of this (420), which is itself divisible by twelve, for the number of the tribe. Every divisor is a gift of God, and corresponds to the months of the year and to the revolution of the universe. All cities have a number, but none is more fortunate than our own, which can be divided by all numbers up to 12, with the exception of 11, and even by 11, if two families are deducted. And now let us divide the state, assigning to each division some God or demigod, who shall have altars raised to them, and sacrifices offered twice a month; and assemblies shall be held in their honour, twelve for the tribes, and twelve for the city, corresponding to their divisions. The object of them will be first to promote religion, secondly to encourage friendship and intercourse between families; for families must be acquainted before they marry into one another, or great mistakes will occur. At these festivals there shall be innocent dances of young men and maidens, who may have the opportunity of seeing one another in modest undress. To the details of all this the masters of choruses and the guardians will attend, embodying in laws the results of their experience; and, after ten years, making the laws permanent, with the consent of the legislator, if he be alive, or, if he be not alive, of the guardians of the law, who shall perfect them and settle them once for all. At least, if any further changes are required, the magistrates must take the whole people into counsel, and obtain the sanction of all the oracles.

\par  Whenever any one who is between the ages of twenty-five and thirty-five wants to marry, let him do so; but first let him hear the strain which we will address to him:—

\par  My son, you ought to marry, but not in order to gain wealth or to avoid poverty; neither should you, as men are wont to do, choose a wife who is like yourself in property and character. You ought to consult the interests of the state rather than your own pleasure; for by equal marriages a society becomes unequal. And yet to enact a law that the rich and mighty shall not marry the rich and mighty, that the quick shall be united to the slow, and the slow to the quick, will arouse anger in some persons and laughter in others; for they do not understand that opposite elements ought to be mingled in the state, as wine should be mingled with water. The object at which we aim must therefore be left to the influence of public opinion. And do not forget our former precept, that every one should seek to attain immortality and raise up a fair posterity to serve God.—Let this be the prelude of the law about the duty of marriage. But if a man will not listen, and at thirty-five years of age is still unmarried, he shall pay an annual fine: if he be of the first class, 100 drachmas; if of the second, 70; if of the third, 60; and if of the fourth, 30. This fine shall be sacred to Here; and if he refuse to pay, a tenfold penalty shall be exacted by the treasurer of Here, who shall be responsible for the payment. Further, the unmarried man shall receive no honour or obedience from the young, and he shall not retain the right of punishing others. A man is neither to give nor receive a dowry beyond a certain fixed sum; in our state, for his consolation, if he be poor, let him know that he need neither receive nor give one, for every citizen is provided with the necessaries of life. Again, if the woman is not rich, her husband will not be her humble servant. He who disobeys this law shall pay a fine according to his class, which shall be exacted by the treasurers of Here and Zeus.

\par  The betrothal of the parties shall be made by the next of kin, or if there are none, by the guardians. The offerings and ceremonies of marriage shall be determined by the interpreters of sacred rites. Let the wedding party be moderate; five male and five female friends, and a like number of kinsmen, will be enough. The expense should not exceed, for the first class, a mina; and for the second, half a mina; and should be in like proportion for the other classes. Extravagance is to be regarded as vulgarity and ignorance of nuptial proprieties. Much wine is only to be drunk at the festivals of Dionysus, and certainly not on the occasion of a marriage. The bride and bridegroom, who are taking a great step in life, ought to have all their wits about them; they should be especially careful of the night on which God may give them increase, and which this will be none can say. Their bodies and souls should be in the most temperate condition; they should abstain from all that partakes of the nature of disease or vice, which will otherwise become hereditary. There is an original divinity in man which preserves all things, if used with proper respect. He who marries should make one of the two houses on the lot the nest and nursery of his young; he should leave his father and mother, and then his affection for them will be only increased by absence. He will go forth as to a colony, and will there rear up his offspring, handing on the torch of life to another generation.

\par  About property in general there is little difficulty, with the exception of property in slaves, which is an institution of a very doubtful character. The slavery of the Helots is approved by some and condemned by others; and there is some doubt even about the slavery of the Mariandynians at Heraclea and of the Thessalian Penestae. This makes us ask, What shall we do about slaves? To which every one would agree in replying,—Let us have the best and most attached whom we can get. All of us have heard stories of slaves who have been better to their masters than sons or brethren. Yet there is an opposite doctrine, that slaves are never to be trusted; as Homer says, 'Slavery takes away half a man's understanding.' And different persons treat them in different ways: there are some who never trust them, and beat them like dogs, until they make them many times more slavish than they were before; and others pursue the opposite plan. Man is a troublesome animal, as has been often shown, Megillus, notably in the revolts of the Messenians; and great mischiefs have arisen in countries where there are large bodies of slaves of one nationality. Two rules may be given for their management: first that they should not, if possible, be of the same country or have a common language; and secondly, that they should be treated by their master with more justice even than equals, out of regard to himself quite as much as to them. For he who is righteous in the treatment of his slaves, or of any inferiors, will sow in them the seed of virtue. Masters should never jest with their slaves: this, which is a common but foolish practice, increases the difficulty and painfulness of managing them.

\par  Next as to habitations. These ought to have been spoken of before; for no man can marry a wife, and have slaves, who has not a house for them to live in. Let us supply the omission. The temples should be placed round the Agora, and the city built in a circle on the heights. Near the temples, which are holy places and the habitations of the Gods, should be buildings for the magistrates, and the courts of law, including those in which capital offences are to be tried. As to walls, Megillus, I agree with Sparta that they should sleep in the earth; 'cold steel is the best wall,' as the poet finely says. Besides, how absurd to be sending out our youth to fortify and guard the borders of our country, and then to build a city wall, which is very unhealthy, and is apt to make people fancy that they may run there and rest in idleness, not knowing that true repose comes from labour, and that idleness is only a renewal of trouble. If, however, there must be a wall, the private houses had better be so arranged as to form one wall; this will have an agreeable aspect, and the building will be safer and more defensible. These objects should be attended to at the foundation of the city. The wardens of the city must see that they are carried out; and they must also enforce cleanliness, and preserve the public buildings from encroachments. Moreover, they must take care to let the rain flow off easily, and must regulate other matters concerning the general administration of the city. If any further enactments prove to be necessary, the guardians of the law must supply them.

\par  And now, having provided buildings, and having married our citizens, we will proceed to speak of their mode of life. In a well-constituted state, individuals cannot be allowed to live as they please. Why do I say this? Because I am going to enact that the bridegroom shall not absent himself from the common meals. They were instituted originally on the occasion of some war, and, though deemed singular when first founded, they have tended greatly to the security of states. There was a difficulty in introducing them, but there is no difficulty in them now. There is, however, another institution about which I would speak, if I dared. I may preface my proposal by remarking that disorder in a state is the source of all evil, and order of all good. Now in Sparta and Crete there are common meals for men, and this, as I was saying, is a divine and natural institution. But the women are left to themselves; they live in dark places, and, being weaker, and therefore wickeder, than men, they are at the bottom of a good deal more than half the evil of states. This must be corrected, and the institution of common meals extended to both sexes. But, in the present unfortunate state of opinion, who would dare to establish them? And still more, who can compel women to eat and drink in public? They will defy the legislator to drag them out of their holes. And in any other state such a proposal would be drowned in clamour, but in our own I think that I can show the attempt to be just and reasonable. 'There is nothing which we should like to hear better.' Listen, then; having plenty of time, we will go back to the beginning of things, which is an old subject with us. 'Right.' Either the race of mankind never had a beginning and will never have an end, or the time which has elapsed since man first came into being is all but infinite. 'No doubt.' And in this infinity of time there have been changes of every kind, both in the order of the seasons and in the government of states and in the customs of eating and drinking. Vines and olives were at length discovered, and the blessings of Demeter and Persephone, of which one Triptolemus is said to have been the minister; before his time the animals had been eating one another. And there are nations in which mankind still sacrifice their fellow-men, and other nations in which they lead a kind of Orphic existence, and will not sacrifice animals, or so much as taste of a cow—they offer fruits or cakes moistened with honey. Perhaps you will ask me what is the bearing of these remarks? 'We would gladly hear.' I will endeavour to explain their drift. I see that the virtue of human life depends on the due regulation of three wants or desires. The first is the desire of meat, the second of drink; these begin with birth, and make us disobedient to any voice other than that of pleasure. The third and fiercest and greatest need is felt latest; this is love, which is a madness setting men's whole nature on fire. These three disorders of mankind we must endeavour to restrain by three mighty influences—fear, and law, and reason, which, with the aid of the Muses and the Gods of contests, may extinguish our lusts.

\par  But to return. After marriage let us proceed to the generation of children, and then to their nurture and education—thus gradually approaching the subject of syssitia. There are, however, some other points which are suggested by the three words—meat, drink, love. 'Proceed,' the bride and bridegroom ought to set their mind on having a brave offspring. Now a man only succeeds when he takes pains; wherefore the bridegroom ought to take special care of the bride, and the bride of the bridegroom, at the time when their children are about to be born. And let there be a committee of matrons who shall meet every day at the temple of Eilithyia at a time fixed by the magistrates, and inform against any man or woman who does not observe the laws of married life. The time of begetting children and the supervision of the parents shall last for ten years only; if at the expiration of this period they have no children, they may part, with the consent of their relatives and the official matrons, and with a due regard to the interests of either; if a dispute arise, ten of the guardians of the law shall be chosen as arbiters. The matrons shall also have power to enter the houses of the young people, if necessary, and to advise and threaten them. If their efforts fail, let them go to the guardians of the law; and if they too fail, the offender, whether man or woman, shall be forbidden to be present at all family ceremonies. If when the time for begetting children has ceased, either husband or wife have connexion with others who are of an age to beget children, they shall be liable to the same penalties as those who are still having a family. But when both parties have ceased to beget children there shall be no penalties. If men and women live soberly, the enactments of law may be left to slumber; punishment is necessary only when there is great disorder of manners.

\par  The first year of children's lives is to be registered in their ancestral temples; the name of the archon of the year is to be inscribed on a whited wall in every phratry, and the names of the living members of the phratry close to them, to be erased at their decease. The proper time of marriage for a woman shall be from sixteen years to twenty; for a man, from thirty to thirty-five (compare Republic). The age of holding office for a woman is to be forty, for a man thirty years. The time for military service for a man is to be from twenty years to sixty; for a woman, from the time that she has ceased to bear children until fifty.

\par  BOOK VII. Now that we have married our citizens and brought their children into the world, we have to find nurture and education for them. This is a matter of precept rather than of law, and cannot be precisely regulated by the legislator. For minute regulations are apt to be transgressed, and frequent transgressions impair the habit of obedience to the laws. I speak darkly, but I will also try to exhibit my wares in the light of day. Am I not right in saying that a good education tends to the improvement of body and mind? 'Certainly.' And the body is fairest which grows up straight and well-formed from the time of birth. 'Very true.' And we observe that the first shoot of every living thing is the greatest; many even contend that man is not at twenty-five twice the height that he was at five. 'True.' And growth without exercise of the limbs is the source of endless evils in the body. 'Yes.' The body should have the most exercise when growing most. 'What, the bodies of young infants?' Nay, the bodies of unborn infants. I should like to explain to you this singular kind of gymnastics. The Athenians are fond of cock-fighting, and the people who keep cocks carry them about in their hands or under their arms, and take long walks, to improve, not their own health, but the health of the birds. Here is a proof of the usefulness of motion, whether of rocking, swinging, riding, or tossing upon the wave; for all these kinds of motion greatly increase strength and the powers of digestion. Hence we infer that our women, when they are with child, should walk about and fashion the embryo; and the children, when born, should be carried by strong nurses,—there must be more than one of them,—and should not be suffered to walk until they are three years old. Shall we impose penalties for the neglect of these rules? The greatest penalty, that is, ridicule, and the difficulty of making the nurses do as we bid them, will be incurred by ourselves. 'Then why speak of such matters?' In the hope that heads of families may learn that the due regulation of them is the foundation of law and order in the state.

\par  And now, leaving the body, let us proceed to the soul; but we must first repeat that perpetual motion by night and by day is good for the young creature. This is proved by the Corybantian cure of motion, and by the practice of nurses who rock children in their arms, lapping them at the same time in sweet strains. And the reason of this is obvious. The affections, both of the Bacchantes and of the children, arise from fear, and this fear is occasioned by something wrong which is going on within them. Now a violent external commotion tends to calm the violent internal one; it quiets the palpitation of the heart, giving to the children sleep, and bringing back the Bacchantes to their right minds by the help of dances and acceptable sacrifices. But if fear has such power, will not a child who is always in a state of terror grow up timid and cowardly, whereas if he learns from the first to resist fear he will develop a habit of courage? 'Very true.' And we may say that the use of motion will inspire the souls of children with cheerfulness and therefore with courage. 'Of course.' Softness enervates and irritates the temper of the young, and violence renders them mean and misanthropical. 'But how is the state to educate them when they are as yet unable to understand the meaning of words?' Why, surely they roar and cry, like the young of any other animal, and the nurse knows the meaning of these intimations of the child's likes or dislikes, and the occasions which call them forth. About three years is passed by children in a state of imperfect articulation, which is quite long enough time to make them either good- or ill-tempered. And, therefore, during these first three years, the infant should be as free as possible from fear and pain. 'Yes, and he should have as much pleasure as possible.' There, I think, you are wrong; for the influence of pleasure in the beginning of education is fatal. A man should neither pursue pleasure nor wholly avoid pain. He should embrace the mean, and cultivate that state of calm which mankind, taught by some inspiration, attribute to God; and he who would be like God should neither be too fond of pleasure himself, nor should he permit any other to be thus given; above all, not the infant, whose character is just in the making. It may sound ridiculous, but I affirm that a woman in her pregnancy should be carefully tended, and kept from excessive pleasures and pains.

\par  'I quite agree with you about the duty of avoiding extremes and following the mean.'

\par  Let us consider a further point. The matters which are now in question are generally called customs rather than laws; and we have already made the reflection that, though they are not, properly speaking, laws, yet neither can they be neglected. For they fill up the interstices of law, and are the props and ligatures on which the strength of the whole building depends. Laws without customs never last; and we must not wonder if habit and custom sometimes lengthen out our laws. 'Very true.' Up to their third year, then, the life of children may be regulated by customs such as we have described. From three to six their minds have to be amused; but they must not be allowed to become self-willed and spoilt. If punishment is necessary, the same rule will hold as in the case of slaves; they must neither be punished in hot blood nor ruined by indulgence. The children of that age will have their own modes of amusing themselves; they should be brought for their play to the village temples, and placed under the care of nurses, who will be responsible to twelve matrons annually chosen by the women who have authority over marriage. These shall be appointed, one out of each tribe, and their duty shall be to keep order at the meetings: slaves who break the rules laid down by them, they shall punish by the help of some of the public slaves; but citizens who dispute their authority shall be brought before the magistrates. After six years of age there shall be a separation of the sexes; the boys will go to learn riding and the use of arms, and the girls may, if they please, also learn. Here I note a practical error in early training. Mothers and nurses foolishly believe that the left hand is by nature different from the right, whereas the left leg and foot are acknowledged to be the same as the right. But the truth is that nature made all things to balance, and the power of using the left hand, which is of little importance in the case of the plectrum of the lyre, may make a great difference in the art of the warrior, who should be a skilled gymnast and able to fight and balance himself in any position. If a man were a Briareus, he should use all his hundred hands at once; at any rate, let everybody employ the two which they have. To these matters the magistrates, male and female, should attend; the women superintending the nursing and amusement of the children, and the men superintending their education, that all of them, boys and girls alike, may be sound, wind and limb, and not spoil the gifts of nature by bad habits.

\par  Education has two branches—gymnastic, which is concerned with the body; and music, which improves the soul. And gymnastic has two parts, dancing and wrestling. Of dancing one kind imitates musical recitation and aims at stateliness and freedom; another kind is concerned with the training of the body, and produces health, agility, and beauty. There is no military use in the complex systems of wrestling which pass under the names of Antaeus and Cercyon, or in the tricks of boxing, which are attributed to Amycus and Epeius; but good wrestling and the habit of extricating the neck, hands, and sides, should be diligently learnt and taught. In our dances imitations of war should be practised, as in the dances of the Curetes in Crete and of the Dioscuri at Sparta, or as in the dances in complete armour which were taught us Athenians by the goddess Athene. Youths who are not yet of an age to go to war should make religious processions armed and on horseback; and they should also engage in military games and contests. These exercises will be equally useful in peace and war, and will benefit both states and families.

\par  Next follows music, to which we will once more return; and here I shall venture to repeat my old paradox, that amusements have great influence on laws. He who has been taught to play at the same games and with the same playthings will be content with the same laws. There is no greater evil in a state than the spirit of innovation. In the case of the seasons and winds, in the management of our bodies and in the habits of our minds, change is a dangerous thing. And in everything but what is bad the same rule holds. We all venerate and acquiesce in the laws to which we are accustomed; and if they have continued during long periods of time, and there is no remembrance of their ever having been otherwise, people are absolutely afraid to change them. Now how can we create this quality of immobility in the laws? I say, by not allowing innovations in the games and plays of children. The children who are always having new plays, when grown up will be always having new laws. Changes in mere fashions are not serious evils, but changes in our estimate of men's characters are most serious; and rhythms and music are representations of characters, and therefore we must avoid novelties in dance and song. For securing permanence no better method can be imagined than that of the Egyptians. 'What is their method?' They make a calendar for the year, arranging on what days the festivals of the various Gods shall be celebrated, and for each festival they consecrate an appropriate hymn and dance. In our state a similar arrangement shall in the first instance be framed by certain individuals, and afterwards solemnly ratified by all the citizens. He who introduces other hymns or dances shall be excluded by the priests and priestesses and the guardians of the law; and if he refuses to submit, he may be prosecuted for impiety. But we must not be too ready to speak about such great matters. Even a young man, when he hears something unaccustomed, stands and looks this way and that, like a traveller at a place where three ways meet; and at our age a man ought to be very sure of his ground in so singular an argument. 'Very true.' Then, leaving the subject for further examination at some future time, let us proceed with our laws about education, for in this manner we may probably throw light upon our present difficulty. 'Let us do as you say.' The ancients used the term nomoi to signify harmonious strains, and perhaps they fancied that there was a connexion between the songs and laws of a country. And we say—Whosoever shall transgress the strains by law established is a transgressor of the laws, and shall be punished by the guardians of the law and by the priests and priestesses. 'Very good.' How can we legislate about these consecrated strains without incurring ridicule? Moulds or types must be first framed, and one of the types shall be—Abstinence from evil words at sacrifices. When a son or brother blasphemes at a sacrifice there is a sound of ill-omen heard in the family; and many a chorus stands by the altar uttering inauspicious words, and he is crowned victor who excites the hearers most with lamentations. Such lamentations should be reserved for evil days, and should be uttered only by hired mourners; and let the singers not wear circlets or ornaments of gold. To avoid every evil word, then, shall be our first type. 'Agreed.' Our second law or type shall be, that prayers ever accompany sacrifices; and our third, that, inasmuch as all prayers are requests, they shall be only for good; this the poets must be made to understand. 'Certainly.' Have we not already decided that no gold or silver Plutus shall be allowed in our city? And did not this show that we were dissatisfied with the poets? And may we not fear that, if they are allowed to utter injudicious prayers, they will bring the greatest misfortunes on the state? And we must therefore make a law that the poet is not to contradict the laws or ideas of the state; nor is he to show his poems to any private persons until they have first received the imprimatur of the director of education. A fourth musical law will be to the effect that hymns and praises shall be offered to Gods, and to heroes and demigods. Still another law will permit eulogies of eminent citizens, whether men or women, but only after their death. As to songs and dances, we will enact as follows:—There shall be a selection made of the best ancient musical compositions and dances; these shall be chosen by judges, who ought not to be less than fifty years of age. They will accept some, and reject or amend others, for which purpose they will call, if necessary, the poets themselves into council. The severe and orderly music is the style in which to educate children, who, if they are accustomed to this, will deem the opposite kind to be illiberal, but if they are accustomed to the other, will count this to be cold and unpleasing. 'True.' Further, a distinction should be made between the melodies of men and women. Nature herself teaches that the grand or manly style should be assigned to men, and to women the moderate and temperate. So much for the subjects of education. But to whom are they to be taught, and when? I must try, like the shipwright, who lays down the keel of a vessel, to build a secure foundation for the vessel of the soul in her voyage through life. Human affairs are hardly serious, and yet a sad necessity compels us to be serious about them. Let us, therefore, do our best to bring the matter to a conclusion. 'Very good.' I say then, that God is the object of a man's most serious endeavours. But man is created to be the plaything of the Gods; and therefore the aim of every one should be to pass through life, not in grim earnest, but playing at the noblest of pastimes, in another spirit from that which now prevails. For the common opinion is, that work is for the sake of play, war of peace; whereas in war there is neither amusement nor instruction worth speaking of. The life of peace is that which men should chiefly desire to lengthen out and improve. They should live sacrificing, singing, and dancing, with the view of propitiating Gods and heroes. I have already told you the types of song and dance which they should follow: and 'Some things,' as the poet well says, 'you will devise for yourself—others, God will suggest to you.'

\par  These words of his may be applied to our pupils. They will partly teach themselves, and partly will be taught by God, the art of propitiating Him; for they are His puppets, and have only a small portion in truth. 'You have a poor opinion of man.' No wonder, when I compare him with God; but, if you are offended, I will place him a little higher.

\par  Next follow the building for gymnasia and schools; these will be in the midst of the city, and outside will be riding-schools and archery-grounds. In all of them there ought to be instructors of the young, drawn from foreign parts by pay, and they will teach them music and war. Education shall be compulsory; the children must attend school, whether their parents like it or not; for they belong to the state more than to their parents. And I say further, without hesitation, that the same education in riding and gymnastic shall be given both to men and women. The ancient tradition about the Amazons confirms my view, and at the present day there are myriads of women, called Sauromatides, dwelling near the Pontus, who practise the art of riding as well as archery and the use of arms. But if I am right, nothing can be more foolish than our modern fashion of training men and women differently, whereby the power the city is reduced to a half. For reflect—if women are not to have the education of men, some other must be found for them, and what other can we propose? Shall they, like the women of Thrace, tend cattle and till the ground; or, like our own, spin and weave, and take care of the house? or shall they follow the Spartan custom, which is between the two?—there the maidens share in gymnastic exercises and in music; and the grown women, no longer engaged in spinning, weave the web of life, although they are not skilled in archery, like the Amazons, nor can they imitate our warrior goddess and carry shield or spear, even in the extremity of their country's need. Compared with our women, the Sauromatides are like men. But your legislators, Megillus, as I maintain, only half did their work; they took care of the men, and left the women to take care of themselves.

\par  'Shall we suffer the Stranger, Cleinias, to run down Sparta in this way?'

\par  'Why, yes; for we cannot withdraw the liberty which we have already conceded to him.'

\par  What will be the manner of life of men in moderate circumstances, freed from the toils of agriculture and business, and having common tables for themselves and their families which are under the inspection of magistrates, male and female? Are men who have these institutions only to eat and fatten like beasts? If they do, how can they escape the fate of a fatted beast, which is to be torn in pieces by some other beast more valiant than himself? True, theirs is not the perfect way of life, for they have not all things in common; but the second best way of life also confers great blessings. Even those who live in the second state have a work to do twice as great as the work of any Pythian or Olympic victor; for their labour is for the body only, but ours both for body and soul. And this higher work ought to be pursued night and day to the exclusion of every other. The magistrates who keep the city should be wakeful, and the master of the household should be up early and before all his servants; and the mistress, too, should awaken her handmaidens, and not be awakened by them. Much sleep is not required either for our souls or bodies. When a man is asleep, he is no better than if he were dead; and he who loves life and wisdom will take no more sleep than is necessary for health. Magistrates who are wide awake at night are terrible to the bad; but they are honoured by the good, and are useful to themselves and the state.

\par  When the morning dawns, let the boy go to school. As the sheep need the shepherd, so the boy needs a master; for he is at once the most cunning and the most insubordinate of creatures. Let him be taken away from mothers and nurses, and tamed with bit and bridle, being treated as a freeman in that he learns and is taught, but as a slave in that he may be chastised by all other freemen; and the freeman who neglects to chastise him shall be disgraced. All these matters will be under the supervision of the Director of Education.

\par  Him we will address as follows: We have spoken to you, O illustrious teacher of youth, of the song, the time, and the dance, and of martial strains; but of the learning of letters and of prose writings, and of music, and of the use of calculation for military and domestic purposes we have not spoken, nor yet of the higher use of numbers in reckoning divine things—such as the revolutions of the stars, or the arrangements of days, months, and years, of which the true calculation is necessary in order that seasons and festivals may proceed in regular course, and arouse and enliven the city, rendering to the Gods their due, and making men know them better. There are, we say, many things about which we have not as yet instructed you—and first, as to reading and music: Shall the pupil be a perfect scholar and musician, or not even enter on these studies? He should certainly enter on both:—to letters he will apply himself from the age of ten to thirteen, and at thirteen he will begin to handle the lyre, and continue to learn music until he is sixteen; no shorter and no longer time will be allowed, however fond he or his parents may be of the pursuit. The study of letters he should carry to the extent of simple reading and writing, but he need not care for calligraphy and tachygraphy, if his natural gifts do not enable him to acquire them in the three years. And here arises a question as to the learning of compositions when unaccompanied with music, I mean, prose compositions. They are a dangerous species of literature. Speak then, O guardians of the law, and tell us what we shall do about them. 'You seem to be in a difficulty.' Yes; it is difficult to go against the opinion of all the world. 'But have we not often already done so?' Very true. And you imply that the road which we are taking, though disagreeable to many, is approved by those whose judgment is most worth having. 'Certainly.' Then I would first observe that we have many poets, comic as well as tragic, with whose compositions, as people say, youth are to be imbued and saturated. Some would have them learn by heart entire poets; others prefer extracts. Now I believe, and the general opinion is, that some of the things which they learn are good, and some bad. 'Then how shall we reject some and select others?' A happy thought occurs to me; this long discourse of ours is a sample of what we want, and is moreover an inspired work and a kind of poem. I am naturally pleased in reflecting upon all our words, which appear to me to be just the thing for a young man to hear and learn. I would venture, then, to offer to the Director of Education this treatise of laws as a pattern for his guidance; and in case he should find any similar compositions, written or oral, I would have him carefully preserve them, and commit them in the first place to the teachers who are willing to learn them (he should turn off the teacher who refuses), and let them communicate the lesson to the young.

\par  I have said enough to the teacher of letters; and now we will proceed to the teacher of the lyre. He must be reminded of the advice which we gave to the sexagenarian minstrels; like them he should be quick to perceive the rhythms suited to the expression of virtue, and to reject the opposite. With a view to the attainment of this object, the pupil and his instructor are to use the lyre because its notes are pure; the voice and string should coincide note for note: nor should there be complex harmonies and contrasts of intervals, or variations of times or rhythms. Three years' study is not long enough to give a knowledge of these intricacies; and our pupils will have many things of more importance to learn. The tunes and hymns which are to be consecrated for each festival have been already determined by us.

\par  Having given these instructions to the Director of Music, let us now proceed to dancing and gymnastic, which must also be taught to boys and girls by masters and mistresses. Our minister of education will have a great deal to do; and being an old man, how will he get through so much work? There is no difficulty;—the law will provide him with assistants, male and female; and he will consider how important his office is, and how great the responsibility of choosing them. For if education prospers, the vessel of state sails merrily along; or if education fails, the consequences are not even to be mentioned. Of dancing and gymnastics something has been said already. We include under the latter military exercises, the various uses of arms, all that relates to horsemanship, and military evolutions and tactics. There should be public teachers of both arts, paid by the state, and women as well as men should be trained in them. The maidens should learn the armed dance, and the grown-up women be practised in drill and the use of arms, if only in case of extremity, when the men are gone out to battle, and they are left to guard their families. Birds and beasts defend their young, but women instead of fighting run to the altars, thus degrading man below the level of the animals. 'Such a lack of education, Stranger, is both unseemly and dangerous.'

\par  Wrestling is to be pursued as a military exercise, but the meaning of this, and the nature of the art, can only be explained when action is combined with words. Next follows dancing, which is of two kinds; imitative, first, of the serious and beautiful; and, secondly, of the ludicrous and grotesque. The first kind may be further divided into the dance of war and the dance of peace. The former is called the Pyrrhic; in this the movements of attack and defence are imitated in a direct and manly style, which indicates strength and sufficiency of body and mind. The latter of the two, the dance of peace, is suitable to orderly and law-abiding men. These must be distinguished from the Bacchic dances which imitate drunken revelry, and also from the dances by which purifications are effected and mysteries celebrated. Such dances cannot be characterized either as warlike or peaceful, and are unsuited to a civilized state. Now the dances of peace are of two classes:—the first of them is the more violent, being an expression of joy and triumph after toil and danger; the other is more tranquil, symbolizing the continuance and preservation of good. In speaking or singing we naturally move our bodies, and as we have more or less courage or self-control we become less or more violent and excited. Thus from the imitation of words in gestures the art of dancing arises. Now one man imitates in an orderly, another in a disorderly manner: and so the peaceful kinds of dance have been appropriately called Emmeleiai, or dances of order, as the warlike have been called Pyrrhic. In the latter a man imitates all sorts of blows and the hurling of weapons and the avoiding of them; in the former he learns to bear himself gracefully and like a gentleman. The types of these dances are to be fixed by the legislator, and when the guardians of the law have assigned them to the several festivals, and consecrated them in due order, no further change shall be allowed.

\par  Thus much of the dances which are appropriate to fair forms and noble souls. Comedy, which is the opposite of them, remains to be considered. For the serious implies the ludicrous, and opposites cannot be understood without opposites. But a man of repute will desire to avoid doing what is ludicrous. He should leave such performances to slaves,—they are not fit for freemen; and there should be some element of novelty in them. Concerning tragedy, let our law be as follows: When the inspired poet comes to us with a request to be admitted into our state, we will reply in courteous words—We also are tragedians and your rivals; and the drama which we enact is the best and noblest, being the imitation of the truest and noblest life, with a view to which our state is ordered. And we cannot allow you to pitch your stage in the agora, and make your voices to be heard above ours, or suffer you to address our women and children and the common people on opposite principles to our own. Come then, ye children of the Lydian Muse, and present yourselves first to the magistrates, and if they decide that your hymns are as good or better than ours, you shall have your chorus; but if not, not.

\par  There remain three kinds of knowledge which should be learnt by freemen—arithmetic, geometry of surfaces and of solids, and thirdly, astronomy. Few need make an accurate study of such sciences; and of special students we will speak at another time. But most persons must be content with the study of them which is absolutely necessary, and may be said to be a necessity of that nature against which God himself is unable to contend. 'What are these divine necessities of knowledge?' Necessities of a knowledge without which neither gods, nor demigods, can govern mankind. And far is he from being a divine man who cannot distinguish one, two, odd and even; who cannot number day and night, and is ignorant of the revolutions of the sun and stars; for to every higher knowledge a knowledge of number is necessary—a fool may see this; how much, is a matter requiring more careful consideration. 'Very true.' But the legislator cannot enter into such details, and therefore we must defer the more careful consideration of these matters to another occasion. 'You seem to fear our habitual want of training in these subjects.' Still more do I fear the danger of bad training, which is often worse than none at all. 'Very true.' I think that a gentleman and a freeman may be expected to know as much as an Egyptian child. In Egypt, arithmetic is taught to children in their sports by a distribution of apples or garlands among a greater or less number of people; or a calculation is made of the various combinations which are possible among a set of boxers or wrestlers; or they distribute cups among the children, sometimes of gold, brass, and silver intermingled, sometimes of one metal only. The knowledge of arithmetic which is thus acquired is a great help, either to the general or to the manager of a household; wherever measure is employed, men are more wide-awake in their dealings, and they get rid of their ridiculous ignorance. 'What do you mean?' I have observed this ignorance among my countrymen—they are like pigs—and I am heartily ashamed both on my own behalf and on that of all the Hellenes. 'In what respect?' Let me ask you a question. You know that there are such things as length, breadth, and depth? 'Yes.' And the Hellenes imagine that they are commensurable (1) with themselves, and (2) with each other; whereas they are only commensurable with themselves. But if this is true, then we are in an unfortunate case, and may well say to our compatriots that not to possess necessary knowledge is a disgrace, though to possess such knowledge is nothing very grand. 'Certainly.' The discussion of arithmetical problems is a much better amusement for old men than their favourite game of draughts. 'True.' Mathematics, then, will be one of the subjects in which youth should be trained. They may be regarded as an amusement, as well as a useful and innocent branch of knowledge;—I think that we may include them provisionally. 'Yes; that will be the way.' The next question is, whether astronomy shall be made a part of education. About the stars there is a strange notion prevalent. Men often suppose that it is impious to enquire into the nature of God and the world, whereas the very reverse is the truth. 'How do you mean?' What I am going to say may seem absurd and at variance with the usual language of age, and yet if true and advantageous to the state, and pleasing to God, ought not to be withheld. 'Let us hear.' My dear friend, how falsely do we and all the Hellenes speak about the sun and moon! 'In what respect?' We are always saying that they and certain of the other stars do not keep the same path, and we term them planets. 'Yes; and I have seen the morning and evening stars go all manner of ways, and the sun and moon doing what we know that they always do. But I wish that you would explain your meaning further.' You will easily understand what I have had no difficulty in understanding myself, though we are both of us past the time of learning. 'True; but what is this marvellous knowledge which youth are to acquire, and of which we are ignorant?' Men say that the sun, moon, and stars are planets or wanderers; but this is the reverse of the fact. Each of them moves in one orbit only, which is circular, and not in many; nor is the swiftest of them the slowest, as appears to human eyes. What an insult should we offer to Olympian runners if we were to put the first last and the last first! And if that is a ridiculous error in speaking of men, how much more in speaking of the Gods? They cannot be pleased at our telling falsehoods about them. 'They cannot.' Then people should at least learn so much about them as will enable them to avoid impiety.

\par  Enough of education. Hunting and similar pursuits now claim our attention. These require for their regulation that mixture of law and admonition of which we have often spoken; e.g., in what we were saying about the nurture of young children. And therefore the whole duty of the citizen will not consist in mere obedience to the laws; he must regard not only the enactments but also the precepts of the legislator. I will illustrate my meaning by an example. Of hunting there are many kinds—hunting of fish and fowl, man and beast, enemies and friends; and the legislator can neither omit to speak about these things, nor make penal ordinances about them all. 'What is he to do then?' He will praise and blame hunting, having in view the discipline and exercise of youth. And the young man will listen obediently and will regard his praises and censures; neither pleasure nor pain should hinder him. The legislator will express himself in the form of a pious wish for the welfare of the young:—O my friends, he will say, may you never be induced to hunt for fish in the waters, either by day or night; or for men, whether by sea or land. Never let the wish to steal enter into your minds; neither be ye fowlers, which is not an occupation for gentlemen. As to land animals, the legislator will discourage hunting by night, and also the use of nets and snares by day; for these are indolent and unmanly methods. The only mode of hunting which he can praise is with horses and dogs, running, shooting, striking at close quarters. Enough of the prelude: the law shall be as follows:—

\par  Let no one hinder the holy order of huntsmen; but let the nightly hunters who lay snares and nets be everywhere prohibited. Let the fowler confine himself to waste places and to the mountains. The fisherman is also permitted to exercise his calling, except in harbours and sacred streams, marshes and lakes; in all other places he may fish, provided he does not make use of poisonous mixtures.

\par  BOOK VIII. Next, with the help of the Delphian Oracle, we will appoint festivals and sacrifices. There shall be 365 of them, one for every day in the year; and one magistrate, at least, shall offer sacrifice daily according to rites prescribed by a convocation of priests and interpreters, who shall co-operate with the guardians of the law, and supply what the legislator has omitted. Moreover there shall be twelve festivals to the twelve Gods after whom the twelve tribes are named: these shall be celebrated every month with appropriate musical and gymnastic contests. There shall also be festivals for women, to be distinguished from the men's festivals. Nor shall the Gods below be forgotten, but they must be separated from the Gods above—Pluto shall have his own in the twelfth month. He is not the enemy, but the friend of man, who releases the soul from the body, which is at least as good a work as to unite them. Further, those who have to regulate these matters should consider that our state has leisure and abundance, and wishing to be happy, like an individual, should lead a good life; for he who leads such a life neither does nor suffers injury, of which the first is very easy, and the second very difficult of attainment, and is only to be acquired by perfect virtue. A good city has peace, but the evil city is full of wars within and without. To guard against the danger of external enemies the citizens should practise war at least one day in every month; they should go out en masse, including their wives and children, or in divisions, as the magistrates determine, and have mimic contests, imitating in a lively manner real battles; they should also have prizes and encomiums of valour, both for the victors in these contests, and for the victors in the battle of life. The poet who celebrates the victors should be fifty years old at least, and himself a man who has done great deeds. Of such an one the poems may be sung, even though he is not the best of poets. To the director of education and the guardians of the law shall be committed the judgment, and no song, however sweet, which has not been licensed by them shall be recited. These regulations about poetry, and about military expeditions, apply equally to men and to women.

\par  The legislator may be conceived to make the following address to himself:—With what object am I training my citizens? Are they not strivers for mastery in the greatest of combats? Certainly, will be the reply. And if they were boxers or wrestlers, would they think of entering the lists without many days' practice? Would they not as far as possible imitate all the circumstances of the contest; and if they had no one to box with, would they not practise on a lifeless image, heedless of the laughter of the spectators? And shall our soldiers go out to fight for life and kindred and property unprepared, because sham fights are thought to be ridiculous? Will not the legislator require that his citizens shall practise war daily, performing lesser exercises without arms, while the combatants on a greater scale will carry arms, and take up positions, and lie in ambuscade? And let their combats be not without danger, that opportunity may be given for distinction, and the brave man and the coward may receive their meed of honour or disgrace. If occasionally a man is killed, there is no great harm done—there are others as good as he is who will replace him; and the state can better afford to lose a few of her citizens than to lose the only means of testing them.

\par  'We agree, Stranger, that such warlike exercises are necessary.' But why are they so rarely practised? Or rather, do we not all know the reasons? One of them (1) is the inordinate love of wealth. This absorbs the soul of a man, and leaves him no time for any other pursuit. Knowledge is valued by him only as it tends to the attainment of wealth. All is lost in the desire of heaping up gold and silver; anybody is ready to do anything, right or wrong, for the sake of eating and drinking, and the indulgence of his animal passions. 'Most true.' This is one of the causes which prevents a man being a good soldier, or anything else which is good; it converts the temperate and orderly into shopkeepers or servants, and the brave into burglars or pirates. Many of these latter are men of ability, and are greatly to be pitied, because their souls are hungering and thirsting all their lives long. The bad forms of government (2) are another reason—democracy, oligarchy, tyranny, which, as I was saying, are not states, but states of discord, in which the rulers are afraid of their subjects, and therefore do not like them to become rich, or noble, or valiant. Now our state will escape both these causes of evil; the society is perfectly free, and has plenty of leisure, and is not allowed by the laws to be absorbed in the pursuit of wealth; hence we have an excellent field for a perfect education, and for the introduction of martial pastimes. Let us proceed to describe the character of these pastimes. All gymnastic exercises in our state must have a military character; no other will be allowed. Activity and quickness are most useful in war; and yet these qualities do not attain their greatest efficiency unless the competitors are armed. The runner should enter the lists in armour, and in the races which our heralds proclaim, no prize is to be given except to armed warriors. Let there be six courses—first, the stadium; secondly, the diaulos or double course; thirdly, the horse course; fourthly, the long course; fifthly, races (1) between heavy-armed soldiers who shall pass over sixty stadia and finish at a temple of Ares, and (2) between still more heavily-armed competitors who run over smoother ground; sixthly, a race for archers, who shall run over hill and dale a distance of a hundred stadia, and their goal shall be a temple of Apollo and Artemis. There shall be three contests of each kind—one for boys, another for youths, a third for men; the course for the boys we will fix at half, and that for the youths at two-thirds of the entire length. Women shall join in the races: young girls who are not grown up shall run naked; but after thirteen they shall be suitably dressed; from thirteen to eighteen they shall be obliged to share in these contests, and from eighteen to twenty they may if they please and if they are unmarried. As to trials of strength, single combats in armour, or battles between two and two, or of any number up to ten, shall take the place of wrestling and the heavy exercises. And there must be umpires, as there are now in wrestling, to determine what is a fair hit and who is conqueror. Instead of the pancratium, let there be contests in which the combatants carry bows and wear light shields and hurl javelins and throw stones. The next provision of the law will relate to horses, which, as we are in Crete, need be rarely used by us, and chariots never; our horse-racing prizes will only be given to single horses, whether colts, half-grown, or full-grown. Their riders are to wear armour, and there shall be a competition between mounted archers. Women, if they have a mind, may join in the exercises of men.

\par  But enough of gymnastics, and nearly enough of music. All musical contests will take place at festivals, whether every third or every fifth year, which are to be fixed by the guardians of the law, the judges of the games, and the director of education, who for this purpose shall become legislators and arrange times and conditions. The principles on which such contests are to be ordered have been often repeated by the first legislator; no more need be said of them, nor are the details of them important. But there is another subject of the highest importance, which, if possible, should be determined by the laws, not of man, but of God; or, if a direct revelation is impossible, there is need of some bold man who, alone against the world, will speak plainly of the corruption of human nature, and go to war with the passions of mankind. 'We do not understand you.' I will try to make my meaning plainer. In speaking of education, I seemed to see young men and maidens in friendly intercourse with one another; and there arose in my mind a natural fear about a state, in which the young of either sex are well nurtured, and have little to do, and occupy themselves chiefly with festivals and dances. How can they be saved from those passions which reason forbids them to indulge, and which are the ruin of so many? The prohibition of wealth, and the influence of education, and the all-seeing eye of the ruler, will alike help to promote temperance; but they will not wholly extirpate the unnatural loves which have been the destruction of states; and against this evil what remedy can be devised? Lacedaemon and Crete give no assistance here; on the subject of love, as I may whisper in your ear, they are against us. Suppose a person were to urge that you ought to restore the natural use which existed before the days of Laius; he would be quite right, but he would not be supported by public opinion in either of your states. Or try the matter by the test which we apply to all laws,—who will say that the permission of such things tends to virtue? Will he who is seduced learn the habit of courage; or will the seducer acquire temperance? And will any legislator be found to make such actions legal?

\par  But to judge of this matter truly, we must understand the nature of love and friendship, which may take very different forms. For we speak of friendship, first, when there is some similarity or equality of virtue; secondly, when there is some want; and either of these, when in excess, is termed love. The first kind is gentle and sociable; the second is fierce and unmanageable; and there is also a third kind, which is akin to both, and is under the dominion of opposite principles. The one is of the body, and has no regard for the character of the beloved; but he who is under the influence of the other disregards the body, and is a looker rather than a lover, and desires only with his soul to be knit to the soul of his friend; while the intermediate sort is both of the body and of the soul. Here are three kinds of love: ought the legislator to prohibit all of them equally, or to allow the virtuous love to remain? 'The latter, clearly.' I expected to gain your approval; but I will reserve the task of convincing our friend Cleinias for another occasion. 'Very good.' To make right laws on this subject is in one point of view easy, and in another most difficult; for we know that in some cases most men abstain willingly from intercourse with the fair. The unwritten law which prohibits members of the same family from such intercourse is strictly obeyed, and no thought of anything else ever enters into the minds of men in general. A little word puts out the fire of their lusts. 'What is it?' The declaration that such things are hateful to the Gods, and most abominable and unholy. The reason is that everywhere, in jest and earnest alike, this is the doctrine which is repeated to all from their earliest youth. They see on the stage that an Oedipus or a Thyestes or a Macareus, when undeceived, are ready to kill themselves. There is an undoubted power in public opinion when no breath is heard adverse to the law; and the legislator who would enslave these enslaving passions must consecrate such a public opinion all through the city. 'Good: but how can you create it?' A fair objection; but I promised to try and find some means of restraining loves to their natural objects. A law which would extirpate unnatural love as effectually as incest is at present extirpated, would be the source of innumerable blessings, because it would be in accordance with nature, and would get rid of excess in eating and drinking and of adulteries and frenzies, making men love their wives, and having other excellent effects. I can imagine that some lusty youth overhears what we are saying, and roars out in abusive terms that we are legislating for impossibilities. And so a person might have said of the syssitia, or common meals; but this is refuted by facts, although even now they are not extended to women. 'True.' There is no impossibility or super-humanity in my proposed law, as I shall endeavour to prove. 'Do so.' Will not a man find abstinence more easy when his body is sound than when he is in ill-condition? 'Yes.' Have we not heard of Iccus of Tarentum and other wrestlers who abstained wholly for a time? Yet they were infinitely worse educated than our citizens, and far more lusty in their bodies. And shall they have abstained for the sake of an athletic contest, and our citizens be incapable of a similar endurance for the sake of a much nobler victory,—the victory over pleasure, which is true happiness? Will not the fear of impiety enable them to conquer that which many who were inferior to them have conquered? 'I dare say.' And therefore the law must plainly declare that our citizens should not fall below the other animals, who live all together in flocks, and yet remain pure and chaste until the time of procreation comes, when they pair, and are ever after faithful to their compact. But if the corruption of public opinion is too great to allow our first law to be carried out, then our guardians of the law must turn legislators, and try their hand at a second law. They must minimize the appetites, diverting the vigour of youth into other channels, allowing the practice of love in secret, but making detection shameful. Three higher principles may be brought to bear on all these corrupt natures. 'What are they?' Religion, honour, and the love of the higher qualities of the soul. Perhaps this is a dream only, yet it is the best of dreams; and if not the whole, still, by the grace of God, a part of what we desire may be realized. Either men may learn to abstain wholly from any loves, natural or unnatural, except of their wedded wives; or, at least, they may give up unnatural loves; or, if detected, they shall be punished with loss of citizenship, as aliens from the state in their morals. 'I entirely agree with you,' said Megillus, 'but Cleinias must speak for himself.' 'I will give my opinion by-and-by.'

\par  We were speaking of the syssitia, which will be a natural institution in a Cretan colony. Whether they shall be established after the model of Crete or Lacedaemon, or shall be different from either, is an unimportant question which may be determined without difficulty. We may, therefore, proceed to speak of the mode of life among our citizens, which will be far less complex than in other cities; a state which is inland and not maritime requires only half the number of laws. There is no trouble about trade and commerce, and a thousand other things. The legislator has only to regulate the affairs of husbandmen and shepherds, which will be easily arranged, now that the principal questions, such as marriage, education, and government, have been settled.

\par  Let us begin with husbandry: First, let there be a law of Zeus against removing a neighbour's landmark, whether he be a citizen or stranger. For this is 'to move the immoveable'; and Zeus, the God of kindred, witnesses to the wrongs of citizens, and Zeus, the God of strangers, to the wrongs of strangers. The offence of removing a boundary shall receive two punishments—the first will be inflicted by the God himself; the second by the judges. In the next place, the differences between neighbours about encroachments must be guarded against. He who encroaches shall pay twofold the amount of the injury; of all such matters the wardens of the country shall be the judges, in lesser cases the officers, and in greater the whole number of them belonging to any one division. Any injury done by cattle, the decoying of bees, the careless firing of woods, the planting unduly near a neighbour's ground, shall all be visited with proper damages. Such details have been determined by previous legislators, and need not now be mixed up with greater matters. Husbandmen have had of old excellent rules about streams and waters; and we need not 'divert their course.' Anybody may take water from a common stream, if he does not thereby cut off a private spring; he may lead the water in any direction, except through a house or temple, but he must do no harm beyond the channel. If land is without water the occupier shall dig down to the clay, and if at this depth he find no water, he shall have a right of getting water from his neighbours for his household; and if their supply is limited, he shall receive from them a measure of water fixed by the wardens of the country. If there be heavy rains, the dweller on the higher ground must not recklessly suffer the water to flow down upon a neighbour beneath him, nor must he who lives upon lower ground or dwells in an adjoining house refuse an outlet. If the two parties cannot agree, they shall go before the wardens of the city or country, and if a man refuse to abide by their decision, he shall pay double the damage which he has caused.

\par  In autumn God gives us two boons—one the joy of Dionysus not to be laid up—the other to be laid up. About the fruits of autumn let the law be as follows: He who gathers the storing fruits of autumn, whether grapes or figs, before the time of the vintage, which is the rising of Arcturus, shall pay fifty drachmas as a fine to Dionysus, if he gathers on his own ground; if on his neighbour's ground, a mina, and two-thirds of a mina if on that of any one else. The grapes or figs not used for storing a man may gather when he pleases on his own ground, but on that of others he must pay the penalty of removing what he has not laid down. If he be a slave who has gathered, he shall receive a stroke for every grape or fig. A metic must purchase the choice fruit; but a stranger may pluck for himself and his attendant. This right of hospitality, however, does not extend to storing grapes. A slave who eats of the storing grapes or figs shall be beaten, and the freeman be dismissed with a warning. Pears, apples, pomegranates, may be taken secretly, but he who is detected in the act of taking them shall be lightly beaten off, if he be not more than thirty years of age. The stranger and the elder may partake of them, but not carry any away; the latter, if he does not obey the law, shall fail in the competition of virtue, if anybody brings up his offence against him.

\par  Water is also in need of protection, being the greatest element of nutrition, and, unlike the other elements—soil, air, and sun—which conspire in the growth of plants, easily polluted. And therefore he who spoils another's water, whether in springs or reservoirs, either by trenching, or theft, or by means of poisonous substances, shall pay the damage and purify the stream. At the getting-in of the harvest everybody shall have a right of way over his neighbour's ground, provided he is careful to do no damage beyond the trespass, or if he himself will gain three times as much as his neighbour loses. Of all this the magistrates are to take cognizance, and they are to assess the damage where the injury does not exceed three minae; cases of greater damage can be tried only in the public courts. A charge against a magistrate is to be referred to the public courts, and any one who is found guilty of deciding corruptly shall pay twofold to the aggrieved person. Matters of detail relating to punishments and modes of procedure, and summonses, and witnesses to summonses, do not require the mature wisdom of the aged legislator; the younger generation may determine them according to their experience; but when once determined, they shall remain unaltered.

\par  The following are to be the regulations respecting handicrafts:—No citizen, or servant of a citizen, is to practise them. For the citizen has already an art and mystery, which is the care of the state; and no man can practise two arts, or practise one and superintend another. No smith should be a carpenter, and no carpenter, having many slaves who are smiths, should look after them himself; but let each man practise one art which shall be his means of livelihood. The wardens of the city should see to this, punishing the citizen who offends with temporary deprival of his rights—the foreigner shall be imprisoned, fined, exiled. Any disputes about contracts shall be determined by the wardens of the city up to fifty drachmae—above that sum by the public courts. No customs are to be exacted either on imports or exports. Nothing unnecessary is to be imported from abroad, whether for the service of the Gods or for the use of man—neither purple, nor other dyes, nor frankincense,—and nothing needed in the country is to be exported. These things are to be decided on by the twelve guardians of the law who are next in seniority to the five elders. Arms and the materials of war are to be imported and exported only with the consent of the generals, and then only by the state. There is to be no retail trade either in these or any other articles. For the distribution of the produce of the country, the Cretan laws afford a rule which may be usefully followed. All shall be required to distribute corn, grain, animals, and other valuable produce, into twelve portions. Each of these shall be subdivided into three parts—one for freemen, another for servants, and the third shall be sold for the supply of artisans, strangers, and metics. These portions must be equal whether the produce be much or little; and the master of a household may distribute the two portions among his family and his slaves as he pleases—the remainder is to be measured out to the animals.

\par  Next as to the houses in the country—there shall be twelve villages, one in the centre of each of the twelve portions; and in every village there shall be temples and an agora—also shrines for heroes or for any old Magnesian deities who linger about the place. In every division there shall be temples of Hestia, Zeus, and Athene, as well as of the local deity, surrounded by buildings on eminences, which will be the guard-houses of the rural police. The dwellings of the artisans will be thus arranged:—The artisans shall be formed into thirteen guilds, one of which will be divided into twelve parts and settled in the city; of the rest there shall be one in each division of the country. And the magistrates will fix them on the spots where they will cause the least inconvenience and be most serviceable in supplying the wants of the husbandmen.

\par  The care of the agora will fall to the wardens of the agora. Their first duty will be the regulation of the temples which surround the market-place; and their second to see that the markets are orderly and that fair dealing is observed. They will also take care that the sales which the citizens are required to make to strangers are duly executed. The law shall be, that on the first day of each month the auctioneers to whom the sale is entrusted shall offer grain; and at this sale a twelfth part of the whole shall be exposed, and the foreigner shall supply his wants for a month. On the tenth, there shall be a sale of liquids, and on the twenty-third of animals, skins, woven or woollen stuffs, and other things which husbandmen have to sell and foreigners want to buy. None of these commodities, any more than barley or flour, or any other food, may be retailed by a citizen to a citizen; but foreigners may sell them to one another in the foreigners' market. There must also be butchers who will sell parts of animals to foreigners and craftsmen, and their servants; and foreigners may buy firewood wholesale of the commissioners of woods, and may sell retail to foreigners. All other goods must be sold in the market, at some place indicated by the magistrates, and shall be paid for on the spot. He who gives credit, and is cheated, will have no redress. In buying or selling, any excess or diminution of what the law allows shall be registered. The same rule is to be observed about the property of metics. Anybody who practises a handicraft may come and remain twenty years from the day on which he is enrolled; at the expiration of this time he shall take what he has and depart. The only condition which is to be imposed upon him as the tax of his sojourn is good conduct; and he is not to pay any tax for being allowed to buy or sell. But if he wants to extend the time of his sojourn, and has done any service to the state, and he can persuade the council and assembly to grant his request, he may remain. The children of metics may also be metics; and the period of twenty years, during which they are permitted to sojourn, is to count, in their case, from their fifteenth year.

\par  No mention occurs in the Laws of the doctrine of Ideas. The will of God, the authority of the legislator, and the dignity of the soul, have taken their place in the mind of Plato. If we ask what is that truth or principle which, towards the end of his life, seems to have absorbed him most, like the idea of good in the Republic, or of beauty in the Symposium, or of the unity of virtue in the Protagoras, we should answer—The priority of the soul to the body: his later system mainly hangs upon this. In the Laws, as in the Sophist and Statesman, we pass out of the region of metaphysical or transcendental ideas into that of psychology.

\par  The opening of the fifth book, though abrupt and unconnected in style, is one of the most elevated passages in Plato. The religious feeling which he seeks to diffuse over the commonest actions of life, the blessedness of living in the truth, the great mistake of a man living for himself, the pity as well as anger which should be felt at evil, the kindness due to the suppliant and the stranger, have the temper of Christian philosophy. The remark that elder men, if they want to educate others, should begin by educating themselves; the necessity of creating a spirit of obedience in the citizens; the desirableness of limiting property; the importance of parochial districts, each to be placed under the protection of some God or demigod, have almost the tone of a modern writer. In many of his views of politics, Plato seems to us, like some politicians of our own time, to be half socialist, half conservative.

\par  In the Laws, we remark a change in the place assigned by him to pleasure and pain. There are two ways in which even the ideal systems of morals may regard them: either like the Stoics, and other ascetics, we may say that pleasure must be eradicated; or if this seems unreal to us, we may affirm that virtue is the true pleasure; and then, as Aristotle says, 'to be brought up to take pleasure in what we ought, exercises a great and paramount influence on human life' (Arist. Eth. Nic.). Or as Plato says in the Laws, 'A man will recognize the noblest life as having the greatest pleasure and the least pain, if he have a true taste.' If we admit that pleasures differ in kind, the opposition between these two modes of speaking is rather verbal than real; and in the greater part of the writings of Plato they alternate with each other. In the Republic, the mere suggestion that pleasure may be the chief good, is received by Socrates with a cry of abhorrence; but in the Philebus, innocent pleasures vindicate their right to a place in the scale of goods. In the Protagoras, speaking in the person of Socrates rather than in his own, Plato admits the calculation of pleasure to be the true basis of ethics, while in the Phaedo he indignantly denies that the exchange of one pleasure for another is the exchange of virtue. So wide of the mark are they who would attribute to Plato entire consistency in thoughts or words.

\par  He acknowledges that the second state is inferior to the first—in this, at any rate, he is consistent; and he still casts longing eyes upon the ideal. Several features of the first are retained in the second: the education of men and women is to be as far as possible the same; they are to have common meals, though separate, the men by themselves, the women with their children; and they are both to serve in the army; the citizens, if not actually communists, are in spirit communistic; they are to be lovers of equality; only a certain amount of wealth is permitted to them, and their burdens and also their privileges are to be proportioned to this. The constitution in the Laws is a timocracy of wealth, modified by an aristocracy of merit. Yet the political philosopher will observe that the first of these two principles is fixed and permanent, while the latter is uncertain and dependent on the opinion of the multitude. Wealth, after all, plays a great part in the Second Republic of Plato. Like other politicians, he deems that a property qualification will contribute stability to the state. The four classes are derived from the constitution of Athens, just as the form of the city, which is clustered around a citadel set on a hill, is suggested by the Acropolis at Athens. Plato, writing under Pythagorean influences, seems really to have supposed that the well-being of the city depended almost as much on the number 5040 as on justice and moderation. But he is not prevented by Pythagoreanism from observing the effects which climate and soil exercise on the characters of nations.

\par  He was doubtful in the Republic whether the ideal or communistic state could be realized, but was at the same time prepared to maintain that whether it existed or not made no difference to the philosopher, who will in any case regulate his life by it (Republic). He has now lost faith in the practicability of his scheme—he is speaking to 'men, and not to Gods or sons of Gods' (Laws). Yet he still maintains it to be the true pattern of the state, which we must approach as nearly as possible: as Aristotle says, 'After having created a more general form of state, he gradually brings it round to the other' (Pol.). He does not observe, either here or in the Republic, that in such a commonwealth there would be little room for the development of individual character. In several respects the second state is an improvement on the first, especially in being based more distinctly on the dignity of the soul. The standard of truth, justice, temperance, is as high as in the Republic;—in one respect higher, for temperance is now regarded, not as a virtue, but as the condition of all virtue. It is finally acknowledged that the virtues are all one and connected, and that if they are separated, courage is the lowest of them. The treatment of moral questions is less speculative but more human. The idea of good has disappeared; the excellences of individuals—of him who is faithful in a civil broil, of the examiner who is incorruptible, are the patterns to which the lives of the citizens are to conform. Plato is never weary of speaking of the honour of the soul, which can only be honoured truly by being improved. To make the soul as good as possible, and to prepare her for communion with the Gods in another world by communion with divine virtue in this, is the end of life. If the Republic is far superior to the Laws in form and style, and perhaps in reach of thought, the Laws leave on the mind of the modern reader much more strongly the impression of a struggle against evil, and an enthusiasm for human improvement. When Plato says that he must carry out that part of his ideal which is practicable, he does not appear to have reflected that part of an ideal cannot be detached from the whole.

\par  The great defect of both his constitutions is the fixedness which he seeks to impress upon them. He had seen the Athenian empire, almost within the limits of his own life, wax and wane, but he never seems to have asked himself what would happen if, a century from the time at which he was writing, the Greek character should have as much changed as in the century which had preceded. He fails to perceive that the greater part of the political life of a nation is not that which is given them by their legislators, but that which they give themselves. He has never reflected that without progress there cannot be order, and that mere order can only be preserved by an unnatural and despotic repression. The possibility of a great nation or of an universal empire arising never occurred to him. He sees the enfeebled and distracted state of the Hellenic world in his own later life, and thinks that the remedy is to make the laws unchangeable. The same want of insight is apparent in his judgments about art. He would like to have the forms of sculpture and of music fixed as in Egypt. He does not consider that this would be fatal to the true principles of art, which, as Socrates had himself taught, was to give life (Xen. Mem.). We wonder how, familiar as he was with the statues of Pheidias, he could have endured the lifeless and half-monstrous works of Egyptian sculpture. The 'chants of Isis' (Laws), we might think, would have been barbarous in an Athenian ear. But although he is aware that there are some things which are not so well among 'the children of the Nile,' he is deeply struck with the stability of Egyptian institutions. Both in politics and in art Plato seems to have seen no way of bringing order out of disorder, except by taking a step backwards. Antiquity, compared with the world in which he lived, had a sacredness and authority for him: the men of a former age were supposed by him to have had a sense of reverence which was wanting among his contemporaries. He could imagine the early stages of civilization; he never thought of what the future might bring forth. His experience is confined to two or three centuries, to a few Greek states, and to an uncertain report of Egypt and the East. There are many ways in which the limitations of their knowledge affected the genius of the Greeks. In criticism they were like children, having an acute vision of things which were near to them, blind to possibilities which were in the distance.

\par  The colony is to receive from the mother-country her original constitution, and some of the first guardians of the law. The guardians of the law are to be ministers of justice, and the president of education is to take precedence of them all. They are to keep the registers of property, to make regulations for trade, and they are to be superannuated at seventy years of age. Several questions of modern politics, such as the limitation of property, the enforcement of education, the relations of classes, are anticipated by Plato. He hopes that in his state will be found neither poverty nor riches; every man having the necessaries of life, he need not go fortune-hunting in marriage. Almost in the spirit of the Gospel he would say, 'How hardly can a rich man dwell in a perfect state.' For he cannot be a good man who is always gaining too much and spending too little (Laws; compare Arist. Eth. Nic.). Plato, though he admits wealth as a political element, would deny that material prosperity can be the foundation of a really great community. A man's soul, as he often says, is more to be esteemed than his body; and his body than external goods. He repeats the complaint which has been made in all ages, that the love of money is the corruption of states. He has a sympathy with thieves and burglars, 'many of whom are men of ability and greatly to be pitied, because their souls are hungering and thirsting all their lives long;' but he has little sympathy with shopkeepers or retailers, although he makes the reflection, which sometimes occurs to ourselves, that such occupations, if they were carried on honestly by the best men and women, would be delightful and honourable. For traders and artisans a moderate gain was, in his opinion, best. He has never, like modern writers, idealized the wealth of nations, any more than he has worked out the problems of political economy, which among the ancients had not yet grown into a science. The isolation of Greek states, their constant wars, the want of a free industrial population, and of the modern methods and instruments of 'credit,' prevented any great extension of commerce among them; and so hindered them from forming a theory of the laws which regulate the accumulation and distribution of wealth.

\par  The constitution of the army is aristocratic and also democratic; official appointment is combined with popular election. The two principles are carried out as follows: The guardians of the law nominate generals out of whom three are chosen by those who are or have been of the age for military service; and the generals elected have the nomination of certain of the inferior officers. But if either in the case of generals or of the inferior officers any one is ready to swear that he knows of a better man than those nominated, he may put the claims of his candidate to the vote of the whole army, or of the division of the service which he will, if elected, command. There is a general assembly, but its functions, except at elections, are hardly noticed. In the election of the Boule, Plato again attempts to mix aristocracy and democracy. This is effected, first as in the Servian constitution, by balancing wealth and numbers; for it cannot be supposed that those who possessed a higher qualification were equal in number with those who had a lower, and yet they have an equal number of representatives. In the second place, all classes are compelled to vote in the election of senators from the first and second class; but the fourth class is not compelled to elect from the third, nor the third and fourth from the fourth. Thirdly, out of the 180 persons who are thus chosen from each of the four classes, 720 in all, 360 are to be taken by lot; these form the council for the year.

\par  These political adjustments of Plato's will be criticised by the practical statesman as being for the most part fanciful and ineffectual. He will observe, first of all, that the only real check on democracy is the division into classes. The second of the three proposals, though ingenious, and receiving some light from the apathy to politics which is often shown by the higher classes in a democracy, would have little power in times of excitement and peril, when the precaution was most needed. At such political crises, all the lower classes would vote equally with the higher. The subtraction of half the persons chosen at the first election by the chances of the lot would not raise the character of the senators, and is open to the objection of uncertainty, which necessarily attends this and similar schemes of double representative government. Nor can the voters be expected to retain the continuous political interest required for carrying out such a proposal as Plato's. Who could select 180 persons of each class, fitted to be senators? And whoever were chosen by the voter in the first instance, his wishes might be neutralized by the action of the lot. Yet the scheme of Plato is not really so extravagant as the actual constitution of Athens, in which all the senators appear to have been elected by lot (apo kuamou bouleutai), at least, after the revolution made by Cleisthenes; for the constitution of the senate which was established by Solon probably had some aristocratic features, though their precise nature is unknown to us. The ancients knew that election by lot was the most democratic of all modes of appointment, seeming to say in the objectionable sense, that 'one man is as good as another.' Plato, who is desirous of mingling different elements, makes a partial use of the lot, which he applies to candidates already elected by vote. He attempts also to devise a system of checks and balances such as he supposes to have been intended by the ancient legislators. We are disposed to say to him, as he himself says in a remarkable passage, that 'no man ever legislates, but accidents of all sorts, which legislate for us in all sorts of ways. The violence of war and the hard necessity of poverty are constantly overturning governments and changing laws.' And yet, as he adds, the true legislator is still required: he must co-operate with circumstances. Many things which are ascribed to human foresight are the result of chance. Ancient, and in a less degree modern political constitutions, are never consistent with themselves, because they are never framed on a single design, but are added to from time to time as new elements arise and gain the preponderance in the state. We often attribute to the wisdom of our ancestors great political effects which have sprung unforeseen from the accident of the situation. Power, not wisdom, is most commonly the source of political revolutions. And the result, as in the Roman Republic, of the co-existence of opposite elements in the same state is, not a balance of power or an equable progress of liberal principles, but a conflict of forces, of which one or other may happen to be in the ascendant. In Greek history, as well as in Plato's conception of it, this 'progression by antagonism' involves reaction: the aristocracy expands into democracy and returns again to tyranny.

\par  The constitution of the Laws may be said to consist, besides the magistrates, mainly of three elements,—an administrative Council, the judiciary, and the Nocturnal Council, which is an intellectual aristocracy, composed of priests and the ten eldest guardians of the law and some younger co-opted members. To this latter chiefly are assigned the functions of legislation, but to be exercised with a sparing hand. The powers of the ordinary council are administrative rather than legislative. The whole number of 360, as in the Athenian constitution, is distributed among the months of the year according to the number of the tribes. Not more than one-twelfth is to be in office at once, so that the government would be made up of twelve administrations succeeding one another in the course of the year. They are to exercise a general superintendence, and, like the Athenian counsellors, are to preside in monthly divisions over all assemblies. Of the ecclesia over which they presided little is said, and that little relates to comparatively trifling duties. Nothing is less present to the mind of Plato than a House of Commons, carrying on year by year the work of legislation. For he supposes the laws to be already provided. As little would he approve of a body like the Roman Senate. The people and the aristocracy alike are to be represented, not by assemblies, but by officers elected for one or two years, except the guardians of the law, who are elected for twenty years.

\par  The evils of this system are obvious. If in any state, as Plato says in the Statesman, it is easier to find fifty good draught-players than fifty good rulers, the greater part of the 360 who compose the council must be unfitted to rule. The unfitness would be increased by the short period during which they held office. There would be no traditions of government among them, as in a Greek or Italian oligarchy, and no individual would be responsible for any of their acts. Everything seems to have been sacrificed to a false notion of equality, according to which all have a turn of ruling and being ruled. In the constitution of the Magnesian state Plato has not emancipated himself from the limitations of ancient politics. His government may be described as a democracy of magistrates elected by the people. He never troubles himself about the political consistency of his scheme. He does indeed say that the greater part of the good of this world arises, not from equality, but from proportion, which he calls the judgment of Zeus (compare Aristotle's Distributive Justice), but he hardly makes any attempt to carry out the principle in practice. There is no attempt to proportion representation to merit; nor is there any body in his commonwealth which represents the life either of a class or of the whole state. The manner of appointing magistrates is taken chiefly from the old democratic constitution of Athens, of which it retains some of the worst features, such as the use of the lot, while by doing away with the political character of the popular assembly the mainspring of the machine is taken out. The guardians of the law, thirty-seven in number, of whom the ten eldest reappear as a part of the Nocturnal Council at the end of the twelfth book, are to be elected by the whole military class, but they are to hold office for twenty years, and would therefore have an oligarchical rather than a democratic character. Nothing is said of the manner in which the functions of the Nocturnal Council are to be harmonized with those of the guardians of the law, or as to how the ordinary council is related to it.

\par  Similar principles are applied to inferior offices. To some the appointment is made by vote, to others by lot. In the elections to the priesthood, Plato endeavours to mix or balance in a friendly manner 'demus and not demus.' The commonwealth of the Laws, like the Republic, cannot dispense with a spiritual head, which is the same in both—the oracle of Delphi. From this the laws about all divine things are to be derived. The final selection of the Interpreters, the choice of an heir for a vacant lot, the punishment for removing a deposit, are also to be determined by it. Plato is not disposed to encourage amateur attempts to revive religion in states. For, as he says in the Laws, 'To institute religious rites is the work of a great intelligence.'

\par  Though the council is framed on the model of the Athenian Boule, the law courts of Plato do not equally conform to the pattern of the Athenian dicasteries. Plato thinks that the judges should speak and ask questions:—this is not possible if they are numerous; he would, therefore, have a few judges only, but good ones. He is nevertheless aware that both in public and private suits there must be a popular element. He insists that the whole people must share in the administration of justice—in public causes they are to take the first step, and the final decision is to remain with them. In private suits they are also to retain a share; 'for the citizen who has no part in the administration of justice is apt to think that he has no share in the state. For this reason there is to be a court of law in every tribe (i.e. for about every 2,000 citizens), and the judges are to be chosen by lot.' Of the courts of law he gives what he calls a superficial sketch. Nor, indeed is it easy to reconcile his various accounts of them. It is however clear that although some officials, like the guardians of the law, the wardens of the agora, city, and country have power to inflict minor penalties, the administration of justice is in the main popular. The ingenious expedient of dividing the questions of law and fact between a judge and jury, which would have enabled Plato to combine the popular element with the judicial, did not occur to him or to any other ancient political philosopher. Though desirous of limiting the number of judges, and thereby confining the office to persons specially fitted for it, he does not seem to have understood that a body of law must be formed by decisions as well as by legal enactments.

\par  He would have men in the first place seek justice from their friends and neighbours, because, as he truly remarks, they know best the questions at issue; these are called in another passage arbiters rather than judges. But if they cannot settle the matter, it is to be referred to the courts of the tribes, and a higher penalty is to be paid by the party who is unsuccessful in the suit. There is a further appeal allowed to the select judges, with a further increase of penalty. The select judges are to be appointed by the magistrates, who are to choose one from every magistracy. They are to be elected annually, and therefore probably for a year only, and are liable to be called to account before the guardians of the law. In cases of which death is the penalty, the trial takes place before a special court, which is composed of the guardians of the law and of the judges of appeal.

\par  In treating of the subject in Book ix, he proposes to leave for the most part the methods of procedure to a younger generation of legislators; the procedure in capital causes he determines himself. He insists that the vote of the judges shall be given openly, and before they vote they are to hear speeches from the plaintiff and defendant. They are then to take evidence in support of what has been said, and to examine witnesses. The eldest judge is to ask his questions first, and then the second, and then the third. The interrogatories are to continue for three days, and the evidence is to be written down. Apparently he does not expect the judges to be professional lawyers, any more than he expects the members of the council to be trained statesmen.

\par  In forming marriage connexions, Plato supposes that the public interest will prevail over private inclination. There was nothing in this very shocking to the notions of Greeks, among whom the feeling of love towards the other sex was almost deprived of sentiment or romance. Married life is to be regulated solely with a view to the good of the state. The newly-married couple are not allowed to absent themselves from their respective syssitia, even during their honeymoon; they are to give their whole mind to the procreation of children; their duties to one another at a later period of life are not a matter about which the state is equally solicitous. Divorces are readily allowed for incompatibility of temper. As in the Republic, physical considerations seem almost to exclude moral and social ones. To modern feelings there is a degree of coarseness in Plato's treatment of the subject. Yet he also makes some shrewd remarks on marriage, as for example, that a man who does not marry for money will not be the humble servant of his wife. And he shows a true conception of the nature of the family, when he requires that the newly-married couple 'should leave their father and mother,' and have a separate home. He also provides against extravagance in marriage festivals, which in some states of society, for instance in the case of the Hindoos, has been a social evil of the first magnitude.

\par  In treating of property, Plato takes occasion to speak of property in slaves. They are to be treated with perfect justice; but, for their own sake, to be kept at a distance. The motive is not so much humanity to the slave, of which there are hardly any traces (although Plato allows that many in the hour of peril have found a slave more attached than members of their own family), but the self-respect which the freeman and citizen owes to himself (compare Republic). If they commit crimes, they are doubly punished; if they inform against illegal practices of their masters, they are to receive a protection, which would probably be ineffectual, from the guardians of the law; in rare cases they are to be set free. Plato still breathes the spirit of the old Hellenic world, in which slavery was a necessity, because leisure must be provided for the citizen.

\par  The education propounded in the Laws differs in several points from that of the Republic. Plato seems to have reflected as deeply and earnestly on the importance of infancy as Rousseau, or Jean Paul (compare the saying of the latter—'Not the moment of death, but the moment of birth, is probably the more important'). He would fix the amusements of children in the hope of fixing their characters in after-life. In the spirit of the statesman who said, 'Let me make the ballads of a country, and I care not who make their laws,' Plato would say, 'Let the amusements of children be unchanged, and they will not want to change the laws. The 'Goddess Harmonia' plays a great part in Plato's ideas of education. The natural restless force of life in children, 'who do nothing but roar until they are three years old,' is gradually to be reduced to law and order. As in the Republic, he fixes certain forms in which songs are to be composed: (1) they are to be strains of cheerfulness and good omen; (2) they are to be hymns or prayers addressed to the Gods; (3) they are to sing only of the lawful and good. The poets are again expelled or rather ironically invited to depart; and those who remain are required to submit their poems to the censorship of the magistrates. Youth are no longer compelled to commit to memory many thousand lyric and tragic Greek verses; yet, perhaps, a worse fate is in store for them. Plato has no belief in 'liberty of prophesying'; and having guarded against the dangers of lyric poetry, he remembers that there is an equal danger in other writings. He cannot leave his old enemies, the Sophists, in possession of the field; and therefore he proposes that youth shall learn by heart, instead of the compositions of poets or prose writers, his own inspired work on laws. These, and music and mathematics, are the chief parts of his education.

\par  Mathematics are to be cultivated, not as in the Republic with a view to the science of the idea of good,—though the higher use of them is not altogether excluded,—but rather with a religious and political aim. They are a sacred study which teaches men how to distribute the portions of a state, and which is to be pursued in order that they may learn not to blaspheme about astronomy. Against three mathematical errors Plato is in profound earnest. First, the error of supposing that the three dimensions of length, breadth, and height, are really commensurable with one another. The difficulty which he feels is analogous to the difficulty which he formerly felt about the connexion of ideas, and is equally characteristic of ancient philosophy: he fixes his mind on the point of difference, and cannot at the same time take in the similarity. Secondly, he is puzzled about the nature of fractions: in the Republic, he is disposed to deny the possibility of their existence. Thirdly, his optimism leads him to insist (unlike the Spanish king who thought that he could have improved on the mechanism of the heavens) on the perfect or circular movement of the heavenly bodies. He appears to mean, that instead of regarding the stars as overtaking or being overtaken by one another, or as planets wandering in many paths, a more comprehensive survey of the heavens would enable us to infer that they all alike moved in a circle around a centre (compare Timaeus; Republic). He probably suspected, though unacquainted with the true cause, that the appearance of the heavens did not agree with the reality: at any rate, his notions of what was right or fitting easily overpowered the results of actual observation. To the early astronomers, who lived at the revival of science, as to Plato, there was nothing absurd in a priori astronomy, and they would probably have made fewer real discoveries of they had followed any other track. (Compare Introduction to the Republic.)

\par  The science of dialectic is nowhere mentioned by name in the Laws, nor is anything said of the education of after-life. The child is to begin to learn at ten years of age: he is to be taught reading and writing for three years, from ten to thirteen, and no longer; and for three years more, from thirteen to sixteen, he is to be instructed in music. The great fault which Plato finds in the contemporary education is the almost total ignorance of arithmetic and astronomy, in which the Greeks would do well to take a lesson from the Egyptians (compare Republic). Dancing and wrestling are to have a military character, and women as well as men are to be taught the use of arms. The military spirit which Plato has vainly endeavoured to expel in the first two books returns again in the seventh and eighth. He has evidently a sympathy with the soldier, as well as with the poet, and he is no mean master of the art, or at least of the theory, of war (compare Laws; Republic), though inclining rather to the Spartan than to the Athenian practice of it (Laws). Of a supreme or master science which was to be the 'coping-stone' of the rest, few traces appear in the Laws. He seems to have lost faith in it, or perhaps to have realized that the time for such a science had not yet come, and that he was unable to fill up the outline which he had sketched. There is no requirement that the guardians of the law shall be philosophers, although they are to know the unity of virtue, and the connexion of the sciences. Nor are we told that the leisure of the citizens, when they are grown up, is to be devoted to any intellectual employment. In this respect we note a falling off from the Republic, but also there is 'the returning to it' of which Aristotle speaks in the Politics. The public and family duties of the citizens are to be their main business, and these would, no doubt, take up a great deal more time than in the modern world we are willing to allow to either of them. Plato no longer entertains the idea of any regular training to be pursued under the superintendence of the state from eighteen to thirty, or from thirty to thirty-five; he has taken the first step downwards on 'Constitution Hill' (Republic). But he maintains as earnestly as ever that 'to men living under this second polity there remains the greatest of all works, the education of the soul,' and that no bye-work should be allowed to interfere with it. Night and day are not long enough for the consummation of it.

\par  Few among us are either able or willing to carry education into later life; five or six years spent at school, three or four at a university, or in the preparation for a profession, an occasional attendance at a lecture to which we are invited by friends when we have an hour to spare from house-keeping or money-making—these comprise, as a matter of fact, the education even of the educated; and then the lamp is extinguished 'more truly than Heracleitus' sun, never to be lighted again' (Republic). The description which Plato gives in the Republic of the state of adult education among his contemporaries may be applied almost word for word to our own age. He does not however acquiesce in this widely-spread want of a higher education; he would rather seek to make every man something of a philosopher before he enters on the duties of active life. But in the Laws he no longer prescribes any regular course of study which is to be pursued in mature years. Nor does he remark that the education of after-life is of another kind, and must consist with the majority of the world rather in the improvement of character than in the acquirement of knowledge. It comes from the study of ourselves and other men: from moderation and experience: from reflection on circumstances: from the pursuit of high aims: from a right use of the opportunities of life. It is the preservation of what we have been, and the addition of something more. The power of abstract study or continuous thought is very rare, but such a training as this can be given by every one to himself.

\par  The singular passage in Book vii., in which Plato describes life as a pastime, like many other passages in the Laws is imperfectly expressed. Two thoughts seem to be struggling in his mind: first, the reflection, to which he returns at the end of the passage, that men are playthings or puppets, and that God only is the serious aim of human endeavours; this suggests to him the afterthought that, although playthings, they are the playthings of the Gods, and that this is the best of them. The cynical, ironical fancy of the moment insensibly passes into a religious sentiment. In another passage he says that life is a game of which God, who is the player, shifts the pieces so as to procure the victory of good on the whole. Or once more: Tragedies are acted on the stage; but the best and noblest of them is the imitation of the noblest life, which we affirm to be the life of our whole state. Again, life is a chorus, as well as a sort of mystery, in which we have the Gods for playmates. Men imagine that war is their serious pursuit, and they make war that they may return to their amusements. But neither wars nor amusements are the true satisfaction of men, which is to be found only in the society of the Gods, in sacrificing to them and propitiating them. Like a Christian ascetic, Plato seems to suppose that life should be passed wholly in the enjoyment of divine things. And after meditating in amazement on the sadness and unreality of the world, he adds, in a sort of parenthesis, 'Be cheerful, Sirs' (Shakespeare, Tempest.)

\par  In one of the noblest passages of Plato, he speaks of the relation of the sexes. Natural relations between members of the same family have been established of old; a 'little word' has put a stop to incestuous connexions. But unnatural unions of another kind continued to prevail at Crete and Lacedaemon, and were even justified by the example of the Gods. They, too, might be banished, if the feeling that they were unholy and abominable could sink into the minds of men. The legislator is to cry aloud, and spare not, 'Let not men fall below the level of the beasts.' Plato does not shrink, like some modern philosophers, from 'carrying on war against the mightiest lusts of mankind;' neither does he expect to extirpate them, but only to confine them to their natural use and purpose, by the enactments of law, and by the influence of public opinion. He will not feed them by an over-luxurious diet, nor allow the healthier instincts of the soul to be corrupted by music and poetry. The prohibition of excessive wealth is, as he says, a very considerable gain in the way of temperance, nor does he allow of those enthusiastic friendships between older and younger persons which in his earlier writings appear to be alluded to with a certain degree of amusement and without reproof (compare Introduction to the Symposium). Sappho and Anacreon are celebrated by him in the Charmides and the Phaedrus; but they would have been expelled from the Magnesian state.

\par  Yet he does not suppose that the rule of absolute purity can be enforced on all mankind. Something must be conceded to the weakness of human nature. He therefore adopts a 'second legal standard of honourable and dishonourable, having a second standard of right.' He would abolish altogether 'the connexion of men with men...As to women, if any man has to do with any but those who come into his house duly married by sacred rites, and he offends publicly in the face of all mankind, we shall be right in enacting that he be deprived of civic honours and privileges.' But feeling also that it is impossible wholly to control the mightiest passions of mankind,' Plato, like other legislators, makes a compromise. The offender must not be found out; decency, if not morality, must be respected. In this he appears to agree with the practice of all civilized ages and countries. Much may be truly said by the moralist on the comparative harm of open and concealed vice. Nor do we deny that some moral evils are better turned out to the light, because, like diseases, when exposed, they are more easily cured. And secrecy introduces mystery which enormously exaggerates their power; a mere animal want is thus elevated into a sentimental ideal. It may very well be that a word spoken in season about things which are commonly concealed may have an excellent effect. But having regard to the education of youth, to the innocence of children, to the sensibilities of women, to the decencies of society, Plato and the world in general are not wrong in insisting that some of the worst vices, if they must exist, should be kept out of sight; this, though only a second-best rule, is a support to the weakness of human nature. There are some things which may be whispered in the closet, but should not be shouted on the housetop. It may be said of this, as of many other things, that it is a great part of education to know to whom they are to be spoken of, and when, and where.

\par  BOOK IX. Punishments of offences and modes of procedure come next in order. We have a sense of disgrace in making regulations for all the details of crime in a virtuous and well-ordered state. But seeing that we are legislating for men and not for Gods, there is no uncharitableness in apprehending that some one of our citizens may have a heart, like the seed which has touched the ox's horn, so hard as to be impenetrable to the law. Let our first enactment be directed against the robbing of temples. No well-educated citizen will be guilty of such a crime, but one of their servants, or some stranger, may, and with a view to him, and at the same time with a remoter eye to the general infirmity of human nature, I will lay down the law, beginning with a prelude. To the intending robber we will say—O sir, the complaint which troubles you is not human; but some curse has fallen upon you, inherited from the crimes of your ancestors, of which you must purge yourself: go and sacrifice to the Gods, associate with the good, avoid the wicked; and if you are cured of the fatal impulse, well; but if not, acknowledge death to be better than life, and depart.

\par  These are the accents, soft and low, in which we address the would-be criminal. And if he will not listen, then cry aloud as with the sound of a trumpet: Whosoever robs a temple, if he be a slave or foreigner shall be branded in the face and hands, and scourged, and cast naked beyond the border. And perhaps this may improve him: for the law aims either at the reformation of the criminal, or the repression of crime. No punishment is designed to inflict useless injury. But if the offender be a citizen, he must be incurable, and for him death is the only fitting penalty. His iniquity, however, shall not be visited on his children, nor shall his property be confiscated.

\par  As to the exaction of penalties, any person who is fined for an offence shall not be liable to pay the fine, unless he have property in excess of his lot. For the lots must never go uncultivated for lack of means; the guardians of the law are to provide against this. If a fine is inflicted upon a man which he cannot pay, and for which his friends are unwilling to give security, he shall be imprisoned and otherwise dishonoured. But no criminal shall go unpunished:—whether death, or imprisonment, or stripes, or fines, or the stocks, or banishment to a remote temple, be the penalty. Capital offences shall come under the cognizance of the guardians of the law, and a college of the best of the last year's magistrates. The order of suits and similar details we shall leave to the lawgivers of the future, and only determine the mode of voting. The judges are to sit in order of seniority, and the proceedings shall begin with the speeches of the plaintiff and the defendant; and then the judges, beginning with the eldest, shall ask questions and collect evidence during three days, which, at the end of each day, shall be deposited in writing under their seals on the altar of Hestia; and when they have evidence enough, after a solemn declaration that they will decide justly, they shall vote and end the case. The votes are to be given openly in the presence of the citizens.

\par  Next to religion, the preservation of the constitution is the first object of the law. The greatest enemy of the state is he who attempts to set up a tyrant, or breeds plots and conspiracies; not far below him in guilt is a magistrate who either knowingly, or in ignorance, fails to bring the offender to justice. Any one who is good for anything will give information against traitors. The mode of proceeding at such trials will be the same as at trials for sacrilege; the penalty, death. But neither in this case nor in any other is the son to bear the iniquity of the father, unless father, grandfather, great-grandfather, have all of them been capitally convicted, and then the family of the criminal are to be sent off to the country of their ancestor, retaining their property, with the exception of the lot and its fixtures. And ten are to be selected from the younger sons of the other citizens—one of whom is to be chosen by the oracle of Delphi to be heir of the lot.

\par  Our third law will be a general one, concerning the procedure and the judges in cases of treason. As regards the remaining or departure of the family of the offender, the same law shall apply equally to the traitor, the sacrilegious, and the conspirator.

\par  A thief, whether he steals much or little, must refund twice the amount, if he can do so without impairing his lot; if he cannot, he must go to prison until he either pays the plaintiff, or in case of a public theft, the city, or they agree to forgive him. 'But should all kinds of theft incur the same penalty?' You remind me of what I know—that legislation is never perfect. The men for whom laws are now made may be compared to the slave who is being doctored, according to our old image, by the unscientific doctor. For the empirical practitioner, if he chance to meet the educated physician talking to his patient, and entering into the philosophy of his disease, would burst out laughing and say, as doctors delight in doing, 'Foolish fellow, instead of curing the patient you are educating him!' 'And would he not be right?' Perhaps; and he might add, that he who discourses in our fashion preaches to the citizens instead of legislating for them. 'True.' There is, however, one advantage which we possess—that being amateurs only, we may either take the most ideal, or the most necessary and utilitarian view. 'But why offer such an alternative? As if all our legislation must be done to-day, and nothing put off until the morrow. We may surely rough-hew our materials first, and shape and place them afterwards.' That will be the natural way of proceeding. There is a further point. Of all writings either in prose or verse the writings of the legislator are the most important. For it is he who has to determine the nature of good and evil, and how they should be studied with a view to our instruction. And is it not as disgraceful for Solon and Lycurgus to lay down false precepts about the institutions of life as for Homer and Tyrtaeus? The laws of states ought to be the models of writing, and what is at variance with them should be deemed ridiculous. And we may further imagine them to express the affection and good sense of a father or mother, and not to be the fiats of a tyrant. 'Very true.'

\par  Let us enquire more particularly about sacrilege, theft and other crimes, for which we have already legislated in part. And this leads us to ask, first of all, whether we are agreed or disagreed about the nature of the honourable and just. 'To what are you referring?' I will endeavour to explain. All are agreed that justice is honourable, whether in men or things, and no one who maintains that a very ugly men who is just, is in his mind fair, would be thought extravagant. 'Very true.' But if honour is to be attributed to justice, are just sufferings honourable, or only just actions? 'What do you mean?' Our laws supply a case in point; for we enacted that the robber of temples and the traitor should die; and this was just, but the reverse of honourable. In this way does the language of the many rend asunder the just and honourable. 'That is true.' But is our own language consistent? I have already said that the evil are involuntarily evil; and the evil are the unjust. Now the voluntary cannot be the involuntary; and if you two come to me and say, 'Then shall we legislate for our city?' Of course, I shall reply.—'Then will you distinguish what crimes are voluntary and what involuntary, and shall we impose lighter penalties on the latter, and heavier on the former? Or shall we refuse to determine what is the meaning of voluntary and involuntary, and maintain that our words have come down from heaven, and that they should be at once embodied in a law?' All states legislate under the idea that there are two classes of actions, the voluntary and the involuntary, but there is great confusion about them in the minds of men; and the law can never act unless they are distinguished. Either we must abstain from affirming that unjust actions are involuntary, or explain the meaning of this statement. Believing, then, that acts of injustice cannot be divided into voluntary and involuntary, I must endeavour to find some other mode of classifying them. Hurts are voluntary and involuntary, but all hurts are not injuries: on the other hand, a benefit when wrongly conferred may be an injury. An act which gives or takes away anything is not simply just; but the legislator who has to decide whether the case is one of hurt or injury, must consider the animus of the agent; and when there is hurt, he must as far as possible, provide a remedy and reparation: but if there is injustice, he must, when compensation has been made, further endeavour to reconcile the two parties. 'Excellent.' Where injustice, like disease, is remediable, there the remedy must be applied in word or deed, with the assistance of pleasures and pains, of bounties and penalties, or any other influence which may inspire man with the love of justice, or hatred of injustice; and this is the noblest work of law. But when the legislator perceives the evil to be incurable, he will consider that the death of the offender will be a good to himself, and in two ways a good to society: first, as he becomes an example to others; secondly, because the city will be quit of a rogue; and in such a case, but in no other, the legislator will punish with death. 'There is some truth in what you say. I wish, however, that you would distinguish more clearly the difference of injury and hurt, and the complications of voluntary and involuntary.' You will admit that anger is of a violent and destructive nature? 'Certainly.' And further, that pleasure is different from anger, and has an opposite power, working by persuasion and deceit? 'Yes.' Ignorance is the third source of crimes; this is of two kinds—simple ignorance and ignorance doubled by conceit of knowledge; the latter, when accompanied with power, is a source of terrible errors, but is excusable when only weak and childish. 'True.' We often say that one man masters, and another is mastered by pleasure and anger. 'Just so.' But no one says that one man masters, and another is mastered by ignorance. 'You are right.' All these motives actuate men and sometimes drive them in different ways. 'That is so.' Now, then, I am in a position to define the nature of just and unjust. By injustice I mean the dominion of anger and fear, pleasure and pain, envy and desire, in the soul, whether doing harm or not: by justice I mean the rule of the opinion of the best, whether in states or individuals, extending to the whole of life; although actions done in error are often thought to be involuntary injustice. No controversy need be raised about names at present; we are only desirous of fixing in our memories the heads of error. And the pain which is called fear and anger is our first head of error; the second is the class of pleasures and desires; and the third, of hopes which aim at true opinion about the best;—this latter falls into three divisions (i.e. (1) when accompanied by simple ignorance, (2) when accompanied by conceit of wisdom combined with power, or (3) with weakness), so that there are in all five. And the laws relating to them may be summed up under two heads, laws which deal with acts of open violence and with acts of deceit; to which may be added acts both violent and deceitful, and these last should be visited with the utmost rigour of the law. 'Very properly.'

\par  Let us now return to the enactment of laws. We have treated of sacrilege, and of conspiracy, and of treason. Any of these crimes may be committed by a person not in his right mind, or in the second childhood of old age. If this is proved to be the fact before the judges, the person in question shall only have to pay for the injury, and not be punished further, unless he have on his hands the stain of blood. In this case he shall be exiled for a year, and if he return before the expiration of the year, he shall be retained in the public prison two years.

\par  Homicides may be divided into voluntary and involuntary: and first of involuntary homicide. He who unintentionally kills another man at the games or in military exercises duly authorized by the magistrates, whether death follow immediately or after an interval, shall be acquitted, subject only to the purification required by the Delphian Oracle. Any physician whose patient dies against his will shall in like manner be acquitted. Any one who unintentionally kills the slave of another, believing that he is his own, with or without weapons, shall bear the master of the slave harmless, or pay a penalty amounting to twice the value of the slave, and to this let him add a purification greater than in the case of homicide at the games. If a man kill his own slave, a purification only is required of him. If he kill a freeman unintentionally, let him also make purification; and let him remember the ancient tradition which says that the murdered man is indignant when he sees the murderer walk about in his own accustomed haunts, and that he terrifies him with the remembrance of his crime. And therefore the homicide should keep away from his native land for a year, or, if he have slain a stranger, let him avoid the land of the stranger for a like period. If he complies with this condition, the nearest kinsman of the deceased shall take pity upon him and be reconciled to him; but if he refuses to remain in exile, or visits the temples unpurified, then let the kinsman proceed against him, and demand a double penalty. The kinsman who neglects this duty shall himself incur the curse, and any one who likes may proceed against him, and compel him to leave his country for five years. If a stranger involuntarily kill a stranger, any one may proceed against him in the same manner: and the homicide, if he be a metic, shall be banished for a year; but if he be an entire stranger, whether he have murdered metic, citizen, or stranger, he shall be banished for ever; and if he return, he shall be punished with death, and his property shall go to the next of kin of the murdered man. If he come back by sea against his will, he shall remain on the seashore, wetting his feet in the water while he waits for a vessel to sail; or if he be brought back by land, the magistrates shall send him unharmed beyond the border.

\par  Next follows murder done from anger, which is of two kinds—either arising out of a sudden impulse, and attended with remorse; or committed with premeditation, and unattended with remorse. The cause of both is anger, and both are intermediate between voluntary and involuntary. The one which is committed from sudden impulse, though not wholly involuntary, bears the image of the involuntary, and is therefore the more excusable of the two, and should receive a gentler punishment. The act of him who nurses his wrath is more voluntary, and therefore more culpable. The degree of culpability depends on the presence or absence of intention, to which the degree of punishment should correspond. For the first kind of murder, that which is done on a momentary impulse, let two years' exile be the penalty; for the second, that which is accompanied with malice prepense, three. When the time of any one's exile has expired, the guardians shall send twelve judges to the borders of the land, who shall have authority to decide whether he may return or not. He who after returning repeats the offence, shall be exiled and return no more, and, if he return, shall be put to death, like the stranger in a similar case. He who in a fit of anger kills his own slave, shall purify himself; and he who kills another man's slave, shall pay to his master double the value. Any one may proceed against the offender if he appear in public places, not having been purified; and may bring to trial both the next of kin to the dead man and the homicide, and compel the one to exact, and the other to pay, a double penalty. If a slave kill his master, or a freeman who is not his master, in anger, the kinsmen of the murdered person may do with the murderer whatever they please, but they must not spare his life. If a father or mother kill their son or daughter in anger, let the slayer remain in exile for three years; and on the return of the exile let the parents separate, and no longer continue to cohabit, or have the same sacred rites with those whom he or she has deprived of a brother or sister. The same penalty is decreed against the husband who murders his wife, and also against the wife who murders her husband. Let them be absent three years, and on their return never again share in the same sacred rites with their children, or sit at the same table with them. Nor is a brother or sister who have lifted up their hands against a brother or sister, ever to come under the same roof or share in the same rites with those whom they have robbed of a child. If a son feels such hatred against his father or mother as to take the life of either of them, then, if the parent before death forgive him, he shall only suffer the penalty due to involuntary homicide; but if he be unforgiven, there are many laws against which he has offended; he is guilty of outrage, impiety, sacrilege all in one, and deserves to be put to death many times over. For if the law will not allow a man to kill the authors of his being even in self-defence, what other penalty than death can be inflicted upon him who in a fit of passion wilfully slays his father or mother? If a brother kill a brother in self-defence during a civil broil, or a citizen a citizen, or a slave a slave, or a stranger a stranger, let them be free from blame, as he is who slays an enemy in battle. But if a slave kill a freeman, let him be as a parricide. In all cases, however, the forgiveness of the injured party shall acquit the agents; and then they shall only be purified, and remain in exile for a year.

\par  Enough of actions that are involuntary, or done in anger; let us proceed to voluntary and premeditated actions. The great source of voluntary crime is the desire of money, which is begotten by evil education; and this arises out of the false praise of riches, common both among Hellenes and barbarians; they think that to be the first of goods which is really the third. For the body is not for the sake of wealth, but wealth for the body, as the body is for the soul. If this were better understood, the crime of murder, of which avarice is the chief cause, would soon cease among men. Next to avarice, ambition is a source of crime, troublesome to the ambitious man himself, as well as to the chief men of the state. And next to ambition, base fear is a motive, which has led many an one to commit murder in order that he may get rid of the witnesses of his crimes. Let this be said as a prelude to all enactments about crimes of violence; and the tradition must not be forgotten, which tells that the murderer is punished in the world below, and that when he returns to this world he meets the fate which he has dealt out to others. If a man is deterred by the prelude and the fear of future punishment, he will have no need of the law; but in case he disobey, let the law be declared against him as follows:—He who of malice prepense kills one of his kindred, shall in the first place be outlawed; neither temple, harbour, nor agora shall be polluted by his presence. And if a kinsman of the deceased refuse to proceed against his slayer, he shall take the curse of pollution upon himself, and also be liable to be prosecuted by any one who will avenge the dead. The prosecutor, however, must observe the customary ceremonial before he proceeds against the offender. The details of these observances will be best determined by a conclave of prophets and interpreters and guardians of the law, and the judges of the cause itself shall be the same as in cases of sacrilege. He who is convicted shall be punished with death, and not be buried within the country of the murdered person. He who flies from the law shall undergo perpetual banishment; if he return, he may be put to death with impunity by any relative of the murdered man or by any other citizen, or bound and delivered to the magistrates. He who accuses a man of murder shall demand satisfactory bail of the accused, and if this is not forthcoming, the magistrate shall keep him in prison against the day of trial. If a man commit murder by the hand of another, he shall be tried in the same way as in the cases previously supposed, but if the offender be a citizen, his body after execution shall be buried within the land.

\par  If a slave kill a freeman, either with his own hand or by contrivance, let him be led either to the grave or to a place whence he can see the grave of the murdered man, and there receive as many stripes at the hand of the public executioner as the person who took him pleases; and if he survive he shall be put to death. If a slave be put out of the way to prevent his informing of some crime, his death shall be punished like that of a citizen. If there are any of those horrible murders of kindred which sometimes occur even in well-regulated societies, and of which the legislator, however unwilling, cannot avoid taking cognizance, he will repeat the old myth of the divine vengeance against the perpetrators of such atrocities. The myth will say that the murderer must suffer what he has done: if he have slain his father, he must be slain by his children; if his mother, he must become a woman and perish at the hands of his offspring in another age of the world. Such a preamble may terrify him; but if, notwithstanding, in some evil hour he murders father or mother or brethren or children, the mode of proceeding shall be as follows:—Him who is convicted, the officers of the judges shall lead to a spot without the city where three ways meet, and there slay him and expose his body naked; and each of the magistrates shall cast a stone upon his head and justify the city, and he shall be thrown unburied beyond the border. But what shall we say of him who takes the life which is dearest to him, that is to say, his own; and this not from any disgrace or calamity, but from cowardice and indolence? The manner of his burial and the purification of his crime is a matter for God and the interpreters to decide and for his kinsmen to execute. Let him, at any rate, be buried alone in some uncultivated and nameless spot, and be without name or monument. If a beast kill a man, not in a public contest, let it be prosecuted for murder, and after condemnation slain and cast without the border. Also inanimate things which have caused death, except in the case of lightning and other visitations from heaven, shall be carried without the border. If the body of a dead man be found, and the murderer remain unknown, the trial shall take place all the same, and the unknown murderer shall be warned not to set foot in the temples or come within the borders of the land; if discovered, he shall die, and his body shall be cast out. A man is justified in taking the life of a burglar, of a footpad, of a violator of women or youth; and he may take the life of another with impunity in defence of father, mother, brother, wife, or other relations.

\par  The nurture and education which are necessary to the existence of men have been considered, and the punishment of acts of violence which destroy life. There remain maiming, wounding, and the like, which admit of a similar division into voluntary and involuntary. About this class of actions the preamble shall be: Whereas men would be like wild beasts unless they obeyed the laws, the first duty of citizens is the care of the public interests, which unite and preserve states, as private interests distract them. A man may know what is for the public good, but if he have absolute power, human nature will impel him to seek pleasure instead of virtue, and so darkness will come over his soul and over the state. If he had mind, he would have no need of law; for mind is the perfection of law. But such a freeman, 'whom the truth makes free,' is hardly to be found; and therefore law and order are necessary, which are the second-best, and they regulate things as they exist in part only, but cannot take in the whole. For actions have innumerable characteristics, which must be partly determined by the law and partly left to the judge. The judge must determine the fact; and to him also the punishment must sometimes be left. What shall the law prescribe, and what shall be left to the judge? A city is unfortunate in which the tribunals are either secret and speechless, or, what is worse, noisy and public, when the people, as if they were in a theatre, clap and hoot the various speakers. Such courts a legislator would rather not have; but if he is compelled to have them, he will speak distinctly, and leave as little as possible to their discretion. But where the courts are good, and presided over by well-trained judges, the penalties to be inflicted may be in a great measure left to them; and as there are to be good courts among our colonists, we need not determine beforehand the exact proportion of the penalty and the crime. Returning, then, to our legislator, let us indite a law about wounding, which shall run as follows:—He who wounds with intent to kill, and fails in his object, shall be tried as if he had succeeded. But since God has favoured both him and his victim, instead of being put to death, he shall be allowed to go into exile and take his property with him, the damage due to the sufferer having been previously estimated by the court, which shall be the same as would have tried the case if death had ensued. If a child should intentionally wound a parent, or a servant his master, or brother or sister wound brother or sister with malice prepense, the penalty shall be death. If a husband or wife wound one another with intent to kill, the penalty which is inflicted upon them shall be perpetual exile; and if they have young children, the guardians shall take care of them and administer their property as if they were orphans. If they have no children, their kinsmen male and female shall meet, and after a consultation with the priests and guardians of the law, shall appoint an heir of the house; for the house and family belong to the state, being a 5040th portion of the whole. And the state is bound to preserve her families happy and holy; therefore, when the heir of a house has committed a capital offence, or is in exile for life, the house is to be purified, and then the kinsmen of the house and the guardians of the law are to find out a family which has a good name and in which there are many sons, and introduce one of them to be the heir and priest of the house. He shall assume the fathers and ancestors of the family, while the first son dies in dishonour and his name is blotted out.

\par  Some actions are intermediate between the voluntary and involuntary. Those done from anger are of this class. If a man wound another in anger, let him pay double the damage, if the injury is curable; or fourfold, if curable, and at the same time dishonourable; and fourfold, if incurable; the amount is to be assessed by the judges. If the wounded person is rendered incapable of military service, the injurer, besides the other penalties, shall serve in his stead, or be liable to a suit for refusing to serve. If brother wounds brother, then their parents and kindred, of both sexes, shall meet and judge the crime. The damages shall be assessed by the parents; and if the amount fixed by them is disputed, an appeal shall be made to the male kindred; or in the last resort to the guardians of the law. Parents who wound their children are to be tried by judges of at least sixty years of age, who have children of their own; and they are to determine whether death, or some lesser punishment, is to be inflicted upon them—no relatives are to take part in the trial. If a slave in anger smite a freeman, he is to be delivered up by his master to the injured person. If the master suspect collusion between the slave and the injured person, he may bring the matter to trial: and if he fail he shall pay three times the injury; or if he obtain a conviction, the contriver of the conspiracy shall be liable to an action for kidnapping. He who wounds another unintentionally shall only pay for the actual harm done.

\par  In all outrages and acts of violence, the elder is to be more regarded than the younger. An injury done by a younger man to an elder is abominable and hateful; but the younger man who is struck by an elder is to bear with him patiently, considering that he who is twenty years older is loco parentis, and remembering the reverence which is due to the Gods who preside over birth. Let him keep his hands, too, from the stranger; instead of taking upon himself to chastise him when he is insolent, he shall bring him before the wardens of the city, who shall examine into the case, and if they find him guilty, shall scourge him with as many blows as he has given; or if he be innocent, they shall warn and threaten his accuser. When an equal strikes an equal, whether an old man an old man, or a young man a young man, let them use only their fists and have no weapons. He who being above forty years of age commences a fight, or retaliates, shall be counted mean and base.

\par  To this preamble, let the law be added: If a man smite another who is his elder by twenty years or more, let the bystander, in case he be older than the combatants, part them; or if he be younger than the person struck, or of the same age with him, let him defend him as he would a father or brother; and let the striker be brought to trial, and if convicted imprisoned for a year or more at the discretion of the judges. If a stranger smite one who is his elder by twenty years or more, he shall be imprisoned for two years, and a metic, in like case, shall suffer three years' imprisonment. He who is standing by and gives no assistance, shall be punished according to his class in one of four penalties—a mina, fifty, thirty, twenty drachmas. The generals and other superior officers of the army shall form the court which tries this class of offences.

\par  Laws are made to instruct the good, and in the hope that there may be no need of them; also to control the bad, whose hardness of heart will not be hindered from crime. The uttermost penalty will fall upon those who lay violent hands upon a parent, having no fear of the Gods above, or of the punishments which will pursue them in the world below. They are too wise in their own conceits to believe in such things: wherefore the tortures which await them in another life must be anticipated in this. Let the law be as follows:—

\par  If a man, being in his right mind, dare to smite his father and mother, or his grandfather and grandmother, let the passer-by come to the rescue; and if he be a metic or stranger who comes to the rescue, he shall have the first place at the games; or if he do not come to the rescue, he shall be a perpetual exile. Let the citizen in the like case be praised or blamed, and the slave receive freedom or a hundred stripes. The wardens of the agora, the city, or the country, as the case may be, shall see to the execution of the law. And he who is an inhabitant of the same place and is present shall come to the rescue, or he shall fall under a curse.

\par  If a man be convicted of assaulting his parents, let him be banished for ever from the city into the country, and let him abstain from all sacred rites; and if he do not abstain, let him be punished by the wardens of the country; and if he return to the city, let him be put to death. If any freeman consort with him, let him be purified before he returns to the city. If a slave strike a freeman, whether citizen or stranger, let the bystander be obliged to seize and deliver him into the hands of the injured person, who may inflict upon him as many blows as he pleases, and shall then return him to his master. The law will be as follows:—The slave who strikes a freeman shall be bound by his master, and not set at liberty without the consent of the person whom he has injured. All these laws apply to women as well as to men.

\par  BOOK X. The greatest wrongs arise out of youthful insolence, and the greatest of all are committed against public temples; they are in the second degree great when private rites and sepulchres are insulted; in the third degree, when committed against parents; in the fourth degree, when they are done against the authority or property of the rulers; in the fifth degree, when the rights of individuals are violated. Most of these offences have been already considered; but there remains the question of admonition and punishment of offences against the Gods. Let the admonition be in the following terms:—No man who ever intentionally did or said anything impious, had a true belief in the existence of the Gods; but either he thought that there were no Gods, or that they did not care about men, or that they were easily appeased by sacrifices and prayers. 'What shall we say or do to such persons?' My good sir, let us first hear the jests which they in their superiority will make upon us. 'What will they say?' Probably something of this kind:—'Strangers you are right in thinking that some of us do not believe in the existence of the Gods; while others assert that they do not care for us, and others that they are propitiated by prayers and offerings. But we want you to argue with us before you threaten; you should prove to us by reasonable evidence that there are Gods, and that they are too good to be bribed. Poets, priests, prophets, rhetoricians, even the best of them, speak to us of atoning for evil, and not of avoiding it. From legislators who profess to be gentle we ask for instruction, which may, at least, have the persuasive power of truth, if no other.' What have you to say? 'Well, there is no difficulty in proving the being of the Gods. The sun, and earth, and stars, moving in their courses, the recurring seasons, furnish proofs of their existence; and there is the general opinion of mankind.' I fear that the unbelievers—not that I care for their opinion—will despise us. You are not aware that their impiety proceeds, not from sensuality, but from ignorance taking the garb of wisdom. 'What do you mean?' At Athens there are tales current both in prose and verse of a kind which are not tolerated in a well-regulated state like yours. The oldest of them relate the origin of the world, and the birth and life of the Gods. These narratives have a bad influence on family relations; but as they are old we will let them pass, and consider another kind of tales, invented by the wisdom of a younger generation, who, if any one argues for the existence of the Gods and claims that the stars have a divine being, insist that these are mere earth and stones, which can have no care of human things, and that all theology is a cooking up of words. Now what course ought we to take? Shall we suppose some impious man to charge us with assuming the existence of the Gods, and make a defence? Or shall we leave the preamble and go on to the laws? 'There is no hurry, and we have often said that the shorter and worse method should not be preferred to the longer and better. The proof that there are Gods who are good, and the friends of justice, is the best preamble of all our laws.' Come, let us talk with the impious, who have been brought up from their infancy in the belief of religion, and have heard their own fathers and mothers praying for them and talking with the Gods as if they were absolutely convinced of their existence; who have seen mankind prostrate in prayer at the rising and setting of the sun and moon and at every turn of fortune, and have dared to despise and disbelieve all this. Can we keep our temper with them, when they compel us to argue on such a theme? We must; or like them we shall go mad, though with more reason. Let us select one of them and address him as follows:

\par  O my son, you are young; time and experience will make you change many of your opinions. Do not be hasty in forming a conclusion about the divine nature; and let me mention to you a fact which I know. You and your friends are not the first or the only persons who have had these notions about the Gods. There are always a considerable number who are infected by them: I have known many myself, and can assure you that no one who was an unbeliever in his youth ever persisted till he was old in denying the existence of the Gods. The two other opinions, first, that the Gods exist and have no care of men, secondly, that they care for men, but may be propitiated by sacrifices and prayers, may indeed last through life in a few instances, but even this is not common. I would beg of you to be patient, and learn the truth of the legislator and others; in the mean time abstain from impiety. 'So far, our discourse has gone well.'

\par  I will now speak of a strange doctrine, which is regarded by many as the crown of philosophy. They affirm that all things come into being either by art or nature or chance, and that the greater things are done by nature and chance, and the lesser things by art, which receiving from nature the greater creations, moulds and fashions all those lesser works which are termed works of art. Their meaning is that fire, water, earth, and air all exist by nature and chance, and not by art; and that out of these, according to certain chance affinities of opposites, the sun, the moon, the stars, and the earth have been framed, not by any action of mind, but by nature and chance only. Thus, in their opinion, the heaven and earth were created, as well as the animals and plants. Art came later, and is of mortal birth; by her power were invented certain images and very partial imitations of the truth, of which kind are the creations of musicians and painters: but they say that there are other arts which combine with nature, and have a deeper truth, such as medicine, husbandry, gymnastic. Also the greater part of politics they imagine to co-operate with nature, but in a less degree, having more of art, while legislation is declared by them to be wholly a work of art. 'How do you mean?' In the first place, they say that the Gods exist neither by nature nor by art, but by the laws of states, which are different in different countries; and that virtue is one thing by nature and another by convention; and that justice is altogether conventional, made by law, and having authority for the moment only. This is repeated to young men by sages and poets, and leads to impiety, and the pretended life according to nature and in disobedience to law; for nobody believes the Gods to be such as the law affirms. 'How true! and oh! how injurious to states and to families!' But then, what should the lawgiver do? Should he stand up in the state and threaten mankind with the severest penalties if they persist in their unbelief, while he makes no attempt to win them by persuasion? 'Nay, Stranger, the legislator ought never to weary of trying to persuade the world that there are Gods; and he should declare that law and art exist by nature.' Yes, Cleinias; but these are difficult and tedious questions. 'And shall our patience, which was not exhausted in the enquiry about music or drink, fail now that we are discoursing about the Gods? There may be a difficulty in framing laws, but when written down they remain, and time and diligence will make them clear; if they are useful there would be neither reason nor religion in rejecting them on account of their length.' Most true. And the general spread of unbelief shows that the legislator should do something in vindication of the laws, when they are being undermined by bad men. 'He should.' You agree with me, Cleinias, that the heresy consists in supposing earth, air, fire, and water to be the first of all things. These the heretics call nature, conceiving them to be prior to the soul. 'I agree.' You would further agree that natural philosophy is the source of this impiety—the study appears to be pursued in a wrong way. 'In what way do you mean?' The error consists in transposing first and second causes. They do not see that the soul is before the body, and before all other things, and the author and ruler of them all. And if the soul is prior to the body, then the things of the soul are prior to the things of the body. In other words, opinion, attention, mind, art, law, are prior to sensible qualities; and the first and greater works of creation are the results of art and mind, whereas the works of nature, as they are improperly termed, are secondary and subsequent. 'Why do you say "improperly"?' Because when they speak of nature they seem to mean the first creative power. But if the soul is first, and not fire and air, then the soul above all things may be said to exist by nature. And this can only be on the supposition that the soul is prior to the body. Shall we try to prove that it is so? 'By all means.' I fear that the greenness of our argument will ludicrously contrast with the ripeness of our ages. But as we must go into the water, and the stream is strong, I will first attempt to cross by myself, and if I arrive at the bank, you shall follow. Remembering that you are unaccustomed to such discussions, I will ask and answer the questions myself, while you listen in safety. But first I must pray the Gods to assist at the demonstration of their own existence—if ever we are to call upon them, now is the time. Let me hold fast to the rope, and enter into the depths: Shall I put the question to myself in this form?—Are all things at rest, and is nothing in motion? or are some things in motion, and some things at rest? 'The latter.' And do they move and rest, some in one place, some in more? 'Yes.' There may be (1) motion in the same place, as in revolution on an axis, which is imparted swiftly to the larger and slowly to the lesser circle; and there may be motion in different places, having sometimes (2) one centre of motion and sometimes (3) more. (4) When bodies in motion come against other bodies which are at rest, they are divided by them, and (5) when they are caught between other bodies coming from opposite directions they unite with them; and (6) they grow by union and (7) waste by dissolution while their constitution remains the same, but are (8) destroyed when their constitution fails. There is a growth from one dimension to two, and from a second to a third, which then becomes perceptible to sense; this process is called generation, and the opposite, destruction. We have now enumerated all possible motions with the exception of two. 'What are they?' Just the two with which our enquiry is concerned; for our enquiry relates to the soul. There is one kind of motion which is only able to move other things; there is another which can move itself as well, working in composition and decomposition, by increase and diminution, by generation and destruction. 'Granted.' (9) That which moves and is moved by another is the ninth kind of motion; (10) that which is self-moved and moves others is the tenth. And this tenth kind of motion is the mightiest, and is really the first, and is followed by that which was improperly called the ninth. 'How do you mean?' Must not that which is moved by others finally depend upon that which is moved by itself? Nothing can be affected by any transition prior to self-motion. Then the first and eldest principle of motion, whether in things at rest or not at rest, will be the principle of self-motion; and that which is moved by others and can move others will be the second. 'True.' Let me ask another question:

\par  What is the name which is given to self-motion when manifested in any material substance? 'Life.' And soul too is life? 'Very good.' And are there not three kinds of knowledge—a knowledge (1) of the essence, (2) of the definition, (3) of the name? And sometimes the name leads us to ask the definition, sometimes the definition to ask the name. For example, number can be divided into equal parts, and when thus divided is termed even, and the definition of even and the word 'even' refer to the same thing. 'Very true.' And what is the definition of the thing which is named 'soul'? Must we not reply, 'The self-moved'? And have we not proved that the self-moved is the source of motion in other things? 'Yes.' And the motion which is not self-moved will be inferior to this? 'True.' And if so, we shall be right in saying that the soul is prior and superior to the body, and the body by nature subject and inferior to the soul? 'Quite right.' And we agreed that if the soul was prior to the body, the things of the soul were prior to the things of the body? 'Certainly.' And therefore desires, and manners, and thoughts, and true opinions, and recollections, are prior to the length and breadth and force of bodies. 'To be sure.' In the next place, we acknowledge that the soul is the cause of good and evil, just and unjust, if we suppose her to be the cause of all things? 'Certainly.' And the soul which orders all things must also order the heavens? 'Of course.' One soul or more? More; for less than two are inconceivable, one good, the other evil. 'Most true.' The soul directs all things by her movements, which we call will, consideration, attention, deliberation, opinion true and false, joy, sorrow, courage, fear, hatred, love, and similar affections. These are the primary movements, and they receive the secondary movements of bodies, and guide all things to increase and diminution, separation and union, and to all the qualities which accompany them—cold, hot, heavy, light, hard, soft, white, black, sweet, bitter; these and other such qualities the soul, herself a goddess, uses, when truly receiving the divine mind she leads all things rightly to their happiness; but under the impulse of folly she works out an opposite result. For the controller of heaven and earth and the circle of the world is either the wise and good soul, or the foolish and vicious soul, working in them. 'What do you mean?' If we say that the whole course and motion of heaven and earth is in accordance with the workings and reasonings of mind, clearly the best soul must have the care of the heaven, and guide it along that better way. 'True.' But if the heavens move wildly and disorderly, then they must be under the guidance of the evil soul. 'True again.' What is the nature of the movement of the soul? We must not suppose that we can see and know the soul with our bodily eyes, any more than we can fix them on the midday sun; it will be safer to look at an image only. 'How do you mean?' Let us find among the ten kinds of motion an image of the motion of the mind. You remember, as we said, that all things are divided into two classes; and some of them were moved and some at rest. 'Yes.' And of those which were moved, some were moved in the same place, others in more places than one. 'Just so.' The motion which was in one place was circular, like the motion of a spherical body; and such a motion in the same place, and in the same relations, is an excellent image of the motion of mind. 'Very true.' The motion of the other sort, which has no fixed place or manner or relation or order or proportion, is akin to folly and nonsense. 'Very true.' After what has been said, it is clear that, since the soul carries round all things, some soul which is either very good or the opposite carries round the circumference of heaven. But that soul can be no other than the best. Again, the soul carries round the sun, moon, and stars, and if the sun has a soul, then either the soul of the sun is within and moves the sun as the human soul moves the body; or, secondly, the sun is contained in some external air or fire, which the soul provides and through which she operates; or, thirdly, the course of the sun is guided by the soul acting in a wonderful manner without a body. 'Yes, in one of those ways the soul must guide all things.' And this soul of the sun, which is better than the sun, whether driving him in a chariot or employing any other agency, is by every man called a God? 'Yes, by every man who has any sense.' And of the seasons, stars, moon, and year, in like manner, it may be affirmed that the soul or souls from which they derive their excellence are divine; and without insisting on the manner of their working, no one can deny that all things are full of Gods. 'No one.' And now let us offer an alternative to him who denies that there are Gods. Either he must show that the soul is not the origin of all things, or he must live for the future in the belief that there are Gods.

\par  Next, as to the man who believes in the Gods, but refuses to acknowledge that they take care of human things—let him too have a word of admonition. 'Best of men,' we will say to him, 'some affinity to the Gods leads you to honour them and to believe in them. But you have heard the happiness of wicked men sung by poets and admired by the world, and this has drawn you away from your natural piety. Or you have seen the wicked growing old in prosperity, and leaving great offices to their children; or you have watched the tyrant succeeding in his career of crime; and considering all these things you have been led to believe in an irrational way that the Gods take no care of human affairs. That your error may not increase, I will endeavour to purify your soul.' Do you, Megillus and Cleinias, make answer for the youth, and when we come to a difficulty, I will carry you over the water as I did before. 'Very good.' He will easily be convinced that the Gods care for the small as well as the great; for he heard what was said of their goodness and of their having all things under their care. 'He certainly heard.' Then now let us enquire what is meant by the virtue of the Gods. To possess mind belongs to virtue, and the contrary to vice. 'That is what we say.' And is not courage a part of virtue, and cowardice of vice? 'Certainly.' And to the Gods we ascribe virtues; but idleness and indolence are not virtues. 'Of course not.' And is God to be conceived of as a careless, indolent fellow, such as the poet would compare to a stingless drone? 'Impossible.' Can we be right in praising any one who cares for great matters and leaves the small to take care of themselves? Whether God or man, he who does so, must either think the neglect of such matters to be of no consequence, or he is indolent and careless. For surely neither of them can be charged with neglect if they fail to attend to something which is beyond their power? 'Certainly not.'

\par  And now we will examine the two classes of offenders who admit that there are Gods, but say,—the one that they may be appeased, the other that they take no care of small matters: do they not acknowledge that the Gods are omnipotent and omniscient, and also good and perfect? 'Certainly.' Then they cannot be indolent, for indolence is the offspring of idleness, and idleness of cowardice, and there is no cowardice in God. 'True.' If the Gods neglect small matters, they must either know or not know that such things are not to be regarded. But of course they know that they should be regarded, and knowing, they cannot be supposed to neglect their duty, overcome by the seductions of pleasure or pain. 'Impossible.' And do not all human things share in soul, and is not man the most religious of animals and the possession of the Gods? And the Gods, who are the best of owners, will surely take care of their property, small or great. Consider further, that the greater the power of perception, the less the power of action. For it is harder to see and hear the small than the great, but easier to control them. Suppose a physician who had to cure a patient—would he ever succeed if he attended to the great and neglected the little? 'Impossible.' Is not life made up of littles?—the pilot, general, householder, statesman, all attend to small matters; and the builder will tell you that large stones do not lie well without small ones. And God is not inferior to mortal craftsmen, who in proportion to their skill are careful in the details of their work; we must not imagine the best and wisest to be a lazy good-for-nothing, who wearies of his work and hurries over small and easy matters. 'Never, never!' He who charges the Gods with neglect has been forced to admit his error; but I should like further to persuade him that the author of all has made every part for the sake of the whole, and that the smallest part has an appointed state of action or passion, and that the least action or passion of any part has a presiding minister. You, we say to him, are a minute fraction of this universe, created with a view to the whole; the world is not made for you, but you for the world; for the good artist considers the whole first, and afterwards the parts. And you are annoyed at not seeing how you and the universe are all working together for the best, so far as the laws of the common creation admit. The soul undergoes many changes from her contact with bodies; and all that the player does is to put the pieces into their right places. 'What do you mean?' I mean that God acts in the way which is simplest and easiest. Had each thing been formed without any regard to the rest, the transposition of the Cosmos would have been endless; but now there is not much trouble in the government of the world. For when the king saw the actions of the living souls and bodies, and the virtue and vice which were in them, and the indestructibility of the soul and body (although they were not eternal), he contrived so to arrange them that virtue might conquer and vice be overcome as far as possible; giving them a seat and room adapted to them, but leaving the direction of their separate actions to men's own wills, which make our characters to be what they are. 'That is very probable.' All things which have a soul possess in themselves the principle of change, and in changing move according to fate and law; natures which have undergone lesser changes move on the surface; but those which have changed utterly for the worse, sink into Hades and the infernal world. And in all great changes for good and evil which are produced either by the will of the soul or the influence of others, there is a change of place. The good soul, which has intercourse with the divine nature, passes into a holier and better place; and the evil soul, as she grows worse, changes her place for the worse. This,—as we declare to the youth who fancies that he is neglected of the Gods,—is the law of divine justice—the worse to the worse, the better to the better, like to like, in life and in death. And from this law no man will ever boast that he has escaped. Even if you say—'I am small, and will creep into the earth,' or 'I am high, and will mount to heaven'—you are not so small or so high that you shall not pay the fitting penalty, either here or in the world below. This is also the explanation of the seeming prosperity of the wicked, in whose actions as in a mirror you imagined that you saw the neglect of the Gods, not considering that they make all things contribute to the whole. And how then could you form any idea of true happiness?—If Cleinias and Megillus and I have succeeded in persuading you that you know not what you say about the Gods, God will help you; but if there is still any deficiency of proof, hear our answer to the third opponent.

\par  Enough has been said to prove that the Gods exist and care for us; that they can be propitiated, or that they receive gifts, is not to be allowed or admitted for an instant. 'Let us proceed with the argument.' Tell me, by the Gods, I say, how the Gods are to be propitiated by us? Are they not rulers, who may be compared to charioteers, pilots, perhaps generals, or physicians providing against the assaults of disease, husbandmen observing the perils of the seasons, shepherds watching their flocks? To whom shall we compare them? We acknowledged that the world is full both of good and evil, but having more of evil than of good. There is an immortal conflict going on, in which Gods and demigods are our allies, and we their property; for injustice and folly and wickedness make war in our souls upon justice and temperance and wisdom. There is little virtue to be found on earth; and evil natures fawn upon the Gods, like wild beasts upon their keepers, and believe that they can win them over by flattery and prayers. And this sin, which is termed dishonesty, is to the soul what disease is to the body, what pestilence is to the seasons, what injustice is to states. 'Quite so.' And they who maintain that the Gods can be appeased must say that they forgive the sins of men, if they are allowed to share in their spoils; as you might suppose wolves to mollify the dogs by throwing them a portion of the prey. 'That is the argument.' But let us apply our images to the Gods—are they the pilots who are won by gifts to wreck their own ships—or the charioteers who are bribed to lose the race—or the generals, or doctors, or husbandmen, who are perverted from their duty—or the dogs who are silenced by wolves? 'God forbid.' Are they not rather our best guardians; and shall we suppose them to fall short even of a moderate degree of human or even canine virtue, which will not betray justice for reward? 'Impossible.' He, then, who maintains such a doctrine, is the most blasphemous of mankind.

\par  And now our three points are proven; and we are agreed (1) that there are Gods, (2) that they care for men, (3) that they cannot be bribed to do injustice. I have spoken warmly, from a fear lest this impiety of theirs should lead to a perversion of life. And our warmth will not have been in vain, if we have succeeded in persuading these men to abominate themselves, and to change their ways. 'So let us hope.' Then now that the preamble is completed, we will make a proclamation commanding the impious to renounce their evil ways; and in case they refuse, the law shall be added:—If a man is guilty of impiety in word or deed, let the bystander inform the magistrates, and let the magistrates bring the offender before the court; and if any of the magistrates refuses to act, he likewise shall be tried for impiety. Any one who is found guilty of such an offence shall be fined at the discretion of the court, and shall also be punished by a term of imprisonment. There shall be three prisons—one for common offences against life and property; another, near by the spot where the Nocturnal Council will assemble, which is to be called the 'House of Reformation'; the third, to be situated in some desolate region in the centre of the country, shall be called by a name indicating retribution. There are three causes of impiety, and from each of them spring impieties of two kinds, six in all. First, there is the impiety of those who deny the existence of the Gods; these may be honest men, haters of evil, who are only dangerous because they talk loosely about the Gods and make others like themselves; but there is also a more vicious class, who are full of craft and licentiousness. To this latter belong diviners, jugglers, despots, demagogues, generals, hierophants of private mysteries, and sophists. The first class shall be only imprisoned and admonished. The second class should be put to death, if they could be, many times over. The two other sorts of impiety, first of those who deny the care of the Gods, and secondly, of those who affirm that they may be propitiated, have similar subdivisions, varying in degree of guilt. Those who have learnt to blaspheme from mere ignorance shall be imprisoned in the House of Reformation for five years at least, and not allowed to see any one but members of the Nocturnal Council, who shall converse with them touching their souls health. If any of the prisoners come to their right mind, at the end of five years let them be restored to sane company; but he who again offends shall die. As to that class of monstrous natures who not only believe that the Gods are negligent, or may be propitiated, but pretend to practise on the souls of quick and dead, and promise to charm the Gods, and to effect the ruin of houses and states—he, I say, who is guilty of these things, shall be bound in the central prison, and shall have no intercourse with any freeman, receiving only his daily rations of food from the public slaves; and when he dies, let him be cast beyond the border; and if any freeman assist to bury him, he shall be liable to a suit for impiety. But the sins of the father shall not be visited upon his children, who, like other orphans, shall be educated by the state. Further, let there be a general law which will have a tendency to repress impiety. No man shall have religious services in his house, but he shall go with his friends to pray and sacrifice in the temples. The reason of this is, that religious institutions can only be framed by a great intelligence. But women and weak men are always consecrating the event of the moment; they are under the influence of dreams and apparitions, and they build altars and temples in every village and in any place where they have had a vision. The law is designed to prevent this, and also to deter men from attempting to propitiate the Gods by secret sacrifices, which only multiply their sins. Therefore let the law run:—No one shall have private religious rites; and if a man or woman who has not been previously noted for any impiety offend in this way, let them be admonished to remove their rites to a public temple; but if the offender be one of the obstinate sort, he shall be brought to trial before the guardians, and if he be found guilty, let him die.

\par  BOOK XI. As to dealings between man and man, the principle of them is simple—Thou shalt not take what is not thine; and shalt do to others as thou wouldst that they should do to thee. First, of treasure trove:—May I never desire to find, or lift, if I find, or be induced by the counsel of diviners to lift, a treasure which one who was not my ancestor has laid down; for I shall not gain so much in money as I shall lose in virtue. The saying, 'Move not the immovable,' may be repeated in a new sense; and there is a common belief which asserts that such deeds prevent a man from having a family. To him who is careless of such consequences, and, despising the word of the wise, takes up a treasure which is not his—what will be done by the hand of the Gods, God only knows,—but I would have the first person who sees the offender, inform the wardens of the city or the country; and they shall send to Delphi for a decision, and whatever the oracle orders, they shall carry out. If the informer be a freeman, he shall be honoured, and if a slave, set free; but he who does not inform, if he be a freeman, shall be dishonoured, and if a slave, shall be put to death. If a man leave anywhere anything great or small, intentionally or unintentionally, let him who may find the property deem the deposit sacred to the Goddess of ways. And he who appropriates the same, if he be a slave, shall be beaten with many stripes; if a freeman, he shall pay tenfold, and be held to have done a dishonourable action. If a person says that another has something of his, and the other allows that he has the property in dispute, but maintains it to be his own, let the ownership be proved out of the registers of property. If the property is registered as belonging to some one who is absent, possession shall be given to him who offers sufficient security on behalf of the absentee; or if the property is not registered, let it remain with the three eldest magistrates, and if it should be an animal, the defeated party must pay the cost of its keep. A man may arrest his own slave, and he may also imprison for safe-keeping the runaway slave of a friend. Any one interfering with him must produce three sureties; otherwise, he will be liable to an action for violence, and if he be cast, must pay a double amount of damages to him from whom he has taken the slave. A freedman who does not pay due respect to his patron, may also be seized. Due respect consists in going three times a month to the house of his patron, and offering to perform any lawful service for him; he must also marry as his master pleases; and if his property be greater than his master's, he must hand over to him the excess. A freedman may not remain in the state, except with the consent of the magistrates and of his master, for more than twenty years; and whenever his census exceeds that of the third class, he must in any case leave the country within thirty days, taking his property with him. If he break this regulation, the penalty shall be death, and his property shall be confiscated. Suits about these matters are to be decided in the courts of the tribes, unless the parties have settled the matter before a court of neighbours or before arbiters. If anybody claim a beast, or anything else, let the possessor refer to the seller or giver of the property within thirty days, if the latter reside in the city, or, if the goods have been received from a stranger, within five months, of which the middle month shall include the summer solstice. All purchases and exchanges are to be made in the agora, and paid for on the spot; the law will not allow credit to be given. No law shall protect the money subscribed for clubs. He who sells anything of greater value than fifty drachmas shall abide in the city for ten days, and let his whereabouts be known to the buyer, in case of any reclamation. When a slave is sold who is subject to epilepsy, stone, or any other invisible disorder, the buyer, if he be a physician or trainer, or if he be warned, shall have no redress; but in other cases within six months, or within twelve months in epileptic disorders, he may bring the matter before a jury of physicians to be agreed upon by both parties; and the seller who loses the suit, if he be an expert, shall pay twice the price; or if he be a private person, the bargain shall be rescinded, and he shall simply refund. If a person knowingly sells a homicide to another, who is informed of his character, there is no redress. But if the judges—who are to be the five youngest guardians of the law—decide that the purchaser was not aware, then the seller is to pay threefold, and to purify the house of the buyer.

\par  He who exchanges money for money, or beast for beast, must warrant either of them to be sound and good. As in the case of other laws, let us have a preamble, relating to all this class of crime. Adulteration is a kind of falsehood about which the many commonly say that at proper times the practice may often be right, but they do not define at what times. But the legislator will tell them, that no man should invoke the Gods when he is practising deceit or fraud, in word or deed. For he is the enemy of heaven, first, who swears falsely, not thinking of the Gods by whom he swears, and secondly, he who lies to his superiors. (Now the superiors are the betters of inferiors,—the elder of the younger, parents of children, men of women, and rulers of subjects.) The trader who cheats in the agora is a liar and is perjured—he respects neither the name of God nor the regulations of the magistrates. If after hearing this he will still be dishonest, let him listen to the law:—The seller shall not have two prices on the same day, neither must he puff his goods, nor offer to swear about them. If he break the law, any citizen not less than thirty years of age may smite him. If he sell adulterated goods, the slave or metic who informs against him shall have the goods; the citizen who brings such a charge, if he prove it, shall offer up the goods in question to the Gods of the agora; or if he fail to prove it, shall be dishonoured. He who is detected in selling adulterated goods shall be deprived of them, and shall receive a stripe for every drachma of their value. The wardens of the agora and the guardians of the law shall take experienced persons into counsel, and draw up regulations for the agora. These shall be inscribed on a column in front of the court of the wardens of the agora.—As to the wardens of the city, enough has been said already. But if any omissions in the law are afterwards discovered, the wardens and the guardians shall supply them, and have them inscribed after the original regulations on a column before the court of the wardens of the city.

\par  Next in order follows the subject of retail trades, which in their natural use are the reverse of mischievous; for every man is a benefactor who reduces what is unequal to symmetry and proportion. Money is the instrument by which this is accomplished, and the shop-keeper, the merchant, and hotel-keeper do but supply the wants and equalize the possessions of mankind. Why, then, does any dishonour attach to a beneficent occupation? Let us consider the nature of the accusation first, and then see whether it can be removed. 'What is your drift?' Dear Cleinias, there are few men who are so gifted by nature, and improved by education, as to be able to control the desire of making money; or who are sober in their wishes and prefer moderation to accumulation. The great majority think that they can never have enough, and the consequence is that retail trade has become a reproach. Whereas, however ludicrous the idea may seem, if noble men and noble women could be induced to open a shop, and to trade upon incorruptible principles, then the aspect of things would change, and retail traders would be regarded as nursing fathers and mothers. In our own day the trader goes and settles in distant places, and receives the weary traveller hospitably at first, but in the end treats him as an enemy and a captive, whom he only liberates for an enormous ransom. This is what has brought retail trade into disrepute, and against this the legislator ought to provide. Men have said of old, that to fight against two opponents is hard; and the two opponents of whom I am thinking are wealth and poverty—the one corrupting men by luxury; the other, through misery, depriving them of the sense of shame. What remedies can a city find for this disease? First, to have as few retail traders as possible; secondly, to give retail trade over to a class whose corruption will not injure the state; and thirdly, to restrain the insolence and meanness of the retailers.

\par  Let us make the following laws:—(1) In the city of the Magnetes none of the 5040 citizens shall be a retailer or merchant, or do any service to any private persons who do not equally serve him, except to his father and mother and their fathers and mothers, and generally to his elders who are freemen, and whom he serves as a freeman. He who follows an illiberal pursuit may be cited for dishonouring his family, and kept in bonds for a year; and if he offend again, he shall be bound for two years; and for every offence his punishment shall be doubled: (2) Every retailer shall be a metic or a foreigner: (3) The guardians of the law shall have a special care of this part of the community, whose calling exposes them to peculiar temptations. They shall consult with persons of experience, and find out what prices will yield the traders a moderate profit, and fix them.

\par  When a man does not fulfil his contract, he being under no legal or other impediment, the case shall be brought before the court of the tribes, if not previously settled by arbitration. The class of artisans is consecrated to Hephaestus and Athene; the makers of weapons to Ares and Athene: all of whom, remembering that the Gods are their ancestors, should be ashamed to deceive in the practice of their craft. If any man is lazy in the fulfilment of his work, and fancies, foolish fellow, that his patron God will not deal hardly with him, he will be punished by the God; and let the law follow:—He who fails in his undertaking shall pay the value, and do the work gratis in a specified time. The contractor, like the seller, is enjoined by law to charge the simple value of his work; in a free city, art should be a true thing, and the artist must not practise on the ignorance of others. On the other hand, he who has ordered any work and does not pay the workman according to agreement, dishonours Zeus and Athene, and breaks the bonds of society. And if he does not pay at the time agreed, let him pay double; and although interest is forbidden in other cases, let the workman receive after the expiration of a year interest at the rate of an obol a month for every drachma (equal to 200 per cent. per ann.). And we may observe by the way, in speaking of craftsmen, that if our military craft do their work well, the state will praise those who honour them, and blame those who do not honour them. Not that the first place of honour is to be assigned to the warrior; a higher still is reserved for those who obey the laws.

\par  Most of the dealings between man and man are now settled, with the exception of such as relate to orphans and guardianships. These lead us to speak of the intentions of the dying, about which we must make regulations. I say 'must'; for mankind cannot be allowed to dispose of their property as they please, in ways at variance with one another and with law and custom. But a dying person is a strange being, and is not easily managed; he wants to be master of all he has, and is apt to use angry words. He will say,—'May I not do what I will with my own, and give much to my friends, and little to my enemies?' 'There is reason in that.' O Cleinias, in my judgment the older lawgivers were too soft-hearted, and wanting in insight into human affairs. They were too ready to listen to the outcry of a dying man, and hence they were induced to give him an absolute power of bequest. But I would say to him:—O creature of a day, you know neither what is yours nor yourself: for you and your property are not your own, but belong to your whole family, past and to come, and property and family alike belong to the State. And therefore I must take out of your hands the charge of what you leave behind you, with a view to the interests of all. And I hope that you will not quarrel with us, now that you are going the way of all mankind; we will do our best for you and yours when you are no longer here. Let this be our address to the living and dying, and let the law be as follows:—The father who has sons shall appoint one of them to be the heir of the lot; and if he has given any other son to be adopted by another, the adoption shall also be recorded; and if he has still a son who has no lot, and has a chance of going to a colony, he may give him what he has more than the lot; or if he has more than one son unprovided for, he may divide the money between them. A son who has a house of his own, and a daughter who is betrothed, are not to share in the bequest of money; and the son or daughter who, having inherited one lot, acquires another, is to bequeath the new inheritance to the next of kin. If a man have only daughters, he may adopt the husband of any one of them; or if he have lost a son, let him make mention of the circumstance in his will and adopt another. If he have no children, he may give away a tenth of his acquired property to whomsoever he likes; but he must adopt an heir to inherit the lot, and may leave the remainder to him. Also he may appoint guardians for his children; or if he die without appointing them or without making a will, the nearest kinsmen,—two on the father's and two on the mother's side,—and one friend of the departed, shall be appointed guardians. The fifteen eldest guardians of the law are to have special charge of all orphans, the whole number of fifteen being divided into bodies of three, who will succeed one another according to seniority every year for five years. If a man dying intestate leave daughters, he must pardon the law which marries them for looking, first to kinship, and secondly to the preservation of the lot. The legislator cannot regard the character of the heir, which to the father is the first consideration. The law will therefore run as follows:—If the intestate leave daughters, husbands are to be found for them among their kindred according to the following table of affinity: first, their father's brothers; secondly, the sons of their father's brothers; thirdly, of their father's sisters; fourthly, their great-uncles; fifthly, the sons of a great-uncle; sixthly, the sons of a great-aunt. The kindred in such cases shall always be reckoned in this way; the relationship shall proceed upwards through brothers and sisters and brothers' and sisters' children, and first the male line must be taken and then the female. If there is a dispute in regard to fitness of age for marriage, this the judge shall decide, after having made an inspection of the youth naked, and of the maiden naked down to the waist. If the maiden has no relations within the degree of third cousin, she may choose whom she likes, with the consent of her guardians; or she may even select some one who has gone to a colony, and he, if he be a kinsman, will take the lot by law; if not, he must have her guardians' consent, as well as hers. When a man dies without children and without a will, let a young man and a young woman go forth from the family and take up their abode in the desolate house. The woman shall be selected from the kindred in the following order of succession:—first, a sister of the deceased; second, a brother's daughter; third, a sister's daughter; fourth, a father's sister; fifth, a daughter of a father's brother; sixth, a daughter of a father's sister. For the man the same order shall be observed as in the preceding case. The legislator foresees that laws of this kind will sometimes press heavily, and that his intention cannot always be fulfilled; as for example, when there are mental and bodily defects in the persons who are enjoined to marry. But he must be excused for not being always able to reconcile the general principles of public interest with the particular circumstances of individuals; and he is willing to allow, in like manner, that the individual cannot always do what the lawgiver wishes. And then arbiters must be chosen, who will determine equitably the cases which may arise under the law: e.g. a rich cousin may sometimes desire a grander match, or the requirements of the law can only be fulfilled by marrying a madwoman. To meet such cases let the following law be enacted:—If any one comes forward and says that the lawgiver, had he been alive, would not have required the carrying out of the law in a particular case, let him go to the fifteen eldest guardians of the law who have the care of orphans; but if he thinks that too much power is thus given to them, he may bring the case before the court of select judges.

\par  Thus will orphans have a second birth. In order to make their sad condition as light as possible, the guardians of the law shall be their parents, and shall be admonished to take care of them. And what admonition can be more appropriate than the assurance which we formerly gave, that the souls of the dead watch over mortal affairs? About this there are many ancient traditions, which may be taken on trust from the legislator. Let men fear, in the first place, the Gods above; secondly, the souls of the departed, who naturally care for their own descendants; thirdly, the aged living, who are quick to hear of any neglect of family duties, especially in the case of orphans. For they are the holiest and most sacred of all deposits, and the peculiar care of guardians and magistrates; and those who try to bring them up well will contribute to their own good and to that of their families. He who listens to the preamble of the law will never know the severity of the legislator; but he who disobeys, and injures the orphan, will pay twice the penalty he would have paid if the parents had been alive. More laws might have been made about orphans, did we not suppose that the guardians have children and property of their own which are protected by the laws; and the duty of the guardian in our state is the same as that of a father, though his honour or disgrace is greater. A legal admonition and threat may, however, be of service: the guardian of the orphan and the guardian of the law who is over him, shall love the orphan as their own children, and take more care of his or her property than of their own. If the guardian of the child neglect his duty, the guardian of the law shall fine him; and the guardian may also have the magistrate tried for neglect in the court of select judges, and he shall pay, if convicted, a double penalty. Further, the guardian of the orphan who is careless or dishonest may be fined on the information of any of the citizens in a fourfold penalty, half to go to the orphan and half to the prosecutor of the suit. When the orphan is of age, if he thinks that he has been ill-used, his guardian may be brought to trial by him within five years, and the penalty shall be fixed by the court. Or if the magistrate has neglected the orphan, he shall pay damages to him; but if he have defrauded him, he shall make compensation and also be deposed from his office of guardian of the law.

\par  If irremediable differences arise between fathers and sons, the father may want to renounce his son, or the son may indict his father for imbecility: such violent separations only take place when the family are 'a bad lot'; if only one of the two parties is bad, the differences do not grow to so great a height. But here arises a difficulty. Although in any other state a son who is disinherited does not cease to be a citizen, in ours he does; for the number of citizens cannot exceed 5040. And therefore he who is to suffer such a penalty ought to be abjured, not only by his father, but by the whole family. The law, then, should run as follows:—If any man's evil fortune or temper incline him to disinherit his son, let him not do so lightly or on the instant; but let him have a council of his own relations and of the maternal relations of his son, and set forth to them the propriety of disinheriting him, and allow his son to answer. And if more than half of the kindred male and female, being of full age, condemn the son, let him be disinherited. If any other citizen desires to adopt him, he may, for young men's characters often change in the course of life. But if, after ten years, he remains unadopted, let him be sent to a colony. If disease, or old age, or evil disposition cause a man to go out of his mind, and he is ruining his house and property, and his son doubts about indicting him for insanity, let him lay the case before the eldest guardians of the law, and consult with them. And if they advise him to proceed, and the father is decided to be imbecile, he shall have no more control over his property, but shall live henceforward like a child in the house.

\par  If a man and his wife are of incompatible tempers, ten guardians of the law and ten of the matrons who regulate marriage shall take their case in hand, and reconcile them, if possible. If, however, their swelling souls cannot be pacified, the wife may try and find a new husband, and the husband a new wife; probably they are not very gentle creatures, and should therefore be joined to milder natures. The younger of those who are separated should also select their partners with a view to the procreation of children; while the older should seek a companion for their declining years. If a woman dies, leaving children male or female, the law will advise, but not compel, the widower to abstain from a second marriage; if she leave no children, he shall be compelled to marry. Also a widow, if she is not old enough to live honestly without marriage, shall marry again; and in case she have no children, she should marry for the sake of them. There is sometimes an uncertainty which parent the offspring is to follow: in unions of a female slave with a male slave, or with a freedman or free man, or of a free woman with a male slave, the offspring is to belong to the master; but if the master or mistress be themselves the parent of the child, the slave and the child are to be sent away to another land.

\par  Concerning duty to parents, let the preamble be as follows:—We honour the Gods in their lifeless images, and believe that we thus propitiate them. But he who has an aged father or mother has a living image, which if he cherish it will do him far more good than any statue. 'What do you mean by cherishing them?' I will tell you. Oedipus and Amyntor and Theseus cursed their children, and their curses took effect. This proves that the Gods hear the curses of parents who are wronged; and shall we doubt that they hear and fulfil their blessings too?' 'Surely not.' And, as we were saying, no image is more honoured by the Gods than an aged father and mother, to whom when honour is done, the God who hears their prayers is rejoiced, and their influence is greater than that of the lifeless statue; for they pray that good or evil may come to us in proportion as they are honoured or dishonoured, but the statue is silent. 'Excellent.' Good men are glad when their parents live to extreme old age, or if they depart early, lament their loss; but to bad man their parents are always terrible. Wherefore let every one honour his parents, and if this preamble fails of influencing him, let him hear the law:—If any one does not take sufficient care of his parents, let the aggrieved person inform the three eldest guardians of the law and three of the women who are concerned with marriages. Women up to forty years of age, and men up to thirty, who thus offend, shall be beaten and imprisoned. After that age they are to be brought before a court composed of the eldest citizens, who may inflict any punishment upon them which they please. If the injured party cannot inform, let any freeman who hears of the case inform; a slave who does so shall be set free,—if he be the slave of the one of the parties, by the magistrate,—if owned by another, at the cost of the state; and let the magistrates, take care that he is not wronged by any one out of revenge.

\par  The injuries which one person does to another by the use of poisons are of two kinds;—one affects the body by the employment of drugs and potions; the other works on the mind by the practice of sorcery and magic. Fatal cases of either sort have been already mentioned; and now we must have a law respecting cases which are not fatal. There is no use in arguing with a man whose mind is disturbed by waxen images placed at his own door, or on the sepulchre of his father or mother, or at a spot where three ways meet. But to the wizards themselves we must address a solemn preamble, begging them not to treat the world as if they were children, or compel the legislator to expose them, and to show men that the poisoner who is not a physician and the wizard who is not a prophet or diviner are equally ignorant of what they are doing. Let the law be as follows:—He who by the use of poison does any injury not fatal to a man or his servants, or any injury whether fatal or not to another's cattle or bees, is to be punished with death if he be a physician, and if he be not a physician he is to suffer the punishment awarded by the court: and he who injures another by sorcery, if he be a diviner or prophet, shall be put to death; and, if he be not a diviner, the court shall determine what he ought to pay or suffer.

\par  Any one who injures another by theft or violence shall pay damages at least equal to the injury; and besides the compensation, a suitable punishment shall be inflicted. The foolish youth who is the victim of others is to have a lighter punishment; he whose folly is occasioned by his own jealousy or desire or anger is to suffer more heavily. Punishment is to be inflicted, not for the sake of vengeance, for what is done cannot be undone, but for the sake of prevention and reformation. And there should be a proportion between the punishment and the crime, in which the judge, having a discretion left him, must, by estimating the crime, second the legislator, who, like a painter, furnishes outlines for him to fill up.

\par  A madman is not to go about at large in the city, but is to be taken care of by his relatives. Neglect on their part is to be punished in the first class by a fine of a hundred drachmas, and proportionally in the others. Now madness is of various kinds; in addition to that which arises from disease there is the madness which originates in a passionate temperament, and makes men when engaged in a quarrel use foul and abusive language against each other. This is intolerable in a well-ordered state; and therefore our law shall be as follows:—No one is to speak evil of another, but when men differ in opinion they are to instruct one another without speaking evil. Nor should any one seek to rouse the passions which education has calmed; for he who feeds and nurses his wrath is apt to make ribald jests at his opponent, with a loss of character or dignity to himself. And for this reason no one may use any abusive word in a temple, or at sacrifices, or games, or in any public assembly, and he who offends shall be censured by the proper magistrate; and the magistrate, if he fail to censure him, shall not claim the prize of virtue. In any other place the angry man who indulges in revilings, whether he be the beginner or not, may be chastised by an elder. The reviler is always trying to make his opponent ridiculous; and the use of ridicule in anger we cannot allow. We forbid the comic poet to ridicule our citizens, under a penalty of expulsion from the country or a fine of three minae. Jest in which there is no offence may be allowed; but the question of offence shall be determined by the director of education, who is to be the licenser of theatrical performances.

\par  The righteous man who is in adversity will not be allowed to starve in a well-ordered city; he will never be a beggar. Nor is a man to be pitied, merely because he is hungry, unless he be temperate. Therefore let the law be as follows:—Let there be no beggars in our state; and he who begs shall be expelled by the magistrates both from town and country.

\par  If a slave, male or female, does any harm to the property of another, who is not himself a party to the harm, the master shall compensate the injury or give up the offending slave. But if the master argue that the charge has arisen by collusion, with the view of obtaining the slave, he may put the plaintiff on his trial for malpractices, and recover from him twice the value of the slave; or if he is cast he must make good the damage and deliver up the slave. The injury done by a horse or other animal shall be compensated in like manner.

\par  A witness who will not come of himself may be summoned, and if he fail in appearing, he shall be liable for any harm which may ensue: if he swears that he does not know, he may leave the court. A judge who is called upon as a witness must not vote. A free woman, if she is over forty, may bear witness and plead, and, if she have no husband, she may also bring an action. A slave, male or female, and a child may witness and plead only in case of murder, but they must give sureties that they will appear at the trial, if they should be charged with false witness. Such charges must be made pending the trial, and the accusations shall be sealed by both parties and kept by the magistrates until the trial for perjury comes off. If a man is twice convicted of perjury, he is not to be required, if three times, he is not to be allowed to bear witness, or, if he persists in bearing witness, is to be punished with death. When more than half the evidence is proved to be false there must be a new trial.

\par  The best and noblest things in human life are liable to be defiled and perverted. Is not justice the civilizer of mankind? And yet upon the noble profession of the advocate has come an evil name. For he is said to make the worse appear the better cause, and only requires money in return for his services. Such an art will be forbidden by the legislator, and if existing among us will be requested to depart to another city. To the disobedient let the voice of the law be heard saying:—He who tries to pervert justice in the minds of the judges, or to increase litigation, shall be brought before the supreme court. If he does so from contentiousness, let him be silenced for a time, and, if he offend again, put to death. If he have acted from a love of gain, let him be sent out of the country if he be a foreigner, or if he be a citizen let him be put to death.

\par  BOOK XII. If a false message be taken to or brought from other states, whether friendly or hostile, by ambassadors or heralds, they shall be indicted for having dishonoured their sacred office, and, if convicted, shall suffer a penalty.—Stealing is mean; robbery is shameless. Let no man deceive himself by the supposed example of the Gods, for no God or son of a God ever really practised either force or fraud. On this point the legislator is better informed than all the poets put together. He who listens to him shall be for ever happy, but he who will not listen shall have the following law directed against him:—He who steals much, or he who steals little of the public property is deserving of the same penalty; for they are both impelled by the same evil motive. When the law punishes one man more lightly than another, this is done under the idea, not that he is less guilty, but that he is more curable. Now a thief who is a foreigner or slave may be curable; but the thief who is a citizen, and has had the advantages of education, should be put to death, for he is incurable.

\par  Much consideration and many regulations are necessary about military expeditions; the great principal of all is that no one, male or female, in war or peace, in great matters or small, shall be without a commander. Whether men stand or walk, or drill, or pursue, or retreat, or wash, or eat, they should all act together and in obedience to orders. We should practise from our youth upwards the habits of command and obedience. All dances, relaxations, endurances of meats and drinks, of cold and heat, and of hard couches, should have a view to war, and care should be taken not to destroy the natural covering and use of the head and feet by wearing shoes and caps; for the head is the lord of the body, and the feet are the best of servants. The soldier should have thoughts like these; and let him hear the law:—He who is enrolled shall serve, and if he absent himself without leave he shall be indicted for failure of service before his own branch of the army when the expedition returns, and if he be found guilty he shall suffer the penalty which the courts award, and never be allowed to contend for any prize of valour, or to accuse another of misbehaviour in military matters. Desertion shall also be tried and punished in the same manner. After the courts for trying failure of service and desertion have been held, the generals shall hold another court, in which the several arms of the service will award prizes for the expedition which has just concluded. The prize is to be a crown of olive, which the victor shall offer up at the temple of his favourite war God...In any suit which a man brings, let the indictment be scrupulously true, for justice is an honourable maiden, to whom falsehood is naturally hateful. For example, when men are prosecuted for having lost their arms, great care should be taken by the witnesses to distinguish between cases in which they have been lost from necessity and from cowardice. If the hero Patroclus had not been killed but had been brought back alive from the field, he might have been reproached with having lost the divine armour. And a man may lose his arms in a storm at sea, or from a fall, and under many other circumstances. There is a distinction of language to be observed in the use of the two terms, 'thrower away of a shield' (ripsaspis), and 'loser of arms' (apoboleus oplon), one being the voluntary, the other the involuntary relinquishment of them. Let the law then be as follows:—If any one is overtaken by the enemy, having arms in his hands, and he leaves them behind him voluntarily, choosing base life instead of honourable death, let justice be done. The old legend of Caeneus, who was changed by Poseidon from a woman into a man, may teach by contraries the appropriate punishment. Let the thrower away of his shield be changed from a man into a woman—that is to say, let him be all his life out of danger, and never again be admitted by any commander into the ranks of his army; and let him pay a heavy fine according to his class. And any commander who permits him to serve shall also be punished by a fine.

\par  All magistrates, whatever be their tenure of office, must give an account of their magistracy. But where shall we find the magistrate who is worthy to supervise them or look into their short-comings and crooked ways? The examiner must be more than man who is sufficient for these things. For the truth is that there are many causes of the dissolution of states; which, like ships or animals, have their cords, and girders, and sinews easily relaxed, and nothing tends more to their welfare and preservation than the supervision of them by examiners who are better than the magistrates; failing in this they fall to pieces, and each becomes many instead of one. Wherefore let the people meet after the summer solstice, in the precincts of Apollo and the Sun, and appoint three men of not less than fifty years of age. They shall proceed as follows:—Each citizen shall select some one, not himself, whom he thinks the best. The persons selected shall be reduced to one half, who have the greatest number of votes, if they are an even number; but if an odd number, he who has the smallest number of votes shall be previously withdrawn. The voting shall continue in the same manner until three only remain; and if the number of votes cast for them be equal, a distinction between the first, second, and third shall be made by lot. The three shall be crowned with an olive wreath, and proclamation made, that the city of the Magnetes, once more preserved by the Gods, presents her three best men to Apollo and the Sun, to whom she dedicates them as long as their lives answer to the judgment formed of them. They shall choose in the first year of their office twelve examiners, to continue until they are seventy-five years of age; afterwards three shall be added annually. While they hold office, they shall dwell within the precinct of the God. They are to divide all the magistracies into twelve classes, and may apply any methods of enquiry, and inflict any punishments which they please; in some cases singly, in other cases together, announcing the acquittal or punishment of the magistrate on a tablet which they will place in the agora. A magistrate who has been condemned by the examiners may appeal to the select judges, and, if he gain his suit, may in turn prosecute the examiners; but if the appellant is cast, his punishment shall be doubled, unless he was previously condemned to death.

\par  And what honours shall be paid to these examiners, whom the whole state counts worthy of the rewards of virtue? They shall have the first place at all sacrifices and other ceremonies, and in all assemblies and public places; they shall go on sacred embassies, and have the exclusive privilege of wearing a crown of laurel. They are priests of Apollo and the Sun, and he of their number who is judged first shall be high priest, and give his name to the year. The manner of their burial, too, shall be different from that of the other citizens. The colour of their funeral array shall be white, and, instead of the voice of lamentation, around the bier shall stand a chorus of fifteen boys and fifteen maidens, chanting hymns in honour of the deceased in alternate strains during an entire day; and at dawn a band of a hundred youths shall carry the bier to the grave, marching in the garb of warriors, and the boys in front of the bier shall sing their national hymn, while the maidens and women past child-bearing follow after. Priests and priestesses may also follow, unless the Pythian oracle forbids. The sepulchre shall be a vault built underground, which will last for ever, having couches of stone placed side by side; on one of these they shall lay the departed saint, and then cover the tomb with a mound, and plant trees on every side except one, where an opening shall be left for other interments. Every year there shall be games—musical, gymnastic, or equestrian, in honour of those who have passed every ordeal. But if any of them, after having been acquitted on any occasion, begin to show the wickedness of human nature, he who pleases may bring them to trial before a court composed of the guardians of the law, and of the select judges, and of any of the examiners who are alive. If he be convicted he shall be deprived of his honours, and if the accuser do not obtain a fifth part of the votes, he shall pay a fine according to his class.

\par  What is called the judgment of Rhadamanthus is suited to 'ages of faith,' but not to our days. He knew that his contemporaries believed in the Gods, for many of them were the sons of Gods; and he thought that the easiest and surest method of ending litigation was to commit the decision to Heaven. In our own day, men either deny the existence of Gods or their care of men, or maintain that they may be bribed by attentions and gifts; and the procedure of Rhadamanthus would therefore be out of date. When the religious ideas of mankind change, their laws should also change. Thus oaths should no longer be taken from plaintiff and defendant; simple statements of affirmation and denial should be substituted. For there is something dreadful in the thought, that nearly half the citizens of a state are perjured men. There is no objection to an oath, where a man has no interest in forswearing himself; as, for example, when a judge is about to give his decision, or in voting at an election, or in the judgment of games and contests. But where there would be a premium on perjury, oaths and imprecations should be prohibited as irrelevant, like appeals to feeling. Let the principles of justice be learned and taught without words of evil omen. The oaths of a stranger against a stranger may be allowed, because strangers are not permitted to become permanent residents in our state.

\par  Trials in private causes are to be decided in the same manner as lesser offences against the state. The non-attendance at a chorus or sacrifice, or the omission to pay a war-tax, may be regarded as in the first instance remediable, and the defaulter may give security; but if he forfeits the security, the goods pledged shall be sold and the money given to the state. And for obstinate disobedience, the magistrate shall have the power of inflicting greater penalties.

\par  A city which is without trade or commerce must consider what it will do about the going abroad of its own people and the admission of strangers. For out of intercourse with strangers there arises great confusion of manners, which in most states is not of any consequence, because the confusion exists already; but in a well-ordered state it may be a great evil. Yet the absolute prohibition of foreign travel, or the exclusion of strangers, is impossible, and would appear barbarous to the rest of mankind. Public opinion should never be lightly regarded, for the many are not so far wrong in their judgments as in their lives. Even the worst of men have often a divine instinct, which enables them to judge of the differences between the good and bad. States are rightly advised when they desire to have the praise of men; and the greatest and truest praise is that of virtue. And our Cretan colony should, and probably will, have a character for virtue, such as few cities have. Let this, then, be our law about foreign travel and the reception of strangers:—No one shall be allowed to leave the country who is under forty years of age—of course military service abroad is not included in this regulation—and no one at all except in a public capacity. To the Olympic, and Pythian, and Nemean, and Isthmian games, shall be sent the fairest and best and bravest, who shall support the dignity of the city in time of peace. These, when they come home, shall teach the youth the inferiority of all other governments. Besides those who go on sacred missions, other persons shall be sent out by permission of the guardians to study the institutions of foreign countries. For a people which has no experience, and no knowledge of the characters of men or the reason of things, but lives by habit only, can never be perfectly civilized. Moreover, in all states, bad as well as good, there are holy and inspired men; these the citizen of a well-ordered city should be ever seeking out; he should go forth to find them over sea and over land, that he may more firmly establish institutions in his own state which are good already and amend the bad. 'What will be the best way of accomplishing such an object?' In the first place, let the visitor of foreign countries be between fifty and sixty years of age, and let him be a citizen of repute, especially in military matters. On his return he shall appear before the Nocturnal Council: this is a body which sits from dawn to sunrise, and includes amongst its members the priests who have gained the prize of virtue, and the ten oldest guardians of the law, and the director and past directors of education; each of whom has power to bring with him a younger friend of his own selection, who is between thirty and forty. The assembly thus constituted shall consider the laws of their own and other states, and gather information relating to them. Anything of the sort which is approved by the elder members of the council shall be studied with all diligence by the younger; who are to be specially watched by the rest of the citizens, and shall receive honour, if they are deserving of honour, or dishonour, if they prove inferior. This is the assembly to which the visitor of foreign countries shall come and tell anything which he has heard from others in the course of his travels, or which he has himself observed. If he be made neither better nor worse, let him at least be praised for his zeal; and let him receive still more praise, and special honour after death, if he be improved. But if he be deteriorated by his travels, let him be prohibited from speaking to any one; and if he submit, he may live as a private individual: but if he be convicted of attempting to make innovations in education and the laws, let him die.

\par  Next, as to the reception of strangers. Of these there are four classes:—First, merchants, who, like birds of passage, find their way over the sea at a certain time of the year, that they may exhibit their wares. These should be received in markets and public buildings without the city, by proper officers, who shall see that justice is done them, and shall also watch against any political designs which they may entertain; no more intercourse is to be held with them than is absolutely necessary. Secondly, there are the visitors at the festivals, who shall be entertained by hospitable persons at the temples for a reasonable time; the priests and ministers of the temples shall have a care of them. In small suits brought by them or against them, the priests shall be the judges; but in the more important, the wardens of the agora. Thirdly, there are ambassadors of foreign states; these are to be honourably received by the generals and commanders, and placed under the care of the Prytanes and of the persons with whom they are lodged. Fourthly, there is the philosophical stranger, who, like our own spectators, from time to time goes to see what is rich and rare in foreign countries. Like them he must be fifty years of age: and let him go unbidden to the doors of the wise and rich, that he may learn from them, and they from him.

\par  These are the rules of missions into foreign countries, and of the reception of strangers. Let Zeus, the God of hospitality, be honoured; and let not the stranger be excluded, as in Egypt, from meals and sacrifices, or, (as at Sparta,) driven away by savage proclamations.

\par  Let guarantees be clearly given in writing and before witnesses. The number of witnesses shall be three when the sum lent is under a thousand drachmas, or five when above. The agent and principal at a fraudulent sale shall be equally liable. He who would search another man's house for anything must swear that he expects to find it there; and he shall enter naked, or having on a single garment and no girdle. The owner shall place at the disposal of the searcher all his goods, sealed as well as unsealed; if he refuse, he shall be liable in double the value of the property, if it shall prove to be in his possession. If the owner be absent, the searcher may counter-seal the property which is under seal, and place watchers. If the owner remain absent more than five days, the searcher shall take the magistrates, and open the sealed property, and seal it up again in their presence. The recovery of goods disputed, except in the case of lands and houses, (about which there can be no dispute in our state), is to be barred by time. The public and unimpeached use of anything for a year in the city, or for five years in the country, or the private possession and domestic use for three years in the city, or for ten years in the country, is to give a right of ownership. But if the possessor have the property in a foreign country, there shall be no bar as to time. The proceedings of any trial are to be void, in which either the parties or the witnesses, whether bond or free, have been prevented by violence from attending:—if a slave be prevented, the suit shall be invalid; or if a freeman, he who is guilty of the violence shall be imprisoned for a year, and shall also be liable to an action for kidnapping. If one competitor forcibly prevents another from attending at the games, the other may be inscribed as victor in the temples, and the first, whether victor or not, shall be liable to an action for damages. The receiver of stolen goods shall undergo the same punishment as the thief. The receiver of an exile shall be punished with death. A man ought to have the same friends and enemies as his country; and he who makes war or peace for himself shall be put to death. And if a party in the state make war or peace, their leaders shall be indicted by the generals, and, if convicted, they shall be put to death. The ministers and officers of a country ought not to receive gifts, even as the reward of good deeds. He who disobeys shall die.

\par  With a view to taxation a man should have his property and income valued: and the government may, at their discretion, levy the tax upon the annual return, or take a portion of the whole.

\par  The good man will offer moderate gifts to the Gods; his land or hearth cannot be offered, because they are already consecrated to all Gods. Gold and silver, which arouse envy, and ivory, which is taken from the dead body of an animal, are unsuitable offerings; iron and brass are materials of war. Wood and stone of a single piece may be offered; also woven work which has not occupied one woman more than a month in making. White is a colour which is acceptable to the Gods; figures of birds and similar offerings are the best of gifts, but they must be such as the painter can execute in a day.

\par  Next concerning lawsuits. Judges, or rather arbiters, may be agreed upon by the plaintiff and defendant; and if no decision is obtained from them, their fellow-tribesmen shall judge. At this stage there shall be an increase of the penalty: the defendant, if he be cast, shall pay a fifth more than the damages claimed. If he further persist, and appeal a second time, the case shall be heard before the select judges; and he shall pay, if defeated, the penalty and half as much again. And the pursuer, if on the first appeal he is defeated, shall pay one fifth of the damages claimed by him; and if on the second, one half. Other matters relating to trials, such as the assignment of judges to courts, the times of sitting, the number of judges, the modes of pleading and procedure, as we have already said, may be determined by younger legislators.

\par  These are to be the rules of private courts. As regards public courts, many states have excellent modes of procedure which may serve for models; these, when duly tested by experience, should be ratified and made permanent by us.

\par  Let the judge be accomplished in the laws. He should possess writings about them, and make a study of them; for laws are the highest instrument of mental improvement, and derive their name from mind (nous, nomos). They afford a measure of all censure and praise, whether in verse or prose, in conversation or in books, and are an antidote to the vain disputes of men and their equally vain acquiescence in each other's opinions. The just judge, who imbibes their spirit, makes the city and himself to stand upright. He establishes justice for the good, and cures the tempers of the bad, if they can be cured; but denounces death, which is the only remedy, to the incurable, the threads of whose life cannot be reversed.

\par  When the suits of the year are completed, execution is to follow. The court is to award to the plaintiff the property of the defendant, if he is cast, reserving to him only his lot of land. If the plaintiff is not satisfied within a month, the court shall put into his hands the property of the defendant. If the defendant fails in payment to the amount of a drachma, he shall lose the use and protection of the court; or if he rebel against the authority of the court, he shall be brought before the guardians of the law, and if found guilty he shall be put to death.

\par  Man having been born, educated, having begotten and brought up children, and gone to law, fulfils the debt of nature. The rites which are to be celebrated after death in honour of the Gods above and below shall be determined by the Interpreters. The dead shall be buried in uncultivated places, where they will be out of the way and do least injury to the living. For no one either in life or after death has any right to deprive other men of the sustenance which mother earth provides for them. No sepulchral mound is to be piled higher than five men can raise it in five days, and the grave-stone shall not be larger than is sufficient to contain an inscription of four heroic verses. The dead are only to be exposed for three days, which is long enough to test the reality of death. The legislator will instruct the people that the body is a mere shadow or image, and that the soul, which is our true being, is gone to give an account of herself before the Gods below. When they hear this, the good are full of hope, and the evil are terrified. It is also said that not much can be done for any one after death. And therefore while in life all man should be helped by their kindred to pass their days justly and holily, that they may depart in peace. When a man loses a son or a brother, he should consider that the beloved one has gone away to fulfil his destiny in another place, and should not waste money over his lifeless remains. Let the law then order a moderate funeral of five minae for the first class, of three for the second, of two for the third, of one for the fourth. One of the guardians of the law, to be selected by the relatives, shall assist them in arranging the affairs of the deceased. There would be a want of delicacy in prescribing that there should or should not be mourning for the dead. But, at any rate, such mourning is to be confined to the house; there must be no processions in the streets, and the dead body shall be taken out of the city before daybreak. Regulations about other forms of burial and about the non-burial of parricides and other sacrilegious persons have already been laid down. The work of legislation is therefore nearly completed; its end will be finally accomplished when we have provided for the continuance of the state.

\par  Do you remember the names of the Fates? Lachesis, the giver of the lots, is the first of them; Clotho, the spinster, the second; Atropos, the unchanging one, is the third and last, who makes the threads of the web irreversible. And we too want to make our laws irreversible, for the unchangeable quality in them will be the salvation of the state, and the source of health and order in the bodies and souls of our citizens. 'But can such a quality be implanted?' I think that it may; and at any rate we must try; for, after all our labour, to have been piling up a fabric which has no foundation would be too ridiculous. 'What foundation would you lay?' We have already instituted an assembly which was composed of the ten oldest guardians of the law, and secondly, of those who had received prizes of virtue, and thirdly, of the travellers who had gone abroad to enquire into the laws of other countries. Moreover, each of the members was to choose a young man, of not less than thirty years of age, to be approved by the rest; and they were to meet at dawn, when all the world is at leisure. This assembly will be an anchor to the vessel of state, and provide the means of permanence; for the constitutions of states, like all other things, have their proper saviours, which are to them what the head and soul are to the living being. 'How do you mean?' Mind in the soul, and sight and hearing in the head, or rather, the perfect union of mind and sense, may be justly called every man's salvation. 'Certainly.' Yes; but of what nature is this union? In the case of a ship, for example, the senses of the sailors are added to the intelligence of the pilot, and the two together save the ship and the men in the ship. Again, the physician and the general have their objects; and the object of the one is health, of the other victory. States, too, have their objects, and the ruler must understand, first, their nature, and secondly, the means of attaining them, whether in laws or men. The state which is wanting in this knowledge cannot be expected to be wise when the time for action arrives. Now what class or institution is there in our state which has such a saving power? 'I suspect that you are referring to the Nocturnal Council.' Yes, to that council which is to have all virtue, and which should aim directly at the mark. 'Very true.' The inconsistency of legislation in most states is not surprising, when the variety of their objects is considered. One of them makes their rule of justice the government of a class; another aims at wealth; another at freedom, or at freedom and power; and some who call themselves philosophers maintain that you should seek for all of them at once. But our object is unmistakeably virtue, and virtue is of four kinds. 'Yes; and we said that mind is the chief and ruler of the three other kinds of virtue and of all else.' True, Cleinias; and now, having already declared the object which is present to the mind of the pilot, the general, the physician, we will interrogate the mind of the statesman. Tell me, I say, as the physician and general have told us their object, what is the object of the statesman. Can you tell me? 'We cannot.' Did we not say that there are four virtues—courage, wisdom, and two others, all of which are called by the common name of virtue, and are in a sense one? 'Certainly we did.' The difficulty is, not in understanding the differences of the virtues, but in apprehending their unity. Why do we call virtue, which is a single thing, by the two names of wisdom and courage? The reason is that courage is concerned with fear, and is found both in children and in brutes; for the soul may be courageous without reason, but no soul was, or ever will be, wise without reason. 'That is true.' I have explained to you the difference, and do you in return explain to me the unity. But first let us consider whether any one who knows the name of a thing without the definition has any real knowledge of it. Is not such knowledge a disgrace to a man of sense, especially where great and glorious truths are concerned? and can any subject be more worthy of the attention of our legislators than the four virtues of which we are speaking—courage, temperance, justice, wisdom? Ought not the magistrates and officers of the state to instruct the citizens in the nature of virtue and vice, instead of leaving them to be taught by some chance poet or sophist? A city which is without instruction suffers the usual fate of cities in our day. What then shall we do? How shall we perfect the ideas of our guardians about virtue? how shall we give our state a head and eyes? 'Yes, but how do you apply the figure?' The city will be the body or trunk; the best of our young men will mount into the head or acropolis and be our eyes; they will look about them, and inform the elders, who are the mind and use the younger men as their instruments: together they will save the state. Shall this be our constitution, or shall all be educated alike, and the special training be given up? 'That is impossible.' Let us then endeavour to attain to some more exact idea of education. Did we not say that the true artist or guardian ought to have an eye, not only to the many, but to the one, and to order all things with a view to the one? Can there be any more philosophical speculation than how to reduce many things which are unlike to one idea? 'Perhaps not.' Say rather, 'Certainly not.' And the rulers of our divine state ought to have an exact knowledge of the common principle in courage, temperance, justice, wisdom, which is called by the name of virtue; and unless we know whether virtue is one or many, we shall hardly know what virtue is. Shall we contrive some means of engrafting this knowledge on our state, or give the matter up? 'Anything rather than that.' Let us begin by making an agreement. 'By all means, if we can.' Well, are we not agreed that our guardians ought to know, not only how the good and the honourable are many, but also how they are one? 'Yes, certainly.' The true guardian of the laws ought to know their truth, and should also be able to interpret and execute them? 'He should.' And is there any higher knowledge than the knowledge of the existence and power of the Gods? The people may be excused for following tradition; but the guardian must be able to give a reason of the faith which is in him. And there are two great evidences of religion—the priority of the soul and the order of the heavens. For no man of sense, when he contemplates the universe, will be likely to substitute necessity for reason and will. Those who maintain that the sun and the stars are inanimate beings are utterly wrong in their opinions. The men of a former generation had a suspicion, which has been confirmed by later thinkers, that things inanimate could never without mind have attained such scientific accuracy; and some (Anaxagoras) even in those days ventured to assert that mind had ordered all things in heaven; but they had no idea of the priority of mind, and they turned the world, or more properly themselves, upside down, and filled the universe with stones, and earth, and other inanimate bodies. This led to great impiety, and the poets said many foolish things against the philosophers, whom they compared to 'yelping she-dogs,' besides making other abusive remarks. No man can now truly worship the Gods who does not believe that the soul is eternal, and prior to the body, and the ruler of all bodies, and does not perceive also that there is mind in the stars; or who has not heard the connexion of these things with music, and has not harmonized them with manners and laws, giving a reason of things which are matters of reason. He who is unable to acquire this knowledge, as well as the ordinary virtues of a citizen, can only be a servant, and not a ruler in the state.

\par  Let us then add another law to the effect that the Nocturnal Council shall be a guard set for the salvation of the state. 'Very good.' To establish this will be our aim, and I hope that others besides myself will assist. 'Let us proceed along the road in which God seems to guide us.' We cannot, Megillus and Cleinias, anticipate the details which will hereafter be needed; they must be supplied by experience. 'What do you mean?' First of all a register will have to be made of all those whose age, character, or education would qualify them to be guardians. The subjects which they are to learn, and the order in which they are to be learnt, are mysteries which cannot be explained beforehand, but not mysteries in any other sense. 'If that is the case, what is to be done?' We must stake our all on a lucky throw, and I will share the risk by stating my views on education. And I would have you, Cleinias, who are the founder of the Magnesian state, and will obtain the greatest glory if you succeed, and will at least be praised for your courage, if you fail, take especial heed of this matter. If we can only establish the Nocturnal Council, we will hand over the city to its keeping; none of the present company will hesitate about that. Our dream will then become a reality; and our citizens, if they are carefully chosen and well educated, will be saviours and guardians such as the world hitherto has never seen.

\par  The want of completeness in the Laws becomes more apparent in the later books. There is less arrangement in them, and the transitions are more abrupt from one subject to another. Yet they contain several noble passages, such as the 'prelude to the discourse concerning the honour and dishonour of parents,' or the picture of the dangers attending the 'friendly intercourse of young men and maidens with one another,' or the soothing remonstrance which is addressed to the dying man respecting his right to do what he will with his own, or the fine description of the burial of the dead. The subject of religion in Book X is introduced as a prelude to offences against the Gods, and this portion of the work appears to be executed in Plato's best manner.

\par  In the last four books, several questions occur for consideration: among them are (I) the detection and punishment of offences; (II) the nature of the voluntary and involuntary; (III) the arguments against atheism, and against the opinion that the Gods have no care of human affairs; (IV) the remarks upon retail trade; (V) the institution of the Nocturnal Council.

\par  I. A weak point in the Laws of Plato is the amount of inquisition into private life which is to be made by the rulers. The magistrate is always watching and waylaying the citizens. He is constantly to receive information against improprieties of life. Plato does not seem to be aware that espionage can only have a negative effect. He has not yet discovered the boundary line which parts the domain of law from that of morality or social life. Men will not tell of one another; nor will he ever be the most honoured citizen, who gives the most frequent information about offenders to the magistrates.

\par  As in some writers of fiction, so also in philosophers, we may observe the effect of age. Plato becomes more conservative as he grows older, and he would govern the world entirely by men like himself, who are above fifty years of age; for in them he hopes to find a principle of stability. He does not remark that, in destroying the freedom he is destroying also the life of the State. In reducing all the citizens to rule and measure, he would have been depriving the Magnesian colony of those great men 'whose acquaintance is beyond all price;' and he would have found that in the worst-governed Hellenic State, there was more of a carriere ouverte for extraordinary genius and virtue than in his own.

\par  Plato has an evident dislike of the Athenian dicasteries; he prefers a few judges who take a leading part in the conduct of trials to a great number who only listen in silence. He allows of two appeals—in each case however with an increase of the penalty. Modern jurists would disapprove of the redress of injustice being purchased only at an increasing risk; though indirectly the burden of legal expenses, which seems to have been little felt among the Athenians, has a similar effect. The love of litigation, which is a remnant of barbarism quite as much as a corruption of civilization, and was innate in the Athenian people, is diminished in the new state by the imposition of severe penalties. If persevered in, it is to be punished with death.

\par  In the Laws murder and homicide besides being crimes, are also pollutions. Regarded from this point of view, the estimate of such offences is apt to depend on accidental circumstances, such as the shedding of blood, and not on the real guilt of the offender or the injury done to society. They are measured by the horror which they arouse in a barbarous age. For there is a superstition in law as well as in religion, and the feelings of a primitive age have a traditional hold on the mass of the people. On the other hand, Plato is innocent of the barbarity which would visit the sins of the fathers upon the children, and he is quite aware that punishment has an eye to the future, and not to the past. Compared with that of most European nations in the last century his penal code, though sometimes capricious, is reasonable and humane.

\par  A defect in Plato's criminal jurisprudence is his remission of the punishment when the homicide has obtained the forgiveness of the murdered person; as if crime were a personal affair between individuals, and not an offence against the State. There is a ridiculous disproportion in his punishments. Because a slave may fairly receive a blow for stealing one fig or one bunch of grapes, or a tradesman for selling adulterated goods to the value of one drachma, it is rather hard upon the slave that he should receive as many blows as he has taken grapes or figs, or upon the tradesman who has sold adulterated goods to the value of a thousand drachmas that he should receive a thousand blows.

\par  II. But before punishment can be inflicted at all, the legislator must determine the nature of the voluntary and involuntary. The great question of the freedom of the will, which in modern times has been worn threadbare with purely abstract discussion, was approached both by Plato and Aristotle—first, from the judicial; secondly, from the sophistical point of view. They were puzzled by the degrees and kinds of crime; they observed also that the law only punished hurts which are inflicted by a voluntary agent on an involuntary patient.

\par  In attempting to distinguish between hurt and injury, Plato says that mere hurt is not injury; but that a benefit when done in a wrong spirit may sometimes injure, e.g. when conferred without regard to right and wrong, or to the good or evil consequences which may follow. He means to say that the good or evil disposition of the agent is the principle which characterizes actions; and this is not sufficiently described by the terms voluntary and involuntary. You may hurt another involuntarily, and no one would suppose that you had injured him; and you may hurt him voluntarily, as in inflicting punishment—neither is this injury; but if you hurt him from motives of avarice, ambition, or cowardly fear, this is injury. Injustice is also described as the victory of desire or passion or self-conceit over reason, as justice is the subordination of them to reason. In some paradoxical sense Plato is disposed to affirm all injustice to be involuntary; because no man would do injustice who knew that it never paid and could calculate the consequences of what he was doing. Yet, on the other hand, he admits that the distinction of voluntary and involuntary, taken in another and more obvious sense, is the basis of legislation. His conception of justice and injustice is complicated (1) by the want of a distinction between justice and virtue, that is to say, between the quality which primarily regards others, and the quality in which self and others are equally regarded; (2) by the confusion of doing and suffering justice; (3) by the unwillingness to renounce the old Socratic paradox, that evil is involuntary.

\par  III. The Laws rest on a religious foundation; in this respect they bear the stamp of primitive legislation. They do not escape the almost inevitable consequence of making irreligion penal. If laws are based upon religion, the greatest offence against them must be irreligion. Hence the necessity for what in modern language, and according to a distinction which Plato would scarcely have understood, might be termed persecution. But the spirit of persecution in Plato, unlike that of modern religious bodies, arises out of the desire to enforce a true and simple form of religion, and is directed against the superstitions which tend to degrade mankind. Sir Thomas More, in his Utopia, is in favour of tolerating all except the intolerant, though he would not promote to high offices those who disbelieved in the immortality of the soul. Plato has not advanced quite so far as this in the path of toleration. But in judging of his enlightenment, we must remember that the evils of necromancy and divination were far greater than those of intolerance in the ancient world. Human nature is always having recourse to the first; but only when organized into some form of priesthood falls into the other; although in primitive as in later ages the institution of a priesthood may claim probably to be an advance on some form of religion which preceded. The Laws would have rested on a sounder foundation, if Plato had ever distinctly realized to his mind the difference between crime and sin or vice. Of this, as of many other controversies, a clear definition might have been the end. But such a definition belongs to a later age of philosophy.

\par  The arguments which Plato uses for the being of a God, have an extremely modern character: first, the consensus gentium; secondly, the argument which has already been adduced in the Phaedrus, of the priority of the self-moved. The answer to those who say that God 'cares not,' is, that He governs by general laws; and that he who takes care of the great will assuredly take care of the small. Plato did not feel, and has not attempted to consider, the difficulty of reconciling the special with the general providence of God. Yet he is on the road to a solution, when he regards the world as a whole, of which all the parts work together towards the final end.

\par  We are surprised to find that the scepticism, which we attribute to young men in our own day, existed then (compare Republic); that the Epicureanism expressed in the line of Horace (borrowed from Lucretius)—

\par  'Namque Deos didici securum agere aevum,'

\par  was already prevalent in the age of Plato; and that the terrors of another world were freely used in order to gain advantages over other men in this. The same objection which struck the Psalmist—'when I saw the prosperity of the wicked'—is supposed to lie at the root of the better sort of unbelief. And the answer is substantially the same which the modern theologian would offer:—that the ways of God in this world cannot be justified unless there be a future state of rewards and punishments. Yet this future state of rewards and punishments is in Plato's view not any addition of happiness or suffering imposed from without, but the permanence of good and evil in the soul: here he is in advance of many modern theologians. The Greek, too, had his difficulty about the existence of evil, which in one solitary passage, remarkable for being inconsistent with his general system, Plato explains, after the Magian fashion, by a good and evil spirit (compare Theaet., Statesman). This passage is also remarkable for being at variance with the general optimism of the Tenth Book—not 'all things are ordered by God for the best,' but some things by a good, others by an evil spirit.

\par  The Tenth Book of the Laws presents a picture of the state of belief among the Greeks singularly like that of the world in which we live. Plato is disposed to attribute the incredulity of his own age to several causes. First, to the bad effect of mythological tales, of which he retains his disapproval; but he has a weak side for antiquity, and is unwilling, as in the Republic, wholly to proscribe them. Secondly, he remarks the self-conceit of a newly-fledged generation of philosophers, who declare that the sun, moon, and stars, are earth and stones only; and who also maintain that the Gods are made by the laws of the state. Thirdly, he notes a confusion in the minds of men arising out of their misinterpretation of the appearances of the world around them: they do not always see the righteous rewarded and the wicked punished. So in modern times there are some whose infidelity has arisen from doubts about the inspiration of ancient writings; others who have been made unbelievers by physical science, or again by the seemingly political character of religion; while there is a third class to whose minds the difficulty of 'justifying the ways of God to man' has been the chief stumblingblock. Plato is very much out of temper at the impiety of some of his contemporaries; yet he is determined to reason with the victims, as he regards them, of these illusions before he punishes them. His answer to the unbelievers is twofold: first, that the soul is prior to the body; secondly, that the ruler of the universe being perfect has made all things with a view to their perfection. The difficulties arising out of ancient sacred writings were far less serious in the age of Plato than in our own.

\par  We too have our popular Epicureanism, which would allow the world to go on as if there were no God. When the belief in Him, whether of ancient or modern times, begins to fade away, men relegate Him, either in theory or practice, into a distant heaven. They do not like expressly to deny God when it is more convenient to forget Him; and so the theory of the Epicurean becomes the practice of mankind in general. Nor can we be said to be free from that which Plato justly considers to be the worst unbelief—of those who put superstition in the place of true religion. For the larger half of Christians continue to assert that the justice of God may be turned aside by gifts, and, if not by the 'odour of fat, and the sacrifice steaming to heaven,' still by another kind of sacrifice placed upon the altar—by masses for the quick and dead, by dispensations, by building churches, by rites and ceremonies—by the same means which the heathen used, taking other names and shapes. And the indifference of Epicureanism and unbelief is in two ways the parent of superstition, partly because it permits, and also because it creates, a necessity for its development in religious and enthusiastic temperaments. If men cannot have a rational belief, they will have an irrational. And hence the most superstitious countries are also at a certain point of civilization the most unbelieving, and the revolution which takes one direction is quickly followed by a reaction in the other. So we may read 'between the lines' ancient history and philosophy into modern, and modern into ancient. Whether we compare the theory of Greek philosophy with the Christian religion, or the practice of the Gentile world with the practice of the Christian world, they will be found to differ more in words and less in reality than we might have supposed. The greater opposition which is sometimes made between them seems to arise chiefly out of a comparison of the ideal of the one with the practice of the other.

\par  To the errors of superstition and unbelief Plato opposes the simple and natural truth of religion; the best and highest, whether conceived in the form of a person or a principle—as the divine mind or as the idea of good—is believed by him to be the basis of human life. That all things are working together for good to the good and evil to the evil in this or in some other world to which human actions are transferred, is the sum of his faith or theology. Unlike Socrates, he is absolutely free from superstition. Religion and morality are one and indivisible to him. He dislikes the 'heathen mythology,' which, as he significantly remarks, was not tolerated in Crete, and perhaps (for the meaning of his words is not quite clear) at Sparta. He gives no encouragement to individual enthusiasm; 'the establishment of religion could only be the work of a mighty intellect.' Like the Hebrews, he prohibits private rites; for the avoidance of superstition, he would transfer all worship of the Gods to the public temples. He would not have men and women consecrating the accidents of their lives. He trusts to human punishments and not to divine judgments; though he is not unwilling to repeat the old tradition that certain kinds of dishonesty 'prevent a man from having a family.' He considers that the 'ages of faith' have passed away and cannot now be recalled. Yet he is far from wishing to extirpate the sentiment of religion, which he sees to be common to all mankind—Barbarians as well as Hellenes. He remarks that no one passes through life without, sooner or later, experiencing its power. To which we may add the further remark that the greater the irreligion, the more violent has often been the religious reaction.

\par  It is remarkable that Plato's account of mind at the end of the Laws goes beyond Anaxagoras, and beyond himself in any of his previous writings. Aristotle, in a well-known passage (Met.) which is an echo of the Phaedo, remarks on the inconsistency of Anaxagoras in introducing the agency of mind, and yet having recourse to other and inferior, probably material causes. But Plato makes the further criticism, that the error of Anaxagoras consisted, not in denying the universal agency of mind, but in denying the priority, or, as we should say, the eternity of it. Yet in the Timaeus he had himself allowed that God made the world out of pre-existing materials: in the Statesman he says that there were seeds of evil in the world arising out of the remains of a former chaos which could not be got rid of; and even in the Tenth Book of the Laws he has admitted that there are two souls, a good and evil. In the Meno, the Phaedrus, and the Phaedo, he had spoken of the recovery of ideas from a former state of existence. But now he has attained to a clearer point of view: he has discarded these fancies. From meditating on the priority of the human soul to the body, he has learnt the nature of soul absolutely. The power of the best, of which he gave an intimation in the Phaedo and in the Republic, now, as in the Philebus, takes the form of an intelligence or person. He no longer, like Anaxagoras, supposes mind to be introduced at a certain time into the world and to give order to a pre-existing chaos, but to be prior to the chaos, everlasting and evermoving, and the source of order and intelligence in all things. This appears to be the last form of Plato's religious philosophy, which might almost be summed up in the words of Kant, 'the starry heaven above and the moral law within.' Or rather, perhaps, 'the starry heaven above and mind prior to the world.'

\par  IV. The remarks about retail trade, about adulteration, and about mendicity, have a very modern character. Greek social life was more like our own than we are apt to suppose. There was the same division of ranks, the same aristocratic and democratic feeling, and, even in a democracy, the same preference for land and for agricultural pursuits. Plato may be claimed as the first free trader, when he prohibits the imposition of customs on imports and exports, though he was clearly not aware of the importance of the principle which he enunciated. The discredit of retail trade he attributes to the rogueries of traders, and is inclined to believe that if a nobleman would keep a shop, which heaven forbid! retail trade might become honourable. He has hardly lighted upon the true reason, which appears to be the essential distinction between buyers and sellers, the one class being necessarily in some degree dependent on the other. When he proposes to fix prices 'which would allow a moderate gain,' and to regulate trade in several minute particulars, we must remember that this is by no means so absurd in a city consisting of 5040 citizens, in which almost every one would know and become known to everybody else, as in our own vast population. Among ourselves we are very far from allowing every man to charge what he pleases. Of many things the prices are fixed by law. Do we not often hear of wages being adjusted in proportion to the profits of employers? The objection to regulating them by law and thus avoiding the conflicts which continually arise between the buyers and sellers of labour, is not so much the undesirableness as the impossibility of doing so. Wherever free competition is not reconcileable either with the order of society, or, as in the case of adulteration, with common honesty, the government may lawfully interfere. The only question is,—Whether the interference will be effectual, and whether the evil of interference may not be greater than the evil which is prevented by it.

\par  He would prohibit beggars, because in a well-ordered state no good man would be left to starve. This again is a prohibition which might have been easily enforced, for there is no difficulty in maintaining the poor when the population is small. In our own times the difficulty of pauperism is rendered far greater, (1) by the enormous numbers, (2) by the facility of locomotion, (3) by the increasing tenderness for human life and suffering. And the only way of meeting the difficulty seems to be by modern nations subdividing themselves into small bodies having local knowledge and acting together in the spirit of ancient communities (compare Arist. Pol.)

\par  V. Regarded as the framework of a polity the Laws are deemed by Plato to be a decline from the Republic, which is the dream of his earlier years. He nowhere imagines that he has reached a higher point of speculation. He is only descending to the level of human things, and he often returns to his original idea. For the guardians of the Republic, who were the elder citizens, and were all supposed to be philosophers, is now substituted a special body, who are to review and amend the laws, preserving the spirit of the legislator. These are the Nocturnal Council, who, although they are not specially trained in dialectic, are not wholly destitute of it; for they must know the relation of particular virtues to the general principle of virtue. Plato has been arguing throughout the Laws that temperance is higher than courage, peace than war, that the love of both must enter into the character of the good citizen. And at the end the same thought is summed up by him in an abstract form. The true artist or guardian must be able to reduce the many to the one, than which, as he says with an enthusiasm worthy of the Phaedrus or Philebus, 'no more philosophical method was ever devised by the wit of man.' But the sense of unity in difference can only be acquired by study; and Plato does not explain to us the nature of this study, which we may reasonably infer, though there is a remarkable omission of the word, to be akin to the dialectic of the Republic.

\par  The Nocturnal Council is to consist of the priests who have obtained the rewards of virtue, of the ten eldest guardians of the law, and of the director and ex-directors of education; each of whom is to select for approval a younger coadjutor. To this council the 'Spectator,' who is sent to visit foreign countries, has to make his report. It is not an administrative body, but an assembly of sages who are to make legislation their study. Plato is not altogether disinclined to changes in the law where experience shows them to be necessary; but he is also anxious that the original spirit of the constitution should never be lost sight of.

\par  The Laws of Plato contain the latest phase of his philosophy, showing in many respects an advance, and in others a decline, in his views of life and the world. His Theory of Ideas in the next generation passed into one of Numbers, the nature of which we gather chiefly from the Metaphysics of Aristotle. Of the speculative side of this theory there are no traces in the Laws, but doubtless Plato found the practical value which he attributed to arithmetic greatly confirmed by the possibility of applying number and measure to the revolution of the heavens, and to the regulation of human life. In the return to a doctrine of numbers there is a retrogression rather than an advance; for the most barren logical abstraction is of a higher nature than number and figure. Philosophy fades away into the distance; in the Laws it is confined to the members of the Nocturnal Council. The speculative truth which was the food of the guardians in the Republic, is for the majority of the citizens to be superseded by practical virtues. The law, which is the expression of mind written down, takes the place of the living word of the philosopher. (Compare the contrast of Phaedrus, and Laws; also the plays on the words nous, nomos, nou dianome; and the discussion in the Statesman of the difference between the personal rule of a king and the impersonal reign of law.) The State is based on virtue and religion rather than on knowledge; and virtue is no longer identified with knowledge, being of the commoner sort, and spoken of in the sense generally understood. Yet there are many traces of advance as well as retrogression in the Laws of Plato. The attempt to reconcile the ideal with actual life is an advance; to 'have brought philosophy down from heaven to earth,' is a praise which may be claimed for him as well as for his master Socrates. And the members of the Nocturnal Council are to continue students of the 'one in many' and of the nature of God. Education is the last word with which Plato supposes the theory of the Laws to end and the reality to begin.

\par  Plato's increasing appreciation of the difficulties of human affairs, and of the element of chance which so largely influences them, is an indication not of a narrower, but of a maturer mind, which had become more conversant with realities. Nor can we fairly attribute any want of originality to him, because he has borrowed many of his provisions from Sparta and Athens. Laws and institutions grow out of habits and customs; and they have 'better opinion, better confirmation,' if they have come down from antiquity and are not mere literary inventions. Plato would have been the first to acknowledge that the Book of Laws was not the creation of his fancy, but a collection of enactments which had been devised by inspired legislators, like Minos, Lycurgus, and Solon, to meet the actual needs of men, and had been approved by time and experience.

\par  In order to do justice therefore to the design of the work, it is necessary to examine how far it rests on an historical foundation and coincides with the actual laws of Sparta and Athens. The consideration of the historical aspect of the Laws has been reserved for this place. In working out the comparison the writer has been greatly assisted by the excellent essays of C.F. Hermann ('De vestigiis institutorum veterum, imprimis Atticorum, per Platonis de Legibus libros indagandis,' and 'Juris domestici et familiaris apud Platonem in Legibus cum veteris Graeciae inque primis Athenarum institutis comparatio': Marburg, 1836), and by J.B. Telfy's 'Corpus Juris Attici' (Leipzig, 1868).

\par 
\section{
      EXCURSUS ON THE RELATION OF THE LAWS OF PLATO TO THE INSTITUTIONS OF CRETE
      AND LACEDAEMON AND TO THE LAWS AND CONSTITUTION OF ATHENS.
    }
\par  The Laws of Plato are essentially Greek: unlike Xenophon's Cyropaedia, they contain nothing foreign or oriental. Their aim is to reconstruct the work of the great lawgivers of Hellas in a literary form. They partake both of an Athenian and a Spartan character. Some of them too are derived from Crete, and are appropriately transferred to a Cretan colony. But of Crete so little is known to us, that although, as Montesquieu (Esprit des Lois) remarks, 'the Laws of Crete are the original of those of Sparta and the Laws of Plato the correction of these latter,' there is only one point, viz. the common meals, in which they can be compared. Most of Plato's provisions resemble the laws and customs which prevailed in these three states (especially in the two former), and which the personifying instinct of the Greeks attributed to Minos, Lycurgus, and Solon. A very few particulars may have been borrowed from Zaleucus (Cic. de Legibus), and Charondas, who is said to have first made laws against perjury (Arist. Pol.) and to have forbidden credit (Stob. Florileg., Gaisford). Some enactments are Plato's own, and were suggested by his experience of defects in the Athenian and other Greek states. The Laws also contain many lesser provisions, which are not found in the ordinary codes of nations, because they cannot be properly defined, and are therefore better left to custom and common sense. 'The greater part of the work,' as Aristotle remarks (Pol. ), 'is taken up with laws': yet this is not wholly true, and applies to the latter rather than to the first half of it. The book rests on an ethical and religious foundation: the actual laws begin with a hymn of praise in honour of the soul. And the same lofty aspiration after the good is perpetually recurring, especially in Books X, XI, XII, and whenever Plato's mind is filled with his highest themes. In prefixing to most of his laws a prooemium he has two ends in view, to persuade and also to threaten. They are to have the sanction of laws and the effect of sermons. And Plato's 'Book of Laws,' if described in the language of modern philosophy, may be said to be as much an ethical and educational, as a political or legal treatise.

\par  But although the Laws partake both of an Athenian and a Spartan character, the elements which are borrowed from either state are necessarily very different, because the character and origin of the two governments themselves differed so widely. Sparta was the more ancient and primitive: Athens was suited to the wants of a later stage of society. The relation of the two states to the Laws may be conceived in this manner:—The foundation and ground-plan of the work are more Spartan, while the superstructure and details are more Athenian. At Athens the laws were written down and were voluminous; more than a thousand fragments of them have been collected by Telfy. Like the Roman or English law, they contained innumerable particulars. Those of them which regulated daily life were familiarly known to the Athenians; for every citizen was his own lawyer, and also a judge, who decided the rights of his fellow-citizens according to the laws, often after hearing speeches from the parties interested or from their advocates. It is to Rome and not to Athens that the invention of law, in the modern sense of the term, is commonly ascribed. But it must be remembered that long before the times of the Twelve Tables (B.C. 451), regular courts and forms of law had existed at Athens and probably in the Greek colonies. And we may reasonably suppose, though without any express proof of the fact, that many Roman institutions and customs, like Latin literature and mythology, were partly derived from Hellas and had imperceptibly drifted from one shore of the Ionian Sea to the other (compare especially the constitutions of Servius Tullius and of Solon).

\par  It is not proved that the laws of Sparta were in ancient times either written down in books or engraved on tablets of marble or brass. Nor is it certain that, if they had been, the Spartans could have read them. They were ancient customs, some of them older probably than the settlement in Laconia, of which the origin is unknown; they occasionally received the sanction of the Delphic oracle, but there was a still stronger obligation by which they were enforced,—the necessity of self-defence: the Spartans were always living in the presence of their enemies. They belonged to an age when written law had not yet taken the place of custom and tradition. The old constitution was very rarely affected by new enactments, and these only related to the duties of the Kings or Ephors, or the new relations of classes which arose as time went on. Hence there was as great a difference as could well be conceived between the Laws of Athens and Sparta: the one was the creation of a civilized state, and did not differ in principle from our modern legislation, the other of an age in which the people were held together and also kept down by force of arms, and which afterwards retained many traces of its barbaric origin 'surviving in culture.'

\par  Nevertheless the Lacedaemonian was the ideal of a primitive Greek state. According to Thucydides it was the first which emerged out of confusion and became a regular government. It was also an army devoted to military exercises, but organized with a view to self-defence and not to conquest. It was not quick to move or easily excited; but stolid, cautious, unambitious, procrastinating. For many centuries it retained the same character which was impressed upon it by the hand of the legislator. This singular fabric was partly the result of circumstances, partly the invention of some unknown individual in prehistoric times, whose ideal of education was military discipline, and who, by the ascendency of his genius, made a small tribe into a nation which became famous in the world's history. The other Hellenes wondered at the strength and stability of his work. The rest of Hellas, says Thucydides, undertook the colonisation of Heraclea the more readily, having a feeling of security now that they saw the Lacedaemonians taking part in it. The Spartan state appears to us in the dawn of history as a vision of armed men, irresistible by any other power then existing in the world. It can hardly be said to have understood at all the rights or duties of nations to one another, or indeed to have had any moral principle except patriotism and obedience to commanders. Men were so trained to act together that they lost the freedom and spontaneity of human life in cultivating the qualities of the soldier and ruler. The Spartan state was a composite body in which kings, nobles, citizens, perioeci, artisans, slaves, had to find a 'modus vivendi' with one another. All of them were taught some use of arms. The strength of the family tie was diminished among them by an enforced absence from home and by common meals. Sparta had no life or growth; no poetry or tradition of the past; no art, no thought. The Athenians started on their great career some centuries later, but the Spartans would have been easily conquered by them, if Athens had not been deficient in the qualities which constituted the strength (and also the weakness) of her rival.

\par  The ideal of Athens has been pictured for all time in the speech which Thucydides puts into the mouth of Pericles, called the Funeral Oration. He contrasts the activity and freedom and pleasantness of Athenian life with the immobility and severe looks and incessant drill of the Spartans. The citizens of no city were more versatile, or more readily changed from land to sea or more quickly moved about from place to place. They 'took their pleasures' merrily, and yet, when the time for fighting arrived, were not a whit behind the Spartans, who were like men living in a camp, and, though always keeping guard, were often too late for the fray. Any foreigner might visit Athens; her ships found a way to the most distant shores; the riches of the whole earth poured in upon her. Her citizens had their theatres and festivals; they 'provided their souls with many relaxations'; yet they were not less manly than the Spartans or less willing to sacrifice this enjoyable existence for their country's good. The Athenian was a nobler form of life than that of their rivals, a life of music as well as of gymnastic, the life of a citizen as well as of a soldier. Such is the picture which Thucydides has drawn of the Athenians in their glory. It is the spirit of this life which Plato would infuse into the Magnesian state and which he seeks to combine with the common meals and gymnastic discipline of Sparta.

\par  The two great types of Athens and Sparta had deeply entered into his mind. He had heard of Sparta at a distance and from common Hellenic fame: he was a citizen of Athens and an Athenian of noble birth. He must often have sat in the law-courts, and may have had personal experience of the duties of offices such as he is establishing. There is no need to ask the question, whence he derived his knowledge of the Laws of Athens: they were a part of his daily life. Many of his enactments are recognized to be Athenian laws from the fragments preserved in the Orators and elsewhere: many more would be found to be so if we had better information. Probably also still more of them would have been incorporated in the Magnesian code, if the work had ever been finally completed. But it seems to have come down to us in a form which is partly finished and partly unfinished, having a beginning and end, but wanting arrangement in the middle. The Laws answer to Plato's own description of them, in the comparison which he makes of himself and his two friends to gatherers of stones or the beginners of some composite work, 'who are providing materials and partly putting them together:—having some of their laws, like stones, already fixed in their places, while others lie about.'

\par  Plato's own life coincided with the period at which Athens rose to her greatest heights and sank to her lowest depths. It was impossible that he should regard the blessings of democracy in the same light as the men of a former generation, whose view was not intercepted by the evil shadow of the taking of Athens, and who had only the glories of Marathon and Salamis and the administration of Pericles to look back upon. On the other hand the fame and prestige of Sparta, which had outlived so many crimes and blunders, was not altogether lost at the end of the life of Plato. Hers was the only great Hellenic government which preserved something of its ancient form; and although the Spartan citizens were reduced to almost one-tenth of their original number (Arist. Pol. ), she still retained, until the rise of Thebes and Macedon, a certain authority and predominance due to her final success in the struggle with Athens and to the victories which Agesilaus won in Asia Minor.

\par  Plato, like Aristotle, had in his mind some form of a mean state which should escape the evils and secure the advantages of both aristocracy and democracy. It may however be doubted whether the creation of such a state is not beyond the legislator's art, although there have been examples in history of forms of government, which through some community of interest or of origin, through a balance of parties in the state itself, or through the fear of a common enemy, have for a while preserved such a character of moderation. But in general there arises a time in the history of a state when the struggle between the few and the many has to be fought out. No system of checks and balances, such as Plato has devised in the Laws, could have given equipoise and stability to an ancient state, any more than the skill of the legislator could have withstood the tide of democracy in England or France during the last hundred years, or have given life to China or India.

\par  The basis of the Magnesian constitution is the equal division of land. In the new state, as in the Republic, there was to be neither poverty nor riches. Every citizen under all circumstances retained his lot, and as much money as was necessary for the cultivation of it, and no one was allowed to accumulate property to the amount of more than five times the value of the lot, inclusive of it. The equal division of land was a Spartan institution, not known to have existed elsewhere in Hellas. The mention of it in the Laws of Plato affords considerable presumption that it was of ancient origin, and not first introduced, as Mr. Grote and others have imagined, in the reformation of Cleomenes III. But at Sparta, if we may judge from the frequent complaints of the accumulation of property in the hands of a few persons (Arist. Pol. ), no provision could have been made for the maintenance of the lot. Plutarch indeed speaks of a law introduced by the Ephor Epitadeus soon after the Peloponnesian War, which first allowed the Spartans to sell their land (Agis): but from the manner in which Aristotle refers to the subject, we should imagine this evil in the state to be of a much older standing. Like some other countries in which small proprietors have been numerous, the original equality passed into inequality, and, instead of a large middle class, there was probably at Sparta greater disproportion in the property of the citizens than in any other state of Hellas. Plato was aware of the danger, and has improved on the Spartan custom. The land, as at Sparta, must have been tilled by slaves, since other occupations were found for the citizens. Bodies of young men between the ages of twenty-five and thirty were engaged in making biennial peregrinations of the country. They and their officers are to be the magistrates, police, engineers, aediles, of the twelve districts into which the colony was divided. Their way of life may be compared with that of the Spartan secret police or Crypteia, a name which Plato freely applies to them without apparently any consciousness of the odium which has attached to the word in history.

\par  Another great institution which Plato borrowed from Sparta (or Crete) is the Syssitia or common meals. These were established in both states, and in some respects were considered by Aristotle to be better managed in Crete than at Lacedaemon (Pol.). In the Laws the Cretan custom appears to be adopted (This is not proved, as Hermann supposes ('De Vestigiis,' etc. )): that is to say, if we may interpret Plato by Aristotle, the cost of them was defrayed by the state and not by the individuals (Arist. Pol); so that the members of the mess, who could not pay their quota, still retained their rights of citizenship. But this explanation is hardly consistent with the Laws, where contributions to the Syssitia from private estates are expressly mentioned. Plato goes further than the legislators of Sparta and Crete, and would extend the common meals to women as well as men: he desires to curb the disorders, which existed among the female sex in both states, by the application to women of the same military discipline to which the men were already subject. It was an extension of the custom of Syssitia from which the ancient legislators shrank, and which Plato himself believed to be very difficult of enforcement.

\par  Like Sparta, the new colony was not to be surrounded by walls,—a state should learn to depend upon the bravery of its citizens only—a fallacy or paradox, if it is not to be regarded as a poetical fancy, which is fairly enough ridiculed by Aristotle (Pol.). Women, too, must be ready to assist in the defence of their country: they are not to rush to the temples and altars, but to arm themselves with shield and spear. In the regulation of the Syssitia, in at least one of his enactments respecting property, and in the attempt to correct the licence of women, Plato shows, that while he borrowed from the institutions of Sparta and favoured the Spartan mode of life, he also sought to improve upon them.

\par  The enmity to the sea is another Spartan feature which is transferred by Plato to the Magnesian state. He did not reflect that a non-maritime power would always be at the mercy of one which had a command of the great highway. Their many island homes, the vast extent of coast which had to be protected by them, their struggles first of all with the Phoenicians and Carthaginians, and secondly with the Persian fleets, forced the Greeks, mostly against their will, to devote themselves to the sea. The islanders before the inhabitants of the continent, the maritime cities before the inland, the Corinthians and Athenians before the Spartans, were compelled to fit out ships: last of all the Spartans, by the pressure of the Peloponnesian War, were driven to establish a naval force, which, after the battle of Aegospotami, for more than a generation commanded the Aegean. Plato, like the Spartans, had a prejudice against a navy, because he regarded it as the nursery of democracy. But he either never considered, or did not care to explain, how a city, set upon an island and 'distant not more than ten miles from the sea, having a seaboard provided with excellent harbours,' could have safely subsisted without one.

\par  Neither the Spartans nor the Magnesian colonists were permitted to engage in trade or commerce. In order to limit their dealings as far as possible to their own country, they had a separate coinage; the Magnesians were only allowed to use the common currency of Hellas when they travelled abroad, which they were forbidden to do unless they received permission from the government. Like the Spartans, Plato was afraid of the evils which might be introduced into his state by intercourse with foreigners; but he also shrinks from the utter exclusiveness of Sparta, and is not unwilling to allow visitors of a suitable age and rank to come from other states to his own, as he also allows citizens of his own state to go to foreign countries and bring back a report of them. Such international communication seemed to him both honourable and useful.

\par  We may now notice some points in which the commonwealth of the Laws approximates to the Athenian model. These are much more numerous than the previous class of resemblances; we are better able to compare the laws of Plato with those of Athens, because a good deal more is known to us of Athens than of Sparta.

\par  The information which we possess about Athenian law, though comparatively fuller, is still fragmentary. The sources from which our knowledge is derived are chiefly the following:—

\par  (1) The Orators,—Antiphon, Andocides, Lysias, Isocrates, Demosthenes, Aeschines, Lycurgus, and others.

\par  (2) Herodotus, Thucydides, Xenophon, Plato, Aristotle, as well as later writers, such as Cicero de Legibus, Plutarch, Aelian, Pausanias.

\par  (3) Lexicographers, such as Harpocration, Pollux, Hesychius, Suidas, and the compiler of the Etymologicum Magnum, many of whom are of uncertain date, and to a great extent based upon one another. Their writings extend altogether over more than eight hundred years, from the second to the tenth century.

\par  (4) The Scholia on Aristophanes, Plato, Demosthenes.

\par  (5) A few inscriptions.

\par  Our knowledge of a subject derived from such various sources and for the most part of uncertain date and origin, is necessarily precarious. No critic can separate the actual laws of Solon from those which passed under his name in later ages. Nor do the Scholiasts and Lexicographers attempt to distinguish how many of these laws were still in force at the time when they wrote, or when they fell into disuse and were to be found in books only. Nor can we hastily assume that enactments which occur in the Laws of Plato were also a part of Athenian law, however probable this may appear.

\par  There are two classes of similarities between Plato's Laws and those of Athens: (i) of institutions (ii) of minor enactments.

\par  (i) The constitution of the Laws in its general character resembles much more nearly the Athenian constitution of Solon's time than that which succeeded it, or the extreme democracy which prevailed in Plato's own day. It was a mean state which he hoped to create, equally unlike a Syracusan tyranny or the mob-government of the Athenian assembly. There are various expedients by which he sought to impart to it the quality of moderation. (1) The whole people were to be educated: they could not be all trained in philosophy, but they were to acquire the simple elements of music, arithmetic, geometry, astronomy; they were also to be subject to military discipline, archontes kai archomenoi. (2) The majority of them were, or had been at some time in their lives, magistrates, and had the experience which is given by office. (3) The persons who held the highest offices were to have a further education, not much inferior to that provided for the guardians in the Republic, though the range of their studies is narrowed to the nature and divisions of virtue: here their philosophy comes to an end. (4) The entire number of the citizens (5040) rarely, if ever, assembled, except for purposes of elections. The whole people were divided into four classes, each having the right to be represented by the same number of members in the Council. The result of such an arrangement would be, as in the constitution of Servius Tullius, to give a disproportionate share of power to the wealthier classes, who may be supposed to be always much fewer in number than the poorer. This tendency was qualified by the complicated system of selection by vote, previous to the final election by lot, of which the object seems to be to hand over to the wealthy few the power of selecting from the many poor, and vice versa. (5) The most important body in the state was the Nocturnal Council, which is borrowed from the Areopagus at Athens, as it existed, or was supposed to have existed, in the days before Ephialtes and the Eumenides of Aeschylus, when its power was undiminished. In some particulars Plato appears to have copied exactly the customs and procedure of the Areopagus: both assemblies sat at night (Telfy). There was a resemblance also in more important matters. Like the Areopagus, the Nocturnal Council was partly composed of magistrates and other state officials, whose term of office had expired. (7) The constitution included several diverse and even opposing elements, such as the Assembly and the Nocturnal Council. (8) There was much less exclusiveness than at Sparta; the citizens were to have an interest in the government of neighbouring states, and to know what was going on in the rest of the world.—All these were moderating influences.

\par  A striking similarity between Athens and the constitution of the Magnesian colony is the use of the lot in the election of judges and other magistrates. That such a mode of election should have been resorted to in any civilized state, or that it should have been transferred by Plato to an ideal or imaginary one, is very singular to us. The most extreme democracy of modern times has never thought of leaving government wholly to chance. It was natural that Socrates should scoff at it, and ask, 'Who would choose a pilot or carpenter or flute-player by lot' (Xen. Mem.)? Yet there were many considerations which made this mode of choice attractive both to the oligarch and to the democrat:—(1) It seemed to recognize that one man was as good as another, and that all the members of the governing body, whether few or many, were on a perfect equality in every sense of the word. (2) To the pious mind it appeared to be a choice made, not by man, but by heaven (compare Laws). (3) It afforded a protection against corruption and intrigue...It must also be remembered that, although elected by lot, the persons so elected were subject to a scrutiny before they entered on their office, and were therefore liable, after election, if disqualified, to be rejected (Laws). They were, moreover, liable to be called to account after the expiration of their office. In the election of councillors Plato introduces a further check: they are not to be chosen directly by lot from all the citizens, but from a select body previously elected by vote. In Plato's state at least, as we may infer from his silence on this point, judges and magistrates performed their duties without pay, which was a guarantee both of their disinterestedness and of their belonging probably to the higher class of citizens (compare Arist. Pol.). Hence we are not surprised that the use of the lot prevailed, not only in the election of the Athenian Council, but also in many oligarchies, and even in Plato's colony. The evil consequences of the lot are to a great extent avoided, if the magistrates so elected do not, like the dicasts at Athens, receive pay from the state.

\par  Another parallel is that of the Popular Assembly, which at Athens was omnipotent, but in the Laws has only a faded and secondary existence. In Plato it was chiefly an elective body, having apparently no judicial and little political power entrusted to it. At Athens it was the mainspring of the democracy; it had the decision of war or peace, of life and death; the acts of generals or statesmen were authorized or condemned by it; no office or person was above its control. Plato was far from allowing such a despotic power to exist in his model community, and therefore he minimizes the importance of the Assembly and narrows its functions. He probably never asked himself a question, which naturally occurs to the modern reader, where was to be the central authority in this new community, and by what supreme power would the differences of inferior powers be decided. At the same time he magnifies and brings into prominence the Nocturnal Council (which is in many respects a reflection of the Areopagus), but does not make it the governing body of the state.

\par  Between the judicial system of the Laws and that of Athens there was very great similarity, and a difference almost equally great. Plato not unfrequently adopts the details when he rejects the principle. At Athens any citizen might be a judge and member of the great court of the Heliaea. This was ordinarily subdivided into a number of inferior courts, but an occasion is recorded on which the whole body, in number six thousand, met in a single court (Andoc. de Myst.). Plato significantly remarks that a few judges, if they are good, are better than a great number. He also, at least in capital cases, confines the plaintiff and defendant to a single speech each, instead of allowing two apiece, as was the common practice at Athens. On the other hand, in all private suits he gives two appeals, from the arbiters to the courts of the tribes, and from the courts of the tribes to the final or supreme court. There was nothing answering to this at Athens. The three courts were appointed in the following manner:—the arbiters were to be agreed upon by the parties to the cause; the judges of the tribes to be elected by lot; the highest tribunal to be chosen at the end of each year by the great officers of state out of their own number—they were to serve for a year, to undergo a scrutiny, and, unlike the Athenian judges, to vote openly. Plato does not dwell upon methods of procedure: these are the lesser matters which he leaves to the younger legislators. In cases of murder and some other capital offences, the cause was to be tried by a special tribunal, as was the custom at Athens: military offences, too, as at Athens, were decided by the soldiers. Public causes in the Laws, as sometimes at Athens, were voted upon by the whole people: because, as Plato remarks, they are all equally concerned in them. They were to be previously investigated by three of the principal magistrates. He believes also that in private suits all should take part; 'for he who has no share in the administration of justice is apt to imagine that he has no share in the state at all.' The wardens of the country, like the Forty at Athens, also exercised judicial power in small matters, as well as the wardens of the agora and city. The department of justice is better organized in Plato than in an ordinary Greek state, proceeding more by regular methods, and being more restricted to distinct duties.

\par  The executive of Plato's Laws, like the Athenian, was different from that of a modern civilized state. The difference chiefly consists in this, that whereas among ourselves there are certain persons or classes of persons set apart for the execution of the duties of government, in ancient Greece, as in all other communities in the earlier stages of their development, they were not equally distinguished from the rest of the citizens. The machinery of government was never so well organized as in the best modern states. The judicial department was not so completely separated from the legislative, nor the executive from the judicial, nor the people at large from the professional soldier, lawyer, or priest. To Aristotle (Pol.) it was a question requiring serious consideration—Who should execute a sentence? There was probably no body of police to whom were entrusted the lives and properties of the citizens in any Hellenic state. Hence it might be reasonably expected that every man should be the watchman of every other, and in turn be watched by him. The ancients do not seem to have remembered the homely adage that, 'What is every man's business is no man's business,' or always to have thought of applying the principle of a division of labour to the administration of law and to government. Every Athenian was at some time or on some occasion in his life a magistrate, judge, advocate, soldier, sailor, policeman. He had not necessarily any private business; a good deal of his time was taken up with the duties of office and other public occupations. So, too, in Plato's Laws. A citizen was to interfere in a quarrel, if older than the combatants, or to defend the outraged party, if his junior. He was especially bound to come to the rescue of a parent who was ill-treated by his children. He was also required to prosecute the murderer of a kinsman. In certain cases he was allowed to arrest an offender. He might even use violence to an abusive person. Any citizen who was not less than thirty years of age at times exercised a magisterial authority, to be enforced even by blows. Both in the Magnesian state and at Athens many thousand persons must have shared in the highest duties of government, if a section only of the Council, consisting of thirty or of fifty persons, as in the Laws, or at Athens after the days of Cleisthenes, held office for a month, or for thirty-five days only. It was almost as if, in our own country, the Ministry or the Houses of Parliament were to change every month. The average ability of the Athenian and Magnesian councillors could not have been very high, considering there were so many of them. And yet they were entrusted with the performance of the most important executive duties. In these respects the constitution of the Laws resembles Athens far more than Sparta. All the citizens were to be, not merely soldiers, but politicians and administrators.

\par  (ii) There are numerous minor particulars in which the Laws of Plato resemble those of Athens. These are less interesting than the preceding, but they show even more strikingly how closely in the composition of his work Plato has followed the laws and customs of his own country.

\par  (1) Evidence. (a) At Athens a child was not allowed to give evidence (Telfy). Plato has a similar law: 'A child shall be allowed to give evidence only in cases of murder.' (b) At Athens an unwilling witness might be summoned; but he was not required to appear if he was ready to declare on oath that he knew nothing about the matter in question (Telfy). So in the Laws. (c) Athenian law enacted that when more than half the witnesses in a case had been convicted of perjury, there was to be a new trial (anadikos krisis—Telfy). There is a similar provision in the Laws. (d) False-witness was punished at Athens by atimia and a fine (Telfy). Plato is at once more lenient and more severe: 'If a man be twice convicted of false-witness, he shall not be required, and if thrice, he shall not be allowed to bear witness; and if he dare to witness after he has been convicted three times,...he shall be punished with death.'

\par  (2) Murder. (a) Wilful murder was punished in Athenian law by death, perpetual exile, and confiscation of property (Telfy). Plato, too, has the alternative of death or exile, but he does not confiscate the murderer's property. (b) The Parricide was not allowed to escape by going into exile at Athens (Telfy), nor, apparently, in the Laws. (c) A homicide, if forgiven by his victim before death, received no punishment, either at Athens (Telfy), or in the Magnesian state. In both (Telfy) the contriver of a murder is punished as severely as the doer; and persons accused of the crime are forbidden to enter temples or the agora until they have been tried (Telfy). (d) At Athens slaves who killed their masters and were caught red-handed, were not to be put to death by the relations of the murdered man, but to be handed over to the magistrates (Telfy). So in the Laws, the slave who is guilty of wilful murder has a public execution: but if the murder is committed in anger, it is punished by the kinsmen of the victim.

\par  (3) Involuntary homicide. (a) The guilty person, according to the Athenian law, had to go into exile, and might not return, until the family of the man slain were conciliated. Then he must be purified (Telfy). If he is caught before he has obtained forgiveness, he may be put to death. These enactments reappear in the Laws. (b) The curious provision of Plato, that a stranger who has been banished for involuntary homicide and is subsequently wrecked upon the coast, must 'take up his abode on the sea-shore, wetting his feet in the sea, and watching for an opportunity of sailing,' recalls the procedure of the Judicium Phreatteum at Athens, according to which an involuntary homicide, who, having gone into exile, is accused of a wilful murder, was tried at Phreatto for this offence in a boat by magistrates on the shore. (c) A still more singular law, occurring both in the Athenian and Magnesian code, enacts that a stone or other inanimate object which kills a man is to be tried, and cast over the border (Telfy).

\par  (4) Justifiable or excusable homicide. Plato and Athenian law agree in making homicide justifiable or excusable in the following cases:—(1) at the games (Telfy); (2) in war (Telfy); (3) if the person slain was found doing violence to a free woman (Telfy); (4) if a doctor's patient dies; (5) in the case of a robber (Telfy); (6) in self-defence (Telfy).

\par  (5) Impiety. Death or expulsion was the Athenian penalty for impiety (Telfy). In the Laws it is punished in various cases by imprisonment for five years, for life, and by death.

\par  (6) Sacrilege. Robbery of temples at Athens was punished by death, refusal of burial in the land, and confiscation of property (Telfy). In the Laws the citizen who is guilty of such a crime is to 'perish ingloriously and be cast beyond the borders of the land,' but his property is not confiscated.

\par  (7) Sorcery. The sorcerer at Athens was to be executed (Telfy): compare Laws, where it is enacted that the physician who poisons and the professional sorcerer shall be punished with death.

\par  (8) Treason. Both at Athens and in the Laws the penalty for treason was death (Telfy), and refusal of burial in the country (Telfy).

\par  (9) Sheltering exiles. 'If a man receives an exile, he shall be punished with death.' So, too, in Athenian law (Telfy. ).

\par  (10) Wounding. Athenian law compelled a man who had wounded another to go into exile; if he returned, he was to be put to death (Telfy). Plato only punishes the offence with death when children wound their parents or one another, or a slave wounds his master.

\par  (11) Bribery. Death was the punishment for taking a bribe, both at Athens (Telfy) and in the Laws; but Athenian law offered an alternative—the payment of a fine of ten times the amount of the bribe.

\par  (12) Theft. Plato, like Athenian law (Telfy), punishes the theft of public property by death; the theft of private property in both involves a fine of double the value of the stolen goods (Telfy).

\par  (13) Suicide. He 'who slays him who of all men, as they say, is his own best friend,' is regarded in the same spirit by Plato and by Athenian law. Plato would have him 'buried ingloriously on the borders of the twelve portions of the land, in such places as are uncultivated and nameless,' and 'no column or inscription is to mark the place of his interment.' Athenian law enacted that the hand which did the deed should be separated from the body and be buried apart (Telfy).

\par  (14) Injury. In cases of wilful injury, Athenian law compelled the guilty person to pay double the damage; in cases of involuntary injury, simple damages (Telfy). Plato enacts that if a man wounds another in passion, and the wound is curable, he shall pay double the damage, if incurable or disfiguring, fourfold damages. If, however, the wounding is accidental, he shall simply pay for the harm done.

\par  (15) Treatment of parents. Athenian law allowed any one to indict another for neglect or illtreatment of parents (Telfy). So Plato bids bystanders assist a father who is assaulted by his son, and allows any one to give information against children who neglect their parents.

\par  (16) Execution of sentences. Both Plato and Athenian law give to the winner of a suit power to seize the goods of the loser, if he does not pay within the appointed time (Telfy). At Athens the penalty was also doubled (Telfy); not so in Plato. Plato however punishes contempt of court by death, which at Athens seems only to have been visited with a further fine (Telfy).

\par  (17) Property. (a) Both at Athens and in the Laws a man who has disputed property in his possession must give the name of the person from whom he received it (Telfy); and any one searching for lost property must enter a house naked (Telfy), or, as Plato says, 'naked, or wearing only a short tunic and without a girdle. (b) Athenian law, as well as Plato, did not allow a father to disinherit his son without good reason and the consent of impartial persons (Telfy). Neither grants to the eldest son any special claim on the paternal estate (Telfy). In the law of inheritance both prefer males to females (Telfy). (c) Plato and Athenian law enacted that a tree should be planted at a fair distance from a neighbour's property (Telfy), and that when a man could not get water, his neighbour must supply him (Telfy). Both at Athens and in Plato there is a law about bees, the former providing that a beehive must be set up at not less a distance than 300 feet from a neighbour's (Telfy), and the latter forbidding the decoying of bees.

\par  (18) Orphans. A ward must proceed against a guardian whom he suspects of fraud within five years of the expiration of the guardianship. This provision is common to Plato and to Athenian law (Telfy). Further, the latter enacted that the nearest male relation should marry or provide a husband for an heiress (Telfy),—a point in which Plato follows it closely.

\par  (19) Contracts. Plato's law that 'when a man makes an agreement which he does not fulfil, unless the agreement be of a nature which the law or a vote of the assembly does not allow, or which he has made under the influence of some unjust compulsion, or which he is prevented from fulfilling against his will by some unexpected chance,—the other party may go to law with him,' according to Pollux (quoted in Telfy's note) prevailed also at Athens.

\par  (20) Trade regulations. (a) Lying was forbidden in the agora both by Plato and at Athens (Telfy). (b) Athenian law allowed an action of recovery against a man who sold an unsound slave as sound (Telfy). Plato's enactment is more explicit: he allows only an unskilled person (i.e. one who is not a trainer or physician) to take proceedings in such a case. (c) Plato diverges from Athenian practice in the disapproval of credit, and does not even allow the supply of goods on the deposit of a percentage of their value (Telfy). He enacts that 'when goods are exchanged by buying and selling, a man shall deliver them and receive the price of them at a fixed place in the agora, and have done with the matter,' and that 'he who gives credit must be satisfied whether he obtain his money or not, for in such exchanges he will not be protected by law. (d) Athenian law forbad an extortionate rate of interest (Telfy); Plato allows interest in one case only—if a contractor does not receive the price of his work within a year of the time agreed—and at the rate of 200 per cent. per annum for every drachma a monthly interest of an obol. (e) Both at Athens and in the Laws sales were to be registered (Telfy), as well as births (Telfy).

\par  (21) Sumptuary laws. Extravagance at weddings (Telfy), and at funerals (Telfy) was forbidden at Athens and also in the Magnesian state.

\par  There remains the subject of family life, which in Plato's Laws partakes both of an Athenian and Spartan character. Under this head may conveniently be included the condition of women and of slaves. To family life may be added citizenship.

\par  As at Sparta, marriages are to be contracted for the good of the state; and they may be dissolved on the same ground, where there is a failure of issue,—the interest of the state requiring that every one of the 5040 lots should have an heir. Divorces are likewise permitted by Plato where there is an incompatibility of temper, as at Athens by mutual consent. The duty of having children is also enforced by a still higher motive, expressed by Plato in the noble words:—'A man should cling to immortality, and leave behind him children's children to be the servants of God in his place.' Again, as at Athens, the father is allowed to put away his undutiful son, but only with the consent of impartial persons (Telfy), and the only suit which may be brought by a son against a father is for imbecility. The class of elder and younger men and women are still to regard one another, as in the Republic, as standing in the relation of parents and children. This is a trait of Spartan character rather than of Athenian. A peculiar sanctity and tenderness was to be shown towards the aged; the parent or grandparent stricken with years was to be loved and worshipped like the image of a God, and was to be deemed far more able than any lifeless statue to bring good or ill to his descendants. Great care is to be taken of orphans: they are entrusted to the fifteen eldest Guardians of the Law, who are to be 'lawgivers and fathers to them not inferior to their natural fathers,' as at Athens they were entrusted to the Archons. Plato wishes to make the misfortune of orphanhood as little sad to them as possible.

\par  Plato, seeing the disorder into which half the human race had fallen at Athens and Sparta, is minded to frame for them a new rule of life. He renounces his fanciful theory of communism, but still desires to place women as far as possible on an equality with men. They were to be trained in the use of arms, they are to live in public. Their time was partly taken up with gymnastic exercises; there could have been little family or private life among them. Their lot was to be neither like that of Spartan women, who were made hard and common by excessive practice of gymnastic and the want of all other education,—nor yet like that of Athenian women, who, at least among the upper classes, retired into a sort of oriental seclusion,—but something better than either. They were to be the perfect mothers of perfect children, yet not wholly taken up with the duties of motherhood, which were to be made easy to them as far as possible (compare Republic), but able to share in the perils of war and to be the companions of their husbands. Here, more than anywhere else, the spirit of the Laws reverts to the Republic. In speaking of them as the companions of their husbands we must remember that it is an Athenian and not a Spartan way of life which they are invited to share, a life of gaiety and brightness, not of austerity and abstinence, which often by a reaction degenerated into licence and grossness.

\par  In Plato's age the subject of slavery greatly interested the minds of thoughtful men; and how best to manage this 'troublesome piece of goods' exercised his own mind a good deal. He admits that they have often been found better than brethren or sons in the hour of danger, and are capable of rendering important public services by informing against offenders—for this they are to be rewarded; and the master who puts a slave to death for the sake of concealing some crime which he has committed, is held guilty of murder. But they are not always treated with equal consideration. The punishments inflicted on them bear no proportion to their crimes. They are to be addressed only in the language of command. Their masters are not to jest with them, lest they should increase the hardship of their lot. Some privileges were granted to them by Athenian law of which there is no mention in Plato; they were allowed to purchase their freedom from their master, and if they despaired of being liberated by him they could demand to be sold, on the chance of falling into better hands. But there is no suggestion in the Laws that a slave who tried to escape should be branded with the words—kateche me, pheugo, or that evidence should be extracted from him by torture, that the whole household was to be executed if the master was murdered and the perpetrator remained undetected: all these were provisions of Athenian law. Plato is more consistent than either the Athenians or the Spartans; for at Sparta too the Helots were treated in a manner almost unintelligible to us. On the one hand, they had arms put into their hands, and served in the army, not only, as at Plataea, in attendance on their masters, but, after they had been manumitted, as a separate body of troops called Neodamodes: on the other hand, they were the victims of one of the greatest crimes recorded in Greek history (Thucyd.). The two great philosophers of Hellas sought to extricate themselves from this cruel condition of human life, but acquiesced in the necessity of it. A noble and pathetic sentiment of Plato, suggested by the thought of their misery, may be quoted in this place:—'The right treatment of slaves is to behave properly to them, and to do to them, if possible, even more justice than to those who are our equals; for he who naturally and genuinely reverences justice, and hates injustice, is discovered in his dealings with any class of men to whom he can easily be unjust. And he who in regard to the natures and actions of his slaves is undefiled by impiety and injustice, will best sow the seeds of virtue in them; and this may be truly said of every master, and tyrant, and of every other having authority in relation to his inferiors.'

\par  All the citizens of the Magnesian state were free and equal; there was no distinction of rank among them, such as is believed to have prevailed at Sparta. Their number was a fixed one, corresponding to the 5040 lots. One of the results of this is the requirement that younger sons or those who have been disinherited shall go out to a colony. At Athens, where there was not the same religious feeling against increasing the size of the city, the number of citizens must have been liable to considerable fluctuations. Several classes of persons, who were not citizens by birth, were admitted to the privilege. Perpetual exiles from other countries, people who settled there to practise a trade (Telfy), any one who had shown distinguished valour in the cause of Athens, the Plataeans who escaped from the siege, metics and strangers who offered to serve in the army, the slaves who fought at Arginusae,—all these could or did become citizens. Even those who were only on one side of Athenian parentage were at more than one period accounted citizens. But at times there seems to have arisen a feeling against this promiscuous extension of the citizen body, an expression of which is to be found in the law of Pericles—monous Athenaious einai tous ek duoin Athenaion gegonotas (Plutarch, Pericles); and at no time did the adopted citizen enjoy the full rights of citizenship—e.g. he might not be elected archon or to the office of priest (Telfy), although this prohibition did not extend to his children, if born of a citizen wife. Plato never thinks of making the metic, much less the slave, a citizen. His treatment of the former class is at once more gentle and more severe than that which prevailed at Athens. He imposes upon them no tax but good behaviour, whereas at Athens they were required to pay twelve drachmae per annum, and to have a patron: on the other hand, he only allows them to reside in the Magnesian state on condition of following a trade; they were required to depart when their property exceeded that of the third class, and in any case after a residence of twenty years, unless they could show that they had conferred some great benefit on the state. This privileged position reflects that of the isoteleis at Athens, who were excused from the metoikion. It is Plato's greatest concession to the metic, as the bestowal of freedom is his greatest concession to the slave.

\par  Lastly, there is a more general point of view under which the Laws of Plato may be considered,—the principles of Jurisprudence which are contained in them. These are not formally announced, but are scattered up and down, to be observed by the reflective reader for himself. Some of them are only the common principles which all courts of justice have gathered from experience; others are peculiar and characteristic. That judges should sit at fixed times and hear causes in a regular order, that evidence should be laid before them, that false witnesses should be disallowed, and corruption punished, that defendants should be heard before they are convicted,—these are the rules, not only of the Hellenic courts, but of courts of law in all ages and countries. But there are also points which are peculiar, and in which ancient jurisprudence differs considerably from modern; some of them are of great importance...It could not be said at Athens, nor was it ever contemplated by Plato, that all men, including metics and slaves, should be equal 'in the eye of the law.' There was some law for the slave, but not much; no adequate protection was given him against the cruelty of his master...It was a singular privilege granted, both by the Athenian and Magnesian law, to a murdered man, that he might, before he died, pardon his murderer, in which case no legal steps were afterwards to be taken against him. This law is the remnant of an age in which the punishment of offences against the person was the concern rather of the individual and his kinsmen than of the state...Plato's division of crimes into voluntary and involuntary and those done from passion, only partially agrees with the distinction which modern law has drawn between murder and manslaughter; his attempt to analyze them is confused by the Socratic paradox, that 'All vice is involuntary'...It is singular that both in the Laws and at Athens theft is commonly punished by a twofold restitution of the article stolen. The distinction between civil and criminal courts or suits was not yet recognized...Possession gives a right of property after a certain time...The religious aspect under which certain offences were regarded greatly interfered with a just and natural estimate of their guilt...As among ourselves, the intent to murder was distinguished by Plato from actual murder...We note that both in Plato and the laws of Athens, libel in the market-place and personality in the theatre were forbidden...Both in Plato and Athenian law, as in modern times, the accomplice of a crime is to be punished as well as the principal...Plato does not allow a witness in a cause to act as a judge of it...Oaths are not to be taken by the parties to a suit...Both at Athens and in Plato's Laws capital punishment for murder was not to be inflicted, if the offender was willing to go into exile...Respect for the dead, duty towards parents, are to be enforced by the law as well as by public opinion...Plato proclaims the noble sentiment that the object of all punishment is the improvement of the offender... Finally, he repeats twice over, as with the voice of a prophet, that the crimes of the fathers are not to be visited upon the children. In this respect he is nobly distinguished from the Oriental, and indeed from the spirit of Athenian law (compare Telfy,—dei kai autous kai tous ek touton atimous einai), as the Hebrew in the age of Ezekial is from the Jewish people of former ages.

\par  Of all Plato's provisions the object is to bring the practice of the law more into harmony with reason and philosophy; to secure impartiality, and while acknowledging that every citizen has a right to share in the administration of justice, to counteract the tendency of the courts to become mere popular assemblies.

\par  ...

\par  Thus we have arrived at the end of the writings of Plato, and at the last stage of philosophy which was really his. For in what followed, which we chiefly gather from the uncertain intimations of Aristotle, the spirit of the master no longer survived. The doctrine of Ideas passed into one of numbers; instead of advancing from the abstract to the concrete, the theories of Plato were taken out of their context, and either asserted or refuted with a provoking literalism; the Socratic or Platonic element in his teaching was absorbed into the Pythagorean or Megarian. His poetry was converted into mysticism; his unsubstantial visions were assailed secundum artem by the rules of logic. His political speculations lost their interest when the freedom of Hellas had passed away. Of all his writings the Laws were the furthest removed from the traditions of the Platonic school in the next generation. Both his political and his metaphysical philosophy are for the most part misinterpreted by Aristotle. The best of him—his love of truth, and his 'contemplation of all time and all existence,' was soonest lost; and some of his greatest thoughts have slept in the ear of mankind almost ever since they were first uttered.

\par  We have followed him during his forty or fifty years of authorship, from the beginning when he first attempted to depict the teaching of Socrates in a dramatic form, down to the time at which the character of Socrates had disappeared, and we have the latest reflections of Plato's own mind upon Hellas and upon philosophy. He, who was 'the last of the poets,' in his book of Laws writes prose only; he has himself partly fallen under the rhetorical influences which in his earlier dialogues he was combating. The progress of his writings is also the history of his life; we have no other authentic life of him. They are the true self of the philosopher, stripped of the accidents of time and place. The great effort which he makes is, first, to realize abstractions, secondly, to connect them. In the attempt to realize them, he was carried into a transcendental region in which he isolated them from experience, and we pass out of the range of science into poetry or fiction. The fancies of mythology for a time cast a veil over the gulf which divides phenomena from onta (Meno, Phaedrus, Symposium, Phaedo). In his return to earth Plato meets with a difficulty which has long ceased to be a difficulty to us. He cannot understand how these obstinate, unmanageable ideas, residing alone in their heaven of abstraction, can be either combined with one another, or adapted to phenomena (Parmenides, Philebus, Sophist). That which is the most familiar process of our own minds, to him appeared to be the crowning achievement of the dialectical art. The difficulty which in his own generation threatened to be the destruction of philosophy, he has rendered unmeaning and ridiculous. For by his conquests in the world of mind our thoughts are widened, and he has furnished us with new dialectical instruments which are of greater compass and power. We have endeavoured to see him as he truly was, a great original genius struggling with unequal conditions of knowledge, not prepared with a system nor evolving in a series of dialogues ideas which he had long conceived, but contradictory, enquiring as he goes along, following the argument, first from one point of view and then from another, and therefore arriving at opposite conclusions, hovering around the light, and sometimes dazzled with excess of light, but always moving in the same element of ideal truth. We have seen him also in his decline, when the wings of his imagination have begun to droop, but his experience of life remains, and he turns away from the contemplation of the eternal to take a last sad look at human affairs.

\par  ...

\par  And so having brought into the world 'noble children' (Phaedr. ), he rests from the labours of authorship. More than two thousand two hundred years have passed away since he returned to the place of Apollo and the Muses. Yet the echo of his words continues to be heard among men, because of all philosophers he has the most melodious voice. He is the inspired prophet or teacher who can never die, the only one in whom the outward form adequately represents the fair soul within; in whom the thoughts of all who went before him are reflected and of all who come after him are partly anticipated. Other teachers of philosophy are dried up and withered,—after a few centuries they have become dust; but he is fresh and blooming, and is always begetting new ideas in the minds of men. They are one-sided and abstract; but he has many sides of wisdom. Nor is he always consistent with himself, because he is always moving onward, and knows that there are many more things in philosophy than can be expressed in words, and that truth is greater than consistency. He who approaches him in the most reverent spirit shall reap most of the fruit of his wisdom; he who reads him by the light of ancient commentators will have the least understanding of him.

\par  We may see him with the eye of the mind in the groves of the Academy, or on the banks of the Ilissus, or in the streets of Athens, alone or walking with Socrates, full of those thoughts which have since become the common possession of mankind. Or we may compare him to a statue hid away in some temple of Zeus or Apollo, no longer existing on earth, a statue which has a look as of the God himself. Or we may once more imagine him following in another state of being the great company of heaven which he beheld of old in a vision (Phaedr.). So, 'partly trifling, but with a certain degree of seriousness' (Symp. ), we linger around the memory of a world which has passed away (Phaedr. ).

\par 
\section{
      LAWS
    }
\par 
\section{
      BOOK I.
    } 
\par \textbf{ATHENIAN}
\par   Tell me, Strangers, is a God or some man supposed to be the author of your laws?

\par \textbf{CLEINIAS}
\par   A God, Stranger; in very truth a God:  among us Cretans he is said to have been Zeus, but in Lacedaemon, whence our friend here comes, I believe they would say that Apollo is their lawgiver:  would they not, Megillus?

\par \textbf{MEGILLUS}
\par   Certainly.

\par \textbf{ATHENIAN}
\par   And do you, Cleinias, believe, as Homer tells, that every ninth year Minos went to converse with his Olympian sire, and was inspired by him to make laws for your cities?

\par \textbf{CLEINIAS}
\par   Yes, that is our tradition; and there was Rhadamanthus, a brother of his, with whose name you are familiar; he is reputed to have been the justest of men, and we Cretans are of opinion that he earned this reputation from his righteous administration of justice when he was alive.

\par \textbf{ATHENIAN}
\par   Yes, and a noble reputation it was, worthy of a son of Zeus. As you and Megillus have been trained in these institutions, I dare say that you will not be unwilling to give an account of your government and laws; on our way we can pass the time pleasantly in talking about them, for I am told that the distance from Cnosus to the cave and temple of Zeus is considerable; and doubtless there are shady places under the lofty trees, which will protect us from this scorching sun. Being no longer young, we may often stop to rest beneath them, and get over the whole journey without difficulty, beguiling the time by conversation.

\par \textbf{CLEINIAS}
\par   Yes, Stranger, and if we proceed onward we shall come to groves of cypresses, which are of rare height and beauty, and there are green meadows, in which we may repose and converse.

\par \textbf{ATHENIAN}
\par   Very good.

\par \textbf{CLEINIAS}
\par   Very good, indeed; and still better when we see them; let us move on cheerily.

\par \textbf{ATHENIAN}
\par   I am willing—And first, I want to know why the law has ordained that you shall have common meals and gymnastic exercises, and wear arms.

\par \textbf{CLEINIAS}
\par   I think, Stranger, that the aim of our institutions is easily intelligible to any one. Look at the character of our country:  Crete is not like Thessaly, a large plain; and for this reason they have horsemen in Thessaly, and we have runners—the inequality of the ground in our country is more adapted to locomotion on foot; but then, if you have runners you must have light arms—no one can carry a heavy weight when running, and bows and arrows are convenient because they are light. Now all these regulations have been made with a view to war, and the legislator appears to me to have looked to this in all his arrangements: —the common meals, if I am not mistaken, were instituted by him for a similar reason, because he saw that while they are in the field the citizens are by the nature of the case compelled to take their meals together for the sake of mutual protection. He seems to me to have thought the world foolish in not understanding that all men are always at war with one another; and if in war there ought to be common meals and certain persons regularly appointed under others to protect an army, they should be continued in peace. For what men in general term peace would be said by him to be only a name; in reality every city is in a natural state of war with every other, not indeed proclaimed by heralds, but everlasting. And if you look closely, you will find that this was the intention of the Cretan legislator; all institutions, private as well as public, were arranged by him with a view to war; in giving them he was under the impression that no possessions or institutions are of any value to him who is defeated in battle; for all the good things of the conquered pass into the hands of the conquerors.

\par \textbf{ATHENIAN}
\par   You appear to me, Stranger, to have been thoroughly trained in the Cretan institutions, and to be well informed about them; will you tell me a little more explicitly what is the principle of government which you would lay down? You seem to imagine that a well-governed state ought to be so ordered as to conquer all other states in war:  am I right in supposing this to be your meaning?

\par \textbf{CLEINIAS}
\par   Certainly; and our Lacedaemonian friend, if I am not mistaken, will agree with me.

\par \textbf{MEGILLUS}
\par   Why, my good friend, how could any Lacedaemonian say anything else?

\par \textbf{ATHENIAN}
\par   And is what you say applicable only to states, or also to villages?

\par \textbf{CLEINIAS}
\par   To both alike.

\par \textbf{ATHENIAN}
\par   The case is the same?

\par \textbf{CLEINIAS}
\par   Yes.

\par \textbf{ATHENIAN}
\par   And in the village will there be the same war of family against family, and of individual against individual?

\par \textbf{CLEINIAS}
\par   The same.

\par \textbf{ATHENIAN}
\par   And should each man conceive himself to be his own enemy: —what shall we say?

\par \textbf{CLEINIAS}
\par   O Athenian Stranger—inhabitant of Attica I will not call you, for you seem to deserve rather to be named after the goddess herself, because you go back to first principles,—you have thrown a light upon the argument, and will now be better able to understand what I was just saying,—that all men are publicly one another's enemies, and each man privately his own.

\par  (ATHENIAN: My good sir, what do you mean? )—

\par \textbf{CLEINIAS}
\par  ...Moreover, there is a victory and defeat—the first and best of victories, the lowest and worst of defeats—which each man gains or sustains at the hands, not of another, but of himself; this shows that there is a war against ourselves going on within every one of us.

\par \textbf{ATHENIAN}
\par   Let us now reverse the order of the argument:  Seeing that every individual is either his own superior or his own inferior, may we say that there is the same principle in the house, the village, and the state?

\par \textbf{CLEINIAS}
\par   You mean that in each of them there is a principle of superiority or inferiority to self?

\par \textbf{ATHENIAN}
\par   Yes.

\par \textbf{CLEINIAS}
\par   You are quite right in asking the question, for there certainly is such a principle, and above all in states; and the state in which the better citizens win a victory over the mob and over the inferior classes may be truly said to be better than itself, and may be justly praised, where such a victory is gained, or censured in the opposite case.

\par \textbf{ATHENIAN}
\par   Whether the better is ever really conquered by the worse, is a question which requires more discussion, and may be therefore left for the present. But I now quite understand your meaning when you say that citizens who are of the same race and live in the same cities may unjustly conspire, and having the superiority in numbers may overcome and enslave the few just; and when they prevail, the state may be truly called its own inferior and therefore bad; and when they are defeated, its own superior and therefore good.

\par \textbf{CLEINIAS}
\par   Your remark, Stranger, is a paradox, and yet we cannot possibly deny it.

\par \textbf{ATHENIAN}
\par   Here is another case for consideration;—in a family there may be several brothers, who are the offspring of a single pair; very possibly the majority of them may be unjust, and the just may be in a minority.

\par \textbf{CLEINIAS}
\par   Very possibly.

\par \textbf{ATHENIAN}
\par   And you and I ought not to raise a question of words as to whether this family and household are rightly said to be superior when they conquer, and inferior when they are conquered; for we are not now considering what may or may not be the proper or customary way of speaking, but we are considering the natural principles of right and wrong in laws.

\par \textbf{CLEINIAS}
\par   What you say, Stranger, is most true.

\par \textbf{MEGILLUS}
\par   Quite excellent, in my opinion, as far as we have gone.

\par \textbf{ATHENIAN}
\par   Again; might there not be a judge over these brethren, of whom we were speaking?

\par \textbf{CLEINIAS}
\par   Certainly.

\par \textbf{ATHENIAN}
\par   Now, which would be the better judge—one who destroyed the bad and appointed the good to govern themselves; or one who, while allowing the good to govern, let the bad live, and made them voluntarily submit? Or third, I suppose, in the scale of excellence might be placed a judge, who, finding the family distracted, not only did not destroy any one, but reconciled them to one another for ever after, and gave them laws which they mutually observed, and was able to keep them friends.

\par \textbf{CLEINIAS}
\par   The last would be by far the best sort of judge and legislator.

\par \textbf{ATHENIAN}
\par   And yet the aim of all the laws which he gave would be the reverse of war.

\par \textbf{CLEINIAS}
\par   Very true.

\par \textbf{ATHENIAN}
\par   And will he who constitutes the state and orders the life of man have in view external war, or that kind of intestine war called civil, which no one, if he could prevent, would like to have occurring in his own state; and when occurring, every one would wish to be quit of as soon as possible?

\par \textbf{CLEINIAS}
\par   He would have the latter chiefly in view.

\par \textbf{ATHENIAN}
\par   And would he prefer that this civil war should be terminated by the destruction of one of the parties, and by the victory of the other, or that peace and friendship should be re-established, and that, being reconciled, they should give their attention to foreign enemies?

\par \textbf{CLEINIAS}
\par   Every one would desire the latter in the case of his own state.

\par \textbf{ATHENIAN}
\par   And would not that also be the desire of the legislator?

\par \textbf{CLEINIAS}
\par   Certainly.

\par \textbf{ATHENIAN}
\par   And would not every one always make laws for the sake of the best?

\par \textbf{CLEINIAS}
\par   To be sure.

\par \textbf{ATHENIAN}
\par   But war, whether external or civil, is not the best, and the need of either is to be deprecated; but peace with one another, and good will, are best. Nor is the victory of the state over itself to be regarded as a really good thing, but as a necessity; a man might as well say that the body was in the best state when sick and purged by medicine, forgetting that there is also a state of the body which needs no purge. And in like manner no one can be a true statesman, whether he aims at the happiness of the individual or state, who looks only, or first of all, to external warfare; nor will he ever be a sound legislator who orders peace for the sake of war, and not war for the sake of peace.

\par \textbf{CLEINIAS}
\par   I suppose that there is truth, Stranger, in that remark of yours; and yet I am greatly mistaken if war is not the entire aim and object of our own institutions, and also of the Lacedaemonian.

\par \textbf{ATHENIAN}
\par   I dare say; but there is no reason why we should rudely quarrel with one another about your legislators, instead of gently questioning them, seeing that both we and they are equally in earnest. Please follow me and the argument closely: —And first I will put forward Tyrtaeus, an Athenian by birth, but also a Spartan citizen, who of all men was most eager about war:  Well, he says,
 
\par  even if he were the richest of men, and possessed every good (and then he gives a whole list of them), if he be not at all times a brave warrior.' I imagine that you, too, must have heard his poems; our Lacedaemonian friend has probably heard more than enough of them.

\par \textbf{MEGILLUS}
\par   Very true.

\par \textbf{CLEINIAS}
\par   And they have found their way from Lacedaemon to Crete.

\par \textbf{ATHENIAN}
\par   Come now and let us all join in asking this question of Tyrtaeus:  O most divine poet, we will say to him, the excellent praise which you have bestowed on those who excel in war sufficiently proves that you are wise and good, and I and Megillus and Cleinias of Cnosus do, as I believe, entirely agree with you. But we should like to be quite sure that we are speaking of the same men; tell us, then, do you agree with us in thinking that there are two kinds of war; or what would you say? A far inferior man to Tyrtaeus would have no difficulty in replying quite truly, that war is of two kinds,—one which is universally called civil war, and is, as we were just now saying, of all wars the worst; the other, as we should all admit, in which we fall out with other nations who are of a different race, is a far milder form of warfare.

\par \textbf{CLEINIAS}
\par   Certainly, far milder.

\par \textbf{ATHENIAN}
\par   Well, now, when you praise and blame war in this high-flown strain, whom are you praising or blaming, and to which kind of war are you referring? I suppose that you must mean foreign war, if I am to judge from expressions of yours in which you say that you abominate those

\par  'Who refuse to look upon fields of blood, and will not draw near and strike at their enemies.'

\par  And we shall naturally go on to say to him,—You, Tyrtaeus, as it seems, praise those who distinguish themselves in external and foreign war; and he must admit this.

\par \textbf{CLEINIAS}
\par   Evidently.

\par \textbf{ATHENIAN}
\par   They are good; but we say that there are still better men whose virtue is displayed in the greatest of all battles. And we too have a poet whom we summon as a witness, Theognis, citizen of Megara in Sicily:

\par  'Cyrnus,' he says, 'he who is faithful in a civil broil is worth his weight in gold and silver.'

\par  And such an one is far better, as we affirm, than the other in a more difficult kind of war, much in the same degree as justice and temperance and wisdom, when united with courage, are better than courage only; for a man cannot be faithful and good in civil strife without having all virtue. But in the war of which Tyrtaeus speaks, many a mercenary soldier will take his stand and be ready to die at his post, and yet they are generally and almost without exception insolent, unjust, violent men, and the most senseless of human beings. You will ask what the conclusion is, and what I am seeking to prove: I maintain that the divine legislator of Crete, like any other who is worthy of consideration, will always and above all things in making laws have regard to the greatest virtue; which, according to Theognis, is loyalty in the hour of danger, and may be truly called perfect justice. Whereas, that virtue which Tyrtaeus highly praises is well enough, and was praised by the poet at the right time, yet in place and dignity may be said to be only fourth rate (i.e., it ranks after justice, temperance, and wisdom. ).

\par \textbf{CLEINIAS}
\par   Stranger, we are degrading our inspired lawgiver to a rank which is far beneath him.

\par \textbf{ATHENIAN}
\par   Nay, I think that we degrade not him but ourselves, if we imagine that Lycurgus and Minos laid down laws both in Lacedaemon and Crete mainly with a view to war.

\par \textbf{CLEINIAS}
\par   What ought we to say then?

\par \textbf{ATHENIAN}
\par   What truth and what justice require of us, if I am not mistaken, when speaking in behalf of divine excellence;—that the legislator when making his laws had in view not a part only, and this the lowest part of virtue, but all virtue, and that he devised classes of laws answering to the kinds of virtue; not in the way in which modern inventors of laws make the classes, for they only investigate and offer laws whenever a want is felt, and one man has a class of laws about allotments and heiresses, another about assaults; others about ten thousand other such matters. But we maintain that the right way of examining into laws is to proceed as we have now done, and I admired the spirit of your exposition; for you were quite right in beginning with virtue, and saying that this was the aim of the giver of the law, but I thought that you went wrong when you added that all his legislation had a view only to a part, and the least part of virtue, and this called forth my subsequent remarks. Will you allow me then to explain how I should have liked to have heard you expound the matter?

\par \textbf{CLEINIAS}
\par   By all means.

\par \textbf{ATHENIAN}
\par   You ought to have said, Stranger—The Cretan laws are with reason famous among the Hellenes; for they fulfil the object of laws, which is to make those who use them happy; and they confer every sort of good. Now goods are of two kinds:  there are human and there are divine goods, and the human hang upon the divine; and the state which attains the greater, at the same time acquires the less, or, not having the greater, has neither. Of the lesser goods the first is health, the second beauty, the third strength, including swiftness in running and bodily agility generally, and the fourth is wealth, not the blind god (Pluto), but one who is keen of sight, if only he has wisdom for his companion. For wisdom is chief and leader of the divine class of goods, and next follows temperance; and from the union of these two with courage springs justice, and fourth in the scale of virtue is courage. All these naturally take precedence of the other goods, and this is the order in which the legislator must place them, and after them he will enjoin the rest of his ordinances on the citizens with a view to these, the human looking to the divine, and the divine looking to their leader mind. Some of his ordinances will relate to contracts of marriage which they make one with another, and then to the procreation and education of children, both male and female; the duty of the lawgiver will be to take charge of his citizens, in youth and age, and at every time of life, and to give them punishments and rewards; and in reference to all their intercourse with one another, he ought to consider their pains and pleasures and desires, and the vehemence of all their passions; he should keep a watch over them, and blame and praise them rightly by the mouth of the laws themselves. Also with regard to anger and terror, and the other perturbations of the soul, which arise out of misfortune, and the deliverances from them which prosperity brings, and the experiences which come to men in diseases, or in war, or poverty, or the opposite of these; in all these states he should determine and teach what is the good and evil of the condition of each. In the next place, the legislator has to be careful how the citizens make their money and in what way they spend it, and to have an eye to their mutual contracts and dissolutions of contracts, whether voluntary or involuntary:  he should see how they order all this, and consider where justice as well as injustice is found or is wanting in their several dealings with one another; and honour those who obey the law, and impose fixed penalties on those who disobey, until the round of civil life is ended, and the time has come for the consideration of the proper funeral rites and honours of the dead. And the lawgiver reviewing his work, will appoint guardians to preside over these things,—some who walk by intelligence, others by true opinion only, and then mind will bind together all his ordinances and show them to be in harmony with temperance and justice, and not with wealth or ambition. This is the spirit, Stranger, in which I was and am desirous that you should pursue the subject. And I want to know the nature of all these things, and how they are arranged in the laws of Zeus, as they are termed, and in those of the Pythian Apollo, which Minos and Lycurgus gave; and how the order of them is discovered to his eyes, who has experience in laws gained either by study or habit, although they are far from being self-evident to the rest of mankind like ourselves.

\par \textbf{CLEINIAS}
\par   How shall we proceed, Stranger?

\par \textbf{ATHENIAN}
\par   I think that we must begin again as before, and first consider the habit of courage; and then we will go on and discuss another and then another form of virtue, if you please. In this way we shall have a model of the whole; and with these and similar discourses we will beguile the way. And when we have gone through all the virtues, we will show, by the grace of God, that the institutions of which I was speaking look to virtue.

\par \textbf{MEGILLUS}
\par   Very good; and suppose that you first criticize this praiser of Zeus and the laws of Crete.

\par \textbf{ATHENIAN}
\par   I will try to criticize you and myself, as well as him, for the argument is a common concern. Tell me,—were not first the syssitia, and secondly the gymnasia, invented by your legislator with a view to war?

\par \textbf{MEGILLUS}
\par   Yes.

\par \textbf{ATHENIAN}
\par   And what comes third, and what fourth? For that, I think, is the sort of enumeration which ought to be made of the remaining parts of virtue, no matter whether you call them parts or what their name is, provided the meaning is clear.

\par \textbf{MEGILLUS}
\par   Then I, or any other Lacedaemonian, would reply that hunting is third in order.

\par \textbf{ATHENIAN}
\par   Let us see if we can discover what comes fourth and fifth.

\par \textbf{MEGILLUS}
\par   I think that I can get as far as the fourth head, which is the frequent endurance of pain, exhibited among us Spartans in certain hand-to-hand fights; also in stealing with the prospect of getting a good beating; there is, too, the so-called Crypteia, or secret service, in which wonderful endurance is shown,—our people wander over the whole country by day and by night, and even in winter have not a shoe to their foot, and are without beds to lie upon, and have to attend upon themselves. Marvellous, too, is the endurance which our citizens show in their naked exercises, contending against the violent summer heat; and there are many similar practices, to speak of which in detail would be endless.

\par \textbf{ATHENIAN}
\par   Excellent, O Lacedaemonian Stranger. But how ought we to define courage? Is it to be regarded only as a combat against fears and pains, or also against desires and pleasures, and against flatteries; which exercise such a tremendous power, that they make the hearts even of respectable citizens to melt like wax?

\par \textbf{MEGILLUS}
\par   I should say the latter.

\par \textbf{ATHENIAN}
\par   In what preceded, as you will remember, our Cnosian friend was speaking of a man or a city being inferior to themselves: —Were you not, Cleinias?

\par \textbf{CLEINIAS}
\par   I was.

\par \textbf{ATHENIAN}
\par   Now, which is in the truest sense inferior, the man who is overcome by pleasure or by pain?

\par \textbf{CLEINIAS}
\par   I should say the man who is overcome by pleasure; for all men deem him to be inferior in a more disgraceful sense, than the other who is overcome by pain.

\par \textbf{ATHENIAN}
\par   But surely the lawgivers of Crete and Lacedaemon have not legislated for a courage which is lame of one leg, able only to meet attacks which come from the left, but impotent against the insidious flatteries which come from the right?

\par \textbf{CLEINIAS}
\par   Able to meet both, I should say.

\par \textbf{ATHENIAN}
\par   Then let me once more ask, what institutions have you in either of your states which give a taste of pleasures, and do not avoid them any more than they avoid pains; but which set a person in the midst of them, and compel or induce him by the prospect of reward to get the better of them? Where is an ordinance about pleasure similar to that about pain to be found in your laws? Tell me what there is of this nature among you: —What is there which makes your citizen equally brave against pleasure and pain, conquering what they ought to conquer, and superior to the enemies who are most dangerous and nearest home?

\par \textbf{MEGILLUS}
\par   I was able to tell you, Stranger, many laws which were directed against pain; but I do not know that I can point out any great or obvious examples of similar institutions which are concerned with pleasure; there are some lesser provisions, however, which I might mention.

\par \textbf{CLEINIAS}
\par   Neither can I show anything of that sort which is at all equally prominent in the Cretan laws.

\par \textbf{ATHENIAN}
\par   No wonder, my dear friends; and if, as is very likely, in our search after the true and good, one of us may have to censure the laws of the others, we must not be offended, but take kindly what another says.

\par \textbf{CLEINIAS}
\par   You are quite right, Athenian Stranger, and we will do as you say.

\par \textbf{ATHENIAN}
\par   At our time of life, Cleinias, there should be no feeling of irritation.

\par \textbf{CLEINIAS}
\par   Certainly not.

\par \textbf{ATHENIAN}
\par   I will not at present determine whether he who censures the Cretan or Lacedaemonian polities is right or wrong. But I believe that I can tell better than either of you what the many say about them. For assuming that you have reasonably good laws, one of the best of them will be the law forbidding any young men to enquire which of them are right or wrong; but with one mouth and one voice they must all agree that the laws are all good, for they came from God; and any one who says the contrary is not to be listened to. But an old man who remarks any defect in your laws may communicate his observation to a ruler or to an equal in years when no young man is present.

\par \textbf{CLEINIAS}
\par   Exactly so, Stranger; and like a diviner, although not there at the time, you seem to me quite to have hit the meaning of the legislator, and to say what is most true.

\par \textbf{ATHENIAN}
\par   As there are no young men present, and the legislator has given old men free licence, there will be no impropriety in our discussing these very matters now that we are alone.

\par \textbf{CLEINIAS}
\par   True. And therefore you may be as free as you like in your censure of our laws, for there is no discredit in knowing what is wrong; he who receives what is said in a generous and friendly spirit will be all the better for it.

\par \textbf{ATHENIAN}
\par   Very good; however, I am not going to say anything against your laws until to the best of my ability I have examined them, but I am going to raise doubts about them. For you are the only people known to us, whether Greek or barbarian, whom the legislator commanded to eschew all great pleasures and amusements and never to touch them; whereas in the matter of pains or fears which we have just been discussing, he thought that they who from infancy had always avoided pains and fears and sorrows, when they were compelled to face them would run away from those who were hardened in them, and would become their subjects. Now the legislator ought to have considered that this was equally true of pleasure; he should have said to himself, that if our citizens are from their youth upward unacquainted with the greatest pleasures, and unused to endure amid the temptations of pleasure, and are not disciplined to refrain from all things evil, the sweet feeling of pleasure will overcome them just as fear would overcome the former class; and in another, and even a worse manner, they will be the slaves of those who are able to endure amid pleasures, and have had the opportunity of enjoying them, they being often the worst of mankind. One half of their souls will be a slave, the other half free; and they will not be worthy to be called in the true sense men and freemen. Tell me whether you assent to my words?

\par \textbf{CLEINIAS}
\par   On first hearing, what you say appears to be the truth; but to be hasty in coming to a conclusion about such important matters would be very childish and simple.

\par \textbf{ATHENIAN}
\par   Suppose, Cleinias and Megillus, that we consider the virtue which follows next of those which we intended to discuss (for after courage comes temperance), what institutions shall we find relating to temperance, either in Crete or Lacedaemon, which, like your military institutions, differ from those of any ordinary state.

\par \textbf{MEGILLUS}
\par   That is not an easy question to answer; still I should say that the common meals and gymnastic exercises have been excellently devised for the promotion both of temperance and courage.

\par \textbf{ATHENIAN}
\par   There seems to be a difficulty, Stranger, with regard to states, in making words and facts coincide so that there can be no dispute about them. As in the human body, the regimen which does good in one way does harm in another; and we can hardly say that any one course of treatment is adapted to a particular constitution. Now the gymnasia and common meals do a great deal of good, and yet they are a source of evil in civil troubles; as is shown in the case of the Milesian, and Boeotian, and Thurian youth, among whom these institutions seem always to have had a tendency to degrade the ancient and natural custom of love below the level, not only of man, but of the beasts. The charge may be fairly brought against your cities above all others, and is true also of most other states which especially cultivate gymnastics. Whether such matters are to be regarded jestingly or seriously, I think that the pleasure is to be deemed natural which arises out of the intercourse between men and women; but that the intercourse of men with men, or of women with women, is contrary to nature, and that the bold attempt was originally due to unbridled lust. The Cretans are always accused of having invented the story of Ganymede and Zeus because they wanted to justify themselves in the enjoyment of unnatural pleasures by the practice of the god whom they believe to have been their lawgiver. Leaving the story, we may observe that any speculation about laws turns almost entirely on pleasure and pain, both in states and in individuals:  these are two fountains which nature lets flow, and he who draws from them where and when, and as much as he ought, is happy; and this holds of men and animals—of individuals as well as states; and he who indulges in them ignorantly and at the wrong time, is the reverse of happy.

\par \textbf{MEGILLUS}
\par   I admit, Stranger, that your words are well spoken, and I hardly know what to say in answer to you; but still I think that the Spartan lawgiver was quite right in forbidding pleasure. Of the Cretan laws, I shall leave the defence to my Cnosian friend. But the laws of Sparta, in as far as they relate to pleasure, appear to me to be the best in the world; for that which leads mankind in general into the wildest pleasure and licence, and every other folly, the law has clean driven out; and neither in the country nor in towns which are under the control of Sparta, will you find revelries and the many incitements of every kind of pleasure which accompany them; and any one who meets a drunken and disorderly person, will immediately have him most severely punished, and will not let him off on any pretence, not even at the time of a Dionysiac festival; although I have remarked that this may happen at your performances 'on the cart,' as they are called; and among our Tarentine colonists I have seen the whole city drunk at a Dionysiac festival; but nothing of the sort happens among us.

\par \textbf{ATHENIAN}
\par   O Lacedaemonian Stranger, these festivities are praiseworthy where there is a spirit of endurance, but are very senseless when they are under no regulations. In order to retaliate, an Athenian has only to point out the licence which exists among your women. To all such accusations, whether they are brought against the Tarentines, or us, or you, there is one answer which exonerates the practice in question from impropriety. When a stranger expresses wonder at the singularity of what he sees, any inhabitant will naturally answer him: —Wonder not, O stranger; this is our custom, and you may very likely have some other custom about the same things. Now we are speaking, my friends, not about men in general, but about the merits and defects of the lawgivers themselves. Let us then discourse a little more at length about intoxication, which is a very important subject, and will seriously task the discrimination of the legislator. I am not speaking of drinking, or not drinking, wine at all, but of intoxication. Are we to follow the custom of the Scythians, and Persians, and Carthaginians, and Celts, and Iberians, and Thracians, who are all warlike nations, or that of your countrymen, for they, as you say, altogether abstain? But the Scythians and Thracians, both men and women, drink unmixed wine, which they pour on their garments, and this they think a happy and glorious institution. The Persians, again, are much given to other practices of luxury which you reject, but they have more moderation in them than the Thracians and Scythians.

\par \textbf{MEGILLUS}
\par   O best of men, we have only to take arms into our hands, and we send all these nations flying before us.

\par \textbf{ATHENIAN}
\par   Nay, my good friend, do not say that; there have been, as there always will be, flights and pursuits of which no account can be given, and therefore we cannot say that victory or defeat in battle affords more than a doubtful proof of the goodness or badness of institutions. For when the greater states conquer and enslave the lesser, as the Syracusans have done the Locrians, who appear to be the best-governed people in their part of the world, or as the Athenians have done the Ceans (and there are ten thousand other instances of the same sort of thing), all this is not to the point; let us endeavour rather to form a conclusion about each institution in itself and say nothing, at present, of victories and defeats. Let us only say that such and such a custom is honourable, and another not. And first permit me to tell you how good and bad are to be estimated in reference to these very matters.

\par \textbf{MEGILLUS}
\par   How do you mean?

\par \textbf{ATHENIAN}
\par   All those who are ready at a moment's notice to praise or censure any practice which is matter of discussion, seem to me to proceed in a wrong way. Let me give you an illustration of what I mean: —You may suppose a person to be praising wheat as a good kind of food, whereupon another person instantly blames wheat, without ever enquiring into its effect or use, or in what way, or to whom, or with what, or in what state and how, wheat is to be given. And that is just what we are doing in this discussion. At the very mention of the word intoxication, one side is ready with their praises and the other with their censures; which is absurd. For either side adduce their witnesses and approvers, and some of us think that we speak with authority because we have many witnesses; and others because they see those who abstain conquering in battle, and this again is disputed by us. Now I cannot say that I shall be satisfied, if we go on discussing each of the remaining laws in the same way. And about this very point of intoxication I should like to speak in another way, which I hold to be the right one; for if number is to be the criterion, are there not myriads upon myriads of nations ready to dispute the point with you, who are only two cities?

\par \textbf{MEGILLUS}
\par   I shall gladly welcome any method of enquiry which is right.

\par \textbf{ATHENIAN}
\par   Let me put the matter thus: —Suppose a person to praise the keeping of goats, and the creatures themselves as capital things to have, and then some one who had seen goats feeding without a goatherd in cultivated spots, and doing mischief, were to censure a goat or any other animal who has no keeper, or a bad keeper, would there be any sense or justice in such censure?

\par \textbf{MEGILLUS}
\par   Certainly not.

\par \textbf{ATHENIAN}
\par   Does a captain require only to have nautical knowledge in order to be a good captain, whether he is sea-sick or not? What do you say?

\par \textbf{MEGILLUS}
\par   I say that he is not a good captain if, although he have nautical skill, he is liable to sea-sickness.

\par \textbf{ATHENIAN}
\par   And what would you say of the commander of an army? Will he be able to command merely because he has military skill if he be a coward, who, when danger comes, is sick and drunk with fear?

\par \textbf{MEGILLUS}
\par   Impossible.

\par \textbf{ATHENIAN}
\par   And what if besides being a coward he has no skill?

\par \textbf{MEGILLUS}
\par   He is a miserable fellow, not fit to be a commander of men, but only of old women.

\par \textbf{ATHENIAN}
\par   And what would you say of some one who blames or praises any sort of meeting which is intended by nature to have a ruler, and is well enough when under his presidency? The critic, however, has never seen the society meeting together at an orderly feast under the control of a president, but always without a ruler or with a bad one: —when observers of this class praise or blame such meetings, are we to suppose that what they say is of any value?

\par \textbf{MEGILLUS}
\par   Certainly not, if they have never seen or been present at such a meeting when rightly ordered.

\par \textbf{ATHENIAN}
\par   Reflect; may not banqueters and banquets be said to constitute a kind of meeting?

\par \textbf{MEGILLUS}
\par   Of course.

\par \textbf{ATHENIAN}
\par   And did any one ever see this sort of convivial meeting rightly ordered? Of course you two will answer that you have never seen them at all, because they are not customary or lawful in your country; but I have come across many of them in many different places, and moreover I have made enquiries about them wherever I went, as I may say, and never did I see or hear of anything of the kind which was carried on altogether rightly; in some few particulars they might be right, but in general they were utterly wrong.

\par \textbf{CLEINIAS}
\par   What do you mean, Stranger, by this remark? Explain. For we, as you say, from our inexperience in such matters, might very likely not know, even if they came in our way, what was right or wrong in such societies.

\par \textbf{ATHENIAN}
\par   Likely enough; then let me try to be your instructor:  You would acknowledge, would you not, that in all gatherings of mankind, of whatever sort, there ought to be a leader?

\par \textbf{CLEINIAS}
\par   Certainly I should.

\par \textbf{ATHENIAN}
\par   And we were saying just now, that when men are at war the leader ought to be a brave man?

\par \textbf{CLEINIAS}
\par   We were.

\par \textbf{ATHENIAN}
\par   The brave man is less likely than the coward to be disturbed by fears?

\par \textbf{CLEINIAS}
\par   That again is true.

\par \textbf{ATHENIAN}
\par   And if there were a possibility of having a general of an army who was absolutely fearless and imperturbable, should we not by all means appoint him?

\par \textbf{CLEINIAS}
\par   Assuredly.

\par \textbf{ATHENIAN}
\par   Now, however, we are speaking not of a general who is to command an army, when foe meets foe in time of war, but of one who is to regulate meetings of another sort, when friend meets friend in time of peace.

\par \textbf{CLEINIAS}
\par   True.

\par \textbf{ATHENIAN}
\par   And that sort of meeting, if attended with drunkenness, is apt to be unquiet.

\par \textbf{CLEINIAS}
\par   Certainly; the reverse of quiet.

\par \textbf{ATHENIAN}
\par   In the first place, then, the revellers as well as the soldiers will require a ruler?

\par \textbf{CLEINIAS}
\par   To be sure; no men more so.

\par \textbf{ATHENIAN}
\par   And we ought, if possible, to provide them with a quiet ruler?

\par \textbf{CLEINIAS}
\par   Of course.

\par \textbf{ATHENIAN}
\par   And he should be a man who understands society; for his duty is to preserve the friendly feelings which exist among the company at the time, and to increase them for the future by his use of the occasion.

\par \textbf{CLEINIAS}
\par   Very true.

\par \textbf{ATHENIAN}
\par   Must we not appoint a sober man and a wise to be our master of the revels? For if the ruler of drinkers be himself young and drunken, and not over-wise, only by some special good fortune will he be saved from doing some great evil.

\par \textbf{CLEINIAS}
\par   It will be by a singular good fortune that he is saved.

\par \textbf{ATHENIAN}
\par   Now suppose such associations to be framed in the best way possible in states, and that some one blames the very fact of their existence—he may very likely be right. But if he blames a practice which he only sees very much mismanaged, he shows in the first place that he is not aware of the mismanagement, and also not aware that everything done in this way will turn out to be wrong, because done without the superintendence of a sober ruler. Do you not see that a drunken pilot or a drunken ruler of any sort will ruin ship, chariot, army—anything, in short, of which he has the direction?

\par \textbf{CLEINIAS}
\par   The last remark is very true, Stranger; and I see quite clearly the advantage of an army having a good leader—he will give victory in war to his followers, which is a very great advantage; and so of other things. But I do not see any similar advantage which either individuals or states gain from the good management of a feast; and I want you to tell me what great good will be effected, supposing that this drinking ordinance is duly established.

\par \textbf{ATHENIAN}
\par   If you mean to ask what great good accrues to the state from the right training of a single youth, or of a single chorus—when the question is put in that form, we cannot deny that the good is not very great in any particular instance. But if you ask what is the good of education in general, the answer is easy—that education makes good men, and that good men act nobly, and conquer their enemies in battle, because they are good. Education certainly gives victory, although victory sometimes produces forgetfulness of education; for many have grown insolent from victory in war, and this insolence has engendered in them innumerable evils; and many a victory has been and will be suicidal to the victors; but education is never suicidal.

\par \textbf{CLEINIAS}
\par   You seem to imply, my friend, that convivial meetings, when rightly ordered, are an important element of education.

\par \textbf{ATHENIAN}
\par   Certainly I do.

\par \textbf{CLEINIAS}
\par   And can you show that what you have been saying is true?

\par \textbf{ATHENIAN}
\par   To be absolutely sure of the truth of matters concerning which there are many opinions, is an attribute of the Gods not given to man, Stranger; but I shall be very happy to tell you what I think, especially as we are now proposing to enter on a discussion concerning laws and constitutions.

\par \textbf{CLEINIAS}
\par   Your opinion, Stranger, about the questions which are now being raised, is precisely what we want to hear.

\par \textbf{ATHENIAN}
\par   Very good; I will try to find a way of explaining my meaning, and you shall try to have the gift of understanding me. But first let me make an apology. The Athenian citizen is reputed among all the Hellenes to be a great talker, whereas Sparta is renowned for brevity, and the Cretans have more wit than words. Now I am afraid of appearing to elicit a very long discourse out of very small materials. For drinking indeed may appear to be a slight matter, and yet is one which cannot be rightly ordered according to nature, without correct principles of music; these are

\par  necessary to any clear or satisfactory treatment of the subject, and music again runs up into education generally, and there is much to be said about all this. What would you say then to leaving these matters for the present, and passing on to some other question of law?

\par \textbf{MEGILLUS}
\par   O Athenian Stranger, let me tell you what perhaps you do not know, that our family is the proxenus of your state. I imagine that from their earliest youth all boys, when they are told that they are the proxeni of a particular state, feel kindly towards their second country; and this has certainly been my own feeling. I can well remember from the days of my boyhood, how, when any Lacedaemonians praised or blamed the Athenians, they used to say to me,—'See, Megillus, how ill or how well,' as the case might be, 'has your state treated us'; and having always had to fight your battles against detractors when I heard you assailed, I became warmly attached to you. And I always like to hear the Athenian tongue spoken; the common saying is quite true, that a good Athenian is more than ordinarily good, for he is the only man who is freely and genuinely good by the divine inspiration of his own nature, and is not manufactured. Therefore be assured that I shall like to hear you say whatever you have to say.

\par \textbf{CLEINIAS}
\par   Yes, Stranger; and when you have heard me speak, say boldly what is in your thoughts. Let me remind you of a tie which unites you to Crete. You must have heard here the story of the prophet Epimenides, who was of my family, and came to Athens ten years before the Persian war, in accordance with the response of the Oracle, and offered certain sacrifices which the God commanded. The Athenians were at that time in dread of the Persian invasion; and he said that for ten years they would not come, and that when they came, they would go away again without accomplishing any of their objects, and would suffer more evil than they inflicted. At that time my forefathers formed ties of hospitality with you; thus ancient is the friendship which I and my parents have had for you.

\par \textbf{ATHENIAN}
\par   You seem to be quite ready to listen; and I am also ready to perform as much as I can of an almost impossible task, which I will nevertheless attempt. At the outset of the discussion, let me define the nature and power of education; for this is the way by which our argument must travel onwards to the God Dionysus.

\par \textbf{CLEINIAS}
\par   Let us proceed, if you please.

\par \textbf{ATHENIAN}
\par   Well, then, if I tell you what are my notions of education, will you consider whether they satisfy you?

\par \textbf{CLEINIAS}
\par   Let us hear.

\par \textbf{ATHENIAN}
\par   According to my view, any one who would be good at anything must practise that thing from his youth upwards, both in sport and earnest, in its several branches:  for example, he who is to be a good builder, should play at building children's houses; he who is to be a good husbandman, at tilling the ground; and those who have the care of their education should provide them when young with mimic tools. They should learn beforehand the knowledge which they will afterwards require for their art. For example, the future carpenter should learn to measure or apply the line in play; and the future warrior should learn riding, or some other exercise, for amusement, and the teacher should endeavour to direct the children's inclinations and pleasures, by the help of amusements, to their final aim in life. The most important part of education is right training in the nursery. The soul of the child in his play should be guided to the love of that sort of excellence in which when he grows up to manhood he will have to be perfected. Do you agree with me thus far?

\par \textbf{CLEINIAS}
\par   Certainly.

\par \textbf{ATHENIAN}
\par   Then let us not leave the meaning of education ambiguous or ill-defined. At present, when we speak in terms of praise or blame about the bringing-up of each person, we call one man educated and another uneducated, although the uneducated man may be sometimes very well educated for the calling of a retail trader, or of a captain of a ship, and the like. For we are not speaking of education in this narrower sense, but of that other education in virtue from youth upwards, which makes a man eagerly pursue the ideal perfection of citizenship, and teaches him how rightly to rule and how to obey. This is the only education which, upon our view, deserves the name; that other sort of training, which aims at the acquisition of wealth or bodily strength, or mere cleverness apart from intelligence and justice, is mean and illiberal, and is not worthy to be called education at all. But let us not quarrel with one another about a word, provided that the proposition which has just been granted hold good:  to wit, that those who are rightly educated generally become good men. Neither must we cast a slight upon education, which is the first and fairest thing that the best of men can ever have, and which, though liable to take a wrong direction, is capable of reformation. And this work of reformation is the great business of every man while he lives.

\par \textbf{CLEINIAS}
\par   Very true; and we entirely agree with you.

\par \textbf{ATHENIAN}
\par   And we agreed before that they are good men who are able to rule themselves, and bad men who are not.

\par \textbf{CLEINIAS}
\par   You are quite right.

\par \textbf{ATHENIAN}
\par   Let me now proceed, if I can, to clear up the subject a little further by an illustration which I will offer you.

\par \textbf{CLEINIAS}
\par   Proceed.

\par \textbf{ATHENIAN}
\par   Do we not consider each of ourselves to be one?

\par \textbf{CLEINIAS}
\par   We do.

\par \textbf{ATHENIAN}
\par   And each one of us has in his bosom two counsellors, both foolish and also antagonistic; of which we call the one pleasure, and the other pain.

\par \textbf{CLEINIAS}
\par   Exactly.

\par \textbf{ATHENIAN}
\par   Also there are opinions about the future, which have the general name of expectations; and the specific name of fear, when the expectation is of pain; and of hope, when of pleasure; and further, there is reflection about the good or evil of them, and this, when embodied in a decree by the State, is called Law.

\par \textbf{CLEINIAS}
\par   I am hardly able to follow you; proceed, however, as if I were.

\par \textbf{MEGILLUS}
\par   I am in the like case.

\par \textbf{ATHENIAN}
\par   Let us look at the matter thus:  May we not conceive each of us living beings to be a puppet of the Gods, either their plaything only, or created with a purpose—which of the two we cannot certainly know? But we do know, that these affections in us are like cords and strings, which pull us different and opposite ways, and to opposite actions; and herein lies the difference between virtue and vice. According to the argument there is one among these cords which every man ought to grasp and never let go, but to pull with it against all the rest; and this is the sacred and golden cord of reason, called by us the common law of the State; there are others which are hard and of iron, but this one is soft because golden; and there are several other kinds. Now we ought always to cooperate with the lead of the best, which is law. For inasmuch as reason is beautiful and gentle, and not violent, her rule must needs have ministers in order to help the golden principle in vanquishing the other principles. And thus the moral of the tale about our being puppets will not have been lost, and the meaning of the expression 'superior or inferior to a man's self' will become clearer; and the individual, attaining to right reason in this matter of pulling the strings of the puppet, should live according to its rule; while the city, receiving the same from some god or from one who has knowledge of these things, should embody it in a law, to be her guide in her dealings with herself and with other states. In this way virtue and vice will be more clearly distinguished by us. And when they have become clearer, education and other institutions will in like manner become clearer; and in particular that question of convivial entertainment, which may seem, perhaps, to have been a very trifling matter, and to have taken a great many more words than were necessary.

\par \textbf{CLEINIAS}
\par   Perhaps, however, the theme may turn out not to be unworthy of the length of discourse.

\par \textbf{ATHENIAN}
\par   Very good; let us proceed with any enquiry which really bears on our present object.

\par \textbf{CLEINIAS}
\par   Proceed.

\par \textbf{ATHENIAN}
\par   Suppose that we give this puppet of ours drink,—what will be the effect on him?

\par \textbf{CLEINIAS}
\par   Having what in view do you ask that question?

\par \textbf{ATHENIAN}
\par   Nothing as yet; but I ask generally, when the puppet is brought to the drink, what sort of result is likely to follow. I will endeavour to explain my meaning more clearly:  what I am now asking is this—Does the drinking of wine heighten and increase pleasures and pains, and passions and loves?

\par \textbf{CLEINIAS}
\par   Very greatly.

\par \textbf{ATHENIAN}
\par   And are perception and memory, and opinion and prudence, heightened and increased? Do not these qualities entirely desert a man if he becomes saturated with drink?

\par \textbf{CLEINIAS}
\par   Yes, they entirely desert him.

\par \textbf{ATHENIAN}
\par   Does he not return to the state of soul in which he was when a young child?

\par \textbf{CLEINIAS}
\par   He does.

\par \textbf{ATHENIAN}
\par   Then at that time he will have the least control over himself?

\par \textbf{CLEINIAS}
\par   The least.

\par \textbf{ATHENIAN}
\par   And will he not be in a most wretched plight?

\par \textbf{CLEINIAS}
\par   Most wretched.

\par \textbf{ATHENIAN}
\par   Then not only an old man but also a drunkard becomes a second time a child?

\par \textbf{CLEINIAS}
\par   Well said, Stranger.

\par \textbf{ATHENIAN}
\par   Is there any argument which will prove to us that we ought to encourage the taste for drinking instead of doing all we can to avoid it?

\par \textbf{CLEINIAS}
\par   I suppose that there is; you at any rate, were just now saying that you were ready to maintain such a doctrine.

\par \textbf{ATHENIAN}
\par   True, I was; and I am ready still, seeing that you have both declared that you are anxious to hear me.

\par \textbf{CLEINIAS}
\par   To be sure we are, if only for the strangeness of the paradox, which asserts that a man ought of his own accord to plunge into utter degradation.

\par \textbf{ATHENIAN}
\par   Are you speaking of the soul?

\par \textbf{CLEINIAS}
\par   Yes.

\par \textbf{ATHENIAN}
\par   And what would you say about the body, my friend? Are you not surprised at any one of his own accord bringing upon himself deformity, leanness, ugliness, decrepitude?

\par \textbf{CLEINIAS}
\par   Certainly.

\par \textbf{ATHENIAN}
\par   Yet when a man goes of his own accord to a doctor's shop, and takes medicine, is he not aware that soon, and for many days afterwards, he will be in a state of body which he would die rather than accept as the permanent condition of his life? Are not those who train in gymnasia, at first beginning reduced to a state of weakness?

\par \textbf{CLEINIAS}
\par   Yes, all that is well known.

\par \textbf{ATHENIAN}
\par   Also that they go of their own accord for the sake of the subsequent benefit?

\par \textbf{CLEINIAS}
\par   Very good.

\par \textbf{ATHENIAN}
\par   And we may conceive this to be true in the same way of other practices?

\par \textbf{CLEINIAS}
\par   Certainly.

\par \textbf{ATHENIAN}
\par   And the same view may be taken of the pastime of drinking wine, if we are right in supposing that the same good effect follows?

\par \textbf{CLEINIAS}
\par   To be sure.

\par \textbf{ATHENIAN}
\par   If such convivialities should turn out to have any advantage equal in importance to that of gymnastic, they are in their very nature to be preferred to mere bodily exercise, inasmuch as they have no accompaniment of pain.

\par \textbf{CLEINIAS}
\par   True; but I hardly think that we shall be able to discover any such benefits to be derived from them.

\par \textbf{ATHENIAN}
\par   That is just what we must endeavour to show. And let me ask you a question: —Do we not distinguish two kinds of fear, which are very different?

\par \textbf{CLEINIAS}
\par   What are they?

\par \textbf{ATHENIAN}
\par   There is the fear of expected evil.

\par \textbf{CLEINIAS}
\par   Yes.

\par \textbf{ATHENIAN}
\par   And there is the fear of an evil reputation; we are afraid of being thought evil, because we do or say some dishonourable thing, which fear we and all men term shame.

\par \textbf{CLEINIAS}
\par   Certainly.

\par \textbf{ATHENIAN}
\par   These are the two fears, as I called them; one of which is the opposite of pain and other fears, and the opposite also of the greatest and most numerous sort of pleasures.

\par \textbf{CLEINIAS}
\par   Very true.

\par \textbf{ATHENIAN}
\par   And does not the legislator and every one who is good for anything, hold this fear in the greatest honour? This is what he terms reverence, and the confidence which is the reverse of this he terms insolence; and the latter he always deems to be a very great evil both to individuals and to states.

\par \textbf{CLEINIAS}
\par   True.

\par \textbf{ATHENIAN}
\par   Does not this kind of fear preserve us in many important ways? What is there which so surely gives victory and safety in war? For there are two things which give victory—confidence before enemies, and fear of disgrace before friends.

\par \textbf{CLEINIAS}
\par   There are.

\par \textbf{ATHENIAN}
\par   Then each of us should be fearless and also fearful; and why we should be either has now been determined.

\par \textbf{CLEINIAS}
\par   Certainly.

\par \textbf{ATHENIAN}
\par   And when we want to make any one fearless, we and the law bring him face to face with many fears.

\par \textbf{CLEINIAS}
\par   Clearly.

\par \textbf{ATHENIAN}
\par   And when we want to make him rightly fearful, must we not introduce him to shameless pleasures, and train him to take up arms against them, and to overcome them? Or does this principle apply to courage only, and must he who would be perfect in valour fight against and overcome his own natural character,—since if he be unpractised and inexperienced in such conflicts, he will not be half the man which he might have been,—and are we to suppose, that with temperance it is otherwise, and that he who has never fought with the shameless and unrighteous temptations of his pleasures and lusts, and conquered them, in earnest and in play, by word, deed, and act, will still be perfectly temperate?

\par \textbf{CLEINIAS}
\par   A most unlikely supposition.

\par \textbf{ATHENIAN}
\par   Suppose that some God had given a fear-potion to men, and that the more a man drank of this the more he regarded himself at every draught as a child of misfortune, and that he feared everything happening or about to happen to him; and that at last the most courageous of men utterly lost his presence of mind for a time, and only came to himself again when he had slept off the influence of the draught.

\par \textbf{CLEINIAS}
\par   But has such a draught, Stranger, ever really been known among men?

\par \textbf{ATHENIAN}
\par   No; but, if there had been, might not such a draught have been of use to the legislator as a test of courage? Might we not go and say to him, 'O legislator, whether you are legislating for the Cretan, or for any other state, would you not like to have a touchstone of the courage and cowardice of your citizens?'

\par \textbf{CLEINIAS}
\par   'I should,' will be the answer of every one.

\par \textbf{ATHENIAN}
\par   'And you would rather have a touchstone in which there is no risk and no great danger than the reverse?'

\par \textbf{CLEINIAS}
\par   In that proposition every one may safely agree.

\par \textbf{ATHENIAN}
\par   'And in order to make use of the draught, you would lead them amid these imaginary terrors, and prove them, when the affection of fear was working upon them, and compel them to be fearless, exhorting and admonishing them; and also honouring them, but dishonouring any one who will not be persuaded by you to be in all respects such as you command him; and if he underwent the trial well and manfully, you would let him go unscathed; but if ill, you would inflict a punishment upon him? Or would you abstain from using the potion altogether, although you have no reason for abstaining?'

\par \textbf{CLEINIAS}
\par   He would be certain, Stranger, to use the potion.

\par \textbf{ATHENIAN}
\par   This would be a mode of testing and training which would be wonderfully easy in comparison with those now in use, and might be applied to a single person, or to a few, or indeed to any number; and he would do well who provided himself with the potion only, rather than with any number of other things, whether he preferred to be by himself in solitude, and there contend with his fears, because he was ashamed to be seen by the eye of man until he was perfect; or trusting to the force of his own nature and habits, and believing that he had been already disciplined sufficiently, he did not hesitate to train himself in company with any number of others, and display his power in conquering the irresistible change effected by the draught—his virtue being such, that he never in any instance fell into any great unseemliness, but was always himself, and left off before he arrived at the last cup, fearing that he, like all other men, might be overcome by the potion.

\par \textbf{CLEINIAS}
\par   Yes, Stranger, in that last case, too, he might equally show his self-control.

\par \textbf{ATHENIAN}
\par   Let us return to the lawgiver, and say to him: —'Well, lawgiver, there is certainly no such fear-potion which man has either received from the Gods or himself discovered; for witchcraft has no place at our board. But is there any potion which might serve as a test of overboldness and excessive and indiscreet boasting?

\par \textbf{CLEINIAS}
\par   I suppose that he will say, Yes,—meaning that wine is such a potion.

\par \textbf{ATHENIAN}
\par   Is not the effect of this quite the opposite of the effect of the other? When a man drinks wine he begins to be better pleased with himself, and the more he drinks the more he is filled full of brave hopes, and conceit of his power, and at last the string of his tongue is loosened, and fancying himself wise, he is brimming over with lawlessness, and has no more fear or respect, and is ready to do or say anything.

\par \textbf{CLEINIAS}
\par   I think that every one will admit the truth of your description.

\par \textbf{MEGILLUS}
\par   Certainly.

\par \textbf{ATHENIAN}
\par   Now, let us remember, as we were saying, that there are two things which should be cultivated in the soul:  first, the greatest courage; secondly, the greatest fear—

\par \textbf{CLEINIAS}
\par   Which you said to be characteristic of reverence, if I am not mistaken.

\par \textbf{ATHENIAN}
\par   Thank you for reminding me. But now, as the habit of courage and fearlessness is to be trained amid fears, let us consider whether the opposite quality is not also to be trained among opposites.

\par \textbf{CLEINIAS}
\par   That is probably the case.

\par \textbf{ATHENIAN}
\par   There are times and seasons at which we are by nature more than commonly valiant and bold; now we ought to train ourselves on these occasions to be as free from impudence and shamelessness as possible, and to be afraid to say or suffer or do anything that is base.

\par \textbf{CLEINIAS}
\par   True.

\par \textbf{ATHENIAN}
\par   Are not the moments in which we are apt to be bold and shameless such as these?—when we are under the influence of anger, love, pride, ignorance, avarice, cowardice? or when wealth, beauty, strength, and all the intoxicating workings of pleasure madden us? What is better adapted than the festive use of wine, in the first place to test, and in the second place to train the character of a man, if care be taken in the use of it? What is there cheaper, or more innocent? For do but consider which is the greater risk: —Would you rather test a man of a morose and savage nature, which is the source of ten thousand acts of injustice, by making bargains with him at a risk to yourself, or by having him as a companion at the festival of Dionysus? Or would you, if you wanted to apply a touchstone to a man who is prone to love, entrust your wife, or your sons, or daughters to him, perilling your dearest interests in order to have a view of the condition of his soul? I might mention numberless cases, in which the advantage would be manifest of getting to know a character in sport, and without paying dearly for experience. And I do not believe that either a Cretan, or any other man, will doubt that such a test is a fair test, and safer, cheaper, and speedier than any other.

\par \textbf{CLEINIAS}
\par   That is certainly true.

\par \textbf{ATHENIAN}
\par   And this knowledge of the natures and habits of men's souls will be of the greatest use in that art which has the management of them; and that art, if I am not mistaken, is politics.

\par \textbf{CLEINIAS}
\par   Exactly so.

\par 
\section{
      BOOK II.
    }
\par \textbf{ATHENIAN}
\par   And now we have to consider whether the insight into human nature is the only benefit derived from well-ordered potations, or whether there are not other advantages great and much to be desired. The argument seems to imply that there are. But how and in what way these are to be attained, will have to be considered attentively, or we may be entangled in error.

\par \textbf{CLEINIAS}
\par   Proceed.

\par \textbf{ATHENIAN}
\par   Let me once more recall our doctrine of right education; which, if I am not mistaken, depends on the due regulation of convivial intercourse.

\par \textbf{CLEINIAS}
\par   You talk rather grandly.

\par \textbf{ATHENIAN}
\par   Pleasure and pain I maintain to be the first perceptions of children, and I say that they are the forms under which virtue and vice are originally present to them. As to wisdom and true and fixed opinions, happy is the man who acquires them, even when declining in years; and we may say that he who possesses them, and the blessings which are contained in them, is a perfect man. Now I mean by education that training which is given by suitable habits to the first instincts of virtue in children;—when pleasure, and friendship, and pain, and hatred, are rightly implanted in souls not yet capable of understanding the nature of them, and who find them, after they have attained reason, to be in harmony with her. This harmony of the soul, taken as a whole, is virtue; but the particular training in respect of pleasure and pain, which leads you always to hate what you ought to hate, and love what you ought to love from the beginning of life to the end, may be separated off; and, in my view, will be rightly called education.

\par \textbf{CLEINIAS}
\par   I think, Stranger, that you are quite right in all that you have said and are saying about education.

\par \textbf{ATHENIAN}
\par   I am glad to hear that you agree with me; for, indeed, the discipline of pleasure and pain which, when rightly ordered, is a principle of education, has been often relaxed and corrupted in human life. And the Gods, pitying the toils which our race is born to undergo, have appointed holy festivals, wherein men alternate rest with labour; and have given them the Muses and Apollo, the leader of the Muses, and Dionysus, to be companions in their revels, that they may improve their education by taking part in the festivals of the Gods, and with their help. I should like to know whether a common saying is in our opinion true to nature or not. For men say that the young of all creatures cannot be quiet in their bodies or in their voices; they are always wanting to move and cry out; some leaping and skipping, and overflowing with sportiveness and delight at something, others uttering all sorts of cries. But, whereas the animals have no perception of order or disorder in their movements, that is, of rhythm or harmony, as they are called, to us, the Gods, who, as we say, have been appointed to be our companions in the dance, have given the pleasurable sense of harmony and rhythm; and so they stir us into life, and we follow them, joining hands together in dances and songs; and these they call choruses, which is a term naturally expressive of cheerfulness. Shall we begin, then, with the acknowledgment that education is first given through Apollo and the Muses? What do you say?

\par \textbf{CLEINIAS}
\par   I assent.

\par \textbf{ATHENIAN}
\par   And the uneducated is he who has not been trained in the chorus, and the educated is he who has been well trained?

\par \textbf{CLEINIAS}
\par   Certainly.

\par \textbf{ATHENIAN}
\par   And the chorus is made up of two parts, dance and song?

\par \textbf{CLEINIAS}
\par   True.

\par \textbf{ATHENIAN}
\par   Then he who is well educated will be able to sing and dance well?

\par \textbf{CLEINIAS}
\par   I suppose that he will.

\par \textbf{ATHENIAN}
\par   Let us see; what are we saying?

\par \textbf{CLEINIAS}
\par   What?

\par \textbf{ATHENIAN}
\par   He sings well and dances well; now must we add that he sings what is good and dances what is good?

\par \textbf{CLEINIAS}
\par   Let us make the addition.

\par \textbf{ATHENIAN}
\par   We will suppose that he knows the good to be good, and the bad to be bad, and makes use of them accordingly:  which now is the better trained in dancing and music—he who is able to move his body and to use his voice in what is understood to be the right manner, but has no delight in good or hatred of evil; or he who is incorrect in gesture and voice, but is right in his sense of pleasure and pain, and welcomes what is good, and is offended at what is evil?

\par \textbf{CLEINIAS}
\par   There is a great difference, Stranger, in the two kinds of education.

\par \textbf{ATHENIAN}
\par   If we three know what is good in song and dance, then we truly know also who is educated and who is uneducated; but if not, then we certainly shall not know wherein lies the safeguard of education, and whether there is any or not.

\par \textbf{CLEINIAS}
\par   True.

\par \textbf{ATHENIAN}
\par   Let us follow the scent like hounds, and go in pursuit of beauty of figure, and melody, and song, and dance; if these escape us, there will be no use in talking about true education, whether Hellenic or barbarian.

\par \textbf{CLEINIAS}
\par   Yes.

\par \textbf{ATHENIAN}
\par   And what is beauty of figure, or beautiful melody? When a manly soul is in trouble, and when a cowardly soul is in similar case, are they likely to use the same figures and gestures, or to give utterance to the same sounds?

\par \textbf{CLEINIAS}
\par   How can they, when the very colours of their faces differ?

\par \textbf{ATHENIAN}
\par   Good, my friend; I may observe, however, in passing, that in music there certainly are figures and there are melodies:  and music is concerned with harmony and rhythm, so that you may speak of a melody or figure having good rhythm or good harmony—the term is correct enough; but to speak metaphorically of a melody or figure having a 'good colour,' as the masters of choruses do, is not allowable, although you can speak of the melodies or figures of the brave and the coward, praising the one and censuring the other. And not to be tedious, let us say that the figures and melodies which are expressive of virtue of soul or body, or of images of virtue, are without exception good, and those which are expressive of vice are the reverse of good.

\par \textbf{CLEINIAS}
\par   Your suggestion is excellent; and let us answer that these things are so.

\par \textbf{ATHENIAN}
\par   Once more, are all of us equally delighted with every sort of dance?

\par \textbf{CLEINIAS}
\par   Far otherwise.

\par \textbf{ATHENIAN}
\par   What, then, leads us astray? Are beautiful things not the same to us all, or are they the same in themselves, but not in our opinion of them? For no one will admit that forms of vice in the dance are more beautiful than forms of virtue, or that he himself delights in the forms of vice, and others in a muse of another character. And yet most persons say, that the excellence of music is to give pleasure to our souls. But this is intolerable and blasphemous; there is, however, a much more plausible account of the delusion.

\par \textbf{CLEINIAS}
\par   What?

\par \textbf{ATHENIAN}
\par   The adaptation of art to the characters of men. Choric movements are imitations of manners occurring in various actions, fortunes, dispositions,—each particular is imitated, and those to whom the words, or songs, or dances are suited, either by nature or habit or both, cannot help feeling pleasure in them and applauding them, and calling them beautiful. But those whose natures, or ways, or habits are unsuited to them, cannot delight in them or applaud them, and they call them base. There are others, again, whose natures are right and their habits wrong, or whose habits are right and their natures wrong, and they praise one thing, but are pleased at another. For they say that all these imitations are pleasant, but not good. And in the presence of those whom they think wise, they are ashamed of dancing and singing in the baser manner, or of deliberately lending any countenance to such proceedings; and yet, they have a secret pleasure in them.

\par \textbf{CLEINIAS}
\par   Very true.

\par \textbf{ATHENIAN}
\par   And is any harm done to the lover of vicious dances or songs, or any good done to the approver of the opposite sort of pleasure?

\par \textbf{CLEINIAS}
\par   I think that there is.

\par \textbf{ATHENIAN}
\par   'I think' is not the word, but I would say, rather, 'I am certain.' For must they not have the same effect as when a man associates with bad characters, whom he likes and approves rather than dislikes, and only censures playfully because he has a suspicion of his own badness? In that case, he who takes pleasure in them will surely become like those in whom he takes pleasure, even though he be ashamed to praise them. And what greater good or evil can any destiny ever make us undergo?

\par \textbf{CLEINIAS}
\par   I know of none.

\par \textbf{ATHENIAN}
\par   Then in a city which has good laws, or in future ages is to have them, bearing in mind the instruction and amusement which are given by music, can we suppose that the poets are to be allowed to teach in the dance anything which they themselves like, in the way of rhythm, or melody, or words, to the young children of any well-conditioned parents? Is the poet to train his choruses as he pleases, without reference to virtue or vice?

\par \textbf{CLEINIAS}
\par   That is surely quite unreasonable, and is not to be thought of.

\par \textbf{ATHENIAN}
\par   And yet he may do this in almost any state with the exception of Egypt.

\par \textbf{CLEINIAS}
\par   And what are the laws about music and dancing in Egypt?

\par \textbf{ATHENIAN}
\par   You will wonder when I tell you:  Long ago they appear to have recognized the very principle of which we are now speaking—that their young citizens must be habituated to forms and strains of virtue. These they fixed, and exhibited the patterns of them in their temples; and no painter or artist is allowed to innovate upon them, or to leave the traditional forms and invent new ones. To this day, no alteration is allowed either in these arts, or in music at all. And you will find that their works of art are painted or moulded in the same forms which they had ten thousand years ago;—this is literally true and no exaggeration,—their ancient paintings and sculptures are not a whit better or worse than the work of to-day, but are made with just the same skill.

\par \textbf{CLEINIAS}
\par   How extraordinary!

\par \textbf{ATHENIAN}
\par   I should rather say, How statesmanlike, how worthy of a legislator! I know that other things in Egypt are not so well. But what I am telling you about music is true and deserving of consideration, because showing that a lawgiver may institute melodies which have a natural truth and correctness without any fear of failure. To do this, however, must be the work of God, or of a divine person; in Egypt they have a tradition that their ancient chants which have been preserved for so many ages are the composition of the Goddess Isis. And therefore, as I was saying, if a person can only find in any way the natural melodies, he may confidently embody them in a fixed and legal form. For the love of novelty which arises out of pleasure in the new and weariness of the old, has not strength enough to corrupt the consecrated song and dance, under the plea that they have become antiquated. At any rate, they are far from being corrupted in Egypt.

\par \textbf{CLEINIAS}
\par   Your arguments seem to prove your point.

\par \textbf{ATHENIAN}
\par   May we not confidently say that the true use of music and of choral festivities is as follows:  We rejoice when we think that we prosper, and again we think that we prosper when we rejoice?

\par \textbf{CLEINIAS}
\par   Exactly.

\par \textbf{ATHENIAN}
\par   And when rejoicing in our good fortune, we are unable to be still?

\par \textbf{CLEINIAS}
\par   True.

\par \textbf{ATHENIAN}
\par   Our young men break forth into dancing and singing, and we who are their elders deem that we are fulfilling our part in life when we look on at them. Having lost our agility, we delight in their sports and merry-making, because we love to think of our former selves; and gladly institute contests for those who are able to awaken in us the memory of our youth.

\par \textbf{CLEINIAS}
\par   Very true.

\par \textbf{ATHENIAN}
\par   Is it altogether unmeaning to say, as the common people do about festivals, that he should be adjudged the wisest of men, and the winner of the palm, who gives us the greatest amount of pleasure and mirth? For on such occasions, and when mirth is the order of the day, ought not he to be honoured most, and, as I was saying, bear the palm, who gives most mirth to the greatest number? Now is this a true way of speaking or of acting?

\par \textbf{CLEINIAS}
\par   Possibly.

\par \textbf{ATHENIAN}
\par   But, my dear friend, let us distinguish between different cases, and not be hasty in forming a judgment:  One way of considering the question will be to imagine a festival at which there are entertainments of all sorts, including gymnastic, musical, and equestrian contests:  the citizens are assembled; prizes are offered, and proclamation is made that any one who likes may enter the lists, and that he is to bear the palm who gives the most pleasure to the spectators—there is to be no regulation about the manner how; but he who is most successful in giving pleasure is to be crowned victor, and deemed to be the pleasantest of the candidates:  What is likely to be the result of such a proclamation?

\par \textbf{CLEINIAS}
\par   In what respect?

\par \textbf{ATHENIAN}
\par   There would be various exhibitions:  one man, like Homer, will exhibit a rhapsody, another a performance on the lute; one will have a tragedy, and another a comedy. Nor would there be anything astonishing in some one imagining that he could gain the prize by exhibiting a puppet-show. Suppose these competitors to meet, and not these only, but innumerable others as well—can you tell me who ought to be the victor?

\par \textbf{CLEINIAS}
\par   I do not see how any one can answer you, or pretend to know, unless he has heard with his own ears the several competitors; the question is absurd.

\par \textbf{ATHENIAN}
\par   Well, then, if neither of you can answer, shall I answer this question which you deem so absurd?

\par \textbf{CLEINIAS}
\par   By all means.

\par \textbf{ATHENIAN}
\par   If very small children are to determine the question, they will decide for the puppet show.

\par \textbf{CLEINIAS}
\par   Of course.

\par \textbf{ATHENIAN}
\par   The older children will be advocates of comedy; educated women, and young men, and people in general, will favour tragedy.

\par \textbf{CLEINIAS}
\par   Very likely.

\par \textbf{ATHENIAN}
\par   And I believe that we old men would have the greatest pleasure in hearing a rhapsodist recite well the Iliad and Odyssey, or one of the Hesiodic poems, and would award the victory to him. But, who would really be the victor?—that is the question.

\par \textbf{CLEINIAS}
\par   Yes.

\par \textbf{ATHENIAN}
\par   Clearly you and I will have to declare that those whom we old men adjudge victors ought to win; for our ways are far and away better than any which at present exist anywhere in the world.

\par \textbf{CLEINIAS}
\par   Certainly.

\par \textbf{ATHENIAN}
\par   Thus far I too should agree with the many, that the excellence of music is to be measured by pleasure. But the pleasure must not be that of chance persons; the fairest music is that which delights the best and best educated, and especially that which delights the one man who is pre-eminent in virtue and education. And therefore the judges must be men of character, for they will require both wisdom and courage; the true judge must not draw his inspiration from the theatre, nor ought he to be unnerved by the clamour of the many and his own incapacity; nor again, knowing the truth, ought he through cowardice and unmanliness carelessly to deliver a lying judgment, with the very same lips which have just appealed to the Gods before he judged. He is sitting not as the disciple of the theatre, but, in his proper place, as their instructor, and he ought to be the enemy of all pandering to the pleasure of the spectators. The ancient and common custom of Hellas, which still prevails in Italy and Sicily, did certainly leave the judgment to the body of spectators, who determined the victor by show of hands. But this custom has been the destruction of the poets; for they are now in the habit of composing with a view to please the bad taste of their judges, and the result is that the spectators instruct themselves;—and also it has been the ruin of the theatre; they ought to be having characters put before them better than their own, and so receiving a higher pleasure, but now by their own act the opposite result follows. What inference is to be drawn from all this? Shall I tell you?

\par \textbf{CLEINIAS}
\par   What?

\par \textbf{ATHENIAN}
\par   The inference at which we arrive for the third or fourth time is, that education is the constraining and directing of youth towards that right reason, which the law affirms, and which the experience of the eldest and best has agreed to be truly right. In order, then, that the soul of the child may not be habituated to feel joy and sorrow in a manner at variance with the law, and those who obey the law, but may rather follow the law and rejoice and sorrow at the same things as the aged—in order, I say, to produce this effect, chants appear to have been invented, which really enchant, and are designed to implant that harmony of which we speak. And, because the mind of the child is incapable of enduring serious training, they are called plays and songs, and are performed in play; just as when men are sick and ailing in their bodies, their attendants give them wholesome diet in pleasant meats and drinks, but unwholesome diet in disagreeable things, in order that they may learn, as they ought, to like the one, and to dislike the other. And similarly the true legislator will persuade, and, if he cannot persuade, will compel the poet to express, as he ought, by fair and noble words, in his rhythms, the figures, and in his melodies, the music of temperate and brave and in every way good men.

\par \textbf{CLEINIAS}
\par   But do you really imagine, Stranger, that this is the way in which poets generally compose in States at the present day? As far as I can observe, except among us and among the Lacedaemonians, there are no regulations like those of which you speak; in other places novelties are always being introduced in dancing and in music, generally not under the authority of any law, but at the instigation of lawless pleasures; and these pleasures are so far from being the same, as you describe the Egyptian to be, or having the same principles, that they are never the same.

\par \textbf{ATHENIAN}
\par   Most true, Cleinias; and I daresay that I may have expressed myself obscurely, and so led you to imagine that I was speaking of some really existing state of things, whereas I was only saying what regulations I would like to have about music; and hence there occurred a misapprehension on your part. For when evils are far gone and irremediable, the task of censuring them is never pleasant, although at times necessary. But as we do not really differ, will you let me ask you whether you consider such institutions to be more prevalent among the Cretans and Lacedaemonians than among the other Hellenes?

\par \textbf{CLEINIAS}
\par   Certainly they are.

\par \textbf{ATHENIAN}
\par   And if they were extended to the other Hellenes, would it be an improvement on the present state of things?

\par \textbf{CLEINIAS}
\par   A very great improvement, if the customs which prevail among them were such as prevail among us and the Lacedaemonians, and such as you were just now saying ought to prevail.

\par \textbf{ATHENIAN}
\par   Let us see whether we understand one another: —Are not the principles of education and music which prevail among you as follows:  you compel your poets to say that the good man, if he be temperate and just, is fortunate and happy; and this whether he be great and strong or small and weak, and whether he be rich or poor; and, on the other hand, if he have a wealth passing that of Cinyras or Midas, and be unjust, he is wretched and lives in misery? As the poet says, and with truth:  I sing not, I care not about him who accomplishes all noble things, not having justice; let him who 'draws near and stretches out his hand against his enemies be a just man.' But if he be unjust, I would not have him 'look calmly upon bloody death,' nor 'surpass in swiftness the Thracian Boreas;' and let no other thing that is called good ever be his. For the goods of which the many speak are not really good:  first in the catalogue is placed health, beauty next, wealth third; and then innumerable others, as for example to have a keen eye or a quick ear, and in general to have all the senses perfect; or, again, to be a tyrant and do as you like; and the final consummation of happiness is to have acquired all these things, and when you have acquired them to become at once immortal. But you and I say, that while to the just and holy all these things are the best of possessions, to the unjust they are all, including even health, the greatest of evils. For in truth, to have sight, and hearing, and the use of the senses, or to live at all without justice and virtue, even though a man be rich in all the so-called goods of fortune, is the greatest of evils, if life be immortal; but not so great, if the bad man lives only a very short time. These are the truths which, if I am not mistaken, you will persuade or compel your poets to utter with suitable accompaniments of harmony and rhythm, and in these they must train up your youth. Am I not right? For I plainly declare that evils as they are termed are goods to the unjust, and only evils to the just, and that goods are truly good to the good, but evil to the evil. Let me ask again, Are you and I agreed about this?

\par \textbf{CLEINIAS}
\par   I think that we partly agree and partly do not.

\par \textbf{ATHENIAN}
\par   When a man has health and wealth and a tyranny which lasts, and when he is pre-eminent in strength and courage, and has the gift of immortality, and none of the so-called evils which counter-balance these goods, but only the injustice and insolence of his own nature—of such an one you are, I suspect, unwilling to believe that he is miserable rather than happy.

\par \textbf{CLEINIAS}
\par   That is quite true.

\par \textbf{ATHENIAN}
\par   Once more:  Suppose that he be valiant and strong, and handsome and rich, and does throughout his whole life whatever he likes, still, if he be unrighteous and insolent, would not both of you agree that he will of necessity live basely? You will surely grant so much?

\par \textbf{CLEINIAS}
\par   Certainly.

\par \textbf{ATHENIAN}
\par   And an evil life too?

\par \textbf{CLEINIAS}
\par   I am not equally disposed to grant that.

\par \textbf{ATHENIAN}
\par   Will he not live painfully and to his own disadvantage?

\par \textbf{CLEINIAS}
\par   How can I possibly say so?

\par \textbf{ATHENIAN}
\par   How! Then may Heaven make us to be of one mind, for now we are of two. To me, dear Cleinias, the truth of what I am saying is as plain as the fact that Crete is an island. And, if I were a lawgiver, I would try to make the poets and all the citizens speak in this strain, and I would inflict the heaviest penalties on any one in all the land who should dare to say that there are bad men who lead pleasant lives, or that the profitable and gainful is one thing, and the just another; and there are many other matters about which I should make my citizens speak in a manner different from the Cretans and Lacedaemonians of this age, and I may say, indeed, from the world in general. For tell me, my good friends, by Zeus and Apollo tell me, if I were to ask these same Gods who were your legislators,—Is not the most just life also the pleasantest? or are there two lives, one of which is the justest and the other the pleasantest?—and they were to reply that there are two; and thereupon I proceeded to ask, (that would be the right way of pursuing the enquiry), Which are the happier—those who lead the justest, or those who lead the pleasantest life? and they replied, Those who lead the pleasantest—that would be a very strange answer, which I should not like to put into the mouth of the Gods. The words will come with more propriety from the lips of fathers and legislators, and therefore I will repeat my former questions to one of them, and suppose him to say again that he who leads the pleasantest life is the happiest. And to that I rejoin: —O my father, did you not wish me to live as happily as possible? And yet you also never ceased telling me that I should live as justly as possible. Now, here the giver of the rule, whether he be legislator or father, will be in a dilemma, and will in vain endeavour to be consistent with himself. But if he were to declare that the justest life is also the happiest, every one hearing him would enquire, if I am not mistaken, what is that good and noble principle in life which the law approves, and which is superior to pleasure. For what good can the just man have which is separated from pleasure? Shall we say that glory and fame, coming from Gods and men, though good and noble, are nevertheless unpleasant, and infamy pleasant? Certainly not, sweet legislator. Or shall we say that the not-doing of wrong and there being no wrong done is good and honourable, although there is no pleasure in it, and that the doing wrong is pleasant, but evil and base?

\par \textbf{CLEINIAS}
\par   Impossible.

\par \textbf{ATHENIAN}
\par   The view which identifies the pleasant and the pleasant and the just and the good and the noble has an excellent moral and religious tendency. And the opposite view is most at variance with the designs of the legislator, and is, in his opinion, infamous; for no one, if he can help, will be persuaded to do that which gives him more pain than pleasure. But as distant prospects are apt to make us dizzy, especially in childhood, the legislator will try to purge away the darkness and exhibit the truth; he will persuade the citizens, in some way or other, by customs and praises and words, that just and unjust are shadows only, and that injustice, which seems opposed to justice, when contemplated by the unjust and evil man appears pleasant and the just most unpleasant; but that from the just man's point of view, the very opposite is the appearance of both of them.

\par \textbf{CLEINIAS}
\par   True.

\par \textbf{ATHENIAN}
\par   And which may be supposed to be the truer judgment—that of the inferior or of the better soul?

\par \textbf{CLEINIAS}
\par   Surely, that of the better soul.

\par \textbf{ATHENIAN}
\par   Then the unjust life must not only be more base and depraved, but also more unpleasant than the just and holy life?

\par \textbf{CLEINIAS}
\par   That seems to be implied in the present argument.

\par \textbf{ATHENIAN}
\par   And even supposing this were otherwise, and not as the argument has proven, still the lawgiver, who is worth anything, if he ever ventures to tell a lie to the young for their good, could not invent a more useful lie than this, or one which will have a better effect in making them do what is right, not on compulsion but voluntarily.

\par \textbf{CLEINIAS}
\par   Truth, Stranger, is a noble thing and a lasting, but a thing of which men are hard to be persuaded.

\par \textbf{ATHENIAN}
\par   And yet the story of the Sidonian Cadmus, which is so improbable, has been readily believed, and also innumerable other tales.

\par \textbf{CLEINIAS}
\par   What is that story?

\par \textbf{ATHENIAN}
\par   The story of armed men springing up after the sowing of teeth, which the legislator may take as a proof that he can persuade the minds of the young of anything; so that he has only to reflect and find out what belief will be of the greatest public advantage, and then use all his efforts to make the whole community utter one and the same word in their songs and tales and discourses all their life long. But if you do not agree with me, there is no reason why you should not argue on the other side.

\par \textbf{CLEINIAS}
\par   I do not see that any argument can fairly be raised by either of us against what you are now saying.

\par \textbf{ATHENIAN}
\par   The next suggestion which I have to offer is, that all our three choruses shall sing to the young and tender souls of children, reciting in their strains all the noble thoughts of which we have already spoken, or are about to speak; and the sum of them shall be, that the life which is by the Gods deemed to be the happiest is also the best;—we shall affirm this to be a most certain truth; and the minds of our young disciples will be more likely to receive these words of ours than any others which we might address to them.

\par \textbf{CLEINIAS}
\par   I assent to what you say.

\par \textbf{ATHENIAN}
\par   First will enter in their natural order the sacred choir composed of children, which is to sing lustily the heaven-taught lay to the whole city. Next will follow the choir of young men under the age of thirty, who will call upon the God Paean to testify to the truth of their words, and will pray him to be gracious to the youth and to turn their hearts. Thirdly, the choir of elder men, who are from thirty to sixty years of age, will also sing. There remain those who are too old to sing, and they will tell stories, illustrating the same virtues, as with the voice of an oracle.

\par \textbf{CLEINIAS}
\par   Who are those who compose the third choir, Stranger? for I do not clearly understand what you mean to say about them.

\par \textbf{ATHENIAN}
\par   And yet almost all that I have been saying has been said with a view to them.

\par \textbf{CLEINIAS}
\par   Will you try to be a little plainer?

\par \textbf{ATHENIAN}
\par   I was speaking at the commencement of our discourse, as you will remember, of the fiery nature of young creatures:  I said that they were unable to keep quiet either in limb or voice, and that they called out and jumped about in a disorderly manner; and that no other animal attained to any perception of order, but man only. Now the order of motion is called rhythm, and the order of the voice, in which high and low are duly mingled, is called harmony; and both together are termed choric song. And I said that the Gods had pity on us, and gave us Apollo and the Muses to be our playfellows and leaders in the dance; and Dionysus, as I dare say that you will remember, was the third.

\par \textbf{CLEINIAS}
\par   I quite remember.

\par \textbf{ATHENIAN}
\par   Thus far I have spoken of the chorus of Apollo and the Muses, and I have still to speak of the remaining chorus, which is that of Dionysus.

\par \textbf{CLEINIAS}
\par   How is that arranged? There is something strange, at any rate on first hearing, in a Dionysiac chorus of old men, if you really mean that those who are above thirty, and may be fifty, or from fifty to sixty years of age, are to dance in his honour.

\par \textbf{ATHENIAN}
\par   Very true; and therefore it must be shown that there is good reason for the proposal.

\par \textbf{CLEINIAS}
\par   Certainly.

\par \textbf{ATHENIAN}
\par   Are we agreed thus far?

\par \textbf{CLEINIAS}
\par   About what?

\par \textbf{ATHENIAN}
\par   That every man and boy, slave and free, both sexes, and the whole city, should never cease charming themselves with the strains of which we have spoken; and that there should be every sort of change and variation of them in order to take away the effect of sameness, so that the singers may always receive pleasure from their hymns, and may never weary of them?

\par \textbf{CLEINIAS}
\par   Every one will agree.

\par \textbf{ATHENIAN}
\par   Where, then, will that best part of our city which, by reason of age and intelligence, has the greatest influence, sing these fairest of strains, which are to do so much good? Shall we be so foolish as to let them off who would give us the most beautiful and also the most useful of songs?

\par \textbf{CLEINIAS}
\par   But, says the argument, we cannot let them off.

\par \textbf{ATHENIAN}
\par   Then how can we carry out our purpose with decorum? Will this be the way?

\par \textbf{CLEINIAS}
\par   What?

\par \textbf{ATHENIAN}
\par   When a man is advancing in years, he is afraid and reluctant to sing;—he has no pleasure in his own performances; and if compulsion is used, he will be more and more ashamed, the older and more discreet he grows;—is not this true?

\par \textbf{CLEINIAS}
\par   Certainly.

\par \textbf{ATHENIAN}
\par   Well, and will he not be yet more ashamed if he has to stand up and sing in the theatre to a mixed audience?—and if moreover when he is required to do so, like the other choirs who contend for prizes, and have been trained under a singing master, he is pinched and hungry, he will certainly have a feeling of shame and discomfort which will make him very unwilling to exhibit.

\par \textbf{CLEINIAS}
\par   No doubt.

\par \textbf{ATHENIAN}
\par   How, then, shall we reassure him, and get him to sing? Shall we begin by enacting that boys shall not taste wine at all until they are eighteen years of age; we will tell them that fire must not be poured upon fire, whether in the body or in the soul, until they begin to go to work—this is a precaution which has to be taken against the excitableness of youth;—afterwards they may taste wine in moderation up to the age of thirty, but while a man is young he should abstain altogether from intoxication and from excess of wine; when, at length, he has reached forty years, after dinner at a public mess, he may invite not only the other Gods, but Dionysus above all, to the mystery and festivity of the elder men, making use of the wine which he has given men to lighten the sourness of old age; that in age we may renew our youth, and forget our sorrows; and also in order that the nature of the soul, like iron melted in the fire, may become softer and so more impressible. In the first place, will not any one who is thus mellowed be more ready and less ashamed to sing—I do not say before a large audience, but before a moderate company; nor yet among strangers, but among his familiars, and, as we have often said, to chant, and to enchant?

\par \textbf{CLEINIAS}
\par   He will be far more ready.

\par \textbf{ATHENIAN}
\par   There will be no impropriety in our using such a method of persuading them to join with us in song.

\par \textbf{CLEINIAS}
\par   None at all.

\par \textbf{ATHENIAN}
\par   And what strain will they sing, and what muse will they hymn? The strain should clearly be one suitable to them.

\par \textbf{CLEINIAS}
\par   Certainly.

\par \textbf{ATHENIAN}
\par   And what strain is suitable for heroes? Shall they sing a choric strain?

\par \textbf{CLEINIAS}
\par   Truly, Stranger, we of Crete and Lacedaemon know no strain other than that which we have learnt and been accustomed to sing in our chorus.

\par \textbf{ATHENIAN}
\par   I dare say; for you have never acquired the knowledge of the most beautiful kind of song, in your military way of life, which is modelled after the camp, and is not like that of dwellers in cities; and you have your young men herding and feeding together like young colts. No one takes his own individual colt and drags him away from his fellows against his will, raging and foaming, and gives him a groom to attend to him alone, and trains and rubs him down privately, and gives him the qualities in education which will make him not only a good soldier, but also a governor of a state and of cities. Such an one, as we said at first, would be a greater warrior than he of whom Tyrtaeus sings; and he would honour courage everywhere, but always as the fourth, and not as the first part of virtue, either in individuals or states.

\par \textbf{CLEINIAS}
\par   Once more, Stranger, I must complain that you depreciate our lawgivers.

\par \textbf{ATHENIAN}
\par   Not intentionally, if at all, my good friend; but whither the argument leads, thither let us follow; for if there be indeed some strain of song more beautiful than that of the choruses or the public theatres, I should like to impart it to those who, as we say, are ashamed of these, and want to have the best.

\par \textbf{CLEINIAS}
\par   Certainly.

\par \textbf{ATHENIAN}
\par   When things have an accompanying charm, either the best thing in them is this very charm, or there is some rightness or utility possessed by them;—for example, I should say that eating and drinking, and the use of food in general, have an accompanying charm which we call pleasure; but that this rightness and utility is just the healthfulness of the things served up to us, which is their true rightness.

\par \textbf{CLEINIAS}
\par   Just so.

\par \textbf{ATHENIAN}
\par   Thus, too, I should say that learning has a certain accompanying charm which is the pleasure; but that the right and the profitable, the good and the noble, are qualities which the truth gives to it.

\par \textbf{CLEINIAS}
\par   Exactly.

\par \textbf{ATHENIAN}
\par   And so in the imitative arts—if they succeed in making likenesses, and are accompanied by pleasure, may not their works be said to have a charm?

\par \textbf{CLEINIAS}
\par   Yes.

\par \textbf{ATHENIAN}
\par   But equal proportions, whether of quality or quantity, and not pleasure, speaking generally, would give them truth or rightness.

\par \textbf{CLEINIAS}
\par   Yes.

\par \textbf{ATHENIAN}
\par   Then that only can be rightly judged by the standard of pleasure, which makes or furnishes no utility or truth or likeness, nor on the other hand is productive of any hurtful quality, but exists solely for the sake of the accompanying charm; and the term 'pleasure' is most appropriately applied to it when these other qualities are absent.

\par \textbf{CLEINIAS}
\par   You are speaking of harmless pleasure, are you not?

\par \textbf{ATHENIAN}
\par   Yes; and this I term amusement, when doing neither harm nor good in any degree worth speaking of.

\par \textbf{CLEINIAS}
\par   Very true.

\par \textbf{ATHENIAN}
\par   Then, if such be our principles, we must assert that imitation is not to be judged of by pleasure and false opinion; and this is true of all equality, for the equal is not equal or the symmetrical symmetrical, because somebody thinks or likes something, but they are to be judged of by the standard of truth, and by no other whatever.

\par \textbf{CLEINIAS}
\par   Quite true.

\par \textbf{ATHENIAN}
\par   Do we not regard all music as representative and imitative?

\par \textbf{CLEINIAS}
\par   Certainly.

\par \textbf{ATHENIAN}
\par   Then, when any one says that music is to be judged of by pleasure, his doctrine cannot be admitted; and if there be any music of which pleasure is the criterion, such music is not to be sought out or deemed to have any real excellence, but only that other kind of music which is an imitation of the good.

\par \textbf{CLEINIAS}
\par   Very true.

\par \textbf{ATHENIAN}
\par   And those who seek for the best kind of song and music ought not to seek for that which is pleasant, but for that which is true; and the truth of imitation consists, as we were saying, in rendering the thing imitated according to quantity and quality.

\par \textbf{CLEINIAS}
\par   Certainly.

\par \textbf{ATHENIAN}
\par   And every one will admit that musical compositions are all imitative and representative. Will not poets and spectators and actors all agree in this?

\par \textbf{CLEINIAS}
\par   They will.

\par \textbf{ATHENIAN}
\par   Surely then he who would judge correctly must know what each composition is; for if he does not know what is the character and meaning of the piece, and what it represents, he will never discern whether the intention is true or false.

\par \textbf{CLEINIAS}
\par   Certainly not.

\par \textbf{ATHENIAN}
\par   And will he who does not know what is true be able to distinguish what is good and bad? My statement is not very clear; but perhaps you will understand me better if I put the matter in another way.

\par \textbf{CLEINIAS}
\par   How?

\par \textbf{ATHENIAN}
\par   There are ten thousand likenesses of objects of sight?

\par \textbf{CLEINIAS}
\par   Yes.

\par \textbf{ATHENIAN}
\par   And can he who does not know what the exact object is which is imitated, ever know whether the resemblance is truthfully executed? I mean, for example, whether a statue has the proportions of a body, and the true situation of the parts; what those proportions are, and how the parts fit into one another in due order; also their colours and conformations, or whether this is all confused in the execution:  do you think that any one can know about this, who does not know what the animal is which has been imitated?

\par \textbf{CLEINIAS}
\par   Impossible.

\par \textbf{ATHENIAN}
\par   But even if we know that the thing pictured or sculptured is a man, who has received at the hand of the artist all his proper parts and colours and shapes, must we not also know whether the work is beautiful or in any respect deficient in beauty?

\par \textbf{CLEINIAS}
\par   If this were not required, Stranger, we should all of us be judges of beauty.

\par \textbf{ATHENIAN}
\par   Very true; and may we not say that in everything imitated, whether in drawing, music, or any other art, he who is to be a competent judge must possess three things;—he must know, in the first place, of what the imitation is; secondly, he must know that it is true; and thirdly, that it has been well executed in words and melodies and rhythms?

\par \textbf{CLEINIAS}
\par   Certainly.

\par \textbf{ATHENIAN}
\par   Then let us not faint in discussing the peculiar difficulty of music. Music is more celebrated than any other kind of imitation, and therefore requires the greatest care of them all. For if a man makes a mistake here, he may do himself the greatest injury by welcoming evil dispositions, and the mistake may be very difficult to discern, because the poets are artists very inferior in character to the Muses themselves, who would never fall into the monstrous error of assigning to the words of men the gestures and songs of women; nor after combining the melodies with the gestures of freemen would they add on the rhythms of slaves and men of the baser sort; nor, beginning with the rhythms and gestures of freemen, would they assign to them a melody or words which are of an opposite character; nor would they mix up the voices and sounds of animals and of men and instruments, and every other sort of noise, as if they were all one. But human poets are fond of introducing this sort of inconsistent mixture, and so make themselves ridiculous in the eyes of those who, as Orpheus says, 'are ripe for true pleasure.' The experienced see all this confusion, and yet the poets go on and make still further havoc by separating the rhythm and the figure of the dance from the melody, setting bare words to metre, and also separating the melody and the rhythm from the words, using the lyre or the flute alone. For when there are no words, it is very difficult to recognize the meaning of the harmony and rhythm, or to see that any worthy object is imitated by them. And we must acknowledge that all this sort of thing, which aims only at swiftness and smoothness and a brutish noise, and uses the flute and the lyre not as the mere accompaniments of the dance and song, is exceedingly coarse and tasteless. The use of either instrument, when unaccompanied, leads to every sort of irregularity and trickery. This is all rational enough. But we are considering not how our choristers, who are from thirty to fifty years of age, and may be over fifty, are not to use the Muses, but how they are to use them. And the considerations which we have urged seem to show in what way these fifty years' old choristers who are to sing, may be expected to be better trained. For they need to have a quick perception and knowledge of harmonies and rhythms; otherwise, how can they ever know whether a melody would be rightly sung to the Dorian mode, or to the rhythm which the poet has assigned to it?

\par \textbf{CLEINIAS}
\par   Clearly they cannot.

\par \textbf{ATHENIAN}
\par   The many are ridiculous in imagining that they know what is in proper harmony and rhythm, and what is not, when they can only be made to sing and step in rhythm by force; it never occurs to them that they are ignorant of what they are doing. Now every melody is right when it has suitable harmony and rhythm, and wrong when unsuitable.

\par \textbf{CLEINIAS}
\par   That is most certain.

\par \textbf{ATHENIAN}
\par   But can a man who does not know a thing, as we were saying, know that the thing is right?

\par \textbf{CLEINIAS}
\par   Impossible.

\par \textbf{ATHENIAN}
\par   Then now, as would appear, we are making the discovery that our newly-appointed choristers, whom we hereby invite and, although they are their own masters, compel to sing, must be educated to such an extent as to be able to follow the steps of the rhythm and the notes of the song, that they may know the harmonies and rhythms, and be able to select what are suitable for men of their age and character to sing; and may sing them, and have innocent pleasure from their own performance, and also lead younger men to welcome with dutiful delight good dispositions. Having such training, they will attain a more accurate knowledge than falls to the lot of the common people, or even of the poets themselves. For the poet need not know the third point, viz., whether the imitation is good or not, though he can hardly help knowing the laws of melody and rhythm. But the aged chorus must know all the three, that they may choose the best, and that which is nearest to the best; for otherwise they will never be able to charm the souls of young men in the way of virtue. And now the original design of the argument which was intended to bring eloquent aid to the Chorus of Dionysus, has been accomplished to the best of our ability, and let us see whether we were right: —I should imagine that a drinking assembly is likely to become more and more tumultuous as the drinking goes on:  this, as we were saying at first, will certainly be the case.

\par \textbf{CLEINIAS}
\par   Certainly.

\par \textbf{ATHENIAN}
\par   Every man has a more than natural elevation; his heart is glad within him, and he will say anything and will be restrained by nobody at such a time; he fancies that he is able to rule over himself and all mankind.

\par \textbf{CLEINIAS}
\par   Quite true.

\par \textbf{ATHENIAN}
\par   Were we not saying that on such occasions the souls of the drinkers become like iron heated in the fire, and grow softer and younger, and are easily moulded by him who knows how to educate and fashion them, just as when they were young, and that this fashioner of them is the same who prescribed for them in the days of their youth, viz., the good legislator; and that he ought to enact laws of the banquet, which, when a man is confident, bold, and impudent, and unwilling to wait his turn and have his share of silence and speech, and drinking and music, will change his character into the opposite—such laws as will infuse into him a just and noble fear, which will take up arms at the approach of insolence, being that divine fear which we have called reverence and shame?

\par \textbf{CLEINIAS}
\par   True.

\par \textbf{ATHENIAN}
\par   And the guardians of these laws and fellow-workers with them are the calm and sober generals of the drinkers; and without their help there is greater difficulty in fighting against drink than in fighting against enemies when the commander of an army is not himself calm; and he who is unwilling to obey them and the commanders of Dionysiac feasts who are more than sixty years of age, shall suffer a disgrace as great as he who disobeys military leaders, or even greater.

\par \textbf{CLEINIAS}
\par   Right.

\par \textbf{ATHENIAN}
\par   If, then, drinking and amusement were regulated in this way, would not the companions of our revels be improved? they would part better friends than they were, and not, as now, enemies. Their whole intercourse would be regulated by law and observant of it, and the sober would be the leaders of the drunken.

\par \textbf{CLEINIAS}
\par   I think so too, if drinking were regulated as you propose.

\par \textbf{ATHENIAN}
\par   Let us not then simply censure the gift of Dionysus as bad and unfit to be received into the State. For wine has many excellences, and one pre-eminent one, about which there is a difficulty in speaking to the many, from a fear of their misconceiving and misunderstanding what is said.

\par \textbf{CLEINIAS}
\par   To what do you refer?

\par \textbf{ATHENIAN}
\par   There is a tradition or story, which has somehow crept about the world, that Dionysus was robbed of his wits by his stepmother Here, and that out of revenge he inspires Bacchic furies and dancing madnesses in others; for which reason he gave men wine. Such traditions concerning the Gods I leave to those who think that they may be safely uttered (compare Euthyph. ; Republic); I only know that no animal at birth is mature or perfect in intelligence; and in the intermediate period, in which he has not yet acquired his own proper sense, he rages and roars without rhyme or reason; and when he has once got on his legs he jumps about without rhyme or reason; and this, as you will remember, has been already said by us to be the origin of music and gymnastic.

\par \textbf{CLEINIAS}
\par   To be sure, I remember.

\par \textbf{ATHENIAN}
\par   And did we not say that the sense of harmony and rhythm sprang from this beginning among men, and that Apollo and the Muses and Dionysus were the Gods whom we had to thank for them?

\par \textbf{CLEINIAS}
\par   Certainly.

\par \textbf{ATHENIAN}
\par   The other story implied that wine was given man out of revenge, and in order to make him mad; but our present doctrine, on the contrary, is, that wine was given him as a balm, and in order to implant modesty in the soul, and health and strength in the body.

\par \textbf{CLEINIAS}
\par   That, Stranger, is precisely what was said.

\par \textbf{ATHENIAN}
\par   Then half the subject may now be considered to have been discussed; shall we proceed to the consideration of the other half?

\par \textbf{CLEINIAS}
\par   What is the other half, and how do you divide the subject?

\par \textbf{ATHENIAN}
\par   The whole choral art is also in our view the whole of education; and of this art, rhythms and harmonies form the part which has to do with the voice.

\par \textbf{CLEINIAS}
\par   Yes.

\par \textbf{ATHENIAN}
\par   The movement of the body has rhythm in common with the movement of the voice, but gesture is peculiar to it, whereas song is simply the movement of the voice.

\par \textbf{CLEINIAS}
\par   Most true.

\par \textbf{ATHENIAN}
\par   And the sound of the voice which reaches and educates the soul, we have ventured to term music.

\par \textbf{CLEINIAS}
\par   We were right.

\par \textbf{ATHENIAN}
\par   And the movement of the body, when regarded as an amusement, we termed dancing; but when extended and pursued with a view to the excellence of the body, this scientific training may be called gymnastic.

\par \textbf{CLEINIAS}
\par   Exactly.

\par \textbf{ATHENIAN}
\par   Music, which was one half of the choral art, may be said to have been completely discussed. Shall we proceed to the other half or not? What would you like?

\par \textbf{CLEINIAS}
\par   My good friend, when you are talking with a Cretan and Lacedaemonian, and we have discussed music and not gymnastic, what answer are either of us likely to make to such an enquiry?

\par \textbf{ATHENIAN}
\par   An answer is contained in your question; and I understand and accept what you say not only as an answer, but also as a command to proceed with gymnastic.

\par \textbf{CLEINIAS}
\par   You quite understand me; do as you say.

\par \textbf{ATHENIAN}
\par   I will; and there will not be any difficulty in speaking intelligibly to you about a subject with which both of you are far more familiar than with music.

\par \textbf{CLEINIAS}
\par   There will not.

\par \textbf{ATHENIAN}
\par   Is not the origin of gymnastics, too, to be sought in the tendency to rapid motion which exists in all animals; man, as we were saying, having attained the sense of rhythm, created and invented dancing; and melody arousing and awakening rhythm, both united formed the choral art?

\par \textbf{CLEINIAS}
\par   Very true.

\par \textbf{ATHENIAN}
\par   And one part of this subject has been already discussed by us, and there still remains another to be discussed?

\par \textbf{CLEINIAS}
\par   Exactly.

\par \textbf{ATHENIAN}
\par   I have first a final word to add to my discourse about drink, if you will allow me to do so.

\par \textbf{CLEINIAS}
\par   What more have you to say?

\par \textbf{ATHENIAN}
\par   I should say that if a city seriously means to adopt the practice of drinking under due regulation and with a view to the enforcement of temperance, and in like manner, and on the same principle, will allow of other pleasures, designing to gain the victory over them—in this way all of them may be used. But if the State makes drinking an amusement only, and whoever likes may drink whenever he likes, and with whom he likes, and add to this any other indulgences, I shall never agree or allow that this city or this man should practise drinking. I would go further than the Cretans and Lacedaemonians, and am disposed rather to the law of the Carthaginians, that no one while he is on a campaign should be allowed to taste wine at all, but that he should drink water during all that time, and that in the city no slave, male or female, should ever drink wine; and that no magistrates should drink during their year of office, nor should pilots of vessels or judges while on duty taste wine at all, nor any one who is going to hold a consultation about any matter of importance; nor in the day-time at all, unless in consequence of exercise or as medicine; nor again at night, when any one, either man or woman, is minded to get children. There are numberless other cases also in which those who have good sense and good laws ought not to drink wine, so that if what I say is true, no city will need many vineyards. Their husbandry and their way of life in general will follow an appointed order, and their cultivation of the vine will be the most limited and the least common of their employments. And this, Stranger, shall be the crown of my discourse about wine, if you agree.

\par \textbf{CLEINIAS}
\par   Excellent:  we agree.

\par 
\section{
      BOOK III.
    }
\par \textbf{ATHENIAN}
\par   Enough of this. And what, then, is to be regarded as the origin of government? Will not a man be able to judge of it best from a point of view in which he may behold the progress of states and their transitions to good or evil?

\par \textbf{CLEINIAS}
\par   What do you mean?

\par \textbf{ATHENIAN}
\par   I mean that he might watch them from the point of view of time, and observe the changes which take place in them during infinite ages.

\par \textbf{CLEINIAS}
\par   How so?

\par \textbf{ATHENIAN}
\par   Why, do you think that you can reckon the time which has elapsed since cities first existed and men were citizens of them?

\par \textbf{CLEINIAS}
\par   Hardly.

\par \textbf{ATHENIAN}
\par   But are sure that it must be vast and incalculable?

\par \textbf{CLEINIAS}
\par   Certainly.

\par \textbf{ATHENIAN}
\par   And have not thousands and thousands of cities come into being during this period and as many perished? And has not each of them had every form of government many times over, now growing larger, now smaller, and again improving or declining?

\par \textbf{CLEINIAS}
\par   To be sure.

\par \textbf{ATHENIAN}
\par   Let us endeavour to ascertain the cause of these changes; for that will probably explain the first origin and development of forms of government.

\par \textbf{CLEINIAS}
\par   Very good. You shall endeavour to impart your thoughts to us, and we will make an effort to understand you.

\par \textbf{ATHENIAN}
\par   Do you believe that there is any truth in ancient traditions?

\par \textbf{CLEINIAS}
\par   What traditions?

\par \textbf{ATHENIAN}
\par   The traditions about the many destructions of mankind which have been occasioned by deluges and pestilences, and in many other ways, and of the survival of a remnant?

\par \textbf{CLEINIAS}
\par   Every one is disposed to believe them.

\par \textbf{ATHENIAN}
\par   Let us consider one of them, that which was caused by the famous deluge.

\par \textbf{CLEINIAS}
\par   What are we to observe about it?

\par \textbf{ATHENIAN}
\par   I mean to say that those who then escaped would only be hill shepherds,—small sparks of the human race preserved on the tops of mountains.

\par \textbf{CLEINIAS}
\par   Clearly.

\par \textbf{ATHENIAN}
\par   Such survivors would necessarily be unacquainted with the arts and the various devices which are suggested to the dwellers in cities by interest or ambition, and with all the wrongs which they contrive against one another.

\par \textbf{CLEINIAS}
\par   Very true.

\par \textbf{ATHENIAN}
\par   Let us suppose, then, that the cities in the plain and on the sea-coast were utterly destroyed at that time.

\par \textbf{CLEINIAS}
\par   Very good.

\par \textbf{ATHENIAN}
\par   Would not all implements have then perished and every other excellent invention of political or any other sort of wisdom have utterly disappeared?

\par \textbf{CLEINIAS}
\par   Why, yes, my friend; and if things had always continued as they are at present ordered, how could any discovery have ever been made even in the least particular? For it is evident that the arts were unknown during ten thousand times ten thousand years. And no more than a thousand or two thousand years have elapsed since the discoveries of Daedalus, Orpheus and Palamedes,—since Marsyas and Olympus invented music, and Amphion the lyre—not to speak of numberless other inventions which are but of yesterday.

\par \textbf{ATHENIAN}
\par   Have you forgotten, Cleinias, the name of a friend who is really of yesterday?

\par \textbf{CLEINIAS}
\par   I suppose that you mean Epimenides.

\par \textbf{ATHENIAN}
\par   The same, my friend; he does indeed far overleap the heads of all mankind by his invention; for he carried out in practice, as you declare, what of old Hesiod (Works and Days) only preached.

\par \textbf{CLEINIAS}
\par   Yes, according to our tradition.

\par \textbf{ATHENIAN}
\par   After the great destruction, may we not suppose that the state of man was something of this sort: —In the beginning of things there was a fearful illimitable desert and a vast expanse of land; a herd or two of oxen would be the only survivors of the animal world; and there might be a few goats, these too hardly enough to maintain the shepherds who tended them?

\par \textbf{CLEINIAS}
\par   True.

\par \textbf{ATHENIAN}
\par   And of cities or governments or legislation, about which we are now talking, do you suppose that they could have any recollection at all?

\par \textbf{CLEINIAS}
\par   None whatever.

\par \textbf{ATHENIAN}
\par   And out of this state of things has there not sprung all that we now are and have:  cities and governments, and arts and laws, and a great deal of vice and a great deal of virtue?

\par \textbf{CLEINIAS}
\par   What do you mean?

\par \textbf{ATHENIAN}
\par   Why, my good friend, how can we possibly suppose that those who knew nothing of all the good and evil of cities could have attained their full development, whether of virtue or of vice?

\par \textbf{CLEINIAS}
\par   I understand your meaning, and you are quite right.

\par \textbf{ATHENIAN}
\par   But, as time advanced and the race multiplied, the world came to be what the world is.

\par \textbf{CLEINIAS}
\par   Very true.

\par \textbf{ATHENIAN}
\par   Doubtless the change was not made all in a moment, but little by little, during a very long period of time.

\par \textbf{CLEINIAS}
\par   A highly probable supposition.

\par \textbf{ATHENIAN}
\par   At first, they would have a natural fear ringing in their ears which would prevent their descending from the heights into the plain.

\par \textbf{CLEINIAS}
\par   Of course.

\par \textbf{ATHENIAN}
\par   The fewness of the survivors at that time would have made them all the more desirous of seeing one another; but then the means of travelling either by land or sea had been almost entirely lost, as I may say, with the loss of the arts, and there was great difficulty in getting at one another; for iron and brass and all metals were jumbled together and had disappeared in the chaos; nor was there any possibility of extracting ore from them; and they had scarcely any means of felling timber. Even if you suppose that some implements might have been preserved in the mountains, they must quickly have worn out and vanished, and there would be no more of them until the art of metallurgy had again revived.

\par \textbf{CLEINIAS}
\par   There could not have been.

\par \textbf{ATHENIAN}
\par   In how many generations would this be attained?

\par \textbf{CLEINIAS}
\par   Clearly, not for many generations.

\par \textbf{ATHENIAN}
\par   During this period, and for some time afterwards, all the arts which require iron and brass and the like would disappear.

\par \textbf{CLEINIAS}
\par   Certainly.

\par \textbf{ATHENIAN}
\par   Faction and war would also have died out in those days, and for many reasons.

\par \textbf{CLEINIAS}
\par   How would that be?

\par \textbf{ATHENIAN}
\par   In the first place, the desolation of these primitive men would create in them a feeling of affection and goodwill towards one another; and, secondly, they would have no occasion to quarrel about their subsistence, for they would have pasture in abundance, except just at first, and in some particular cases; and from their pasture-land they would obtain the greater part of their food in a primitive age, having plenty of milk and flesh; moreover they would procure other food by the chase, not to be despised either in quantity or quality. They would also have abundance of clothing, and bedding, and dwellings, and utensils either capable of standing on the fire or not; for the plastic and weaving arts do not require any use of iron:  and God has given these two arts to man in order to provide him with all such things, that, when reduced to the last extremity, the human race may still grow and increase. Hence in those days mankind were not very poor; nor was poverty a cause of difference among them; and rich they could not have been, having neither gold nor silver: —such at that time was their condition. And the community which has neither poverty nor riches will always have the noblest principles; in it there is no insolence or injustice, nor, again, are there any contentions or envyings. And therefore they were good, and also because they were what is called simple-minded; and when they were told about good and evil, they in their simplicity believed what they heard to be very truth and practised it. No one had the wit to suspect another of a falsehood, as men do now; but what they heard about Gods and men they believed to be true, and lived accordingly; and therefore they were in all respects such as we have described them.

\par \textbf{CLEINIAS}
\par   That quite accords with my views, and with those of my friend here.

\par \textbf{ATHENIAN}
\par   Would not many generations living on in a simple manner, although ruder, perhaps, and more ignorant of the arts generally, and in particular of those of land or naval warfare, and likewise of other arts, termed in cities legal practices and party conflicts, and including all conceivable ways of hurting one another in word and deed;—although inferior to those who lived before the deluge, or to the men of our day in these respects, would they not, I say, be simpler and more manly, and also more temperate and altogether more just? The reason has been already explained.

\par \textbf{CLEINIAS}
\par   Very true.

\par \textbf{ATHENIAN}
\par   I should wish you to understand that what has preceded and what is about to follow, has been, and will be said, with the intention of explaining what need the men of that time had of laws, and who was their lawgiver.

\par \textbf{CLEINIAS}
\par   And thus far what you have said has been very well said.

\par \textbf{ATHENIAN}
\par   They could hardly have wanted lawgivers as yet; nothing of that sort was likely to have existed in their days, for they had no letters at this early period; they lived by habit and the customs of their ancestors, as they are called.

\par \textbf{CLEINIAS}
\par   Probably.

\par \textbf{ATHENIAN}
\par   But there was already existing a form of government which, if I am not mistaken, is generally termed a lordship, and this still remains in many places, both among Hellenes and barbarians (compare Arist. Pol. ), and is the government which is declared by Homer to have prevailed among the Cyclopes: —

\par  'They have neither councils nor judgments, but they dwell in hollow caves on the tops of high mountains, and every one gives law to his wife and children, and they do not busy themselves about one another.' (Odyss.)

\par \textbf{CLEINIAS}
\par   That seems to be a charming poet of yours; I have read some other verses of his, which are very clever; but I do not know much of him, for foreign poets are very little read among the Cretans.

\par \textbf{MEGILLUS}
\par   But they are in Lacedaemon, and he appears to be the prince of them all; the manner of life, however, which he describes is not Spartan, but rather Ionian, and he seems quite to confirm what you are saying, when he traces up the ancient state of mankind by the help of tradition to barbarism.

\par \textbf{ATHENIAN}
\par   Yes, he does confirm it; and we may accept his witness to the fact that such forms of government sometimes arise.

\par \textbf{CLEINIAS}
\par   We may.

\par \textbf{ATHENIAN}
\par   And were not such states composed of men who had been dispersed in single habitations and families by the poverty which attended the devastations; and did not the eldest then rule among them, because with them government originated in the authority of a father and a mother, whom, like a flock of birds, they followed, forming one troop under the patriarchal rule and sovereignty of their parents, which of all sovereignties is the most just?

\par \textbf{CLEINIAS}
\par   Very true.

\par \textbf{ATHENIAN}
\par   After this they came together in greater numbers, and increased the size of their cities, and betook themselves to husbandry, first of all at the foot of the mountains, and made enclosures of loose walls and works of defence, in order to keep off wild beasts; thus creating a single large and common habitation.

\par \textbf{CLEINIAS}
\par   Yes; at least we may suppose so.

\par \textbf{ATHENIAN}
\par   There is another thing which would probably happen.

\par \textbf{CLEINIAS}
\par   What?

\par \textbf{ATHENIAN}
\par   When these larger habitations grew up out of the lesser original ones, each of the lesser ones would survive in the larger; every family would be under the rule of the eldest, and, owing to their separation from one another, would have peculiar customs in things divine and human, which they would have received from their several parents who had educated them; and these customs would incline them to order, when the parents had the element of order in their nature, and to courage, when they had the element of courage. And they would naturally stamp upon their children, and upon their children's children, their own likings; and, as we are saying, they would find their way into the larger society, having already their own peculiar laws.

\par \textbf{CLEINIAS}
\par   Certainly.

\par \textbf{ATHENIAN}
\par   And every man surely likes his own laws best, and the laws of others not so well.

\par \textbf{CLEINIAS}
\par   True.

\par \textbf{ATHENIAN}
\par   Then now we seem to have stumbled upon the beginnings of legislation.

\par \textbf{CLEINIAS}
\par   Exactly.

\par \textbf{ATHENIAN}
\par   The next step will be that these persons who have met together, will select some arbiters, who will review the laws of all of them, and will publicly present such as they approve to the chiefs who lead the tribes, and who are in a manner their kings, allowing them to choose those which they think best. These persons will themselves be called legislators, and will appoint the magistrates, framing some sort of aristocracy, or perhaps monarchy, out of the dynasties or lordships, and in this altered state of the government they will live.

\par \textbf{CLEINIAS}
\par   Yes, that would be the natural order of things.

\par \textbf{ATHENIAN}
\par   Then, now let us speak of a third form of government, in which all other forms and conditions of polities and cities concur.

\par \textbf{CLEINIAS}
\par   What is that?

\par \textbf{ATHENIAN}
\par   The form which in fact Homer indicates as following the second. This third form arose when, as he says, Dardanus founded Dardania: —

\par  'For not as yet had the holy Ilium been built on the plain to be a city of speaking men; but they were still dwelling at the foot of many-fountained Ida.'

\par  For indeed, in these verses, and in what he said of the Cyclopes, he speaks the words of God and nature; for poets are a divine race, and often in their strains, by the aid of the Muses and the Graces, they attain truth.

\par \textbf{CLEINIAS}
\par   Yes.

\par \textbf{ATHENIAN}
\par   Then now let us proceed with the rest of our tale, which will probably be found to illustrate in some degree our proposed design: —Shall we do so?

\par \textbf{CLEINIAS}
\par   By all means.

\par \textbf{ATHENIAN}
\par   Ilium was built, when they descended from the mountain, in a large and fair plain, on a sort of low hill, watered by many rivers descending from Ida.

\par \textbf{CLEINIAS}
\par   Such is the tradition.

\par \textbf{ATHENIAN}
\par   And we must suppose this event to have taken place many ages after the deluge?

\par \textbf{ATHENIAN}
\par   A marvellous forgetfulness of the former destruction would appear to have come over them, when they placed their town right under numerous streams flowing from the heights, trusting for their security to not very high hills, either.

\par \textbf{CLEINIAS}
\par   There must have been a long interval, clearly.

\par \textbf{ATHENIAN}
\par   And, as population increased, many other cities would begin to be inhabited.

\par \textbf{CLEINIAS}
\par   Doubtless.

\par \textbf{ATHENIAN}
\par   Those cities made war against Troy—by sea as well as land—for at that time men were ceasing to be afraid of the sea.

\par \textbf{CLEINIAS}
\par   Clearly.

\par \textbf{ATHENIAN}
\par   The Achaeans remained ten years, and overthrew Troy.

\par \textbf{CLEINIAS}
\par   True.

\par \textbf{ATHENIAN}
\par   And during the ten years in which the Achaeans were besieging Ilium, the homes of the besiegers were falling into an evil plight. Their youth revolted; and when the soldiers returned to their own cities and families, they did not receive them properly, and as they ought to have done, and numerous deaths, murders, exiles, were the consequence. The exiles came again, under a new name, no longer Achaeans, but Dorians,—a name which they derived from Dorieus; for it was he who gathered them together. The rest of the story is told by you Lacedaemonians as part of the history of Sparta.

\par \textbf{MEGILLUS}
\par   To be sure.

\par \textbf{ATHENIAN}
\par   Thus, after digressing from the original subject of laws into music and drinking-bouts, the argument has, providentially, come back to the same point, and presents to us another handle. For we have reached the settlement of Lacedaemon; which, as you truly say, is in laws and in institutions the sister of Crete. And we are all the better for the digression, because we have gone through various governments and settlements, and have been present at the foundation of a first, second, and third state, succeeding one another in infinite time. And now there appears on the horizon a fourth state or nation which was once in process of settlement and has continued settled to this day. If, out of all this, we are able to discern what is well or ill settled, and what laws are the salvation and what are the destruction of cities, and what changes would make a state happy, O Megillus and Cleinias, we may now begin again, unless we have some fault to find with the previous discussion.

\par \textbf{MEGILLUS}
\par   If some God, Stranger, would promise us that our new enquiry about legislation would be as good and full as the present, I would go a great way to hear such another, and would think that a day as long as this—and we are now approaching the longest day of the year—was too short for the discussion.

\par \textbf{ATHENIAN}
\par   Then I suppose that we must consider this subject?

\par \textbf{MEGILLUS}
\par   Certainly.

\par \textbf{ATHENIAN}
\par   Let us place ourselves in thought at the moment when Lacedaemon and Argos and Messene and the rest of the Peloponnesus were all in complete subjection, Megillus, to your ancestors; for afterwards, as the legend informs us, they divided their army into three portions, and settled three cities, Argos, Messene, Lacedaemon.

\par \textbf{MEGILLUS}
\par   True.

\par \textbf{ATHENIAN}
\par   Temenus was the king of Argos, Cresphontes of Messene, Procles and Eurysthenes of Lacedaemon.

\par \textbf{MEGILLUS}
\par   Certainly.

\par \textbf{ATHENIAN}
\par   To these kings all the men of that day made oath that they would assist them, if any one subverted their kingdom.

\par \textbf{MEGILLUS}
\par   True.

\par \textbf{ATHENIAN}
\par   But can a kingship be destroyed, or was any other form of government ever destroyed, by any but the rulers themselves? No indeed, by Zeus. Have we already forgotten what was said a little while ago?

\par \textbf{MEGILLUS}
\par   No.

\par \textbf{ATHENIAN}
\par   And may we not now further confirm what was then mentioned? For we have come upon facts which have brought us back again to the same principle; so that, in resuming the discussion, we shall not be enquiring about an empty theory, but about events which actually happened. The case was as follows: —Three royal heroes made oath to three cities which were under a kingly government, and the cities to the kings, that both rulers and subjects should govern and be governed according to the laws which were common to all of them:  the rulers promised that as time and the race went forward they would not make their rule more arbitrary; and the subjects said that, if the rulers observed these conditions, they would never subvert or permit others to subvert those kingdoms; the kings were to assist kings and peoples when injured, and the peoples were to assist peoples and kings in like manner. Is not this the fact?

\par \textbf{MEGILLUS}
\par   Yes.

\par \textbf{ATHENIAN}
\par   And the three states to whom these laws were given, whether their kings or any others were the authors of them, had therefore the greatest security for the maintenance of their constitutions?

\par \textbf{MEGILLUS}
\par   What security?

\par \textbf{ATHENIAN}
\par   That the other two states were always to come to the rescue against a rebellious third.

\par \textbf{MEGILLUS}
\par   True.

\par \textbf{ATHENIAN}
\par   Many persons say that legislators ought to impose such laws as the mass of the people will be ready to receive; but this is just as if one were to command gymnastic masters or physicians to treat or cure their pupils or patients in an agreeable manner.

\par \textbf{MEGILLUS}
\par   Exactly.

\par \textbf{ATHENIAN}
\par   Whereas the physician may often be too happy if he can restore health, and make the body whole, without any very great infliction of pain.

\par \textbf{MEGILLUS}
\par   Certainly.

\par \textbf{ATHENIAN}
\par   There was also another advantage possessed by the men of that day, which greatly lightened the task of passing laws.

\par \textbf{MEGILLUS}
\par   What advantage?

\par \textbf{ATHENIAN}
\par   The legislators of that day, when they equalized property, escaped the great accusation which generally arises in legislation, if a person attempts to disturb the possession of land, or to abolish debts, because he sees that without this reform there can never be any real equality. Now, in general, when the legislator attempts to make a new settlement of such matters, every one meets him with the cry, that 'he is not to disturb vested interests,'—declaring with imprecations that he is introducing agrarian laws and cancelling of debts, until a man is at his wits' end; whereas no one could quarrel with the Dorians for distributing the land,—there was nothing to hinder them; and as for debts, they had none which were considerable or of old standing.

\par \textbf{MEGILLUS}
\par   Very true.

\par \textbf{ATHENIAN}
\par   But then, my good friends, why did the settlement and legislation of their country turn out so badly?

\par \textbf{MEGILLUS}
\par   How do you mean; and why do you blame them?

\par \textbf{ATHENIAN}
\par   There were three kingdoms, and of these, two quickly corrupted their original constitution and laws, and the only one which remained was the Spartan.

\par \textbf{MEGILLUS}
\par   The question which you ask is not easily answered.

\par \textbf{ATHENIAN}
\par   And yet must be answered when we are enquiring about laws, this being our old man's sober game of play, whereby we beguile the way, as I was saying when we first set out on our journey.

\par \textbf{MEGILLUS}
\par   Certainly; and we must find out why this was.

\par \textbf{ATHENIAN}
\par   What laws are more worthy of our attention than those which have regulated such cities? or what settlements of states are greater or more famous?

\par \textbf{MEGILLUS}
\par   I know of none.

\par \textbf{ATHENIAN}
\par   Can we doubt that your ancestors intended these institutions not only for the protection of Peloponnesus, but of all the Hellenes, in case they were attacked by the barbarian? For the inhabitants of the region about Ilium, when they provoked by their insolence the Trojan war, relied upon the power of the Assyrians and the Empire of Ninus, which still existed and had a great prestige; the people of those days fearing the united Assyrian Empire just as we now fear the Great King. And the second capture of Troy was a serious offence against them, because Troy was a portion of the Assyrian Empire. To meet the danger the single army was distributed between three cities by the royal brothers, sons of Heracles,—a fair device, as it seemed, and a far better arrangement than the expedition against Troy. For, firstly, the people of that day had, as they thought, in the Heraclidae better leaders than the Pelopidae; in the next place, they considered that their army was superior in valour to that which went against Troy; for, although the latter conquered the Trojans, they were themselves conquered by the Heraclidae—Achaeans by Dorians. May we not suppose that this was the intention with which the men of those days framed the constitutions of their states?

\par \textbf{MEGILLUS}
\par   Quite true.

\par \textbf{ATHENIAN}
\par   And would not men who had shared with one another many dangers, and were governed by a single race of royal brothers, and had taken the advice of oracles, and in particular of the Delphian Apollo, be likely to think that such states would be firmly and lastingly established?

\par \textbf{MEGILLUS}
\par   Of course they would.

\par \textbf{ATHENIAN}
\par   Yet these institutions, of which such great expectations were entertained, seem to have all rapidly vanished away; with the exception, as I was saying, of that small part of them which existed in your land. And this third part has never to this day ceased warring against the two others; whereas, if the original idea had been carried out, and they had agreed to be one, their power would have been invincible in war.

\par \textbf{MEGILLUS}
\par   No doubt.

\par \textbf{ATHENIAN}
\par   But what was the ruin of this glorious confederacy? Here is a subject well worthy of consideration.

\par \textbf{MEGILLUS}
\par   Certainly, no one will ever find more striking instances of laws or governments being the salvation or destruction of great and noble interests, than are here presented to his view.

\par \textbf{ATHENIAN}
\par   Then now we seem to have happily arrived at a real and important question.

\par \textbf{MEGILLUS}
\par   Very true.

\par \textbf{ATHENIAN}
\par   Did you never remark, sage friend, that all men, and we ourselves at this moment, often fancy that they see some beautiful thing which might have effected wonders if any one had only known how to make a right use of it in some way; and yet this mode of looking at things may turn out after all to be a mistake, and not according to nature, either in our own case or in any other?

\par \textbf{MEGILLUS}
\par   To what are you referring, and what do you mean?

\par \textbf{ATHENIAN}
\par   I was thinking of my own admiration of the aforesaid Heracleid expedition, which was so noble, and might have had such wonderful results for the Hellenes, if only rightly used; and I was just laughing at myself.

\par \textbf{MEGILLUS}
\par   But were you not right and wise in speaking as you did, and we in assenting to you?

\par \textbf{ATHENIAN}
\par   Perhaps; and yet I cannot help observing that any one who sees anything great or powerful, immediately has the feeling that—'If the owner only knew how to use his great and noble possession, how happy would he be, and what great results would he achieve!'

\par \textbf{MEGILLUS}
\par   And would he not be justified?

\par \textbf{ATHENIAN}
\par   Reflect; in what point of view does this sort of praise appear just:  First, in reference to the question in hand: —If the then commanders had known how to arrange their army properly, how would they have attained success? Would not this have been the way? They would have bound them all firmly together and preserved them for ever, giving them freedom and dominion at pleasure, combined with the power of doing in the whole world, Hellenic and barbarian, whatever they and their descendants desired. What other aim would they have had?

\par \textbf{MEGILLUS}
\par   Very good.

\par \textbf{ATHENIAN}
\par   Suppose any one were in the same way to express his admiration at the sight of great wealth or family honour, or the like, he would praise them under the idea that through them he would attain either all or the greater and chief part of what he desires.

\par \textbf{MEGILLUS}
\par   He would.

\par \textbf{ATHENIAN}
\par   Well, now, and does not the argument show that there is one common desire of all mankind?

\par \textbf{MEGILLUS}
\par   What is it?

\par \textbf{ATHENIAN}
\par   The desire which a man has, that all things, if possible,—at any rate, things human,—may come to pass in accordance with his soul's desire.

\par \textbf{MEGILLUS}
\par   Certainly.

\par \textbf{ATHENIAN}
\par   And having this desire always, and at every time of life, in youth, in manhood, in age, he cannot help always praying for the fulfilment of it.

\par \textbf{MEGILLUS}
\par   No doubt.

\par \textbf{ATHENIAN}
\par   And we join in the prayers of our friends, and ask for them what they ask for themselves.

\par \textbf{MEGILLUS}
\par   We do.

\par \textbf{ATHENIAN}
\par   Dear is the son to the father—the younger to the elder.

\par \textbf{MEGILLUS}
\par   Of course.

\par \textbf{ATHENIAN}
\par   And yet the son often prays to obtain things which the father prays that he may not obtain.

\par \textbf{MEGILLUS}
\par   When the son is young and foolish, you mean?

\par \textbf{ATHENIAN}
\par   Yes; or when the father, in the dotage of age or the heat of youth, having no sense of right and justice, prays with fervour, under the influence of feelings akin to those of Theseus when he cursed the unfortunate Hippolytus, do you imagine that the son, having a sense of right and justice, will join in his father's prayers?

\par \textbf{MEGILLUS}
\par   I understand you to mean that a man should not desire or be in a hurry to have all things according to his wish, for his wish may be at variance with his reason. But every state and every individual ought to pray and strive for wisdom.

\par \textbf{ATHENIAN}
\par   Yes; and I remember, and you will remember, what I said at first, that a statesman and legislator ought to ordain laws with a view to wisdom; while you were arguing that the good lawgiver ought to order all with a view to war. And to this I replied that there were four virtues, but that upon your view one of them only was the aim of legislation; whereas you ought to regard all virtue, and especially that which comes first, and is the leader of all the rest—I mean wisdom and mind and opinion, having affection and desire in their train. And now the argument returns to the same point, and I say once more, in jest if you like, or in earnest if you like, that the prayer of a fool is full of danger, being likely to end in the opposite of what he desires. And if you would rather receive my words in earnest, I am willing that you should; and you will find, I suspect, as I have said already, that not cowardice was the cause of the ruin of the Dorian kings and of their whole design, nor ignorance of military matters, either on the part of the rulers or of their subjects; but their misfortunes were due to their general degeneracy, and especially to their ignorance of the most important human affairs. That was then, and is still, and always will be the case, as I will endeavour, if you will allow me, to make out and demonstrate as well as I am able to you who are my friends, in the course of the argument.

\par \textbf{CLEINIAS}
\par   Pray go on, Stranger;—compliments are troublesome, but we will show, not in word but in deed, how greatly we prize your words, for we will give them our best attention; and that is the way in which a freeman best shows his approval or disapproval.

\par \textbf{MEGILLUS}
\par   Excellent, Cleinias; let us do as you say.

\par \textbf{CLEINIAS}
\par   By all means, if Heaven wills. Go on.

\par \textbf{ATHENIAN}
\par   Well, then, proceeding in the same train of thought, I say that the greatest ignorance was the ruin of the Dorian power, and that now, as then, ignorance is ruin. And if this be true, the legislator must endeavour to implant wisdom in states, and banish ignorance to the utmost of his power.

\par \textbf{CLEINIAS}
\par   That is evident.

\par \textbf{ATHENIAN}
\par   Then now consider what is really the greatest ignorance. I should like to know whether you and Megillus would agree with me in what I am about to say; for my opinion is—

\par \textbf{CLEINIAS}
\par   What?

\par \textbf{ATHENIAN}
\par   That the greatest ignorance is when a man hates that which he nevertheless thinks to be good and noble, and loves and embraces that which he knows to be unrighteous and evil. This disagreement between the sense of pleasure and the judgment of reason in the soul is, in my opinion, the worst ignorance; and also the greatest, because affecting the great mass of the human soul; for the principle which feels pleasure and pain in the individual is like the mass or populace in a state. And when the soul is opposed to knowledge, or opinion, or reason, which are her natural lords, that I call folly, just as in the state, when the multitude refuses to obey their rulers and the laws; or, again, in the individual, when fair reasonings have their habitation in the soul and yet do no good, but rather the reverse of good. All these cases I term the worst ignorance, whether in individuals or in states. You will understand, Stranger, that I am speaking of something which is very different from the ignorance of handicraftsmen.

\par \textbf{CLEINIAS}
\par   Yes, my friend, we understand and agree.

\par \textbf{ATHENIAN}
\par   Let us, then, in the first place declare and affirm that the citizen who does not know these things ought never to have any kind of authority entrusted to him:  he must be stigmatized as ignorant, even though he be versed in calculation and skilled in all sorts of accomplishments, and feats of mental dexterity; and the opposite are to be called wise, even although, in the words of the proverb, they know neither how to read nor how to swim; and to them, as to men of sense, authority is to be committed. For, O my friends, how can there be the least shadow of wisdom when there is no harmony? There is none; but the noblest and greatest of harmonies may be truly said to be the greatest wisdom; and of this he is a partaker who lives according to reason; whereas he who is devoid of reason is the destroyer of his house and the very opposite of a saviour of the state:  he is utterly ignorant of political wisdom. Let this, then, as I was saying, be laid down by us.

\par \textbf{CLEINIAS}
\par   Let it be so laid down.

\par \textbf{ATHENIAN}
\par   I suppose that there must be rulers and subjects in states?

\par \textbf{CLEINIAS}
\par   Certainly.

\par \textbf{ATHENIAN}
\par   And what are the principles on which men rule and obey in cities, whether great or small; and similarly in families? What are they, and how many in number? Is there not one claim of authority which is always just,—that of fathers and mothers and in general of progenitors to rule over their offspring?

\par \textbf{CLEINIAS}
\par   There is.

\par \textbf{ATHENIAN}
\par   Next follows the principle that the noble should rule over the ignoble; and, thirdly, that the elder should rule and the younger obey?

\par \textbf{CLEINIAS}
\par   To be sure.

\par \textbf{ATHENIAN}
\par   And, fourthly, that slaves should be ruled, and their masters rule?

\par \textbf{CLEINIAS}
\par   Of course.

\par \textbf{ATHENIAN}
\par   Fifthly, if I am not mistaken, comes the principle that the stronger shall rule, and the weaker be ruled?

\par \textbf{CLEINIAS}
\par   That is a rule not to be disobeyed.

\par \textbf{ATHENIAN}
\par   Yes, and a rule which prevails very widely among all creatures, and is according to nature, as the Theban poet Pindar once said; and the sixth principle, and the greatest of all, is, that the wise should lead and command, and the ignorant follow and obey; and yet, O thou most wise Pindar, as I should reply him, this surely is not contrary to nature, but according to nature, being the rule of law over willing subjects, and not a rule of compulsion.

\par \textbf{CLEINIAS}
\par   Most true.

\par \textbf{ATHENIAN}
\par   There is a seventh kind of rule which is awarded by lot, and is dear to the Gods and a token of good fortune:  he on whom the lot falls is a ruler, and he who fails in obtaining the lot goes away and is the subject; and this we affirm to be quite just.

\par \textbf{CLEINIAS}
\par   Certainly.

\par \textbf{ATHENIAN}
\par   'Then now,' as we say playfully to any of those who lightly undertake the making of laws, 'you see, legislator, the principles of government, how many they are, and that they are naturally opposed to each other. There we have discovered a fountain-head of seditions, to which you must attend. And, first, we will ask you to consider with us, how and in what respect the kings of Argos and Messene violated these our maxims, and ruined themselves and the great and famous Hellenic power of the olden time. Was it because they did not know how wisely Hesiod spoke when he said that the half is often more than the whole? His meaning was, that when to take the whole would be dangerous, and to take the half would be the safe and moderate course, then the moderate or better was more than the immoderate or worse.'

\par \textbf{CLEINIAS}
\par   Very true.

\par \textbf{ATHENIAN}
\par   And may we suppose this immoderate spirit to be more fatal when found among kings than when among peoples?

\par \textbf{CLEINIAS}
\par   The probability is that ignorance will be a disorder especially prevalent among kings, because they lead a proud and luxurious life.

\par \textbf{ATHENIAN}
\par   Is it not palpable that the chief aim of the kings of that time was to get the better of the established laws, and that they were not in harmony with the principles which they had agreed to observe by word and oath? This want of harmony may have had the appearance of wisdom, but was really, as we assert, the greatest ignorance, and utterly overthrew the whole empire by dissonance and harsh discord.

\par \textbf{CLEINIAS}
\par   Very likely.

\par \textbf{ATHENIAN}
\par   Good; and what measures ought the legislator to have then taken in order to avert this calamity? Truly there is no great wisdom in knowing, and no great difficulty in telling, after the evil has happened; but to have foreseen the remedy at the time would have taken a much wiser head than ours.

\par \textbf{MEGILLUS}
\par   What do you mean?

\par \textbf{ATHENIAN}
\par   Any one who looks at what has occurred with you Lacedaemonians, Megillus, may easily know and may easily say what ought to have been done at that time.

\par \textbf{MEGILLUS}
\par   Speak a little more clearly.

\par \textbf{ATHENIAN}
\par   Nothing can be clearer than the observation which I am about to make.

\par \textbf{MEGILLUS}
\par   What is it?

\par \textbf{ATHENIAN}
\par   That if any one gives too great a power to anything, too large a sail to a vessel, too much food to the body, too much authority to the mind, and does not observe the mean, everything is overthrown, and, in the wantonness of excess, runs in the one case to disorders, and in the other to injustice, which is the child of excess. I mean to say, my dear friends, that there is no soul of man, young and irresponsible, who will be able to sustain the temptation of arbitrary power—no one who will not, under such circumstances, become filled with folly, that worst of diseases, and be hated by his nearest and dearest friends:  when this happens his kingdom is undermined, and all his power vanishes from him. And great legislators who know the mean should take heed of the danger. As far as we can guess at this distance of time, what happened was as follows: —

\par \textbf{MEGILLUS}
\par   What?

\par \textbf{ATHENIAN}
\par   A God, who watched over Sparta, seeing into the future, gave you two families of kings instead of one; and thus brought you more within the limits of moderation. In the next place, some human wisdom mingled with divine power, observing that the constitution of your government was still feverish and excited, tempered your inborn strength and pride of birth with the moderation which comes of age, making the power of your twenty-eight elders equal with that of the kings in the most important matters. But your third saviour, perceiving that your government was still swelling and foaming, and desirous to impose a curb upon it, instituted the Ephors, whose power he made to resemble that of magistrates elected by lot; and by this arrangement the kingly office, being compounded of the right elements and duly moderated, was preserved, and was the means of preserving all the rest. Since, if there had been only the original legislators, Temenus, Cresphontes, and their contemporaries, as far as they were concerned not even the portion of Aristodemus would have been preserved; for they had no proper experience in legislation, or they would surely not have imagined that oaths would moderate a youthful spirit invested with a power which might be converted into a tyranny. Now that God has instructed us what sort of government would have been or will be lasting, there is no wisdom, as I have already said, in judging after the event; there is no difficulty in learning from an example which has already occurred. But if any one could have foreseen all this at the time, and had been able to moderate the government of the three kingdoms and unite them into one, he might have saved all the excellent institutions which were then conceived; and no Persian or any other armament would have dared to attack us, or would have regarded Hellas as a power to be despised.

\par \textbf{CLEINIAS}
\par   True.

\par \textbf{ATHENIAN}
\par   There was small credit to us, Cleinias, in defeating them; and the discredit was, not that the conquerors did not win glorious victories both by land and sea, but what, in my opinion, brought discredit was, first of all, the circumstance that of the three cities one only fought on behalf of Hellas, and the two others were so utterly good for nothing that the one was waging a mighty war against Lacedaemon, and was thus preventing her from rendering assistance, while the city of Argos, which had the precedence at the time of the distribution, when asked to aid in repelling the barbarian, would not answer to the call, or give aid. Many things might be told about Hellas in connexion with that war which are far from honourable; nor, indeed, can we rightly say that Hellas repelled the invader; for the truth is, that unless the Athenians and Lacedaemonians, acting in concert, had warded off the impending yoke, all the tribes of Hellas would have been fused in a chaos of Hellenes mingling with one another, of barbarians mingling with Hellenes, and Hellenes with barbarians; just as nations who are now subject to the Persian power, owing to unnatural separations and combinations of them, are dispersed and scattered, and live miserably. These, Cleinias and Megillus, are the reproaches which we have to make against statesmen and legislators, as they are called, past and present, if we would analyse the causes of their failure, and find out what else might have been done. We said, for instance, just now, that there ought to be no great and unmixed powers; and this was under the idea that a state ought to be free and wise and harmonious, and that a legislator ought to legislate with a view to this end. Nor is there any reason to be surprised at our continually proposing aims for the legislator which appear not to be always the same; but we should consider when we say that temperance is to be the aim, or wisdom is to be the aim, or friendship is to be the aim, that all these aims are really the same; and if so, a variety in the modes of expression ought not to disturb us.

\par \textbf{CLEINIAS}
\par   Let us resume the argument in that spirit. And now, speaking of friendship and wisdom and freedom, I wish that you would tell me at what, in your opinion, the legislator should aim.

\par \textbf{ATHENIAN}
\par   Hear me, then:  there are two mother forms of states from which the rest may be truly said to be derived; and one of them may be called monarchy and the other democracy:  the Persians have the highest form of the one, and we of the other; almost all the rest, as I was saying, are variations of these. Now, if you are to have liberty and the combination of friendship with wisdom, you must have both these forms of government in a measure; the argument emphatically declares that no city can be well governed which is not made up of both.

\par \textbf{CLEINIAS}
\par   Impossible.

\par \textbf{ATHENIAN}
\par   Neither the one, if it be exclusively and excessively attached to monarchy, nor the other, if it be similarly attached to freedom, observes moderation; but your states, the Laconian and Cretan, have more of it; and the same was the case with the Athenians and Persians of old time, but now they have less. Shall I tell you why?

\par \textbf{CLEINIAS}
\par   By all means, if it will tend to elucidate our subject.

\par \textbf{ATHENIAN}
\par   Hear, then: —There was a time when the Persians had more of the state which is a mean between slavery and freedom. In the reign of Cyrus they were freemen and also lords of many others:  the rulers gave a share of freedom to the subjects, and being treated as equals, the soldiers were on better terms with their generals, and showed themselves more ready in the hour of danger. And if there was any wise man among them, who was able to give good counsel, he imparted his wisdom to the public; for the king was not jealous, but allowed him full liberty of speech, and gave honour to those who could advise him in any matter. And the nation waxed in all respects, because there was freedom and friendship and communion of mind among them.

\par \textbf{CLEINIAS}
\par   That certainly appears to have been the case.

\par \textbf{ATHENIAN}
\par   How, then, was this advantage lost under Cambyses, and again recovered under Darius? Shall I try to divine?

\par \textbf{CLEINIAS}
\par   The enquiry, no doubt, has a bearing upon our subject.

\par \textbf{ATHENIAN}
\par   I imagine that Cyrus, though a great and patriotic general, had never given his mind to education, and never attended to the order of his household.

\par \textbf{CLEINIAS}
\par   What makes you say so?

\par \textbf{ATHENIAN}
\par   I think that from his youth upwards he was a soldier, and entrusted the education of his children to the women; and they brought them up from their childhood as the favourites of fortune, who were blessed already, and needed no more blessings. They thought that they were happy enough, and that no one should be allowed to oppose them in any way, and they compelled every one to praise all that they said or did. This was how they brought them up.

\par \textbf{CLEINIAS}
\par   A splendid education truly!

\par \textbf{ATHENIAN}
\par   Such an one as women were likely to give them, and especially princesses who had recently grown rich, and in the absence of the men, too, who were occupied in wars and dangers, and had no time to look after them.

\par \textbf{CLEINIAS}
\par   What would you expect?

\par \textbf{ATHENIAN}
\par   Their father had possessions of cattle and sheep, and many herds of men and other animals, but he did not consider that those to whom he was about to make them over were not trained in his own calling, which was Persian; for the Persians are shepherds—sons of a rugged land, which is a stern mother, and well fitted to produce a sturdy race able to live in the open air and go without sleep, and also to fight, if fighting is required (compare Arist. Pol.). He did not observe that his sons were trained differently; through the so-called blessing of being royal they were educated in the Median fashion by women and eunuchs, which led to their becoming such as people do become when they are brought up unreproved. And so, after the death of Cyrus, his sons, in the fulness of luxury and licence, took the kingdom, and first one slew the other because he could not endure a rival; and, afterwards, the slayer himself, mad with wine and brutality, lost his kingdom through the Medes and the Eunuch, as they called him, who despised the folly of Cambyses.

\par \textbf{CLEINIAS}
\par   So runs the tale, and such probably were the facts.

\par \textbf{ATHENIAN}
\par   Yes; and the tradition says, that the empire came back to the Persians, through Darius and the seven chiefs.

\par \textbf{CLEINIAS}
\par   True.

\par \textbf{ATHENIAN}
\par   Let us note the rest of the story. Observe, that Darius was not the son of a king, and had not received a luxurious education. When he came to the throne, being one of the seven, he divided the country into seven portions, and of this arrangement there are some shadowy traces still remaining; he made laws upon the principle of introducing universal equality in the order of the state, and he embodied in his laws the settlement of the tribute which Cyrus promised,—thus creating a feeling of friendship and community among all the Persians, and attaching the people to him with money and gifts. Hence his armies cheerfully acquired for him countries as large as those which Cyrus had left behind him. Darius was succeeded by his son Xerxes; and he again was brought up in the royal and luxurious fashion. Might we not most justly say:  'O Darius, how came you to bring up Xerxes in the same way in which Cyrus brought up Cambyses, and not to see his fatal mistake?' For Xerxes, being the creation of the same education, met with much the same fortune as Cambyses; and from that time until now there has never been a really great king among the Persians, although they are all called Great. And their degeneracy is not to be attributed to chance, as I maintain; the reason is rather the evil life which is generally led by the sons of very rich and royal persons; for never will boy or man, young or old, excel in virtue, who has been thus educated. And this, I say, is what the legislator has to consider, and what at the present moment has to be considered by us. Justly may you, O Lacedaemonians, be praised, in that you do not give special honour or a special education to wealth rather than to poverty, or to a royal rather than to a private station, where the divine and inspired lawgiver has not originally commanded them to be given. For no man ought to have pre-eminent honour in a state because he surpasses others in wealth, any more than because he is swift of foot or fair or strong, unless he have some virtue in him; nor even if he have virtue, unless he have this particular virtue of temperance.

\par \textbf{MEGILLUS}
\par   What do you mean, Stranger?

\par \textbf{ATHENIAN}
\par   I suppose that courage is a part of virtue?

\par \textbf{MEGILLUS}
\par   To be sure.

\par \textbf{ATHENIAN}
\par   Then, now hear and judge for yourself: —Would you like to have for a fellow-lodger or neighbour a very courageous man, who had no control over himself?

\par \textbf{MEGILLUS}
\par   Heaven forbid!

\par \textbf{ATHENIAN}
\par   Or an artist, who was clever in his profession, but a rogue?

\par \textbf{MEGILLUS}
\par   Certainly not.

\par \textbf{ATHENIAN}
\par   And surely justice does not grow apart from temperance?

\par \textbf{MEGILLUS}
\par   Impossible.

\par \textbf{ATHENIAN}
\par   Any more than our pattern wise man, whom we exhibited as having his pleasures and pains in accordance with and corresponding to true reason, can be intemperate?

\par \textbf{MEGILLUS}
\par   No.

\par \textbf{ATHENIAN}
\par   There is a further consideration relating to the due and undue award of honours in states.

\par \textbf{MEGILLUS}
\par   What is it?

\par \textbf{ATHENIAN}
\par   I should like to know whether temperance without the other virtues, existing alone in the soul of man, is rightly to be praised or blamed?

\par \textbf{MEGILLUS}
\par   I cannot tell.

\par \textbf{ATHENIAN}
\par   And that is the best answer; for whichever alternative you had chosen, I think that you would have gone wrong.

\par \textbf{MEGILLUS}
\par   I am fortunate.

\par \textbf{ATHENIAN}
\par   Very good; a quality, which is a mere appendage of things which can be praised or blamed, does not deserve an expression of opinion, but is best passed over in silence.

\par \textbf{MEGILLUS}
\par   You are speaking of temperance?

\par \textbf{ATHENIAN}
\par   Yes; but of the other virtues, that which having this appendage is also most beneficial, will be most deserving of honour, and next that which is beneficial in the next degree; and so each of them will be rightly honoured according to a regular order.

\par \textbf{MEGILLUS}
\par   True.

\par \textbf{ATHENIAN}
\par   And ought not the legislator to determine these classes?

\par \textbf{MEGILLUS}
\par   Certainly he should.

\par \textbf{ATHENIAN}
\par   Suppose that we leave to him the arrangement of details. But the general division of laws according to their importance into a first and second and third class, we who are lovers of law may make ourselves.

\par \textbf{MEGILLUS}
\par   Very good.

\par \textbf{ATHENIAN}
\par   We maintain, then, that a State which would be safe and happy, as far as the nature of man allows, must and ought to distribute honour and dishonour in the right way. And the right way is to place the goods of the soul first and highest in the scale, always assuming temperance to be the condition of them; and to assign the second place to the goods of the body; and the third place to money and property. And if any legislator or state departs from this rule by giving money the place of honour, or in any way preferring that which is really last, may we not say, that he or the state is doing an unholy and unpatriotic thing?

\par \textbf{MEGILLUS}
\par   Yes; let that be plainly declared.

\par \textbf{ATHENIAN}
\par   The consideration of the Persian governments led us thus far to enlarge. We remarked that the Persians grew worse and worse. And we affirm the reason of this to have been, that they too much diminished the freedom of the people, and introduced too much of despotism, and so destroyed friendship and community of feeling. And when there is an end of these, no longer do the governors govern on behalf of their subjects or of the people, but on behalf of themselves; and if they think that they can gain ever so small an advantage for themselves, they devastate cities, and send fire and desolation among friendly races. And as they hate ruthlessly and horribly, so are they hated; and when they want the people to fight for them, they find no community of feeling or willingness to risk their lives on their behalf; their untold myriads are useless to them on the field of battle, and they think that their salvation depends on the employment of mercenaries and strangers whom they hire, as if they were in want of more men. And they cannot help being stupid, since they proclaim by their actions that the ordinary distinctions of right and wrong which are made in a state are a trifle, when compared with gold and silver.

\par \textbf{MEGILLUS}
\par   Quite true.

\par \textbf{ATHENIAN}
\par   And now enough of the Persians, and their present mal-administration of their government, which is owing to the excess of slavery and despotism among them.

\par \textbf{MEGILLUS}
\par   Good.

\par \textbf{ATHENIAN}
\par   Next, we must pass in review the government of Attica in like manner, and from this show that entire freedom and the absence of all superior authority is not by any means so good as government by others when properly limited, which was our ancient Athenian constitution at the time when the Persians made their attack on Hellas, or, speaking more correctly, on the whole continent of Europe. There were four classes, arranged according to a property census, and reverence was our queen and mistress, and made us willing to live in obedience to the laws which then prevailed. Also the vastness of the Persian armament, both by sea and on land, caused a helpless terror, which made us more and more the servants of our rulers and of the laws; and for all these reasons an exceeding harmony prevailed among us. About ten years before the naval engagement at Salamis, Datis came, leading a Persian host by command of Darius, which was expressly directed against the Athenians and Eretrians, having orders to carry them away captive; and these orders he was to execute under pain of death. Now Datis and his myriads soon became complete masters of Eretria, and he sent a fearful report to Athens that no Eretrian had escaped him; for the soldiers of Datis had joined hands and netted the whole of Eretria. And this report, whether well or ill founded, was terrible to all the Hellenes, and above all to the Athenians, and they dispatched embassies in all directions, but no one was willing to come to their relief, with the exception of the Lacedaemonians; and they, either because they were detained by the Messenian war, which was then going on, or for some other reason of which we are not told, came a day too late for the battle of Marathon. After a while, the news arrived of mighty preparations being made, and innumerable threats came from the king. Then, as time went on, a rumour reached us that Darius had died, and that his son, who was young and hot-headed, had come to the throne and was persisting in his design. The Athenians were under the impression that the whole expedition was directed against them, in consequence of the battle of Marathon; and hearing of the bridge over the Hellespont, and the canal of Athos, and the host of ships, considering that there was no salvation for them either by land or by sea, for there was no one to help them, and remembering that in the first expedition, when the Persians destroyed Eretria, no one came to their help, or would risk the danger of an alliance with them, they thought that this would happen again, at least on land; nor, when they looked to the sea, could they descry any hope of salvation; for they were attacked by a thousand vessels and more. One chance of safety remained, slight indeed and desperate, but their only one. They saw that on the former occasion they had gained a seemingly impossible victory, and borne up by this hope, they found that their only refuge was in themselves and in the Gods. All these things created in them the spirit of friendship; there was the fear of the moment, and there was that higher fear, which they had acquired by obedience to their ancient laws, and which I have several times in the preceding discourse called reverence, of which the good man ought to be a willing servant, and of which the coward is independent and fearless. If this fear had not possessed them, they would never have met the enemy, or defended their temples and sepulchres and their country, and everything that was near and dear to them, as they did; but little by little they would have been all scattered and dispersed.

\par \textbf{MEGILLUS}
\par   Your words, Athenian, are quite true, and worthy of yourself and of your country.

\par \textbf{ATHENIAN}
\par   They are true, Megillus; and to you, who have inherited the virtues of your ancestors, I may properly speak of the actions of that day. And I would wish you and Cleinias to consider whether my words have not also a bearing on legislation; for I am not discoursing only for the pleasure of talking, but for the argument's sake. Please to remark that the experience both of ourselves and the Persians was, in a certain sense, the same; for as they led their people into utter servitude, so we too led ours into all freedom. And now, how shall we proceed? for I would like you to observe that our previous arguments have good deal to say for themselves.

\par \textbf{MEGILLUS}
\par   True; but I wish that you would give us a fuller explanation.

\par \textbf{ATHENIAN}
\par   I will. Under the ancient laws, my friends, the people was not as now the master, but rather the willing servant of the laws.

\par \textbf{MEGILLUS}
\par   What laws do you mean?

\par \textbf{ATHENIAN}
\par   In the first place, let us speak of the laws about music,—that is to say, such music as then existed—in order that we may trace the growth of the excess of freedom from the beginning. Now music was early divided among us into certain kinds and manners. One sort consisted of prayers to the Gods, which were called hymns; and there was another and opposite sort called lamentations, and another termed paeans, and another, celebrating the birth of Dionysus, called, I believe, 'dithyrambs.' And they used the actual word 'laws,' or nomoi, for another kind of song; and to this they added the term 'citharoedic.' All these and others were duly distinguished, nor were the performers allowed to confuse one style of music with another. And the authority which determined and gave judgment, and punished the disobedient, was not expressed in a hiss, nor in the most unmusical shouts of the multitude, as in our days, nor in applause and clapping of hands. But the directors of public instruction insisted that the spectators should listen in silence to the end; and boys and their tutors, and the multitude in general, were kept quiet by a hint from a stick. Such was the good order which the multitude were willing to observe; they would never have dared to give judgment by noisy cries. And then, as time went on, the poets themselves introduced the reign of vulgar and lawless innovation. They were men of genius, but they had no perception of what is just and lawful in music; raging like Bacchanals and possessed with inordinate delights—mingling lamentations with hymns, and paeans with dithyrambs; imitating the sounds of the flute on the lyre, and making one general confusion; ignorantly affirming that music has no truth, and, whether good or bad, can only be judged of rightly by the pleasure of the hearer (compare Republic). And by composing such licentious works, and adding to them words as licentious, they have inspired the multitude with lawlessness and boldness, and made them fancy that they can judge for themselves about melody and song. And in this way the theatres from being mute have become vocal, as though they had understanding of good and bad in music and poetry; and instead of an aristocracy, an evil sort of theatrocracy has grown up (compare Arist. Pol.). For if the democracy which judged had only consisted of educated persons, no fatal harm would have been done; but in music there first arose the universal conceit of omniscience and general lawlessness;—freedom came following afterwards, and men, fancying that they knew what they did not know, had no longer any fear, and the absence of fear begets shamelessness. For what is this shamelessness, which is so evil a thing, but the insolent refusal to regard the opinion of the better by reason of an over-daring sort of liberty?

\par \textbf{MEGILLUS}
\par   Very true.

\par \textbf{ATHENIAN}
\par   Consequent upon this freedom comes the other freedom, of disobedience to rulers (compare Republic); and then the attempt to escape the control and exhortation of father, mother, elders, and when near the end, the control of the laws also; and at the very end there is the contempt of oaths and pledges, and no regard at all for the Gods,—herein they exhibit and imitate the old so-called Titanic nature, and come to the same point as the Titans when they rebelled against God, leading a life of endless evils. But why have I said all this? I ask, because the argument ought to be pulled up from time to time, and not be allowed to run away, but held with bit and bridle, and then we shall not, as the proverb says, fall off our ass. Let us then once more ask the question, To what end has all this been said?

\par \textbf{MEGILLUS}
\par   Very good.

\par \textbf{ATHENIAN}
\par   This, then, has been said for the sake—

\par \textbf{MEGILLUS}
\par   Of what?

\par \textbf{ATHENIAN}
\par   We were maintaining that the lawgiver ought to have three things in view:  first, that the city for which he legislates should be free; and secondly, be at unity with herself; and thirdly, should have understanding;—these were our principles, were they not?

\par \textbf{MEGILLUS}
\par   Certainly.

\par \textbf{ATHENIAN}
\par   With a view to this we selected two kinds of government, the one the most despotic, and the other the most free; and now we are considering which of them is the right form:  we took a mean in both cases, of despotism in the one, and of liberty in the other, and we saw that in a mean they attained their perfection; but that when they were carried to the extreme of either, slavery or licence, neither party were the gainers.

\par \textbf{MEGILLUS}
\par   Very true.

\par \textbf{ATHENIAN}
\par   And that was our reason for considering the settlement of the Dorian army, and of the city built by Dardanus at the foot of the mountains, and the removal of cities to the seashore, and of our mention of the first men, who were the survivors of the deluge. And all that was previously said about music and drinking, and what preceded, was said with the view of seeing how a state might be best administered, and how an individual might best order his own life. And now, Megillus and Cleinias, how can we put to the proof the value of our words?

\par \textbf{CLEINIAS}
\par   Stranger, I think that I see how a proof of their value may be obtained. This discussion of ours appears to me to have been singularly fortunate, and just what I at this moment want; most auspiciously have you and my friend Megillus come in my way. For I will tell you what has happened to me; and I regard the coincidence as a sort of omen. The greater part of Crete is going to send out a colony, and they have entrusted the management of the affair to the Cnosians; and the Cnosian government to me and nine others. And they desire us to give them any laws which we please, whether taken from the Cretan model or from any other; and they do not mind about their being foreign if they are better. Grant me then this favour, which will also be a gain to yourselves: —Let us make a selection from what has been said, and then let us imagine a State of which we will suppose ourselves to be the original founders. Thus we shall proceed with our enquiry, and, at the same time, I may have the use of the framework which you are constructing, for the city which is in contemplation.

\par \textbf{ATHENIAN}
\par   Good news, Cleinias; if Megillus has no objection, you may be sure that I will do all in my power to please you.

\par \textbf{CLEINIAS}
\par   Thank you.

\par \textbf{MEGILLUS}
\par   And so will I.

\par \textbf{CLEINIAS}
\par   Excellent; and now let us begin to frame the State.

\par 
\section{
      BOOK IV.
    }
\par \textbf{ATHENIAN}
\par   And now, what will this city be? I do not mean to ask what is or will hereafter be the name of the place; that may be determined by the accident of locality or of the original settlement—a river or fountain, or some local deity may give the sanction of a name to the newly-founded city; but I do want to know what the situation is, whether maritime or inland.

\par \textbf{CLEINIAS}
\par   I should imagine, Stranger, that the city of which we are speaking is about eighty stadia distant from the sea.

\par \textbf{ATHENIAN}
\par   And are there harbours on the seaboard?

\par \textbf{CLEINIAS}
\par   Excellent harbours, Stranger; there could not be better.

\par \textbf{ATHENIAN}
\par   Alas! what a prospect! And is the surrounding country productive, or in need of importations?

\par \textbf{CLEINIAS}
\par   Hardly in need of anything.

\par \textbf{ATHENIAN}
\par   And is there any neighbouring State?

\par \textbf{CLEINIAS}
\par   None whatever, and that is the reason for selecting the place; in days of old, there was a migration of the inhabitants, and the region has been deserted from time immemorial.

\par \textbf{ATHENIAN}
\par   And has the place a fair proportion of hill, and plain, and wood?

\par \textbf{CLEINIAS}
\par   Like the rest of Crete in that.

\par \textbf{ATHENIAN}
\par   You mean to say that there is more rock than plain?

\par \textbf{CLEINIAS}
\par   Exactly.

\par \textbf{ATHENIAN}
\par   Then there is some hope that your citizens may be virtuous:  had you been on the sea, and well provided with harbours, and an importing rather than a producing country, some mighty saviour would have been needed, and lawgivers more than mortal, if you were ever to have a chance of preserving your state from degeneracy and discordance of manners (compare Ar. Pol.). But there is comfort in the eighty stadia; although the sea is too near, especially if, as you say, the harbours are so good. Still we may be content. The sea is pleasant enough as a daily companion, but has indeed also a bitter and brackish quality; filling the streets with merchants and shopkeepers, and begetting in the souls of men uncertain and unfaithful ways—making the state unfriendly and unfaithful both to her own citizens, and also to other nations. There is a consolation, therefore, in the country producing all things at home; and yet, owing to the ruggedness of the soil, not providing anything in great abundance. Had there been abundance, there might have been a great export trade, and a great return of gold and silver; which, as we may safely affirm, has the most fatal results on a State whose aim is the attainment of just and noble sentiments:  this was said by us, if you remember, in the previous discussion.

\par \textbf{CLEINIAS}
\par   I remember, and am of opinion that we both were and are in the right.

\par \textbf{ATHENIAN}
\par   Well, but let me ask, how is the country supplied with timber for ship-building?

\par \textbf{CLEINIAS}
\par   There is no fir of any consequence, nor pine, and not much cypress; and you will find very little stone-pine or plane-wood, which shipwrights always require for the interior of ships.

\par \textbf{ATHENIAN}
\par   These are also natural advantages.

\par \textbf{CLEINIAS}
\par   Why so?

\par \textbf{ATHENIAN}
\par   Because no city ought to be easily able to imitate its enemies in what is mischievous.

\par \textbf{CLEINIAS}
\par   How does that bear upon any of the matters of which we have been speaking?

\par \textbf{ATHENIAN}
\par   Remember, my good friend, what I said at first about the Cretan laws, that they looked to one thing only, and this, as you both agreed, was war; and I replied that such laws, in so far as they tended to promote virtue, were good; but in that they regarded a part only, and not the whole of virtue, I disapproved of them. And now I hope that you in your turn will follow and watch me if I legislate with a view to anything but virtue, or with a view to a part of virtue only. For I consider that the true lawgiver, like an archer, aims only at that on which some eternal beauty is always attending, and dismisses everything else, whether wealth or any other benefit, when separated from virtue. I was saying that the imitation of enemies was a bad thing; and I was thinking of a case in which a maritime people are harassed by enemies, as the Athenians were by Minos (I do not speak from any desire to recall past grievances); but he, as we know, was a great naval potentate, who compelled the inhabitants of Attica to pay him a cruel tribute; and in those days they had no ships of war as they now have, nor was the country filled with ship-timber, and therefore they could not readily build them. Hence they could not learn how to imitate their enemy at sea, and in this way, becoming sailors themselves, directly repel their enemies. Better for them to have lost many times over the seven youths, than that heavy-armed and stationary troops should have been turned into sailors, and accustomed to be often leaping on shore, and again to come running back to their ships; or should have fancied that there was no disgrace in not awaiting the attack of an enemy and dying boldly; and that there were good reasons, and plenty of them, for a man throwing away his arms, and betaking himself to flight,—which is not dishonourable, as people say, at certain times. This is the language of naval warfare, and is anything but worthy of extraordinary praise. For we should not teach bad habits, least of all to the best part of the citizens. You may learn the evil of such a practice from Homer, by whom Odysseus is introduced, rebuking Agamemnon, because he desires to draw down the ships to the sea at a time when the Achaeans are hard pressed by the Trojans,—he gets angry with him, and says:

\par  'Who, at a time when the battle is in full cry, biddest to drag the well-benched ships into the sea, that the prayers of the Trojans may be accomplished yet more, and high ruin fall upon us. For the Achaeans will not maintain the battle, when the ships are drawn into the sea, but they will look behind and will cease from strife; in that the counsel which you give will prove injurious.'

\par  You see that he quite knew triremes on the sea, in the neighbourhood of fighting men, to be an evil;—lions might be trained in that way to fly from a herd of deer. Moreover, naval powers which owe their safety to ships, do not give honour to that sort of warlike excellence which is most deserving of it. For he who owes his safety to the pilot and the captain, and the oarsman, and all sorts of rather inferior persons, cannot rightly give honour to whom honour is due. But how can a state be in a right condition which cannot justly award honour?

\par \textbf{CLEINIAS}
\par   It is hardly possible, I admit; and yet, Stranger, we Cretans are in the habit of saying that the battle of Salamis was the salvation of Hellas.

\par \textbf{ATHENIAN}
\par   Why, yes; and that is an opinion which is widely spread both among Hellenes and barbarians. But Megillus and I say rather, that the battle of Marathon was the beginning, and the battle of Plataea the completion, of the great deliverance, and that these battles by land made the Hellenes better; whereas the sea-fights of Salamis and Artemisium—for I may as well put them both together—made them no better, if I may say so without offence about the battles which helped to save us. And in estimating the goodness of a state, we regard both the situation of the country and the order of the laws, considering that the mere preservation and continuance of life is not the most honourable thing for men, as the vulgar think, but the continuance of the best life, while we live; and that again, if I am not mistaken, is a remark which has been made already.

\par \textbf{CLEINIAS}
\par   Yes.

\par \textbf{ATHENIAN}
\par   Then we have only to ask, whether we are taking the course which we acknowledge to be the best for the settlement and legislation of states.

\par \textbf{CLEINIAS}
\par   The best by far.

\par \textbf{ATHENIAN}
\par   And now let me proceed to another question:  Who are to be the colonists? May any one come out of all Crete; and is the idea that the population in the several states is too numerous for the means of subsistence? For I suppose that you are not going to send out a general invitation to any Hellene who likes to come. And yet I observe that to your country settlers have come from Argos and Aegina and other parts of Hellas. Tell me, then, whence do you draw your recruits in the present enterprise?

\par \textbf{CLEINIAS}
\par   They will come from all Crete; and of other Hellenes, Peloponnesians will be most acceptable. For, as you truly observe, there are Cretans of Argive descent; and the race of Cretans which has the highest character at the present day is the Gortynian, and this has come from Gortys in the Peloponnesus.

\par \textbf{ATHENIAN}
\par   Cities find colonization in some respects easier if the colonists are one race, which like a swarm of bees is sent out from a single country, either when friends leave friends, owing to some pressure of population or other similar necessity, or when a portion of a state is driven by factions to emigrate. And there have been whole cities which have taken flight when utterly conquered by a superior power in war. This, however, which is in one way an advantage to the colonist or legislator, in another point of view creates a difficulty. There is an element of friendship in the community of race, and language, and laws, and in common temples and rites of worship; but colonies which are of this homogeneous sort are apt to kick against any laws or any form of constitution differing from that which they had at home; and although the badness of their own laws may have been the cause of the factions which prevailed among them, yet from the force of habit they would fain preserve the very customs which were their ruin, and the leader of the colony, who is their legislator, finds them troublesome and rebellious. On the other hand, the conflux of several populations might be more disposed to listen to new laws; but then, to make them combine and pull together, as they say of horses, is a most difficult task, and the work of years. And yet there is nothing which tends more to the improvement of mankind than legislation and colonization.

\par \textbf{CLEINIAS}
\par   No doubt; but I should like to know why you say so.

\par \textbf{ATHENIAN}
\par   My good friend, I am afraid that the course of my speculations is leading me to say something depreciatory of legislators; but if the word be to the purpose, there can be no harm. And yet, why am I disquieted, for I believe that the same principle applies equally to all human things?

\par \textbf{CLEINIAS}
\par   To what are you referring?

\par \textbf{ATHENIAN}
\par   I was going to say that man never legislates, but accidents of all sorts, which legislate for us in all sorts of ways. The violence of war and the hard necessity of poverty are constantly overturning governments and changing laws. And the power of disease has often caused innovations in the state, when there have been pestilences, or when there has been a succession of bad seasons continuing during many years. Any one who sees all this, naturally rushes to the conclusion of which I was speaking, that no mortal legislates in anything, but that in human affairs chance is almost everything. And this may be said of the arts of the sailor, and the pilot, and the physician, and the general, and may seem to be well said; and yet there is another thing which may be said with equal truth of all of them.

\par \textbf{CLEINIAS}
\par   What is it?

\par \textbf{ATHENIAN}
\par   That God governs all things, and that chance and opportunity co-operate with Him in the government of human affairs. There is, however, a third and less extreme view, that art should be there also; for I should say that in a storm there must surely be a great advantage in having the aid of the pilot's art. You would agree?

\par \textbf{CLEINIAS}
\par   Yes.

\par \textbf{ATHENIAN}
\par   And does not a like principle apply to legislation as well as to other things:  even supposing all the conditions to be favourable which are needed for the happiness of the state, yet the true legislator must from time to time appear on the scene?

\par \textbf{CLEINIAS}
\par   Most true.

\par \textbf{ATHENIAN}
\par   In each case the artist would be able to pray rightly for certain conditions, and if these were granted by fortune, he would then only require to exercise his art?

\par \textbf{CLEINIAS}
\par   Certainly.

\par \textbf{ATHENIAN}
\par   And all the other artists just now mentioned, if they were bidden to offer up each their special prayer, would do so?

\par \textbf{CLEINIAS}
\par   Of course.

\par \textbf{ATHENIAN}
\par   And the legislator would do likewise?

\par \textbf{CLEINIAS}
\par   I believe that he would.

\par \textbf{ATHENIAN}
\par   'Come, legislator,' we will say to him; 'what are the conditions which you require in a state before you can organize it?' How ought he to answer this question? Shall I give his answer?

\par \textbf{CLEINIAS}
\par   Yes.

\par \textbf{ATHENIAN}
\par   He will say—'Give me a state which is governed by a tyrant, and let the tyrant be young and have a good memory; let him be quick at learning, and of a courageous and noble nature; let him have that quality which, as I said before, is the inseparable companion of all the other parts of virtue, if there is to be any good in them.'

\par \textbf{CLEINIAS}
\par   I suppose, Megillus, that this companion virtue of which the Stranger speaks, must be temperance?

\par \textbf{ATHENIAN}
\par   Yes, Cleinias, temperance in the vulgar sense; not that which in the forced and exaggerated language of some philosophers is called prudence, but that which is the natural gift of children and animals, of whom some live continently and others incontinently, but when isolated, was, as we said, hardly worth reckoning in the catalogue of goods. I think that you must understand my meaning.

\par \textbf{CLEINIAS}
\par   Certainly.

\par \textbf{ATHENIAN}
\par   Then our tyrant must have this as well as the other qualities, if the state is to acquire in the best manner and in the shortest time the form of government which is most conducive to happiness; for there neither is nor ever will be a better or speedier way of establishing a polity than by a tyranny.

\par \textbf{CLEINIAS}
\par   By what possible arguments, Stranger, can any man persuade himself of such a monstrous doctrine?

\par \textbf{ATHENIAN}
\par   There is surely no difficulty in seeing, Cleinias, what is in accordance with the order of nature?

\par \textbf{CLEINIAS}
\par   You would assume, as you say, a tyrant who was young, temperate, quick at learning, having a good memory, courageous, of a noble nature?

\par \textbf{ATHENIAN}
\par   Yes; and you must add fortunate; and his good fortune must be that he is the contemporary of a great legislator, and that some happy chance brings them together. When this has been accomplished, God has done all that he ever does for a state which he desires to be eminently prosperous; He has done second best for a state in which there are two such rulers, and third best for a state in which there are three. The difficulty increases with the increase, and diminishes with the diminution of the number.

\par \textbf{CLEINIAS}
\par   You mean to say, I suppose, that the best government is produced from a tyranny, and originates in a good lawgiver and an orderly tyrant, and that the change from such a tyranny into a perfect form of government takes place most easily; less easily when from an oligarchy; and, in the third degree, from a democracy:  is not that your meaning?

\par \textbf{ATHENIAN}
\par   Not so; I mean rather to say that the change is best made out of a tyranny; and secondly, out of a monarchy; and thirdly, out of some sort of democracy:  fourth, in the capacity for improvement, comes oligarchy, which has the greatest difficulty in admitting of such a change, because the government is in the hands of a number of potentates. I am supposing that the legislator is by nature of the true sort, and that his strength is united with that of the chief men of the state; and when the ruling element is numerically small, and at the same time very strong, as in a tyranny, there the change is likely to be easiest and most rapid.

\par \textbf{CLEINIAS}
\par   How? I do not understand.

\par \textbf{ATHENIAN}
\par   And yet I have repeated what I am saying a good many times; but I suppose that you have never seen a city which is under a tyranny?

\par \textbf{CLEINIAS}
\par   No, and I cannot say that I have any great desire to see one.

\par \textbf{ATHENIAN}
\par   And yet, where there is a tyranny, you might certainly see that of which I am now speaking.

\par \textbf{CLEINIAS}
\par   What do you mean?

\par \textbf{ATHENIAN}
\par   I mean that you might see how, without trouble and in no very long period of time, the tyrant, if he wishes, can change the manners of a state:  he has only to go in the direction of virtue or of vice, whichever he prefers, he himself indicating by his example the lines of conduct, praising and rewarding some actions and reproving others, and degrading those who disobey.

\par \textbf{CLEINIAS}
\par   But how can we imagine that the citizens in general will at once follow the example set to them; and how can he have this power both of persuading and of compelling them?

\par \textbf{ATHENIAN}
\par   Let no one, my friends, persuade us that there is any quicker and easier way in which states change their laws than when the rulers lead:  such changes never have, nor ever will, come to pass in any other way. The real impossibility or difficulty is of another sort, and is rarely surmounted in the course of ages; but when once it is surmounted, ten thousand or rather all blessings follow.

\par \textbf{CLEINIAS}
\par   Of what are you speaking?

\par \textbf{ATHENIAN}
\par   The difficulty is to find the divine love of temperate and just institutions existing in any powerful forms of government, whether in a monarchy or oligarchy of wealth or of birth. You might as well hope to reproduce the character of Nestor, who is said to have excelled all men in the power of speech, and yet more in his temperance. This, however, according to the tradition, was in the times of Troy; in our own days there is nothing of the sort; but if such an one either has or ever shall come into being, or is now among us, blessed is he and blessed are they who hear the wise words that flow from his lips. And this may be said of power in general:  When the supreme power in man coincides with the greatest wisdom and temperance, then the best laws and the best constitution come into being; but in no other way. And let what I have been saying be regarded as a kind of sacred legend or oracle, and let this be our proof that, in one point of view, there may be a difficulty for a city to have good laws, but that there is another point of view in which nothing can be easier or sooner effected, granting our supposition.

\par \textbf{CLEINIAS}
\par   How do you mean?

\par \textbf{ATHENIAN}
\par   Let us try to amuse ourselves, old boys as we are, by moulding in words the laws which are suitable to your state.

\par \textbf{CLEINIAS}
\par   Let us proceed without delay.

\par \textbf{ATHENIAN}
\par   Then let us invoke God at the settlement of our state; may He hear and be propitious to us, and come and set in order the State and the laws!

\par \textbf{CLEINIAS}
\par   May He come!

\par \textbf{ATHENIAN}
\par   But what form of polity are we going to give the city?

\par \textbf{CLEINIAS}
\par   Tell us what you mean a little more clearly. Do you mean some form of democracy, or oligarchy, or aristocracy, or monarchy? For we cannot suppose that you would include tyranny.

\par \textbf{ATHENIAN}
\par   Which of you will first tell me to which of these classes his own government is to be referred?

\par \textbf{MEGILLUS}
\par   Ought I to answer first, since I am the elder?

\par \textbf{CLEINIAS}
\par   Perhaps you should.

\par \textbf{MEGILLUS}
\par   And yet, Stranger, I perceive that I cannot say, without more thought, what I should call the government of Lacedaemon, for it seems to me to be like a tyranny,—the power of our Ephors is marvellously tyrannical; and sometimes it appears to me to be of all cities the most democratical; and who can reasonably deny that it is an aristocracy (compare Ar. Pol.)? We have also a monarchy which is held for life, and is said by all mankind, and not by ourselves only, to be the most ancient of all monarchies; and, therefore, when asked on a sudden, I cannot precisely say which form of government the Spartan is.

\par \textbf{CLEINIAS}
\par   I am in the same difficulty, Megillus; for I do not feel confident that the polity of Cnosus is any of these.

\par \textbf{ATHENIAN}
\par   The reason is, my excellent friends, that you really have polities, but the states of which we were just now speaking are merely aggregations of men dwelling in cities who are the subjects and servants of a part of their own state, and each of them is named after the dominant power; they are not polities at all. But if states are to be named after their rulers, the true state ought to be called by the name of the God who rules over wise men.

\par \textbf{CLEINIAS}
\par   And who is this God?

\par \textbf{ATHENIAN}
\par   May I still make use of fable to some extent, in the hope that I may be better able to answer your question:  shall I?

\par \textbf{CLEINIAS}
\par   By all means.

\par \textbf{ATHENIAN}
\par   In the primeval world, and a long while before the cities came into being whose settlements we have described, there is said to have been in the time of Cronos a blessed rule and life, of which the best-ordered of existing states is a copy (compare Statesman).

\par \textbf{CLEINIAS}
\par   It will be very necessary to hear about that.

\par \textbf{ATHENIAN}
\par   I quite agree with you; and therefore I have introduced the subject.

\par \textbf{CLEINIAS}
\par   Most appropriately; and since the tale is to the point, you will do well in giving us the whole story.

\par \textbf{ATHENIAN}
\par   I will do as you suggest. There is a tradition of the happy life of mankind in days when all things were spontaneous and abundant. And of this the reason is said to have been as follows: —Cronos knew what we ourselves were declaring, that no human nature invested with supreme power is able to order human affairs and not overflow with insolence and wrong. Which reflection led him to appoint not men but demigods, who are of a higher and more divine race, to be the kings and rulers of our cities; he did as we do with flocks of sheep and other tame animals. For we do not appoint oxen to be the lords of oxen, or goats of goats; but we ourselves are a superior race, and rule over them. In like manner God, in His love of mankind, placed over us the demons, who are a superior race, and they with great ease and pleasure to themselves, and no less to us, taking care of us and giving us peace and reverence and order and justice never failing, made the tribes of men happy and united. And this tradition, which is true, declares that cities of which some mortal man and not God is the ruler, have no escape from evils and toils. Still we must do all that we can to imitate the life which is said to have existed in the days of Cronos, and, as far as the principle of immortality dwells in us, to that we must hearken, both in private and public life, and regulate our cities and houses according to law, meaning by the very term 'law,' the distribution of mind. But if either a single person or an oligarchy or a democracy has a soul eager after pleasures and desires—wanting to be filled with them, yet retaining none of them, and perpetually afflicted with an endless and insatiable disorder; and this evil spirit, having first trampled the laws under foot, becomes the master either of a state or of an individual,—then, as I was saying, salvation is hopeless. And now, Cleinias, we have to consider whether you will or will not accept this tale of mine.

\par \textbf{CLEINIAS}
\par   Certainly we will.

\par \textbf{ATHENIAN}
\par   You are aware,—are you not?—that there are often said to be as many forms of laws as there are of governments, and of the latter we have already mentioned all those which are commonly recognized. Now you must regard this as a matter of first-rate importance. For what is to be the standard of just and unjust, is once more the point at issue. Men say that the law ought not to regard either military virtue, or virtue in general, but only the interests and power and preservation of the established form of government; this is thought by them to be the best way of expressing the natural definition of justice.

\par \textbf{CLEINIAS}
\par   How?

\par \textbf{ATHENIAN}
\par   Justice is said by them to be the interest of the stronger (Republic).

\par \textbf{CLEINIAS}
\par   Speak plainer.

\par \textbf{ATHENIAN}
\par   I will: —'Surely,' they say, 'the governing power makes whatever laws have authority in any state'?

\par \textbf{CLEINIAS}
\par   True.

\par \textbf{ATHENIAN}
\par   'Well,' they would add, 'and do you suppose that tyranny or democracy, or any other conquering power, does not make the continuance of the power which is possessed by them the first or principal object of their laws'?

\par \textbf{CLEINIAS}
\par   How can they have any other?

\par \textbf{ATHENIAN}
\par   'And whoever transgresses these laws is punished as an evil-doer by the legislator, who calls the laws just'?

\par \textbf{CLEINIAS}
\par   Naturally.

\par \textbf{ATHENIAN}
\par   'This, then, is always the mode and fashion in which justice exists.'

\par \textbf{CLEINIAS}
\par   Certainly, if they are correct in their view.

\par \textbf{ATHENIAN}
\par   Why, yes, this is one of those false principles of government to which we were referring.

\par \textbf{CLEINIAS}
\par   Which do you mean?

\par \textbf{ATHENIAN}
\par   Those which we were examining when we spoke of who ought to govern whom. Did we not arrive at the conclusion that parents ought to govern their children, and the elder the younger, and the noble the ignoble? And there were many other principles, if you remember, and they were not always consistent. One principle was this very principle of might, and we said that Pindar considered violence natural and justified it.

\par \textbf{CLEINIAS}
\par   Yes; I remember.

\par \textbf{ATHENIAN}
\par   Consider, then, to whom our state is to be entrusted. For there is a thing which has occurred times without number in states—

\par \textbf{CLEINIAS}
\par   What thing?

\par \textbf{ATHENIAN}
\par   That when there has been a contest for power, those who gain the upper hand so entirely monopolize the government, as to refuse all share to the defeated party and their descendants—they live watching one another, the ruling class being in perpetual fear that some one who has a recollection of former wrongs will come into power and rise up against them. Now, according to our view, such governments are not polities at all, nor are laws right which are passed for the good of particular classes and not for the good of the whole state. States which have such laws are not polities but parties, and their notions of justice are simply unmeaning. I say this, because I am going to assert that we must not entrust the government in your state to any one because he is rich, or because he possesses any other advantage, such as strength, or stature, or again birth:  but he who is most obedient to the laws of the state, he shall win the palm; and to him who is victorious in the first degree shall be given the highest office and chief ministry of the gods; and the second to him who bears the second palm; and on a similar principle shall all the other offices be assigned to those who come next in order. And when I call the rulers servants or ministers of the law, I give them this name not for the sake of novelty, but because I certainly believe that upon such service or ministry depends the well- or ill-being of the state. For that state in which the law is subject and has no authority, I perceive to be on the highway to ruin; but I see that the state in which the law is above the rulers, and the rulers are the inferiors of the law, has salvation, and every blessing which the Gods can confer.

\par \textbf{CLEINIAS}
\par   Truly, Stranger, you see with the keen vision of age.

\par \textbf{ATHENIAN}
\par   Why, yes; every man when he is young has that sort of vision dullest, and when he is old keenest.

\par \textbf{CLEINIAS}
\par   Very true.

\par \textbf{ATHENIAN}
\par   And now, what is to be the next step? May we not suppose the colonists to have arrived, and proceed to make our speech to them?

\par \textbf{CLEINIAS}
\par   Certainly.

\par \textbf{ATHENIAN}
\par   'Friends,' we say to them,—'God, as the old tradition declares, holding in his hand the beginning, middle, and end of all that is, travels according to His nature in a straight line towards the accomplishment of His end. Justice always accompanies Him, and is the punisher of those who fall short of the divine law. To justice, he who would be happy holds fast, and follows in her company with all humility and order; but he who is lifted up with pride, or elated by wealth or rank, or beauty, who is young and foolish, and has a soul hot with insolence, and thinks that he has no need of any guide or ruler, but is able himself to be the guide of others, he, I say, is left deserted of God; and being thus deserted, he takes to him others who are like himself, and dances about, throwing all things into confusion, and many think that he is a great man, but in a short time he pays a penalty which justice cannot but approve, and is utterly destroyed, and his family and city with him. Wherefore, seeing that human things are thus ordered, what should a wise man do or think, or not do or think'?

\par \textbf{CLEINIAS}
\par   Every man ought to make up his mind that he will be one of the followers of God; there can be no doubt of that.

\par \textbf{ATHENIAN}
\par   Then what life is agreeable to God, and becoming in His followers? One only, expressed once for all in the old saying that 'like agrees with like, with measure measure,' but things which have no measure agree neither with themselves nor with the things which have. Now God ought to be to us the measure of all things, and not man (compare Crat. ; Theaet. ), as men commonly say (Protagoras):  the words are far more true of Him. And he who would be dear to God must, as far as is possible, be like Him and such as He is. Wherefore the temperate man is the friend of God, for he is like Him; and the intemperate man is unlike Him, and different from Him, and unjust. And the same applies to other things; and this is the conclusion, which is also the noblest and truest of all sayings,—that for the good man to offer sacrifice to the Gods, and hold converse with them by means of prayers and offerings and every kind of service, is the noblest and best of all things, and also the most conducive to a happy life, and very fit and meet. But with the bad man, the opposite of this is true:  for the bad man has an impure soul, whereas the good is pure; and from one who is polluted, neither a good man nor God can without impropriety receive gifts. Wherefore the unholy do only waste their much service upon the Gods, but when offered by any holy man, such service is most acceptable to them. This is the mark at which we ought to aim. But what weapons shall we use, and how shall we direct them? In the first place, we affirm that next after the Olympian Gods and the Gods of the State, honour should be given to the Gods below; they should receive everything in even numbers, and of the second choice, and ill omen, while the odd numbers, and the first choice, and the things of lucky omen, are given to the Gods above, by him who would rightly hit the mark of piety. Next to these Gods, a wise man will do service to the demons or spirits, and then to the heroes, and after them will follow the private and ancestral Gods, who are worshipped as the law prescribes in the places which are sacred to them. Next comes the honour of living parents, to whom, as is meet, we have to pay the first and greatest and oldest of all debts, considering that all which a man has belongs to those who gave him birth and brought him up, and that he must do all that he can to minister to them, first, in his property, secondly, in his person, and thirdly, in his soul, in return for the endless care and travail which they bestowed upon him of old, in the days of his infancy, and which he is now to pay back to them when they are old and in the extremity of their need. And all his life long he ought never to utter, or to have uttered, an unbecoming word to them; for of light and fleeting words the penalty is most severe; Nemesis, the messenger of justice, is appointed to watch over all such matters. When they are angry and want to satisfy their feelings in word or deed, he should give way to them; for a father who thinks that he has been wronged by his son may be reasonably expected to be very angry. At their death, the most moderate funeral is best, neither exceeding the customary expense, nor yet falling short of the honour which has been usually shown by the former generation to their parents. And let a man not forget to pay the yearly tribute of respect to the dead, honouring them chiefly by omitting nothing that conduces to a perpetual remembrance of them, and giving a reasonable portion of his fortune to the dead. Doing this, and living after this manner, we shall receive our reward from the Gods and those who are above us (i.e. the demons); and we shall spend our days for the most part in good hope. And how a man ought to order what relates to his descendants and his kindred and friends and fellow-citizens, and the rites of hospitality taught by Heaven, and the intercourse which arises out of all these duties, with a view to the embellishment and orderly regulation of his own life—these things, I say, the laws, as we proceed with them, will accomplish, partly persuading, and partly when natures do not yield to the persuasion of custom, chastising them by might and right, and will thus render our state, if the Gods co-operate with us, prosperous and happy. But of what has to be said, and must be said by the legislator who is of my way of thinking, and yet, if said in the form of law, would be out of place—of this I think that he may give a sample for the instruction of himself and of those for whom he is legislating; and then when, as far as he is able, he has gone through all the preliminaries, he may proceed to the work of legislation. Now, what will be the form of such prefaces? There may be a difficulty in including or describing them all under a single form, but I think that we may get some notion of them if we can guarantee one thing.

\par \textbf{CLEINIAS}
\par   What is that?

\par \textbf{ATHENIAN}
\par   I should wish the citizens to be as readily persuaded to virtue as possible; this will surely be the aim of the legislator in all his laws.

\par \textbf{CLEINIAS}
\par   Certainly.

\par \textbf{ATHENIAN}
\par   The proposal appears to me to be of some value; and I think that a person will listen with more gentleness and good-will to the precepts addressed to him by the legislator, when his soul is not altogether unprepared to receive them. Even a little done in the way of conciliation gains his ear, and is always worth having. For there is no great inclination or readiness on the part of mankind to be made as good, or as quickly good, as possible. The case of the many proves the wisdom of Hesiod, who says that the road to wickedness is smooth and can be travelled without perspiring, because it is so very short:

\par  'But before virtue the immortal Gods have placed the sweat of labour, and long and steep is the way thither, and rugged at first; but when you have reached the top, although difficult before, it is then easy.' (Works and Days.)

\par \textbf{CLEINIAS}
\par   Yes; and he certainly speaks well.

\par \textbf{ATHENIAN}
\par   Very true:  and now let me tell you the effect which the preceding discourse has had upon me.

\par \textbf{CLEINIAS}
\par   Proceed.

\par \textbf{ATHENIAN}
\par   Suppose that we have a little conversation with the legislator, and say to him—'O, legislator, speak; if you know what we ought to say and do, you can surely tell.'

\par \textbf{CLEINIAS}
\par   Of course he can.

\par \textbf{ATHENIAN}
\par   'Did we not hear you just now saying, that the legislator ought not to allow the poets to do what they liked? For that they would not know in which of their words they went against the laws, to the hurt of the state.'

\par \textbf{CLEINIAS}
\par   That is true.

\par \textbf{ATHENIAN}
\par   May we not fairly make answer to him on behalf of the poets?

\par \textbf{CLEINIAS}
\par   What answer shall we make to him?

\par \textbf{ATHENIAN}
\par   That the poet, according to the tradition which has ever prevailed among us, and is accepted of all men, when he sits down on the tripod of the muse, is not in his right mind; like a fountain, he allows to flow out freely whatever comes in, and his art being imitative, he is often compelled to represent men of opposite dispositions, and thus to contradict himself; neither can he tell whether there is more truth in one thing that he has said than in another. This is not the case in a law; the legislator must give not two rules about the same thing, but one only. Take an example from what you have just been saying. Of three kinds of funerals, there is one which is too extravagant, another is too niggardly, the third in a mean; and you choose and approve and order the last without qualification. But if I had an extremely rich wife, and she bade me bury her and describe her burial in a poem, I should praise the extravagant sort; and a poor miserly man, who had not much money to spend, would approve of the niggardly; and the man of moderate means, who was himself moderate, would praise a moderate funeral. Now you in the capacity of legislator must not barely say 'a moderate funeral,' but you must define what moderation is, and how much; unless you are definite, you must not suppose that you are speaking a language that can become law.

\par \textbf{CLEINIAS}
\par   Certainly not.

\par \textbf{ATHENIAN}
\par   And is our legislator to have no preface to his laws, but to say at once Do this, avoid that—and then holding the penalty in terrorem, to go on to another law; offering never a word of advice or exhortation to those for whom he is legislating, after the manner of some doctors? For of doctors, as I may remind you, some have a gentler, others a ruder method of cure; and as children ask the doctor to be gentle with them, so we will ask the legislator to cure our disorders with the gentlest remedies. What I mean to say is, that besides doctors there are doctors' servants, who are also styled doctors.

\par \textbf{CLEINIAS}
\par   Very true.

\par \textbf{ATHENIAN}
\par   And whether they are slaves or freemen makes no difference; they acquire their knowledge of medicine by obeying and observing their masters; empirically and not according to the natural way of learning, as the manner of freemen is, who have learned scientifically themselves the art which they impart scientifically to their pupils. You are aware that there are these two classes of doctors?

\par \textbf{CLEINIAS}
\par   To be sure.

\par \textbf{ATHENIAN}
\par   And did you ever observe that there are two classes of patients in states, slaves and freemen; and the slave doctors run about and cure the slaves, or wait for them in the dispensaries—practitioners of this sort never talk to their patients individually, or let them talk about their own individual complaints? The slave doctor prescribes what mere experience suggests, as if he had exact knowledge; and when he has given his orders, like a tyrant, he rushes off with equal assurance to some other servant who is ill; and so he relieves the master of the house of the care of his invalid slaves. But the other doctor, who is a freeman, attends and practices upon freemen; and he carries his enquiries far back, and goes into the nature of the disorder; he enters into discourse with the patient and with his friends, and is at once getting information from the sick man, and also instructing him as far as he is able, and he will not prescribe for him until he has first convinced him; at last, when he has brought the patient more and more under his persuasive influences and set him on the road to health, he attempts to effect a cure. Now which is the better way of proceeding in a physician and in a trainer? Is he the better who accomplishes his ends in a double way, or he who works in one way, and that the ruder and inferior?

\par \textbf{CLEINIAS}
\par   I should say, Stranger, that the double way is far better.

\par \textbf{ATHENIAN}
\par   Should you like to see an example of the double and single method in legislation?

\par \textbf{CLEINIAS}
\par   Certainly I should.

\par \textbf{ATHENIAN}
\par   What will be our first law? Will not the legislator, observing the order of nature, begin by making regulations for states about births?

\par \textbf{CLEINIAS}
\par   He will.

\par \textbf{ATHENIAN}
\par   In all states the birth of children goes back to the connexion of marriage?

\par \textbf{CLEINIAS}
\par   Very true.

\par \textbf{ATHENIAN}
\par   And, according to the true order, the laws relating to marriage should be those which are first determined in every state?

\par \textbf{CLEINIAS}
\par   Quite so.

\par \textbf{ATHENIAN}
\par   Then let me first give the law of marriage in a simple form; it may run as follows: —A man shall marry between the ages of thirty and thirty-five, or, if he does not, he shall pay such and such a fine, or shall suffer the loss of such and such privileges. This would be the simple law about marriage. The double law would run thus: —A man shall marry between the ages of thirty and thirty-five, considering that in a manner the human race naturally partakes of immortality, which every man is by nature inclined to desire to the utmost; for the desire of every man that he may become famous, and not lie in the grave without a name, is only the love of continuance. Now mankind are coeval with all time, and are ever following, and will ever follow, the course of time; and so they are immortal, because they leave children's children behind them, and partake of immortality in the unity of generation. And for a man voluntarily to deprive himself of this gift, as he deliberately does who will not have a wife or children, is impiety. He who obeys the law shall be free, and shall pay no fine; but he who is disobedient, and does not marry, when he has arrived at the age of thirty-five, shall pay a yearly fine of a certain amount, in order that he may not imagine his celibacy to bring ease and profit to him; and he shall not share in the honours which the young men in the state give to the aged. Comparing now the two forms of the law, you will be able to arrive at a judgment about any other laws—whether they should be double in length even when shortest, because they have to persuade as well as threaten, or whether they shall only threaten and be of half the length.

\par \textbf{MEGILLUS}
\par   The shorter form, Stranger, would be more in accordance with Lacedaemonian custom; although, for my own part, if any one were to ask me which I myself prefer in the state, I should certainly determine in favour of the longer; and I would have every law made after the same pattern, if I had to choose. But I think that Cleinias is the person to be consulted, for his is the state which is going to use these laws.

\par \textbf{CLEINIAS}
\par   Thank you, Megillus.

\par \textbf{ATHENIAN}
\par   Whether, in the abstract, words are to be many or few, is a very foolish question; the best form, and not the shortest, is to be approved; nor is length at all to be regarded. Of the two forms of law which have been recited, the one is not only twice as good in practical usefulness as the other, but the case is like that of the two kinds of doctors, which I was just now mentioning. And yet legislators never appear to have considered that they have two instruments which they might use in legislation—persuasion and force; for in dealing with the rude and uneducated multitude, they use the one only as far as they can; they do not mingle persuasion with coercion, but employ force pure and simple. Moreover, there is a third point, sweet friends, which ought to be, and never is, regarded in our existing laws.

\par \textbf{CLEINIAS}
\par   What is it?

\par \textbf{ATHENIAN}
\par   A point arising out of our previous discussion, which comes into my mind in some mysterious way. All this time, from early dawn until noon, have we been talking about laws in this charming retreat:  now we are going to promulgate our laws, and what has preceded was only the prelude of them. Why do I mention this? For this reason: —Because all discourses and vocal exercises have preludes and overtures, which are a sort of artistic beginnings intended to help the strain which is to be performed; lyric measures and music of every other kind have preludes framed with wonderful care. But of the truer and higher strain of law and politics, no one has ever yet uttered any prelude, or composed or published any, as though there was no such thing in nature. Whereas our present discussion seems to me to imply that there is;—these double laws, of which we were speaking, are not exactly double, but they are in two parts, the law and the prelude of the law. The arbitrary command, which was compared to the commands of doctors, whom we described as of the meaner sort, was the law pure and simple; and that which preceded, and was described by our friend here as being hortatory only, was, although in fact, an exhortation, likewise analogous to the preamble of a discourse. For I imagine that all this language of conciliation, which the legislator has been uttering in the preface of the law, was intended to create good-will in the person whom he addressed, in order that, by reason of this good-will, he might more intelligently receive his command, that is to say, the law. And therefore, in my way of speaking, this is more rightly described as the preamble than as the matter of the law. And I must further proceed to observe, that to all his laws, and to each separately, the legislator should prefix a preamble; he should remember how great will be the difference between them, according as they have, or have not, such preambles, as in the case already given.

\par \textbf{CLEINIAS}
\par   The lawgiver, if he asks my opinion, will certainly legislate in the form which you advise.

\par \textbf{ATHENIAN}
\par   I think that you are right, Cleinias, in affirming that all laws have preambles, and that throughout the whole of this work of legislation every single law should have a suitable preamble at the beginning; for that which is to follow is most important, and it makes all the difference whether we clearly remember the preambles or not. Yet we should be wrong in requiring that all laws, small and great alike, should have preambles of the same kind, any more than all songs or speeches; although they may be natural to all, they are not always necessary, and whether they are to be employed or not has in each case to be left to the judgment of the speaker or the musician, or, in the present instance, of the lawgiver.

\par \textbf{CLEINIAS}
\par   That I think is most true. And now, Stranger, without delay let us return to the argument, and, as people say in play, make a second and better beginning, if you please, with the principles which we have been laying down, which we never thought of regarding as a preamble before, but of which we may now make a preamble, and not merely consider them to be chance topics of discourse. Let us acknowledge, then, that we have a preamble. About the honour of the Gods and the respect of parents, enough has been already said; and we may proceed to the topics which follow next in order, until the preamble is deemed by you to be complete; and after that you shall go through the laws themselves.

\par \textbf{ATHENIAN}
\par   I understand you to mean that we have made a sufficient preamble about Gods and demigods, and about parents living or dead; and now you would have us bring the rest of the subject into the light of day?

\par \textbf{CLEINIAS}
\par   Exactly.

\par \textbf{ATHENIAN}
\par   After this, as is meet and for the interest of us all, I the speaker, and you the listeners, will try to estimate all that relates to the souls and bodies and properties of the citizens, as regards both their occupations and amusements, and thus arrive, as far as in us lies, at the nature of education. These then are the topics which follow next in order.

\par \textbf{CLEINIAS}
\par   Very good.

\par 
\section{
      BOOK V.
    }
\par \textbf{ATHENIAN}
\par   Listen, all ye who have just now heard the laws about Gods, and about our dear forefathers: —Of all the things which a man has, next to the Gods, his soul is the most divine and most truly his own. Now in every man there are two parts:  the better and superior, which rules, and the worse and inferior, which serves; and the ruling part of him is always to be preferred to the subject. Wherefore I am right in bidding every one next to the Gods, who are our masters, and those who in order follow them (i.e. the demons), to honour his own soul, which every one seems to honour, but no one honours as he ought; for honour is a divine good, and no evil thing is honourable; and he who thinks that he can honour the soul by word or gift, or any sort of compliance, without making her in any way better, seems to honour her, but honours her not at all. For example, every man, from his very boyhood, fancies that he is able to know everything, and thinks that he honours his soul by praising her, and he is very ready to let her do whatever she may like. But I mean to say that in acting thus he injures his soul, and is far from honouring her; whereas, in our opinion, he ought to honour her as second only to the Gods. Again, when a man thinks that others are to be blamed, and not himself, for the errors which he has committed from time to time, and the many and great evils which befell him in consequence, and is always fancying himself to be exempt and innocent, he is under the idea that he is honouring his soul; whereas the very reverse is the fact, for he is really injuring her. And when, disregarding the word and approval of the legislator, he indulges in pleasure, then again he is far from honouring her; he only dishonours her, and fills her full of evil and remorse; or when he does not endure to the end the labours and fears and sorrows and pains which the legislator approves, but gives way before them, then, by yielding, he does not honour the soul, but by all such conduct he makes her to be dishonourable; nor when he thinks that life at any price is a good, does he honour her, but yet once more he dishonours her; for the soul having a notion that the world below is all evil, he yields to her, and does not resist and teach or convince her that, for aught she knows, the world of the Gods below, instead of being evil, may be the greatest of all goods. Again, when any one prefers beauty to virtue, what is this but the real and utter dishonour of the soul? For such a preference implies that the body is more honourable than the soul; and this is false, for there is nothing of earthly birth which is more honourable than the heavenly, and he who thinks otherwise of the soul has no idea how greatly he undervalues this wonderful possession; nor, again, when a person is willing, or not unwilling, to acquire dishonest gains, does he then honour his soul with gifts—far otherwise; he sells her glory and honour for a small piece of gold; but all the gold which is under or upon the earth is not enough to give in exchange for virtue. In a word, I may say that he who does not estimate the base and evil, the good and noble, according to the standard of the legislator, and abstain in every possible way from the one and practise the other to the utmost of his power, does not know that in all these respects he is most foully and disgracefully abusing his soul, which is the divinest part of man; for no one, as I may say, ever considers that which is declared to be the greatest penalty of evil-doing—namely, to grow into the likeness of bad men, and growing like them to fly from the conversation of the good, and be cut off from them, and cleave to and follow after the company of the bad. And he who is joined to them must do and suffer what such men by nature do and say to one another,—a suffering which is not justice but retribution; for justice and the just are noble, whereas retribution is the suffering which waits upon injustice; and whether a man escape or endure this, he is miserable,—in the former case, because he is not cured; while in the latter, he perishes in order that the rest of mankind may be saved.

\par  Speaking generally, our glory is to follow the better and improve the inferior, which is susceptible of improvement, as far as this is possible. And of all human possessions, the soul is by nature most inclined to avoid the evil, and track out and find the chief good; which when a man has found, he should take up his abode with it during the remainder of his life. Wherefore the soul also is second (or next to God) in honour; and third, as every one will perceive, comes the honour of the body in natural order. Having determined this, we have next to consider that there is a natural honour of the body, and that of honours some are true and some are counterfeit. To decide which are which is the business of the legislator; and he, I suspect, would intimate that they are as follows:—Honour is not to be given to the fair body, or to the strong or the swift or the tall, or to the healthy body (although many may think otherwise), any more than to their opposites; but the mean states of all these habits are by far the safest and most moderate; for the one extreme makes the soul braggart and insolent, and the other, illiberal and base; and money, and property, and distinction all go to the same tune. The excess of any of these things is apt to be a source of hatreds and divisions among states and individuals; and the defect of them is commonly a cause of slavery. And, therefore, I would not have any one fond of heaping up riches for the sake of his children, in order that he may leave them as rich as possible. For the possession of great wealth is of no use, either to them or to the state. The condition of youth which is free from flattery, and at the same time not in need of the necessaries of life, is the best and most harmonious of all, being in accord and agreement with our nature, and making life to be most entirely free from sorrow. Let parents, then, bequeath to their children not a heap of riches, but the spirit of reverence. We, indeed, fancy that they will inherit reverence from us, if we rebuke them when they show a want of reverence. But this quality is not really imparted to them by the present style of admonition, which only tells them that the young ought always to be reverential. A sensible legislator will rather exhort the elders to reverence the younger, and above all to take heed that no young man sees or hears one of themselves doing or saying anything disgraceful; for where old men have no shame, there young men will most certainly be devoid of reverence. The best way of training the young is to train yourself at the same time; not to admonish them, but to be always carrying out your own admonitions in practice. He who honours his kindred, and reveres those who share in the same Gods and are of the same blood and family, may fairly expect that the Gods who preside over generation will be propitious to him, and will quicken his seed. And he who deems the services which his friends and acquaintances do for him, greater and more important than they themselves deem them, and his own favours to them less than theirs to him, will have their good-will in the intercourse of life. And surely in his relations to the state and his fellow citizens, he is by far the best, who rather than the Olympic or any other victory of peace or war, desires to win the palm of obedience to the laws of his country, and who, of all mankind, is the person reputed to have obeyed them best through life. In his relations to strangers, a man should consider that a contract is a most holy thing, and that all concerns and wrongs of strangers are more directly dependent on the protection of God, than wrongs done to citizens; for the stranger, having no kindred and friends, is more to be pitied by Gods and men. Wherefore, also, he who is most able to avenge him is most zealous in his cause; and he who is most able is the genius and the god of the stranger, who follow in the train of Zeus, the god of strangers. And for this reason, he who has a spark of caution in him, will do his best to pass through life without sinning against the stranger. And of offences committed, whether against strangers or fellow-countrymen, that against suppliants is the greatest. For the God who witnessed to the agreement made with the suppliant, becomes in a special manner the guardian of the sufferer; and he will certainly not suffer unavenged.

\par  Thus we have fairly described the manner in which a man is to act about his parents, and himself, and his own affairs; and in relation to the state, and his friends, and kindred, both in what concerns his own countrymen, and in what concerns the stranger. We will now consider what manner of man he must be who would best pass through life in respect of those other things which are not matters of law, but of praise and blame only; in which praise and blame educate a man, and make him more tractable and amenable to the laws which are about to be imposed.

\par  Truth is the beginning of every good thing, both to Gods and men; and he who would be blessed and happy, should be from the first a partaker of the truth, that he may live a true man as long as possible, for then he can be trusted; but he is not to be trusted who loves voluntary falsehood, and he who loves involuntary falsehood is a fool. Neither condition is enviable, for the untrustworthy and ignorant has no friend, and as time advances he becomes known, and lays up in store for himself isolation in crabbed age when life is on the wane: so that, whether his children or friends are alive or not, he is equally solitary.—Worthy of honour is he who does no injustice, and of more than twofold honour, if he not only does no injustice himself, but hinders others from doing any; the first may count as one man, the second is worth many men, because he informs the rulers of the injustice of others. And yet more highly to be esteemed is he who co-operates with the rulers in correcting the citizens as far as he can—he shall be proclaimed the great and perfect citizen, and bear away the palm of virtue. The same praise may be given about temperance and wisdom, and all other goods which may be imparted to others, as well as acquired by a man for himself; he who imparts them shall be honoured as the man of men, and he who is willing, yet is not able, may be allowed the second place; but he who is jealous and will not, if he can help, allow others to partake in a friendly way of any good, is deserving of blame: the good, however, which he has, is not to be undervalued by us because it is possessed by him, but must be acquired by us also to the utmost of our power. Let every man, then, freely strive for the prize of virtue, and let there be no envy. For the unenvious nature increases the greatness of states—he himself contends in the race, blasting the fair fame of no man; but the envious, who thinks that he ought to get the better by defaming others, is less energetic himself in the pursuit of true virtue, and reduces his rivals to despair by his unjust slanders of them. And so he makes the whole city to enter the arena untrained in the practice of virtue, and diminishes her glory as far as in him lies. Now every man should be valiant, but he should also be gentle. From the cruel, or hardly curable, or altogether incurable acts of injustice done to him by others, a man can only escape by fighting and defending himself and conquering, and by never ceasing to punish them; and no man who is not of a noble spirit is able to accomplish this. As to the actions of those who do evil, but whose evil is curable, in the first place, let us remember that the unjust man is not unjust of his own free will. For no man of his own free will would choose to possess the greatest of evils, and least of all in the most honourable part of himself. And the soul, as we said, is of a truth deemed by all men the most honourable. In the soul, then, which is the most honourable part of him, no one, if he could help, would admit, or allow to continue the greatest of evils (compare Republic). The unrighteous and vicious are always to be pitied in any case; and one can afford to forgive as well as pity him who is curable, and refrain and calm one's anger, not getting into a passion, like a woman, and nursing ill-feeling. But upon him who is incapable of reformation and wholly evil, the vials of our wrath should be poured out; wherefore I say that good men ought, when occasion demands, to be both gentle and passionate.

\par  Of all evils the greatest is one which in the souls of most men is innate, and which a man is always excusing in himself and never correcting; I mean, what is expressed in the saying that 'Every man by nature is and ought to be his own friend.' Whereas the excessive love of self is in reality the source to each man of all offences; for the lover is blinded about the beloved, so that he judges wrongly of the just, the good, and the honourable, and thinks that he ought always to prefer himself to the truth. But he who would be a great man ought to regard, not himself or his interests, but what is just, whether the just act be his own or that of another. Through a similar error men are induced to fancy that their own ignorance is wisdom, and thus we who may be truly said to know nothing, think that we know all things; and because we will not let others act for us in what we do not know, we are compelled to act amiss ourselves. Wherefore let every man avoid excess of self-love, and condescend to follow a better man than himself, not allowing any false shame to stand in the way. There are also minor precepts which are often repeated, and are quite as useful; a man should recollect them and remind himself of them. For when a stream is flowing out, there should be water flowing in too; and recollection flows in while wisdom is departing. Therefore I say that a man should refrain from excess either of laughter or tears, and should exhort his neighbour to do the same; he should veil his immoderate sorrow or joy, and seek to behave with propriety, whether the genius of his good fortune remains with him, or whether at the crisis of his fate, when he seems to be mounting high and steep places, the Gods oppose him in some of his enterprises. Still he may ever hope, in the case of good men, that whatever afflictions are to befall them in the future God will lessen, and that present evils He will change for the better; and as to the goods which are the opposite of these evils, he will not doubt that they will be added to them, and that they will be fortunate. Such should be men's hopes, and such should be the exhortations with which they admonish one another, never losing an opportunity, but on every occasion distinctly reminding themselves and others of all these things, both in jest and earnest.

\par  Enough has now been said of divine matters, both as touching the practices which men ought to follow, and as to the sort of persons who they ought severally to be. But of human things we have not as yet spoken, and we must; for to men we are discoursing and not to Gods. Pleasures and pains and desires are a part of human nature, and on them every mortal being must of necessity hang and depend with the most eager interest. And therefore we must praise the noblest life, not only as the fairest in appearance, but as being one which, if a man will only taste, and not, while still in his youth, desert for another, he will find to surpass also in the very thing which we all of us desire,—I mean in having a greater amount of pleasure and less of pain during the whole of life. And this will be plain, if a man has a true taste of them, as will be quickly and clearly seen. But what is a true taste? That we have to learn from the argument—the point being what is according to nature, and what is not according to nature. One life must be compared with another, the more pleasurable with the more painful, after this manner:—We desire to have pleasure, but we neither desire nor choose pain; and the neutral state we are ready to take in exchange, not for pleasure but for pain; and we also wish for less pain and greater pleasure, but less pleasure and greater pain we do not wish for; and an equal balance of either we cannot venture to assert that we should desire. And all these differ or do not differ severally in number and magnitude and intensity and equality, and in the opposites of these when regarded as objects of choice, in relation to desire. And such being the necessary order of things, we wish for that life in which there are many great and intense elements of pleasure and pain, and in which the pleasures are in excess, and do not wish for that in which the opposites exceed; nor, again, do we wish for that in which the elements of either are small and few and feeble, and the pains exceed. And when, as I said before, there is a balance of pleasure and pain in life, this is to be regarded by us as the balanced life; while other lives are preferred by us because they exceed in what we like, or are rejected by us because they exceed in what we dislike. All the lives of men may be regarded by us as bound up in these, and we must also consider what sort of lives we by nature desire. And if we wish for any others, I say that we desire them only through some ignorance and inexperience of the lives which actually exist.

\par  Now, what lives are they, and how many in which, having searched out and beheld the objects of will and desire and their opposites, and making of them a law, choosing, I say, the dear and the pleasant and the best and noblest, a man may live in the happiest way possible? Let us say that the temperate life is one kind of life, and the rational another, and the courageous another, and the healthful another; and to these four let us oppose four other lives—the foolish, the cowardly, the intemperate, the diseased. He who knows the temperate life will describe it as in all things gentle, having gentle pains and gentle pleasures, and placid desires and loves not insane; whereas the intemperate life is impetuous in all things, and has violent pains and pleasures, and vehement and stinging desires, and loves utterly insane; and in the temperate life the pleasures exceed the pains, but in the intemperate life the pains exceed the pleasures in greatness and number and frequency. Hence one of the two lives is naturally and necessarily more pleasant and the other more painful, and he who would live pleasantly cannot possibly choose to live intemperately. And if this is true, the inference clearly is that no man is voluntarily intemperate; but that the whole multitude of men lack temperance in their lives, either from ignorance, or from want of self-control, or both. And the same holds of the diseased and healthy life; they both have pleasures and pains, but in health the pleasure exceeds the pain, and in sickness the pain exceeds the pleasure. Now our intention in choosing the lives is not that the painful should exceed, but the life in which pain is exceeded by pleasure we have determined to be the more pleasant life. And we should say that the temperate life has the elements both of pleasure and pain fewer and smaller and less frequent than the intemperate, and the wise life than the foolish life, and the life of courage than the life of cowardice; one of each pair exceeding in pleasure and the other in pain, the courageous surpassing the cowardly, and the wise exceeding the foolish. And so the one class of lives exceeds the other class in pleasure; the temperate and courageous and wise and healthy exceed the cowardly and foolish and intemperate and diseased lives; and generally speaking, that which has any virtue, whether of body or soul, is pleasanter than the vicious life, and far superior in beauty and rectitude and excellence and reputation, and causes him who lives accordingly to be infinitely happier than the opposite.

\par  Enough of the preamble; and now the laws should follow; or, to speak more correctly, an outline of them. As, then, in the case of a web or any other tissue, the warp and the woof cannot be made of the same materials (compare Statesman), but the warp is necessarily superior as being stronger, and having a certain character of firmness, whereas the woof is softer and has a proper degree of elasticity;—in a similar manner those who are to hold great offices in states, should be distinguished truly in each case from those who have been but slenderly proven by education. Let us suppose that there are two parts in the constitution of a state—one the creation of offices, the other the laws which are assigned to them to administer.

\par  But, before all this, comes the following consideration:—The shepherd or herdsman, or breeder of horses or the like, when he has received his animals will not begin to train them until he has first purified them in a manner which befits a community of animals; he will divide the healthy and unhealthy, and the good breed and the bad breed, and will send away the unhealthy and badly bred to other herds, and tend the rest, reflecting that his labours will be vain and have no effect, either on the souls or bodies of those whom nature and ill nurture have corrupted, and that they will involve in destruction the pure and healthy nature and being of every other animal, if he should neglect to purify them. Now the case of other animals is not so important—they are only worth introducing for the sake of illustration; but what relates to man is of the highest importance; and the legislator should make enquiries, and indicate what is proper for each one in the way of purification and of any other procedure. Take, for example, the purification of a city—there are many kinds of purification, some easier and others more difficult; and some of them, and the best and most difficult of them, the legislator, if he be also a despot, may be able to effect; but the legislator, who, not being a despot, sets up a new government and laws, even if he attempt the mildest of purgations, may think himself happy if he can complete his work. The best kind of purification is painful, like similar cures in medicine, involving righteous punishment and inflicting death or exile in the last resort. For in this way we commonly dispose of great sinners who are incurable, and are the greatest injury of the whole state. But the milder form of purification is as follows:—when men who have nothing, and are in want of food, show a disposition to follow their leaders in an attack on the property of the rich—these, who are the natural plague of the state, are sent away by the legislator in a friendly spirit as far as he is able; and this dismissal of them is euphemistically termed a colony. And every legislator should contrive to do this at once. Our present case, however, is peculiar. For there is no need to devise any colony or purifying separation under the circumstances in which we are placed. But as, when many streams flow together from many sources, whether springs or mountain torrents, into a single lake, we ought to attend and take care that the confluent waters should be perfectly clear, and in order to effect this, should pump and draw off and divert impurities, so in every political arrangement there may be trouble and danger. But, seeing that we are now only discoursing and not acting, let our selection be supposed to be completed, and the desired purity attained. Touching evil men, who want to join and be citizens of our state, after we have tested them by every sort of persuasion and for a sufficient time, we will prevent them from coming; but the good we will to the utmost of our ability receive as friends with open arms.

\par  Another piece of good fortune must not be forgotten, which, as we were saying, the Heraclid colony had, and which is also ours,—that we have escaped division of land and the abolition of debts; for these are always a source of dangerous contention, and a city which is driven by necessity to legislate upon such matters can neither allow the old ways to continue, nor yet venture to alter them. We must have recourse to prayers, so to speak, and hope that a slight change may be cautiously effected in a length of time. And such a change can be accomplished by those who have abundance of land, and having also many debtors, are willing, in a kindly spirit, to share with those who are in want, sometimes remitting and sometimes giving, holding fast in a path of moderation, and deeming poverty to be the increase of a man's desires and not the diminution of his property. For this is the great beginning of salvation to a state, and upon this lasting basis may be erected afterwards whatever political order is suitable under the circumstances; but if the change be based upon an unsound principle, the future administration of the country will be full of difficulties. That is a danger which, as I am saying, is escaped by us, and yet we had better say how, if we had not escaped, we might have escaped; and we may venture now to assert that no other way of escape, whether narrow or broad, can be devised but freedom from avarice and a sense of justice—upon this rock our city shall be built; for there ought to be no disputes among citizens about property. If there are quarrels of long standing among them, no legislator of any degree of sense will proceed a step in the arrangement of the state until they are settled. But that they to whom God has given, as He has to us, to be the founders of a new state as yet free from enmity—that they should create themselves enmities by their mode of distributing lands and houses, would be superhuman folly and wickedness.

\par  How then can we rightly order the distribution of the land? In the first place, the number of the citizens has to be determined, and also the number and size of the divisions into which they will have to be formed; and the land and the houses will then have to be apportioned by us as fairly as we can. The number of citizens can only be estimated satisfactorily in relation to the territory and the neighbouring states. The territory must be sufficient to maintain a certain number of inhabitants in a moderate way of life—more than this is not required; and the number of citizens should be sufficient to defend themselves against the injustice of their neighbours, and also to give them the power of rendering efficient aid to their neighbours when they are wronged. After having taken a survey of their's and their neighbours' territory, we will determine the limits of them in fact as well as in theory. And now, let us proceed to legislate with a view to perfecting the form and outline of our state. The number of our citizens shall be 5040—this will be a convenient number; and these shall be owners of the land and protectors of the allotment. The houses and the land will be divided in the same way, so that every man may correspond to a lot. Let the whole number be first divided into two parts, and then into three; and the number is further capable of being divided into four or five parts, or any number of parts up to ten. Every legislator ought to know so much arithmetic as to be able to tell what number is most likely to be useful to all cities; and we are going to take that number which contains the greatest and most regular and unbroken series of divisions. The whole of number has every possible division, and the number 5040 can be divided by exactly fifty-nine divisors, and ten of these proceed without interval from one to ten: this will furnish numbers for war and peace, and for all contracts and dealings, including taxes and divisions of the land. These properties of number should be ascertained at leisure by those who are bound by law to know them; for they are true, and should be proclaimed at the foundation of the city, with a view to use. Whether the legislator is establishing a new state or restoring an old and decayed one, in respect of Gods and temples,—the temples which are to be built in each city, and the Gods or demi-gods after whom they are to be called,—if he be a man of sense, he will make no change in anything which the oracle of Delphi, or Dodona, or the God Ammon, or any ancient tradition has sanctioned in whatever manner, whether by apparitions or reputed inspiration of Heaven, in obedience to which mankind have established sacrifices in connexion with mystic rites, either originating on the spot, or derived from Tyrrhenia or Cyprus or some other place, and on the strength of which traditions they have consecrated oracles and images, and altars and temples, and portioned out a sacred domain for each of them. The least part of all these ought not to be disturbed by the legislator; but he should assign to the several districts some God, or demi-god, or hero, and, in the distribution of the soil, should give to these first their chosen domain and all things fitting, that the inhabitants of the several districts may meet at fixed times, and that they may readily supply their various wants, and entertain one another with sacrifices, and become friends and acquaintances; for there is no greater good in a state than that the citizens should be known to one another. When not light but darkness and ignorance of each other's characters prevails among them, no one will receive the honour of which he is deserving, or the power or the justice to which he is fairly entitled: wherefore, in every state, above all things, every man should take heed that he have no deceit in him, but that he be always true and simple; and that no deceitful person take any advantage of him.

\par  The next move in our pastime of legislation, like the withdrawal of the stone from the holy line in the game of draughts, being an unusual one, will probably excite wonder when mentioned for the first time. And yet, if a man will only reflect and weigh the matter with care, he will see that our city is ordered in a manner which, if not the best, is the second best. Perhaps also some one may not approve this form, because he thinks that such a constitution is ill adapted to a legislator who has not despotic power. The truth is, that there are three forms of government, the best, the second and the third best, which we may just mention, and then leave the selection to the ruler of the settlement. Following this method in the present instance, let us speak of the states which are respectively first, second, and third in excellence, and then we will leave the choice to Cleinias now, or to any one else who may hereafter have to make a similar choice among constitutions, and may desire to give to his state some feature which is congenial to him and which he approves in his own country.

\par  The first and highest form of the state and of the government and of the law is that in which there prevails most widely the ancient saying, that 'Friends have all things in common.' Whether there is anywhere now, or will ever be, this communion of women and children and of property, in which the private and individual is altogether banished from life, and things which are by nature private, such as eyes and ears and hands, have become common, and in some way see and hear and act in common, and all men express praise and blame and feel joy and sorrow on the same occasions, and whatever laws there are unite the city to the utmost (compare Republic),—whether all this is possible or not, I say that no man, acting upon any other principle, will ever constitute a state which will be truer or better or more exalted in virtue. Whether such a state is governed by Gods or sons of Gods, one, or more than one, happy are the men who, living after this manner, dwell there; and therefore to this we are to look for the pattern of the state, and to cling to this, and to seek with all our might for one which is like this. The state which we have now in hand, when created, will be nearest to immortality and the only one which takes the second place; and after that, by the grace of God, we will complete the third one. And we will begin by speaking of the nature and origin of the second.

\par  Let the citizens at once distribute their land and houses, and not till the land in common, since a community of goods goes beyond their proposed origin, and nurture, and education. But in making the distribution, let the several possessors feel that their particular lots also belong to the whole city; and seeing that the earth is their parent, let them tend her more carefully than children do their mother. For she is a goddess and their queen, and they are her mortal subjects. Such also are the feelings which they ought to entertain to the Gods and demi-gods of the country. And in order that the distribution may always remain, they ought to consider further that the present number of families should be always retained, and neither increased nor diminished. This may be secured for the whole city in the following manner:—Let the possessor of a lot leave the one of his children who is his best beloved, and one only, to be the heir of his dwelling, and his successor in the duty of ministering to the Gods, the state and the family, as well the living members of it as those who are departed when he comes into the inheritance; but of his other children, if he have more than one, he shall give the females in marriage according to the law to be hereafter enacted, and the males he shall distribute as sons to those citizens who have no children, and are disposed to receive them; or if there should be none such, and particular individuals have too many children, male or female, or too few, as in the case of barrenness—in all these cases let the highest and most honourable magistracy created by us judge and determine what is to be done with the redundant or deficient, and devise a means that the number of 5040 houses shall always remain the same. There are many ways of regulating numbers; for they in whom generation is affluent may be made to refrain (compare Arist. Pol. ), and, on the other hand, special care may be taken to increase the number of births by rewards and stigmas, or we may meet the evil by the elder men giving advice and administering rebuke to the younger—in this way the object may be attained. And if after all there be very great difficulty about the equal preservation of the 5040 houses, and there be an excess of citizens, owing to the too great love of those who live together, and we are at our wits' end, there is still the old device often mentioned by us of sending out a colony, which will part friends with us, and be composed of suitable persons. If, on the other hand, there come a wave bearing a deluge of disease, or a plague of war, and the inhabitants become much fewer than the appointed number by reason of bereavement, we ought not to introduce citizens of spurious birth and education, if this can be avoided; but even God is said not to be able to fight against necessity.

\par  Wherefore let us suppose this 'high argument' of ours to address us in the following terms:—Best of men, cease not to honour according to nature similarity and equality and sameness and agreement, as regards number and every good and noble quality. And, above all, observe the aforesaid number 5040 throughout life; in the second place, do not disparage the small and modest proportions of the inheritances which you received in the distribution, by buying and selling them to one another. For then neither will the God who gave you the lot be your friend, nor will the legislator; and indeed the law declares to the disobedient that these are the terms upon which he may or may not take the lot. In the first place, the earth as he is informed is sacred to the Gods; and in the next place, priests and priestesses will offer up prayers over a first, and second, and even a third sacrifice, that he who buys or sells the houses or lands which he has received, may suffer the punishment which he deserves; and these their prayers they shall write down in the temples, on tablets of cypress-wood, for the instruction of posterity. Moreover they will set a watch over all these things, that they may be observed;—the magistracy which has the sharpest eyes shall keep watch that any infringement of these commands may be discovered and punished as offences both against the law and the God. How great is the benefit of such an ordinance to all those cities, which obey and are administered accordingly, no bad man can ever know, as the old proverb says; but only a man of experience and good habits. For in such an order of things there will not be much opportunity for making money; no man either ought, or indeed will be allowed, to exercise any ignoble occupation, of which the vulgarity is a matter of reproach to a freeman, and should never want to acquire riches by any such means.

\par  Further, the law enjoins that no private man shall be allowed to possess gold and silver, but only coin for daily use, which is almost necessary in dealing with artisans, and for payment of hirelings, whether slaves or immigrants, by all those persons who require the use of them. Wherefore our citizens, as we say, should have a coin passing current among themselves, but not accepted among the rest of mankind; with a view, however, to expeditions and journeys to other lands,—for embassies, or for any other occasion which may arise of sending out a herald, the state must also possess a common Hellenic currency. If a private person is ever obliged to go abroad, let him have the consent of the magistrates and go; and if when he returns he has any foreign money remaining, let him give the surplus back to the treasury, and receive a corresponding sum in the local currency. And if he is discovered to appropriate it, let it be confiscated, and let him who knows and does not inform be subject to curse and dishonour equally him who brought the money, and also to a fine not less in amount than the foreign money which has been brought back. In marrying and giving in marriage, no one shall give or receive any dowry at all; and no one shall deposit money with another whom he does not trust as a friend, nor shall he lend money upon interest; and the borrower should be under no obligation to repay either capital or interest. That these principles are best, any one may see who compares them with the first principle and intention of a state. The intention, as we affirm, of a reasonable statesman, is not what the many declare to be the object of a good legislator, namely, that the state for the true interests of which he is advising should be as great and as rich as possible, and should possess gold and silver, and have the greatest empire by sea and land;—this they imagine to be the real object of legislation, at the same time adding, inconsistently, that the true legislator desires to have the city the best and happiest possible. But they do not see that some of these things are possible, and some of them are impossible; and he who orders the state will desire what is possible, and will not indulge in vain wishes or attempts to accomplish that which is impossible. The citizen must indeed be happy and good, and the legislator will seek to make him so; but very rich and very good at the same time he cannot be, not, at least, in the sense in which the many speak of riches. For they mean by 'the rich' the few who have the most valuable possessions, although the owner of them may quite well be a rogue. And if this is true, I can never assent to the doctrine that the rich man will be happy—he must be good as well as rich. And good in a high degree, and rich in a high degree at the same time, he cannot be. Some one will ask, why not? And we shall answer—Because acquisitions which come from sources which are just and unjust indifferently, are more than double those which come from just sources only; and the sums which are expended neither honourably nor disgracefully, are only half as great as those which are expended honourably and on honourable purposes. Thus, if the one acquires double and spends half, the other who is in the opposite case and is a good man cannot possibly be wealthier than he. The first—I am speaking of the saver and not of the spender—is not always bad; he may indeed in some cases be utterly bad, but, as I was saying, a good man he never is. For he who receives money unjustly as well as justly, and spends neither nor unjustly, will be a rich man if he be also thrifty. On the other hand, the utterly bad is in general profligate, and therefore very poor; while he who spends on noble objects, and acquires wealth by just means only, can hardly be remarkable for riches, any more than he can be very poor. Our statement, then, is true, that the very rich are not good, and, if they are not good, they are not happy. But the intention of our laws was, that the citizens should be as happy as may be, and as friendly as possible to one another. And men who are always at law with one another, and amongst whom there are many wrongs done, can never be friends to one another, but only those among whom crimes and lawsuits are few and slight. Therefore we say that gold and silver ought not to be allowed in the city, nor much of the vulgar sort of trade which is carried on by lending money, or rearing the meaner kinds of live stock; but only the produce of agriculture, and only so much of this as will not compel us in pursuing it to neglect that for the sake of which riches exist—I mean, soul and body, which without gymnastics, and without education, will never be worth anything; and therefore, as we have said not once but many times, the care of riches should have the last place in our thoughts. For there are in all three things about which every man has an interest; and the interest about money, when rightly regarded, is the third and lowest of them: midway comes the interest of the body; and, first of all, that of the soul; and the state which we are describing will have been rightly constituted if it ordains honours according to this scale. But if, in any of the laws which have been ordained, health has been preferred to temperance, or wealth to health and temperate habits, that law must clearly be wrong. Wherefore, also, the legislator ought often to impress upon himself the question—'What do I want?' and 'Do I attain my aim, or do I miss the mark?' In this way, and in this way only, he may acquit himself and free others from the work of legislation.

\par  Let the allottee then hold his lot upon the conditions which we have mentioned.

\par  It would be well that every man should come to the colony having all things equal; but seeing that this is not possible, and one man will have greater possessions than another, for many reasons and in particular in order to preserve equality in special crises of the state, qualifications of property must be unequal, in order that offices and contributions and distributions may be proportioned to the value of each person's wealth, and not solely to the virtue of his ancestors or himself, nor yet to the strength and beauty of his person, but also to the measure of his wealth or poverty; and so by a law of inequality, which will be in proportion to his wealth, he will receive honours and offices as equally as possible, and there will be no quarrels and disputes. To which end there should be four different standards appointed according to the amount of property: there should be a first and a second and a third and a fourth class, in which the citizens will be placed, and they will be called by these or similar names: they may continue in the same rank, or pass into another in any individual case, on becoming richer from being poorer, or poorer from being richer. The form of law which I should propose as the natural sequel would be as follows:—In a state which is desirous of being saved from the greatest of all plagues—not faction, but rather distraction;—there should exist among the citizens neither extreme poverty, nor, again, excess of wealth, for both are productive of both these evils. Now the legislator should determine what is to be the limit of poverty or wealth. Let the limit of poverty be the value of the lot; this ought to be preserved, and no ruler, nor any one else who aspires after a reputation for virtue, will allow the lot to be impaired in any case. This the legislator gives as a measure, and he will permit a man to acquire double or triple, or as much as four times the amount of this (compare Arist. Pol.). But if a person have yet greater riches, whether he has found them, or they have been given to him, or he has made them in business, or has acquired by any stroke of fortune that which is in excess of the measure, if he give back the surplus to the state, and to the Gods who are the patrons of the state, he shall suffer no penalty or loss of reputation; but if he disobeys this our law, any one who likes may inform against him and receive half the value of the excess, and the delinquent shall pay a sum equal to the excess out of his own property, and the other half of the excess shall belong to the Gods. And let every possession of every man, with the exception of the lot, be publicly registered before the magistrates whom the law appoints, so that all suits about money may be easy and quite simple.

\par  The next thing to be noted is, that the city should be placed as nearly as possible in the centre of the country; we should choose a place which possesses what is suitable for a city, and this may easily be imagined and described. Then we will divide the city into twelve portions, first founding temples to Hestia, to Zeus and to Athene, in a spot which we will call the Acropolis, and surround with a circular wall, making the division of the entire city and country radiate from this point. The twelve portions shall be equalized by the provision that those which are of good land shall be smaller, while those of inferior quality shall be larger. The number of the lots shall be 5040, and each of them shall be divided into two, and every allotment shall be composed of two such sections; one of land near the city, the other of land which is at a distance (compare Arist. Pol.). This arrangement shall be carried out in the following manner: The section which is near the city shall be added to that which is on the borders, and form one lot, and the portion which is next nearest shall be added to the portion which is next farthest; and so of the rest. Moreover, in the two sections of the lots the same principle of equalization of the soil ought to be maintained; the badness and goodness shall be compensated by more and less. And the legislator shall divide the citizens into twelve parts, and arrange the rest of their property, as far as possible, so as to form twelve equal parts; and there shall be a registration of all. After this they shall assign twelve lots to twelve Gods, and call them by their names, and dedicate to each God their several portions, and call the tribes after them. And they shall distribute the twelve divisions of the city in the same way in which they divided the country; and every man shall have two habitations, one in the centre of the country, and the other at the extremity. Enough of the manner of settlement.

\par  Now we ought by all means to consider that there can never be such a happy concurrence of circumstances as we have described; neither can all things coincide as they are wanted. Men who will not take offence at such a mode of living together, and will endure all their life long to have their property fixed at a moderate limit, and to beget children in accordance with our ordinances, and will allow themselves to be deprived of gold and other things which the legislator, as is evident from these enactments, will certainly forbid them; and will endure, further, the situation of the land with the city in the middle and dwellings round about;—all this is as if the legislator were telling his dreams, or making a city and citizens of wax. There is truth in these objections, and therefore every one should take to heart what I am going to say. Once more, then, the legislator shall appear and address us:—'O my friends,' he will say to us, 'do not suppose me ignorant that there is a certain degree of truth in your words; but I am of opinion that, in matters which are not present but future, he who exhibits a pattern of that at which he aims, should in nothing fall short of the fairest and truest; and that if he finds any part of this work impossible of execution he should avoid and not execute it, but he should contrive to carry out that which is nearest and most akin to it; you must allow the legislator to perfect his design, and when it is perfected, you should join with him in considering what part of his legislation is expedient and what will arouse opposition; for surely the artist who is to be deemed worthy of any regard at all, ought always to make his work self-consistent.'

\par  Having determined that there is to be a distribution into twelve parts, let us now see in what way this may be accomplished. There is no difficulty in perceiving that the twelve parts admit of the greatest number of divisions of that which they include, or in seeing the other numbers which are consequent upon them, and are produced out of them up to 5040; wherefore the law ought to order phratries and demes and villages, and also military ranks and movements, as well as coins and measures, dry and liquid, and weights, so as to be commensurable and agreeable to one another. Nor should we fear the appearance of minuteness, if the law commands that all the vessels which a man possesses should have a common measure, when we consider generally that the divisions and variations of numbers have a use in respect of all the variations of which they are susceptible, both in themselves and as measures of height and depth, and in all sounds, and in motions, as well those which proceed in a straight direction, upwards or downwards, as in those which go round and round. The legislator is to consider all these things and to bid the citizens, as far as possible, not to lose sight of numerical order; for no single instrument of youthful education has such mighty power, both as regards domestic economy and politics, and in the arts, as the study of arithmetic. Above all, arithmetic stirs up him who is by nature sleepy and dull, and makes him quick to learn, retentive, shrewd, and aided by art divine he makes progress quite beyond his natural powers (compare Republic). All such things, if only the legislator, by other laws and institutions, can banish meanness and covetousness from the souls of men, so that they can use them properly and to their own good, will be excellent and suitable instruments of education. But if he cannot, he will unintentionally create in them, instead of wisdom, the habit of craft, which evil tendency may be observed in the Egyptians and Phoenicians, and many other races, through the general vulgarity of their pursuits and acquisitions, whether some unworthy legislator of theirs has been the cause, or some impediment of chance or nature. For we must not fail to observe, O Megillus and Cleinias, that there is a difference in places, and that some beget better men and others worse; and we must legislate accordingly. Some places are subject to strange and fatal influences by reason of diverse winds and violent heats, some by reason of waters; or, again, from the character of the food given by the earth, which not only affects the bodies of men for good or evil, but produces similar results in their souls. And in all such qualities those spots excel in which there is a divine inspiration, and in which the demigods have their appointed lots, and are propitious, not adverse, to the settlers in them. To all these matters the legislator, if he have any sense in him, will attend as far as man can, and frame his laws accordingly. And this is what you, Cleinias, must do, and to matters of this kind you must turn your mind since you are going to colonize a new country.

\par \textbf{CLEINIAS}
\par   Your words, Athenian Stranger, are excellent, and I will do as you say.

\par 
\section{
      BOOK VI.
    }
\par \textbf{ATHENIAN}
\par   And now having made an end of the preliminaries we will proceed to the appointment of magistracies.

\par \textbf{CLEINIAS}
\par   Very good.

\par \textbf{ATHENIAN}
\par   In the ordering of a state there are two parts:  first, the number of the magistracies, and the mode of establishing them; and, secondly, when they have been established, laws again will have to be provided for each of them, suitable in nature and number. But before electing the magistrates let us stop a little and say a word in season about the election of them.

\par \textbf{CLEINIAS}
\par   What have you got to say?

\par \textbf{ATHENIAN}
\par   This is what I have to say;—every one can see, that although the work of legislation is a most important matter, yet if a well-ordered city superadd to good laws unsuitable offices, not only will there be no use in having the good laws,—not only will they be ridiculous and useless, but the greatest political injury and evil will accrue from them.

\par \textbf{CLEINIAS}
\par   Of course.

\par \textbf{ATHENIAN}
\par   Then now, my friend, let us observe what will happen in the constitution of out intended state. In the first place, you will acknowledge that those who are duly appointed to magisterial power, and their families, should severally have given satisfactory proof of what they are, from youth upward until the time of election; in the next place, those who are to elect should have been trained in habits of law, and be well educated, that they may have a right judgment, and may be able to select or reject men whom they approve or disapprove, as they are worthy of either. But how can we imagine that those who are brought together for the first time, and are strangers to one another, and also uneducated, will avoid making mistakes in the choice of magistrates?

\par \textbf{CLEINIAS}
\par   Impossible.

\par \textbf{ATHENIAN}
\par   The matter is serious, and excuses will not serve the turn. I will tell you, then, what you and I will have to do, since you, as you tell me, with nine others, have offered to settle the new state on behalf of the people of Crete, and I am to help you by the invention of the present romance. I certainly should not like to leave the tale wandering all over the world without a head;—a headless monster is such a hideous thing.

\par \textbf{CLEINIAS}
\par   Excellent, Stranger.

\par \textbf{ATHENIAN}
\par   Yes; and I will be as good as my word.

\par \textbf{CLEINIAS}
\par   Let us by all means do as you propose.

\par \textbf{ATHENIAN}
\par   That we will, by the grace of God, if old age will only permit us.

\par \textbf{CLEINIAS}
\par   But God will be gracious.

\par \textbf{ATHENIAN}
\par   Yes; and under his guidance let us consider a further point.

\par \textbf{CLEINIAS}
\par   What is it?

\par \textbf{ATHENIAN}
\par   Let us remember what a courageously mad and daring creation this our city is.

\par \textbf{CLEINIAS}
\par   What had you in your mind when you said that?

\par \textbf{ATHENIAN}
\par   I had in my mind the free and easy manner in which we are ordaining that the inexperienced colonists shall receive our laws. Now a man need not be very wise, Cleinias, in order to see that no one can easily receive laws at their first imposition. But if we could anyhow wait until those who have been imbued with them from childhood, and have been nurtured in them, and become habituated to them, take their part in the public elections of the state; I say, if this could be accomplished, and rightly accomplished by any way or contrivance—then, I think that there would be very little danger, at the end of the time, of a state thus trained not being permanent.

\par \textbf{CLEINIAS}
\par   A reasonable supposition.

\par \textbf{ATHENIAN}
\par   Then let us consider if we can find any way out of the difficulty; for I maintain, Cleinias, that the Cnosians, above all the other Cretans, should not be satisfied with barely discharging their duty to the colony, but they ought to take the utmost pains to establish the offices which are first created by them in the best and surest manner. Above all, this applies to the selection of the guardians of the law, who must be chosen first of all, and with the greatest care; the others are of less importance.

\par \textbf{CLEINIAS}
\par   What method can we devise of electing them?

\par \textbf{ATHENIAN}
\par   This will be the method: —Sons of the Cretans, I shall say to them, inasmuch as the Cnosians have precedence over the other states, they should, in common with those who join this settlement, choose a body of thirty-seven in all, nineteen of them being taken from the settlers, and the remainder from the citizens of Cnosus. Of these latter the Cnosians shall make a present to your colony, and you yourself shall be one of the eighteen, and shall become a citizen of the new state; and if you and they cannot be persuaded to go, the Cnosians may fairly use a little violence in order to make you.

\par \textbf{CLEINIAS}
\par   But why, Stranger, do not you and Megillus take a part in our new city?

\par \textbf{ATHENIAN}
\par   O, Cleinias, Athens is proud, and Sparta too; and they are both a long way off. But you and likewise the other colonists are conveniently situated as you describe. I have been speaking of the way in which the new citizens may be best managed under present circumstances; but in after-ages, if the city continues to exist, let the election be on this wise. All who are horse or foot soldiers, or have seen military service at the proper ages when they were severally fitted for it (compare Arist. Pol. ), shall share in the election of magistrates; and the election shall be held in whatever temple the state deems most venerable, and every one shall carry his vote to the altar of the God, writing down on a tablet the name of the person for whom he votes, and his father's name, and his tribe, and ward; and at the side he shall write his own name in like manner. Any one who pleases may take away any tablet which he does not think properly filled up, and exhibit it in the Agora for a period of not less than thirty days. The tablets which are judged to be first, to the number of 300, shall be shown by the magistrates to the whole city, and the citizens shall in like manner select from these the candidates whom they prefer; and this second selection, to the number of 100, shall be again exhibited to the citizens; in the third, let any one who pleases select whom he pleases out of the 100, walking through the parts of victims, and let them choose for magistrates and proclaim the seven-and-thirty who have the greatest number of votes. But who, Cleinias and Megillus, will order for us in the colony all this matter of the magistrates, and the scrutinies of them? If we reflect, we shall see that cities which are in process of construction like ours must have some such persons, who cannot possibly be elected before there are any magistrates; and yet they must be elected in some way, and they are not to be inferior men, but the best possible. For as the proverb says, 'a good beginning is half the business'; and 'to have begun well' is praised by all, and in my opinion is a great deal more than half the business, and has never been praised by any one enough.

\par \textbf{CLEINIAS}
\par   That is very true.

\par \textbf{ATHENIAN}
\par   Then let us recognize the difficulty, and make clear to our own minds how the beginning is to be accomplished. There is only one proposal which I have to offer, and that is one which, under our circumstances, is both necessary and expedient.

\par \textbf{CLEINIAS}
\par   What is it?

\par \textbf{ATHENIAN}
\par   I maintain that this colony of ours has a father and mother, who are no other than the colonizing state. Well I know that many colonies have been, and will be, at enmity with their parents. But in early days the child, as in a family, loves and is beloved; even if there come a time later when the tie is broken, still, while he is in want of education, he naturally loves his parents and is beloved by them, and flies to his relatives for protection, and finds in them his only natural allies in time of need; and this parental feeling already exists in the Cnosians, as is shown by their care of the new city; and there is a similar feeling on the part of the young city towards Cnosus. And I repeat what I was saying—for there is no harm in repeating a good thing—that the Cnosians should take a common interest in all these matters, and choose, as far as they can, the eldest and best of the colonists, to the number of not less than a hundred; and let there be another hundred of the Cnosians themselves. These, I say, on their arrival, should have a joint care that the magistrates should be appointed according to law, and that when they are appointed they should undergo a scrutiny. When this has been effected, the Cnosians shall return home, and the new city do the best she can for her own preservation and happiness. I would have the seven-and-thirty now, and in all future time, chosen to fulfil the following duties: —Let them, in the first place, be the guardians of the law; and, secondly, of the registers in which each one registers before the magistrate the amount of his property, excepting four minae which are allowed to citizens of the first class, three allowed to the second, two to the third, and a single mina to the fourth. And if any one, despising the laws for the sake of gain, be found to possess anything more which has not been registered, let all that he has in excess be confiscated, and let him be liable to a suit which shall be the reverse of honourable or fortunate. And let any one who will, indict him on the charge of loving base gains, and proceed against him before the guardians of the law. And if he be cast, let him lose his share of the public possessions, and when there is any public distribution, let him have nothing but his original lot; and let him be written down a condemned man as long as he lives, in some place in which any one who pleases can read about his offences. The guardian of the law shall not hold office longer than twenty years, and shall not be less than fifty years of age when he is elected; or if he is elected when he is sixty years of age, he shall hold office for ten years only; and upon the same principle, he must not imagine that he will be permitted to hold such an important office as that of guardian of the laws after he is seventy years of age, if he live so long.

\par  These are the three first ordinances about the guardians of the law; as the work of legislation progresses, each law in turn will assign to them their further duties. And now we may proceed in order to speak of the election of other officers; for generals have to be elected, and these again must have their ministers, commanders, and colonels of horse, and commanders of brigades of foot, who would be more rightly called by their popular name of brigadiers. The guardians of the law shall propose as generals men who are natives of the city, and a selection from the candidates proposed shall be made by those who are or have been of the age for military service. And if one who is not proposed is thought by somebody to be better than one who is, let him name whom he prefers in the place of whom, and make oath that he is better, and propose him; and whichever of them is approved by vote shall be admitted to the final selection; and the three who have the greatest number of votes shall be appointed generals, and superintendents of military affairs, after previously undergoing a scrutiny, like the guardians of the law. And let the generals thus elected propose twelve brigadiers, one for each tribe; and there shall be a right of counter-proposal as in the case of the generals, and the voting and decision shall take place in the same way. Until the prytanes and council are elected, the guardians of the law shall convene the assembly in some holy spot which is suitable to the purpose, placing the hoplites by themselves, and the cavalry by themselves, and in a third division all the rest of the army. All are to vote for the generals (and for the colonels of horse), but the brigadiers are to be voted for only by those who carry shields (i.e. the hoplites). Let the body of cavalry choose phylarchs for the generals; but captains of light troops, or archers, or any other division of the army, shall be appointed by the generals for themselves. There only remains the appointment of officers of cavalry: these shall be proposed by the same persons who proposed the generals, and the election and the counter-proposal of other candidates shall be arranged in the same way as in the case of the generals, and let the cavalry vote and the infantry look on at the election; the two who have the greatest number of votes shall be the leaders of all the horse. Disputes about the voting may be raised once or twice; but if the dispute be raised a third time, the officers who preside at the several elections shall decide.

\par  The council shall consist of 30 x 12 members—360 will be a convenient number for sub-division. If we divide the whole number into four parts of ninety each, we get ninety counsellors for each class. First, all the citizens shall select candidates from the first class; they shall be compelled to vote, and, if they do not, shall be duly fined. When the candidates have been selected, some one shall mark them down; this shall be the business of the first day. And on the following day, candidates shall be selected from the second class in the same manner and under the same conditions as on the previous day; and on the third day a selection shall be made from the third class, at which every one may, if he likes vote, and the three first classes shall be compelled to vote; but the fourth and lowest class shall be under no compulsion, and any member of this class who does not vote shall not be punished. On the fourth day candidates shall be selected from the fourth and smallest class; they shall be selected by all, but he who is of the fourth class shall suffer no penalty, nor he who is of the third, if he be not willing to vote; but he who is of the first or second class, if he does not vote shall be punished;—he who is of the second class shall pay a fine of triple the amount which was exacted at first, and he who is of the first class quadruple. On the fifth day the rulers shall bring out the names noted down, for all the citizens to see, and every man shall choose out of them, under pain, if he do not, of suffering the first penalty; and when they have chosen 180 out of each of the classes, they shall choose one-half of them by lot, who shall undergo a scrutiny:—These are to form the council for the year.

\par  The mode of election which has been described is in a mean between monarchy and democracy, and such a mean the state ought always to observe; for servants and masters never can be friends, nor good and bad, merely because they are declared to have equal privileges. For to unequals equals become unequal, if they are not harmonised by measure; and both by reason of equality, and by reason of inequality, cities are filled with seditions. The old saying, that 'equality makes friendship,' is happy and also true; but there is obscurity and confusion as to what sort of equality is meant. For there are two equalities which are called by the same name, but are in reality in many ways almost the opposite of one another; one of them may be introduced without difficulty, by any state or any legislator in the distribution of honours: this is the rule of measure, weight, and number, which regulates and apportions them. But there is another equality, of a better and higher kind, which is not so easily recognized. This is the judgment of Zeus; among men it avails but little; that little, however, is the source of the greatest good to individuals and states. For it gives to the greater more, and to the inferior less and in proportion to the nature of each; and, above all, greater honour always to the greater virtue, and to the less less; and to either in proportion to their respective measure of virtue and education. And this is justice, and is ever the true principle of states, at which we ought to aim, and according to this rule order the new city which is now being founded, and any other city which may be hereafter founded. To this the legislator should look,—not to the interests of tyrants one or more, or to the power of the people, but to justice always; which, as I was saying, is the distribution of natural equality among unequals in each case. But there are times at which every state is compelled to use the words, 'just,' 'equal,' in a secondary sense, in the hope of escaping in some degree from factions. For equity and indulgence are infractions of the perfect and strict rule of justice. And this is the reason why we are obliged to use the equality of the lot, in order to avoid the discontent of the people; and so we invoke God and fortune in our prayers, and beg that they themselves will direct the lot with a view to supreme justice. And therefore, although we are compelled to use both equalities, we should use that into which the element of chance enters as seldom as possible.

\par  Thus, O my friends, and for the reasons given, should a state act which would endure and be saved. But as a ship sailing on the sea has to be watched night and day, in like manner a city also is sailing on a sea of politics, and is liable to all sorts of insidious assaults; and therefore from morning to night, and from night to morning, rulers must join hands with rulers, and watchers with watchers, receiving and giving up their trust in a perpetual succession. Now a multitude can never fulfil a duty of this sort with anything like energy. Moreover, the greater number of the senators will have to be left during the greater part of the year to order their concerns at their own homes. They will therefore have to be arranged in twelve portions, answering to the twelve months, and furnish guardians of the state, each portion for a single month. Their business is to be at hand and receive any foreigner or citizen who comes to them, whether to give information, or to put one of those questions, to which, when asked by other cities, a city should give an answer, and to which, if she ask them herself, she should receive an answer; or again, when there is a likelihood of internal commotions, which are always liable to happen in some form or other, they will, if they can, prevent their occurring; or if they have already occurred, will lose no time in making them known to the city, and healing the evil. Wherefore, also, this which is the presiding body of the state ought always to have the control of their assemblies, and of the dissolutions of them, ordinary as well as extraordinary. All this is to be ordered by the twelfth part of the council, which is always to keep watch together with the other officers of the state during one portion of the year, and to rest during the remaining eleven portions.

\par  Thus will the city be fairly ordered. And now, who is to have the superintendence of the country, and what shall be the arrangement? Seeing that the whole city and the entire country have been both of them divided into twelve portions, ought there not to be appointed superintendents of the streets of the city, and of the houses, and buildings, and harbours, and the agora, and fountains, and sacred domains, and temples, and the like?

\par \textbf{CLEINIAS}
\par   To be sure there ought.

\par \textbf{ATHENIAN}
\par   Let us assume, then, that there ought to be servants of the temples, and priests and priestesses. There must also be superintendents of roads and buildings, who will have a care of men, that they may do no harm, and also of beasts, both within the enclosure and in the suburbs. Three kinds of officers will thus have to be appointed, in order that the city may be suitably provided according to her needs. Those who have the care of the city shall be called wardens of the city; and those who have the care of the agora shall be called wardens of the agora; and those who have the care of the temples shall be called priests. Those who hold hereditary offices as priests or priestesses, shall not be disturbed; but if there be few or none such, as is probable at the foundation of a new city, priests and priestesses shall be appointed to be servants of the Gods who have no servants. Some of our officers shall be elected, and others appointed by lot, those who are of the people and those who are not of the people mingling in a friendly manner in every place and city, that the state may be as far as possible of one mind. The officers of the temples shall be appointed by lot; in this way their election will be committed to God, that He may do what is agreeable to Him. And he who obtains a lot shall undergo a scrutiny, first, as to whether he is sound of body and of legitimate birth; and in the second place, in order to show that he is of a perfectly pure family, not stained with homicide or any similar impiety in his own person, and also that his father and mother have led a similar unstained life. Now the laws about all divine things should be brought from Delphi, and interpreters appointed, under whose direction they should be used. The tenure of the priesthood should always be for a year and no longer; and he who will duly execute the sacred office, according to the laws of religion, must be not less than sixty years of age—the laws shall be the same about priestesses. As for the interpreters, they shall be appointed thus: —Let the twelve tribes be distributed into groups of four, and let each group select four, one out of each tribe within the group, three times; and let the three who have the greatest number of votes (out of the twelve appointed by each group), after undergoing a scrutiny, nine in all, be sent to Delphi, in order that the God may return one out of each triad; their age shall be the same as that of the priests, and the scrutiny of them shall be conducted in the same manner; let them be interpreters for life, and when any one dies let the four tribes select another from the tribe of the deceased. Moreover, besides priests and interpreters, there must be treasurers, who will take charge of the property of the several temples, and of the sacred domains, and shall have authority over the produce and the letting of them; and three of them shall be chosen from the highest classes for the greater temples, and two for the lesser, and one for the least of all; the manner of their election and the scrutiny of them shall be the same as that of the generals. This shall be the order of the temples.

\par  Let everything have a guard as far as possible. Let the defence of the city be commited to the generals, and taxiarchs, and hipparchs, and phylarchs, and prytanes, and the wardens of the city, and of the agora, when the election of them has been completed. The defence of the country shall be provided for as follows:—The entire land has been already distributed into twelve as nearly as possible equal parts, and let the tribe allotted to a division provide annually for it five wardens of the country and commanders of the watch; and let each body of five have the power of selecting twelve others out of the youth of their own tribe,—these shall be not less than twenty-five years of age, and not more than thirty. And let there be allotted to them severally every month the various districts, in order that they may all acquire knowledge and experience of the whole country. The term of service for commanders and for watchers shall continue during two years. After having had their stations allotted to them, they will go from place to place in regular order, making their round from left to right as their commanders direct them; (when I speak of going to the right, I mean that they are to go to the east). And at the commencement of the second year, in order that as many as possible of the guards may not only get a knowledge of the country at any one season of the year, but may also have experience of the manner in which different places are affected at different seasons of the year, their then commanders shall lead them again towards the left, from place to place in succession, until they have completed the second year. In the third year other wardens of the country shall be chosen and commanders of the watch, five for each division, who are to be the superintendents of the bands of twelve. While on service at each station, their attention shall be directed to the following points:—In the first place, they shall see that the country is well protected against enemies; they shall trench and dig wherever this is required, and, as far as they can, they shall by fortifications keep off the evil-disposed, in order to prevent them from doing any harm to the country or the property; they shall use the beasts of burden and the labourers whom they find on the spot: these will be their instruments whom they will superintend, taking them, as far as possible, at the times when they are not engaged in their regular business. They shall make every part of the country inaccessible to enemies, and as accessible as possible to friends (compare Arist. Pol. ); there shall be ways for man and beasts of burden and for cattle, and they shall take care to have them always as smooth as they can; and shall provide against the rains doing harm instead of good to the land, when they come down from the mountains into the hollow dells; and shall keep in the overflow by the help of works and ditches, in order that the valleys, receiving and drinking up the rain from heaven, and providing fountains and streams in the fields and regions which lie underneath, may furnish even to the dry places plenty of good water. The fountains of water, whether of rivers or of springs, shall be ornamented with plantations and buildings for beauty; and let them bring together the streams in subterraneous channels, and make all things plenteous; and if there be a sacred grove or dedicated precinct in the neighbourhood, they shall conduct the water to the actual temples of the Gods, and so beautify them at all seasons of the year. Everywhere in such places the youth shall make gymnasia for themselves, and warm baths for the aged, placing by them abundance of dry wood, for the benefit of those labouring under disease—there the weary frame of the rustic, worn with toil, will receive a kindly welcome, far better than he would at the hands of a not over-wise doctor.

\par  The building of these and the like works will be useful and ornamental; they will provide a pleasing amusement, but they will be a serious employment too; for the sixty wardens will have to guard their several divisions, not only with a view to enemies, but also with an eye to professing friends. When a quarrel arises among neighbours or citizens, and any one whether slave or freeman wrongs another, let the five wardens decide small matters on their own authority; but where the charge against another relates to greater matters, the seventeen composed of the fives and twelves, shall determine any charges which one man brings against another, not involving more than three minae. Every judge and magistrate shall be liable to give an account of his conduct in office, except those who, like kings, have the final decision. Moreover, as regards the aforesaid wardens of the country, if they do any wrong to those of whom they have the care, whether by imposing upon them unequal tasks, or by taking the produce of the soil or implements of husbandry without their consent; also if they receive anything in the way of a bribe, or decide suits unjustly, or if they yield to the influences of flattery, let them be publicly dishonoured; and in regard to any other wrong which they do to the inhabitants of the country, if the question be of a mina, let them submit to the decision of the villagers in the neighbourhood; but in suits of greater amount, or in case of lesser, if they refuse to submit, trusting that their monthly removal into another part of the country will enable them to escape—in such cases the injured party may bring his suit in the common court, and if he obtain a verdict he may exact from the defendant, who refused to submit, a double penalty.

\par  The wardens and the overseers of the country, while on their two years' service, shall have common meals at their several stations, and shall all live together; and he who is absent from the common meal, or sleeps out, if only for one day or night, unless by order of his commanders, or by reason of absolute necessity, if the five denounce him and inscribe his name in the agora as not having kept his guard, let him be deemed to have betrayed the city, as far as lay in his power, and let him be disgraced and beaten with impunity by any one who meets him and is willing to punish him. If any of the commanders is guilty of such an irregularity, the whole company of sixty shall see to it, and he who is cognisant of the offence, and does not bring the offender to trial, shall be amenable to the same laws as the younger offender himself, and shall pay a heavier fine, and be incapable of ever commanding the young. The guardians of the law are to be careful inspectors of these matters, and shall either prevent or punish offenders. Every man should remember the universal rule, that he who is not a good servant will not be a good master; a man should pride himself more upon serving well than upon commanding well: first upon serving the laws, which is also the service of the Gods; in the second place, upon having served ancient and honourable men in the days of his youth. Furthermore, during the two years in which any one is a warden of the country, his daily food ought to be of a simple and humble kind. When the twelve have been chosen, let them and the five meet together, and determine that they will be their own servants, and, like servants, will not have other slaves and servants for their own use, neither will they use those of the villagers and husbandmen for their private advantage, but for the public service only; and in general they should make up their minds to live independently by themselves, servants of each other and of themselves. Further, at all seasons of the year, summer and winter alike, let them be under arms and survey minutely the whole country; thus they will at once keep guard, and at the same time acquire a perfect knowledge of every locality. There can be no more important kind of information than the exact knowledge of a man's own country; and for this as well as for more general reasons of pleasure and advantage, hunting with dogs and other kinds of sports should be pursued by the young. The service to whom this is committed may be called the secret police or wardens of the country; the name does not much signify, but every one who has the safety of the state at heart will use his utmost diligence in this service.

\par  After the wardens of the country, we have to speak of the election of wardens of the agora and of the city. The wardens of the country were sixty in number, and the wardens of the city will be three, and will divide the twelve parts of the city into three; like the former, they shall have care of the ways, and of the different high roads which lead out of the country into the city, and of the buildings, that they may be all made according to law;—also of the waters, which the guardians of the supply preserve and convey to them, care being taken that they may reach the fountains pure and abundant, and be both an ornament and a benefit to the city. These also should be men of influence, and at leisure to take care of the public interest. Let every man propose as warden of the city any one whom he likes out of the highest class, and when the vote has been given on them, and the number is reduced to the six who have the greatest number of votes, let the electing officers choose by lot three out of the six, and when they have undergone a scrutiny let them hold office according to the laws laid down for them. Next, let the wardens of the agora be elected in like manner, out of the first and second class, five in number: ten are to be first elected, and out of the ten five are to be chosen by lot, as in the election of the wardens of the city:—these when they have undergone a scrutiny are to be declared magistrates. Every one shall vote for every one, and he who will not vote, if he be informed against before the magistrates, shall be fined fifty drachmae, and shall also be deemed a bad citizen. Let any one who likes go to the assembly and to the general council; it shall be compulsory to go on citizens of the first and second class, and they shall pay a fine of ten drachmae if they be found not answering to their names at the assembly. But the third and fourth class shall be under no compulsion, and shall be let off without a fine, unless the magistrates have commanded all to be present, in consequence of some urgent necessity. The wardens of the agora shall observe the order appointed by law for the agora, and shall have the charge of the temples and fountains which are in the agora; and they shall see that no one injures anything, and punish him who does, with stripes and bonds, if he be a slave or stranger; but if he be a citizen who misbehaves in this way, they shall have the power themselves of inflicting a fine upon him to the amount of a hundred drachmae, or with the consent of the wardens of the city up to double that amount. And let the wardens of the city have a similar power of imposing punishments and fines in their own department; and let them impose fines by their own department; and let them impose fines by their own authority, up to a mina, or up to two minae with the consent of the wardens of the agora.

\par  In the next place, it will be proper to appoint directors of music and gymnastic, two kinds of each—of the one kind the business will be education, of the other, the superintendence of contests. In speaking of education, the law means to speak of those who have the care of order and instruction in gymnasia and schools, and of the going to school, and of school buildings for boys and girls; and in speaking of contests, the law refers to the judges of gymnastics and of music; these again are divided into two classes, the one having to do with music, the other with gymnastics; and the same who judge of the gymnastic contests of men, shall judge of horses; but in music there shall be one set of judges of solo singing, and of imitation—I mean of rhapsodists, players on the harp, the flute and the like, and another who shall judge of choral song. First of all, we must choose directors for the choruses of boys, and men, and maidens, whom they shall follow in the amusement of the dance, and for our other musical arrangements;—one director will be enough for the choruses, and he should be not less than forty years of age. One director will also be enough to introduce the solo singers, and to give judgment on the competitors, and he ought not to be less than thirty years of age. The director and manager of the choruses shall be elected after the following manner:—Let any persons who commonly take an interest in such matters go to the meeting, and be fined if they do not go (the guardians of the law shall judge of their fault), but those who have no interest shall not be compelled. The elector shall propose as director some one who understands music, and he in the scrutiny may be challenged on the one part by those who say he has no skill, and defended on the other hand by those who say that he has. Ten are to be elected by vote, and he of the ten who is chosen by lot shall undergo a scrutiny, and lead the choruses for a year according to law. And in like manner the competitor who wins the lot shall be leader of the solo and concert music for that year; and he who is thus elected shall deliver the award to the judges. In the next place, we have to choose judges in the contests of horses and of men; these shall be selected from the third and also from the second class of citizens, and three first classes shall be compelled to go to the election, but the lowest may stay away with impunity; and let there be three elected by lot out of the twenty who have been chosen previously, and they must also have the vote and approval of the examiners. But if any one is rejected in the scrutiny at any ballot or decision, others shall be chosen in the same manner, and undergo a similar scrutiny.

\par  There remains the minister of the education of youth, male and female; he too will rule according to law; one such minister will be sufficient, and he must be fifty years old, and have children lawfully begotten, both boys and girls by preference, at any rate, one or the other. He who is elected, and he who is the elector, should consider that of all the great offices of state this is the greatest; for the first shoot of any plant, if it makes a good start towards the attainment of its natural excellence, has the greatest effect on its maturity; and this is not only true of plants, but of animals wild and tame, and also of men. Man, as we say, is a tame or civilized animal; nevertheless, he requires proper instruction and a fortunate nature, and then of all animals he becomes the most divine and most civilized (Arist. Pol. ); but if he be insufficiently or ill educated he is the most savage of earthly creatures. Wherefore the legislator ought not to allow the education of children to become a secondary or accidental matter. In the first place, he who would be rightly provident about them, should begin by taking care that he is elected, who of all the citizens is in every way best; him the legislator shall do his utmost to appoint guardian and superintendent. To this end all the magistrates, with the exception of the council and prytanes, shall go to the temple of Apollo, and elect by ballot him of the guardians of the law whom they severally think will be the best superintendent of education. And he who has the greatest number of votes, after he has undergone a scrutiny at the hands of all the magistrates who have been his electors, with the exception of the guardians of the law,—shall hold office for five years; and in the sixth year let another be chosen in like manner to fill his office.

\par  If any one dies while he is holding a public office, and more than thirty days before his term of office expires, let those whose business it is elect another to the office in the same manner as before. And if any one who is entrusted with orphans dies, let the relations both on the father's and mother's side, who are residing at home, including cousins, appoint another guardian within ten days, or be fined a drachma a day for neglect to do so.

\par  A city which has no regular courts of law ceases to be a city; and again, if a judge is silent and says no more in preliminary proceedings than the litigants, as is the case in arbitrations, he will never be able to decide justly; wherefore a multitude of judges will not easily judge well, nor a few if they are bad. The point in dispute between the parties should be made clear; and time, and deliberation, and repeated examination, greatly tend to clear up doubts. For this reason, he who goes to law with another, should go first of all to his neighbours and friends who know best the questions at issue. And if he be unable to obtain from them a satisfactory decision, let him have recourse to another court; and if the two courts cannot settle the matter, let a third put an end to the suit.

\par  Now the establishment of courts of justice may be regarded as a choice of magistrates, for every magistrate must also be a judge of some things; and the judge, though he be not a magistrate, yet in certain respects is a very important magistrate on the day on which he is determining a suit. Regarding then the judges also as magistrates, let us say who are fit to be judges, and of what they are to be judges, and how many of them are to judge in each suit. Let that be the supreme tribunal which the litigants appoint in common for themselves, choosing certain persons by agreement. And let there be two other tribunals: one for private causes, when a citizen accuses another of wronging him and wishes to get a decision; the other for public causes, in which some citizen is of opinion that the public has been wronged by an individual, and is willing to vindicate the common interests. And we must not forget to mention how the judges are to be qualified, and who they are to be. In the first place, let there be a tribunal open to all private persons who are trying causes one against another for the third time, and let this be composed as follows:—All the officers of state, as well annual as those holding office for a longer period, when the new year is about to commence, in the month following after the summer solstice, on the last day but one of the year, shall meet in some temple, and calling God to witness, shall dedicate one judge from every magistracy to be their first-fruits, choosing in each office him who seems to them to be the best, and whom they deem likely to decide the causes of his fellow-citizens during the ensuing year in the best and holiest manner. And when the election is completed, a scrutiny shall be held in the presence of the electors themselves, and if any one be rejected another shall be chosen in the same manner. Those who have undergone the scrutiny shall judge the causes of those who have declined the inferior courts, and shall give their vote openly. The councillors and other magistrates who have elected them shall be required to be hearers and spectators of the causes; and any one else may be present who pleases. If one man charges another with having intentionally decided wrong, let him go to the guardians of the law and lay his accusation before them, and he who is found guilty in such a case shall pay damages to the injured party equal to half the injury; but if he shall appear to deserve a greater penalty, the judges shall determine what additional punishment he shall suffer, and how much more he ought to pay to the public treasury, and to the party who brought the suit.

\par  In the judgment of offences against the state, the people ought to participate, for when any one wrongs the state all are wronged, and may reasonably complain if they are not allowed to share in the decision. Such causes ought to originate with the people, and the ought also to have the final decision of them, but the trial of them shall take place before three of the highest magistrates, upon whom the plaintiff and the defendant shall agree; and if they are not able to come to an agreement themselves, the council shall choose one of the two proposed. And in private suits, too, as far as is possible, all should have a share; for he who has no share in the administration of justice, is apt to imagine that he has no share in the state at all. And for this reason there shall be a court of law in every tribe, and the judges shall be chosen by lot;—they shall give their decisions at once, and shall be inaccessible to entreaties. The final judgment shall rest with that court which, as we maintain, has been established in the most incorruptible form of which human things admit: this shall be the court established for those who are unable to get rid of their suits either in the courts of neighbours or of the tribes.

\par  Thus much of the courts of law, which, as I was saying, cannot be precisely defined either as being or not being offices; a superficial sketch has been given of them, in which some things have been told and others omitted. For the right place of an exact statement of the laws respecting suits, under their several heads, will be at the end of the body of legislation;—let us then expect them at the end. Hitherto our legislation has been chiefly occupied with the appointment of offices. Perfect unity and exactness, extending to the whole and every particular of political administration, cannot be attained to the full, until the discussion shall have a beginning, middle, and end, and is complete in every part. At present we have reached the election of magistrates, and this may be regarded as a sufficient termination of what preceded. And now there need no longer be any delay or hesitation in beginning the work of legislation.

\par \textbf{CLEINIAS}
\par   I like what you have said, Stranger; and I particularly like your manner of tacking on the beginning of your new discourse to the end of the former one.

\par \textbf{ATHENIAN}
\par   Thus far, then, the old men's rational pastime has gone off well.

\par \textbf{CLEINIAS}
\par   You mean, I suppose, their serious and noble pursuit?

\par \textbf{ATHENIAN}
\par   Perhaps; but I should like to know whether you and I are agreed about a certain thing.

\par \textbf{CLEINIAS}
\par   About what thing?

\par \textbf{ATHENIAN}
\par   You know the endless labour which painters expend upon their pictures—they are always putting in or taking out colours, or whatever be the term which artists employ; they seem as if they would never cease touching up their works, which are always being made brighter and more beautiful.

\par \textbf{CLEINIAS}
\par   I know something of these matters from report, although I have never had any great acquaintance with the art.

\par \textbf{ATHENIAN}
\par   No matter; we may make use of the illustration notwithstanding: —Suppose that some one had a mind to paint a figure in the most beautiful manner, in the hope that his work instead of losing would always improve as time went on—do you not see that being a mortal, unless he leaves some one to succeed him who will correct the flaws which time may introduce, and be able to add what is left imperfect through the defect of the artist, and who will further brighten up and improve the picture, all his great labour will last but a short time?

\par \textbf{CLEINIAS}
\par   True.

\par \textbf{ATHENIAN}
\par   And is not the aim of the legislator similar? First, he desires that his laws should be written down with all possible exactness; in the second place, as time goes on and he has made an actual trial of his decrees, will he not find omissions? Do you imagine that there ever was a legislator so foolish as not to know that many things are necessarily omitted, which some one coming after him must correct, if the constitution and the order of government is not to deteriorate, but to improve in the state which he has established?

\par \textbf{CLEINIAS}
\par   Assuredly, that is the sort of thing which every one would desire.

\par \textbf{ATHENIAN}
\par   And if any one possesses any means of accomplishing this by word or deed, or has any way great or small by which he can teach a person to understand how he can maintain and amend the laws, he should finish what he has to say, and not leave the work incomplete.

\par \textbf{CLEINIAS}
\par   By all means.

\par \textbf{ATHENIAN}
\par   And is not this what you and I have to do at the present moment?

\par \textbf{CLEINIAS}
\par   What have we to do?

\par \textbf{ATHENIAN}
\par   As we are about to legislate and have chosen our guardians of the law, and are ourselves in the evening of life, and they as compared with us are young men, we ought not only to legislate for them, but to endeavour to make them not only guardians of the law but legislators themselves, as far as this is possible.

\par \textbf{CLEINIAS}
\par   Certainly; if we can.

\par \textbf{ATHENIAN}
\par   At any rate, we must do our best.

\par \textbf{CLEINIAS}
\par   Of course.

\par \textbf{ATHENIAN}
\par   We will say to them—O friends and saviours of our laws, in laying down any law, there are many particulars which we shall omit, and this cannot be helped; at the same time, we will do our utmost to describe what is important, and will give an outline which you shall fill up. And I will explain on what principle you are to act. Megillus and Cleinias and I have often spoken to one another touching these matters, and we are of opinion that we have spoken well. And we hope that you will be of the same mind with us, and become our disciples, and keep in view the things which in our united opinion the legislator and guardian of the law ought to keep in view. There was one main point about which we were agreed—that a man's whole energies throughout life should be devoted to the acquisition of the virtue proper to a man, whether this was to be gained by study, or habit, or some mode of acquisition, or desire, or opinion, or knowledge—and this applies equally to men and women, old and young—the aim of all should always be such as I have described; anything which may be an impediment, the good man ought to show that he utterly disregards. And if at last necessity plainly compels him to be an outlaw from his native land, rather than bow his neck to the yoke of slavery and be ruled by inferiors, and he has to fly, an exile he must be and endure all such trials, rather than accept another form of government, which is likely to make men worse. These are our original principles; and do you now, fixing your eyes upon the standard of what a man and a citizen ought or ought not to be, praise and blame the laws—blame those which have not this power of making the citizen better, but embrace those which have; and with gladness receive and live in them; bidding a long farewell to other institutions which aim at goods, as they are termed, of a different kind.

\par  Let us proceed to another class of laws, beginning with their foundation in religion. And we must first return to the number 5040—the entire number had, and has, a great many convenient divisions, and the number of the tribes which was a twelfth part of the whole, being correctly formed by 21 x 20 (5040/(21 x 20), i.e., 5040/420 = 12), also has them. And not only is the whole number divisible by twelve, but also the number of each tribe is divisible by twelve. Now every portion should be regarded by us as a sacred gift of Heaven, corresponding to the months and to the revolution of the universe (compare Tim.). Every city has a guiding and sacred principle given by nature, but in some the division or distribution has been more right than in others, and has been more sacred and fortunate. In our opinion, nothing can be more right than the selection of the number 5040, which may be divided by all numbers from one to twelve with the single exception of eleven, and that admits of a very easy correction; for if, turning to the dividend (5040), we deduct two families, the defect in the division is cured. And the truth of this may be easily proved when we have leisure. But for the present, trusting to the mere assertion of this principle, let us divide the state; and assigning to each portion some God or son of a God, let us give them altars and sacred rites, and at the altars let us hold assemblies for sacrifice twice in the month—twelve assemblies for the tribes, and twelve for the city, according to their divisions; the first in honour of the Gods and divine things, and the second to promote friendship and 'better acquaintance,' as the phrase is, and every sort of good fellowship with one another. For people must be acquainted with those into whose families and whom they marry and with those to whom they give in marriage; in such matters, as far as possible, a man should deem it all important to avoid a mistake, and with this serious purpose let games be instituted (compare Republic) in which youths and maidens shall dance together, seeing one another and being seen naked, at a proper age, and on a suitable occasion, not transgressing the rules of modesty.

\par  The directors of choruses will be the superintendents and regulators of these games, and they, together with the guardians of the law, will legislate in any matters which we have omitted; for, as we said, where there are numerous and minute details, the legislator must leave out something. And the annual officers who have experience, and know what is wanted, must make arrangements and improvements year by year, until such enactments and provisions are sufficiently determined. A ten years' experience of sacrifices and dances, if extending to all particulars, will be quite sufficient; and if the legislator be alive they shall communicate with him, but if he be dead then the several officers shall refer the omissions which come under their notice to the guardians of the law, and correct them, until all is perfect; and from that time there shall be no more change, and they shall establish and use the new laws with the others which the legislator originally gave them, and of which they are never, if they can help, to change aught; or, if some necessity overtakes them, the magistrates must be called into counsel, and the whole people, and they must go to all the oracles of the Gods; and if they are all agreed, in that case they may make the change, but if they are not agreed, by no manner of means, and any one who dissents shall prevail, as the law ordains.

\par  Whenever any one over twenty-five years of age, having seen and been seen by others, believes himself to have found a marriage connexion which is to his mind, and suitable for the procreation of children, let him marry if he be still under the age of five-and-thirty years; but let him first hear how he ought to seek after what is suitable and appropriate (compare Arist. Pol.). For, as Cleinias says, every law should have a suitable prelude.

\par \textbf{CLEINIAS}
\par   You recollect at the right moment, Stranger, and do not miss the opportunity which the argument affords of saying a word in season.

\par \textbf{ATHENIAN}
\par   I thank you. We will say to him who is born of good parents—O my son, you ought to make such a marriage as wise men would approve. Now they would advise you neither to avoid a poor marriage, nor specially to desire a rich one; but if other things are equal, always to honour inferiors, and with them to form connexions;—this will be for the benefit of the city and of the families which are united; for the equable and symmetrical tends infinitely more to virtue than the unmixed. And he who is conscious of being too headstrong, and carried away more than is fitting in all his actions, ought to desire to become the relation of orderly parents; and he who is of the opposite temper ought to seek the opposite alliance. Let there be one word concerning all marriages: —Every man shall follow, not after the marriage which is most pleasing to himself, but after that which is most beneficial to the state. For somehow every one is by nature prone to that which is likest to himself, and in this way the whole city becomes unequal in property and in disposition; and hence there arise in most states the very results which we least desire to happen. Now, to add to the law an express provision, not only that the rich man shall not marry into the rich family, nor the powerful into the family of the powerful, but that the slower natures shall be compelled to enter into marriage with the quicker, and the quicker with the slower, may awaken anger as well as laughter in the minds of many; for there is a difficulty in perceiving that the city ought to be well mingled like a cup, in which the maddening wine is hot and fiery, but when chastened by a soberer God, receives a fair associate and becomes an excellent and temperate drink (compare Statesman). Yet in marriage no one is able to see that the same result occurs. Wherefore also the law must let alone such matters, but we should try to charm the spirits of men into believing the equability of their children's disposition to be of more importance than equality in excessive fortune when they marry; and him who is too desirous of making a rich marriage we should endeavour to turn aside by reproaches, not, however, by any compulsion of written law.

\par  Let this then be our exhortation concerning marriage, and let us remember what was said before—that a man should cling to immortality, and leave behind him children's children to be the servants of God in his place for ever. All this and much more may be truly said by way of prelude about the duty of marriage. But if a man will not listen, and remains unsocial and alien among his fellow-citizens, and is still unmarried at thirty-five years of age, let him pay a yearly fine;—he who of the highest class shall pay a fine of a hundred drachmae, and he who is of the second class a fine of seventy drachmae; the third class shall pay sixty drachmae, and the fourth thirty drachmae, and let the money be sacred to Here; he who does not pay the fine annually shall owe ten times the sum, which the treasurer of the goddess shall exact; and if he fails in doing so, let him be answerable and give an account of the money at his audit. He who refuses to marry shall be thus punished in money, and also be deprived of all honour which the younger show to the elder; let no young man voluntarily obey him, and, if he attempt to punish any one, let every one come to the rescue and defend the injured person, and he who is present and does not come to the rescue, shall be pronounced by the law to be a coward and a bad citizen. Of the marriage portion I have already spoken; and again I say for the instruction of poor men that he who neither gives nor receives a dowry on account of poverty, has a compensation; for the citizens of our state are provided with the necessaries of life, and wives will be less likely to be insolent, and husbands to be mean and subservient to them on account of property. And he who obeys this law will do a noble action; but he who will not obey, and gives or receives more than fifty drachmae as the price of the marriage garments if he be of the lowest, or more than a mina, or a mina-and-a-half, if he be of the third or second classes, or two minae if he be of the highest class, shall owe to the public treasury a similar sum, and that which is given or received shall be sacred to Here and Zeus; and let the treasurers of these Gods exact the money, as was said before about the unmarried—that the treasurers of Here were to exact the money, or pay the fine themselves.

\par  The betrothal by a father shall be valid in the first degree, that by a grandfather in the second degree, and in the third degree, betrothal by brothers who have the same father; but if there are none of these alive, the betrothal by a mother shall be valid in like manner; in cases of unexampled fatality, the next of kin and the guardians shall have authority. What are to be the rites before marriages, or any other sacred acts, relating either to future, present, or past marriages, shall be referred to the interpreters; and he who follows their advice may be satisfied. Touching the marriage festival, they shall assemble not more than five male and five female friends of both families; and a like number of members of the family of either sex, and no man shall spend more than his means will allow; he who is of the richest class may spend a mina,—he who is of the second, half a mina, and in the same proportion as the census of each decreases: all men shall praise him who is obedient to the law; but he who is disobedient shall be punished by the guardians of the law as a man wanting in true taste, and uninstructed in the laws of bridal song. Drunkenness is always improper, except at the festivals of the God who gave wine; and peculiarly dangerous, when a man is engaged in the business of marriage; at such a crisis of their lives a bride and bridegroom ought to have all their wits about them—they ought to take care that their offspring may be born of reasonable beings; for on what day or night Heaven will give them increase, who can say? Moreover, they ought not to begetting children when their bodies are dissipated by intoxication, but their offspring should be compact and solid, quiet and compounded properly; whereas the drunkard is all abroad in all his actions, and beside himself both in body and soul. Wherefore, also, the drunken man is bad and unsteady in sowing the seed of increase, and is likely to beget offspring who will be unstable and untrustworthy, and cannot be expected to walk straight either in body or mind. Hence during the whole year and all his life long, and especially while he is begetting children, he ought to take care and not intentionally do what is injurious to health, or what involves insolence and wrong; for he cannot help leaving the impression of himself on the souls and bodies of his offspring, and he begets children in every way inferior. And especially on the day and night of marriage should a man abstain from such things. For the beginning, which is also a God dwelling in man, preserves all things, if it meet with proper respect from each individual. He who marries is further to consider, that one of the two houses in the lot is the nest and nursery of his young, and there he is to marry and make a home for himself and bring up his children, going away from his father and mother. For in friendships there must be some degree of desire, in order to cement and bind together diversities of character; but excessive intercourse not having the desire which is created by time, insensibly dissolves friendships from a feeling of satiety; wherefore a man and his wife shall leave to his and her father and mother their own dwelling-places, and themselves go as to a colony and dwell there, and visit and be visited by their parents; and they shall beget and bring up children, handing on the torch of life from one generation to another, and worshipping the Gods according to law for ever.

\par  In the next place, we have to consider what sort of property will be most convenient. There is no difficulty either in understanding or acquiring most kinds of property, but there is great difficulty in what relates to slaves. And the reason is, that we speak about them in a way which is right and which is not right; for what we say about our slaves is consistent and also inconsistent with our practice about them.

\par \textbf{MEGILLUS}
\par   I do not understand, Stranger, what you mean.

\par \textbf{ATHENIAN}
\par   I am not surprised, Megillus, for the state of the Helots among the Lacedaemonians is of all Hellenic forms of slavery the most controverted and disputed about, some approving and some condemning it; there is less dispute about the slavery which exists among the Heracleots, who have subjugated the Mariandynians, and about the Thessalian Penestae. Looking at these and the like examples, what ought we to do concerning property in slaves? I made a remark, in passing, which naturally elicited a question about my meaning from you. It was this: —We know that all would agree that we should have the best and most attached slaves whom we can get. For many a man has found his slaves better in every way than brethren or sons, and many times they have saved the lives and property of their masters and their whole house—such tales are well known.

\par \textbf{MEGILLUS}
\par   To be sure.

\par \textbf{ATHENIAN}
\par   But may we not also say that the soul of the slave is utterly corrupt, and that no man of sense ought to trust them? And the wisest of our poets, speaking of Zeus, says:

\par  'Far-seeing Zeus takes away half the understanding of men whom the day of slavery subdues.'

\par  Different persons have got these two different notions of slaves in their minds—some of them utterly distrust their servants, and, as if they were wild beasts, chastise them with goads and whips, and make their souls three times, or rather many times, as slavish as they were before;—and others do just the opposite.

\par \textbf{MEGILLUS}
\par   True.

\par \textbf{CLEINIAS}
\par   Then what are we to do in our own country, Stranger, seeing that there are such differences in the treatment of slaves by their owners?

\par \textbf{ATHENIAN}
\par   Well, Cleinias, there can be no doubt that man is a troublesome animal, and therefore he is not very manageable, nor likely to become so, when you attempt to introduce the necessary division of slave, and freeman, and master.

\par \textbf{CLEINIAS}
\par   That is obvious.

\par \textbf{ATHENIAN}
\par   He is a troublesome piece of goods, as has been often shown by the frequent revolts of the Messenians, and the great mischiefs which happen in states having many slaves who speak the same language, and the numerous robberies and lawless life of the Italian banditti, as they are called. A man who considers all this is fairly at a loss. Two remedies alone remain to us,—not to have the slaves of the same country, nor if possible, speaking the same language (compare Aris. Pol. ); in this way they will more easily be held in subjection:  secondly, we should tend them carefully, not only out of regard to them, but yet more out of respect to ourselves. And the right treatment of slaves is to behave properly to them, and to do to them, if possible, even more justice than to those who are our equals; for he who naturally and genuinely reverences justice, and hates injustice, is discovered in his dealings with any class of men to whom he can easily be unjust. And he who in regard to the natures and actions of his slaves is undefiled by impiety and injustice, will best sow the seeds of virtue in them; and this may be truly said of every master, and tyrant, and of every other having authority in relation to his inferiors. Slaves ought to be punished as they deserve, and not admonished as if they were freemen, which will only make them conceited. The language used to a servant ought always to be that of a command (compare Arist. Pol. ), and we ought not to jest with them, whether they are males or females—this is a foolish way which many people have of setting up their slaves, and making the life of servitude more disagreeable both for them and for their masters.

\par \textbf{CLEINIAS}
\par   True.

\par \textbf{ATHENIAN}
\par   Now that each of the citizens is provided, as far as possible, with a sufficient number of suitable slaves who can help him in what he has to do, we may next proceed to describe their dwellings.

\par \textbf{CLEINIAS}
\par   Very good.

\par \textbf{ATHENIAN}
\par   The city being new and hitherto uninhabited, care ought to be taken of all the buildings, and the manner of building each of them, and also of the temples and walls. These, Cleinias, were matters which properly came before the marriages;—but, as we are only talking, there is no objection to changing the order. If, however, our plan of legislation is ever to take effect, then the house shall precede the marriage if God so will, and afterwards we will come to the regulations about marriage; but at present we are only describing these matters in a general outline.

\par \textbf{CLEINIAS}
\par   Quite true.

\par \textbf{ATHENIAN}
\par   The temples are to be placed all round the agora, and the whole city built on the heights in a circle (compare Arist. Pol. ), for the sake of defence and for the sake of purity. Near the temples are to be placed buildings for the magistrates and the courts of law; in these plaintiff and defendant will receive their due, and the places will be regarded as most holy, partly because they have to do with holy things:  and partly because they are the dwelling-places of holy Gods:  and in them will be held the courts in which cases of homicide and other trials of capital offences may fitly take place. As to the walls, Megillus, I agree with Sparta in thinking that they should be allowed to sleep in the earth, and that we should not attempt to disinter them (compare Arist. Pol. ); there is a poetical saying, which is finely expressed, that 'walls ought to be of steel and iron, and not of earth;' besides, how ridiculous of us to be sending out our young men annually into the country to dig and to trench, and to keep off the enemy by fortifications, under the idea that they are not to be allowed to set foot in our territory, and then, that we should surround ourselves with a wall, which, in the first place, is by no means conducive to the health of cities, and is also apt to produce a certain effeminacy in the minds of the inhabitants, inviting men to run thither instead of repelling their enemies, and leading them to imagine that their safety is due not to their keeping guard day and night, but that when they are protected by walls and gates, then they may sleep in safety; as if they were not meant to labour, and did not know that true repose comes from labour, and that disgraceful indolence and a careless temper of mind is only the renewal of trouble. But if men must have walls, the private houses ought to be so arranged from the first that the whole city may be one wall, having all the houses capable of defence by reason of their uniformity and equality towards the streets (compare Arist. Pol.). The form of the city being that of a single dwelling will have an agreeable aspect, and being easily guarded will be infinitely better for security. Until the original building is completed, these should be the principal objects of the inhabitants; and the wardens of the city should superintend the work, and should impose a fine on him who is negligent; and in all that relates to the city they should have a care of cleanliness, and not allow a private person to encroach upon any public property either by buildings or excavations. Further, they ought to take care that the rains from heaven flow off easily, and of any other matters which may have to be administered either within or without the city. The guardians of the law shall pass any further enactments which their experience may show to be necessary, and supply any other points in which the law may be deficient. And now that these matters, and the buildings about the agora, and the gymnasia, and places of instruction, and theatres, are all ready and waiting for scholars and spectators, let us proceed to the subjects which follow marriage in the order of legislation.

\par \textbf{CLEINIAS}
\par   By all means.

\par \textbf{ATHENIAN}
\par   Assuming that marriages exist already, Cleinias, the mode of life during the year after marriage, before children are born, will follow next in order. In what way bride and bridegroom ought to live in a city which is to be superior to other cities, is a matter not at all easy for us to determine. There have been many difficulties already, but this will be the greatest of them, and the most disagreeable to the many. Still I cannot but say what appears to me to be right and true, Cleinias.

\par \textbf{CLEINIAS}
\par   Certainly.

\par \textbf{ATHENIAN}
\par   He who imagines that he can give laws for the public conduct of states, while he leaves the private life of citizens wholly to take care of itself; who thinks that individuals may pass the day as they please, and that there is no necessity of order in all things; he, I say, who gives up the control of their private lives, and supposes that they will conform to law in their common and public life, is making a great mistake. Why have I made this remark? Why, because I am going to enact that the bridegrooms should live at the common tables, just as they did before marriage. This was a singularity when first enacted by the legislator in your parts of the world, Megillus and Cleinias, as I should suppose, on the occasion of some war or other similar danger, which caused the passing of the law, and which would be likely to occur in thinly-peopled places, and in times of pressure. But when men had once tried and been accustomed to a common table, experience showed that the institution greatly conduced to security; and in some such manner the custom of having common tables arose among you.

\par \textbf{CLEINIAS}
\par   Likely enough.

\par \textbf{ATHENIAN}
\par   I said that there may have been singularity and danger in imposing such a custom at first, but that now there is not the same difficulty. There is, however, another institution which is the natural sequel to this, and would be excellent, if it existed anywhere, but at present it does not. The institution of which I am about to speak is not easily described or executed; and would be like the legislator 'combing wool into the fire,' as people say, or performing any other impossible and useless feat.

\par \textbf{CLEINIAS}
\par   What is the cause, Stranger, of this extreme hesitation?

\par \textbf{ATHENIAN}
\par   You shall hear without any fruitless loss of time. That which has law and order in a state is the cause of every good, but that which is disordered or ill-ordered is often the ruin of that which is well-ordered; and at this point the argument is now waiting. For with you, Cleinias and Megillus, the common tables of men are, as I said, a heaven-born and admirable institution, but you are mistaken in leaving the women unregulated by law. They have no similar institution of public tables in the light of day, and just that part of the human race which is by nature prone to secrecy and stealth on account of their weakness—I mean the female sex—has been left without regulation by the legislator, which is a great mistake. And, in consequence of this neglect, many things have grown lax among you, which might have been far better, if they had been only regulated by law; for the neglect of regulations about women may not only be regarded as a neglect of half the entire matter (Arist. Pol. ), but in proportion as woman's nature is inferior to that of men in capacity for virtue, in that degree the consequence of such neglect is more than twice as important. The careful consideration of this matter, and the arranging and ordering on a common principle of all our institutions relating both to men and women, greatly conduces to the happiness of the state. But at present, such is the unfortunate condition of mankind, that no man of sense will even venture to speak of common tables in places and cities in which they have never been established at all; and how can any one avoid being utterly ridiculous, who attempts to compel women to show in public how much they eat and drink? There is nothing at which the sex is more likely to take offence. For women are accustomed to creep into dark places, and when dragged out into the light they will exert their utmost powers of resistance, and be far too much for the legislator. And therefore, as I said before, in most places they will not endure to have the truth spoken without raising a tremendous outcry, but in this state perhaps they may. And if we may assume that our whole discussion about the state has not been mere idle talk, I should like to prove to you, if you will consent to listen, that this institution is good and proper; but if you had rather not, I will refrain.

\par \textbf{CLEINIAS}
\par   There is nothing which we should both of us like better, Stranger, than to hear what you have to say.

\par \textbf{ATHENIAN}
\par   Very good; and you must not be surprised if I go back a little, for we have plenty of leisure, and there is nothing to prevent us from considering in every point of view the subject of law.

\par \textbf{CLEINIAS}
\par   True.

\par \textbf{ATHENIAN}
\par   Then let us return once more to what we were saying at first. Every man should understand that the human race either had no beginning at all, and will never have an end, but always will be and has been; or that it began an immense while ago.

\par \textbf{CLEINIAS}
\par   Certainly.

\par \textbf{ATHENIAN}
\par   Well, and have there not been constitutions and destructions of states, and all sorts of pursuits both orderly and disorderly, and diverse desires of meats and drinks always, and in all the world, and all sorts of changes of the seasons in which animals may be expected to have undergone innumerable transformations of themselves?

\par \textbf{CLEINIAS}
\par   No doubt.

\par \textbf{ATHENIAN}
\par   And may we not suppose that vines appeared, which had previously no existence, and also olives, and the gifts of Demeter and her daughter, of which one Triptolemus was the minister, and that, before these existed, animals took to devouring each other as they do still?

\par \textbf{CLEINIAS}
\par   True.

\par \textbf{ATHENIAN}
\par   Again, the practice of men sacrificing one another still exists among many nations; while, on the other hand, we hear of other human beings who did not even venture to taste the flesh of a cow and had no animal sacrifices, but only cakes and fruits dipped in honey, and similar pure offerings, but no flesh of animals; from these they abstained under the idea that they ought not to eat them, and might not stain the altars of the Gods with blood. For in those days men are said to have lived a sort of Orphic life, having the use of all lifeless things, but abstaining from all living things.

\par \textbf{CLEINIAS}
\par   Such has been the constant tradition, and is very likely true.

\par \textbf{ATHENIAN}
\par   Some one might say to us, What is the drift of all this?

\par \textbf{CLEINIAS}
\par   A very pertinent question, Stranger.

\par \textbf{ATHENIAN}
\par   And therefore I will endeavour, Cleinias, if I can, to draw the natural inference.

\par \textbf{CLEINIAS}
\par   Proceed.

\par \textbf{ATHENIAN}
\par   I see that among men all things depend upon three wants and desires, of which the end is virtue, if they are rightly led by them, or the opposite if wrongly. Now these are eating and drinking, which begin at birth—every animal has a natural desire for them, and is violently excited, and rebels against him who says that he must not satisfy all his pleasures and appetites, and get rid of all the corresponding pains—and the third and greatest and sharpest want and desire breaks out last, and is the fire of sexual lust, which kindles in men every species of wantonness and madness. And these three disorders we must endeavour to master by the three great principles of fear and law and right reason; turning them away from that which is called pleasantest to the best, using the Muses and the Gods who preside over contests to extinguish their increase and influx.

\par  But to return:—After marriage let us speak of the birth of children, and after their birth of their nurture and education. In the course of discussion the several laws will be perfected, and we shall at last arrive at the common tables. Whether such associations are to be confined to men, or extended to women also, we shall see better when we approach and take a nearer view of them; and we may then determine what previous institutions are required and will have to precede them. As I said before, we shall see them more in detail, and shall be better able to lay down the laws which are proper or suited to them.

\par \textbf{CLEINIAS}
\par   Very true.

\par \textbf{ATHENIAN}
\par   Let us keep in mind the words which have now been spoken; for hereafter there may be need of them.

\par \textbf{CLEINIAS}
\par   What do you bid us keep in mind?

\par \textbf{ATHENIAN}
\par   That which we comprehended under the three words—first, eating, secondly, drinking, thirdly, the excitement of love.

\par \textbf{CLEINIAS}
\par   We shall be sure to remember, Stranger.

\par \textbf{ATHENIAN}
\par   Very good. Then let us now proceed to marriage, and teach persons in what way they shall beget children, threatening them, if they disobey, with the terrors of the law.

\par \textbf{CLEINIAS}
\par   What do you mean?

\par \textbf{ATHENIAN}
\par   The bride and bridegroom should consider that they are to produce for the state the best and fairest specimens of children which they can. Now all men who are associated in any action always succeed when they attend and give their mind to what they are doing, but when they do not give their mind or have no mind, they fail; wherefore let the bridegroom give his mind to the bride and to the begetting of children, and the bride in like manner give her mind to the bridegroom, and particularly at the time when their children are not yet born. And let the women whom we have chosen be the overseers of such matters, and let them in whatever number, large or small, and at whatever time the magistrates may command, assemble every day in the temple of Eileithyia during a third part of the day, and being there assembled, let them inform one another of any one whom they see, whether man or woman, of those who are begetting children, disregarding the ordinances given at the time when the nuptial sacrifices and ceremonies were performed. Let the begetting of children and the supervision of those who are begetting them continue ten years and no longer, during the time when marriage is fruitful. But if any continue without children up to this time, let them take counsel with their kindred and with the women holding the office of overseer and be divorced for their mutual benefit. If, however, any dispute arises about what is proper and for the interest of either party, they shall choose ten of the guardians of the law and abide by their permission and appointment. The women who preside over these matters shall enter into the houses of the young, and partly by admonitions and partly by threats make them give over their folly and error:  if they persist, let the women go and tell the guardians of the law, and the guardians shall prevent them. But if they too cannot prevent them, they shall bring the matter before the people; and let them write up their names and make oath that they cannot reform such and such an one; and let him who is thus written up, if he cannot in a court of law convict those who have inscribed his name, be deprived of the privileges of a citizen in the following respects: —let him not go to weddings nor to the thanksgivings after the birth of children; and if he go, let any one who pleases strike him with impunity; and let the same regulations hold about women:  let not a woman be allowed to appear abroad, or receive honour, or go to nuptial and birthday festivals, if she in like manner be written up as acting disorderly and cannot obtain a verdict. And if, when they themselves have done begetting children according to the law, a man or woman have connexion with another man or woman who are still begetting children, let the same penalties be inflicted upon them as upon those who are still having a family; and when the time for procreation has passed let the man or woman who refrains in such matters be held in esteem, and let those who do not refrain be held in the contrary of esteem—that is to say, disesteem. Now, if the greater part of mankind behave modestly, the enactments of law may be left to slumber; but, if they are disorderly, the enactments having been passed, let them be carried into execution. To every man the first year is the beginning of life, and the time of birth ought to be written down in the temples of their fathers as the beginning of existence to every child, whether boy or girl. Let every phratria have inscribed on a whited wall the names of the successive archons by whom the years are reckoned. And near to them let the living members of the phratria be inscribed, and when they depart life let them be erased. The limit of marriageable ages for a woman shall be from sixteen to twenty years at the longest,—for a man, from thirty to thirty-five years; and let a woman hold office at forty, and a man at thirty years. Let a man go out to war from twenty to sixty years, and for a woman, if there appear any need to make use of her in military service, let the time of service be after she shall have brought forth children up to fifty years of age; and let regard be had to what is possible and suitable to each.

\par 
\section{
      BOOK VII.
    }
\par  And now, assuming children of both sexes to have been born, it will be proper for us to consider, in the next place, their nurture and education; this cannot be left altogether unnoticed, and yet may be thought a subject fitted rather for precept and admonition than for law. In private life there are many little things, not always apparent, arising out of the pleasures and pains and desires of individuals, which run counter to the intention of the legislator, and make the characters of the citizens various and dissimilar:—this is an evil in states; for by reason of their smallness and frequent occurrence, there would be an unseemliness and want of propriety in making them penal by law; and if made penal, they are the destruction of the written law because mankind get the habit of frequently transgressing the law in small matters. The result is that you cannot legislate about them, and still less can you be silent. I speak somewhat darkly, but I shall endeavour also to bring my wares into the light of day, for I acknowledge that at present there is a want of clearness in what I am saying.

\par \textbf{CLEINIAS}
\par   Very true.

\par  ATHENIAN. Am I not right in maintaining that a good education is that which tends most to the improvement of mind and body?

\par \textbf{CLEINIAS}
\par   Undoubtedly.

\par \textbf{ATHENIAN}
\par   And nothing can be plainer than that the fairest bodies are those which grow up from infancy in the best and straightest manner?

\par \textbf{CLEINIAS}
\par   Certainly.

\par \textbf{ATHENIAN}
\par   And do we not further observe that the first shoot of every living thing is by far the greatest and fullest? Many will even contend that a man at twenty-five does not reach twice the height which he attained at five.

\par \textbf{CLEINIAS}
\par   True.

\par \textbf{ATHENIAN}
\par   Well, and is not rapid growth without proper and abundant exercise the source endless evils in the body?

\par \textbf{CLEINIAS}
\par   Yes.

\par \textbf{ATHENIAN}
\par   And the body should have the most exercise when it receives most nourishment?

\par \textbf{CLEINIAS}
\par   But, Stranger, are we to impose this great amount of exercise upon newly-born infants?

\par \textbf{ATHENIAN}
\par   Nay, rather on the bodies of infants still unborn.

\par \textbf{CLEINIAS}
\par   What do you mean, my good sir? In the process of gestation?

\par \textbf{ATHENIAN}
\par   Exactly. I am not at all surprised that you have never heard of this very peculiar sort of gymnastic applied to such little creatures, which, although strange, I will endeavour to explain to you.

\par \textbf{CLEINIAS}
\par   By all means.

\par \textbf{ATHENIAN}
\par   The practice is more easy for us to understand than for you, by reason of certain amusements which are carried to excess by us at Athens. Not only boys, but often older persons, are in the habit of keeping quails and cocks (compare Republic), which they train to fight one another. And they are far from thinking that the contests in which they stir them up to fight with one another are sufficient exercise; for, in addition to this, they carry them about tucked beneath their armpits, holding the smaller birds in their hands, the larger under their arms, and go for a walk of a great many miles for the sake of health, that is to say, not their own health, but the health of the birds; whereby they prove to any intelligent person, that all bodies are benefited by shakings and movements, when they are moved without weariness, whether the motion proceeds from themselves, or is caused by a swing, or at sea, or on horseback, or by other bodies in whatever way moving, and that thus gaining the mastery over food and drink, they are able to impart beauty and health and strength. But admitting all this, what follows? Shall we make a ridiculous law that the pregnant woman shall walk about and fashion the embryo within as we fashion wax before it hardens, and after birth swathe the infant for two years? Suppose that we compel nurses, under penalty of a legal fine, to be always carrying the children somewhere or other, either to the temples, or into the country, or to their relations' houses, until they are well able to stand, and to take care that their limbs are not distorted by leaning on them when they are too young (compare Arist. Pol. ),—they should continue to carry them until the infant has completed its third year; the nurses should be strong, and there should be more than one of them. Shall these be our rules, and shall we impose a penalty for the neglect of them? No, no; the penalty of which we were speaking will fall upon our own heads more than enough.

\par \textbf{CLEINIAS}
\par   What penalty?

\par \textbf{ATHENIAN}
\par   Ridicule, and the difficulty of getting the feminine and servant-like dispositions of the nurses to comply.

\par \textbf{CLEINIAS}
\par   Then why was there any need to speak of the matter at all?

\par \textbf{ATHENIAN}
\par   The reason is, that masters and freemen in states, when they hear of it, are very likely to arrive at a true conviction that without due regulation of private life in cities, stability in the laying down of laws is hardly to be expected (compare Republic); and he who makes this reflection may himself adopt the laws just now mentioned, and, adopting them, may order his house and state well and be happy.

\par \textbf{CLEINIAS}
\par   Likely enough.

\par \textbf{ATHENIAN}
\par   And therefore let us proceed with our legislation until we have determined the exercises which are suited to the souls of young children, in the same manner in which we have begun to go through the rules relating to their bodies.

\par \textbf{CLEINIAS}
\par   By all means.

\par \textbf{ATHENIAN}
\par   Let us assume, then, as a first principle in relation both to the body and soul of very young creatures, that nursing and moving about by day and night is good for them all, and that the younger they are, the more they will need it (compare Arist. Pol. ); infants should live, if that were possible, as if they were always rocking at sea. This is the lesson which we may gather from the experience of nurses, and likewise from the use of the remedy of motion in the rites of the Corybantes; for when mothers want their restless children to go to sleep they do not employ rest, but, on the contrary, motion—rocking them in their arms; nor do they give them silence, but they sing to them and lap them in sweet strains; and the Bacchic women are cured of their frenzy in the same manner by the use of the dance and of music.

\par \textbf{CLEINIAS}
\par   Well, Stranger, and what is the reason of this?

\par \textbf{ATHENIAN}
\par   The reason is obvious.

\par \textbf{CLEINIAS}
\par   What?

\par \textbf{ATHENIAN}
\par   The affection both of the Bacchantes and of the children is an emotion of fear, which springs out of an evil habit of the soul. And when some one applies external agitation to affections of this sort, the motion coming from without gets the better of the terrible and violent internal one, and produces a peace and calm in the soul, and quiets the restless palpitation of the heart, which is a thing much to be desired, sending the children to sleep, and making the Bacchantes, although they remain awake, to dance to the pipe with the help of the Gods to whom they offer acceptable sacrifices, and producing in them a sound mind, which takes the place of their frenzy. And, to express what I mean in a word, there is a good deal to be said in favour of this treatment.

\par \textbf{CLEINIAS}
\par   Certainly.

\par \textbf{ATHENIAN}
\par   But if fear has such a power we ought to infer from these facts, that every soul which from youth upward has been familiar with fears, will be made more liable to fear (compare Republic), and every one will allow that this is the way to form a habit of cowardice and not of courage.

\par \textbf{CLEINIAS}
\par   No doubt.

\par \textbf{ATHENIAN}
\par   And, on the other hand, the habit of overcoming, from our youth upwards, the fears and terrors which beset us, may be said to be an exercise of courage.

\par \textbf{CLEINIAS}
\par   True.

\par \textbf{ATHENIAN}
\par   And we may say that the use of exercise and motion in the earliest years of life greatly contributes to create a part of virtue in the soul.

\par \textbf{CLEINIAS}
\par   Quite true.

\par \textbf{ATHENIAN}
\par   Further, a cheerful temper, or the reverse, may be regarded as having much to do with high spirit on the one hand, or with cowardice on the other.

\par \textbf{CLEINIAS}
\par   To be sure.

\par \textbf{ATHENIAN}
\par   Then now we must endeavour to show how and to what extent we may, if we please, without difficulty implant either character in the young.

\par \textbf{CLEINIAS}
\par   Certainly.

\par \textbf{ATHENIAN}
\par   There is a common opinion, that luxury makes the disposition of youth discontented and irascible and vehemently excited by trifles; that on the other hand excessive and savage servitude makes men mean and abject, and haters of their kind, and therefore makes them undesirable associates.

\par \textbf{CLEINIAS}
\par   But how must the state educate those who do not as yet understand the language of the country, and are therefore incapable of appreciating any sort of instruction?

\par \textbf{ATHENIAN}
\par   I will tell you how: —Every animal that is born is wont to utter some cry, and this is especially the case with man, and he is also affected with the inclination to weep more than any other animal.

\par \textbf{CLEINIAS}
\par   Quite true.

\par \textbf{ATHENIAN}
\par   Do not nurses, when they want to know what an infant desires, judge by these signs?—when anything is brought to the infant and he is silent, then he is supposed to be pleased, but, when he weeps and cries out, then he is not pleased. For tears and cries are the inauspicious signs by which children show what they love and hate. Now the time which is thus spent is no less than three years, and is a very considerable portion of life to be passed ill or well.

\par \textbf{CLEINIAS}
\par   True.

\par \textbf{ATHENIAN}
\par   Does not the discontented and ungracious nature appear to you to be full of lamentations and sorrows more than a good man ought to be?

\par \textbf{CLEINIAS}
\par   Certainly.

\par \textbf{ATHENIAN}
\par   Well, but if during these three years every possible care were taken that our nursling should have as little of sorrow and fear, and in general of pain as was possible, might we not expect in early childhood to make his soul more gentle and cheerful? (Compare Arist. Pol.)

\par \textbf{CLEINIAS}
\par   To be sure, Stranger—more especially if we could procure him a variety of pleasures.

\par \textbf{ATHENIAN}
\par   There I can no longer agree, Cleinias:  you amaze me. To bring him up in such a way would be his utter ruin; for the beginning is always the most critical part of education. Let us see whether I am right.

\par \textbf{CLEINIAS}
\par   Proceed.

\par \textbf{ATHENIAN}
\par   The point about which you and I differ is of great importance, and I hope that you, Megillus, will help to decide between us. For I maintain that the true life should neither seek for pleasures, nor, on the other hand, entirely avoid pains, but should embrace the middle state (compare Republic), which I just spoke of as gentle and benign, and is a state which we by some divine presage and inspiration rightly ascribe to God. Now, I say, he among men, too, who would be divine ought to pursue after this mean habit—he should not rush headlong into pleasures, for he will not be free from pains; nor should we allow any one, young or old, male or female, to be thus given any more than ourselves, and least of all the newly-born infant, for in infancy more than at any other time the character is engrained by habit. Nay, more, if I were not afraid of appearing to be ridiculous, I would say that a woman during her year of pregnancy should of all women be most carefully tended, and kept from violent or excessive pleasures and pains, and should at that time cultivate gentleness and benevolence and kindness.

\par \textbf{CLEINIAS}
\par   You need not ask Megillus, Stranger, which of us has most truly spoken; for I myself agree that all men ought to avoid the life of unmingled pain or pleasure, and pursue always a middle course. And having spoken well, may I add that you have been well answered?

\par \textbf{ATHENIAN}
\par   Very good, Cleinias; and now let us all three consider a further point.

\par \textbf{CLEINIAS}
\par   What is it?

\par \textbf{ATHENIAN}
\par   That all the matters which we are now describing are commonly called by the general name of unwritten customs, and what are termed the laws of our ancestors are all of similar nature. And the reflection which lately arose in our minds, that we can neither call these things laws, nor yet leave them unmentioned, is justified; for they are the bonds of the whole state, and come in between the written laws which are or are hereafter to be laid down; they are just ancestral customs of great antiquity, which, if they are rightly ordered and made habitual, shield and preserve the previously existing written law; but if they depart from right and fall into disorder, then they are like the props of builders which slip away out of their place and cause a universal ruin—one part drags another down, and the fair super-structure falls because the old foundations are undermined. Reflecting upon this, Cleinias, you ought to bind together the new state in every possible way, omitting nothing, whether great or small, of what are called laws or manners or pursuits, for by these means a city is bound together, and all these things are only lasting when they depend upon one another; and, therefore, we must not wonder if we find that many apparently trifling customs or usages come pouring in and lengthening out our laws.

\par \textbf{CLEINIAS}
\par   Very true:  we are disposed to agree with you.

\par \textbf{ATHENIAN}
\par   Up to the age of three years, whether of boy or girl, if a person strictly carries out our previous regulations and makes them a principal aim, he will do much for the advantage of the young creatures. But at three, four, five, and even six years the childish nature will require sports; now is the time to get rid of self-will in him, punishing him, but not so as to disgrace him. We were saying about slaves, that we ought neither to add insult to punishment so as to anger them, nor yet to leave them unpunished lest they become self-willed; and a like rule is to be observed in the case of the free-born. Children at that age have certain natural modes of amusement which they find out for themselves when they meet. And all the children who are between the ages of three and six ought to meet at the temples of the villages, the several families of a village uniting on one spot. The nurses are to see that the children behave properly and orderly—they themselves and all their companies are to be under the control of twelve matrons, one for each company, who are annually selected to inspect them from the women previously mentioned [i.e. the women who have authority over marriage], whom the guardians of the law appoint. These matrons shall be chosen by the women who have authority over marriage, one out of each tribe; all are to be of the same age; and let each of them, as soon as she is appointed, hold office and go to the temples every day, punishing all offenders, male or female, who are slaves or strangers, by the help of some of the public slaves; but if any citizen disputes the punishment, let her bring him before the wardens of the city; or, if there be no dispute, let her punish him herself. After the age of six years the time has arrived for the separation of the sexes—let boys live with boys, and girls in like manner with girls. Now they must begin to learn—the boys going to teachers of horsemanship and the use of the bow, the javelin, and sling, and the girls too, if they do not object, at any rate until they know how to manage these weapons, and especially how to handle heavy arms; for I may note, that the practice which now prevails is almost universally misunderstood.

\par \textbf{CLEINIAS}
\par   In what respect?

\par \textbf{ATHENIAN}
\par   In that the right and left hand are supposed to be by nature differently suited for our various uses of them; whereas no difference is found in the use of the feet and the lower limbs; but in the use of the hands we are, as it were, maimed by the folly of nurses and mothers; for although our several limbs are by nature balanced, we create a difference in them by bad habit. In some cases this is of no consequence, as, for example, when we hold the lyre in the left hand, and the plectrum in the right, but it is downright folly to make the same distinction in other cases. The custom of the Scythians proves our error; for they not only hold the bow from them with the left hand and draw the arrow to them with their right, but use either hand for both purposes. And there are many similar examples in charioteering and other things, from which we may learn that those who make the left side weaker than the right act contrary to nature. In the case of the plectrum, which is of horn only, and similar instruments, as I was saying, it is of no consequence, but makes a great difference, and may be of very great importance to the warrior who has to use iron weapons, bows and javelins, and the like; above all, when in heavy armour, he has to fight against heavy armour. And there is a very great difference between one who has learnt and one who has not, and between one who has been trained in gymnastic exercises and one who has not been. For as he who is perfectly skilled in the Pancratium or boxing or wrestling, is not unable to fight from his left side, and does not limp and draggle in confusion when his opponent makes him change his position, so in heavy-armed fighting, and in all other things, if I am not mistaken, the like holds—he who has these double powers of attack and defence ought not in any case to leave them either unused or untrained, if he can help; and if a person had the nature of Geryon or Briareus he ought to be able with his hundred hands to throw a hundred darts. Now, the magistrates, male and female, should see to all these things, the women superintending the nursing and amusements of the children, and the men superintending their education, that all of them, boys and girls alike, may be sound hand and foot, and may not, if they can help, spoil the gifts of nature by bad habits.

\par  Education has two branches—one of gymnastic, which is concerned with the body, and the other of music, which is designed for the improvement of the soul. And gymnastic has also two branches—dancing and wrestling; and one sort of dancing imitates musical recitation, and aims at preserving dignity and freedom, the other aims at producing health, agility, and beauty in the limbs and parts of the body, giving the proper flexion and extension to each of them, a harmonious motion being diffused everywhere, and forming a suitable accompaniment to the dance. As regards wrestling, the tricks which Antaeus and Cercyon devised in their systems out of a vain spirit of competition, or the tricks of boxing which Epeius or Amycus invented, are useless and unsuitable for war, and do not deserve to have much said about them; but the art of wrestling erect and keeping free the neck and hands and sides, working with energy and constancy, with a composed strength, for the sake of health—these are always useful, and are not to be neglected, but to be enjoined alike on masters and scholars, when we reach that part of legislation; and we will desire the one to give their instructions freely, and the others to receive them thankfully. Nor, again, must we omit suitable imitations of war in our choruses; here in Crete you have the armed dances of the Curetes, and the Lacedaemonians have those of the Dioscuri. And our virgin lady, delighting in the amusement of the dance, thought it not fit to amuse herself with empty hands; she must be clothed in a complete suit of armour, and in this attire go through the dance; and youths and maidens should in every respect imitate her, esteeming highly the favour of the Goddess, both with a view to the necessities of war, and to festive occasions: it will be right also for the boys, until such time as they go out to war, to make processions and supplications to all the Gods in goodly array, armed and on horseback, in dances and marches, fast or slow, offering up prayers to the Gods and to the sons of Gods; and also engaging in contests and preludes of contests, if at all, with these objects. For these sorts of exercises, and no others, are useful both in peace and war, and are beneficial alike to states and to private houses. But other labours and sports and exercises of the body are unworthy of freemen, O Megillus and Cleinias.

\par  I have now completely described the kind of gymnastic which I said at first ought to be described; if you know of any better, will you communicate your thoughts?

\par \textbf{CLEINIAS}
\par   It is not easy, Stranger, to put aside these principles of gymnastic and wrestling and to enunciate better ones.

\par \textbf{ATHENIAN}
\par   Now we must say what has yet to be said about the gifts of the Muses and of Apollo:  before, we fancied that we had said all, and that gymnastic alone remained; but now we see clearly what points have been omitted, and should be first proclaimed; of these, then, let us proceed to speak.

\par \textbf{CLEINIAS}
\par   By all means.

\par \textbf{ATHENIAN}
\par   Let me tell you once more—although you have heard me say the same before—that caution must be always exercised, both by the speaker and by the hearer, about anything that is very singular and unusual. For my tale is one which many a man would be afraid to tell, and yet I have a confidence which makes me go on.

\par \textbf{CLEINIAS}
\par   What have you to say, Stranger?

\par \textbf{ATHENIAN}
\par   I say that in states generally no one has observed that the plays of childhood have a great deal to do with the permanence or want of permanence in legislation. For when plays are ordered with a view to children having the same plays, and amusing themselves after the same manner, and finding delight in the same playthings, the more solemn institutions of the state are allowed to remain undisturbed. Whereas if sports are disturbed, and innovations are made in them, and they constantly change, and the young never speak of their having the same likings, or the same established notions of good and bad taste, either in the bearing of their bodies or in their dress, but he who devises something new and out of the way in figures and colours and the like is held in special honour, we may truly say that no greater evil can happen in a state; for he who changes the sports is secretly changing the manners of the young, and making the old to be dishonoured among them and the new to be honoured. And I affirm that there is nothing which is a greater injury to all states than saying or thinking thus. Will you hear me tell how great I deem the evil to be?

\par \textbf{CLEINIAS}
\par   You mean the evil of blaming antiquity in states?

\par \textbf{ATHENIAN}
\par   Exactly.

\par \textbf{CLEINIAS}
\par   If you are speaking of that, you will find in us hearers who are disposed to receive what you say not unfavourably but most favourably.

\par \textbf{ATHENIAN}
\par   I should expect so.

\par \textbf{CLEINIAS}
\par   Proceed.

\par \textbf{ATHENIAN}
\par   Well, then, let us give all the greater heed to one another's words. The argument affirms that any change whatever except from evil is the most dangerous of all things; this is true in the case of the seasons and of the winds, in the management of our bodies and the habits of our minds—true of all things except, as I said before, of the bad. He who looks at the constitution of individuals accustomed to eat any sort of meat, or drink any drink, or to do any work which they can get, may see that they are at first disordered by them, but afterwards, as time goes on, their bodies grow adapted to them, and they learn to know and like variety, and have good health and enjoyment of life; and if ever afterwards they are confined again to a superior diet, at first they are troubled with disorders, and with difficulty become habituated to their new food. A similar principle we may imagine to hold good about the minds of men and the natures of their souls. For when they have been brought up in certain laws, which by some Divine Providence have remained unchanged during long ages, so that no one has any memory or tradition of their ever having been otherwise than they are, then every one is afraid and ashamed to change that which is established. The legislator must somehow find a way of implanting this reverence for antiquity, and I would propose the following way:  People are apt to fancy, as I was saying before, that when the plays of children are altered they are merely plays, not seeing that the most serious and detrimental consequences arise out of the change; and they readily comply with the child's wishes instead of deterring him, not considering that these children who make innovations in their games, when they grow up to be men, will be different from the last generation of children, and, being different, will desire a different sort of life, and under the influence of this desire will want other institutions and laws; and no one of them reflects that there will follow what I just now called the greatest of evils to states. Changes in bodily fashions are no such serious evils, but frequent changes in the praise and censure of manners are the greatest of evils, and require the utmost prevision.

\par \textbf{CLEINIAS}
\par   To be sure.

\par \textbf{ATHENIAN}
\par   And now do we still hold to our former assertion, that rhythms and music in general are imitations of good and evil characters in men? What say you?

\par \textbf{CLEINIAS}
\par   That is the only doctrine which we can admit.

\par \textbf{ATHENIAN}
\par   Must we not, then, try in every possible way to prevent our youth from even desiring to imitate new modes either in dance or song? nor must any one be allowed to offer them varieties of pleasures.

\par \textbf{CLEINIAS}
\par   Most true.

\par \textbf{ATHENIAN}
\par   Can any of us imagine a better mode of effecting this object than that of the Egyptians?

\par \textbf{CLEINIAS}
\par   What is their method?

\par \textbf{ATHENIAN}
\par   To consecrate every sort of dance or melody. First we should ordain festivals—calculating for the year what they ought to be, and at what time, and in honour of what Gods, sons of Gods, and heroes they ought to be celebrated; and, in the next place, what hymns ought to be sung at the several sacrifices, and with what dances the particular festival is to be honoured. This has to be arranged at first by certain persons, and, when arranged, the whole assembly of the citizens are to offer sacrifices and libations to the Fates and all the other Gods, and to consecrate the several odes to Gods and heroes:  and if any one offers any other hymns or dances to any one of the Gods, the priests and priestesses, acting in concert with the guardians of the law, shall, with the sanction of religion and the law, exclude him, and he who is excluded, if he do not submit, shall be liable all his life long to have a suit of impiety brought against him by any one who likes.

\par \textbf{CLEINIAS}
\par   Very good.

\par \textbf{ATHENIAN}
\par   In the consideration of this subject, let us remember what is due to ourselves.

\par \textbf{CLEINIAS}
\par   To what are you referring?

\par \textbf{ATHENIAN}
\par   I mean that any young man, and much more any old one, when he sees or hears anything strange or unaccustomed, does not at once run to embrace the paradox, but he stands considering, like a person who is at a place where three paths meet, and does not very well know his way—he may be alone or he may be walking with others, and he will say to himself and them, 'Which is the way?' and will not move forward until he is satisfied that he is going right. And this is what we must do in the present instance:  A strange discussion on the subject of law has arisen, which requires the utmost consideration, and we should not at our age be too ready to speak about such great matters, or be confident that we can say anything certain all in a moment.

\par \textbf{CLEINIAS}
\par   Most true.

\par \textbf{ATHENIAN}
\par   Then we will allow time for reflection, and decide when we have given the subject sufficient consideration. But that we may not be hindered from completing the natural arrangement of our laws, let us proceed to the conclusion of them in due order; for very possibly, if God will, the exposition of them, when completed, may throw light on our present perplexity.

\par \textbf{CLEINIAS}
\par   Excellent, Stranger; let us do as you propose.

\par \textbf{ATHENIAN}
\par   Let us then affirm the paradox that strains of music are our laws (nomoi), and this latter being the name which the ancients gave to lyric songs, they probably would not have very much objected to our proposed application of the word. Some one, either asleep or awake, must have had a dreamy suspicion of their nature. And let our decree be as follows:  No one in singing or dancing shall offend against public and consecrated models, and the general fashion among the youth, any more than he would offend against any other law. And he who observes this law shall be blameless; but he who is disobedient, as I was saying, shall be punished by the guardians of the laws, and by the priests and priestesses. Suppose that we imagine this to be our law.

\par \textbf{CLEINIAS}
\par   Very good.

\par \textbf{ATHENIAN}
\par   Can any one who makes such laws escape ridicule? Let us see. I think that our only safety will be in first framing certain models for composers. One of these models shall be as follows:  If when a sacrifice is going on, and the victims are being burnt according to law—if, I say, any one who may be a son or brother, standing by another at the altar and over the victims, horribly blasphemes, will not his words inspire despondency and evil omens and forebodings in the mind of his father and of his other kinsmen?

\par \textbf{CLEINIAS}
\par   Of course.

\par \textbf{ATHENIAN}
\par   And this is just what takes place in almost all our cities. A magistrate offers a public sacrifice, and there come in not one but many choruses, who take up a position a little way from the altar, and from time to time pour forth all sorts of horrible blasphemies on the sacred rites, exciting the souls of the audience with words and rhythms and melodies most sorrowful to hear; and he who at the moment when the city is offering sacrifice makes the citizens weep most, carries away the palm of victory. Now, ought we not to forbid such strains as these? And if ever our citizens must hear such lamentations, then on some unblest and inauspicious day let there be choruses of foreign and hired minstrels, like those hirelings who accompany the departed at funerals with barbarous Carian chants. That is the sort of thing which will be appropriate if we have such strains at all; and let the apparel of the singers be, not circlets and ornaments of gold, but the reverse. Enough of all this. I will simply ask once more whether we shall lay down as one of our principles of song—

\par \textbf{CLEINIAS}
\par   What?

\par \textbf{ATHENIAN}
\par   That we should avoid every word of evil omen; let that kind of song which is of good omen be heard everywhere and always in our state. I need hardly ask again, but shall assume that you agree with me.

\par \textbf{CLEINIAS}
\par   By all means; that law is approved by the suffrages of us all.

\par \textbf{ATHENIAN}
\par   But what shall be our next musical law or type? Ought not prayers to be offered up to the Gods when we sacrifice?

\par \textbf{CLEINIAS}
\par   Certainly.

\par \textbf{ATHENIAN}
\par   And our third law, if I am not mistaken, will be to the effect that our poets, understanding prayers to be requests which we make to the Gods, will take especial heed that they do not by mistake ask for evil instead of good. To make such a prayer would surely be too ridiculous.

\par \textbf{CLEINIAS}
\par   Very true.

\par \textbf{ATHENIAN}
\par   Were we not a little while ago quite convinced that no silver or golden Plutus should dwell in our state?

\par \textbf{CLEINIAS}
\par   To be sure.

\par \textbf{ATHENIAN}
\par   And what has it been the object of our argument to show? Did we not imply that the poets are not always quite capable of knowing what is good or evil? And if one of them utters a mistaken prayer in song or words, he will make our citizens pray for the opposite of what is good in matters of the highest import; than which, as I was saying, there can be few greater mistakes. Shall we then propose as one of our laws and models relating to the Muses—

\par \textbf{CLEINIAS}
\par   What? will you explain the law more precisely?

\par \textbf{ATHENIAN}
\par   Shall we make a law that the poet shall compose nothing contrary to the ideas of the lawful, or just, or beautiful, or good, which are allowed in the state? nor shall he be permitted to communicate his compositions to any private individuals, until he shall have shown them to the appointed judges and the guardians of the law, and they are satisfied with them. As to the persons whom we appoint to be our legislators about music and as to the director of education, these have been already indicated. Once more then, as I have asked more than once, shall this be our third law, and type, and model—What do you say?

\par \textbf{CLEINIAS}
\par   Let it be so, by all means.

\par \textbf{ATHENIAN}
\par   Then it will be proper to have hymns and praises of the Gods, intermingled with prayers; and after the Gods prayers and praises should be offered in like manner to demigods and heroes, suitable to their several characters.

\par \textbf{CLEINIAS}
\par   Certainly.

\par \textbf{ATHENIAN}
\par   In the next place there will be no objection to a law, that citizens who are departed and have done good and energetic deeds, either with their souls or with their bodies, and have been obedient to the laws, should receive eulogies; this will be very fitting.

\par \textbf{CLEINIAS}
\par   Quite true.

\par \textbf{ATHENIAN}
\par   But to honour with hymns and panegyrics those who are still alive is not safe; a man should run his course, and make a fair ending, and then we will praise him; and let praise be given equally to women as well as men who have been distinguished in virtue. The order of songs and dances shall be as follows:  There are many ancient musical compositions and dances which are excellent, and from these the newly-founded city may freely select what is proper and suitable; and they shall choose judges of not less than fifty years of age, who shall make the selection, and any of the old poems which they deem sufficient they shall include; any that are deficient or altogether unsuitable, they shall either utterly throw aside, or examine and amend, taking into their counsel poets and musicians, and making use of their poetical genius; but explaining to them the wishes of the legislator in order that they may regulate dancing, music, and all choral strains, according to the mind of the judges; and not allowing them to indulge, except in some few matters, their individual pleasures and fancies. Now the irregular strain of music is always made ten thousand times better by attaining to law and order, and rejecting the honeyed Muse—not however that we mean wholly to exclude pleasure, which is the characteristic of all music. And if a man be brought up from childhood to the age of discretion and maturity in the use of the orderly and severe music, when he hears the opposite he detests it, and calls it illiberal; but if trained in the sweet and vulgar music, he deems the severer kind cold and displeasing. So that, as I was saying before, while he who hears them gains no more pleasure from the one than from the other, the one has the advantage of making those who are trained in it better men, whereas the other makes them worse.

\par \textbf{CLEINIAS}
\par   Very true.

\par \textbf{ATHENIAN}
\par   Again, we must distinguish and determine on some general principle what songs are suitable to women, and what to men, and must assign to them their proper melodies and rhythms. It is shocking for a whole harmony to be inharmonical, or for a rhythm to be unrhythmical, and this will happen when the melody is inappropriate to them. And therefore the legislator must assign to these also their forms. Now both sexes have melodies and rhythms which of necessity belong to them; and those of women are clearly enough indicated by their natural difference. The grand, and that which tends to courage, may be fairly called manly; but that which inclines to moderation and temperance, may be declared both in law and in ordinary speech to be the more womanly quality. This, then, will be the general order of them.

\par  Let us now speak of the manner of teaching and imparting them, and the persons to whom, and the time when, they are severally to be imparted. As the shipwright first lays down the lines of the keel, and thus, as it were, draws the ship in outline, so do I seek to distinguish the patterns of life, and lay down their keels according to the nature of different men's souls; seeking truly to consider by what means, and in what ways, we may go through the voyage of life best. Now human affairs are hardly worth considering in earnest, and yet we must be in earnest about them—a sad necessity constrains us. And having got thus far, there will be a fitness in our completing the matter, if we can only find some suitable method of doing so. But what do I mean? Some one may ask this very question, and quite rightly, too.

\par \textbf{CLEINIAS}
\par   Certainly.

\par \textbf{ATHENIAN}
\par   I say that about serious matters a man should be serious, and about a matter which is not serious he should not be serious; and that God is the natural and worthy object of our most serious and blessed endeavours, for man, as I said before, is made to be the plaything of God, and this, truly considered, is the best of him; wherefore also every man and woman should walk seriously, and pass life in the noblest of pastimes, and be of another mind from what they are at present.

\par \textbf{CLEINIAS}
\par   In what respect?

\par \textbf{ATHENIAN}
\par   At present they think that their serious pursuits should be for the sake of their sports, for they deem war a serious pursuit, which must be managed well for the sake of peace; but the truth is, that there neither is, nor has been, nor ever will be, either amusement or instruction in any degree worth speaking of in war, which is nevertheless deemed by us to be the most serious of our pursuits. And therefore, as we say, every one of us should live the life of peace as long and as well as he can. And what is the right way of living? Are we to live in sports always? If so, in what kind of sports? We ought to live sacrificing, and singing, and dancing, and then a man will be able to propitiate the Gods, and to defend himself against his enemies and conquer them in battle. The type of song or dance by which he will propitiate them has been described, and the paths along which he is to proceed have been cut for him. He will go forward in the spirit of the poet:

\par  'Telemachus, some things thou wilt thyself find in thy heart, but other things God will suggest; for I deem that thou wast not born or brought up without the will of the Gods.'

\par  And this ought to be the view of our alumni; they ought to think that what has been said is enough for them, and that any other things their Genius and God will suggest to them—he will tell them to whom, and when, and to what Gods severally they are to sacrifice and perform dances, and how they may propitiate the deities, and live according to the appointment of nature; being for the most part puppets, but having some little share of reality.

\par \textbf{MEGILLUS}
\par   You have a low opinion of mankind, Stranger.

\par \textbf{ATHENIAN}
\par   Nay, Megillus, be not amazed, but forgive me:  I was comparing them with the Gods; and under that feeling I spoke. Let us grant, if you wish, that the human race is not to be despised, but is worthy of some consideration.

\par  Next follow the buildings for gymnasia and schools open to all; these are to be in three places in the midst of the city; and outside the city and in the surrounding country, also in three places, there shall be schools for horse exercise, and large grounds arranged with a view to archery and the throwing of missiles, at which young men may learn and practise. Of these mention has already been made; and if the mention be not sufficiently explicit, let us speak further of them and embody them in laws. In these several schools let there be dwellings for teachers, who shall be brought from foreign parts by pay, and let them teach those who attend the schools the art of war and the art of music, and the children shall come not only if their parents please, but if they do not please; there shall be compulsory education, as the saying is, of all and sundry, as far as this is possible; and the pupils shall be regarded as belonging to the state rather than to their parents. My law would apply to females as well as males; they shall both go through the same exercises. I assert without fear of contradiction that gymnastic and horsemanship are as suitable to women as to men. Of the truth of this I am persuaded from ancient tradition, and at the present day there are said to be countless myriads of women in the neighbourhood of the Black Sea, called Sauromatides, who not only ride on horseback like men, but have enjoined upon them the use of bows and other weapons equally with the men. And I further affirm, that if these things are possible, nothing can be more absurd than the practice which prevails in our own country, of men and women not following the same pursuits with all their strength and with one mind, for thus the state, instead of being a whole, is reduced to a half, but has the same imposts to pay and the same toils to undergo; and what can be a greater mistake for any legislator to make than this?

\par \textbf{CLEINIAS}
\par   Very true; yet much of what has been asserted by us, Stranger, is contrary to the custom of states; still, in saying that the discourse should be allowed to proceed, and that when the discussion is completed, we should choose what seems best, you spoke very properly, and I now feel compunction for what I have said. Tell me, then, what you would next wish to say.

\par \textbf{ATHENIAN}
\par   I should wish to say, Cleinias, as I said before, that if the possibility of these things were not sufficiently proven in fact, then there might be an objection to the argument, but the fact being as I have said, he who rejects the law must find some other ground of objection; and, failing this, our exhortation will still hold good, nor will any one deny that women ought to share as far as possible in education and in other ways with men. For consider; if women do not share in their whole life with men, then they must have some other order of life.

\par \textbf{CLEINIAS}
\par   Certainly.

\par \textbf{ATHENIAN}
\par   And what arrangement of life to be found anywhere is preferable to this community which we are now assigning to them? Shall we prefer that which is adopted by the Thracians and many other races who use their women to till the ground and to be shepherds of their herds and flocks, and to minister to them like slaves? Or shall we do as we and people in our part of the world do—getting together, as the phrase is, all our goods and chattels into one dwelling, we entrust them to our women, who are the stewards of them, and who also preside over the shuttles and the whole art of spinning? Or shall we take a middle course, as in Lacedaemon, Megillus—letting the girls share in gymnastic and music, while the grown-up women, no longer employed in spinning wool, are hard at work weaving the web of life, which will be no cheap or mean employment, and in the duty of serving and taking care of the household and bringing up the children, in which they will observe a sort of mean, not participating in the toils of war; and if there were any necessity that they should fight for their city and families, unlike the Amazons, they would be unable to take part in archery or any other skilled use of missiles, nor could they, after the example of the Goddess, carry shield or spear, or stand up nobly for their country when it was being destroyed, and strike terror into their enemies, if only because they were seen in regular order? Living as they do, they would never dare at all to imitate the Sauromatides, who, when compared with ordinary women, would appear to be like men. Let him who will, praise your legislators, but I must say what I think. The legislator ought to be whole and perfect, and not half a man only; he ought not to let the female sex live softly and waste money and have no order of life, while he takes the utmost care of the male sex, and leaves half of life only blest with happiness, when he might have made the whole state happy.

\par \textbf{MEGILLUS}
\par   What shall we do, Cleinias? Shall we allow a stranger to run down Sparta in this fashion?

\par \textbf{CLEINIAS}
\par   Yes; for as we have given him liberty of speech we must let him go on until we have perfected the work of legislation.

\par \textbf{MEGILLUS}
\par   Very true.

\par \textbf{ATHENIAN}
\par   Then now I may proceed?

\par \textbf{CLEINIAS}
\par   By all means.

\par \textbf{ATHENIAN}
\par   What will be the manner of life among men who may be supposed to have their food and clothing provided for them in moderation, and who have entrusted the practice of the arts to others, and whose husbandry committed to slaves paying a part of the produce, brings them a return sufficient for men living temperately; who, moreover, have common tables in which the men are placed apart, and near them are the common tables of their families, of their daughters and mothers, which day by day, the officers, male and female, are to inspect—they shall see to the behaviour of the company, and so dismiss them; after which the presiding magistrate and his attendants shall honour with libations those Gods to whom that day and night are dedicated, and then go home? To men whose lives are thus ordered, is there no work remaining to be done which is necessary and fitting, but shall each one of them live fattening like a beast? Such a life is neither just nor honourable, nor can he who lives it fail of meeting his due; and the due reward of the idle fatted beast is that he should be torn in pieces by some other valiant beast whose fatness is worn down by brave deeds and toil. These regulations, if we duly consider them, will never be exactly carried into execution under present circumstances, nor as long as women and children and houses and all other things are the private property of individuals; but if we can attain the second-best form of polity, we shall be very well off. And to men living under this second polity there remains a work to be accomplished which is far from being small or insignificant, but is the greatest of all works, and ordained by the appointment of righteous law. For the life which may be truly said to be concerned with the virtue of body and soul is twice, or more than twice, as full of toil and trouble as the pursuit after Pythian and Olympic victories, which debars a man from every employment of life. For there ought to be no bye-work interfering with the greater work of providing the necessary exercise and nourishment for the body, and instruction and education for the soul. Night and day are not long enough for the accomplishment of their perfection and consummation; and therefore to this end all freemen ought to arrange the way in which they will spend their time during the whole course of the day, from morning till evening and from evening till the morning of the next sunrise. There may seem to be some impropriety in the legislator determining minutely the numberless details of the management of the house, including such particulars as the duty of wakefulness in those who are to be perpetual watchmen of the whole city; for that any citizen should continue during the whole of any night in sleep, instead of being seen by all his servants, always the first to awake and get up—this, whether the regulation is to be called a law or only a practice, should be deemed base and unworthy of a freeman; also that the mistress of the house should be awakened by her hand-maidens instead of herself first awakening them, is what the slaves, male and female, and the serving-boys, and, if that were possible, everybody and everything in the house should regard as base. If they rise early, they may all of them do much of their public and of their household business, as magistrates in the city, and masters and mistresses in their private houses, before the sun is up. Much sleep is not required by nature, either for our souls or bodies, or for the actions which they perform. For no one who is asleep is good for anything, any more than if he were dead; but he of us who has the most regard for life and reason keeps awake as long as he can, reserving only so much time for sleep as is expedient for health; and much sleep is not required, if the habit of moderation be once rightly formed. Magistrates in states who keep awake at night are terrible to the bad, whether enemies or citizens, and are honoured and reverenced by the just and temperate, and are useful to themselves and to the whole state.

\par  A night which is passed in such a manner, in addition to all the above-mentioned advantages, infuses a sort of courage into the minds of the citizens. When the day breaks, the time has arrived for youth to go to their schoolmasters. Now neither sheep nor any other animals can live without a shepherd, nor can children be left without tutors, or slaves without masters. And of all animals the boy is the most unmanageable, inasmuch as he has the fountain of reason in him not yet regulated; he is the most insidious, sharp-witted, and insubordinate of animals. Wherefore he must be bound with many bridles; in the first place, when he gets away from mothers and nurses, he must be under the management of tutors on account of his childishness and foolishness; then, again, being a freeman, he must be controlled by teachers, no matter what they teach, and by studies; but he is also a slave, and in that regard any freeman who comes in his way may punish him and his tutor and his instructor, if any of them does anything wrong; and he who comes across him and does not inflict upon him the punishment which he deserves, shall incur the greatest disgrace; and let the guardian of the law, who is the director of education, see to him who coming in the way of the offences which we have mentioned, does not chastise them when he ought, or chastises them in a way which he ought not; let him keep a sharp look-out, and take especial care of the training of our children, directing their natures, and always turning them to good according to the law.

\par  But how can our law sufficiently train the director of education himself; for as yet all has been imperfect, and nothing has been said either clear or satisfactory? Now, as far as possible, the law ought to leave nothing to him, but to explain everything, that he may be an interpreter and tutor to others. About dances and music and choral strains, I have already spoken both as to the character of the selection of them, and the manner in which they are to be amended and consecrated. But we have not as yet spoken, O illustrious guardian of education, of the manner in which your pupils are to use those strains which are written in prose, although you have been informed what martial strains they are to learn and practise; what relates in the first place to the learning of letters, and secondly, to the lyre, and also to calculation, which, as we were saying, is needful for them all to learn, and any other things which are required with a view to war and the management of house and city, and, looking to the same object, what is useful in the revolutions of the heavenly bodies—the stars and sun and moon, and the various regulations about these matters which are necessary for the whole state—I am speaking of the arrangements of days in periods of months, and of months in years, which are to be observed, in order that seasons and sacrifices and festivals may have their regular and natural order, and keep the city alive and awake, the Gods receiving the honours due to them, and men having a better understanding about them: all these things, O my friend, have not yet been sufficiently declared to you by the legislator. Attend, then, to what I am now going to say: We were telling you, in the first place, that you were not sufficiently informed about letters, and the objection was to this effect—that you were never told whether he who was meant to be a respectable citizen should apply himself in detail to that sort of learning, or not apply himself at all; and the same remark holds good of the study of the lyre. But now we say that he ought to attend to them. A fair time for a boy of ten years old to spend in letters is three years; the age of thirteen is the proper time for him to begin to handle the lyre, and he may continue at this for another three years, neither more nor less, and whether his father or himself like or dislike the study, he is not to be allowed to spend more or less time in learning music than the law allows. And let him who disobeys the law be deprived of those youthful honours of which we shall hereafter speak. Hear, however, first of all, what the young ought to learn in the early years of life, and what their instructors ought to teach them. They ought to be occupied with their letters until they are able to read and write; but the acquisition of perfect beauty or quickness in writing, if nature has not stimulated them to acquire these accomplishments in the given number of years, they should let alone. And as to the learning of compositions committed to writing which are not set to the lyre, whether metrical or without rhythmical divisions, compositions in prose, as they are termed, having no rhythm or harmony—seeing how dangerous are the writings handed down to us by many writers of this class—what will you do with them, O most excellent guardians of the law? or how can the lawgiver rightly direct you about them? I believe that he will be in great difficulty.

\par \textbf{CLEINIAS}
\par   What troubles you, Stranger? and why are you so perplexed in your mind?

\par \textbf{ATHENIAN}
\par   You naturally ask, Cleinias, and to you and Megillus, who are my partners in the work of legislation, I must state the more difficult as well as the easier parts of the task.

\par \textbf{CLEINIAS}
\par   To what do you refer in this instance?

\par \textbf{ATHENIAN}
\par   I will tell you. There is a difficulty in opposing many myriads of mouths.

\par \textbf{CLEINIAS}
\par   Well, and have we not already opposed the popular voice in many important enactments?

\par \textbf{ATHENIAN}
\par   That is quite true; and you mean to imply that the road which we are taking may be disagreeable to some but is agreeable to as many others, or if not to as many, at any rate to persons not inferior to the others, and in company with them you bid me, at whatever risk, to proceed along the path of legislation which has opened out of our present discourse, and to be of good cheer, and not to faint.

\par \textbf{CLEINIAS}
\par   Certainly.

\par \textbf{ATHENIAN}
\par   And I do not faint; I say, indeed, that we have a great many poets writing in hexameter, trimeter, and all sorts of measures—some who are serious, others who aim only at raising a laugh—and all mankind declare that the youth who are rightly educated should be brought up in them and saturated with them; some insist that they should be constantly hearing them read aloud, and always learning them, so as to get by heart entire poets; while others select choice passages and long speeches, and make compendiums of them, saying that these ought to be committed to memory, if a man is to be made good and wise by experience and learning of many things. And you want me now to tell them plainly in what they are right and in what they are wrong.

\par \textbf{CLEINIAS}
\par   Yes, I do.

\par \textbf{ATHENIAN}
\par   But how can I in one word rightly comprehend all of them? I am of opinion, and, if I am not mistaken, there is a general agreement, that every one of these poets has said many things well and many things the reverse of well; and if this be true, then I do affirm that much learning is dangerous to youth.

\par \textbf{CLEINIAS}
\par   How would you advise the guardian of the law to act?

\par \textbf{ATHENIAN}
\par   In what respect?

\par \textbf{CLEINIAS}
\par   I mean to what pattern should he look as his guide in permitting the young to learn some things and forbidding them to learn others. Do not shrink from answering.

\par \textbf{ATHENIAN}
\par   My good Cleinias, I rather think that I am fortunate.

\par \textbf{CLEINIAS}
\par   How so?

\par \textbf{ATHENIAN}
\par   I think that I am not wholly in want of a pattern, for when I consider the words which we have spoken from early dawn until now, and which, as I believe, have been inspired by Heaven, they appear to me to be quite like a poem. When I reflected upon all these words of ours, I naturally felt pleasure, for of all the discourses which I have ever learnt or heard, either in poetry or prose, this seemed to me to be the justest, and most suitable for young men to hear; I cannot imagine any better pattern than this which the guardian of the law who is also the director of education can have. He cannot do better than advise the teachers to teach the young these words and any which are of a like nature, if he should happen to find them, either in poetry or prose, or if he come across unwritten discourses akin to ours, he should certainly preserve them, and commit them to writing. And, first of all, he shall constrain the teachers themselves to learn and approve them, and any of them who will not, shall not be employed by him, but those whom he finds agreeing in his judgment, he shall make use of and shall commit to them the instruction and education of youth. And here and on this wise let my fanciful tale about letters and teachers of letters come to an end.

\par \textbf{CLEINIAS}
\par   I do not think, Stranger, that we have wandered out of the proposed limits of the argument; but whether we are right or not in our whole conception, I cannot be very certain.

\par \textbf{ATHENIAN}
\par   The truth, Cleinias, may be expected to become clearer when, as we have often said, we arrive at the end of the whole discussion about laws.

\par \textbf{CLEINIAS}
\par   Yes.

\par \textbf{ATHENIAN}
\par   And now that we have done with the teacher of letters, the teacher of the lyre has to receive orders from us.

\par \textbf{CLEINIAS}
\par   Certainly.

\par \textbf{ATHENIAN}
\par   I think that we have only to recollect our previous discussions, and we shall be able to give suitable regulations touching all this part of instruction and education to the teachers of the lyre.

\par \textbf{CLEINIAS}
\par   To what do you refer?

\par \textbf{ATHENIAN}
\par   We were saying, if I remember rightly, that the sixty years old choristers of Dionysus were to be specially quick in their perceptions of rhythm and musical composition, that they might be able to distinguish good and bad imitation, that is to say, the imitation of the good or bad soul when under the influence of passion, rejecting the one and displaying the other in hymns and songs, charming the souls of youth, and inviting them to follow and attain virtue by the way of imitation.

\par \textbf{CLEINIAS}
\par   Very true.

\par \textbf{ATHENIAN}
\par   And with this view the teacher and the learner ought to use the sounds of the lyre, because its notes are pure, the player who teaches and his pupil rendering note for note in unison; but complexity, and variation of notes, when the strings give one sound and the poet or composer of the melody gives another—also when they make concords and harmonies in which lesser and greater intervals, slow and quick, or high and low notes, are combined—or, again, when they make complex variations of rhythms, which they adapt to the notes of the lyre—all that sort of thing is not suited to those who have to acquire speedy and useful knowledge of music in three years; for opposite principles are confusing, and create a difficulty in learning, and our young men should learn quickly, and their mere necessary acquirements are not few or trifling, as will be shown in due course. Let the director of education attend to the principles concerning music which we are laying down. As to the songs and words themselves which the masters of choruses are to teach and the character of them, they have been already described by us, and are the same which, when consecrated and adapted to the different festivals, we said were to benefit cities by affording them an innocent amusement.

\par \textbf{CLEINIAS}
\par   That, again, is true.

\par \textbf{ATHENIAN}
\par   Then let him who has been elected a director of music receive these rules from us as containing the very truth; and may he prosper in his office! Let us now proceed to lay down other rules in addition to the preceding about dancing and gymnastic exercise in general. Having said what remained to be said about the teaching of music, let us speak in like manner about gymnastic. For boys and girls ought to learn to dance and practise gymnastic exercises—ought they not?

\par \textbf{CLEINIAS}
\par   Yes.

\par \textbf{ATHENIAN}
\par   Then the boys ought to have dancing masters, and the girls dancing mistresses to exercise them.

\par \textbf{CLEINIAS}
\par   Very good.

\par \textbf{ATHENIAN}
\par   Then once more let us summon him who has the chief concern in the business, the superintendent of youth [i.e. the director of education]; he will have plenty to do, if he is to have the charge of music and gymnastic.

\par \textbf{CLEINIAS}
\par   But how will an old man be able to attend to such great charges?

\par \textbf{ATHENIAN}
\par   O my friend, there will be no difficulty, for the law has already given and will give him permission to select as his assistants in this charge any citizens, male or female, whom he desires; and he will know whom he ought to choose, and will be anxious not to make a mistake, from a due sense of responsibility, and from a consciousness of the importance of his office, and also because he will consider that if young men have been and are well brought up, then all things go swimmingly, but if not, it is not meet to say, nor do we say, what will follow, lest the regarders of omens should take alarm about our infant state. Many things have been said by us about dancing and about gymnastic movements in general; for we include under gymnastics all military exercises, such as archery, and all hurling of weapons, and the use of the light shield, and all fighting with heavy arms, and military evolutions, and movements of armies, and encampings, and all that relates to horsemanship. Of all these things there ought to be public teachers, receiving pay from the state, and their pupils should be the men and boys in the state, and also the girls and women, who are to know all these things. While they are yet girls they should have practised dancing in arms and the whole art of fighting—when grown-up women, they should apply themselves to evolutions and tactics, and the mode of grounding and taking up arms; if for no other reason, yet in case the whole military force should have to leave the city and carry on operations of war outside, that those who will have to guard the young and the rest of the city may be equal to the task; and, on the other hand, when enemies, whether barbarian or Hellenic, come from without with mighty force and make a violent assault upon them, and thus compel them to fight for the possession of the city, which is far from being an impossibility, great would be the disgrace to the state, if the women had been so miserably trained that they could not fight for their young, as birds will, against any creature however strong, and die or undergo any danger, but must instantly rush to the temples and crowd at the altars and shrines, and bring upon human nature the reproach, that of all animals man is the most cowardly!

\par \textbf{CLEINIAS}
\par   Such a want of education, Stranger, is certainly an unseemly thing to happen in a state, as well as a great misfortune.

\par \textbf{ATHENIAN}
\par   Suppose that we carry our law to the extent of saying that women ought not to neglect military matters, but that all citizens, male and female alike, shall attend to them?

\par \textbf{CLEINIAS}
\par   I quite agree.

\par \textbf{ATHENIAN}
\par   Of wrestling we have spoken in part, but of what I should call the most important part we have not spoken, and cannot easily speak without showing at the same time by gesture as well as in word what we mean; when word and action combine, and not till then, we shall explain clearly what has been said, pointing out that of all movements wrestling is most akin to the military art, and is to be pursued for the sake of this, and not this for the sake of wrestling.

\par \textbf{CLEINIAS}
\par   Excellent. ATHENIAN:  Enough of wrestling; we will now proceed to speak of other movements of the body. Such motion may be in general called dancing, and is of two kinds:  one of nobler figures, imitating the honourable, the other of the more ignoble figures, imitating the mean; and of both these there are two further subdivisions. Of the serious, one kind is of those engaged in war and vehement action, and is the exercise of a noble person and a manly heart; the other exhibits a temperate soul in the enjoyment of prosperity and modest pleasures, and may be truly called and is the dance of peace. The warrior dance is different from the peaceful one, and may be rightly termed Pyrrhic; this imitates the modes of avoiding blows and missiles by dropping or giving way, or springing aside, or rising up or falling down; also the opposite postures which are those of action, as, for example, the imitation of archery and the hurling of javelins, and of all sorts of blows. And when the imitation is of brave bodies and souls, and the action is direct and muscular, giving for the most part a straight movement to the limbs of the body—that, I say, is the true sort; but the opposite is not right. In the dance of peace what we have to consider is whether a man bears himself naturally and gracefully, and after the manner of men who duly conform to the law. But before proceeding I must distinguish the dancing about which there is any doubt, from that about which there is no doubt. Which is the doubtful kind, and how are the two to be distinguished? There are dances of the Bacchic sort, both those in which, as they say, they imitate drunken men, and which are named after the Nymphs, and Pan, and Silenuses, and Satyrs; and also those in which purifications are made or mysteries celebrated—all this sort of dancing cannot be rightly defined as having either a peaceful or a warlike character, or indeed as having any meaning whatever, and may, I think, be most truly described as distinct from the warlike dance, and distinct from the peaceful, and not suited for a city at all. There let it lie; and so leaving it to lie, we will proceed to the dances of war and peace, for with these we are undoubtedly concerned. Now the unwarlike muse, which honours in dance the Gods and the sons of the Gods, is entirely associated with the consciousness of prosperity; this class may be subdivided into two lesser classes, of which one is expressive of an escape from some labour or danger into good, and has greater pleasures, the other expressive of preservation and increase of former good, in which the pleasure is less exciting—in all these cases, every man when the pleasure is greater, moves his body more, and less when the pleasure is less; and, again, if he be more orderly and has learned courage from discipline he moves less, but if he be a coward, and has no training or self-control, he makes greater and more violent movements, and in general when he is speaking or singing he is not altogether able to keep his body still; and so out of the imitation of words in gestures the whole art of dancing has arisen. And in these various kinds of imitation one man moves in an orderly, another in a disorderly manner; and as the ancients may be observed to have given many names which are according to nature and deserving of praise, so there is an excellent one which they have given to the dances of men who in their times of prosperity are moderate in their pleasures—the giver of names, whoever he was, assigned to them a very true, and poetical, and rational name, when he called them Emmeleiai, or dances of order, thus establishing two kinds of dances of the nobler sort, the dance of war which he called the Pyrrhic, and the dance of peace which he called Emmeleia, or the dance of order; giving to each their appropriate and becoming name. These things the legislator should indicate in general outline, and the guardian of the law should enquire into them and search them out, combining dancing with music, and assigning to the several sacrificial feasts that which is suitable to them; and when he has consecrated all of them in due order, he shall for the future change nothing, whether of dance or song. Thenceforward the city and the citizens shall continue to have the same pleasures, themselves being as far as possible alike, and shall live well and happily.

\par  I have described the dances which are appropriate to noble bodies and generous souls. But it is necessary also to consider and know uncomely persons and thoughts, and those which are intended to produce laughter in comedy, and have a comic character in respect of style, song, and dance, and of the imitations which these afford. For serious things cannot be understood without laughable things, nor opposites at all without opposites, if a man is really to have intelligence of either; but he cannot carry out both in action, if he is to have any degree of virtue. And for this very reason he should learn them both, in order that he may not in ignorance do or say anything which is ridiculous and out of place—he should command slaves and hired strangers to imitate such things, but he should never take any serious interest in them himself, nor should any freeman or freewoman be discovered taking pains to learn them; and there should always be some element of novelty in the imitation. Let these then be laid down, both in law and in our discourse, as the regulations of laughable amusements which are generally called comedy. And, if any of the serious poets, as they are termed, who write tragedy, come to us and say—'O strangers, may we go to your city and country or may we not, and shall we bring with us our poetry—what is your will about these matters? '—how shall we answer the divine men? I think that our answer should be as follows: Best of strangers, we will say to them, we also according to our ability are tragic poets, and our tragedy is the best and noblest; for our whole state is an imitation of the best and noblest life, which we affirm to be indeed the very truth of tragedy. You are poets and we are poets, both makers of the same strains, rivals and antagonists in the noblest of dramas, which true law can alone perfect, as our hope is. Do not then suppose that we shall all in a moment allow you to erect your stage in the agora, or introduce the fair voices of your actors, speaking above our own, and permit you to harangue our women and children, and the common people, about our institutions, in language other than our own, and very often the opposite of our own. For a state would be mad which gave you this licence, until the magistrates had determined whether your poetry might be recited, and was fit for publication or not. Wherefore, O ye sons and scions of the softer Muses, first of all show your songs to the magistrates, and let them compare them with our own, and if they are the same or better we will give you a chorus; but if not, then, my friends, we cannot. Let these, then, be the customs ordained by law about all dances and the teaching of them, and let matters relating to slaves be separated from those relating to masters, if you do not object.

\par \textbf{CLEINIAS}
\par   We can have no hesitation in assenting when you put the matter thus.

\par \textbf{ATHENIAN}
\par   There still remain three studies suitable for freemen. Arithmetic is one of them; the measurement of length, surface, and depth is the second; and the third has to do with the revolutions of the stars in relation to one another. Not every one has need to toil through all these things in a strictly scientific manner, but only a few, and who they are to be we will hereafter indicate at the end, which will be the proper place; not to know what is necessary for mankind in general, and what is the truth, is disgraceful to every one:  and yet to enter into these matters minutely is neither easy, nor at all possible for every one; but there is something in them which is necessary and cannot be set aside, and probably he who made the proverb about God originally had this in view when he said, that 'not even God himself can fight against necessity;' he meant, if I am not mistaken, divine necessity; for as to the human necessities of which the many speak, when they talk in this manner, nothing can be more ridiculous than such an application of the words.

\par \textbf{CLEINIAS}
\par   And what necessities of knowledge are there, Stranger, which are divine and not human?

\par \textbf{ATHENIAN}
\par   I conceive them to be those of which he who has no use nor any knowledge at all cannot be a God, or demi-god, or hero to mankind, or able to take any serious thought or charge of them. And very unlike a divine man would he be, who is unable to count one, two, three, or to distinguish odd and even numbers, or is unable to count at all, or reckon night and day, and who is totally unacquainted with the revolution of the sun and moon, and the other stars. There would be great folly in supposing that all these are not necessary parts of knowledge to him who intends to know anything about the highest kinds of knowledge; but which these are, and how many there are of them, and when they are to be learned, and what is to be learned together and what apart, and the whole correlation of them, must be rightly apprehended first; and these leading the way we may proceed to the other parts of knowledge. For so necessity grounded in nature constrains us, against which we say that no God contends, or ever will contend.

\par \textbf{CLEINIAS}
\par   I think, Stranger, that what you have now said is very true and agreeable to nature.

\par \textbf{ATHENIAN}
\par   Yes, Cleinias, that is so. But it is difficult for the legislator to begin with these studies; at a more convenient time we will make regulations for them.

\par \textbf{CLEINIAS}
\par   You seem, Stranger, to be afraid of our habitual ignorance of the subject:  there is no reason why that should prevent you from speaking out.

\par \textbf{ATHENIAN}
\par   I certainly am afraid of the difficulties to which you allude, but I am still more afraid of those who apply themselves to this sort of knowledge, and apply themselves badly. For entire ignorance is not so terrible or extreme an evil, and is far from being the greatest of all; too much cleverness and too much learning, accompanied with an ill bringing up, are far more fatal.

\par \textbf{CLEINIAS}
\par   True.

\par \textbf{ATHENIAN}
\par   All freemen I conceive, should learn as much of these branches of knowledge as every child in Egypt is taught when he learns the alphabet. In that country arithmetical games have been invented for the use of mere children, which they learn as a pleasure and amusement. They have to distribute apples and garlands, using the same number sometimes for a larger and sometimes for a lesser number of persons; and they arrange pugilists and wrestlers as they pair together by lot or remain over, and show how their turns come in natural order. Another mode of amusing them is to distribute vessels, sometimes of gold, brass, silver, and the like, intermixed with one another, sometimes of one metal only; as I was saying they adapt to their amusement the numbers in common use, and in this way make more intelligible to their pupils the arrangements and movements of armies and expeditions, and in the management of a household they make people more useful to themselves, and more wide awake; and again in measurements of things which have length, and breadth, and depth, they free us from that natural ignorance of all these things which is so ludicrous and disgraceful.

\par \textbf{CLEINIAS}
\par   What kind of ignorance do you mean?

\par \textbf{ATHENIAN}
\par   O my dear Cleinias, I, like yourself, have late in life heard with amazement of our ignorance in these matters; to me we appear to be more like pigs than men, and I am quite ashamed, not only of myself, but of all Hellenes.

\par \textbf{CLEINIAS}
\par   About what? Say, Stranger, what you mean.

\par \textbf{ATHENIAN}
\par   I will; or rather I will show you my meaning by a question, and do you please to answer me:  You know, I suppose, what length is?

\par \textbf{CLEINIAS}
\par   Certainly.

\par \textbf{ATHENIAN}
\par   And what breadth is?

\par \textbf{CLEINIAS}
\par   To be sure.

\par \textbf{ATHENIAN}
\par   And you know that these are two distinct things, and that there is a third thing called depth?

\par \textbf{CLEINIAS}
\par   Of course.

\par \textbf{ATHENIAN}
\par   And do not all these seem to you to be commensurable with themselves?

\par \textbf{CLEINIAS}
\par   Yes.

\par \textbf{ATHENIAN}
\par   That is to say, length is naturally commensurable with length, and breadth with breadth, and depth in like manner with depth?

\par \textbf{CLEINIAS}
\par   Undoubtedly.

\par \textbf{ATHENIAN}
\par   But if some things are commensurable and others wholly incommensurable, and you think that all things are commensurable, what is your position in regard to them?

\par \textbf{CLEINIAS}
\par   Clearly, far from good.

\par \textbf{ATHENIAN}
\par   Concerning length and breadth when compared with depth, or breadth and length when compared with one another, are not all the Hellenes agreed that these are commensurable with one another in some way?

\par \textbf{CLEINIAS}
\par   Quite true.

\par \textbf{ATHENIAN}
\par   But if they are absolutely incommensurable, and yet all of us regard them as commensurable, have we not reason to be ashamed of our compatriots; and might we not say to them:  O ye best of Hellenes, is not this one of the things of which we were saying that not to know them is disgraceful, and of which to have a bare knowledge only is no great distinction?

\par \textbf{CLEINIAS}
\par   Certainly.

\par \textbf{ATHENIAN}
\par   And there are other things akin to these, in which there spring up other errors of the same family.

\par \textbf{CLEINIAS}
\par   What are they?

\par \textbf{ATHENIAN}
\par   The natures of commensurable and incommensurable quantities in their relation to one another. A man who is good for anything ought to be able, when he thinks, to distinguish them; and different persons should compete with one another in asking questions, which will be a far better and more graceful way of passing their time than the old man's game of draughts.

\par \textbf{CLEINIAS}
\par   I dare say; and these pastimes are not so very unlike a game of draughts.

\par \textbf{ATHENIAN}
\par   And these, as I maintain, Cleinias, are the studies which our youth ought to learn, for they are innocent and not difficult; the learning of them will be an amusement, and they will benefit the state. If any one is of another mind, let him say what he has to say.

\par \textbf{CLEINIAS}
\par   Certainly.

\par \textbf{ATHENIAN}
\par   Then if these studies are such as we maintain, we will include them; if not, they shall be excluded.

\par \textbf{CLEINIAS}
\par   Assuredly:  but may we not now, Stranger, prescribe these studies as necessary, and so fill up the lacunae of our laws?

\par \textbf{ATHENIAN}
\par   They shall be regarded as pledges which may be hereafter redeemed and removed from our state, if they do not please either us who give them, or you who accept them.

\par \textbf{CLEINIAS}
\par   A fair condition.

\par \textbf{ATHENIAN}
\par   Next let us see whether we are or are not willing that the study of astronomy shall be proposed for our youth.

\par \textbf{CLEINIAS}
\par   Proceed.

\par \textbf{ATHENIAN}
\par   Here occurs a strange phenomenon, which certainly cannot in any point of view be tolerated.

\par \textbf{CLEINIAS}
\par   To what are you referring?

\par \textbf{ATHENIAN}
\par   Men say that we ought not to enquire into the supreme God and the nature of the universe, nor busy ourselves in searching out the causes of things, and that such enquiries are impious; whereas the very opposite is the truth.

\par \textbf{CLEINIAS}
\par   What do you mean?

\par \textbf{ATHENIAN}
\par   Perhaps what I am saying may seem paradoxical, and at variance with the usual language of age. But when any one has any good and true notion which is for the advantage of the state and in every way acceptable to God, he cannot abstain from expressing it.

\par \textbf{CLEINIAS}
\par   Your words are reasonable enough; but shall we find any good or true notion about the stars?

\par \textbf{ATHENIAN}
\par   My good friends, at this hour all of us Hellenes tell lies, if I may use such an expression, about those great Gods, the Sun and the Moon.

\par \textbf{CLEINIAS}
\par   Lies of what nature?

\par \textbf{ATHENIAN}
\par   We say that they and divers other stars do not keep the same path, and we call them planets or wanderers.

\par \textbf{CLEINIAS}
\par   Very true, Stranger; and in the course of my life I have often myself seen the morning star and the evening star and divers others not moving in their accustomed course, but wandering out of their path in all manner of ways, and I have seen the sun and moon doing what we all know that they do.

\par \textbf{ATHENIAN}
\par   Just so, Megillus and Cleinias; and I maintain that our citizens and our youth ought to learn about the nature of the Gods in heaven, so far as to be able to offer sacrifices and pray to them in pious language, and not to blaspheme about them.

\par \textbf{CLEINIAS}
\par   There you are right, if such a knowledge be only attainable; and if we are wrong in our mode of speaking now, and can be better instructed and learn to use better language, then I quite agree with you that such a degree of knowledge as will enable us to speak rightly should be acquired by us. And now do you try to explain to us your whole meaning, and we, on our part, will endeavour to understand you.

\par \textbf{ATHENIAN}
\par   There is some difficulty in understanding my meaning, but not a very great one, nor will any great length of time be required. And of this I am myself a proof; for I did not know these things long ago, nor in the days of my youth, and yet I can explain them to you in a brief space of time; whereas if they had been difficult I could certainly never have explained them all, old as I am, to old men like yourselves.

\par \textbf{CLEINIAS}
\par   True; but what is this study which you describe as wonderful and fitting for youth to learn, but of which we are ignorant? Try and explain the nature of it to us as clearly as you can.

\par \textbf{ATHENIAN}
\par   I will. For, O my good friends, that other doctrine about the wandering of the sun and the moon and the other stars is not the truth, but the very reverse of the truth. Each of them moves in the same path—not in many paths, but in one only, which is circular, and the varieties are only apparent. Nor are we right in supposing that the swiftest of them is the slowest, nor conversely, that the slowest is the quickest. And if what I say is true, only just imagine that we had a similar notion about horses running at Olympia, or about men who ran in the long course, and that we addressed the swiftest as the slowest and the slowest as the swiftest, and sang the praises of the vanquished as though he were the victor—in that case our praises would not be true, nor very agreeable to the runners, though they be but men; and now, to commit the same error about the Gods which would have been ludicrous and erroneous in the case of men—is not that ludicrous and erroneous?

\par \textbf{CLEINIAS}
\par   Worse than ludicrous, I should say.

\par \textbf{ATHENIAN}
\par   At all events, the Gods cannot like us to be spreading a false report of them.

\par \textbf{CLEINIAS}
\par   Most true, if such is the fact.

\par \textbf{ATHENIAN}
\par   And if we can show that such is really the fact, then all these matters ought to be learned so far as is necessary for the avoidance of impiety; but if we cannot, they may be let alone, and let this be our decision.

\par \textbf{CLEINIAS}
\par   Very good.

\par \textbf{ATHENIAN}
\par   Enough of laws relating to education and learning. But hunting and similar pursuits in like manner claim our attention. For the legislator appears to have a duty imposed upon him which goes beyond mere legislation. There is something over and above law which lies in a region between admonition and law, and has several times occurred to us in the course of discussion; for example, in the education of very young children there were things, as we maintain, which are not to be defined, and to regard them as matters of positive law is a great absurdity. Now, our laws and the whole constitution of our state having been thus delineated, the praise of the virtuous citizen is not complete when he is described as the person who serves the laws best and obeys them most, but the higher form of praise is that which describes him as the good citizen who passes through life undefiled and is obedient to the words of the legislator, both when he is giving laws and when he assigns praise and blame. This is the truest word that can be spoken in praise of a citizen; and the true legislator ought not only to write his laws, but also to interweave with them all such things as seem to him honourable and dishonourable. And the perfect citizen ought to seek to strengthen these no less than the principles of law which are sanctioned by punishments. I will adduce an example which will clear up my meaning, and will be a sort of witness to my words. Hunting is of wide extent, and has a name under which many things are included, for there is a hunting of creatures in the water, and of creatures in the air, and there is a great deal of hunting of land animals of all kinds, and not of wild beasts only. The hunting after man is also worthy of consideration; there is the hunting after him in war, and there is often a hunting after him in the way of friendship, which is praised and also blamed; and there is thieving, and the hunting which is practised by robbers, and that of armies against armies. Now the legislator, in laying down laws about hunting, can neither abstain from noting these things, nor can he make threatening ordinances which will assign rules and penalties about all of them. What is he to do? He will have to praise and blame hunting with a view to the exercise and pursuits of youth. And, on the other hand, the young man must listen obediently; neither pleasure nor pain should hinder him, and he should regard as his standard of action the praises and injunctions of the legislator rather than the punishments which he imposes by law. This being premised, there will follow next in order moderate praise and censure of hunting; the praise being assigned to that kind which will make the souls of young men better, and the censure to that which has the opposite effect. And now let us address young men in the form of a prayer for their welfare:  O friends, we will say to them, may no desire or love of hunting in the sea, or of angling or of catching the creatures in the waters, ever take possession of you, either when you are awake or when you are asleep, by hook or with weels, which latter is a very lazy contrivance; and let not any desire of catching men and of piracy by sea enter into your souls and make you cruel and lawless hunters. And as to the desire of thieving in town or country, may it never enter into your most passing thoughts; nor let the insidious fancy of catching birds, which is hardly worthy of freemen, come into the head of any youth. There remains therefore for our athletes only the hunting and catching of land animals, of which the one sort is called hunting by night, in which the hunters sleep in turn and are lazy; this is not to be commended any more than that which has intervals of rest, in which the wild strength of beasts is subdued by nets and snares, and not by the victory of a laborious spirit. Thus, only the best kind of hunting is allowed at all—that of quadrupeds, which is carried on with horses and dogs and men's own persons, and they get the victory over the animals by running them down and striking them and hurling at them, those who have a care of godlike manhood taking them with their own hands. The praise and blame which is assigned to all these things has now been declared; and let the law be as follows:  Let no one hinder these who verily are sacred hunters from following the chase wherever and whithersoever they will; but the hunter by night, who trusts to his nets and gins, shall not be allowed to hunt anywhere. The fowler in the mountains and waste places shall be permitted, but on cultivated ground and on consecrated wilds he shall not be permitted; and any one who meets him may stop him. As to the hunter in waters, he may hunt anywhere except in harbours or sacred streams or marshes or pools, provided only that he do not pollute the water with poisonous juices. And now we may say that all our enactments about education are complete.

\par \textbf{CLEINIAS}
\par   Very good.

\par 
\section{
      BOOK VIII.
    }
\par \textbf{ATHENIAN}
\par   Next, with the help of the Delphian oracle, we have to institute festivals and make laws about them, and to determine what sacrifices will be for the good of the city, and to what Gods they shall be offered; but when they shall be offered, and how often, may be partly regulated by us.

\par \textbf{CLEINIAS}
\par   The number—yes.

\par \textbf{ATHENIAN}
\par   Then we will first determine the number; and let the whole number be 365—one for every day—so that one magistrate at least will sacrifice daily to some God or demi-god on behalf of the city, and the citizens, and their possessions. And the interpreters, and priests, and priestesses, and prophets shall meet, and, in company with the guardians of the law, ordain those things which the legislator of necessity omits; and I may remark that they are the very persons who ought to take note of what is omitted. The law will say that there are twelve feasts dedicated to the twelve Gods, after whom the several tribes are named; and that to each of them they shall sacrifice every month, and appoint choruses, and musical and gymnastic contests, assigning them so as to suit the Gods and seasons of the year. And they shall have festivals for women, distinguishing those which ought to be separated from the men's festivals, and those which ought not. Further, they shall not confuse the infernal deities and their rites with the Gods who are termed heavenly and their rites, but shall separate them, giving to Pluto his own in the twelfth month, which is sacred to him, according to the law. To such a deity warlike men should entertain no aversion, but they should honour him as being always the best friend of man. For the connexion of soul and body is no way better than the dissolution of them, as I am ready to maintain quite seriously. Moreover, those who would regulate these matters rightly should consider, that our city among existing cities has no fellow, either in respect of leisure or command of the necessaries of life, and that like an individual she ought to live happily. And those who would live happily should in the first place do no wrong to one another, and ought not themselves to be wronged by others; to attain the first is not difficult, but there is great difficulty in acquiring the power of not being wronged. No man can be perfectly secure against wrong, unless he has become perfectly good; and cities are like individuals in this, for a city if good has a life of peace, but if evil, a life of war within and without. Wherefore the citizens ought to practise war—not in time of war, but rather while they are at peace. And every city which has any sense, should take the field at least for one day in every month, and for more if the magistrates think fit, having no regard to winter cold or summer heat; and they should go out en masse, including their wives and their children, when the magistrates determine to lead forth the whole people, or in separate portions when summoned by them; and they should always provide that there should be games and sacrificial feasts, and they should have tournaments, imitating in as lively a manner as they can real battles. And they should distribute prizes of victory and valour to the competitors, passing censures and encomiums on one another according to the characters which they bear in the contests and in their whole life, honouring him who seems to be the best, and blaming him who is the opposite. And let poets celebrate the victors—not however every poet, but only one who in the first place is not less than fifty years of age; nor should he be one who, although he may have musical and poetical gifts, has never in his life done any noble or illustrious action; but those who are themselves good and also honourable in the state, creators of noble actions—let their poems be sung, even though they be not very musical. And let the judgment of them rest with the instructor of youth and the other guardians of the laws, who shall give them this privilege, and they alone shall be free to sing; but the rest of the world shall not have this liberty. Nor shall any one dare to sing a song which has not been approved by the judgment of the guardians of the laws, not even if his strain be sweeter than the songs of Thamyras and Orpheus; but only such poems as have been judged sacred and dedicated to the Gods, and such as are the works of good men, in which praise or blame has been awarded and which have been deemed to fulfil their design fairly.

\par  The regulations about war, and about liberty of speech in poetry, ought to apply equally to men and women. The legislator may be supposed to argue the question in his own mind: Who are my citizens for whom I have set in order the city? Are they not competitors in the greatest of all contests, and have they not innumerable rivals? To be sure, will be the natural reply. Well, but if we were training boxers, or pancratiasts, or any other sort of athletes, would they never meet until the hour of contest arrived; and should we do nothing to prepare ourselves previously by daily practice? Surely, if we were boxers, we should have been learning to fight for many days before, and exercising ourselves in imitating all those blows and wards which we were intending to use in the hour of conflict; and in order that we might come as near to reality as possible, instead of cestuses we should put on boxing-gloves, that the blows and the wards might be practised by us to the utmost of our power. And if there were a lack of competitors, the ridicule of fools would not deter us from hanging up a lifeless image and practising at that. Or if we had no adversary at all, animate or inanimate, should we not venture in the dearth of antagonists to spar by ourselves? In what other manner could we ever study the art of self-defence?

\par \textbf{CLEINIAS}
\par   The way which you mention, Stranger, would be the only way.

\par \textbf{ATHENIAN}
\par   And shall the warriors of our city, who are destined when occasion calls to enter the greatest of all contests, and to fight for their lives, and their children, and their property, and the whole city, be worse prepared than boxers? And will the legislator, because he is afraid that their practising with one another may appear to some ridiculous, abstain from commanding them to go out and fight; will he not ordain that soldiers shall perform lesser exercises without arms every day, making dancing and all gymnastic tend to this end; and also will he not require that they shall practise some gymnastic exercises, greater as well as lesser, as often as every month; and that they shall have contests one with another in every part of the country, seizing upon posts and lying in ambush, and imitating in every respect the reality of war; fighting with boxing-gloves and hurling javelins, and using weapons somewhat dangerous, and as nearly as possible like the true ones, in order that the sport may not be altogether without fear, but may have terrors and to a certain degree show the man who has and who has not courage; and that the honour and dishonour which are assigned to them respectively, may prepare the whole city for the true conflict of life? If any one dies in these mimic contests, the homicide is involuntary, and we will make the slayer, when he has been purified according to law, to be pure of blood, considering that if a few men should die, others as good as they will be born; but that if fear is dead, then the citizens will never find a test of superior and inferior natures, which is a far greater evil to the state than the loss of a few.

\par \textbf{CLEINIAS}
\par   We are quite agreed, Stranger, that we should legislate about such things, and that the whole state should practise them.

\par \textbf{ATHENIAN}
\par   And what is the reason that dances and contests of this sort hardly ever exist in states, at least not to any extent worth speaking of? Is this due to the ignorance of mankind and their legislators?

\par \textbf{CLEINIAS}
\par   Perhaps.

\par \textbf{ATHENIAN}
\par   Certainly not, sweet Cleinias; there are two causes, which are quite enough to account for the deficiency.

\par \textbf{CLEINIAS}
\par   What are they?

\par \textbf{ATHENIAN}
\par   One cause is the love of wealth, which wholly absorbs men, and never for a moment allows them to think of anything but their own private possessions; on this the soul of every citizen hangs suspended, and can attend to nothing but his daily gain; mankind are ready to learn any branch of knowledge, and to follow any pursuit which tends to this end, and they laugh at every other:  that is one reason why a city will not be in earnest about such contests or any other good and honourable pursuit. But from an insatiable love of gold and silver, every man will stoop to any art or contrivance, seemly or unseemly, in the hope of becoming rich; and will make no objection to performing any action, holy, or unholy and utterly base; if only like a beast he have the power of eating and drinking all kinds of things, and procuring for himself in every sort of way the gratification of his lusts.

\par \textbf{CLEINIAS}
\par   True.

\par \textbf{ATHENIAN}
\par   Let this, then, be deemed one of the causes which prevent states from pursuing in an efficient manner the art of war, or any other noble aim, but makes the orderly and temperate part of mankind into merchants, and captains of ships, and servants, and converts the valiant sort into thieves and burglars, and robbers of temples, and violent, tyrannical persons; many of whom are not without ability, but they are unfortunate.

\par \textbf{CLEINIAS}
\par   What do you mean?

\par \textbf{ATHENIAN}
\par   Must not they be truly unfortunate whose souls are compelled to pass through life always hungering?

\par \textbf{CLEINIAS}
\par   Then that is one cause, Stranger; but you spoke of another.

\par \textbf{ATHENIAN}
\par   Thank you for reminding me.

\par \textbf{CLEINIAS}
\par   The insatiable lifelong love of wealth, as you were saying, is one cause which absorbs mankind, and prevents them from rightly practising the arts of war:  Granted; and now tell me, what is the other?

\par \textbf{ATHENIAN}
\par   Do you imagine that I delay because I am in a perplexity?

\par \textbf{CLEINIAS}
\par   No; but we think that you are too severe upon the money-loving temper, of which you seem in the present discussion to have a peculiar dislike.

\par \textbf{ATHENIAN}
\par   That is a very fair rebuke, Cleinias; and I will now proceed to the second cause.

\par \textbf{CLEINIAS}
\par   Proceed.

\par \textbf{ATHENIAN}
\par   I say that governments are a cause—democracy, oligarchy, tyranny, concerning which I have often spoken in the previous discourse; or rather governments they are not, for none of them exercises a voluntary rule over voluntary subjects; but they may be truly called states of discord, in which while the government is voluntary, the subjects always obey against their will, and have to be coerced; and the ruler fears the subject, and will not, if he can help, allow him to become either noble, or rich, or strong, or valiant, or warlike at all. These two are the chief causes of almost all evils, and of the evils of which I have been speaking they are notably the causes. But our state has escaped both of them; for her citizens have the greatest leisure, and they are not subject to one another, and will, I think, be made by these laws the reverse of lovers of money. Such a constitution may be reasonably supposed to be the only one existing which will accept the education which we have described, and the martial pastimes which have been perfected according to our idea.

\par \textbf{CLEINIAS}
\par   True.

\par \textbf{ATHENIAN}
\par   Then next we must remember, about all gymnastic contests, that only the warlike sort of them are to be practised and to have prizes of victory; and those which are not military are to be given up. The military sort had better be completely described and established by law; and first, let us speak of running and swiftness.

\par \textbf{CLEINIAS}
\par   Very good.

\par \textbf{ATHENIAN}
\par   Certainly the most military of all qualities is general activity of body, whether of foot or hand. For escaping or for capturing an enemy, quickness of foot is required; but hand-to-hand conflict and combat need vigour and strength.

\par \textbf{CLEINIAS}
\par   Very true.

\par \textbf{ATHENIAN}
\par   Neither of them can attain their greatest efficiency without arms.

\par \textbf{CLEINIAS}
\par   How can they?

\par \textbf{ATHENIAN}
\par   Then our herald, in accordance with the prevailing practice, will first summon the runner—he will appear armed, for to an unarmed competitor we will not give a prize. And he shall enter first who is to run the single course bearing arms; next, he who is to run the double course; third, he who is to run the horse-course; and fourthly, he who is to run the long course; the fifth whom we start, shall be the first sent forth in heavy armour, and shall run a course of sixty stadia to some temple of Ares—and we will send forth another, whom we will style the more heavily armed, to run over smoother ground. There remains the archer; and he shall run in the full equipments of an archer a distance of 100 stadia over mountains, and across every sort of country, to a temple of Apollo and Artemis; this shall be the order of the contest, and we will wait for them until they return, and will give a prize to the conqueror in each.

\par \textbf{CLEINIAS}
\par   Very good.

\par \textbf{ATHENIAN}
\par   Let us suppose that there are three kinds of contests—one of boys, another of beardless youths, and a third of men. For the youths we will fix the length of the contest at two-thirds, and for the boys at half of the entire course, whether they contend as archers or as heavy-armed. Touching the women, let the girls who are not grown up compete naked in the stadium and the double course, and the horse-course and the long course, and let them run on the race-ground itself; those who are thirteen years of age and upwards until their marriage shall continue to share in contests if they are not more than twenty, and shall be compelled to run up to eighteen; and they shall descend into the arena in suitable dresses. Let these be the regulations about contests in running both for men and women.

\par  Respecting contests of strength, instead of wrestling and similar contests of the heavier sort, we will institute conflicts in armour of one against one, and two against two, and so on up to ten against ten. As to what a man ought not to suffer or do, and to what extent, in order to gain the victory—as in wrestling, the masters of the art have laid down what is fair and what is not fair, so in fighting in armour—we ought to call in skilful persons, who shall judge for us and be our assessors in the work of legislation; they shall say who deserves to be victor in combats of this sort, and what he is not to do or have done to him, and in like manner what rule determines who is defeated; and let these ordinances apply to women until they are married as well as to men. The pancration shall have a counterpart in a combat of the light-armed; they shall contend with bows and with light shields and with javelins and in the throwing of stones by slings and by hand: and laws shall be made about it, and rewards and prizes given to him who best fulfils the ordinances of the law.

\par  Next in order we shall have to legislate about the horse contests. Now we do not need many horses, for they cannot be of much use in a country like Crete, and hence we naturally do not take great pains about the rearing of them or about horse races. There is no one who keeps a chariot among us, and any rivalry in such matters would be altogether out of place; there would be no sense nor any shadow of sense in instituting contests which are not after the manner of our country. And therefore we give our prizes for single horses—for colts who have not yet cast their teeth, and for those who are intermediate, and for the full-grown horses themselves; and thus our equestrian games will accord with the nature of the country. Let them have conflict and rivalry in these matters in accordance with the law, and let the colonels and generals of horse decide together about all courses and about the armed competitors in them. But we have nothing to say to the unarmed either in gymnastic exercises or in these contests. On the other hand, the Cretan bowman or javelin-man who fights in armour on horseback is useful, and therefore we may as well place a competition of this sort among our amusements. Women are not to be forced to compete by laws and ordinances; but if from previous training they have acquired the habit and are strong enough and like to take part, let them do so, girls as well as boys, and no blame to them.

\par  Thus the competition in gymnastic and the mode of learning it have been described; and we have spoken also of the toils of the contest, and of daily exercises under the superintendence of masters. Likewise, what relates to music has been, for the most part, completed. But as to rhapsodes and the like, and the contests of choruses which are to perform at feasts, all this shall be arranged when the months and days and years have been appointed for Gods and demi-gods, whether every third year, or again every fifth year, or in whatever way or manner the Gods may put into men's minds the distribution and order of them. At the same time, we may expect that the musical contests will be celebrated in their turn by the command of the judges and the director of education and the guardians of the law meeting together for this purpose, and themselves becoming legislators of the times and nature and conditions of the choral contests and of dancing in general. What they ought severally to be in language and song, and in the admixture of harmony with rhythm and the dance, has been often declared by the original legislator; and his successors ought to follow him, making the games and sacrifices duly to correspond at fitting times, and appointing public festivals. It is not difficult to determine how these and the like matters may have a regular order; nor, again, will the alteration of them do any great good or harm to the state. There is, however, another matter of great importance and difficulty, concerning which God should legislate, if there were any possibility of obtaining from Him an ordinance about it. But seeing that divine aid is not to be had, there appears to be a need of some bold man who specially honours plainness of speech, and will say outright what he thinks best for the city and citizens—ordaining what is good and convenient for the whole state amid the corruptions of human souls, opposing the mightiest lusts, and having no man his helper but himself standing alone and following reason only.

\par \textbf{CLEINIAS}
\par   What is this, Stranger, that you are saying? For we do not as yet understand your meaning.

\par \textbf{ATHENIAN}
\par   Very likely; I will endeavour to explain myself more clearly. When I came to the subject of education, I beheld young men and maidens holding friendly intercourse with one another. And there naturally arose in my mind a sort of apprehension—I could not help thinking how one is to deal with a city in which youths and maidens are well nurtured, and have nothing to do, and are not undergoing the excessive and servile toils which extinguish wantonness, and whose only cares during their whole life are sacrifices and festivals and dances. How, in such a state as this, will they abstain from desires which thrust many a man and woman into perdition; and from which reason, assuming the functions of law, commands them to abstain? The ordinances already made may possibly get the better of most of these desires; the prohibition of excessive wealth is a very considerable gain in the direction of temperance, and the whole education of our youth imposes a law of moderation on them; moreover, the eye of the rulers is required always to watch over the young, and never to lose sight of them; and these provisions do, as far as human means can effect anything, exercise a regulating influence upon the desires in general. But how can we take precautions against the unnatural loves of either sex, from which innumerable evils have come upon individuals and cities? How shall we devise a remedy and way of escape out of so great a danger? Truly, Cleinias, here is a difficulty. In many ways Crete and Lacedaemon furnish a great help to those who make peculiar laws; but in the matter of love, as we are alone, I must confess that they are quite against us. For if any one following nature should lay down the law which existed before the days of Laius, and denounce these lusts as contrary to nature, adducing the animals as a proof that such unions were monstrous, he might prove his point, but he would be wholly at variance with the custom of your states. Further, they are repugnant to a principle which we say that a legislator should always observe; for we are always enquiring which of our enactments tends to virtue and which not. And suppose we grant that these loves are accounted by law to the honourable, or at least not disgraceful, in what degree will they contribute to virtue? Will such passions implant in the soul of him who is seduced the habit of courage, or in the soul of the seducer the principle of temperance? Who will ever believe this? or rather, who will not blame the effeminacy of him who yields to pleasures and is unable to hold out against them? Will not all men censure as womanly him who imitates the woman? And who would ever think of establishing such a practice by law? certainly no one who had in his mind the image of true law. How can we prove that what I am saying is true? He who would rightly consider these matters must see the nature of friendship and desire, and of these so-called loves, for they are of two kinds, and out of the two arises a third kind, having the same name; and this similarity of name causes all the difficulty and obscurity.

\par \textbf{CLEINIAS}
\par   How is that?

\par \textbf{ATHENIAN}
\par   Dear is the like in virtue to the like, and the equal to the equal; dear also, though unlike, is he who has abundance to him who is in want. And when either of these friendships becomes excessive, we term the excess love.

\par \textbf{CLEINIAS}
\par   Very true.

\par \textbf{ATHENIAN}
\par   The friendship which arises from contraries is horrible and coarse, and has often no tie of communion; but that which arises from likeness is gentle, and has a tie of communion which lasts through life. As to the mixed sort which is made up of them both, there is, first of all, a difficulty in determining what he who is possessed by this third love desires; moreover, he is drawn different ways, and is in doubt between the two principles; the one exhorting him to enjoy the beauty of youth, and the other forbidding him. For the one is a lover of the body, and hungers after beauty, like ripe fruit, and would fain satisfy himself without any regard to the character of the beloved; the other holds the desire of the body to be a secondary matter, and looking rather than loving and with his soul desiring the soul of the other in a becoming manner, regards the satisfaction of the bodily love as wantonness; he reverences and respects temperance and courage and magnanimity and wisdom, and wishes to live chastely with the chaste object of his affection. Now the sort of love which is made up of the other two is that which we have described as the third. Seeing then that there are these three sorts of love, ought the law to prohibit and forbid them all to exist among us? Is it not rather clear that we should wish to have in the state the love which is of virtue and which desires the beloved youth to be the best possible; and the other two, if possible, we should hinder? What do you say, friend Megillus?

\par \textbf{MEGILLUS}
\par   I think, Stranger, that you are perfectly right in what you have been now saying.

\par  Athenian: I knew well, my friend, that I should obtain your assent, which I accept, and therefore have no need to analyze your custom any further. Cleinias shall be prevailed upon to give me his assent at some other time. Enough of this; and now let us proceed to the laws.

\par \textbf{MEGILLUS}
\par   Very good.

\par \textbf{ATHENIAN}
\par   Upon reflection I see a way of imposing the law, which, in one respect, is easy, but, in another, is of the utmost difficulty.

\par \textbf{MEGILLUS}
\par   What do you mean?

\par \textbf{ATHENIAN}
\par   We are all aware that most men, in spite of their lawless natures, are very strictly and precisely restrained from intercourse with the fair, and this is not at all against their will, but entirely with their will.

\par \textbf{MEGILLUS}
\par   When do you mean?

\par \textbf{ATHENIAN}
\par   When any one has a brother or sister who is fair; and about a son or daughter the same unwritten law holds, and is a most perfect safeguard, so that no open or secret connexion ever takes place between them. Nor does the thought of such a thing ever enter at all into the minds of most of them.

\par \textbf{MEGILLUS}
\par   Very true.

\par \textbf{ATHENIAN}
\par   Does not a little word extinguish all pleasures of that sort?

\par \textbf{MEGILLUS}
\par   What word?

\par \textbf{ATHENIAN}
\par   The declaration that they are unholy, hated of God, and most infamous; and is not the reason of this that no one has ever said the opposite, but every one from his earliest childhood has heard men speaking in the same manner about them always and everywhere, whether in comedy or in the graver language of tragedy? When the poet introduces on the stage a Thyestes or an Oedipus, or a Macareus having secret intercourse with his sister, he represents him, when found out, ready to kill himself as the penalty of his sin.

\par \textbf{MEGILLUS}
\par   You are very right in saying that tradition, if no breath of opposition ever assails it, has a marvellous power.

\par \textbf{ATHENIAN}
\par   Am I not also right in saying that the legislator who wants to master any of the passions which master man may easily know how to subdue them? He will consecrate the tradition of their evil character among all, slaves and freemen, women and children, throughout the city:  that will be the surest foundation of the law which he can make.

\par \textbf{MEGILLUS}
\par   Yes; but will he ever succeed in making all mankind use the same language about them?

\par \textbf{ATHENIAN}
\par   A good objection; but was I not just now saying that I had a way to make men use natural love and abstain from unnatural, not intentionally destroying the seeds of human increase, or sowing them in stony places, in which they will take no root; and that I would command them to abstain too from any female field of increase in which that which is sown is not likely to grow? Now if a law to this effect could only be made perpetual, and gain an authority such as already prevents intercourse of parents and children—such a law, extending to other sensual desires, and conquering them, would be the source of ten thousand blessings. For, in the first place, moderation is the appointment of nature, and deters men from all frenzy and madness of love, and from all adulteries and immoderate use of meats and drinks, and makes them good friends to their own wives. And innumerable other benefits would result if such a law could only be enforced. I can imagine some lusty youth who is standing by, and who, on hearing this enactment, declares in scurrilous terms that we are making foolish and impossible laws, and fills the world with his outcry. And therefore I said that I knew a way of enacting and perpetuating such a law, which was very easy in one respect, but in another most difficult. There is no difficulty in seeing that such a law is possible, and in what way; for, as I was saying, the ordinance once consecrated would master the soul of every man, and terrify him into obedience. But matters have now come to such a pass that even then the desired result seems as if it could not be attained, just as the continuance of an entire state in the practice of common meals is also deemed impossible. And although this latter is partly disproven by the fact of their existence among you, still even in your cities the common meals of women would be regarded as unnatural and impossible. I was thinking of the rebelliousness of the human heart when I said that the permanent establishment of these things is very difficult.

\par \textbf{MEGILLUS}
\par   Very true.

\par \textbf{ATHENIAN}
\par   Shall I try and find some sort of persuasive argument which will prove to you that such enactments are possible, and not beyond human nature?

\par \textbf{CLEINIAS}
\par   By all means.

\par \textbf{ATHENIAN}
\par   Is a man more likely to abstain from the pleasures of love and to do what he is bidden about them, when his body is in a good condition, or when he is in an ill condition, and out of training?

\par \textbf{CLEINIAS}
\par   He will be far more temperate when he is in training.

\par \textbf{ATHENIAN}
\par   And have we not heard of Iccus of Tarentum, who, with a view to the Olympic and other contests, in his zeal for his art, and also because he was of a manly and temperate disposition, never had any connexion with a woman or a youth during the whole time of his training? And the same is said of Crison and Astylus and Diopompus and many others; and yet, Cleinias, they were far worse educated in their minds than your and my citizens, and in their bodies far more lusty.

\par \textbf{CLEINIAS}
\par   No doubt this fact has been often affirmed positively by the ancients of these athletes.

\par \textbf{ATHENIAN}
\par   And had they the courage to abstain from what is ordinarily deemed a pleasure for the sake of a victory in wrestling, running, and the like; and shall our young men be incapable of a similar endurance for the sake of a much nobler victory, which is the noblest of all, as from their youth upwards we will tell them, charming them, as we hope, into the belief of this by tales and sayings and songs?

\par \textbf{CLEINIAS}
\par   Of what victory are you speaking?

\par \textbf{ATHENIAN}
\par   Of the victory over pleasure, which if they win, they will live happily; or if they are conquered, the reverse of happily. And, further, may we not suppose that the fear of impiety will enable them to master that which other inferior people have mastered?

\par \textbf{CLEINIAS}
\par   I dare say.

\par \textbf{ATHENIAN}
\par   And since we have reached this point in our legislation, and have fallen into a difficulty by reason of the vices of mankind, I affirm that our ordinance should simply run in the following terms:  Our citizens ought not to fall below the nature of birds and beasts in general, who are born in great multitudes, and yet remain until the age for procreation virgin and unmarried, but when they have reached the proper time of life are coupled, male and female, and lovingly pair together, and live the rest of their lives in holiness and innocence, abiding firmly in their original compact:  surely, we will say to them, you should be better than the animals. But if they are corrupted by the other Hellenes and the common practice of barbarians, and they see with their eyes and hear with their ears of the so-called free love everywhere prevailing among them, and they themselves are not able to get the better of the temptation, the guardians of the law, exercising the functions of lawgivers, shall devise a second law against them.

\par \textbf{CLEINIAS}
\par   And what law would you advise them to pass if this one failed?

\par \textbf{ATHENIAN}
\par   Clearly, Cleinias, the one which would naturally follow.

\par \textbf{CLEINIAS}
\par   What is that?

\par \textbf{ATHENIAN}
\par   Our citizens should not allow pleasures to strengthen with indulgence, but should by toil divert the aliment and exuberance of them into other parts of the body; and this will happen if no immodesty be allowed in the practice of love. Then they will be ashamed of frequent intercourse, and they will find pleasure, if seldom enjoyed, to be a less imperious mistress. They should not be found out doing anything of the sort. Concealment shall be honourable, and sanctioned by custom and made law by unwritten prescription; on the other hand, to be detected shall be esteemed dishonourable, but not, to abstain wholly. In this way there will be a second legal standard of honourable and dishonourable, involving a second notion of right. Three principles will comprehend all those corrupt natures whom we call inferior to themselves, and who form but one class, and will compel them not to transgress.

\par \textbf{CLEINIAS}
\par   What are they?

\par \textbf{ATHENIAN}
\par   The principle of piety, the love of honour, and the desire of beauty, not in the body but in the soul. These are, perhaps, romantic aspirations; but they are the noblest of aspirations, if they could only be realised in all states, and, God willing, in the matter of love we may be able to enforce one of two things—either that no one shall venture to touch any person of the freeborn or noble class except his wedded wife, or sow the unconsecrated and bastard seed among harlots, or in barren and unnatural lusts; or at least we may abolish altogether the connection of men with men; and as to women, if any man has to do with any but those who come into his house duly married by sacred rites, whether they be bought or acquired in any other way, and he offends publicly in the face of all mankind, we shall be right in enacting that he be deprived of civic honours and privileges, and be deemed to be, as he truly is, a stranger. Let this law, then, whether it is one, or ought rather to be called two, be laid down respecting love in general, and the intercourse of the sexes which arises out of the desires, whether rightly or wrongly indulged.

\par \textbf{MEGILLUS}
\par   I, for my part, Stranger, would gladly receive this law. Cleinias shall speak for himself, and tell you what is his opinion.

\par \textbf{CLEINIAS}
\par   I will, Megillus, when an opportunity offers; at present, I think that we had better allow the Stranger to proceed with his laws.

\par \textbf{MEGILLUS}
\par   Very good.

\par \textbf{ATHENIAN}
\par   We had got about as far as the establishment of the common tables, which in most places would be difficult, but in Crete no one would think of introducing any other custom. There might arise a question about the manner of them—whether they shall be such as they are here in Crete, or such as they are in Lacedaemon—or is there a third kind which may be better than either of them? The answer to this question might be easily discovered, but the discovery would do no great good, for at present they are very well ordered.

\par  Leaving the common tables, we may therefore proceed to the means of providing food. Now, in cities the means of life are gained in many ways and from divers sources, and in general from two sources, whereas our city has only one. For most of the Hellenes obtain their food from sea and land, but our citizens from land only. And this makes the task of the legislator less difficult—half as many laws will be enough, and much less than half; and they will be of a kind better suited to free men. For he has nothing to do with laws about shipowners and merchants and retailers and inn-keepers and tax collectors and mines and moneylending and compound interest and innumerable other things—bidding good-bye to these, he gives laws to husbandmen and shepherds and bee-keepers, and to the guardians and superintendents of their implements; and he has already legislated for greater matters, as for example, respecting marriage and the procreation and nurture of children, and for education, and the establishment of offices—and now he must direct his laws to those who provide food and labour in preparing it.

\par  Let us first of all, then, have a class of laws which shall be called the laws of husbandmen. And let the first of them be the law of Zeus, the God of boundaries. Let no one shift the boundary line either of a fellow-citizen who is a neighbour, or, if he dwells at the extremity of the land, of any stranger who is conterminous with him, considering that this is truly 'to move the immovable,' and every one should be more willing to move the largest rock which is not a landmark, than the least stone which is the sworn mark of friendship and hatred between neighbours; for Zeus, the god of kindred, is the witness of the citizen, and Zeus, the god of strangers, of the stranger, and when aroused, terrible are the wars which they stir up. He who obeys the law will never know the fatal consequences of disobedience, but he who despises the law shall be liable to a double penalty, the first coming from the Gods, and the second from the law. For let no one wilfully remove the boundaries of his neighbour's land, and if any one does, let him who will inform the landowners, and let them bring him into court, and if he be convicted of re-dividing the land by stealth or by force, let the court determine what he ought to suffer or pay. In the next place, many small injuries done by neighbours to one another, through their multiplication, may cause a weight of enmity, and make neighbourhood a very disagreeable and bitter thing. Wherefore a man ought to be very careful of committing any offence against his neighbour, and especially of encroaching on his neighbour's land; for any man may easily do harm, but not every man can do good to another. He who encroaches on his neighbour's land, and transgresses his boundaries, shall make good the damage, and, to cure him of his impudence and also of his meanness, he shall pay a double penalty to the injured party. Of these and the like matters the wardens of the country shall take cognizance, and be the judges of them and assessors of the damage; in the more important cases, as has been already said, the whole number of them belonging to any one of the twelve divisions shall decide, and in the lesser cases the commanders: or, again, if any one pastures his cattle on his neighbour's land, they shall see the injury, and adjudge the penalty. And if any one, by decoying the bees, gets possession of another's swarms, and draws them to himself by making noises, he shall pay the damage; or if any one sets fire to his own wood and takes no care of his neighbour's property, he shall be fined at the discretion of the magistrates. And if in planting he does not leave a fair distance between his own and his neighbour's land, he shall be punished, in accordance with the enactments of many lawgivers, which we may use, not deeming it necessary that the great legislator of our state should determine all the trifles which might be decided by any body; for example, husbandmen have had of old excellent laws about waters, and there is no reason why we should propose to divert their course: He who likes may draw water from the fountain-head of the common stream on to his own land, if he do not cut off the spring which clearly belongs to some other owner; and he may take the water in any direction which he pleases, except through a house or temple or sepulchre, but he must be careful to do no harm beyond the channel. And if there be in any place a natural dryness of the earth, which keeps in the rain from heaven, and causes a deficiency in the supply of water, let him dig down on his own land as far as the clay, and if at this depth he finds no water, let him obtain water from his neighbours, as much as is required for his servants' drinking, and if his neighbours, too, are limited in their supply, let him have a fixed measure, which shall be determined by the wardens of the country. This he shall receive each day, and on these terms have a share of his neighbours' water. If there be heavy rain, and one of those on the lower ground injures some tiller of the upper ground, or some one who has a common wall, by refusing to give them an outlet for water; or, again, if some one living on the higher ground recklessly lets off the water on his lower neighbour, and they cannot come to terms with one another, let him who will call in a warden of the city, if he be in the city, or if he be in the country, a warden of the country, and let him obtain a decision determining what each of them is to do. And he who will not abide by the decision shall suffer for his malignant and morose temper, and pay a fine to the injured party, equivalent to double the value of the injury, because he was unwilling to submit to the magistrates.

\par  Now the participation of fruits shall be ordered on this wise. The goddess of Autumn has two gracious gifts: one the joy of Dionysus which is not treasured up; the other, which nature intends to be stored. Let this be the law, then, concerning the fruits of autumn: He who tastes the common or storing fruits of autumn, whether grapes or figs, before the season of vintage which coincides with Arcturus, either on his own land or on that of others—let him pay fifty drachmae, which shall be sacred to Dionysus, if he pluck them from his own land; and if from his neighbour's land, a mina, and if from any others', two-thirds of a mina. And he who would gather the 'choice' grapes or the 'choice' figs, as they are now termed, if he take them off his own land, let him pluck them how and when he likes; but if he take them from the ground of others without their leave, let him in that case be always punished in accordance with the law which ordains that he should not move what he has not laid down. And if a slave touches any fruit of this sort, without the consent of the owner of the land, he shall be beaten with as many blows as there are grapes on the bunch, or figs on the fig-tree. Let a metic purchase the 'choice' autumnal fruit, and then, if he pleases, he may gather it; but if a stranger is passing along the road, and desires to eat, let him take of the 'choice' grape for himself and a single follower without payment, as a tribute of hospitality. The law however forbids strangers from sharing in the sort which is not used for eating; and if any one, whether he be master or slave, takes of them in ignorance, let the slave be beaten, and the freeman dismissed with admonitions, and instructed to take of the other autumnal fruits which are unfit for making raisins and wine, or for laying by as dried figs. As to pears, and apples, and pomegranates, and similar fruits, there shall be no disgrace in taking them secretly; but he who is caught, if he be of less than thirty years of age, shall be struck and beaten off, but not wounded; and no freeman shall have any right of satisfaction for such blows. Of these fruits the stranger may partake, just as he may of the fruits of autumn. And if an elder, who is more than thirty years of age, eat of them on the spot, let him, like the stranger, be allowed to partake of all such fruits, but he must carry away nothing. If, however, he will not obey the law, let him run the risk of failing in the competition of virtue, in case any one takes notice of his actions before the judges at the time.

\par  Water is the greatest element of nutrition in gardens, but is easily polluted. You cannot poison the soil, or the sun, or the air, which are the other elements of nutrition in plants, or divert them, or steal them; but all these things may very likely happen in regard to water, which must therefore be protected by law. And let this be the law: If any one intentionally pollutes the water of another, whether the water of a spring, or collected in reservoirs, either by poisonous substances, or by digging, or by theft, let the injured party bring the cause before the wardens of the city, and claim in writing the value of the loss; if the accused be found guilty of injuring the water by deleterious substances, let him not only pay damages, but purify the stream or the cistern which contains the water, in such manner as the laws of the interpreters order the purification to be made by the offender in each case.

\par  With respect to the gathering in of the fruits of the soil, let a man, if he pleases, carry his own fruits through any place in which he either does no harm to any one, or himself gains three times as much as his neighbour loses. Now of these things the magistrates should be cognizant, as of all other things in which a man intentionally does injury to another or to the property of another, by fraud or force, in the use which he makes of his own property. All these matters a man should lay before the magistrates, and receive damages, supposing the injury to be not more than three minae; or if he have a charge against another which involves a larger amount, let him bring his suit into the public courts and have the evil-doer punished. But if any of the magistrates appear to adjudge the penalties which he imposes in an unjust spirit, let him be liable to pay double to the injured party. Any one may bring the offences of magistrates, in any particular case, before the public courts. There are innumerable little matters relating to the modes of punishment, and applications for suits, and summonses and the witnesses to summonses—for example, whether two witnesses should be required for a summons, or how many—and all such details, which cannot be omitted in legislation, but are beneath the wisdom of an aged legislator. These lesser matters, as they indeed are in comparison with the greater ones, let a younger generation regulate by law, after the patterns which have preceded, and according to their own experience of the usefulness and necessity of such laws; and when they are duly regulated let there be no alteration, but let the citizens live in the observance of them.

\par  Now of artisans, let the regulations be as follows: In the first place, let no citizen or servant of a citizen be occupied in handicraft arts; for he who is to secure and preserve the public order of the state, has an art which requires much study and many kinds of knowledge, and does not admit of being made a secondary occupation; and hardly any human being is capable of pursuing two professions or two arts rightly, or of practising one art himself, and superintending some one else who is practising another. Let this, then, be our first principle in the state: No one who is a smith shall also be a carpenter, and if he be a carpenter, he shall not superintend the smith's art rather than his own, under the pretext that in superintending many servants who are working for him, he is likely to superintend them better, because more revenue will accrue to him from them than from his own art; but let every man in the state have one art, and get his living by that. Let the wardens of the city labour to maintain this law, and if any citizen incline to any other art rather than the study of virtue, let them punish him with disgrace and infamy, until they bring him back into his own right course; and if any stranger profess two arts, let them chastise him with bonds and money penalties, and expulsion from the state, until they compel him to be one only and not many.

\par  But as touching payments for hire, and contracts of work, or in case any one does wrong to any of the citizens, or they do wrong to any other, up to fifty drachmae, let the wardens of the city decide the case; but if a greater amount be involved, then let the public courts decide according to law. Let no one pay any duty either on the importation or exportation of goods; and as to frankincense and similar perfumes, used in the service of the Gods, which come from abroad, and purple and other dyes which are not produced in the country, or the materials of any art which have to be imported, and which are not necessary—no one should import them; nor, again, should any one export anything which is wanted in the country. Of all these things let there be inspectors and superintendents, taken from the guardians of the law; and they shall be the twelve next in order to the five seniors. Concerning arms, and all implements which are required for military purposes, if there be need of introducing any art, or plant, or metal, or chains of any kind, or animals for use in war, let the commanders of the horse and the generals have authority over their importation and exportation; the city shall send them out and also receive them, and the guardians of the law shall make fit and proper laws about them. But let there be no retail trade for the sake of moneymaking, either in these or any other articles, in the city or country at all.

\par  With respect to food and the distribution of the produce of the country, the right and proper way seems to be nearly that which is the custom of Crete; for all should be required to distribute the fruits of the soil into twelve parts, and in this way consume them. Let the twelfth portion of each as for instance of wheat and barley, to which the rest of the fruits of the earth shall be added, as well as the animals which are for sale in each of the twelve divisions, be divided in due proportion into three parts; one part for freemen, another for their servants, and a third for craftsmen and in general for strangers, whether sojourners who may be dwelling in the city, and like other men must live, or those who come on some business which they have with the state, or with some individual. Let only this third part of all necessaries be required to be sold; out of the other two-thirds no one shall be compelled to sell. And how will they be best distributed? In the first place, we see clearly that the distribution will be of equals in one point of view, and in another point of view of unequals.

\par \textbf{CLEINIAS}
\par   What do you mean?

\par \textbf{ATHENIAN}
\par   I mean that the earth of necessity produces and nourishes the various articles of food, sometimes better and sometimes worse.

\par \textbf{CLEINIAS}
\par   Of course.

\par \textbf{ATHENIAN}
\par   Such being the case, let no one of the three portions be greater than either of the other two—neither that which is assigned to masters or to slaves, nor again that of the stranger; but let the distribution to all be equal and alike, and let every citizen take his two portions and distribute them among slaves and freemen, he having power to determine the quantity and quality. And what remains he shall distribute by measure and number among the animals who have to be sustained from the earth, taking the whole number of them.

\par  In the second place, our citizens should have separate houses duly ordered; and this will be the order proper for men like them. There shall be twelve hamlets, one in the middle of each twelfth portion, and in each hamlet they shall first set apart a market-place, and the temples of the Gods, and of their attendant demi-gods; and if there be any local deities of the Magnetes, or holy seats of other ancient deities, whose memory has been preserved, to these let them pay their ancient honours. But Hestia, and Zeus, and Athene will have temples everywhere together with the God who presides in each of the twelve districts. And the first erection of houses shall be around these temples, where the ground is highest, in order to provide the safest and most defensible place of retreat for the guards. All the rest of the country they shall settle in the following manner: They shall make thirteen divisions of the craftsmen; one of them they shall establish in the city, and this, again, they shall subdivide into twelve lesser divisions, among the twelve districts of the city, and the remainder shall be distributed in the country round about; and in each village they shall settle various classes of craftsmen, with a view to the convenience of the husbandmen. And the chief officers of the wardens of the country shall superintend all these matters, and see how many of them, and which class of them, each place requires; and fix them where they are likely to be least troublesome, and most useful to the husbandman. And the wardens of the city shall see to similar matters in the city.

\par  Now the wardens of the agora ought to see to the details of the agora. Their first care, after the temples which are in the agora have been seen to, should be to prevent any one from doing any wrong in dealings between man and man; in the second place, as being inspectors of temperance and violence, they should chastise him who requires chastisement. Touching articles of sale, they should first see whether the articles which the citizens are under regulations to sell to strangers are sold to them, as the law ordains. And let the law be as follows: On the first day of the month, the persons in charge, whoever they are, whether strangers or slaves, who have the charge on behalf of the citizens, shall produce to the strangers the portion which falls to them, in the first place, a twelfth portion of the corn—the stranger shall purchase corn for the whole month, and other cereals, on the first market day; and on the tenth day of the month the one party shall sell, and the other buy, liquids sufficient to last during the whole month; and on the twenty-third day there shall be a sale of animals by those who are willing to sell to the people who want to buy, and of implements and other things which husbandmen sell, (such as skins and all kinds of clothing, either woven or made of felt and other goods of the same sort) and which strangers are compelled to buy and purchase of others. As to the retail trade in these things, whether of barley or wheat set apart for meal and flour, or any other kind of food, no one shall sell them to citizens or their slaves, nor shall any one buy of a citizen; but let the stranger sell them in the market of strangers, to artisans and their slaves, making an exchange of wine and food, which is commonly called retail trade. And butchers shall offer for sale parts of dismembered animals to the strangers, and artisans, and their servants. Let any stranger who likes buy fuel from day to day wholesale, from those who have the care of it in the country, and let him sell to the strangers as much as he pleases and when he pleases. As to other goods and implements which are likely to be wanted, they shall sell them in the common market, at any place which the guardians of the law and the wardens of the market and city, choosing according to their judgment, shall determine; at such places they shall exchange money for goods, and goods for money, neither party giving credit to the other; and he who gives credit must be satisfied, whether he obtain his money or not, for in such exchanges he will not be protected by law. But whenever property has been bought or sold, greater in quantity or value than is allowed by the law, which has determined within what limits a man may increase and diminish his possessions, let the excess be registered in the books of the guardians of the law; or in case of diminution, let there be an erasure made. And let the same rule be observed about the registration of the property of the metics. Any one who likes may come and be a metic on certain conditions; a foreigner, if he likes, and is able to settle, may dwell in the land, but he must practise an art, and not abide more than twenty years from the time at which he has registered himself; and he shall pay no sojourner's tax, however small, except good conduct, nor any other tax for buying and selling. But when the twenty years have expired, he shall take his property with him and depart. And if in the course of these years he should chance to distinguish himself by any considerable benefit which he confers on the state, and he thinks that he can persuade the council and assembly, either to grant him delay in leaving the country, or to allow him to remain for the whole of his life, let him go and persuade the city, and whatever they assent to at his instance shall take effect. For the children of the metics, being artisans, and of fifteen years of age, let the time of their sojourn commence after their fifteenth year; and let them remain for twenty years, and then go where they like; but any of them who wishes to remain, may do so, if he can persuade the council and assembly. And if he depart, let him erase all the entries which have been made by him in the register kept by the magistrates.

\par 
\section{
      BOOK IX.
    }
\par  Next to all the matters which have preceded in the natural order of legislation will come suits of law. Of suits those which relate to agriculture have been already described, but the more important have not been described. Having mentioned them severally under their usual names, we will proceed to say what punishments are to be inflicted for each offence, and who are to be the judges of them.

\par \textbf{CLEINIAS}
\par   Very good.

\par \textbf{ATHENIAN}
\par   There is a sense of disgrace in legislating, as we are about to do, for all the details of crime in a state which, as we say, is to be well regulated and will be perfectly adapted to the practice of virtue. To assume that in such a state there will arise some one who will be guilty of crimes as heinous as any which are ever perpetrated in other states, and that we must legislate for him by anticipation, and threaten and make laws against him if he should arise, in order to deter him, and punish his acts, under the idea that he will arise—this, as I was saying, is in a manner disgraceful. Yet seeing that we are not like the ancient legislators, who gave laws to heroes and sons of gods, being, according to the popular belief, themselves the offspring of the gods, and legislating for others, who were also the children of divine parents, but that we are only men who are legislating for the sons of men, there is no uncharitableness in apprehending that some one of our citizens may be like a seed which has touched the ox's horn, having a heart so hard that it cannot be softened any more than those seeds can be softened by fire. Among our citizens there may be those who cannot be subdued by all the strength of the laws; and for their sake, though an ungracious task, I will proclaim my first law about the robbing of temples, in case any one should dare to commit such a crime. I do not expect or imagine that any well-brought-up citizen will ever take the infection, but their servants, and strangers, and strangers' servants may be guilty of many impieties. And with a view to them especially, and yet not without a provident eye to the weakness of human nature generally, I will proclaim the law about robbers of temples and similar incurable, or almost incurable, criminals. Having already agreed that such enactments ought always to have a short prelude, we may speak to the criminal, whom some tormenting desire by night and by day tempts to go and rob a temple, the fewest possible words of admonition and exhortation:  O sir, we will say to him, the impulse which moves you to rob temples is not an ordinary human malady, nor yet a visitation of heaven, but a madness which is begotten in a man from ancient and unexpiated crimes of his race, an ever-recurring curse—against this you must guard with all your might, and how you are to guard we will explain to you. When any such thought comes into your mind, go and perform expiations, go as a suppliant to the temples of the Gods who avert evils, go to the society of those who are called good men among you; hear them tell and yourself try to repeat after them, that every man should honour the noble and the just. Fly from the company of the wicked—fly and turn not back; and if your disorder is lightened by these remedies, well and good, but if not, then acknowledge death to be nobler than life, and depart hence.

\par  Such are the preludes which we sing to all who have thoughts of unholy and treasonable actions, and to him who hearkens to them the law has nothing to say. But to him who is disobedient when the prelude is over, cry with a loud voice—He who is taken in the act of robbing temples, if he be a slave or stranger, shall have his evil deed engraven on his face and hands, and shall be beaten with as many stripes as may seem good to the judges, and be cast naked beyond the borders of the land. And if he suffers this punishment he will probably return to his right mind and be improved; for no penalty which the law inflicts is designed for evil, but always makes him who suffers either better or not so much worse as he would have been. But if any citizen be found guilty of any great or unmentionable wrong, either in relation to the Gods, or his parents, or the state, let the judge deem him to be incurable, remembering that after receiving such an excellent education and training from youth upward, he has not abstained from the greatest of crimes. His punishment shall be death, which to him will be the least of evils; and his example will benefit others, if he perish ingloriously, and be cast beyond the borders of the land. But let his children and family, if they avoid the ways of their father, have glory, and let honourable mention be made of them, as having nobly and manfully escaped out of evil into good. None of them should have their goods confiscated to the state, for the lots of the citizens ought always to continue the same and equal.

\par  Touching the exaction of penalties, when a man appears to have done anything which deserves a fine, he shall pay the fine, if he have anything in excess of the lot which is assigned to him; but more than that he shall not pay. And to secure exactness, let the guardians of the law refer to the registers, and inform the judges of the precise truth, in order that none of the lots may go uncultivated for want of money. But if any one seems to deserve a greater penalty, let him undergo a long and public imprisonment and be dishonoured, unless some of his friends are willing to be surety for him, and liberate him by assisting him to pay the fine. No criminal shall go unpunished, not even for a single offence, nor if he have fled the country; but let the penalty be according to his deserts—death, or bonds, or blows, or degrading places of sitting or standing, or removal to some temple on the borders of the land; or let him pay fines, as we said before. In cases of death, let the judges be the guardians of the law, and a court selected by merit from the last year's magistrates. But how the causes are to be brought into court, how the summonses are to be served, and the like, these things may be left to the younger generation of legislators to determine; the manner of voting we must determine ourselves.

\par  Let the vote be given openly; but before they come to the vote let the judges sit in order of seniority over against plaintiff and defendant, and let all the citizens who can spare time hear and take a serious interest in listening to such causes. First of all the plaintiff shall make one speech, and then the defendant shall make another; and after the speeches have been made the eldest judge shall begin to examine the parties, and proceed to make an adequate enquiry into what has been said; and after the oldest has spoken, the rest shall proceed in order to examine either party as to what he finds defective in the evidence, whether of statement or omission; and he who has nothing to ask shall hand over the examination to another. And on so much of what has been said as is to the purpose all the judges shall set their seals, and place the writings on the altar of Hestia. On the next day they shall meet again, and in like manner put their questions and go through the cause, and again set their seals upon the evidence; and when they have three times done this, and have had witnesses and evidence enough, they shall each of them give a holy vote, after promising by Hestia that they will decide justly and truly to the utmost of their power; and so they shall put an end to the suit.

\par  Next, after what relates to the Gods, follows what relates to the dissolution of the state: Whoever by permitting a man to power enslaves the laws, and subjects the city to factions, using violence and stirring up sedition contrary to law, him we will deem the greatest enemy of the whole state. But he who takes no part in such proceedings, and, being one of the chief magistrates of the state, has no knowledge of treason, or, having knowledge of it, by reason of cowardice does not interfere on behalf of his country, such an one we must consider nearly as bad. Every man who is worth anything will inform the magistrates, and bring the conspirator to trial for making a violent and illegal attempt to change the government. The judges of such cases shall be the same as of the robbers of temples; and let the whole proceeding be carried on in the same way, and the vote of the majority condemn to death. But let there be a general rule, that the disgrace and punishment of the father is not to be visited on the children, except in the case of some one whose father, grandfather, and great-grandfather have successively undergone the penalty of death. Such persons the city shall send away with all their possessions to the city and country of their ancestors, retaining only and wholly their appointed lot. And out of the citizens who have more than one son of not less than ten years of age, they shall select ten whom their father or grandfather by the mother's or father's side shall appoint, and let them send to Delphi the names of those who are selected, and him whom the God chooses they shall establish as heir of the house which has failed; and may he have better fortune than his predecessors!

\par \textbf{CLEINIAS}
\par   Very good.

\par \textbf{ATHENIAN}
\par   Once more let there be a third general law respecting the judges who are to give judgment, and the manner of conducting suits against those who are tried on an accusation of treason; and as concerning the remaining or departure of their descendants—there shall be one law for all three, for the traitor, and the robber of temples, and the subverter by violence of the laws of the state. For a thief, whether he steal much or little, let there be one law, and one punishment for all alike:  in the first place, let him pay double the amount of the theft if he be convicted, and if he have so much over and above the allotment—if he have not, he shall be bound until he pay the penalty, or persuade him who has obtained the sentence against him to forgive him. But if a person be convicted of a theft against the state, then if he can persuade the city, or if he will pay back twice the amount of the theft, he shall be set free from his bonds.

\par \textbf{CLEINIAS}
\par   What makes you say, Stranger, that a theft is all one, whether the thief may have taken much or little, and either from sacred or secular places—and these are not the only differences in thefts—seeing, then, that they are of many kinds, ought not the legislator to adapt himself to them, and impose upon them entirely different penalties?

\par \textbf{ATHENIAN}
\par   Excellent. I was running on too fast, Cleinias, and you impinged upon me, and brought me to my senses, reminding me of what, indeed, had occurred to my mind already, that legislation was never yet rightly worked out, as I may say in passing. Do you remember the image in which I likened the men for whom laws are now made to slaves who are doctored by slaves? For of this you may be very sure, that if one of those empirical physicians, who practise medicine without science, were to come upon the gentleman physician talking to his gentleman patient, and using the language almost of philosophy, beginning at the beginning of the disease and discoursing about the whole nature of the body, he would burst into a hearty laugh—he would say what most of those who are called doctors always have at their tongue's end:  Foolish fellow, he would say, you are not healing the sick man, but you are educating him; and he does not want to be made a doctor, but to get well.

\par \textbf{CLEINIAS}
\par   And would he not be right?

\par \textbf{ATHENIAN}
\par   Perhaps he would; and he might remark upon us, that he who discourses about laws, as we are now doing, is giving the citizens education and not laws; that would be rather a telling observation.

\par \textbf{CLEINIAS}
\par   Very true.

\par \textbf{ATHENIAN}
\par   But we are fortunate.

\par \textbf{CLEINIAS}
\par   In what way?

\par \textbf{ATHENIAN}
\par   Inasmuch as we are not compelled to give laws, but we may take into consideration every form of government, and ascertain what is best and what is most needful, and how they may both be carried into execution; and we may also, if we please, at this very moment choose what is best, or, if we prefer, what is most necessary—which shall we do?

\par \textbf{CLEINIAS}
\par   There is something ridiculous, Stranger, in our proposing such an alternative, as if we were legislators, simply bound under some great necessity which cannot be deferred to the morrow. But we, as I may by the grace of Heaven affirm, like gatherers of stones or beginners of some composite work, may gather a heap of materials, and out of this, at our leisure, select what is suitable for our projected construction. Let us then suppose ourselves to be at leisure, not of necessity building, but rather like men who are partly providing materials, and partly putting them together. And we may truly say that some of our laws, like stones, are already fixed in their places, and others lie at hand.

\par \textbf{ATHENIAN}
\par   Certainly, in that case, Cleinias, our view of law will be more in accordance with nature. For there is another matter affecting legislators, which I must earnestly entreat you to consider.

\par \textbf{CLEINIAS}
\par   What is it?

\par \textbf{ATHENIAN}
\par   There are many writings to be found in cities, and among them there are discourses composed by legislators as well as by other persons.

\par \textbf{CLEINIAS}
\par   To be sure.

\par \textbf{ATHENIAN}
\par   Shall we give heed rather to the writings of those others—poets and the like, who either in metre or out of metre have recorded their advice about the conduct of life, and not to the writings of legislators? or shall we give heed to them above all?

\par \textbf{CLEINIAS}
\par   Yes; to them far above all others.

\par \textbf{ATHENIAN}
\par   And ought the legislator alone among writers to withhold his opinion about the beautiful, the good, and the just, and not to teach what they are, and how they are to be pursued by those who intend to be happy?

\par \textbf{CLEINIAS}
\par   Certainly not.

\par \textbf{ATHENIAN}
\par   And is it disgraceful for Homer and Tyrtaeus and other poets to lay down evil precepts in their writings respecting life and the pursuits of men, but not so disgraceful for Lycurgus and Solon and others who were legislators as well as writers? Is it not true that of all the writings to be found in cities, those which relate to laws, when you unfold and read them, ought to be by far the noblest and the best? and should not other writings either agree with them, or if they disagree, be deemed ridiculous? We should consider whether the laws of states ought not to have the character of loving and wise parents, rather than of tyrants and masters, who command and threaten, and, after writing their decrees on walls, go their ways; and whether, in discoursing of laws, we should not take the gentler view of them which may or may not be attainable—at any rate, we will show our readiness to entertain such a view, and be prepared to undergo whatever may be the result. And may the result be good, and if God be gracious, it will be good!

\par \textbf{CLEINIAS}
\par   Excellent; let us do as you say.

\par \textbf{ATHENIAN}
\par   Then we will now consider accurately, as we proposed, what relates to robbers of temples, and all kinds of thefts, and offences in general; and we must not be annoyed if, in the course of legislation, we have enacted some things, and have not made up our minds about some others; for as yet we are not legislators, but we may soon be. Let us, if you please, consider these matters.

\par \textbf{CLEINIAS}
\par   By all means.

\par \textbf{ATHENIAN}
\par   Concerning all things honourable and just, let us then endeavour to ascertain how far we are consistent with ourselves, and how far we are inconsistent, and how far the many, from whom at any rate we should profess a desire to differ, agree and disagree among themselves.

\par \textbf{CLEINIAS}
\par   What are the inconsistencies which you observe in us?

\par \textbf{ATHENIAN}
\par   I will endeavour to explain. If I am not mistaken, we are all agreed that justice, and just men and things and actions, are all fair, and, if a person were to maintain that just men, even when they are deformed in body, are still perfectly beautiful in respect of the excellent justice of their minds, no one would say that there was any inconsistency in this.

\par \textbf{CLEINIAS}
\par   They would be quite right.

\par \textbf{ATHENIAN}
\par   Perhaps; but let us consider further, that if all things which are just are fair and honourable, in the term 'all' we must include just sufferings which are the correlatives of just actions.

\par \textbf{CLEINIAS}
\par   And what is the inference?

\par \textbf{ATHENIAN}
\par   The inference is, that a just action in partaking of the just partakes also in the same degree of the fair and honourable.

\par \textbf{CLEINIAS}
\par   Certainly.

\par \textbf{ATHENIAN}
\par   And must not a suffering which partakes of the just principle be admitted to be in the same degree fair and honourable, if the argument is consistently carried out?

\par \textbf{CLEINIAS}
\par   True.

\par \textbf{ATHENIAN}
\par   But then if we admit suffering to be just and yet dishonourable, and the term 'dishonourable' is applied to justice, will not the just and the honourable disagree?

\par \textbf{CLEINIAS}
\par   What do you mean?

\par \textbf{ATHENIAN}
\par   A thing not difficult to understand; the laws which have been already enacted would seem to announce principles directly opposed to what we are saying.

\par \textbf{CLEINIAS}
\par   To what?

\par \textbf{ATHENIAN}
\par   We had enacted, if I am not mistaken, that the robber of temples, and he who was the enemy of law and order, might justly be put to death, and we were proceeding to make divers other enactments of a similar nature. But we stopped short, because we saw that these sufferings are infinite in number and degree, and that they are, at once, the most just and also the most dishonourable of all sufferings. And if this be true, are not the just and the honourable at one time all the same, and at another time in the most diametrical opposition?

\par \textbf{CLEINIAS}
\par   Such appears to be the case.

\par \textbf{ATHENIAN}
\par   In this discordant and inconsistent fashion does the language of the many rend asunder the honourable and just.

\par \textbf{CLEINIAS}
\par   Very true, Stranger.

\par \textbf{ATHENIAN}
\par   Then now, Cleinias, let us see how far we ourselves are consistent about these matters.

\par \textbf{CLEINIAS}
\par   Consistent in what?

\par \textbf{ATHENIAN}
\par   I think that I have clearly stated in the former part of the discussion, but if I did not, let me now state—

\par \textbf{CLEINIAS}
\par   What?

\par \textbf{ATHENIAN}
\par   That all bad men are always involuntarily bad; and from this I must proceed to draw a further inference.

\par \textbf{CLEINIAS}
\par   What is it?

\par \textbf{ATHENIAN}
\par   That the unjust man may be bad, but that he is bad against his will. Now that an action which is voluntary should be done involuntarily is a contradiction; wherefore he who maintains that injustice is involuntary will deem that the unjust does injustice involuntarily. I too admit that all men do injustice involuntarily, and if any contentious or disputatious person says that men are unjust against their will, and yet that many do injustice willingly, I do not agree with him. But, then, how can I avoid being inconsistent with myself, if you, Cleinias, and you, Megillus, say to me—Well, Stranger, if all this be as you say, how about legislating for the city of the Magnetes—shall we legislate or not—what do you advise? Certainly we will, I should reply. Then will you determine for them what are voluntary and what are involuntary crimes, and shall we make the punishments greater of voluntary errors and crimes and less for the involuntary? or shall we make the punishment of all to be alike, under the idea that there is no such thing as voluntary crime?

\par \textbf{CLEINIAS}
\par   Very good, Stranger; and what shall we say in answer to these objections?

\par \textbf{ATHENIAN}
\par   That is a very fair question. In the first place, let us—

\par \textbf{CLEINIAS}
\par   Do what?

\par \textbf{ATHENIAN}
\par   Let us remember what has been well said by us already, that our ideas of justice are in the highest degree confused and contradictory. Bearing this in mind, let us proceed to ask ourselves once more whether we have discovered a way out of the difficulty. Have we ever determined in what respect these two classes of actions differ from one another? For in all states and by all legislators whatsoever, two kinds of actions have been distinguished—the one, voluntary, the other, involuntary; and they have legislated about them accordingly. But shall this new word of ours, like an oracle of God, be only spoken, and get away without giving any explanation or verification of itself? How can a word not understood be the basis of legislation? Impossible. Before proceeding to legislate, then, we must prove that they are two, and what is the difference between them, that when we impose the penalty upon either, every one may understand our proposal, and be able in some way to judge whether the penalty is fitly or unfitly inflicted.

\par \textbf{CLEINIAS}
\par   I agree with you, Stranger; for one of two things is certain:  either we must not say that all unjust acts are involuntary, or we must show the meaning and truth of this statement.

\par \textbf{ATHENIAN}
\par   Of these two alternatives, the one is quite intolerable—not to speak what I believe to be the truth would be to me unlawful and unholy. But if acts of injustice cannot be divided into voluntary and involuntary, I must endeavour to find some other distinction between them.

\par \textbf{CLEINIAS}
\par   Very true, Stranger; there cannot be two opinions among us upon that point.

\par \textbf{ATHENIAN}
\par   Reflect, then; there are hurts of various kinds done by the citizens to one another in the intercourse of life, affording plentiful examples both of the voluntary and involuntary.

\par \textbf{CLEINIAS}
\par   Certainly.

\par \textbf{ATHENIAN}
\par   I would not have any one suppose that all these hurts are injuries, and that these injuries are of two kinds—one, voluntary, and the other, involuntary; for the involuntary hurts of all men are quite as many and as great as the voluntary. And please to consider whether I am right or quite wrong in what I am going to say; for I deny, Cleinias and Megillus, that he who harms another involuntarily does him an injury involuntarily, nor should I legislate about such an act under the idea that I am legislating for an involuntary injury. But I should rather say that such a hurt, whether great or small, is not an injury at all; and, on the other hand, if I am right, when a benefit is wrongly conferred, the author of the benefit may often be said to injure. For I maintain, O my friends, that the mere giving or taking away of anything is not to be described either as just or unjust; but the legislator has to consider whether mankind do good or harm to one another out of a just principle and intention. On the distinction between injustice and hurt he must fix his eye; and when there is hurt, he must, as far as he can, make the hurt good by law, and save that which is ruined, and raise up that which is fallen, and make that which is dead or wounded whole. And when compensation has been given for injustice, the law must always seek to win over the doers and sufferers of the several hurts from feelings of enmity to those of friendship.

\par \textbf{CLEINIAS}
\par   Very good.

\par \textbf{ATHENIAN}
\par   Then as to unjust hurts (and gains also, supposing the injustice to bring gain), of these we may heal as many as are capable of being healed, regarding them as diseases of the soul; and the cure of injustice will take the following direction.

\par \textbf{CLEINIAS}
\par   What direction?

\par \textbf{ATHENIAN}
\par   When any one commits any injustice, small or great, the law will admonish and compel him either never at all to do the like again, or never voluntarily, or at any rate in a far less degree; and he must in addition pay for the hurt. Whether the end is to be attained by word or action, with pleasure or pain, by giving or taking away privileges, by means of fines or gifts, or in whatsoever way the law shall proceed to make a man hate injustice, and love or not hate the nature of the just—this is quite the noblest work of law. But if the legislator sees any one who is incurable, for him he will appoint a law and a penalty. He knows quite well that to such men themselves there is no profit in the continuance of their lives, and that they would do a double good to the rest of mankind if they would take their departure, inasmuch as they would be an example to other men not to offend, and they would relieve the city of bad citizens. In such cases, and in such cases only, the legislator ought to inflict death as the punishment of offences.

\par \textbf{CLEINIAS}
\par   What you have said appears to me to be very reasonable, but will you favour me by stating a little more clearly the difference between hurt and injustice, and the various complications of the voluntary and involuntary which enter into them?

\par \textbf{ATHENIAN}
\par   I will endeavour to do as you wish:  Concerning the soul, thus much would be generally said and allowed, that one element in her nature is passion, which may be described either as a state or a part of her, and is hard to be striven against and contended with, and by irrational force overturns many things.

\par \textbf{CLEINIAS}
\par   Very true.

\par \textbf{ATHENIAN}
\par   And pleasure is not the same with passion, but has an opposite power, working her will by persuasion and by the force of deceit in all things.

\par \textbf{CLEINIAS}
\par   Quite true.

\par \textbf{ATHENIAN}
\par   A man may truly say that ignorance is a third cause of crimes. Ignorance, however, may be conveniently divided by the legislator into two sorts:  there is simple ignorance, which is the source of lighter offences, and double ignorance, which is accompanied by a conceit of wisdom; and he who is under the influence of the latter fancies that he knows all about matters of which he knows nothing. This second kind of ignorance, when possessed of power and strength, will be held by the legislator to be the source of great and monstrous crimes, but when attended with weakness, will only result in the errors of children and old men; and these he will treat as errors, and will make laws accordingly for those who commit them, which will be the mildest and most merciful of all laws.

\par \textbf{CLEINIAS}
\par   You are perfectly right.

\par \textbf{ATHENIAN}
\par   We all of us remark of one man that he is superior to pleasure and passion, and of another that he is inferior to them; and this is true.

\par \textbf{CLEINIAS}
\par   Certainly.

\par \textbf{ATHENIAN}
\par   But no one was ever yet heard to say that one of us is superior and another inferior to ignorance.

\par \textbf{CLEINIAS}
\par   Very true.

\par \textbf{ATHENIAN}
\par   We are speaking of motives which incite men to the fulfilment of their will; although an individual may be often drawn by them in opposite directions at the same time.

\par \textbf{CLEINIAS}
\par   Yes, often.

\par \textbf{ATHENIAN}
\par   And now I can define to you clearly, and without ambiguity, what I mean by the just and unjust, according to my notion of them:  When anger and fear, and pleasure and pain, and jealousies and desires, tyrannize over the soul, whether they do any harm or not—I call all this injustice. But when the opinion of the best, in whatever part of human nature states or individuals may suppose that to dwell, has dominion in the soul and orders the life of every man, even if it be sometimes mistaken, yet what is done in accordance therewith, and the principle in individuals which obeys this rule, and is best for the whole life of man, is to be called just; although the hurt done by mistake is thought by many to be involuntary injustice. Leaving the question of names, about which we are not going to quarrel, and having already delineated three sources of error, we may begin by recalling them somewhat more vividly to our memory:  One of them was of the painful sort, which we denominate anger and fear.

\par \textbf{CLEINIAS}
\par   Quite right.

\par \textbf{ATHENIAN}
\par   There was a second consisting of pleasures and desires, and a third of hopes, which aimed at true opinion about the best. The latter being subdivided into three, we now get five sources of actions, and for these five we will make laws of two kinds.

\par \textbf{CLEINIAS}
\par   What are the two kinds?

\par \textbf{ATHENIAN}
\par   There is one kind of actions done by violence and in the light of day, and another kind of actions which are done in darkness and with secret deceit, or sometimes both with violence and deceit; the laws concerning these last ought to have a character of severity.

\par \textbf{CLEINIAS}
\par   Naturally.

\par \textbf{ATHENIAN}
\par   And now let us return from this digression and complete the work of legislation. Laws have been already enacted by us concerning the robbers of the Gods, and concerning traitors, and also concerning those who corrupt the laws for the purpose of subverting the government. A man may very likely commit some of these crimes, either in a state of madness or when affected by disease, or under the influence of extreme old age, or in a fit of childish wantonness, himself no better than a child. And if this be made evident to the judges elected to try the cause, on the appeal of the criminal or his advocate, and he be judged to have been in this state when he committed the offence, he shall simply pay for the hurt which he may have done to another; but he shall be exempt from other penalties, unless he have slain some one, and have on his hands the stain of blood. And in that case he shall go to another land and country, and there dwell for a year; and if he return before the expiration of the time which the law appoints, or even set his foot at all on his native land, he shall be bound by the guardians of the law in the public prison for two years, and then go free.

\par  Having begun to speak of homicide, let us endeavour to lay down laws concerning every different kind of homicide; and, first of all, concerning violent and involuntary homicides. If any one in an athletic contest, and at the public games, involuntarily kills a friend, and he dies either at the time or afterwards of the blows which he has received; or if the like misfortune happens to any one in war, or military exercises, or mimic contests of which the magistrates enjoin the practice, whether with or without arms, when he has been purified according to the law brought from Delphi relating to these matters, he shall be innocent. And so in the case of physicians: if their patient dies against their will, they shall be held guiltless by the law. And if one slay another with his own hand, but unintentionally, whether he be unarmed or have some instrument or dart in his hand; or if he kill him by administering food or drink, or by the application of fire or cold, or by suffocating him, whether he do the deed by his own hand, or by the agency of others, he shall be deemed the agent, and shall suffer one of the following penalties: If he kill the slave of another in the belief that he is his own, he shall bear the master of the dead man harmless from loss, or shall pay a penalty of twice the value of the dead man, which the judges shall assess; but purifications must be used greater and more numerous than for those who committed homicide at the games—what they are to be, the interpreters whom the God appoints shall be authorised to declare. And if a man kills his own slave, when he has been purified according to law, he shall be quit of the homicide. And if a man kills a freeman unintentionally, he shall undergo the same purification as he did who killed the slave. But let him not forget also a tale of olden time, which is to this effect: He who has suffered a violent end, when newly dead, if he has had the soul of a freeman in life, is angry with the author of his death; and being himself full of fear and panic by reason of his violent end, when he sees his murderer walking about in his own accustomed haunts, he is stricken with terror and becomes disordered, and this disorder of his, aided by the guilty recollection of the other, is communicated by him with overwhelming force to the murderer and his deeds. Wherefore also the murderer must go out of the way of his victim for the entire period of a year, and not himself be found in any spot which was familiar to him throughout the country. And if the dead man be a stranger, the homicide shall be kept from the country of the stranger during a like period. If any one voluntarily obeys this law, the next of kin to the deceased, seeing all that has happened, shall take pity on him, and make peace with him, and show him all gentleness. But if any one is disobedient, and either ventures to go to any of the temples and sacrifice unpurified, or will not continue in exile during the appointed time, the next of kin to the deceased shall proceed against him for murder; and if he be convicted, every part of his punishment shall be doubled. And if the next of kin do not proceed against the perpetrator of the crime, then the pollution shall be deemed to fall upon his own head—the murdered man will fix the guilt upon his kinsman, and he who has a mind to proceed against him may compel him to be absent from his country during five years, according to law. If a stranger unintentionally kill a stranger who is dwelling in the city, he who likes shall prosecute the cause according to the same rules. If he be a metic, let him be absent for a year, or if he be an entire stranger, in addition to the purification, whether he have slain a stranger, or a metic, or a citizen, he shall be banished for life from the country which is in possession of our laws. And if he return contrary to law, let the guardians of the law punish him with death; and let them hand over his property, if he have any, to him who is next of kin to the sufferer. And if he be wrecked, and driven on the coast against his will, he shall take up his abode on the seashore, wetting his feet in the sea, and watching for an opportunity of sailing; but if he be brought by land, and is not his own master, let the magistrate whom he first comes across in the city, release him and send him unharmed over the border.

\par  If any one slays a freeman with his own hand, and the deed be done in passion, in the case of such actions we must begin by making a distinction. For a deed is done from passion either when men suddenly, and without intention to kill, cause the death of another by blows and the like on a momentary impulse, and are sorry for the deed immediately afterwards; or again, when after having been insulted in deed or word, men pursue revenge, and kill a person intentionally, and are not sorry for the act. And, therefore, we must assume that these homicides are of two kinds, both of them arising from passion, which may be justly said to be in a mean between the voluntary and involuntary; at the same time, they are neither of them anything more than a likeness or shadow of either. He who treasures up his anger, and avenges himself, not immediately and at the moment, but with insidious design, and after an interval, is like the voluntary; but he who does not treasure up his anger, and takes vengeance on the instant, and without malice prepense, approaches to the involuntary; and yet even he is not altogether involuntary, but is only the image or shadow of the involuntary; wherefore about homicides committed in hot blood, there is a difficulty in determining whether in legislating we shall reckon them as voluntary or as partly involuntary. The best and truest view is to regard them respectively as likenesses only of the voluntary and involuntary, and to distinguish them accordingly as they are done with or without premeditation. And we should make the penalties heavier for those who commit homicide with angry premeditation, and lighter for those who do not premeditate, but smite upon the instant; for that which is like a greater evil should be punished more severely, and that which is like a less evil should be punished less severely: this shall be the rule of our laws.

\par \textbf{CLEINIAS}
\par   Certainly.

\par \textbf{ATHENIAN}
\par   Let us proceed:  If any one slays a freeman with his own hand, and the deed be done in a moment of anger, and without premeditation, let the offender suffer in other respects as the involuntary homicide would have suffered, and also undergo an exile of two years, that he may learn to school his passions. But he who slays another from passion, yet with premeditation, shall in other respects suffer as the former; and to this shall be added an exile of three instead of two years—his punishment is to be longer because his passion is greater. The manner of their return shall be on this wise:  (and here the law has difficulty in determining exactly; for in some cases the murderer who is judged by the law to be the worse may really be the less cruel, and he who is judged the less cruel may be really the worse, and may have executed the murder in a more savage manner, whereas the other may have been gentler. But in general the degrees of guilt will be such as we have described them. Of all these things the guardians of the law must take cognizance):  When a homicide of either kind has completed his term of exile, the guardians shall send twelve judges to the borders of the land; these during the interval shall have informed themselves of the actions of the criminals, and they shall judge respecting their pardon and reception; and the homicides shall abide by their judgment. But if after they have returned home, any one of them in a moment of anger repeats the deed, let him be an exile, and return no more; or if he returns, let him suffer as the stranger was to suffer in a similar case. He who kills his own slave shall undergo a purification, but if he kills the slave of another in anger, he shall pay twice the amount of the loss to his owner. And if any homicide is disobedient to the law, and without purification pollutes the agora, or the games, or the temples, he who pleases may bring to trial the next of kin to the dead man for permitting him, and the murderer with him, and may compel the one to exact and the other to suffer a double amount of fines and purifications; and the accuser shall himself receive the fine in accordance with the law. If a slave in a fit of passion kills his master, the kindred of the deceased man may do with the murderer (provided only they do not spare his life) whatever they please, and they will be pure; or if he kills a freeman, who is not his master, the owner shall give up the slave to the relatives of the deceased, and they shall be under an obligation to put him to death, but this may be done in any manner which they please. And if (which is a rare occurrence, but does sometimes happen) a father or a mother in a moment of passion slays a son or daughter by blows, or some other violence, the slayer shall undergo the same purification as in other cases, and be exiled during three years; but when the exile returns the wife shall separate from the husband, and the husband from the wife, and they shall never afterwards beget children together, or live under the same roof, or partake of the same sacred rites with those whom they have deprived of a child or of a brother. And he who is impious and disobedient in such a case shall be brought to trial for impiety by any one who pleases. If in a fit of anger a husband kills his wedded wife, or the wife her husband, the slayer shall undergo the same purification, and the term of exile shall be three years. And when he who has committed any such crime returns, let him have no communication in sacred rites with his children, neither let him sit at the same table with them, and the father or son who disobeys shall be liable to be brought to trial for impiety by any one who pleases. If a brother or a sister in a fit of passion kills a brother or a sister, they shall undergo purification and exile, as was the case with parents who killed their offspring:  they shall not come under the same roof, or share in the sacred rites of those whom they have deprived of their brethren, or of their children. And he who is disobedient shall be justly liable to the law concerning impiety, which relates to these matters. If any one is so violent in his passion against his parents, that in the madness of his anger he dares to kill one of them, if the murdered person before dying freely forgives the murderer, let him undergo the purification which is assigned to those who have been guilty of involuntary homicide, and do as they do, and he shall be pure. But if he be not acquitted, the perpetrator of such a deed shall be amenable to many laws—he shall be amenable to the extreme punishments for assault, and impiety, and robbing of temples, for he has robbed his parent of life; and if a man could be slain more than once, most justly would he who in a fit of passion has slain father or mother, undergo many deaths. How can he, whom, alone of all men, even in defence of his life, and when about to suffer death at the hands of his parents, no law will allow to kill his father or his mother who are the authors of his being, and whom the legislator will command to endure any extremity rather than do this—how can he, I say, lawfully receive any other punishment? Let death then be the appointed punishment of him who in a fit of passion slays his father or his mother. But if brother kills brother in a civil broil, or under other like circumstances, if the other has begun, and he only defends himself, let him be free from guilt, as he would be if he had slain an enemy; and the same rule will apply if a citizen kill a citizen, or a stranger a stranger. Or if a stranger kill a citizen or a citizen a stranger in self-defence, let him be free from guilt in like manner; and so in the case of a slave who has killed a slave; but if a slave have killed a freeman in self-defence, let him be subject to the same law as he who has killed a father; and let the law about the remission of penalties in the case of parricide apply equally to every other remission. Whenever any sufferer of his own accord remits the guilt of homicide to another, under the idea that his act was involuntary, let the perpetrator of the deed undergo a purification and remain in exile for a year, according to law.

\par  Enough has been said of murders violent and involuntary and committed in passion: we have now to speak of voluntary crimes done with injustice of every kind and with premeditation, through the influence of pleasures, and desires, and jealousies.

\par \textbf{CLEINIAS}
\par   Very good.

\par \textbf{ATHENIAN}
\par   Let us first speak, as far as we are able, of their various kinds. The greatest cause of them is lust, which gets the mastery of the soul maddened by desire; and this is most commonly found to exist where the passion reigns which is strongest and most prevalent among the mass of mankind:  I mean where the power of wealth breeds endless desires of never-to-be-satisfied acquisition, originating in natural disposition, and a miserable want of education. Of this want of education, the false praise of wealth which is bruited about both among Hellenes and barbarians is the cause; they deem that to be the first of goods which in reality is only the third. And in this way they wrong both posterity and themselves, for nothing can be nobler and better than that the truth about wealth should be spoken in all states—namely, that riches are for the sake of the body, as the body is for the sake of the soul. They are good, and wealth is intended by nature to be for the sake of them, and is therefore inferior to them both, and third in order of excellence. This argument teaches us that he who would be happy ought not to seek to be rich, or rather he should seek to be rich justly and temperately, and then there would be no murders in states requiring to be purged away by other murders. But now, as I said at first, avarice is the chiefest cause and source of the worst trials for voluntary homicide. A second cause is ambition:  this creates jealousies, which are troublesome companions, above all to the jealous man himself, and in a less degree to the chiefs of the state. And a third cause is cowardly and unjust fear, which has been the occasion of many murders. When a man is doing or has done something which he desires that no one should know him to be doing or to have done, he will take the life of those who are likely to inform of such things, if he have no other means of getting rid of them. Let this be said as a prelude concerning crimes of violence in general; and I must not omit to mention a tradition which is firmly believed by many, and has been received by them from those who are learned in the mysteries:  they say that such deeds will be punished in the world below, and also that when the perpetrators return to this world they will pay the natural penalty which is due to the sufferer, and end their lives in like manner by the hand of another. If he who is about to commit murder believes this, and is made by the mere prelude to dread such a penalty, there is no need to proceed with the proclamation of the law. But if he will not listen, let the following law be declared and registered against him:  Whoever shall wrongfully and of design slay with his own hand any of his kinsmen, shall in the first place be deprived of legal privileges; and he shall not pollute the temples, or the agora, or the harbours, or any other place of meeting, whether he is forbidden of men or not; for the law, which represents the whole state, forbids him, and always is and will be in the attitude of forbidding him. And if a cousin or nearer relative of the deceased, whether on the male or female side, does not prosecute the homicide when he ought, and have him proclaimed an outlaw, he shall in the first place be involved in the pollution, and incur the hatred of the Gods, even as the curse of the law stirs up the voices of men against him; and in the second place he shall be liable to be prosecuted by any one who is willing to inflict retribution on behalf of the dead. And he who would avenge a murder shall observe all the precautionary ceremonies of lavation, and any others which the God commands in cases of this kind. Let him have proclamation made, and then go forth and compel the perpetrator to suffer the execution of justice according to the law. Now the legislator may easily show that these things must be accomplished by prayers and sacrifices to certain Gods, who are concerned with the prevention of murders in states. But who these Gods are, and what should be the true manner of instituting such trials with due regard to religion, the guardians of the law, aided by the interpreters, and the prophets, and the God, shall determine, and when they have determined let them carry on the prosecution at law. The cause shall have the same judges who are appointed to decide in the case of those who plunder temples. Let him who is convicted be punished with death, and let him not be buried in the country of the murdered man, for this would be shameless as well as impious. But if he fly and will not stand his trial, let him fly for ever; or, if he set foot anywhere on any part of the murdered man's country, let any relation of the deceased, or any other citizen who may first happen to meet with him, kill him with impunity, or bind and deliver him to those among the judges of the case who are magistrates, that they may put him to death. And let the prosecutor demand surety of him whom he prosecutes; three sureties sufficient in the opinion of the magistrates who try the cause shall be provided by him, and they shall undertake to produce him at the trial. But if he be unwilling or unable to provide sureties, then the magistrates shall take him and keep him in bonds, and produce him at the day of trial.

\par  If a man do not commit a murder with his own hand, but contrives the death of another, and is the author of the deed in intention and design, and he continues to dwell in the city, having his soul not pure of the guilt of murder, let him be tried in the same way, except in what relates to the sureties; and also, if he be found guilty, his body after execution may have burial in his native land, but in all other respects his case shall be as the former; and whether a stranger shall kill a citizen, or a citizen a stranger, or a slave a slave, there shall be no difference as touching murder by one's own hand or by contrivance, except in the matter of sureties; and these, as has been said, shall be required of the actual murderer only, and he who brings the accusation shall bind them over at the time. If a slave be convicted of slaying a freeman voluntarily, either by his own hand or by contrivance, let the public executioner take him in the direction of the sepulchre, to a place whence he can see the tomb of the dead man, and inflict upon him as many stripes as the person who caught him orders, and if he survive, let him put him to death. And if any one kills a slave who has done no wrong, because he is afraid that he may inform of some base and evil deeds of his own, or for any similar reason, in such a case let him pay the penalty of murder, as he would have done if he had slain a citizen. There are things about which it is terrible and unpleasant to legislate, but impossible not to legislate. If, for example, there should be murders of kinsmen, either perpetrated by the hands of kinsmen, or by their contrivance, voluntary and purely malicious, which most often happen in ill-regulated and ill-educated states, and may perhaps occur even in a country where a man would not expect to find them, we must repeat once more the tale which we narrated a little while ago, in the hope that he who hears us will be the more disposed to abstain voluntarily on these grounds from murders which are utterly abominable. For the myth, or saying, or whatever we ought to call it, has been plainly set forth by priests of old; they have pronounced that the justice which guards and avenges the blood of kindred, follows the law of retaliation, and ordains that he who has done any murderous act should of necessity suffer that which he has done. He who has slain a father shall himself be slain at some time or other by his children—if a mother, he shall of necessity take a woman's nature, and lose his life at the hands of his offspring in after ages; for where the blood of a family has been polluted there is no other purification, nor can the pollution be washed out until the homicidal soul which did the deed has given life for life, and has propitiated and laid to sleep the wrath of the whole family. These are the retributions of Heaven, and by such punishments men should be deterred. But if they are not deterred, and any one should be incited by some fatality to deprive his father, or mother, or brethren, or children, of life voluntarily and of purpose, for him the earthly lawgiver legislates as follows: There shall be the same proclamations about outlawry, and there shall be the same sureties which have been enacted in the former cases. But in his case, if he be convicted, the servants of the judges and the magistrates shall slay him at an appointed place without the city where three ways meet, and there expose his body naked, and each of the magistrates on behalf of the whole city shall take a stone and cast it upon the head of the dead man, and so deliver the city from pollution; after that, they shall bear him to the borders of the land, and cast him forth unburied, according to law. And what shall he suffer who slays him who of all men, as they say, is his own best friend? I mean the suicide, who deprives himself by violence of his appointed share of life, not because the law of the state requires him, nor yet under the compulsion of some painful and inevitable misfortune which has come upon him, nor because he has had to suffer from irremediable and intolerable shame, but who from sloth or want of manliness imposes upon himself an unjust penalty. For him, what ceremonies there are to be of purification and burial God knows, and about these the next of kin should enquire of the interpreters and of the laws thereto relating, and do according to their injunctions. They who meet their death in this way shall be buried alone, and none shall be laid by their side; they shall be buried ingloriously in the borders of the twelve portions of the land, in such places as are uncultivated and nameless, and no column or inscription shall mark the place of their interment. And if a beast of burden or other animal cause the death of any one, except in the case of anything of that kind happening to a competitor in the public contests, the kinsmen of the deceased shall prosecute the slayer for murder, and the wardens of the country, such, and so many as the kinsmen appoint, shall try the cause, and let the beast when condemned be slain by them, and let them cast it beyond the borders. And if any lifeless thing deprive a man of life, except in the case of a thunderbolt or other fatal dart sent from the Gods—whether a man is killed by lifeless objects falling upon him, or by his falling upon them, the nearest of kin shall appoint the nearest neighbour to be a judge, and thereby acquit himself and the whole family of guilt. And he shall cast forth the guilty thing beyond the border, as has been said about the animals.

\par  If a man is found dead, and his murderer be unknown, and after a diligent search cannot be detected, there shall be the same proclamation as in the previous cases, and the same interdict on the murderer; and having proceeded against him, they shall proclaim in the agora by a herald, that he who has slain such and such a person, and has been convicted of murder, shall not set his foot in the temples, nor at all in the country of the murdered man, and if he appears and is discovered, he shall die, and be cast forth unburied beyond the border. Let this one law then be laid down by us about murder; and let cases of this sort be so regarded.

\par  And now let us say in what cases and under what circumstances the murderer is rightly free from guilt: If a man catch a thief coming into his house by night to steal, and he take and kill him, or if he slay a footpad in self-defence, he shall be guiltless. And any one who does violence to a free woman or a youth, shall be slain with impunity by the injured person, or by his or her father or brothers or sons. If a man find his wife suffering violence, he may kill the violator, and be guiltless in the eye of the law; or if a person kill another in warding off death from his father or mother or children or brethren or wife who are doing no wrong, he shall assuredly be guiltless.

\par  Thus much as to the nurture and education of the living soul of man, having which, he can, and without which, if he unfortunately be without them, he cannot live; and also concerning the punishments which are to be inflicted for violent deaths, let thus much be enacted. Of the nurture and education of the body we have spoken before, and next in order we have to speak of deeds of violence, voluntary and involuntary, which men do to one another; these we will now distinguish, as far as we are able, according to their nature and number, and determine what will be the suitable penalties of each, and so assign to them their proper place in the series of our enactments. The poorest legislator will have no difficulty in determining that wounds and mutilations arising out of wounds should follow next in order after deaths. Let wounds be divided as homicides were divided—into those which are involuntary, and which are given in passion or from fear, and those inflicted voluntarily and with premeditation. Concerning all this, we must make some such proclamation as the following: Mankind must have laws, and conform to them, or their life would be as bad as that of the most savage beast. And the reason of this is that no man's nature is able to know what is best for human society; or knowing, always able and willing to do what is best. In the first place, there is a difficulty in apprehending that the true art of politics is concerned, not with private but with public good (for public good binds together states, but private only distracts them); and that both the public and private good as well of individuals as of states is greater when the state and not the individual is first considered. In the second place, although a person knows in the abstract that this is true, yet if he be possessed of absolute and irresponsible power, he will never remain firm in his principles or persist in regarding the public good as primary in the state, and the private good as secondary. Human nature will be always drawing him into avarice and selfishness, avoiding pain and pursuing pleasure without any reason, and will bring these to the front, obscuring the juster and better; and so working darkness in his soul will at last fill with evils both him and the whole city. For if a man were born so divinely gifted that he could naturally apprehend the truth, he would have no need of laws to rule over him; for there is no law or order which is above knowledge, nor can mind, without impiety, be deemed the subject or slave of any man, but rather the lord of all. I speak of mind, true and free, and in harmony with nature. But then there is no such mind anywhere, or at least not much; and therefore we must choose law and order, which are second best. These look at things as they exist for the most part only, and are unable to survey the whole of them. And therefore I have spoken as I have.

\par  And now we will determine what penalty he ought to pay or suffer who has hurt or wounded another. Any one may easily imagine the questions which have to be asked in all such cases: What did he wound, or whom, or how, or when? for there are innumerable particulars of this sort which greatly vary from one another. And to allow courts of law to determine all these things, or not to determine any of them, is alike impossible. There is one particular which they must determine in all cases—the question of fact. And then, again, that the legislator should not permit them to determine what punishment is to be inflicted in any of these cases, but should himself decide about all of them, small or great, is next to impossible.

\par \textbf{CLEINIAS}
\par   Then what is to be the inference?

\par \textbf{ATHENIAN}
\par   The inference is, that some things should be left to courts of law; others the legislator must decide for himself.

\par \textbf{CLEINIAS}
\par   And what ought the legislator to decide, and what ought he to leave to the courts of law?

\par \textbf{ATHENIAN}
\par   I may reply, that in a state in which the courts are bad and mute, because the judges conceal their opinions and decide causes clandestinely; or what is worse, when they are disorderly and noisy, as in a theatre, clapping or hooting in turn this or that orator—I say that then there is a very serious evil, which affects the whole state. Unfortunate is the necessity of having to legislate for such courts, but where the necessity exists, the legislator should only allow them to ordain the penalties for the smallest offences; if the state for which he is legislating be of this character, he must take most matters into his own hands and speak distinctly. But when a state has good courts, and the judges are well trained and scrupulously tested, the determination of the penalties or punishments which shall be inflicted on the guilty may fairly and with advantage be left to them. And we are not to be blamed for not legislating concerning all that large class of matters which judges far worse educated than ours would be able to determine, assigning to each offence what is due both to the perpetrator and to the sufferer. We believe those for whom we are legislating to be best able to judge, and therefore to them the greater part may be left. At the same time, as I have often said, we should exhibit to the judges, as we have done, the outline and form of the punishments to be inflicted, and then they will not transgress the just rule. That was an excellent practice, which we observed before, and which now that we are resuming the work of legislation, may with advantage be repeated by us.

\par  Let the enactment about wounding be in the following terms: If any one has a purpose and intention to slay another who is not his enemy, and whom the law does not permit him to slay, and he wounds him, but is unable to kill him, he who had the intent and has wounded him is not to be pitied—he deserves no consideration, but should be regarded as a murderer and be tried for murder. Still having respect to the fortune which has in a manner favoured him, and to the providence which in pity to him and to the wounded man saved the one from a fatal blow, and the other from an accursed fate and calamity—as a thank-offering to this deity, and in order not to oppose his will—in such a case the law will remit the punishment of death, and only compel the offender to emigrate to a neighbouring city for the rest of his life, where he shall remain in the enjoyment of all his possessions. But if he have injured the wounded man, he shall make such compensation for the injury as the court deciding the cause shall assess, and the same judges shall decide who would have decided if the man had died of his wounds. And if a child intentionally wound his parents, or a servant his master, death shall be the penalty. And if a brother or a sister intentionally wound a brother or a sister, and is found guilty, death shall be the penalty. And if a husband wound a wife, or a wife a husband, with intent to kill, let him or her undergo perpetual exile; if they have sons or daughters who are still young, the guardians shall take care of their property, and have charge of the children as orphans. If their sons are grown up, they shall be under no obligation to support the exiled parent, but they shall possess the property themselves. And if he who meets with such a misfortune has no children, the kindred of the exiled man to the degree of sons of cousins, both on the male and female side, shall meet together, and after taking counsel with the guardians of the law and the priests, shall appoint a 5040th citizen to be the heir of the house, considering and reasoning that no house of all the 5040 belongs to the inhabitant or to the whole family, but is the public and private property of the state. Now the state should seek to have its houses as holy and happy as possible. And if any one of the houses be unfortunate, and stained with impiety, and the owner leave no posterity, but dies unmarried, or married and childless, having suffered death as the penalty of murder or some other crime committed against the Gods or against his fellow-citizens, of which death is the penalty distinctly laid down in the law; or if any of the citizens be in perpetual exile, and also childless, that house shall first of all be purified and undergo expiation according to law; and then let the kinsmen of the house, as we were just now saying, and the guardians of the law, meet and consider what family there is in the state which is of the highest repute for virtue and also for good fortune, in which there are a number of sons; from that family let them take one and introduce him to the father and forefathers of the dead man as their son, and, for the sake of the omen, let him be called so, that he may be the continuer of their family, the keeper of their hearth, and the minister of their sacred rites with better fortune than his father had; and when they have made this supplication, they shall make him heir according to law, and the offending person they shall leave nameless and childless and portionless when calamities such as these overtake him.

\par  Now the boundaries of some things do not touch one another, but there is a borderland which comes in between, preventing them from touching. And we were saying that actions done from passion are of this nature, and come in between the voluntary and involuntary. If a person be convicted of having inflicted wounds in a passion, in the first place he shall pay twice the amount of the injury, if the wound be curable, or, if incurable, four times the amount of the injury; or if the wound be curable, and at the same time cause great and notable disgrace to the wounded person, he shall pay fourfold. And whenever any one in wounding another injures not only the sufferer, but also the city, and makes him incapable of defending his country against the enemy, he, besides the other penalties, shall pay a penalty for the loss which the state has incurred. And the penalty shall be, that in addition to his own times of service, he shall serve on behalf of the disabled person, and shall take his place in war; or, if he refuse, he shall be liable to be convicted by law of refusal to serve. The compensation for the injury, whether to be twofold or threefold or fourfold, shall be fixed by the judges who convict him. And if, in like manner, a brother wounds a brother, the parents and kindred of either sex, including the children of cousins, whether on the male or female side, shall meet, and when they have judged the cause, they shall entrust the assessment of damages to the parents, as is natural; and if the estimate be disputed, then the kinsmen on the male side shall make the estimate, or if they cannot, they shall commit the matter to the guardians of the law. And when similar charges of wounding are brought by children against their parents, those who are more than sixty years of age, having children of their own, not adopted, shall be required to decide; and if any one is convicted, they shall determine whether he or she ought to die, or suffer some other punishment either greater than death, or, at any rate, not much less. A kinsman of the offender shall not be allowed to judge the cause, not even if he be of the age which is prescribed by the law. If a slave in a fit of anger wound a freeman, the owner of the slave shall give him up to the wounded man, who may do as he pleases with him, and if he do not give him up he shall himself make good the injury. And if any one says that the slave and the wounded man are conspiring together, let him argue the point, and if he is cast, he shall pay for the wrong three times over, but if he gains his case, the freeman who conspired with the slave shall be liable to an action for kidnapping. And if any one unintentionally wounds another he shall simply pay for the harm, for no legislator is able to control chance. In such a case the judges shall be the same as those who are appointed in the case of children suing their parents; and they shall estimate the amount of the injury.

\par  All the preceding injuries and every kind of assault are deeds of violence; and every man, woman, or child ought to consider that the elder has the precedence of the younger in honour, both among the Gods and also among men who would live in security and happiness. Wherefore it is a foul thing and hateful to the Gods to see an elder man assaulted by a younger in the city, and it is reasonable that a young man when struck by an elder should lightly endure his anger, laying up in store for himself a like honour when he is old. Let this be the law: Every one shall reverence his elder in word and deed; he shall respect any one who is twenty years older than himself, whether male or female, regarding him or her as his father or mother; and he shall abstain from laying hands on any one who is of an age to have been his father or mother, out of reverence to the Gods who preside over birth; similarly he shall keep his hands from a stranger, whether he be an old inhabitant or newly arrived; he shall not venture to correct such an one by blows, either as the aggressor or in self-defence. If he thinks that some stranger has struck him out of wantonness or insolence, and ought to be punished, he shall take him to the wardens of the city, but let him not strike him, that the stranger may be kept far away from the possibility of lifting up his hand against a citizen, and let the wardens of the city take the offender and examine him, not forgetting their duty to the God of Strangers, and in case the stranger appears to have struck the citizen unjustly, let them inflict upon him as many blows with the scourge as he was himself inflicted, and quell his presumption. But if he be innocent, they shall threaten and rebuke the man who arrested him, and let them both go. If a person strikes another of the same age or somewhat older than himself, who has no children, whether he be an old man who strikes an old man or a young man who strikes a young man, let the person struck defend himself in the natural way without a weapon and with his hands only. He who, being more than forty years of age, dares to fight with another, whether he be the aggressor or in self-defence, shall be regarded as rude and ill-mannered and slavish—this will be a disgraceful punishment, and therefore suitable to him. The obedient nature will readily yield to such exhortations, but the disobedient, who heeds not the prelude, shall have the law ready for him: If any man smite another who is older than himself, either by twenty or by more years, in the first place, he who is at hand, not being younger than the combatants, nor their equal in age, shall separate them, or be disgraced according to law; but if he be the equal in age of the person who is struck or younger, he shall defend the person injured as he would a brother or father or still older relative. Further, let him who dares to smite an elder be tried for assault, as I have said, and if he be found guilty, let him be imprisoned for a period of not less than a year, or if the judges approve of a longer period, their decision shall be final. But if a stranger or metic smite one who is older by twenty years or more, the same law shall hold about the bystanders assisting, and he who is found guilty in such a suit, if he be a stranger but not resident, shall be imprisoned during a period of two years; and a metic who disobeys the laws shall be imprisoned for three years, unless the court assign him a longer term. And let him who was present in any of these cases and did not assist according to law be punished, if he be of the highest class, by paying a fine of a mina; or if he be of the second class, of fifty drachmas; or if of the third class, by a fine of thirty drachmas; or if he be of the fourth class, by a fine of twenty drachmas; and the generals and taxiarchs and phylarchs and hipparchs shall form the court in such cases.

\par  Laws are partly framed for the sake of good men, in order to instruct them how they may live on friendly terms with one another, and partly for the sake of those who refuse to be instructed, whose spirit cannot be subdued, or softened, or hindered from plunging into evil. These are the persons who cause the word to be spoken which I am about to utter; for them the legislator legislates of necessity, and in the hope that there may be no need of his laws. He who shall dare to lay violent hands upon his father or mother, or any still older relative, having no fear either of the wrath of the Gods above, or of the punishments that are spoken of in the world below, but transgresses in contempt of ancient and universal traditions as though he were too wise to believe in them, requires some extreme measure of prevention. Now death is not the worst that can happen to men; far worse are the punishments which are said to pursue them in the world below. But although they are most true tales, they work on such souls no prevention; for if they had any effect there would be no slayers of mothers, or impious hands lifted up against parents; and therefore the punishments of this world which are inflicted during life ought not in such cases to fall short, if possible, of the terrors of the world below. Let our enactment then be as follows: If a man dare to strike his father or his mother, or their fathers or mothers, he being at the time of sound mind, then let any one who is at hand come to the rescue as has been already said, and the metic or stranger who comes to the rescue shall be called to the first place in the games; but if he do not come he shall suffer the punishment of perpetual exile. He who is not a metic, if he comes to the rescue, shall have praise, and if he do not come, blame. And if a slave come to the rescue, let him be made free, but if he do not come to the rescue, let him receive 100 strokes of the whip, by order of the wardens of the agora, if the occurrence take place in the agora; or if somewhere in the city beyond the limits of the agora, any warden of the city who is in residence shall punish him; or if in the country, then the commanders of the wardens of the country. If those who are near at the time be inhabitants of the same place, whether they be youths, or men, or women, let them come to the rescue and denounce him as the impious one; and he who does not come to the rescue shall fall under the curse of Zeus, the God of kindred and of ancestors, according to law. And if any one is found guilty of assaulting a parent, let him in the first place be forever banished from the city into the country, and let him abstain from the temples; and if he do not abstain, the wardens of the country shall punish him with blows, or in any way which they please, and if he return he shall be put to death. And if any freeman eat or drink, or have any other sort of intercourse with him, or only meeting him have voluntarily touched him, he shall not enter into any temple, nor into the agora, nor into the city, until he is purified; for he should consider that he has become tainted by a curse. And if he disobeys the law, and pollutes the city and the temples contrary to law, and one of the magistrates sees him and does not indict him, when he gives in his account this omission shall be a most serious charge.

\par  If a slave strike a freeman, whether a stranger or a citizen, let any one who is present come to the rescue, or pay the penalty already mentioned; and let the bystanders bind him, and deliver him up to the injured person, and he receiving him shall put him in chains, and inflict on him as many stripes as he pleases; but having punished him he must surrender him to his master according to law, and not deprive him of his property. Let the law be as follows: The slave who strikes a freeman, not at the command of the magistrates, his owner shall receive bound from the man whom he has stricken, and not release him until the slave has persuaded the man whom he has stricken that he ought to be released. And let there be the same laws about women in relation to women, and about men and women in relation to one another.

\par 
\section{
      BOOK X.
    }
\par  And now having spoken of assaults, let us sum up all acts of violence under a single law, which shall be as follows: No one shall take or carry away any of his neighbour's goods, neither shall he use anything which is his neighbour's without the consent of the owner; for these are the offences which are and have been, and will ever be, the source of all the aforesaid evils. The greatest of them are excesses and insolences of youth, and are offences against the greatest when they are done against religion; and especially great when in violation of public and holy rites, or of the partly-common rites in which tribes and phratries share; and in the second degree great when they are committed against private rites and sepulchres, and in the third degree (not to repeat the acts formerly mentioned), when insults are offered to parents; the fourth kind of violence is when any one, regardless of the authority of the rulers, takes or carries away or makes use of anything which belongs to them, not having their consent; and the fifth kind is when the violation of the civil rights of an individual demands reparation. There should be a common law embracing all these cases. For we have already said in general terms what shall be the punishment of sacrilege, whether fraudulent or violent, and now we have to determine what is to be the punishment of those who speak or act insolently toward the Gods. But first we must give them an admonition which may be in the following terms: No one who in obedience to the laws believed that there were Gods, ever intentionally did any unholy act, or uttered any unlawful word; but he who did must have supposed one of three things—either that they did not exist—which is the first possibility, or secondly, that, if they did, they took no care of man, or thirdly, that they were easily appeased and turned aside from their purpose by sacrifices and prayers.

\par \textbf{CLEINIAS}
\par   What shall we say or do to these persons?

\par \textbf{ATHENIAN}
\par   My good friend, let us first hear the jests which I suspect that they in their superiority will utter against us.

\par \textbf{CLEINIAS}
\par   What jests?

\par \textbf{ATHENIAN}
\par   They will make some irreverent speech of this sort:  'O inhabitants of Athens, and Sparta, and Cnosus,' they will reply, 'in that you speak truly; for some of us deny the very existence of the Gods, while others, as you say, are of opinion that they do not care about us; and others that they are turned from their course by gifts. Now we have a right to claim, as you yourself allowed, in the matter of laws, that before you are hard upon us and threaten us, you should argue with us and convince us—you should first attempt to teach and persuade us that there are Gods by reasonable evidences, and also that they are too good to be unrighteous, or to be propitiated, or turned from their course by gifts. For when we hear such things said of them by those who are esteemed to be the best of poets, and orators, and prophets, and priests, and by innumerable others, the thoughts of most of us are not set upon abstaining from unrighteous acts, but upon doing them and atoning for them. When lawgivers profess that they are gentle and not stern, we think that they should first of all use persuasion to us, and show us the existence of Gods, if not in a better manner than other men, at any rate in a truer; and who knows but that we shall hearken to you? If then our request is a fair one, please to accept our challenge.

\par \textbf{CLEINIAS}
\par   But is there any difficulty in proving the existence of the Gods?

\par \textbf{ATHENIAN}
\par   How would you prove it?

\par \textbf{CLEINIAS}
\par   How? In the first place, the earth and the sun, and the stars and the universe, and the fair order of the seasons, and the division of them into years and months, furnish proofs of their existence, and also there is the fact that all Hellenes and barbarians believe in them.

\par \textbf{ATHENIAN}
\par   I fear, my sweet friend, though I will not say that I much regard, the contempt with which the profane will be likely to assail us. For you do not understand the nature of their complaint, and you fancy that they rush into impiety only from a love of sensual pleasure.

\par \textbf{CLEINIAS}
\par   Why, Stranger, what other reason is there?

\par \textbf{ATHENIAN}
\par   One which you who live in a different atmosphere would never guess.

\par \textbf{CLEINIAS}
\par   What is it?

\par \textbf{ATHENIAN}
\par   A very grievous sort of ignorance which is imagined to be the greatest wisdom.

\par \textbf{CLEINIAS}
\par   What do you mean?

\par \textbf{ATHENIAN}
\par   At Athens there are tales preserved in writing which the virtue of your state, as I am informed, refuses to admit. They speak of the Gods in prose as well as verse, and the oldest of them tell of the origin of the heavens and of the world, and not far from the beginning of their story they proceed to narrate the birth of the Gods, and how after they were born they behaved to one another. Whether these stories have in other ways a good or a bad influence, I should not like to be severe upon them, because they are ancient; but, looking at them with reference to the duties of children to their parents, I cannot praise them, or think that they are useful, or at all true. Of the words of the ancients I have nothing more to say; and I should wish to say of them only what is pleasing to the Gods. But as to our younger generation and their wisdom, I cannot let them off when they do mischief. For do but mark the effect of their words:  when you and I argue for the existence of the Gods, and produce the sun, moon, stars, and earth, claiming for them a divine being, if we would listen to the aforesaid philosophers we should say that they are earth and stones only, which can have no care at all of human affairs, and that all religion is a cooking up of words and a make-believe.

\par \textbf{CLEINIAS}
\par   One such teacher, O stranger, would be bad enough, and you imply that there are many of them, which is worse.

\par \textbf{ATHENIAN}
\par   Well, then; what shall we say or do? Shall we assume that some one is accusing us among unholy men, who are trying to escape from the effect of our legislation; and that they say of us—How dreadful that you should legislate on the supposition that there are Gods! Shall we make a defence of ourselves? or shall we leave them and return to our laws, lest the prelude should become longer than the law? For the discourse will certainly extend to great length, if we are to treat the impiously disposed as they desire, partly demonstrating to them at some length the things of which they demand an explanation, partly making them afraid or dissatisfied, and then proceed to the requisite enactments.

\par \textbf{CLEINIAS}
\par   Yes, Stranger; but then how often have we repeated already that on the present occasion there is no reason why brevity should be preferred to length; for who is 'at our heels?' as the saying goes, and it would be paltry and ridiculous to prefer the shorter to the better. It is a matter of no small consequence, in some way or other to prove that there are Gods, and that they are good, and regard justice more than men do. The demonstration of this would be the best and noblest prelude of all our laws. And therefore, without impatience, and without hurry, let us unreservedly consider the whole matter, summoning up all the power of persuasion which we possess.

\par \textbf{ATHENIAN}
\par   Seeing you thus in earnest, I would fain offer up a prayer that I may succeed:  but I must proceed at once. Who can be calm when he is called upon to prove the existence of the Gods? Who can avoid hating and abhorring the men who are and have been the cause of this argument; I speak of those who will not believe the tales which they have heard as babes and sucklings from their mothers and nurses, repeated by them both in jest and earnest, like charms, who have also heard them in the sacrificial prayers, and seen sights accompanying them—sights and sounds delightful to children—and their parents during the sacrifices showing an intense earnestness on behalf of their children and of themselves, and with eager interest talking to the Gods, and beseeching them, as though they were firmly convinced of their existence; who likewise see and hear the prostrations and invocations which are made by Hellenes and barbarians at the rising and setting of the sun and moon, in all the vicissitudes of life, not as if they thought that there were no Gods, but as if there could be no doubt of their existence, and no suspicion of their non-existence; when men, knowing all these things, despise them on no real grounds, as would be admitted by all who have any particle of intelligence, and when they force us to say what we are now saying, how can any one in gentle terms remonstrate with the like of them, when he has to begin by proving to them the very existence of the Gods? Yet the attempt must be made; for it would be unseemly that one half of mankind should go mad in their lust of pleasure, and the other half in their indignation at such persons. Our address to these lost and perverted natures should not be spoken in passion; let us suppose ourselves to select some one of them, and gently reason with him, smothering our anger:  O my son, we will say to him, you are young, and the advance of time will make you reverse many of the opinions which you now hold. Wait awhile, and do not attempt to judge at present of the highest things; and that is the highest of which you now think nothing—to know the Gods rightly and to live accordingly. And in the first place let me indicate to you one point which is of great importance, and about which I cannot be deceived:  You and your friends are not the first who have held this opinion about the Gods. There have always been persons more or less numerous who have had the same disorder. I have known many of them, and can tell you, that no one who had taken up in youth this opinion, that the Gods do not exist, ever continued in the same until he was old; the two other notions certainly do continue in some cases, but not in many; the notion, I mean, that the Gods exist, but take no heed of human things, and the other notion that they do take heed of them, but are easily propitiated with sacrifices and prayers. As to the opinion about the Gods which may some day become clear to you, I advise you to wait and consider if it be true or not; ask of others, and above all of the legislator. In the meantime take care that you do not offend against the Gods. For the duty of the legislator is and always will be to teach you the truth of these matters.

\par \textbf{CLEINIAS}
\par   Our address, Stranger, thus far, is excellent.

\par \textbf{ATHENIAN}
\par   Quite true, Megillus and Cleinias, but I am afraid that we have unconsciously lighted on a strange doctrine.

\par \textbf{CLEINIAS}
\par   What doctrine do you mean?

\par \textbf{ATHENIAN}
\par   The wisest of all doctrines, in the opinion of many.

\par \textbf{CLEINIAS}
\par   I wish that you would speak plainer.

\par \textbf{ATHENIAN}
\par   The doctrine that all things do become, have become, and will become, some by nature, some by art, and some by chance.

\par \textbf{CLEINIAS}
\par   Is not that true?

\par \textbf{ATHENIAN}
\par   Well, philosophers are probably right; at any rate we may as well follow in their track, and examine what is the meaning of them and their disciples.

\par \textbf{CLEINIAS}
\par   By all means.

\par \textbf{ATHENIAN}
\par   They say that the greatest and fairest things are the work of nature and of chance, the lesser of art, which, receiving from nature the greater and primeval creations, moulds and fashions all those lesser works which are generally termed artificial.

\par \textbf{CLEINIAS}
\par   How is that?

\par \textbf{ATHENIAN}
\par   I will explain my meaning still more clearly. They say that fire and water, and earth and air, all exist by nature and chance, and none of them by art, and that as to the bodies which come next in order—earth, and sun, and moon, and stars—they have been created by means of these absolutely inanimate existences. The elements are severally moved by chance and some inherent force according to certain affinities among them—of hot with cold, or of dry with moist, or of soft with hard, and according to all the other accidental admixtures of opposites which have been formed by necessity. After this fashion and in this manner the whole heaven has been created, and all that is in the heaven, as well as animals and all plants, and all the seasons come from these elements, not by the action of mind, as they say, or of any God, or from art, but as I was saying, by nature and chance only. Art sprang up afterwards and out of these, mortal and of mortal birth, and produced in play certain images and very partial imitations of the truth, having an affinity to one another, such as music and painting create and their companion arts. And there are other arts which have a serious purpose, and these co-operate with nature, such, for example, as medicine, and husbandry, and gymnastic. And they say that politics co-operate with nature, but in a less degree, and have more of art; also that legislation is entirely a work of art, and is based on assumptions which are not true.

\par \textbf{CLEINIAS}
\par   How do you mean?

\par \textbf{ATHENIAN}
\par   In the first place, my dear friend, these people would say that the Gods exist not by nature, but by art, and by the laws of states, which are different in different places, according to the agreement of those who make them; and that the honourable is one thing by nature and another thing by law, and that the principles of justice have no existence at all in nature, but that mankind are always disputing about them and altering them; and that the alterations which are made by art and by law have no basis in nature, but are of authority for the moment and at the time at which they are made. These, my friends, are the sayings of wise men, poets and prose writers, which find a way into the minds of youth. They are told by them that the highest right is might, and in this way the young fall into impieties, under the idea that the Gods are not such as the law bids them imagine; and hence arise factions, these philosophers inviting them to lead a true life according to nature, that is, to live in real dominion over others, and not in legal subjection to them.

\par \textbf{CLEINIAS}
\par   What a dreadful picture, Stranger, have you given, and how great is the injury which is thus inflicted on young men to the ruin both of states and families!

\par \textbf{ATHENIAN}
\par   True, Cleinias; but then what should the lawgiver do when this evil is of long standing? should he only rise up in the state and threaten all mankind, proclaiming that if they will not say and think that the Gods are such as the law ordains (and this may be extended generally to the honourable, the just, and to all the highest things, and to all that relates to virtue and vice), and if they will not make their actions conform to the copy which the law gives them, then he who refuses to obey the law shall die, or suffer stripes and bonds, or privation of citizenship, or in some cases be punished by loss of property and exile? Should he not rather, when he is making laws for men, at the same time infuse the spirit of persuasion into his words, and mitigate the severity of them as far as he can?

\par \textbf{CLEINIAS}
\par   Why, Stranger, if such persuasion be at all possible, then a legislator who has anything in him ought never to weary of persuading men; he ought to leave nothing unsaid in support of the ancient opinion that there are Gods, and of all those other truths which you were just now mentioning; he ought to support the law and also art, and acknowledge that both alike exist by nature, and no less than nature, if they are the creations of mind in accordance with right reason, as you appear to me to maintain, and I am disposed to agree with you in thinking.

\par \textbf{ATHENIAN}
\par   Yes, my enthusiastic Cleinias; but are not these things when spoken to a multitude hard to be understood, not to mention that they take up a dismal length of time?

\par \textbf{CLEINIAS}
\par   Why, Stranger, shall we, whose patience failed not when drinking or music were the themes of discourse, weary now of discoursing about the Gods, and about divine things? And the greatest help to rational legislation is that the laws when once written down are always at rest; they can be put to the test at any future time, and therefore, if on first hearing they seem difficult, there is no reason for apprehension about them, because any man however dull can go over them and consider them again and again; nor if they are tedious but useful, is there any reason or religion, as it seems to me, in any man refusing to maintain the principles of them to the utmost of his power.

\par \textbf{MEGILLUS}
\par   Stranger, I like what Cleinias is saying.

\par \textbf{ATHENIAN}
\par   Yes, Megillus, and we should do as he proposes; for if impious discourses were not scattered, as I may say, throughout the world, there would have been no need for any vindication of the existence of the Gods—but seeing that they are spread far and wide, such arguments are needed; and who should come to the rescue of the greatest laws, when they are being undermined by bad men, but the legislator himself?

\par \textbf{MEGILLUS}
\par   There is no more proper champion of them.

\par \textbf{ATHENIAN}
\par   Well, then, tell me, Cleinias—for I must ask you to be my partner—does not he who talks in this way conceive fire and water and earth and air to be the first elements of all things? these he calls nature, and out of these he supposes the soul to be formed afterwards; and this is not a mere conjecture of ours about his meaning, but is what he really means.

\par \textbf{CLEINIAS}
\par   Very true.

\par \textbf{ATHENIAN}
\par   Then, by Heaven, we have discovered the source of this vain opinion of all those physical investigators; and I would have you examine their arguments with the utmost care, for their impiety is a very serious matter; they not only make a bad and mistaken use of argument, but they lead away the minds of others:  that is my opinion of them.

\par \textbf{CLEINIAS}
\par   You are right; but I should like to know how this happens.

\par \textbf{ATHENIAN}
\par   I fear that the argument may seem singular.

\par \textbf{CLEINIAS}
\par   Do not hesitate, Stranger; I see that you are afraid of such a discussion carrying you beyond the limits of legislation. But if there be no other way of showing our agreement in the belief that there are Gods, of whom the law is said now to approve, let us take this way, my good sir.

\par \textbf{ATHENIAN}
\par   Then I suppose that I must repeat the singular argument of those who manufacture the soul according to their own impious notions; they affirm that which is the first cause of the generation and destruction of all things, to be not first, but last, and that which is last to be first, and hence they have fallen into error about the true nature of the Gods.

\par \textbf{CLEINIAS}
\par   Still I do not understand you.

\par \textbf{ATHENIAN}
\par   Nearly all of them, my friends, seem to be ignorant of the nature and power of the soul, especially in what relates to her origin:  they do not know that she is among the first of things, and before all bodies, and is the chief author of their changes and transpositions. And if this is true, and if the soul is older than the body, must not the things which are of the soul's kindred be of necessity prior to those which appertain to the body?

\par \textbf{CLEINIAS}
\par   Certainly.

\par \textbf{ATHENIAN}
\par   Then thought and attention and mind and art and law will be prior to that which is hard and soft and heavy and light; and the great and primitive works and actions will be works of art; they will be the first, and after them will come nature and works of nature, which however is a wrong term for men to apply to them; these will follow, and will be under the government of art and mind.

\par \textbf{CLEINIAS}
\par   But why is the word 'nature' wrong?

\par \textbf{ATHENIAN}
\par   Because those who use the term mean to say that nature is the first creative power; but if the soul turn out to be the primeval element, and not fire or air, then in the truest sense and beyond other things the soul may be said to exist by nature; and this would be true if you proved that the soul is older than the body, but not otherwise.

\par \textbf{CLEINIAS}
\par   You are quite right.

\par \textbf{ATHENIAN}
\par   Shall we, then, take this as the next point to which our attention should be directed?

\par \textbf{CLEINIAS}
\par   By all means.

\par \textbf{ATHENIAN}
\par   Let us be on our guard lest this most deceptive argument with its youthful looks, beguiling us old men, give us the slip and make a laughing-stock of us. Who knows but we may be aiming at the greater, and fail of attaining the lesser? Suppose that we three have to pass a rapid river, and I, being the youngest of the three and experienced in rivers, take upon me the duty of making the attempt first by myself; leaving you in safety on the bank, I am to examine whether the river is passable by older men like yourselves, and if such appears to be the case then I shall invite you to follow, and my experience will help to convey you across; but if the river is impassable by you, then there will have been no danger to anybody but myself—would not that seem to be a very fair proposal? I mean to say that the argument in prospect is likely to be too much for you, out of your depth and beyond your strength, and I should be afraid that the stream of my questions might create in you who are not in the habit of answering, giddiness and confusion of mind, and hence a feeling of unpleasantness and unsuitableness might arise. I think therefore that I had better first ask the questions and then answer them myself while you listen in safety; in that way I can carry on the argument until I have completed the proof that the soul is prior to the body.

\par \textbf{CLEINIAS}
\par   Excellent, Stranger, and I hope that you will do as you propose.

\par \textbf{ATHENIAN}
\par   Come, then, and if ever we are to call upon the Gods, let us call upon them now in all seriousness to come to the demonstration of their own existence. And so holding fast to the rope we will venture upon the depths of the argument. When questions of this sort are asked of me, my safest answer would appear to be as follows:  Some one says to me, 'O Stranger, are all things at rest and nothing in motion, or is the exact opposite of this true, or are some things in motion and others at rest?' To this I shall reply that some things are in motion and others at rest. 'And do not things which move move in a place, and are not the things which are at rest at rest in a place?' Certainly. 'And some move or rest in one place and some in more places than one?' You mean to say, we shall rejoin, that those things which rest at the centre move in one place, just as the circumference goes round of globes which are said to be at rest? 'Yes.' And we observe that, in the revolution, the motion which carries round the larger and the lesser circle at the same time is proportionally distributed to greater and smaller, and is greater and smaller in a certain proportion. Here is a wonder which might be thought an impossibility, that the same motion should impart swiftness and slowness in due proportion to larger and lesser circles. 'Very true.' And when you speak of bodies moving in many places, you seem to me to mean those which move from one place to another, and sometimes have one centre of motion and sometimes more than one because they turn upon their axis; and whenever they meet anything, if it be stationary, they are divided by it; but if they get in the midst between bodies which are approaching and moving towards the same spot from opposite directions, they unite with them. 'I admit the truth of what you are saying.' Also when they unite they grow, and when they are divided they waste away—that is, supposing the constitution of each to remain, or if that fails, then there is a second reason of their dissolution. 'And when are all things created and how?' Clearly, they are created when the first principle receives increase and attains to the second dimension, and from this arrives at the one which is neighbour to this, and after reaching the third becomes perceptible to sense. Everything which is thus changing and moving is in process of generation; only when at rest has it real existence, but when passing into another state it is destroyed utterly. Have we not mentioned all motions that there are, and comprehended them under their kinds and numbered them with the exception, my friends, of two?

\par \textbf{CLEINIAS}
\par   Which are they?

\par \textbf{ATHENIAN}
\par   Just the two, with which our present enquiry is concerned.

\par \textbf{CLEINIAS}
\par   Speak plainer.

\par \textbf{ATHENIAN}
\par   I suppose that our enquiry has reference to the soul?

\par \textbf{CLEINIAS}
\par   Very true.

\par \textbf{ATHENIAN}
\par   Let us assume that there is a motion able to move other things, but not to move itself; that is one kind; and there is another kind which can move itself as well as other things, working in composition and decomposition, by increase and diminution and generation and destruction—that is also one of the many kinds of motion.

\par \textbf{CLEINIAS}
\par   Granted.

\par \textbf{ATHENIAN}
\par   And we will assume that which moves other, and is changed by other, to be the ninth, and that which changes itself and others, and is coincident with every action and every passion, and is the true principle of change and motion in all that is—that we shall be inclined to call the tenth.

\par \textbf{CLEINIAS}
\par   Certainly.

\par \textbf{ATHENIAN}
\par   And which of these ten motions ought we to prefer as being the mightiest and most efficient?

\par \textbf{CLEINIAS}
\par   I must say that the motion which is able to move itself is ten thousand times superior to all the others.

\par \textbf{ATHENIAN}
\par   Very good; but may I make one or two corrections in what I have been saying?

\par \textbf{CLEINIAS}
\par   What are they?

\par \textbf{ATHENIAN}
\par   When I spoke of the tenth sort of motion, that was not quite correct.

\par \textbf{CLEINIAS}
\par   What was the error?

\par \textbf{ATHENIAN}
\par   According to the true order, the tenth was really the first in generation and power; then follows the second, which was strangely enough termed the ninth by us.

\par \textbf{CLEINIAS}
\par   What do you mean?

\par \textbf{ATHENIAN}
\par   I mean this:  when one thing changes another, and that another, of such will there be any primary changing element? How can a thing which is moved by another ever be the beginning of change? Impossible. But when the self-moved changes other, and that again other, and thus thousands upon tens of thousands of bodies are set in motion, must not the beginning of all this motion be the change of the self-moving principle?

\par \textbf{CLEINIAS}
\par   Very true, and I quite agree.

\par \textbf{ATHENIAN}
\par   Or, to put the question in another way, making answer to ourselves:  If, as most of these philosophers have the audacity to affirm, all things were at rest in one mass, which of the above-mentioned principles of motion would first spring up among them?

\par \textbf{CLEINIAS}
\par   Clearly the self-moving; for there could be no change in them arising out of any external cause; the change must first take place in themselves.

\par \textbf{ATHENIAN}
\par   Then we must say that self-motion being the origin of all motions, and the first which arises among things at rest as well as among things in motion, is the eldest and mightiest principle of change, and that which is changed by another and yet moves other is second.

\par \textbf{CLEINIAS}
\par   Quite true.

\par \textbf{ATHENIAN}
\par   At this stage of the argument let us put a question.

\par \textbf{CLEINIAS}
\par   What question?

\par \textbf{ATHENIAN}
\par   If we were to see this power existing in any earthy, watery, or fiery substance, simple or compound—how should we describe it?

\par \textbf{CLEINIAS}
\par   You mean to ask whether we should call such a self-moving power life?

\par \textbf{ATHENIAN}
\par   I do.

\par \textbf{CLEINIAS}
\par   Certainly we should.

\par \textbf{ATHENIAN}
\par   And when we see soul in anything, must we not do the same—must we not admit that this is life?

\par \textbf{CLEINIAS}
\par   We must.

\par \textbf{ATHENIAN}
\par   And now, I beseech you, reflect—you would admit that we have a threefold knowledge of things?

\par \textbf{CLEINIAS}
\par   What do you mean?

\par \textbf{ATHENIAN}
\par   I mean that we know the essence, and that we know the definition of the essence, and the name—these are the three; and there are two questions which may be raised about anything.

\par \textbf{CLEINIAS}
\par   How two?

\par \textbf{ATHENIAN}
\par   Sometimes a person may give the name and ask the definition; or he may give the definition and ask the name. I may illustrate what I mean in this way.

\par \textbf{CLEINIAS}
\par   How?

\par \textbf{ATHENIAN}
\par   Number like some other things is capable of being divided into equal parts; when thus divided, number is named 'even,' and the definition of the name 'even' is 'number divisible into two equal parts'?

\par \textbf{CLEINIAS}
\par   True.

\par \textbf{ATHENIAN}
\par   I mean, that when we are asked about the definition and give the name, or when we are asked about the name and give the definition—in either case, whether we give name or definition, we speak of the same thing, calling 'even' the number which is divided into two equal parts.

\par \textbf{CLEINIAS}
\par   Quite true.

\par \textbf{ATHENIAN}
\par   And what is the definition of that which is named 'soul'? Can we conceive of any other than that which has been already given—the motion which can move itself?

\par \textbf{CLEINIAS}
\par   You mean to say that the essence which is defined as the self-moved is the same with that which has the name soul?

\par \textbf{ATHENIAN}
\par   Yes; and if this is true, do we still maintain that there is anything wanting in the proof that the soul is the first origin and moving power of all that is, or has become, or will be, and their contraries, when she has been clearly shown to be the source of change and motion in all things?

\par \textbf{CLEINIAS}
\par   Certainly not; the soul as being the source of motion, has been most satisfactorily shown to be the oldest of all things.

\par \textbf{ATHENIAN}
\par   And is not that motion which is produced in another, by reason of another, but never has any self-moving power at all, being in truth the change of an inanimate body, to be reckoned second, or by any lower number which you may prefer?

\par \textbf{CLEINIAS}
\par   Exactly.

\par \textbf{ATHENIAN}
\par   Then we are right, and speak the most perfect and absolute truth, when we say that the soul is prior to the body, and that the body is second and comes afterwards, and is born to obey the soul, which is the ruler?

\par \textbf{CLEINIAS}
\par   Nothing can be more true.

\par \textbf{ATHENIAN}
\par   Do you remember our old admission, that if the soul was prior to the body the things of the soul were also prior to those of the body?

\par \textbf{CLEINIAS}
\par   Certainly.

\par \textbf{ATHENIAN}
\par   Then characters and manners, and wishes and reasonings, and true opinions, and reflections, and recollections are prior to length and breadth and depth and strength of bodies, if the soul is prior to the body.

\par \textbf{CLEINIAS}
\par   To be sure.

\par \textbf{ATHENIAN}
\par   In the next place, we must not of necessity admit that the soul is the cause of good and evil, base and honourable, just and unjust, and of all other opposites, if we suppose her to be the cause of all things?

\par \textbf{CLEINIAS}
\par   We must.

\par \textbf{ATHENIAN}
\par   And as the soul orders and inhabits all things that move, however moving, must we not say that she orders also the heavens?

\par \textbf{CLEINIAS}
\par   Of course.

\par \textbf{ATHENIAN}
\par   One soul or more? More than one—I will answer for you; at any rate, we must not suppose that there are less than two—one the author of good, and the other of evil.

\par \textbf{CLEINIAS}
\par   Very true.

\par \textbf{ATHENIAN}
\par   Yes, very true; the soul then directs all things in heaven, and earth, and sea by her movements, and these are described by the terms—will, consideration, attention, deliberation, opinion true and false, joy and sorrow, confidence, fear, hatred, love, and other primary motions akin to these; which again receive the secondary motions of corporeal substances, and guide all things to growth and decay, to composition and decomposition, and to the qualities which accompany them, such as heat and cold, heaviness and lightness, hardness and softness, blackness and whiteness, bitterness and sweetness, and all those other qualities which the soul uses, herself a goddess, when truly receiving the divine mind she disciplines all things rightly to their happiness; but when she is the companion of folly, she does the very contrary of all this. Shall we assume so much, or do we still entertain doubts?

\par \textbf{CLEINIAS}
\par   There is no room at all for doubt.

\par \textbf{ATHENIAN}
\par   Shall we say then that it is the soul which controls heaven and earth, and the whole world? that it is a principle of wisdom and virtue, or a principle which has neither wisdom nor virtue? Suppose that we make answer as follows:

\par \textbf{CLEINIAS}
\par   How would you answer?

\par \textbf{ATHENIAN}
\par   If, my friend, we say that the whole path and movement of heaven, and of all that is therein, is by nature akin to the movement and revolution and calculation of mind, and proceeds by kindred laws, then, as is plain, we must say that the best soul takes care of the world and guides it along the good path.

\par \textbf{CLEINIAS}
\par   True.

\par \textbf{ATHENIAN}
\par   But if the world moves wildly and irregularly, then the evil soul guides it.

\par \textbf{CLEINIAS}
\par   True again.

\par \textbf{ATHENIAN}
\par   Of what nature is the movement of mind? To this question it is not easy to give an intelligent answer; and therefore I ought to assist you in framing one.

\par \textbf{CLEINIAS}
\par   Very good.

\par \textbf{ATHENIAN}
\par   Then let us not answer as if we would look straight at the sun, making ourselves darkness at midday—I mean as if we were under the impression that we could see with mortal eyes, or know adequately the nature of mind—it will be safer to look at the image only.

\par \textbf{CLEINIAS}
\par   What do you mean?

\par \textbf{ATHENIAN}
\par   Let us select of the ten motions the one which mind chiefly resembles; this I will bring to your recollection, and will then make the answer on behalf of us all.

\par \textbf{CLEINIAS}
\par   That will be excellent.

\par \textbf{ATHENIAN}
\par   You will surely remember our saying that all things were either at rest or in motion?

\par \textbf{CLEINIAS}
\par   I do.

\par \textbf{ATHENIAN}
\par   And that of things in motion some were moving in one place, and others in more than one?

\par \textbf{CLEINIAS}
\par   Yes.

\par \textbf{ATHENIAN}
\par   Of these two kinds of motion, that which moves in one place must move about a centre like globes made in a lathe, and is most entirely akin and similar to the circular movement of mind.

\par \textbf{CLEINIAS}
\par   What do you mean?

\par \textbf{ATHENIAN}
\par   In saying that both mind and the motion which is in one place move in the same and like manner, in and about the same, and in relation to the same, and according to one proportion and order, and are like the motion of a globe, we invented a fair image, which does no discredit to our ingenuity.

\par \textbf{CLEINIAS}
\par   It does us great credit.

\par \textbf{ATHENIAN}
\par   And the motion of the other sort which is not after the same manner, nor in the same, nor about the same, nor in relation to the same, nor in one place, nor in order, nor according to any rule or proportion, may be said to be akin to senselessness and folly?

\par \textbf{CLEINIAS}
\par   That is most true.

\par \textbf{ATHENIAN}
\par   Then, after what has been said, there is no difficulty in distinctly stating, that since soul carries all things round, either the best soul or the contrary must of necessity carry round and order and arrange the revolution of the heaven.

\par \textbf{CLEINIAS}
\par   And judging from what has been said, Stranger, there would be impiety in asserting that any but the most perfect soul or souls carries round the heavens.

\par \textbf{ATHENIAN}
\par   You have understood my meaning right well, Cleinias, and now let me ask you another question.

\par \textbf{CLEINIAS}
\par   What are you going to ask?

\par \textbf{ATHENIAN}
\par   If the soul carries round the sun and moon, and the other stars, does she not carry round each individual of them?

\par \textbf{CLEINIAS}
\par   Certainly.

\par \textbf{ATHENIAN}
\par   Then of one of them let us speak, and the same argument will apply to all.

\par \textbf{CLEINIAS}
\par   Which will you take?

\par \textbf{ATHENIAN}
\par   Every one sees the body of the sun, but no one sees his soul, nor the soul of any other body living or dead; and yet there is great reason to believe that this nature, unperceived by any of our senses, is circumfused around them all, but is perceived by mind; and therefore by mind and reflection only let us apprehend the following point.

\par \textbf{CLEINIAS}
\par   What is that?

\par \textbf{ATHENIAN}
\par   If the soul carries round the sun, we shall not be far wrong in supposing one of three alternatives.

\par \textbf{CLEINIAS}
\par   What are they?

\par \textbf{ATHENIAN}
\par   Either the soul which moves the sun this way and that, resides within the circular and visible body, like the soul which carries us about every way; or the soul provides herself with an external body of fire or air, as some affirm, and violently propels body by body; or thirdly, she is without such a body, but guides the sun by some extraordinary and wonderful power.

\par \textbf{CLEINIAS}
\par   Yes, certainly; the soul can only order all things in one of these three ways.

\par \textbf{ATHENIAN}
\par   And this soul of the sun, which is therefore better than the sun, whether taking the sun about in a chariot to give light to men, or acting from without, or in whatever way, ought by every man to be deemed a God.

\par \textbf{CLEINIAS}
\par   Yes, by every man who has the least particle of sense.

\par \textbf{ATHENIAN}
\par   And of the stars too, and of the moon, and of the years and months and seasons, must we not say in like manner, that since a soul or souls having every sort of excellence are the causes of all of them, those souls are Gods, whether they are living beings and reside in bodies, and in this way order the whole heaven, or whatever be the place and mode of their existence—and will any one who admits all this venture to deny that all things are full of Gods?

\par \textbf{CLEINIAS}
\par   No one, Stranger, would be such a madman.

\par \textbf{ATHENIAN}
\par   And now, Megillus and Cleinias, let us offer terms to him who has hitherto denied the existence of the Gods, and leave him.

\par \textbf{CLEINIAS}
\par   What terms?

\par \textbf{ATHENIAN}
\par   Either he shall teach us that we were wrong in saying that the soul is the original of all things, and arguing accordingly; or, if he be not able to say anything better, then he must yield to us and live for the remainder of his life in the belief that there are Gods. Let us see, then, whether we have said enough or not enough to those who deny that there are Gods.

\par \textbf{CLEINIAS}
\par   Certainly, quite enough, Stranger.

\par \textbf{ATHENIAN}
\par   Then to them we will say no more. And now we are to address him who, believing that there are Gods, believes also that they take no heed of human affairs:  To him we say—O thou best of men, in believing that there are Gods you are led by some affinity to them, which attracts you towards your kindred and makes you honour and believe in them. But the fortunes of evil and unrighteous men in private as well as public life, which, though not really happy, are wrongly counted happy in the judgment of men, and are celebrated both by poets and prose writers—these draw you aside from your natural piety. Perhaps you have seen impious men growing old and leaving their children's children in high offices, and their prosperity shakes your faith—you have known or heard or been yourself an eyewitness of many monstrous impieties, and have beheld men by such criminal means from small beginnings attaining to sovereignty and the pinnacle of greatness; and considering all these things you do not like to accuse the Gods of them, because they are your relatives; and so from some want of reasoning power, and also from an unwillingness to find fault with them, you have come to believe that they exist indeed, but have no thought or care of human things. Now, that your present evil opinion may not grow to still greater impiety, and that we may if possible use arguments which may conjure away the evil before it arrives, we will add another argument to that originally addressed to him who utterly denied the existence of the Gods. And do you, Megillus and Cleinias, answer for the young man as you did before; and if any impediment comes in our way, I will take the word out of your mouths, and carry you over the river as I did just now.

\par \textbf{CLEINIAS}
\par   Very good; do as you say, and we will help you as well as we can.

\par \textbf{ATHENIAN}
\par   There will probably be no difficulty in proving to him that the Gods care about the small as well as about the great. For he was present and heard what was said, that they are perfectly good, and that the care of all things is most entirely natural to them.

\par \textbf{CLEINIAS}
\par   No doubt he heard that.

\par \textbf{ATHENIAN}
\par   Let us consider together in the next place what we mean by this virtue which we ascribe to them. Surely we should say that to be temperate and to possess mind belongs to virtue, and the contrary to vice?

\par \textbf{CLEINIAS}
\par   Certainly.

\par \textbf{ATHENIAN}
\par   Yes; and courage is a part of virtue, and cowardice of vice?

\par \textbf{CLEINIAS}
\par   True.

\par \textbf{ATHENIAN}
\par   And the one is honourable, and the other dishonourable?

\par \textbf{CLEINIAS}
\par   To be sure.

\par \textbf{ATHENIAN}
\par   And the one, like other meaner things, is a human quality, but the Gods have no part in anything of the sort?

\par \textbf{CLEINIAS}
\par   That again is what everybody will admit.

\par \textbf{ATHENIAN}
\par   But do we imagine carelessness and idleness and luxury to be virtues? What do you think?

\par \textbf{CLEINIAS}
\par   Decidedly not.

\par \textbf{ATHENIAN}
\par   They rank under the opposite class?

\par \textbf{CLEINIAS}
\par   Yes.

\par \textbf{ATHENIAN}
\par   And their opposites, therefore, would fall under the opposite class?

\par \textbf{CLEINIAS}
\par   Yes.

\par \textbf{ATHENIAN}
\par   But are we to suppose that one who possesses all these good qualities will be luxurious and heedless and idle, like those whom the poet compares to stingless drones?

\par \textbf{CLEINIAS}
\par   And the comparison is a most just one.

\par \textbf{ATHENIAN}
\par   Surely God must not be supposed to have a nature which He Himself hates? he who dares to say this sort of thing must not be tolerated for a moment.

\par \textbf{CLEINIAS}
\par   Of course not. How could he have?

\par \textbf{ATHENIAN}
\par   Should we not on any principle be entirely mistaken in praising any one who has some special business entrusted to him, if he have a mind which takes care of great matters and no care of small ones? Reflect; he who acts in this way, whether he be God or man, must act from one of two principles.

\par \textbf{CLEINIAS}
\par   What are they?

\par \textbf{ATHENIAN}
\par   Either he must think that the neglect of the small matters is of no consequence to the whole, or if he knows that they are of consequence, and he neglects them, his neglect must be attributed to carelessness and indolence. Is there any other way in which his neglect can be explained? For surely, when it is impossible for him to take care of all, he is not negligent if he fails to attend to these things great or small, which a God or some inferior being might be wanting in strength or capacity to manage?

\par \textbf{CLEINIAS}
\par   Certainly not.

\par \textbf{ATHENIAN}
\par   Now, then, let us examine the offenders, who both alike confess that there are Gods, but with a difference—the one saying that they may be appeased, and the other that they have no care of small matters:  there are three of us and two of them, and we will say to them—In the first place, you both acknowledge that the Gods hear and see and know all things, and that nothing can escape them which is matter of sense and knowledge:  do you admit this?

\par \textbf{CLEINIAS}
\par   Yes.

\par \textbf{ATHENIAN}
\par   And do you admit also that they have all power which mortals and immortals can have?

\par \textbf{CLEINIAS}
\par   They will, of course, admit this also.

\par \textbf{ATHENIAN}
\par   And surely we three and they two—five in all—have acknowledged that they are good and perfect?

\par \textbf{CLEINIAS}
\par   Assuredly.

\par \textbf{ATHENIAN}
\par   But, if they are such as we conceive them to be, can we possibly suppose that they ever act in the spirit of carelessness and indolence? For in us inactivity is the child of cowardice, and carelessness of inactivity and indolence.

\par \textbf{CLEINIAS}
\par   Most true.

\par \textbf{ATHENIAN}
\par   Then not from inactivity and carelessness is any God ever negligent; for there is no cowardice in them.

\par \textbf{CLEINIAS}
\par   That is very true.

\par \textbf{ATHENIAN}
\par   Then the alternative which remains is, that if the Gods neglect the lighter and lesser concerns of the universe, they neglect them because they know that they ought not to care about such matters—what other alternative is there but the opposite of their knowing?

\par \textbf{CLEINIAS}
\par   There is none.

\par \textbf{ATHENIAN}
\par   And, O most excellent and best of men, do I understand you to mean that they are careless because they are ignorant, and do not know that they ought to take care, or that they know, and yet like the meanest sort of men, knowing the better, choose the worse because they are overcome by pleasures and pains?

\par \textbf{CLEINIAS}
\par   Impossible.

\par \textbf{ATHENIAN}
\par   Do not all human things partake of the nature of soul? And is not man the most religious of all animals?

\par \textbf{CLEINIAS}
\par   That is not to be denied.

\par \textbf{ATHENIAN}
\par   And we acknowledge that all mortal creatures are the property of the Gods, to whom also the whole of heaven belongs?

\par \textbf{CLEINIAS}
\par   Certainly.

\par \textbf{ATHENIAN}
\par   And, therefore, whether a person says that these things are to the Gods great or small—in either case it would not be natural for the Gods who own us, and who are the most careful and the best of owners, to neglect us. There is also a further consideration.

\par \textbf{CLEINIAS}
\par   What is it?

\par \textbf{ATHENIAN}
\par   Sensation and power are in an inverse ratio to each other in respect to their ease and difficulty.

\par \textbf{CLEINIAS}
\par   What do you mean?

\par \textbf{ATHENIAN}
\par   I mean that there is greater difficulty in seeing and hearing the small than the great, but more facility in moving and controlling and taking care of small and unimportant things than of their opposites.

\par \textbf{CLEINIAS}
\par   Far more.

\par \textbf{ATHENIAN}
\par   Suppose the case of a physician who is willing and able to cure some living thing as a whole—how will the whole fare at his hands if he takes care only of the greater and neglects the parts which are lesser?

\par \textbf{CLEINIAS}
\par   Decidedly not well.

\par \textbf{ATHENIAN}
\par   No better would be the result with pilots or generals, or householders or statesmen, or any other such class, if they neglected the small and regarded only the great—as the builders say, the larger stones do not lie well without the lesser.

\par \textbf{CLEINIAS}
\par   Of course not.

\par \textbf{ATHENIAN}
\par   Let us not, then, deem God inferior to human workmen, who, in proportion to their skill, finish and perfect their works, small as well as great, by one and the same art; or that God, the wisest of beings, who is both willing and able to take care, is like a lazy good-for-nothing, or a coward, who turns his back upon labour and gives no thought to smaller and easier matters, but to the greater only.

\par \textbf{CLEINIAS}
\par   Never, Stranger, let us admit a supposition about the Gods which is both impious and false.

\par \textbf{ATHENIAN}
\par   I think that we have now argued enough with him who delights to accuse the Gods of neglect.

\par \textbf{CLEINIAS}
\par   Yes.

\par \textbf{ATHENIAN}
\par   He has been forced to acknowledge that he is in error, but he still seems to me to need some words of consolation.

\par \textbf{CLEINIAS}
\par   What consolation will you offer him?

\par \textbf{ATHENIAN}
\par   Let us say to the youth:  The ruler of the universe has ordered all things with a view to the excellence and preservation of the whole, and each part, as far as may be, has an action and passion appropriate to it. Over these, down to the least fraction of them, ministers have been appointed to preside, who have wrought out their perfection with infinitesimal exactness. And one of these portions of the universe is thine own, unhappy man, which, however little, contributes to the whole; and you do not seem to be aware that this and every other creation is for the sake of the whole, and in order that the life of the whole may be blessed; and that you are created for the sake of the whole, and not the whole for the sake of you. For every physician and every skilled artist does all things for the sake of the whole, directing his effort towards the common good, executing the part for the sake of the whole, and not the whole for the sake of the part. And you are annoyed because you are ignorant how what is best for you happens to you and to the universe, as far as the laws of the common creation admit. Now, as the soul combining first with one body and then with another undergoes all sorts of changes, either of herself, or through the influence of another soul, all that remains to the player of the game is that he should shift the pieces; sending the better nature to the better place, and the worse to the worse, and so assigning to them their proper portion.

\par \textbf{CLEINIAS}
\par   In what way do you mean?

\par \textbf{ATHENIAN}
\par   In a way which may be supposed to make the care of all things easy to the Gods. If any one were to form or fashion all things without any regard to the whole—if, for example, he formed a living element of water out of fire, instead of forming many things out of one or one out of many in regular order attaining to a first or second or third birth, the transmutation would have been infinite; but now the ruler of the world has a wonderfully easy task.

\par \textbf{CLEINIAS}
\par   How so?

\par \textbf{ATHENIAN}
\par   I will explain:  When the king saw that our actions had life, and that there was much virtue in them and much vice, and that the soul and body, although not, like the Gods of popular opinion, eternal, yet having once come into existence, were indestructible (for if either of them had been destroyed, there would have been no generation of living beings); and when he observed that the good of the soul was ever by nature designed to profit men, and the evil to harm them—he, seeing all this, contrived so to place each of the parts that their position might in the easiest and best manner procure the victory of good and the defeat of evil in the whole. And he contrived a general plan by which a thing of a certain nature found a certain seat and room. But the formation of qualities he left to the wills of individuals. For every one of us is made pretty much what he is by the bent of his desires and the nature of his soul.

\par \textbf{CLEINIAS}
\par   Yes, that is probably true.

\par \textbf{ATHENIAN}
\par   Then all things which have a soul change, and possess in themselves a principle of change, and in changing move according to law and to the order of destiny:  natures which have undergone a lesser change move less and on the earth's surface, but those which have suffered more change and have become more criminal sink into the abyss, that is to say, into Hades and other places in the world below, of which the very names terrify men, and which they picture to themselves as in a dream, both while alive and when released from the body. And whenever the soul receives more of good or evil from her own energy and the strong influence of others—when she has communion with divine virtue and becomes divine, she is carried into another and better place, which is perfect in holiness; but when she has communion with evil, then she also changes the place of her life.

\par  'This is the justice of the Gods who inhabit Olympus.'

\par  O youth or young man, who fancy that you are neglected by the Gods, know that if you become worse you shall go to the worse souls, or if better to the better, and in every succession of life and death you will do and suffer what like may fitly suffer at the hands of like. This is the justice of heaven, which neither you nor any other unfortunate will ever glory in escaping, and which the ordaining powers have specially ordained; take good heed thereof, for it will be sure to take heed of you. If you say: I am small and will creep into the depths of the earth, or I am high and will fly up to heaven, you are not so small or so high but that you shall pay the fitting penalty, either here or in the world below or in some still more savage place whither you shall be conveyed. This is also the explanation of the fate of those whom you saw, who had done unholy and evil deeds, and from small beginnings had grown great, and you fancied that from being miserable they had become happy; and in their actions, as in a mirror, you seemed to see the universal neglect of the Gods, not knowing how they make all things work together and contribute to the great whole. And thinkest thou, bold man, that thou needest not to know this? he who knows it not can never form any true idea of the happiness or unhappiness of life or hold any rational discourse respecting either. If Cleinias and this our reverend company succeed in proving to you that you know not what you say of the Gods, then will God help you; but should you desire to hear more, listen to what we say to the third opponent, if you have any understanding whatsoever. For I think that we have sufficiently proved the existence of the Gods, and that they care for men: The other notion that they are appeased by the wicked, and take gifts, is what we must not concede to any one, and what every man should disprove to the utmost of his power.

\par \textbf{CLEINIAS}
\par   Very good; let us do as you say.

\par \textbf{ATHENIAN}
\par   Well, then, by the Gods themselves I conjure you to tell me—if they are to be propitiated, how are they to be propitiated? Who are they, and what is their nature? Must they not be at least rulers who have to order unceasingly the whole heaven?

\par \textbf{CLEINIAS}
\par   True.

\par \textbf{ATHENIAN}
\par   And to what earthly rulers can they be compared, or who to them? How in the less can we find an image of the greater? Are they charioteers of contending pairs of steeds, or pilots of vessels? Perhaps they might be compared to the generals of armies, or they might be likened to physicians providing against the diseases which make war upon the body, or to husbandmen observing anxiously the effects of the seasons on the growth of plants; or perhaps to shepherds of flocks. For as we acknowledge the world to be full of many goods and also of evils, and of more evils than goods, there is, as we affirm, an immortal conflict going on among us, which requires marvellous watchfulness; and in that conflict the Gods and demigods are our allies, and we are their property. Injustice and insolence and folly are the destruction of us, and justice and temperance and wisdom are our salvation; and the place of these latter is in the life of the Gods, although some vestige of them may occasionally be discerned among mankind. But upon this earth we know that there dwell souls possessing an unjust spirit, who may be compared to brute animals, which fawn upon their keepers, whether dogs or shepherds, or the best and most perfect masters; for they in like manner, as the voices of the wicked declare, prevail by flattery and prayers and incantations, and are allowed to make their gains with impunity. And this sin, which is termed dishonesty, is an evil of the same kind as what is termed disease in living bodies or pestilence in years or seasons of the year, and in cities and governments has another name, which is injustice.

\par \textbf{CLEINIAS}
\par   Quite true.

\par \textbf{ATHENIAN}
\par   What else can he say who declares that the Gods are always lenient to the doers of unjust acts, if they divide the spoil with them? As if wolves were to toss a portion of their prey to the dogs, and they, mollified by the gift, suffered them to tear the flocks. Must not he who maintains that the Gods can be propitiated argue thus?

\par \textbf{CLEINIAS}
\par   Precisely so.

\par \textbf{ATHENIAN}
\par   And to which of the above-mentioned classes of guardians would any man compare the Gods without absurdity? Will he say that they are like pilots, who are themselves turned away from their duty by 'libations of wine and the savour of fat,' and at last overturn both ship and sailors?

\par \textbf{CLEINIAS}
\par   Assuredly not.

\par \textbf{ATHENIAN}
\par   And surely they are not like charioteers who are bribed to give up the victory to other chariots?

\par \textbf{CLEINIAS}
\par   That would be a fearful image of the Gods.

\par \textbf{ATHENIAN}
\par   Nor are they like generals, or physicians, or husbandmen, or shepherds; and no one would compare them to dogs who have been silenced by wolves.

\par \textbf{CLEINIAS}
\par   A thing not to be spoken of.

\par \textbf{ATHENIAN}
\par   And are not all the Gods the chiefest of all guardians, and do they not guard our highest interests?

\par \textbf{CLEINIAS}
\par   Yes; the chiefest.

\par \textbf{ATHENIAN}
\par   And shall we say that those who guard our noblest interests, and are the best of guardians, are inferior in virtue to dogs, and to men even of moderate excellence, who would never betray justice for the sake of gifts which unjust men impiously offer them?

\par \textbf{CLEINIAS}
\par   Certainly not; nor is such a notion to be endured, and he who holds this opinion may be fairly singled out and characterized as of all impious men the wickedest and most impious.

\par \textbf{ATHENIAN}
\par   Then are the three assertions—that the Gods exist, and that they take care of men, and that they can never be persuaded to do injustice, now sufficiently demonstrated? May we say that they are?

\par \textbf{CLEINIAS}
\par   You have our entire assent to your words.

\par \textbf{ATHENIAN}
\par   I have spoken with vehemence because I am zealous against evil men; and I will tell you, dear Cleinias, why I am so. I would not have the wicked think that, having the superiority in argument, they may do as they please and act according to their various imaginations about the Gods; and this zeal has led me to speak too vehemently; but if we have at all succeeded in persuading the men to hate themselves and love their opposites, the prelude of our laws about impiety will not have been spoken in vain.

\par \textbf{CLEINIAS}
\par   So let us hope; and even if we have failed, the style of our argument will not discredit the lawgiver.

\par \textbf{ATHENIAN}
\par   After the prelude shall follow a discourse, which will be the interpreter of the law; this shall proclaim to all impious persons that they must depart from their ways and go over to the pious. And to those who disobey, let the law about impiety be as follows:  If a man is guilty of any impiety in word or deed, any one who happens to be present shall give information to the magistrates, in aid of the law; and let the magistrates who first receive the information bring him before the appointed court according to the law; and if a magistrate, after receiving information, refuses to act, he shall be tried for impiety at the instance of any one who is willing to vindicate the laws; and if any one be cast, the court shall estimate the punishment of each act of impiety; and let all such criminals be imprisoned. There shall be three prisons in the state:  the first of them is to be the common prison in the neighbourhood of the agora for the safe-keeping of the generality of offenders; another is to be in the neighbourhood of the nocturnal council, and is to be called the 'House of Reformation'; another, to be situated in some wild and desolate region in the centre of the country, shall be called by some name expressive of retribution. Now, men fall into impiety from three causes, which have been already mentioned, and from each of these causes arise two sorts of impiety, in all six, which are worth distinguishing, and should not all have the same punishment. For he who does not believe in the Gods, and yet has a righteous nature, hates the wicked and dislikes and refuses to do injustice, and avoids unrighteous men, and loves the righteous. But they who besides believing that the world is devoid of Gods are intemperate, and have at the same time good memories and quick wits, are worse; although both of them are unbelievers, much less injury is done by the one than by the other. The one may talk loosely about the Gods and about sacrifices and oaths, and perhaps by laughing at other men he may make them like himself, if he be not punished. But the other who holds the same opinions and is called a clever man, is full of stratagem and deceit—men of this class deal in prophecy and jugglery of all kinds, and out of their ranks sometimes come tyrants and demagogues and generals and hierophants of private mysteries and the Sophists, as they are termed, with their ingenious devices. There are many kinds of unbelievers, but two only for whom legislation is required; one the hypocritical sort, whose crime is deserving of death many times over, while the other needs only bonds and admonition. In like manner also the notion that the Gods take no thought of men produces two other sorts of crimes, and the notion that they may be propitiated produces two more. Assuming these divisions, let those who have been made what they are only from want of understanding, and not from malice or an evil nature, be placed by the judge in the House of Reformation, and ordered to suffer imprisonment during a period of not less than five years. And in the meantime let them have no intercourse with the other citizens, except with members of the nocturnal council, and with them let them converse with a view to the improvement of their soul's health. And when the time of their imprisonment has expired, if any of them be of sound mind let him be restored to sane company, but if not, and if he be condemned a second time, let him be punished with death. As to that class of monstrous natures who not only believe that there are no Gods, or that they are negligent, or to be propitiated, but in contempt of mankind conjure the souls of the living and say that they can conjure the dead and promise to charm the Gods with sacrifices and prayers, and will utterly overthrow individuals and whole houses and states for the sake of money—let him who is guilty of any of these things be condemned by the court to be bound according to law in the prison which is in the centre of the land, and let no freeman ever approach him, but let him receive the rations of food appointed by the guardians of the law from the hands of the public slaves; and when he is dead let him be cast beyond the borders unburied, and if any freeman assist in burying him, let him pay the penalty of impiety to any one who is willing to bring a suit against him. But if he leaves behind him children who are fit to be citizens, let the guardians of orphans take care of them, just as they would of any other orphans, from the day on which their father is convicted.

\par  In all these cases there should be one law, which will make men in general less liable to transgress in word or deed, and less foolish, because they will not be allowed to practise religious rites contrary to law. And let this be the simple form of the law: No man shall have sacred rites in a private house. When he would sacrifice, let him go to the temples and hand over his offerings to the priests and priestesses, who see to the sanctity of such things, and let him pray himself, and let any one who pleases join with him in prayer. The reason of this is as follows: Gods and temples are not easily instituted, and to establish them rightly is the work of a mighty intellect. And women especially, and men too, when they are sick or in danger, or in any sort of difficulty, or again on their receiving any good fortune, have a way of consecrating the occasion, vowing sacrifices, and promising shrines to Gods, demigods, and sons of Gods; and when they are awakened by terrible apparitions and dreams or remember visions, they find in altars and temples the remedies of them, and will fill every house and village with them, placing them in the open air, or wherever they may have had such visions; and with a view to all these cases we should obey the law. The law has also regard to the impious, and would not have them fancy that by the secret performance of these actions—by raising temples and by building altars in private houses, they can propitiate the God secretly with sacrifices and prayers, while they are really multiplying their crimes infinitely, bringing guilt from heaven upon themselves, and also upon those who permit them, and who are better men than they are; and the consequence is that the whole state reaps the fruit of their impiety, which, in a certain sense, is deserved. Assuredly God will not blame the legislator, who will enact the following law: No one shall possess shrines of the Gods in private houses, and he who is found to possess them, and perform any sacred rites not publicly authorised—supposing the offender to be some man or woman who is not guilty of any other great and impious crime—shall be informed against by him who is acquainted with the fact, which shall be announced by him to the guardians of the law; and let them issue orders that he or she shall carry away their private rites to the public temples, and if they do not persuade them, let them inflict a penalty on them until they comply. And if a person be proven guilty of impiety, not merely from childish levity, but such as grown-up men may be guilty of, whether he have sacrificed publicly or privately to any Gods, let him be punished with death, for his sacrifice is impure. Whether the deed has been done in earnest, or only from childish levity, let the guardians of the law determine, before they bring the matter into court and prosecute the offender for impiety.

\par 
\section{
      BOOK XI.
    }
\par  In the next place, dealings between man and man require to be suitably regulated. The principle of them is very simple: Thou shalt not, if thou canst help, touch that which is mine, or remove the least thing which belongs to me without my consent; and may I be of a sound mind, and do to others as I would that they should do to me. First, let us speak of treasure-trove: May I never pray the Gods to find the hidden treasure, which another has laid up for himself and his family, he not being one of my ancestors, nor lift, if I should find, such a treasure. And may I never have any dealings with those who are called diviners, and who in any way or manner counsel me to take up the deposit entrusted to the earth, for I should not gain so much in the increase of my possessions, if I take up the prize, as I should grow in justice and virtue of soul, if I abstain; and this will be a better possession to me than the other in a better part of myself; for the possession of justice in the soul is preferable to the possession of wealth. And of many things it is well said—'Move not the immovables,' and this may be regarded as one of them. And we shall do well to believe the common tradition which says, that such deeds prevent a man from having a family. Now as to him who is careless about having children and regardless of the legislator, taking up that which neither he deposited, nor any ancestor of his, without the consent of the depositor, violating the simplest and noblest of laws which was the enactment of no mean man: 'Take not up that which was not laid down by thee'—of him, I say, who despises these two legislators, and takes up, not some small matter which he has not deposited, but perhaps a great heap of treasure, what he ought to suffer at the hands of the Gods, God only knows; but I would have the first person who sees him go and tell the wardens of the city, if the occurrence has taken place in the city, or if the occurrence has taken place in the agora he shall tell the wardens of the agora, or if in the country he shall tell the wardens of the country and their commanders. When information has been received the city shall send to Delphi, and, whatever the God answers about the money and the remover of the money, that the city shall do in obedience to the oracle; the informer, if he be a freeman, shall have the honour of doing rightly, and he who informs not, the dishonour of doing wrongly; and if he be a slave who gives information, let him be freed, as he ought to be, by the state, which shall give his master the price of him; but if he do not inform he shall be punished with death. Next in order shall follow a similar law, which shall apply equally to matters great and small: If a man happens to leave behind him some part of his property, whether intentionally or unintentionally, let him who may come upon the left property suffer it to remain, reflecting that such things are under the protection of the Goddess of ways, and are dedicated to her by the law. But if any one defies the law, and takes the property home with him, let him, if the thing is of little worth, and the man who takes it a slave, be beaten with many stripes by him who meets him, being a person of not less than thirty years of age. Or if he be a freeman, in addition to being thought a mean person and a despiser of the laws, let him pay ten times the value of the treasure which he has moved to the leaver. And if some one accuses another of having anything which belongs to him, whether little or much, and the other admits that he has this thing, but denies that the property in dispute belongs to the other, if the property be registered with the magistrates according to law, the claimant shall summon the possessor, who shall bring it before the magistrates; and when it is brought into court, if it be registered in the public registers, to which of the litigants it belonged, let him take it and go his way. Or if the property be registered as belonging to some one who is not present, whoever will offer sufficient surety on behalf of the absent person that he will give it up to him, shall take it away as the representative of the other. But if the property which is deposited be not registered with the magistrates, let it remain until the time of trial with three of the eldest of the magistrates; and if it be an animal which is deposited, then he who loses the suit shall pay the magistrates for its keep, and they shall determine the cause within three days.

\par  Any one who is of sound mind may arrest his own slave, and do with him whatever he will of such things as are lawful; and he may arrest the runaway slave of any of his friends or kindred with a view to his safe-keeping. And if any one takes away him who is being carried off as a slave, intending to liberate him, he who is carrying him off shall let him go; but he who takes him away shall give three sufficient sureties; and if he give them, and not without giving them, he may take him away, but if he take him away after any other manner he shall be deemed guilty of violence, and being convicted shall pay as a penalty double the amount of the damages claimed to him who has been deprived of the slave. Any man may also carry off a freedman, if he do not pay respect or sufficient respect to him who freed him. Now the respect shall be, that the freedman go three times in the month to the hearth of the person who freed him, and offer to do whatever he ought, so far as he can; and he shall agree to make such a marriage as his former master approves. He shall not be permitted to have more property than he who gave him liberty, and what more he has shall belong to his master. The freedman shall not remain in the state more than twenty years, but like other foreigners shall go away, taking his entire property with him, unless he has the consent of the magistrates and of his former master to remain. If a freedman or any other stranger has a property greater than the census of the third class, at the expiration of thirty days from the day on which this comes to pass, he shall take that which is his and go his way, and in this case he shall not be allowed to remain any longer by the magistrates. And if any one disobeys this regulation, and is brought into court and convicted, he shall be punished with death, and his property shall be confiscated. Suits about these matters shall take place before the tribes, unless the plaintiff and defendant have got rid of the accusation either before their neighbours or before judges chosen by them. If a man lay claim to any animal or anything else which he declares to be his, let the possessor refer to the seller or to some honest and trustworthy person, who has given, or in some legitimate way made over the property to him; if he be a citizen or a metic, sojourning in the city, within thirty days, or, if the property have been delivered to him by a stranger, within five months, of which the middle month shall include the summer solstice. When goods are exchanged by selling and buying, a man shall deliver them, and receive the price of them, at a fixed place in the agora, and have done with the matter; but he shall not buy or sell anywhere else, nor give credit. And if in any other manner or in any other place there be an exchange of one thing for another, and the seller give credit to the man who buys from him, he must do this on the understanding that the law gives no protection in cases of things sold not in accordance with these regulations. Again, as to contributions, any man who likes may go about collecting contributions as a friend among friends, but if any difference arises about the collection, he is to act on the understanding that the law gives no protection in such cases. He who sells anything above the value of fifty drachmas shall be required to remain in the city for ten days, and the purchaser shall be informed of the house of the seller, with a view to the sort of charges which are apt to arise in such cases, and the restitutions which the law allows. And let legal restitution be on this wise: If a man sells a slave who is in a consumption, or who has the disease of the stone, or of strangury, or epilepsy, or some other tedious and incurable disorder of body or mind, which is not discernible to the ordinary man, if the purchaser be a physician or trainer, he shall have no right of restitution; nor shall there be any right of restitution if the seller has told the truth beforehand to the buyer. But if a skilled person sells to another who is not skilled, let the buyer appeal for restitution within six months, except in the case of epilepsy, and then the appeal may be made within a year. The cause shall be determined by such physicians as the parties may agree to choose; and the defendant, if he lose the suit, shall pay double the price at which he sold. If a private person sell to another private person, he shall have the right of restitution, and the decision shall be given as before, but the defendant, if he be cast, shall only pay back the price of the slave. If a person sells a homicide to another, and they both know of the fact, let there be no restitution in such a case, but if he do not know of the fact, there shall be a right of restitution, whenever the buyer makes the discovery; and the decision shall rest with the five youngest guardians of the law, and if the decision be that the seller was cognisant of the fact, he shall purify the house of the purchaser, according to the law of the interpreters, and shall pay back three times the purchase-money.

\par  If a man exchanges either money for money, or anything whatever for anything else, either with or without life, let him give and receive them genuine and unadulterated, in accordance with the law. And let us have a prelude about all this sort of roguery, like the preludes of our other laws. Every man should regard adulteration as of one and the same class with falsehood and deceit, concerning which the many are too fond of saying that at proper times and places the practice may often be right. But they leave the occasion, and the when, and the where, undefined and unsettled, and from this want of definiteness in their language they do a great deal of harm to themselves and to others. Now a legislator ought not to leave the matter undetermined; he ought to prescribe some limit, either greater or less. Let this be the rule prescribed: No one shall call the Gods to witness, when he says or does anything false or deceitful or dishonest, unless he would be the most hateful of mankind to them. And he is most hateful to them who takes a false oath, and pays no heed to the Gods; and in the next degree, he who tells a falsehood in the presence of his superiors. Now better men are the superiors of worse men, and in general elders are the superiors of the young; wherefore also parents are the superiors of their offspring, and men of women and children, and rulers of their subjects; for all men ought to reverence any one who is in any position of authority, and especially those who are in state offices. And this is the reason why I have spoken of these matters. For every one who is guilty of adulteration in the agora tells a falsehood, and deceives, and when he invokes the Gods, according to the customs and cautions of the wardens of the agora, he does but swear without any respect for God or man. Certainly, it is an excellent rule not lightly to defile the names of the Gods, after the fashion of men in general, who care little about piety and purity in their religious actions. But if a man will not conform to this rule, let the law be as follows: He who sells anything in the agora shall not ask two prices for that which he sells, but he shall ask one price, and if he do not obtain this, he shall take away his goods; and on that day he shall not value them either at more or less; and there shall be no praising of any goods, or oath taken about them. If a person disobeys this command, any citizen who is present, not being less than thirty years of age, may with impunity chastise and beat the swearer, but if instead of obeying the laws he takes no heed, he shall be liable to the charge of having betrayed them. If a man sells any adulterated goods and will not obey these regulations, he who knows and can prove the fact, and does prove it in the presence of the magistrates, if he be a slave or a metic, shall have the adulterated goods; but if he be a citizen, and do not pursue the charge, he shall be called a rogue, and deemed to have robbed the Gods of the agora; or if he proves the charge, he shall dedicate the goods to the Gods of the agora. He who is proved to have sold any adulterated goods, in addition to losing the goods themselves, shall be beaten with stripes—a stripe for a drachma, according to the price of the goods; and the herald shall proclaim in the agora the offence for which he is going to be beaten. The wardens of the agora and the guardians of the law shall obtain information from experienced persons about the rogueries and adulterations of the sellers, and shall write up what the seller ought and ought not to do in each case; and let them inscribe their laws on a column in front of the court of the wardens of the agora, that they may be clear instructors of those who have business in the agora. Enough has been said in what has preceded about the wardens of the city, and if anything seems to be wanting, let them communicate with the guardians of the law, and write down the omission, and place on a column in the court of the wardens of the city the primary and secondary regulations which are laid down for them about their office.

\par  After the practices of adulteration naturally follow the practices of retail trade. Concerning these, we will first of all give a word of counsel and reason, and the law shall come afterwards. Retail trade in a city is not by nature intended to do any harm, but quite the contrary; for is not he a benefactor who reduces the inequalities and incommensurabilities of goods to equality and common measure? And this is what the power of money accomplishes, and the merchant may be said to be appointed for this purpose. The hireling and the tavern-keeper, and many other occupations, some of them more and others less seemly—all alike have this object—they seek to satisfy our needs and equalize our possessions. Let us then endeavour to see what has brought retail trade into ill-odour, and wherein lies the dishonour and unseemliness of it, in order that if not entirely, we may yet partially, cure the evil by legislation. To effect this is no easy matter, and requires a great deal of virtue.

\par \textbf{CLEINIAS}
\par   What do you mean?

\par \textbf{ATHENIAN}
\par   Dear Cleinias, the class of men is small—they must have been rarely gifted by nature, and trained by education—who, when assailed by wants and desires, are able to hold out and observe moderation, and when they might make a great deal of money are sober in their wishes, and prefer a moderate to a large gain. But the mass of mankind are the very opposite:  their desires are unbounded, and when they might gain in moderation they prefer gains without limit; wherefore all that relates to retail trade, and merchandise, and the keeping of taverns, is denounced and numbered among dishonourable things. For if what I trust may never be and will not be, we were to compel, if I may venture to say a ridiculous thing, the best men everywhere to keep taverns for a time, or carry on retail trade, or do anything of that sort; or if, in consequence of some fate or necessity, the best women were compelled to follow similar callings, then we should know how agreeable and pleasant all these things are; and if all such occupations were managed on incorrupt principles, they would be honoured as we honour a mother or a nurse. But now that a man goes to desert places and builds houses which can only be reached by long journeys, for the sake of retail trade, and receives strangers who are in need at the welcome resting-place, and gives them peace and calm when they are tossed by the storm, or cool shade in the heat; and then instead of behaving to them as friends, and showing the duties of hospitality to his guests, treats them as enemies and captives who are at his mercy, and will not release them until they have paid the most unjust, abominable, and extortionate ransom—these are the sort of practises, and foul evils they are, which cast a reproach upon the succour of adversity. And the legislator ought always to be devising a remedy for evils of this nature. There is an ancient saying, which is also a true one—'To fight against two opponents is a difficult thing,' as is seen in diseases and in many other cases. And in this case also the war is against two enemies—wealth and poverty; one of whom corrupts the soul of man with luxury, while the other drives him by pain into utter shamelessness. What remedy can a city of sense find against this disease? In the first place, they must have as few retail traders as possible; and in the second place, they must assign the occupation to that class of men whose corruption will be the least injury to the state; and in the third place, they must devise some way whereby the followers of these occupations themselves will not readily fall into habits of unbridled shamelessness and meanness.

\par  After this preface let our law run as follows, and may fortune favour us: No landowner among the Magnetes, whose city the God is restoring and resettling—no one, that is, of the 5040 families, shall become a retail trader either voluntarily or involuntarily; neither shall he be a merchant, or do any service for private persons unless they equally serve him, except for his father or his mother, and their fathers and mothers; and in general for his elders who are freemen, and whom he serves as a freeman. Now it is difficult to determine accurately the things which are worthy or unworthy of a freeman, but let those who have obtained the prize of virtue give judgment about them in accordance with their feelings of right and wrong. He who in any way shares in the illiberality of retail trades may be indicted for dishonouring his race by any one who likes, before those who have been judged to be the first in virtue; and if he appear to throw dirt upon his father's house by an unworthy occupation, let him be imprisoned for a year and abstain from that sort of thing; and if he repeat the offence, for two years; and every time that he is convicted let the length of his imprisonment be doubled. This shall be the second law: He who engages in retail trade must be either a metic or a stranger. And a third law shall be: In order that the retail trader who dwells in our city may be as good or as little bad as possible, the guardians of the law shall remember that they are not only guardians of those who may be easily watched and prevented from becoming lawless or bad, because they are well-born and bred; but still more should they have a watch over those who are of another sort, and follow pursuits which have a very strong tendency to make men bad. And, therefore, in respect of the multifarious occupations of retail trade, that is to say, in respect of such of them as are allowed to remain, because they seem to be quite necessary in a state—about these the guardians of the law should meet and take counsel with those who have experience of the several kinds of retail trade, as we before commanded concerning adulteration (which is a matter akin to this), and when they meet they shall consider what amount of receipts, after deducting expenses, will produce a moderate gain to the retail trades, and they shall fix in writing and strictly maintain what they find to be the right percentage of profit; this shall be seen to by the wardens of the agora, and by the wardens of the city, and by the wardens of the country. And so retail trade will benefit every one, and do the least possible injury to those in the state who practise it.

\par  When a man makes an agreement which he does not fulfil, unless the agreement be of a nature which the law or a vote of the assembly does not allow, or which he has made under the influence of some unjust compulsion, or which he is prevented from fulfilling against his will by some unexpected chance, the other party may go to law with him in the courts of the tribes, for not having completed his agreement, if the parties are not able previously to come to terms before arbiters or before their neighbours. The class of craftsmen who have furnished human life with the arts is dedicated to Hephaestus and Athene; and there is a class of craftsmen who preserve the works of all craftsmen by arts of defence, the votaries of Ares and Athene, to which divinities they too are rightly dedicated. All these continue through life serving the country and the people; some of them are leaders in battle; others make for hire implements and works, and they ought not to deceive in such matters, out of respect to the Gods who are their ancestors. If any craftsman through indolence omit to execute his work in a given time, not reverencing the God who gives him the means of life, but considering, foolish fellow, that he is his own God and will let him off easily, in the first place, he shall suffer at the hands of the God, and in the second place, the law shall follow in a similar spirit. He shall owe to him who contracted with him the price of the works which he has failed in performing, and he shall begin again and execute them gratis in the given time. When a man undertakes a work, the law gives him the same advice which was given to the seller, that he should not attempt to raise the price, but simply ask the value; this the law enjoins also on the contractor; for the craftsman assuredly knows the value of his work. Wherefore, in free states the man of art ought not to attempt to impose upon private individuals by the help of his art, which is by nature a true thing; and he who is wronged in a matter of this sort, shall have a right of action against the party who has wronged him. And if any one lets out work to a craftsman, and does not pay him duly according to the lawful agreement, disregarding Zeus the guardian of the city and Athene, who are the partners of the state, and overthrows the foundations of society for the sake of a little gain, in his case let the law and the Gods maintain the common bonds of the state. And let him who, having already received the work in exchange, does not pay the price in the time agreed, pay double the price; and if a year has elapsed, although interest is not to be taken on loans, yet for every drachma which he owes to the contractor let him pay a monthly interest of an obol. Suits about these matters are to be decided by the courts of the tribes; and by the way, since we have mentioned craftsmen at all, we must not forget that other craft of war, in which generals and tacticians are the craftsmen, who undertake voluntarily or involuntarily the work of our safety, as other craftsmen undertake other public works—if they execute their work well the law will never tire of praising him who gives them those honours which are the just rewards of the soldier; but if any one, having already received the benefit of any noble service in war, does not make the due return of honour, the law will blame him. Let this then be the law, having an ingredient of praise, not compelling but advising the great body of the citizens to honour the brave men who are the saviours of the whole state, whether by their courage or by their military skill—they should honour them, I say, in the second place; for the first and highest tribute of respect is to be given to those who are able above other men to honour the words of good legislators.

\par  The greater part of the dealings between man and man have been now regulated by us with the exception of those that relate to orphans and the supervision of orphans by their guardians. These follow next in order, and must be regulated in some way. But to arrive at them we must begin with the testamentary wishes of the dying and the case of those who may have happened to die intestate. When I said, Cleinias, that we must regulate them, I had in my mind the difficulty and perplexity in which all such matters are involved. You cannot leave them unregulated, for individuals would make regulations at variance with one another, and repugnant to the laws and habits of the living and to their own previous habits, if a person were simply allowed to make any will which he pleased, and this were to take effect in whatever state he may have been at the end of his life; for most of us lose our senses in a manner, and feel crushed when we think that we are about to die.

\par \textbf{CLEINIAS}
\par   What do you mean, Stranger?

\par \textbf{ATHENIAN}
\par   O Cleinias, a man when he is about to die is an intractable creature, and is apt to use language which causes a great deal of anxiety and trouble to the legislator.

\par \textbf{CLEINIAS}
\par   In what way?

\par \textbf{ATHENIAN}
\par   He wants to have the entire control of all his property, and will use angry words.

\par \textbf{CLEINIAS}
\par   Such as what?

\par \textbf{ATHENIAN}
\par   O ye Gods, he will say, how monstrous that I am not allowed to give, or not to give, my own to whom I will—less to him who has been bad to me, and more to him who has been good to me, and whose badness and goodness have been tested by me in time of sickness or in old age and in every other sort of fortune!

\par \textbf{CLEINIAS}
\par   Well, Stranger, and may he not very fairly say so?

\par \textbf{ATHENIAN}
\par   In my opinion, Cleinias, the ancient legislators were too good-natured, and made laws without sufficient observation or consideration of human things.

\par \textbf{CLEINIAS}
\par   What do you mean?

\par \textbf{ATHENIAN}
\par   I mean, my friend, that they were afraid of the testator's reproaches, and so they passed a law to the effect that a man should be allowed to dispose of his property in all respects as he liked; but you and I, if I am not mistaken, will have something better to say to our departing citizens.

\par \textbf{CLEINIAS}
\par   What?

\par \textbf{ATHENIAN}
\par   O my friends, we will say to them, hard is it for you, who are creatures of a day, to know what is yours—hard too, as the Delphic oracle says, to know yourselves at this hour. Now I, as the legislator, regard you and your possessions, not as belonging to yourselves, but as belonging to your whole family, both past and future, and yet more do I regard both family and possessions as belonging to the state; wherefore, if some one steals upon you with flattery, when you are tossed on the sea of disease or old age, and persuades you to dispose of your property in a way that is not for the best, I will not, if I can help, allow this; but I will legislate with a view to the whole, considering what is best both for the state and for the family, esteeming as I ought the feelings of an individual at a lower rate; and I hope that you will depart in peace and kindness towards us, as you are going the way of all mankind; and we will impartially take care of all your concerns, not neglecting any of them, if we can possibly help. Let this be our prelude and consolation to the living and dying, Cleinias, and let the law be as follows:  He who makes a disposition in a testament, if he be the father of a family, shall first of all inscribe as his heir any one of his sons whom he may think fit; and if he gives any of his children to be adopted by another citizen, let the adoption be inscribed. And if he has a son remaining over and above who has not been adopted upon any lot, and who may be expected to be sent out to a colony according to law, to him his father may give as much as he pleases of the rest of his property, with the exception of the paternal lot and the fixtures on the lot. And if there are other sons, let him distribute among them what there is more than the lot in such portions as he pleases. And if one of the sons has already a house of his own, he shall not give him of the money, nor shall he give money to a daughter who has been betrothed, but if she is not betrothed he may give her money. And if any of the sons or daughters shall be found to have another lot of land in the country, which has accrued after the testament has been made, they shall leave the lot which they have inherited to the heir of the man who has made the will. If the testator has no sons, but only daughters, let him choose the husband of any one of his daughters whom he pleases, and leave and inscribe him as his son and heir. And if a man have lost his son, when he was a child, and before he could be reckoned among grown up men, whether his own or an adopted son, let the testator make mention of the circumstance and inscribe whom he will to be his second son in hope of better fortune. If the testator has no children at all, he may select and give to any one whom he pleases the tenth part of the property which he has acquired; but let him not be blamed if he gives all the rest to his adopted son, and makes a friend of him according to the law. If the sons of a man require guardians, and the father when he dies leaves a will appointing guardians, those who have been named by him, whoever they are and whatever their number be, if they are able and willing to take charge of the children, shall be recognised according to the provisions of the will. But if he dies and has made no will, or a will in which he has appointed no guardians, then the next of kin, two on the father's and two on the mother's side, and one of the friends of the deceased, shall have the authority of guardians, whom the guardians of the law shall appoint when the orphans require guardians. And the fifteen eldest guardians of the law shall have the whole care and charge of the orphans, divided into threes according to seniority—a body of three for one year, and then another body of three for the next year, until the cycle of the five periods is complete; and this, as far as possible, is to continue always. If a man dies, having made no will at all, and leaves sons who require the care of guardians, they shall share in the protection which is afforded by these laws. And if a man dying by some unexpected fate leaves daughters behind him, let him pardon the legislator if when he gives them in marriage, he have a regard only to two out of three conditions—nearness of kin and the preservation of the lot, and omits the third condition, which a father would naturally consider, for he would choose out of all the citizens a son for himself, and a husband for his daughter, with a view to his character and disposition—the father, I say, shall forgive the legislator if he disregards this, which to him is an impossible consideration. Let the law about these matters where practicable be as follows:  If a man dies without making a will, and leaves behind him daughters, let his brother, being the son of the same father or of the same mother, having no lot, marry the daughter and have the lot of the dead man. And if he have no brother, but only a brother's son, in like manner let them marry, if they be of a suitable age; and if there be not even a brother's son, but only the son of a sister, let them do likewise, and so in the fourth degree, if there be only the testator's father's brother, or in the fifth degree, his father's brother's son, or in the sixth degree, the child of his father's sister. Let kindred be always reckoned in this way:  if a person leaves daughters the relationship shall proceed upwards through brothers and sisters, and brothers' and sisters' children, and first the males shall come, and after them the females in the same family. The judge shall consider and determine the suitableness or unsuitableness of age in marriage; he shall make an inspection of the males naked, and of the women naked down to the navel. And if there be a lack of kinsmen in a family extending to grandchildren of a brother, or to the grandchildren of a grandfather's children, the maiden may choose with the consent of her guardians any one of the citizens who is willing and whom she wills, and he shall be the heir of the dead man, and the husband of his daughter. Circumstances vary, and there may sometimes be a still greater lack of relations within the limits of the state; and if any maiden has no kindred living in the city, and there is some one who has been sent out to a colony, and she is disposed to make him the heir of her father's possessions, if he be indeed of her kindred, let him proceed to take the lot according to the regulation of the law; but if he be not of her kindred, she having no kinsmen within the city, and he be chosen by the daughter of the dead man, and empowered to marry by the guardians, let him return home and take the lot of him who died intestate. And if a man has no children, either male or female, and dies without making a will, let the previous law in general hold; and let a man and a woman go forth from the family and share the deserted house, and let the lot belong absolutely to them; and let the heiress in the first degree be a sister, and in a second degree a daughter of a brother, and in the third, a daughter of a sister, in the fourth degree the sister of a father, and in the fifth degree the daughter of a father's brother, and in a sixth degree of a father's sister; and these shall dwell with their male kinsmen, according to the degree of relationship and right, as we enacted before. Now we must not conceal from ourselves that such laws are apt to be oppressive and that there may sometimes be a hardship in the lawgiver commanding the kinsman of the dead man to marry his relation; he may be thought not to have considered the innumerable hindrances which may arise among men in the execution of such ordinances; for there may be cases in which the parties refuse to obey, and are ready to do anything rather than marry, when there is some bodily or mental malady or defect among those who are bidden to marry or be married. Persons may fancy that the legislator never thought of this, but they are mistaken; wherefore let us make a common prelude on behalf of the lawgiver and of his subjects, the law begging the latter to forgive the legislator, in that he, having to take care of the common weal, cannot order at the same time the various circumstances of individuals, and begging him to pardon them if naturally they are sometimes unable to fulfil the act which he in his ignorance imposes upon them.

\par \textbf{CLEINIAS}
\par   And how, Stranger, can we act most fairly under the circumstances?

\par \textbf{ATHENIAN}
\par   There must be arbiters chosen to deal with such laws and the subjects of them.

\par \textbf{CLEINIAS}
\par   What do you mean?

\par \textbf{ATHENIAN}
\par   I mean to say, that a case may occur in which the nephew, having a rich father, will be unwilling to marry the daughter of his uncle; he will have a feeling of pride, and he will wish to look higher. And there are cases in which the legislator will be imposing upon him the greatest calamity, and he will be compelled to disobey the law, if he is required, for example, to take a wife who is mad, or has some other terrible malady of soul or body, such as makes life intolerable to the sufferer. Then let what we are saying concerning these cases be embodied in a law:  If any one finds fault with the established laws respecting testaments, both as to other matters and especially in what relates to marriage, and asserts that the legislator, if he were alive and present, would not compel him to obey—that is to say, would not compel those who are by our law required to marry or be given in marriage, to do either—and some kinsman or guardian dispute this, the reply is that the legislator left fifteen of the guardians of the law to be arbiters and fathers of orphans, male or female, and to them let the disputants have recourse, and by their aid determine any matters of the kind, admitting their decision to be final. But if any one thinks that too great power is thus given to the guardians of the law, let him bring his adversaries into the court of the select judges, and there have the points in dispute determined. And he who loses the cause shall have censure and blame from the legislator, which, by a man of sense, is felt to be a penalty far heavier than a great loss of money.

\par  Thus will orphan children have a second birth. After their first birth we spoke of their nurture and education, and after their second birth, when they have lost their parents, we ought to take measures that the misfortune of orphanhood may be as little sad to them as possible. In the first place, we say that the guardians of the law are lawgivers and fathers to them, not inferior to their natural fathers. Moreover, they shall take charge of them year by year as of their own kindred; and we have given both to them and to the children's own guardians as suitable admonition concerning the nurture of orphans. And we seem to have spoken opportunely in our former discourse, when we said that the souls of the dead have the power after death of taking an interest in human affairs, about which there are many tales and traditions, long indeed, but true; and seeing that they are so many and so ancient, we must believe them, and we must also believe the lawgivers, who tell us that these things are true, if they are not to be regarded as utter fools. But if these things are really so, in the first place men should have a fear of the Gods above, who regard the loneliness of the orphans; and in the second place of the souls of the departed, who by nature incline to take an especial care of their own children, and are friendly to those who honour, and unfriendly to those who dishonour them. Men should also fear the souls of the living who are aged and high in honour; wherever a city is well ordered and prosperous, their descendants cherish them, and so live happily; old persons are quick to see and hear all that relates to them, and are propitious to those who are just in the fulfilment of such duties, and they punish those who wrong the orphan and the desolate, considering that they are the greatest and most sacred of trusts. To all which matters the guardian and magistrate ought to apply his mind, if he has any, and take heed of the nurture and education of the orphans, seeking in every possible way to do them good, for he is making a contribution to his own good and that of his children. He who obeys the tale which precedes the law, and does no wrong to an orphan, will never experience the wrath of the legislator. But he who is disobedient, and wrongs any one who is bereft of father or mother, shall pay twice the penalty which he would have paid if he had wronged one whose parents had been alive. As touching other legislation concerning guardians in their relation to orphans, or concerning magistrates and their superintendence of the guardians, if they did not possess examples of the manner in which children of freemen would be brought up in the bringing up of their own children, and of the care of their property in the care of their own, or if they had not just laws fairly stated about these very things—there would have been reason in making laws for them, under the idea that they were a peculiar class, and we might distinguish and make separate rules for the life of those who are orphans and of those who are not orphans. But as the case stands, the condition of orphans with us is not different from the case of those who have a father, though in regard to honour and dishonour, and the attention given to them, the two are not usually placed upon a level. Wherefore, touching the legislation about orphans, the law speaks in serious accents, both of persuasion and threatening, and such a threat as the following will be by no means out of place: He who is the guardian of an orphan of either sex, and he among the guardians of the law to whom the superintendence of this guardian has been assigned, shall love the unfortunate orphan as though he were his own child, and he shall be as careful and diligent in the management of his possessions as he would be if they were his own, or even more careful and diligent. Let every one who has the care of an orphan observe this law. But any one who acts contrary to the law on these matters, if he be a guardian of the child, may be fined by a magistrate, or, if he be himself a magistrate, the guardian may bring him before the court of select judges, and punish him, if convicted, by exacting a fine of double the amount of that inflicted by the court. And if a guardian appears to the relations of the orphan, or to any other citizen, to act negligently or dishonestly, let them bring him before the same court, and whatever damages are given against him, let him pay fourfold, and let half belong to the orphan and half to him who procured the conviction. If any orphan arrives at years of discretion, and thinks that he has been ill-used by his guardians, let him within five years of the expiration of the guardianship be allowed to bring them to trial; and if any of them be convicted, the court shall determine what he shall pay or suffer. And if a magistrate shall appear to have wronged the orphan by neglect, and he be convicted, let the court determine what he shall suffer or pay to the orphan, and if there be dishonesty in addition to neglect, besides paying the fine, let him be deposed from his office of guardian of the law, and let the state appoint another guardian of the law for the city and for the country in his room.

\par  Greater differences than there ought to be sometimes arise between fathers and sons, on the part either of fathers who will be of opinion that the legislator should enact that they may, if they wish, lawfully renounce their son by the proclamation of a herald in the face of the world, or of sons who think that they should be allowed to indict their fathers on the charge of imbecility when they are disabled by disease or old age. These things only happen, as a matter of fact, where the natures of men are utterly bad; for where only half is bad, as, for example, if the father be not bad, but the son be bad, or conversely, no great calamity is the result of such an amount of hatred as this. In another state, a son disowned by his father would not of necessity cease to be a citizen, but in our state, of which these are to be the laws, the disinherited must necessarily emigrate into another country, for no addition can be made even of a single family to the 5040 households; and, therefore, he who deserves to suffer these things must be renounced not only by his father, who is a single person, but by the whole family, and what is done in these cases must be regulated by some such law as the following: He who in the sad disorder of his soul has a mind, justly or unjustly, to expel from his family a son whom he has begotten and brought up, shall not lightly or at once execute his purpose; but first of all he shall collect together his own kinsmen, extending to cousins, and in like manner his son's kinsmen by the mother's side, and in their presence he shall accuse his son, setting forth that he deserves at the hands of them all to be dismissed from the family; and the son shall be allowed to address them in a similar manner, and show that he does not deserve to suffer any of these things. And if the father persuades them, and obtains the suffrages of more than half of his kindred, exclusive of the father and mother and the offender himself—I say, if he obtains more than half the suffrages of all the other grown-up members of the family, of both sexes, the father shall be permitted to put away his son, but not otherwise. And if any other citizen is willing to adopt the son who is put away, no law shall hinder him; for the characters of young men are subject to many changes in the course of their lives. And if he has been put away, and in a period of ten years no one is willing to adopt him, let those who have the care of the superabundant population which is sent out into colonies, see to him, in order that he may be suitably provided for in the colony. And if disease or age or harshness of temper, or all these together, makes a man to be more out of his mind than the rest of the world are—but this is not observable, except to those who live with him—and he, being master of his property, is the ruin of the house, and his son doubts and hesitates about indicting his father for insanity, let the law in that case ordain that he shall first of all go to the eldest guardians of the law and tell them of his father's misfortune, and they shall duly look into the matter, and take counsel as to whether he shall indict him or not. And if they advise him to proceed, they shall be both his witnesses and his advocates; and if the father is cast, he shall henceforth be incapable of ordering the least particular of his life; let him be as a child dwelling in the house for the remainder of his days. And if a man and his wife have an unfortunate incompatibility of temper, ten of the guardians of the law, who are impartial, and ten of the women who regulate marriages, shall look to the matter, and if they are able to reconcile them they shall be formally reconciled; but if their souls are too much tossed with passion, they shall endeavour to find other partners. Now they are not likely to have very gentle tempers; and, therefore, we must endeavour to associate with them deeper and softer natures. Those who have no children, or only a few, at the time of their separation, should choose their new partners with a view to the procreation of children; but those who have a sufficient number of children should separate and marry again in order that they may have some one to grow old with and that the pair may take care of one another in age. If a woman dies, leaving children, male or female, the law will advise rather than compel the husband to bring up the children without introducing into the house a stepmother. But if he have no children, then he shall be compelled to marry until he has begotten a sufficient number of sons to his family and to the state. And if a man dies leaving a sufficient number of children, the mother of his children shall remain with them and bring them up. But if she appears to be too young to live virtuously without a husband, let her relations communicate with the women who superintend marriage, and let both together do what they think best in these matters; if there is a lack of children, let the choice be made with a view to having them; two children, one of either sex, shall be deemed sufficient in the eye of the law. When a child is admitted to be the offspring of certain parents and is acknowledged by them, but there is need of a decision as to which parent the child is to follow—in case a female slave have intercourse with a male slave, or with a freeman or freedman, the offspring shall always belong to the master of the female slave. Again, if a free woman have intercourse with a male slave, the offspring shall belong to the master of the slave; but if a child be born either of a slave by her master, or of his mistress by a slave—and this be proven—the offspring of the woman and its father shall be sent away by the women who superintend marriage into another country, and the guardians of the law shall send away the offspring of the man and its mother.

\par  Neither God, nor a man who has understanding, will ever advise any one to neglect his parents. To a discourse concerning the honour and dishonour of parents, a prelude such as the following, about the service of the Gods, will be a suitable introduction: There are ancient customs about the Gods which are universal, and they are of two kinds: some of the Gods we see with our eyes and we honour them, of others we honour the images, raising statues of them which we adore; and though they are lifeless, yet we imagine that the living Gods have a good will and gratitude to us on this account. Now, if a man has a father or mother, or their fathers or mothers treasured up in his house stricken in years, let him consider that no statue can be more potent to grant his requests than they are, who are sitting at his hearth, if only he knows how to show true service to them.

\par \textbf{CLEINIAS}
\par   And what do you call the true mode of service?

\par \textbf{ATHENIAN}
\par   I will tell you, O my friend, for such things are worth listening to.

\par \textbf{CLEINIAS}
\par   Proceed.

\par \textbf{ATHENIAN}
\par   Oedipus, as tradition says, when dishonoured by his sons, invoked on them curses which every one declares to have been heard and ratified by the Gods, and Amyntor in his wrath invoked curses on his son Phoenix, and Theseus upon Hippolytus, and innumerable others have also called down wrath upon their children, whence it is clear that the Gods listen to the imprecations of parents; for the curses of parents are, as they ought to be, mighty against their children as no others are. And shall we suppose that the prayers of a father or mother who is specially dishonoured by his or her children, are heard by the Gods in accordance with nature; and that if a parent is honoured by them, and in the gladness of his heart earnestly entreats the Gods in his prayers to do them good, he is not equally heard, and that they do not minister to his request? If not, they would be very unjust ministers of good, and that we affirm to be contrary to their nature.

\par \textbf{CLEINIAS}
\par   Certainly.

\par \textbf{ATHENIAN}
\par   May we not think, as I was saying just now, that we can possess no image which is more honoured by the Gods, than that of a father or grandfather, or of a mother stricken in years? whom when a man honours, the heart of the God rejoices, and he is ready to answer their prayers. And, truly, the figure of an ancestor is a wonderful thing, far higher than that of a lifeless image. For the living, when they are honoured by us, join in our prayers, and when they are dishonoured, they utter imprecations against us; but lifeless objects do neither. And therefore, if a man makes a right use of his father and grandfather and other aged relations, he will have images which above all others will win him the favour of the Gods.

\par \textbf{CLEINIAS}
\par   Excellent.

\par \textbf{ATHENIAN}
\par   Every man of any understanding fears and respects the prayers of parents, knowing well that many times and to many persons they have been accomplished. Now these things being thus ordered by nature, good men think it a blessing from heaven if their parents live to old age and reach the utmost limit of human life, or if taken away before their time they are deeply regretted by them; but to bad men parents are always a cause of terror. Wherefore let every man honour with every sort of lawful honour his own parents, agreeably to what has now been said. But if this prelude be an unmeaning sound in the ears of any one, let the law follow, which may be rightly imposed in these terms:  If any one in this city be not sufficiently careful of his parents, and do not regard and gratify in every respect their wishes more than those of his sons and of his other offspring or of himself—let him who experiences this sort of treatment either come himself, or send some one to inform the three eldest guardians of the law, and three of the women who have the care of marriages; and let them look to the matter and punish youthful evil-doers with stripes and bonds if they are under thirty years of age, that is to say, if they be men, or if they be women, let them undergo the same punishment up to forty years of age. But if, when they are still more advanced in years, they continue the same neglect of their parents, and do any hurt to any of them, let them be brought before a court in which every single one of the eldest citizens shall be the judges, and if the offender be convicted, let the court determine what he ought to pay or suffer, and any penalty may be imposed on him which a man can pay or suffer. If the person who has been wronged be unable to inform the magistrates, let any freeman who hears of his case inform, and if he do not, he shall be deemed base, and shall be liable to have a suit for damage brought against him by any one who likes. And if a slave inform, he shall receive freedom; and if he be the slave of the injurer or injured party, he shall be set free by the magistrates, or if he belong to any other citizen, the public shall pay a price on his behalf to the owner; and let the magistrates take heed that no one wrongs him out of revenge, because he has given information.

\par  Cases in which one man injures another by poisons, and which prove fatal, have been already discussed; but about other cases in which a person intentionally and of malice harms another with meats, or drinks, or ointments, nothing has as yet been determined. For there are two kinds of poisons used among men, which cannot clearly be distinguished. There is the kind just now explicitly mentioned, which injures bodies by the use of other bodies according to a natural law; there is also another kind which persuades the more daring class that they can do injury by sorceries, and incantations, and magic knots, as they are termed, and makes others believe that they above all persons are injured by the powers of the magician. Now it is not easy to know the nature of all these things; nor if a man do know can he readily persuade others to believe him. And when men are disturbed in their minds at the sight of waxen images fixed either at their doors, or in a place where three ways meet, or on the sepulchres of parents, there is no use in trying to persuade them that they should despise all such things because they have no certain knowledge about them. But we must have a law in two parts, concerning poisoning, in whichever of the two ways the attempt is made, and we must entreat, and exhort, and advise men not to have recourse to such practises, by which they scare the multitude out of their wits, as if they were children, compelling the legislator and the judge to heal the fears which the sorcerer arouses, and to tell them in the first place, that he who attempts to poison or enchant others knows not what he is doing, either as regards the body (unless he has a knowledge of medicine), or as regards his enchantments (unless he happens to be a prophet or diviner). Let the law, then, run as follows about poisoning or witchcraft: He who employs poison to do any injury, not fatal, to a man himself, or to his servants, or any injury, whether fatal or not, to his cattle or his bees, if he be a physician, and be convicted of poisoning, shall be punished with death; or if he be a private person, the court shall determine what he is to pay or suffer. But he who seems to be the sort of man who injures others by magic knots, or enchantments, or incantations, or any of the like practices, if he be a prophet or diviner, let him die; and if, not being a prophet, he be convicted of witchcraft, as in the previous case, let the court fix what he ought to pay or suffer.

\par  When a man does another any injury by theft or violence, for the greater injury let him pay greater damages to the injured man, and less for the smaller injury; but in all cases, whatever the injury may have been, as much as will compensate the loss. And besides the compensation of the wrong, let a man pay a further penalty for the chastisement of his offence: he who has done the wrong instigated by the folly of another, through the lightheartedness of youth or the like, shall pay a lighter penalty; but he who has injured another through his own folly, when overcome by pleasure or pain, in cowardly fear, or lust, or envy, or implacable anger, shall endure a heavier punishment. Not that he is punished because he did wrong, for that which is done can never be undone, but in order that in future times, he, and those who see him corrected, may utterly hate injustice, or at any rate abate much of their evil-doing. Having an eye to all these things, the law, like a good archer, should aim at the right measure of punishment, and in all cases at the deserved punishment. In the attainment of this the judge shall be a fellow-worker with the legislator, whenever the law leaves to him to determine what the offender shall suffer or pay; and the legislator, like a painter, shall give a rough sketch of the cases in which the law is to be applied. This is what we must do, Megillus and Cleinias, in the best and fairest manner that we can, saying what the punishments are to be of all actions of theft and violence, and giving laws of such a kind as the Gods and sons of Gods would have us give.

\par  If a man is mad he shall not be at large in the city, but his relations shall keep him at home in any way which they can; or if not, let them pay a penalty—he who is of the highest class shall pay a penalty of one hundred drachmas, whether he be a slave or a freeman whom he neglects; and he of the second class shall pay four-fifths of a mina; and he of the third class three-fifths; and he of the fourth class two-fifths. Now there are many sorts of madness, some arising out of disease, which we have already mentioned; and there are other kinds, which originate in an evil and passionate temperament, and are increased by bad education; out of a slight quarrel this class of madmen will often raise a storm of abuse against one another, and nothing of that sort ought to be allowed to occur in a well-ordered state. Let this, then, be the law about abuse, which shall relate to all cases: No one shall speak evil of another; and when a man disputes with another he shall teach and learn of the disputant and the company, but he shall abstain from evil-speaking; for out of the imprecations which men utter against one another, and the feminine habit of casting aspersions on one another, and using foul names, out of words light as air, in very deed the greatest enmities and hatreds spring up. For the speaker gratifies his anger, which is an ungracious element of his nature; and nursing up his wrath by the entertainment of evil thoughts, and exacerbating that part of his soul which was formerly civilised by education, he lives in a state of savageness and moroseness, and pays a bitter penalty for his anger. And in such cases almost all men take to saying something ridiculous about their opponent, and there is no man who is in the habit of laughing at another who does not miss virtue and earnestness altogether, or lose the better half of greatness. Wherefore let no one utter any taunting word at a temple, or at the public sacrifices, or at the games, or in the agora, or in a court of justice, or in any public assembly. And let the magistrate who presides on these occasions chastise an offender, and he shall be blameless; but if he fails in doing so, he shall not claim the prize of virtue; for he is one who heeds not the laws, and does not do what the legislator commands. And if in any other place any one indulges in these sort of revilings, whether he has begun the quarrel or is only retaliating, let any elder who is present support the law, and control with blows those who indulge in passion, which is another great evil; and if he do not, let him be liable to pay the appointed penalty. And we say now, that he who deals in reproaches against others cannot reproach them without attempting to ridicule them; and this, when done in a moment of anger, is what we make matter of reproach against him. But then, do we admit into our state the comic writers who are so fond of making mankind ridiculous, if they attempt in a good-natured manner to turn the laugh against our citizens? or do we draw the distinction of jest and earnest, and allow a man to make use of ridicule in jest and without anger about any thing or person; though as we were saying, not if he be angry and have a set purpose? We forbid earnest—that is unalterably fixed; but we have still to say who are to be sanctioned or not to be sanctioned by the law in the employment of innocent humour. A comic poet, or maker of iambic or satirical lyric verse, shall not be permitted to ridicule any of the citizens, either by word or likeness, either in anger or without anger. And if any one is disobedient, the judges shall either at once expel him from the country, or he shall pay a fine of three minae, which shall be dedicated to the God who presides over the contests. Those only who have received permission shall be allowed to write verses at one another, but they shall be without anger and in jest; in anger and in serious earnest they shall not be allowed. The decision of this matter shall be left to the superintendent of the general education of the young, and whatever he may license, the writer shall be allowed to produce, and whatever he rejects let not the poet himself exhibit, or ever teach anybody else, slave or freeman, under the penalty of being dishonoured, and held disobedient to the laws.

\par  Now he is not to be pitied who is hungry, or who suffers any bodily pain, but he who is temperate, or has some other virtue, or part of a virtue, and at the same time suffers from misfortune; it would be an extraordinary thing if such an one, whether slave or freeman, were utterly forsaken and fell into the extremes of poverty in any tolerably well-ordered city or government. Wherefore the legislator may safely make a law applicable to such cases in the following terms: Let there be no beggars in our state; and if anybody begs, seeking to pick up a livelihood by unavailing prayers, let the wardens of the agora turn him out of the agora, and the wardens of the city out of the city, and the wardens of the country send him out of any other parts of the land across the border, in order that the land may be cleared of this sort of animal.

\par  If a slave of either sex injure anything, which is not his or her own, through inexperience, or some improper practice, and the person who suffers damage be not himself in part to blame, the master of the slave who has done the harm shall either make full satisfaction, or give up the slave who has done the injury. But if the master argue that the charge has arisen by collusion between the injured party and the injurer, with the view of obtaining the slave, let him sue the person, who says that he has been injured, for malpractices. And if he gain a conviction, let him receive double the value which the court fixes as the price of the slave; and if he lose his suit, let him make amends for the injury, and give up the slave. And if a beast of burden, or horse, or dog, or any other animal, injure the property of a neighbour, the owner shall in like manner pay for the injury.

\par  If any man refuses to be a witness, he who wants him shall summon him, and he who is summoned shall come to the trial; and if he knows and is willing to bear witness, let him bear witness, but if he says he does not know let him swear by the three divinities Zeus, and Apollo, and Themis, that he does not, and have no more to do with the cause. And he who is summoned to give witness and does not answer to his summoner, shall be liable for the harm which ensues according to law. And if a person calls up as a witness any one who is acting as a judge, let him give his witness, but he shall not afterwards vote in the cause. A free woman may give her witness and plead, if she be more than forty years of age, and may bring an action if she have no husband; but if her husband be alive she shall only be allowed to bear witness. A slave of either sex and a child shall be allowed to give evidence and to plead, but only in cases of murder; and they must produce sufficient sureties that they will certainly remain until the trial, in case they should be charged with false witness. And either of the parties in a cause may bring an accusation of perjury against witnesses, touching their evidence in whole or in part, if he asserts that such evidence has been given; but the accusation must be brought previous to the final decision of the cause. The magistrates shall preserve the accusations of false witness, and have them kept under the seal of both parties, and produce them on the day when the trial for false witness takes place. If a man be twice convicted of false witness, he shall not be required, and if thrice, he shall not be allowed to bear witness; and if he dare to witness after he has been convicted three times, let any one who pleases inform against him to the magistrates, and let the magistrates hand him over to the court, and if he be convicted he shall be punished with death. And in any case in which the evidence is rightly found to be false, and yet to have given the victory to him who wins the suit, and more than half the witnesses are condemned, the decision which was gained by these means shall be rescinded, and there shall be a discussion and a decision as to whether the suit was determined by that false evidence or not; and in whichever way the decision may be given, the previous suit shall be determined accordingly.

\par  There are many noble things in human life, but to most of them attach evils which are fated to corrupt and spoil them. Is not justice noble, which has been the civiliser of humanity? How then can the advocate of justice be other than noble? And yet upon this profession which is presented to us under the fair name of art has come an evil reputation. In the first place, we are told that by ingenious pleas and the help of an advocate the law enables a man to win a particular cause, whether just or unjust; and that both the art, and the power of speech which is thereby imparted, are at the service of him who is willing to pay for them. Now in our state this so-called art, whether really an art or only an experience and practice destitute of any art, ought if possible never to come into existence, or if existing among us should listen to the request of the legislator and go away into another land, and not speak contrary to justice. If the offenders obey we say no more; but for those who disobey, the voice of the law is as follows: If any one thinks that he will pervert the power of justice in the minds of the judges, and unseasonably litigate or advocate, let any one who likes indict him for malpractices of law and dishonest advocacy, and let him be judged in the court of select judges; and if he be convicted, let the court determine whether he may be supposed to act from a love of money or from contentiousness. And if he is supposed to act from contentiousness, the court shall fix a time during which he shall not be allowed to institute or plead a cause; and if he is supposed to act as he does from love of money, in case he be a stranger, he shall leave the country, and never return under penalty of death; but if he be a citizen, he shall die, because he is a lover of money, in whatever manner gained; and equally, if he be judged to have acted more than once from contentiousness, he shall die.

\par 
\section{
      BOOK XII.
    }
\par  If a herald or an ambassador carry a false message from our city to any other, or bring back a false message from the city to which he is sent, or be proved to have brought back, whether from friends or enemies, in his capacity of herald or ambassador, what they have never said, let him be indicted for having violated, contrary to the law, the commands and duties imposed upon him by Hermes and Zeus, and let there be a penalty fixed, which he shall suffer or pay if he be convicted.

\par  Theft is a mean, and robbery a shameless thing; and none of the sons of Zeus delight in fraud and violence, or ever practised either. Wherefore let no one be deluded by poets or mythologers into a mistaken belief of such things, nor let him suppose, when he thieves or is guilty of violence, that he is doing nothing base, but only what the Gods themselves do. For such tales are untrue and improbable; and he who steals or robs contrary to the law, is never either a God or the son of a God; of this the legislator ought to be better informed than all the poets put together. Happy is he and may he be for ever happy, who is persuaded and listens to our words; but he who disobeys shall have to contend against the following law: If a man steal anything belonging to the public, whether that which he steals be much or little, he shall have the same punishment. For he who steals a little steals with the same wish as he who steals much, but with less power, and he who takes up a greater amount, not having deposited it, is wholly unjust. Wherefore the law is not disposed to inflict a less penalty on the one than on the other because his theft is less, but on the ground that the thief may possibly be in one case still curable, and may in another case be incurable. If any one convict in a court of law a stranger or a slave of a theft of public property, let the court determine what punishment he shall suffer, or what penalty he shall pay, bearing in mind that he is probably not incurable. But the citizen who has been brought up as our citizens will have been, if he be found guilty of robbing his country by fraud or violence, whether he be caught in the act or not, shall be punished with death; for he is incurable.

\par  Now for expeditions of war much consideration and many laws are required; the great principle of all is that no one of either sex should be without a commander; nor should the mind of any one be accustomed to do anything, either in jest or earnest, of his own motion, but in war and in peace he should look to and follow his leader, even in the least things being under his guidance; for example, he should stand or move, or exercise, or wash, or take his meals, or get up in the night to keep guard and deliver messages when he is bidden; and in the hour of danger he should not pursue and not retreat except by order of his superior; and in a word, not teach the soul or accustom her to know or understand how to do anything apart from others. Of all soldiers the life should be always and in all things as far as possible in common and together; there neither is nor ever will be a higher, or better, or more scientific principle than this for the attainment of salvation and victory in war. And we ought in time of peace from youth upwards to practise this habit of commanding others, and of being commanded by others; anarchy should have no place in the life of man or of the beasts who are subject to man. I may add that all dances ought to be performed with a view to military excellence; and agility and ease should be cultivated for the same object, and also endurance of the want of meats and drinks, and of winter cold and summer heat, and of hard couches; and, above all, care should be taken not to destroy the peculiar qualities of the head and the feet by surrounding them with extraneous coverings, and so hindering their natural growth of hair and soles. For these are the extremities, and of all the parts of the body, whether they are preserved or not is of the greatest consequence; the one is the servant of the whole body, and the other the master, in whom all the ruling senses are by nature set. Let the young men imagine that he hears in what has preceded the praises of the military life; the law shall be as follows: He shall serve in war who is on the roll or appointed to some special service, and if any one is absent from cowardice, and without the leave of the generals, he shall be indicted before the military commanders for failure of service when the army comes home; and the soldiers shall be his judges; the heavy-armed, and the cavalry, and the other arms of the service shall form separate courts; and they shall bring the heavy-armed before the heavy-armed, and the horsemen before the horsemen, and the others in like manner before their peers; and he who is found guilty shall never be allowed to compete for any prize of valour, or indict another for not serving on an expedition, or be an accuser at all in any military matters. Moreover, the court shall further determine what punishment he shall suffer, or what penalty he shall pay. When the suits for failure of service are completed, the leaders of the several kinds of troops shall again hold an assembly, and they shall adjudge the prizes of valour; and he who likes searching for judgment in his own branch of the service, saying nothing about any former expedition, nor producing any proof or witnesses to confirm his statement, but speaking only of the present occasion. The crown of victory shall be an olive wreath which the victor shall offer up at the temple of any war-god whom he likes, adding an inscription for a testimony to last during life, that such an one has received the first, the second, or the third prize. If any one goes on an expedition, and returns home before the appointed time, when the generals have not withdrawn the army, he shall be indicted for desertion before the same persons who took cognizance of failure of service, and if he be found guilty, the same punishment shall be inflicted on him. Now every man who is engaged in any suit ought to be very careful of bringing false witness against any one, either intentionally or unintentionally, if he can help; for justice is truly said to be an honourable maiden, and falsehood is naturally repugnant to honour and justice. A witness ought to be very careful not to sin against justice, as for example in what relates to the throwing away of arms—he must distinguish the throwing them away when necessary, and not make that a reproach, or bring an action against some innocent person on that account. To make the distinction may be difficult; but still the law must attempt to define the different kinds in some way. Let me endeavour to explain my meaning by an ancient tale: If Patroclus had been brought to the tent still alive but without his arms (and this has happened to innumerable persons), the original arms, which the poet says were presented to Peleus by the Gods as a nuptial gift when he married Thetis, remaining in the hands of Hector, then the base spirits of that day might have reproached the son of Menoetius with having cast away his arms. Again, there is the case of those who have been thrown down precipices and lost their arms; and of those who at sea, and in stormy places, have been suddenly overwhelmed by floods of water; and there are numberless things of this kind which one might adduce by way of extenuation, and with the view of justifying a misfortune which is easily misrepresented. We must, therefore, endeavour to divide to the best of our power the greater and more serious evil from the lesser. And a distinction may be drawn in the use of terms of reproach. A man does not always deserve to be called the thrower away of his shield; he may be only the loser of his arms. For there is a great or rather absolute difference between him who is deprived of his arms by a sufficient force, and him who voluntarily lets his shield go. Let the law then be as follows: If a person having arms is overtaken by the enemy and does not turn round and defend himself, but lets them go voluntarily or throws them away, choosing a base life and a swift escape rather than a courageous and noble and blessed death—in such a case of the throwing away of arms let justice be done, but the judge need take no note of the case just now mentioned; for the bad men ought always to be punished, in the hope that he may be improved, but not the unfortunate, for there is no advantage in that. And what shall be the punishment suited to him who has thrown away his weapons of defence? Tradition says that Caeneus, the Thessalian, was changed by a God from a woman into a man; but the converse miracle cannot now be wrought, or no punishment would be more proper than that the man who throws away his shield should be changed into a woman. This however is impossible, and therefore let us make a law as nearly like this as we can—that he who loves his life too well shall be in no danger for the remainder of his days, but shall live for ever under the stigma of cowardice. And let the law be in the following terms: When a man is found guilty of disgracefully throwing away his arms in war, no general or military officer shall allow him to serve as a soldier, or give him any place at all in the ranks of soldiers; and the officer who gives the coward any place, shall suffer a penalty which the public examiner shall exact of him; and if he be of the highest class, he shall pay a thousand drachmae; or if he be of the second class, five minae; or if he be of the third, three minae; or if he be of the fourth class, one mina. And he who is found guilty of cowardice, shall not only be dismissed from manly dangers, which is a disgrace appropriate to his nature, but he shall pay a thousand drachmae, if he be of the highest class, and five minae if he be of the second class, and three if he be of the third class, and a mina, like the preceding, if he be of the fourth class.

\par  What regulations will be proper about examiners, seeing that some of our magistrates are elected by lot, and for a year, and some for a longer time and from selected persons? Of such magistrates, who will be a sufficient censor or examiner, if any of them, weighed down by the pressure of office or his own inability to support the dignity of his office, be guilty of any crooked practice? It is by no means easy to find a magistrate who excels other magistrates in virtue, but still we must endeavour to discover some censor or examiner who is more than man. For the truth is, that there are many elements of dissolution in a state, as there are also in a ship, or in an animal; they all have their cords, and girders, and sinews—one nature diffused in many places, and called by many names; and the office of examiner is a most important element in the preservation and dissolution of states. For if the examiners are better than the magistrates, and their duty is fulfilled justly and without blame, then the whole state and country flourishes and is happy; but if the examination of the magistrates is carried on in a wrong way, then, by the relaxation of that justice which is the uniting principle of all constitutions, every power in the state is rent asunder from every other; they no longer incline in the same direction, but fill the city with faction, and make many cities out of one, and soon bring all to destruction. Wherefore the examiners ought to be admirable in every sort of virtue. Let us invent a mode of creating them, which shall be as follows: Every year, after the summer solstice, the whole city shall meet in the common precincts of Helios and Apollo, and shall present to the God three men out of their own number in the manner following: Each citizen shall select, not himself, but some other citizen whom he deems in every way the best, and who is not less than fifty years of age. And out of the selected persons who have the greatest number of votes, they shall make a further selection until they reduce them to one-half, if they are an even number; but if they are not an even number, they shall subtract the one who has the smallest number of votes, and make them an even number, and then leave the half which have the greater number of votes. And if two persons have an equal number of votes, and thus increase the number beyond one-half, they shall withdraw the younger of the two and do away the excess; and then including all the rest they shall again vote, until there are left three having an unequal number of votes. But if all the three, or two out of the three, have equal votes, let them commit the election to good fate and fortune, and separate off by lot the first, and the second, and the third; these they shall crown with an olive wreath and give them the prize of excellence, at the same time proclaiming to all the world that the city of the Magnetes, by the providence of the Gods, is again preserved, and presents to the Sun and to Apollo her three best men as first-fruits, to be a common offering to them, according to the ancient law, as long as their lives answer to the judgment formed of them. And these shall appoint in their first year twelve examiners, to continue until each has completed seventy-five years, to whom three shall afterwards be added yearly; and let these divide all the magistracies into twelve parts, and prove the holders of them by every sort of test to which a freeman may be subjected; and let them live while they hold office in the precinct of Helios and Apollo, in which they were chosen, and let each one form a judgment of some things individually, and of others in company with his colleagues; and let him place a writing in the agora about each magistracy, and what the magistrate ought to suffer or pay, according to the decision of the examiners. And if a magistrate does not admit that he has been justly judged, let him bring the examiners before the select judges, and if he be acquitted by their decision, let him, if he will, accuse the examiners themselves; if, however, he be convicted, and have been condemned to death by the examiners, let him die (and of course he can only die once): but any other penalties which admit of being doubled let him suffer twice over.

\par  And now let us pass under review the examiners themselves; what will their examination be, and how conducted? During the life of these men, whom the whole state counts worthy of the rewards of virtue, they shall have the first seat at all public assemblies, and at all Hellenic sacrifices and sacred missions, and other public and holy ceremonies in which they share. The chiefs of each sacred mission shall be selected from them, and they only of all the citizens shall be adorned with a crown of laurel; they shall all be priests of Apollo and Helios; and one of them, who is judged first of the priests created in that year, shall be high priest; and they shall write up his name in each year to be a measure of time as long as the city lasts; and after their death they shall be laid out and carried to the grave and entombed in a manner different from the other citizens. They shall be decked in a robe all of white, and there shall be no crying or lamentation over them; but a chorus of fifteen maidens, and another of boys, shall stand around the bier on either side, hymning the praises of the departed priests in alternate responses, declaring their blessedness in song all day long; and at dawn a hundred of the youths who practise gymnastic exercises, and whom the relations of the departed shall choose, shall carry the bier to the sepulchre, the young men marching first, dressed in the garb of warriors—the cavalry with their horses, the heavy-armed with their arms, and the others in like manner. And boys near the bier and in front of it shall sing their national hymn, and maidens shall follow behind, and with them the women who have passed the age of child-bearing; next, although they are interdicted from other burials, let priests and priestesses follow, unless the Pythian oracle forbid them; for this burial is free from pollution. The place of burial shall be an oblong vaulted chamber underground, constructed of tufa, which will last for ever, having stone couches placed side by side. And here they will lay the blessed person, and cover the sepulchre with a circular mound of earth and plant a grove of trees around on every side but one; and on that side the sepulchre shall be allowed to extend for ever, and a new mound will not be required. Every year they shall have contests in music and gymnastics, and in horsemanship, in honour of the dead. These are the honours which shall be given to those who at the examination are found blameless; but if any of them, trusting to the scrutiny being over, should, after the judgment has been given, manifest the wickedness of human nature, let the law ordain that he who pleases shall indict him, and let the cause be tried in the following manner. In the first place, the court shall be composed of the guardians of the law, and to them the surviving examiners shall be added, as well as the court of select judges; and let the pursuer lay his indictment in this form—he shall say that so-and-so is unworthy of the prize of virtue and of his office; and if the defendant be convicted let him be deprived of his office, and of the burial, and of the other honours given him. But if the prosecutor do not obtain the fifth part of the votes, let him, if he be of the first-class, pay twelve minae, and eight if he be of the second class, and six if he be of the third class, and two minae if he be of the fourth class.

\par  The so-called decision of Rhadamanthus is worthy of all admiration. He knew that the men of his own time believed and had no doubt that there were Gods, which was a reasonable belief in those days, because most men were the sons of Gods, and according to tradition he was one himself. He appears to have thought that he ought to commit judgment to no man, but to the Gods only, and in this way suits were simply and speedily decided by him. For he made the two parties take an oath respecting the points in dispute, and so got rid of the matter speedily and safely. But now that a certain portion of mankind do not believe at all in the existence of the Gods, and others imagine that they have no care of us, and the opinion of most men, and of the worst men, is that in return for a small sacrifice and a few flattering words they will be their accomplices in purloining large sums and save them from many terrible punishments, the way of Rhadamanthus is no longer suited to the needs of justice; for as the opinions of men about the Gods are changed, the laws should also be changed—in the granting of suits a rational legislation ought to do away with the oaths of the parties on either side—he who obtains leave to bring an action should write down the charges, but should not add an oath; and the defendant in like manner should give his denial to the magistrates in writing, and not swear; for it is a dreadful thing to know, when many lawsuits are going on in a state, that almost half the people who meet one another quite unconcernedly at the public meals and in other companies and relations of private life are perjured. Let the law, then, be as follows: A judge who is about to give judgment shall take an oath, and he who is choosing magistrates for the state shall either vote on oath or with a voting tablet which he brings from a temple; so too the judge of dances and of all music, and the superintendents and umpires of gymnastic and equestrian contests, and any matters in which, as far as men can judge, there is nothing to be gained by a false oath; but all cases in which a denial confirmed by an oath clearly results in a great advantage to the taker of the oath, shall be decided without the oath of the parties to the suit, and the presiding judges shall not permit either of them to use an oath for the sake of persuading, nor to call down curses on himself and his race, nor to use unseemly supplications or womanish laments. But they shall ever be teaching and learning what is just in auspicious words; and he who does otherwise shall be supposed to speak beside the point, and the judges shall again bring him back to the question at issue. On the other hand, strangers in their dealings with strangers shall as at present have power to give and receive oaths, for they will not often grow old in the city or leave a fry of young ones like themselves to be the sons and heirs of the land.

\par  As to the initiation of private suits, let the manner of deciding causes between all citizens be the same as in cases in which any freeman is disobedient to the state in minor matters, of which the penalty is not stripes, imprisonment, or death. But as regards attendance at choruses or processions or other shows, and as regards public services, whether the celebration of sacrifice in peace, or the payment of contributions in war—in all these cases, first comes the necessity of providing a remedy for the loss; and by those who will not obey, there shall be security given to the officers whom the city and the law empower to exact the sum due; and if they forfeit their security, let the goods which they have pledged be sold and the money given to the city; but if they ought to pay a larger sum, the several magistrates shall impose upon the disobedient a suitable penalty, and bring them before the court, until they are willing to do what they are ordered.

\par  Now a state which makes money from the cultivation of the soil only, and has no foreign trade, must consider what it will do about the emigration of its own people to other countries, and the reception of strangers from elsewhere. About these matters the legislator has to consider, and he will begin by trying to persuade men as far as he can. The intercourse of cities with one another is apt to create a confusion of manners; strangers are always suggesting novelties to strangers. When states are well governed by good laws the mixture causes the greatest possible injury; but seeing that most cities are the reverse of well-ordered, the confusion which arises in them from the reception of strangers, and from the citizens themselves rushing off into other cities, when any one either young or old desires to travel anywhere abroad at whatever time, is of no consequence. On the other hand, the refusal of states to receive others, and for their own citizens never to go to other places, is an utter impossibility, and to the rest of the world is likely to appear ruthless and uncivilised; it is a practice adopted by people who use harsh words, such as xenelasia or banishment of strangers, and who have harsh and morose ways, as men think. And to be thought or not to be thought well of by the rest of the world is no light matter; for the many are not so far wrong in their judgment of who are bad and who are good, as they are removed from the nature of virtue in themselves. Even bad men have a divine instinct which guesses rightly, and very many who are utterly depraved form correct notions and judgments of the differences between the good and bad. And the generality of cities are quite right in exhorting us to value a good reputation in the world, for there is no truth greater and more important than this—that he who is really good (I am speaking of the men who would be perfect) seeks for reputation with, but not without, the reality of goodness. And our Cretan colony ought also to acquire the fairest and noblest reputation for virtue from other men; and there is every reason to expect that, if the reality answers to the idea, she will be one of the few well-ordered cities which the sun and the other Gods behold. Wherefore, in the matter of journeys to other countries and the reception of strangers, we enact as follows: In the first place, let no one be allowed to go anywhere at all into a foreign country who is less than forty years of age; and no one shall go in a private capacity, but only in some public one, as a herald, or on an embassy, or on a sacred mission. Going abroad on an expedition or in war is not to be included among travels of the class authorised by the state. To Apollo at Delphi and to Zeus at Olympia and to Nemea and to the Isthmus, citizens should be sent to take part in the sacrifices and games there dedicated to the Gods; and they should send as many as possible, and the best and fairest that can be found, and they will make the city renowned at holy meetings in time of peace, procuring a glory which shall be the converse of that which is gained in war; and when they come home they shall teach the young that the institutions of other states are inferior to their own. And they shall send spectators of another sort, if they have the consent of the guardians, being such citizens as desire to look a little more at leisure at the doings of other men; and these no law shall hinder. For a city which has no experience of good and bad men or intercourse with them, can never be thoroughly and perfectly civilised, nor, again, can the citizens of a city properly observe the laws by habit only, and without an intelligent understanding of them. And there always are in the world a few inspired men whose acquaintance is beyond price, and who spring up quite as much in ill-ordered as in well-ordered cities. These are they whom the citizens of a well-ordered city should be ever seeking out, going forth over sea and over land to find him who is incorruptible—that he may establish more firmly institutions in his own state which are good already, and amend what is deficient; for without this examination and enquiry a city will never continue perfect any more than if the examination is ill-conducted.

\par \textbf{CLEINIAS}
\par   How can we have an examination and also a good one?

\par \textbf{ATHENIAN}
\par   In this way:  In the first place, our spectator shall be of not less than fifty years of age; he must be a man of reputation, especially in war, if he is to exhibit to other cities a model of the guardians of the law, but when he is more than sixty years of age he shall no longer continue in his office of spectator. And when he has carried on his inspection during as many out of the ten years of his office as he pleases, on his return home let him go to the assembly of those who review the laws. This shall be a mixed body of young and old men, who shall be required to meet daily between the hour of dawn and the rising of the sun. They shall consist, in the first place, of the priests who have obtained the rewards of virtue; and, in the second place, of guardians of the law, the ten eldest being chosen; the general superintendent of education shall also be a member, as well as the last appointed as those who have been released from the office; and each of them shall take with him as his companion a young man, whomsoever he chooses, between the ages of thirty and forty. These shall be always holding conversation and discourse about the laws of their own city or about any specially good ones which they may hear to be existing elsewhere; also about kinds of knowledge which may appear to be of use and will throw light upon the examination, or of which the want will make the subject of laws dark and uncertain to them. Any knowledge of this sort which the elders approve, the younger men shall learn with all diligence; and if any one of those who have been invited appear to be unworthy, the whole assembly shall blame him who invited him. The rest of the city shall watch over those among the young men who distinguish themselves, having an eye upon them, and especially honouring them if they succeed, but dishonouring them above the rest if they turn out to be inferior. This is the assembly to which he who has visited the institutions of other men, on his return home shall straightway go, and if he have discovered any one who has anything to say about the enactment of laws or education or nurture, or if he have himself made any observations, let him communicate his discoveries to the whole assembly. And if he be seen to have come home neither better nor worse, let him be praised at any rate for his enthusiasm; and if he be much better, let him be praised so much the more; and not only while he lives but after his death let the assembly honour him with fitting honours. But if on his return home he appear to have been corrupted, pretending to be wise when he is not, let him hold no communication with any one, whether young or old; and if he will hearken to the rulers, then he shall be permitted to live as a private individual; but if he will not, let him die, if he be convicted in a court of law of interfering about education and the laws. And if he deserve to be indicted, and none of the magistrates indict him, let that be counted as a disgrace to them when the rewards of virtue are decided.

\par  Let such be the character of the person who goes abroad, and let him go abroad under these conditions. In the next place, the stranger who comes from abroad should be received in a friendly spirit. Now there are four kinds of strangers, of whom we must make some mention—the first is he who comes and stays throughout the summer; this class are like birds of passage, taking wing in pursuit of commerce, and flying over the sea to other cities, while the season lasts; he shall be received in market-places and harbours and public buildings, near the city but outside, by those magistrates who are appointed to superintend these matters; and they shall take care that a stranger, whoever he be, duly receives justice; but he shall not be allowed to make any innovation. They shall hold the intercourse with him which is necessary, and this shall be as little as possible. The second kind is just a spectator who comes to see with his eyes and hear with his ears the festivals of the Muses; such ought to have entertainment provided them at the temples by hospitable persons, and the priests and ministers of the temples should see and attend to them. But they should not remain more than a reasonable time; let them see and hear that for the sake of which they came, and then go away, neither having suffered nor done any harm. The priests shall be their judges, if any of them receive or do any wrong up to the sum of fifty drachmae, but if any greater charge be brought, in such cases the suit shall come before the wardens of the agora. The third kind of stranger is he who comes on some public business from another land, and is to be received with public honours. He is to be received only by the generals and commanders of horse and foot, and the host by whom he is entertained, in conjunction with the Prytanes, shall have the sole charge of what concerns him. There is a fourth class of persons answering to our spectators, who come from another land to look at ours. In the first place, such visits will be rare, and the visitor should be at least fifty years of age; he may possibly be wanting to see something that is rich and rare in other states, or himself to show something in like manner to another city. Let such an one, then, go unbidden to the doors of the wise and rich, being one of them himself: let him go, for example, to the house of the superintendent of education, confident that he is a fitting guest of such a host, or let him go to the house of some of those who have gained the prize of virtue and hold discourse with them, both learning from them, and also teaching them; and when he has seen and heard all, he shall depart, as a friend taking leave of friends, and be honoured by them with gifts and suitable tributes of respect. These are the customs, according to which our city should receive all strangers of either sex who come from other countries, and should send forth her own citizens, showing respect to Zeus, the God of hospitality, not forbidding strangers at meals and sacrifices, as is the manner which prevails among the children of the Nile, nor driving them away by savage proclamations.

\par  When a man becomes surety, let him give the security in a distinct form, acknowledging the whole transaction in a written document, and in the presence of not less than three witnesses if the sum be under a thousand drachmae, and of not less than five witnesses if the sum be above a thousand drachmae. The agent of a dishonest or untrustworthy seller shall himself be responsible; both the agent and the principal shall be equally liable. If a person wishes to find anything in the house of another, he shall enter naked, or wearing only a short tunic and without a girdle, having first taken an oath by the customary Gods that he expects to find it there; he shall then make his search, and the other shall throw open his house and allow him to search things both sealed and unsealed. And if a person will not allow the searcher to make his search, he who is prevented shall go to law with him, estimating the value of the goods after which he is searching, and if the other be convicted he shall pay twice the value of the article. If the master be absent from home, the dwellers in the house shall let him search the unsealed property, and on the sealed property the searcher shall set another seal, and shall appoint any one whom he likes to guard them during five days; and if the master of the house be absent during a longer time, he shall take with him the wardens of the city, and so make his search, opening the sealed property as well as the unsealed, and then, together with the members of the family and the wardens of the city, he shall seal them up again as they were before. There shall be a limit of time in the case of disputed things, and he who has had possession of them during a certain time shall no longer be liable to be disturbed. As to houses and lands there can be no dispute in this state of ours; but if a man has any other possessions which he has used and openly shown in the city and in the agora and in the temples, and no one has put in a claim to them, and some one says that he was looking for them during this time, and the possessor is proved to have made no concealment, if they have continued for a year, the one having the goods and the other looking for them, the claim of the seeker shall not be allowed after the expiration of the year; or if he does not use or show the lost property in the market or in the city, but only in the country, and no one offers himself as the owner during five years, at the expiration of the five years the claim shall be barred for ever after; or if he uses them in the city but within the house, then the appointed time of claiming the goods shall be three years, or ten years if he has them in the country in private. And if he has them in another land, there shall be no limit of time or prescription, but whenever the owner finds them he may claim them.

\par  If any one prevents another by force from being present at a trial, whether a principal party or his witnesses; if the person prevented be a slave, whether his own or belonging to another, the suit shall be incomplete and invalid; but if he who is prevented be a freeman, besides the suit being incomplete, the other who has prevented him shall be imprisoned for a year, and shall be prosecuted for kidnapping by any one who pleases. And if any one hinders by force a rival competitor in gymnastic or music, or any other sort of contest, from being present at the contest, let him who has a mind inform the presiding judges, and they shall liberate him who is desirous of competing; and if they are not able, and he who hinders the other from competing wins the prize, then they shall give the prize of victory to him who is prevented, and inscribe him as the conqueror in any temples which he pleases; and he who hinders the other shall not be permitted to make any offering or inscription having reference to that contest, and in any case he shall be liable for damages, whether he be defeated or whether he conquer.

\par  If any one knowingly receives anything which has been stolen, he shall undergo the same punishment as the thief, and if a man receives an exile he shall be punished with death. Every man should regard the friend and enemy of the state as his own friend and enemy; and if any one makes peace or war with another on his own account, and without the authority of the state, he, like the receiver of the exile, shall undergo the penalty of death. And if any fraction of the city declare war or peace against any, the generals shall indict the authors of this proceeding, and if they are convicted death shall be the penalty. Those who serve their country ought to serve without receiving gifts, and there ought to be no excusing or approving the saying, 'Men should receive gifts as the reward of good, but not of evil deeds'; for to know which we are doing, and to stand fast by our knowledge, is no easy matter. The safest course is to obey the law which says, 'Do no service for a bribe,' and let him who disobeys, if he be convicted, simply die. With a view to taxation, for various reasons, every man ought to have had his property valued: and the tribesmen should likewise bring a register of the yearly produce to the wardens of the country, that in this way there may be two valuations; and the public officers may use annually whichever on consideration they deem the best, whether they prefer to take a certain portion of the whole value, or of the annual revenue, after subtracting what is paid to the common tables.

\par  Touching offerings to the Gods, a moderate man should observe moderation in what he offers. Now the land and the hearth of the house of all men is sacred to all Gods; wherefore let no man dedicate them a second time to the Gods. Gold and silver, whether possessed by private persons or in temples, are in other cities provocative of envy, and ivory, the product of a dead body, is not a proper offering; brass and iron, again, are instruments of war; but of wood let a man bring what offering he likes, provided it be a single block, and in like manner of stone, to the public temples; of woven work let him not offer more than one woman can execute in a month. White is a colour suitable to the Gods, especially in woven works, but dyes should only be used for the adornments of war. The most divine of gifts are birds and images, and they should be such as one painter can execute in a single day. And let all other offerings follow a similar rule.

\par  Now that the whole city has been divided into parts of which the nature and number have been described, and laws have been given about all the most important contracts as far as this was possible, the next thing will be to have justice done. The first of the courts shall consist of elected judges, who shall be chosen by the plaintiff and the defendant in common: these shall be called arbiters rather than judges. And in the second court there shall be judges of the villages and tribes corresponding to the twelvefold division of the land, and before these the litigants shall go to contend for greater damages, if the suit be not decided before the first judges; the defendant, if he be defeated the second time, shall pay a fifth more than the damages mentioned in the indictment; and if he find fault with his judges and would try a third time, let him carry the suit before the select judges, and if he be again defeated, let him pay the whole of the damages and half as much again. And the plaintiff, if when defeated before the first judges he persist in going on to the second, shall if he wins receive in addition to the damages a fifth part more, and if defeated he shall pay a like sum; but if he is not satisfied with the previous decision, and will insist on proceeding to a third court, then if he win he shall receive from the defendant the amount of the damages and, as I said before, half as much again, and the plaintiff, if he lose, shall pay half of the damages claimed. Now the assignment by lot of judges to courts and the completion of the number of them, and the appointment of servants to the different magistrates, and the times at which the several causes should be heard, and the votings and delays, and all the things that necessarily concern suits, and the order of causes, and the time in which answers have to be put in and parties are to appear—of these and other things akin to these we have indeed already spoken, but there is no harm in repeating what is right twice or thrice: All lesser and easier matters which the elder legislator has omitted may be supplied by the younger one. Private courts will be sufficiently regulated in this way, and the public and state courts, and those which the magistrates must use in the administration of their several offices, exist in many other states. Many very respectable institutions of this sort have been framed by good men, and from them the guardians of the law may by reflection derive what is necessary for the order of our new state, considering and correcting them, and bringing them to the test of experience, until every detail appears to be satisfactorily determined; and then putting the final seal upon them, and making them irreversible, they shall use them for ever afterwards. As to what relates to the silence of judges and the abstinence from words of evil omen and the reverse, and the different notions of the just and good and honourable which exist in our own as compared with other states, they have been partly mentioned already, and another part of them will be mentioned hereafter as we draw near the end. To all these matters he who would be an equal judge shall justly look, and he shall possess writings about them that he may learn them. For of all kinds of knowledge the knowledge of good laws has the greatest power of improving the learner; otherwise there would be no meaning in the divine and admirable law possessing a name akin to mind (nous, nomos). And of all other words, such as the praises and censures of individuals which occur in poetry and also in prose, whether written down or uttered in daily conversation, whether men dispute about them in the spirit of contention or weakly assent to them, as is often the case—of all these the one sure test is the writings of the legislator, which the righteous judge ought to have in his mind as the antidote of all other words, and thus make himself and the city stand upright, procuring for the good the continuance and increase of justice, and for the bad, on the other hand, a conversion from ignorance and intemperance, and in general from all unrighteousness, as far as their evil minds can be healed, but to those whose web of life is in reality finished, giving death, which is the only remedy for souls in their condition, as I may say truly again and again. And such judges and chiefs of judges will be worthy of receiving praise from the whole city.

\par  When the suits of the year are completed the following laws shall regulate their execution: In the first place, the judge shall assign to the party who wins the suit the whole property of him who loses, with the exception of mere necessaries, and the assignment shall be made through the herald immediately after each decision in the hearing of the judges; and when the month arrives following the month in which the courts are sitting, (unless the gainer of the suit has been previously satisfied) the court shall follow up the case, and hand over to the winner the goods of the loser; but if they find that he has not the means of paying, and the sum deficient is not less than a drachma, the insolvent person shall not have any right of going to law with any other man until he have satisfied the debt of the winning party; but other persons shall still have the right of bringing suits against him. And if any one after he is condemned refuses to acknowledge the authority which condemned him, let the magistrates who are thus deprived of their authority bring him before the court of the guardians of the law, and if he be cast, let him be punished with death, as a subverter of the whole state and of the laws.

\par  Thus a man is born and brought up, and after this manner he begets and brings up his own children, and has his share of dealings with other men, and suffers if he has done wrong to any one, and receives satisfaction if he has been wronged, and so at length in due time he grows old under the protection of the laws, and his end comes in the order of nature. Concerning the dead of either sex, the religious ceremonies which may fittingly be performed, whether appertaining to the Gods of the under-world or of this, shall be decided by the interpreters with absolute authority. Their sepulchres are not to be in places which are fit for cultivation, and there shall be no monuments in such spots, either large or small, but they shall occupy that part of the country which is naturally adapted for receiving and concealing the bodies of the dead with as little hurt as possible to the living. No man, living or dead, shall deprive the living of the sustenance which the earth, their foster-parent, is naturally inclined to provide for them. And let not the mound be piled higher than would be the work of five men completed in five days; nor shall the stone which is placed over the spot be larger than would be sufficient to receive the praises of the dead included in four heroic lines. Nor shall the laying out of the dead in the house continue for a longer time than is sufficient to distinguish between him who is in a trance only and him who is really dead, and speaking generally, the third day after death will be a fair time for carrying out the body to the sepulchre. Now we must believe the legislator when he tells us that the soul is in all respects superior to the body, and that even in life what makes each one of us to be what we are is only the soul; and that the body follows us about in the likeness of each of us, and therefore, when we are dead, the bodies of the dead are quite rightly said to be our shades or images; for the true and immortal being of each one of us which is called the soul goes on her way to other Gods, before them to give an account—which is an inspiring hope to the good, but very terrible to the bad, as the laws of our fathers tell us; and they also say that not much can be done in the way of helping a man after he is dead. But the living—he should be helped by all his kindred, that while in life he may be the holiest and justest of men, and after death may have no great sins to be punished in the world below. If this be true, a man ought not to waste his substance under the idea that all this lifeless mass of flesh which is in process of burial is connected with him; he should consider that the son, or brother, or the beloved one, whoever he may be, whom he thinks he is laying in the earth, has gone away to complete and fulfil his own destiny, and that his duty is rightly to order the present, and to spend moderately on the lifeless altar of the Gods below. But the legislator does not intend moderation to be taken in the sense of meanness. Let the law, then, be as follows: The expenditure on the entire funeral of him who is of the highest class, shall not exceed five minae; and for him who is of the second class, three minae, and for him who is of the third class, two minae, and for him who is of the fourth class, one mina, will be a fair limit of expense. The guardians of the law ought to take especial care of the different ages of life, whether childhood, or manhood, or any other age. And at the end of all, let there be some one guardian of the law presiding, who shall be chosen by the friends of the deceased to superintend, and let it be glory to him to manage with fairness and moderation what relates to the dead, and a discredit to him if they are not well managed. Let the laying out and other ceremonies be in accordance with custom, but to the statesman who adopts custom as his law we must give way in certain particulars. It would be monstrous for example that he should command any man to weep or abstain from weeping over the dead; but he may forbid cries of lamentation, and not allow the voice of the mourner to be heard outside the house; also, he may forbid the bringing of the dead body into the open streets, or the processions of mourners in the streets, and may require that before daybreak they should be outside the city. Let these, then, be our laws relating to such matters, and let him who obeys be free from penalty; but he who disobeys even a single guardian of the law shall be punished by them all with a fitting penalty. Other modes of burial, or again the denial of burial, which is to be refused in the case of robbers of temples and parricides and the like, have been devised and are embodied in the preceding laws, so that now our work of legislation is pretty nearly at an end; but in all cases the end does not consist in doing something or acquiring something or establishing something—the end will be attained and finally accomplished, when we have provided for the perfect and lasting continuance of our institutions; until then our creation is incomplete.

\par \textbf{CLEINIAS}
\par   That is very good, Stranger; but I wish you would tell me more clearly what you mean.

\par \textbf{ATHENIAN}
\par   O Cleinias, many things of old time were well said and sung; and the saying about the Fates was one of them.

\par \textbf{CLEINIAS}
\par   What is it?

\par \textbf{ATHENIAN}
\par   The saying that Lachesis or the giver of the lots is the first of them, and that Clotho or the spinster is the second of them, and that Atropos or the unchanging one is the third of them; and that she is the preserver of the things which we have spoken, and which have been compared in a figure to things woven by fire, they both (i.e. Atropos and the fire) producing the quality of unchangeableness. I am speaking of the things which in a state and government give not only health and salvation to the body, but law, or rather preservation of the law, in the soul; and, if I am not mistaken, this seems to be still wanting in our laws:  we have still to see how we can implant in them this irreversible nature.

\par \textbf{CLEINIAS}
\par   It will be no small matter if we can only discover how such a nature can be implanted in anything.

\par \textbf{ATHENIAN}
\par   But it certainly can be; so much I clearly see.

\par \textbf{CLEINIAS}
\par   Then let us not think of desisting until we have imparted this quality to our laws; for it is ridiculous, after a great deal of labour has been spent, to place a thing at last on an insecure foundation.

\par \textbf{ATHENIAN}
\par   I approve of your suggestion, and am quite of the same mind with you.

\par \textbf{CLEINIAS}
\par   Very good:  And now what, according to you, is to be the salvation of our government and of our laws, and how is it to be effected?

\par \textbf{ATHENIAN}
\par   Were we not saying that there must be in our city a council which was to be of this sort:  The ten oldest guardians of the law, and all those who have obtained prizes of virtue, were to meet in the same assembly, and the council was also to include those who had visited foreign countries in the hope of hearing something that might be of use in the preservation of the laws, and who, having come safely home, and having been tested in these same matters, had proved themselves to be worthy to take part in the assembly—each of the members was to select some young man of not less than thirty years of age, he himself judging in the first instance whether the young man was worthy by nature and education, and then suggesting him to the others, and if he seemed to them also to be worthy they were to adopt him; but if not, the decision at which they arrived was to be kept a secret from the citizens at large, and, more especially, from the rejected candidate. The meeting of the council was to be held early in the morning, when everybody was most at leisure from all other business, whether public or private—was not something of this sort said by us before?

\par \textbf{CLEINIAS}
\par   True.

\par \textbf{ATHENIAN}
\par   Then, returning to the council, I would say further, that if we let it down to be the anchor of the state, our city, having everything which is suitable to her, will preserve all that we wish to preserve.

\par \textbf{CLEINIAS}
\par   What do you mean?

\par \textbf{ATHENIAN}
\par   Now is the time for me to speak the truth in all earnestness.

\par \textbf{CLEINIAS}
\par   Well said, and I hope that you will fulfil your intention.

\par \textbf{ATHENIAN}
\par   Know, Cleinias, that everything, in all that it does, has a natural saviour, as of an animal the soul and the head are the chief saviours.

\par \textbf{CLEINIAS}
\par   Once more, what do you mean?

\par \textbf{ATHENIAN}
\par   The well-being of those two is obviously the preservation of every living thing.

\par \textbf{CLEINIAS}
\par   How is that?

\par \textbf{ATHENIAN}
\par   The soul, besides other things, contains mind, and the head, besides other things, contains sight and hearing; and the mind, mingling with the noblest of the senses, and becoming one with them, may be truly called the salvation of all.

\par \textbf{CLEINIAS}
\par   Yes, quite so.

\par \textbf{ATHENIAN}
\par   Yes, indeed; but with what is that intellect concerned which, mingling with the senses, is the salvation of ships in storms as well as in fair weather? In a ship, when the pilot and the sailors unite their perceptions with the piloting mind, do they not save both themselves and their craft?

\par \textbf{CLEINIAS}
\par   Very true.

\par \textbf{ATHENIAN}
\par   We do not want many illustrations about such matters:  What aim would the general of an army, or what aim would a physician propose to himself, if he were seeking to attain salvation?

\par \textbf{CLEINIAS}
\par   Very good.

\par \textbf{ATHENIAN}
\par   Does not the general aim at victory and superiority in war, and do not the physician and his assistants aim at producing health in the body?

\par \textbf{CLEINIAS}
\par   Certainly.

\par \textbf{ATHENIAN}
\par   And a physician who is ignorant about the body, that is to say, who knows not that which we just now called health, or a general who knows not victory, or any others who are ignorant of the particulars of the arts which we mentioned, cannot be said to have understanding about any of these matters.

\par \textbf{CLEINIAS}
\par   They cannot.

\par \textbf{ATHENIAN}
\par   And what would you say of the state? If a person proves to be ignorant of the aim to which the statesman should look, ought he, in the first place, to be called a ruler at all; and further, will he ever be able to preserve that of which he does not even know the aim?

\par \textbf{CLEINIAS}
\par   Impossible.

\par \textbf{ATHENIAN}
\par   And therefore, if our settlement of the country is to be perfect, we ought to have some institution, which, as I was saying, will tell what is the aim of the state, and will inform us how we are to attain this, and what law or what man will advise us to that end. Any state which has no such institution is likely to be devoid of mind and sense, and in all her actions will proceed by mere chance.

\par \textbf{CLEINIAS}
\par   Very true.

\par \textbf{ATHENIAN}
\par   In which, then, of the parts or institutions of the state is any such guardian power to be found? Can we say?

\par \textbf{CLEINIAS}
\par   I am not quite certain, Stranger; but I have a suspicion that you are referring to the assembly which you just now said was to meet at night.

\par \textbf{ATHENIAN}
\par   You understand me perfectly, Cleinias; and we must assume, as the argument implies, that this council possesses all virtue; and the beginning of virtue is not to make mistakes by guessing many things, but to look steadily at one thing, and on this to fix all our aims.

\par \textbf{CLEINIAS}
\par   Quite true.

\par \textbf{ATHENIAN}
\par   Then now we shall see why there is nothing wonderful in states going astray—the reason is that their legislators have such different aims; nor is there anything wonderful in some laying down as their rule of justice, that certain individuals should bear rule in the state, whether they be good or bad, and others that the citizens should be rich, not caring whether they are the slaves of other men or not. The tendency of others, again, is towards freedom; and some legislate with a view to two things at once—they want to be at the same time free and the lords of other states; but the wisest men, as they deem themselves to be, look to all these and similar aims, and there is no one of them which they exclusively honour, and to which they would have all things look.

\par \textbf{CLEINIAS}
\par   Then, Stranger, our former assertion will hold; for we were saying that laws generally should look to one thing only; and this, as we admitted, was rightly said to be virtue.

\par \textbf{ATHENIAN}
\par   Yes.

\par \textbf{CLEINIAS}
\par   And we said that virtue was of four kinds?

\par \textbf{ATHENIAN}
\par   Quite true.

\par \textbf{CLEINIAS}
\par   And that mind was the leader of the four, and that to her the three other virtues and all other things ought to have regard?

\par \textbf{ATHENIAN}
\par   You follow me capitally, Cleinias, and I would ask you to follow me to the end, for we have already said that the mind of the pilot, the mind of the physician and of the general look to that one thing to which they ought to look; and now we may turn to mind political, of which, as of a human creature, we will ask a question:  O wonderful being, and to what are you looking? The physician is able to tell his single aim in life, but you, the superior, as you declare yourself to be, of all intelligent beings, when you are asked are not able to tell. Can you, Megillus, and you, Cleinias, say distinctly what is the aim of mind political, in return for the many explanations of things which I have given you?

\par \textbf{CLEINIAS}
\par   We cannot, Stranger.

\par \textbf{ATHENIAN}
\par   Well, but ought we not to desire to see it, and to see where it is to be found?

\par \textbf{CLEINIAS}
\par   For example, where?

\par \textbf{ATHENIAN}
\par   For example, we were saying that there are four kinds of virtue, and as there are four of them, each of them must be one.

\par \textbf{CLEINIAS}
\par   Certainly.

\par \textbf{ATHENIAN}
\par   And further, all four of them we call one; for we say that courage is virtue, and that prudence is virtue, and the same of the two others, as if they were in reality not many but one, that is, virtue.

\par \textbf{CLEINIAS}
\par   Quite so.

\par \textbf{ATHENIAN}
\par   There is no difficulty in seeing in what way the two differ from one another, and have received two names, and so of the rest. But there is more difficulty in explaining why we call these two and the rest of them by the single name of virtue.

\par \textbf{CLEINIAS}
\par   How do you mean?

\par \textbf{ATHENIAN}
\par   I have no difficulty in explaining what I mean. Let us distribute the subject into questions and answers.

\par \textbf{CLEINIAS}
\par   Once more, what do you mean?

\par \textbf{ATHENIAN}
\par   Ask me what is that one thing which I call virtue, and then again speak of as two, one part being courage and the other wisdom. I will tell you how that occurs:  One of them has to do with fear; in this the beasts also participate, and quite young children—I mean courage; for a courageous temper is a gift of nature and not of reason. But without reason there never has been, or is, or will be a wise and understanding soul; it is of a different nature.

\par \textbf{CLEINIAS}
\par   That is true.

\par \textbf{ATHENIAN}
\par   I have now told you in what way the two are different, and do you in return tell me in what way they are one and the same. Suppose that I ask you in what way the four are one, and when you have answered me, you will have a right to ask of me in return in what way they are four; and then let us proceed to enquire whether in the case of things which have a name and also a definition to them, true knowledge consists in knowing the name only and not the definition. Can he who is good for anything be ignorant of all this without discredit where great and glorious truths are concerned?

\par \textbf{CLEINIAS}
\par   I suppose not.

\par \textbf{ATHENIAN}
\par   And is there anything greater to the legislator and the guardian of the law, and to him who thinks that he excels all other men in virtue, and has won the palm of excellence, than these very qualities of which we are now speaking—courage, temperance, wisdom, justice?

\par \textbf{CLEINIAS}
\par   How can there be anything greater?

\par \textbf{ATHENIAN}
\par   And ought not the interpreters, the teachers, the lawgivers, the guardians of the other citizens, to excel the rest of mankind, and perfectly to show him who desires to learn and know or whose evil actions require to be punished and reproved, what is the nature of virtue and vice? Or shall some poet who has found his way into the city, or some chance person who pretends to be an instructor of youth, show himself to be better than him who has won the prize for every virtue? And can we wonder that when the guardians are not adequate in speech or action, and have no adequate knowledge of virtue, the city being unguarded should experience the common fate of cities in our day?

\par \textbf{CLEINIAS}
\par   Wonder! no.

\par \textbf{ATHENIAN}
\par   Well, then, must we do as we said? Or can we give our guardians a more precise knowledge of virtue in speech and action than the many have? or is there any way in which our city can be made to resemble the head and senses of rational beings because possessing such a guardian power?

\par \textbf{CLEINIAS}
\par   What, Stranger, is the drift of your comparison?

\par \textbf{ATHENIAN}
\par   Do we not see that the city is the trunk, and are not the younger guardians, who are chosen for their natural gifts, placed in the head of the state, having their souls all full of eyes, with which they look about the whole city? They keep watch and hand over their perceptions to the memory, and inform the elders of all that happens in the city; and those whom we compared to the mind, because they have many wise thoughts—that is to say, the old men—take counsel, and making use of the younger men as their ministers, and advising with them—in this way both together truly preserve the whole state:  Shall this or some other be the order of our state? Are all our citizens to be equal in acquirements, or shall there be special persons among them who have received a more careful training and education?

\par \textbf{CLEINIAS}
\par   That they should be equal, my good sir, is impossible.

\par \textbf{ATHENIAN}
\par   Then we ought to proceed to some more exact training than any which has preceded.

\par \textbf{CLEINIAS}
\par   Certainly.

\par \textbf{ATHENIAN}
\par   And must not that of which we are in need be the one to which we were just now alluding?

\par \textbf{CLEINIAS}
\par   Very true.

\par \textbf{ATHENIAN}
\par   Did we not say that the workman or guardian, if he be perfect in every respect, ought not only to be able to see the many aims, but he should press onward to the one? This he should know, and knowing, order all things with a view to it.

\par \textbf{CLEINIAS}
\par   True.

\par \textbf{ATHENIAN}
\par   And can any one have a more exact way of considering or contemplating anything, than the being able to look at one idea gathered from many different things?

\par \textbf{CLEINIAS}
\par   Perhaps not.

\par \textbf{ATHENIAN}
\par   Not 'Perhaps not,' but 'Certainly not,' my good sir, is the right answer. There never has been a truer method than this discovered by any man.

\par \textbf{CLEINIAS}
\par   I bow to your authority, Stranger; let us proceed in the way which you propose.

\par \textbf{ATHENIAN}
\par   Then, as would appear, we must compel the guardians of our divine state to perceive, in the first place, what that principle is which is the same in all the four—the same, as we affirm, in courage and in temperance, and in justice and in prudence, and which, being one, we call as we ought, by the single name of virtue. To this, my friends, we will, if you please, hold fast, and not let go until we have sufficiently explained what that is to which we are to look, whether to be regarded as one, or as a whole, or as both, or in whatever way. Are we likely ever to be in a virtuous condition, if we cannot tell whether virtue is many, or four, or one? Certainly, if we take counsel among ourselves, we shall in some way contrive that this principle has a place amongst us; but if you have made up your mind that we should let the matter alone, we will.

\par \textbf{CLEINIAS}
\par   We must not, Stranger, by the God of strangers I swear that we must not, for in our opinion you speak most truly; but we should like to know how you will accomplish your purpose.

\par \textbf{ATHENIAN}
\par   Wait a little before you ask; and let us, first of all, be quite agreed with one another that the purpose has to be accomplished.

\par \textbf{CLEINIAS}
\par   Certainly, it ought to be, if it can be.

\par \textbf{ATHENIAN}
\par   Well, and about the good and the honourable, are we to take the same view? Are our guardians only to know that each of them is many, or also how and in what way they are one?

\par \textbf{CLEINIAS}
\par   They must consider also in what sense they are one.

\par \textbf{ATHENIAN}
\par   And are they to consider only, and to be unable to set forth what they think?

\par \textbf{CLEINIAS}
\par   Certainly not; that would be the state of a slave.

\par \textbf{ATHENIAN}
\par   And may not the same be said of all good things—that the true guardians of the laws ought to know the truth about them, and to be able to interpret them in words, and carry them out in action, judging of what is and of what is not well, according to nature?

\par \textbf{CLEINIAS}
\par   Certainly.

\par \textbf{ATHENIAN}
\par   Is not the knowledge of the Gods which we have set forth with so much zeal one of the noblest sorts of knowledge—to know that they are, and know how great is their power, as far as in man lies? We do indeed excuse the mass of the citizens, who only follow the voice of the laws, but we refuse to admit as guardians any who do not labour to obtain every possible evidence that there is respecting the Gods; our city is forbidden and not allowed to choose as a guardian of the law, or to place in the select order of virtue, him who is not an inspired man, and has not laboured at these things.

\par \textbf{CLEINIAS}
\par   It is certainly just, as you say, that he who is indolent about such matters or incapable should be rejected, and that things honourable should be put away from him.

\par \textbf{ATHENIAN}
\par   Are we assured that there are two things which lead men to believe in the Gods, as we have already stated?

\par \textbf{CLEINIAS}
\par   What are they?

\par \textbf{ATHENIAN}
\par   One is the argument about the soul, which has been already mentioned—that it is the eldest and most divine of all things, to which motion attaining generation gives perpetual existence; the other was an argument from the order of the motion of the stars, and of all things under the dominion of the mind which ordered the universe. If a man look upon the world not lightly or ignorantly, there was never any one so godless who did not experience an effect opposite to that which the many imagine. For they think that those who handle these matters by the help of astronomy, and the accompanying arts of demonstration, may become godless, because they see, as far as they can see, things happening by necessity, and not by an intelligent will accomplishing good.

\par \textbf{CLEINIAS}
\par   But what is the fact?

\par \textbf{ATHENIAN}
\par   Just the opposite, as I said, of the opinion which once prevailed among men, that the sun and stars are without soul. Even in those days men wondered about them, and that which is now ascertained was then conjectured by some who had a more exact knowledge of them—that if they had been things without soul, and had no mind, they could never have moved with numerical exactness so wonderful; and even at that time some ventured to hazard the conjecture that mind was the orderer of the universe. But these same persons again mistaking the nature of the soul, which they conceived to be younger and not older than the body, once more overturned the world, or rather, I should say, themselves; for the bodies which they saw moving in heaven all appeared to be full of stones, and earth, and many other lifeless substances, and to these they assigned the causes of all things. Such studies gave rise to much atheism and perplexity, and the poets took occasion to be abusive—comparing the philosophers to she-dogs uttering vain howlings, and talking other nonsense of the same sort. But now, as I said, the case is reversed.

\par \textbf{CLEINIAS}
\par   How so?

\par \textbf{ATHENIAN}
\par   No man can be a true worshipper of the Gods who does not know these two principles—that the soul is the eldest of all things which are born, and is immortal and rules over all bodies; moreover, as I have now said several times, he who has not contemplated the mind of nature which is said to exist in the stars, and gone through the previous training, and seen the connexion of music with these things, and harmonized them all with laws and institutions, is not able to give a reason of such things as have a reason. And he who is unable to acquire this in addition to the ordinary virtues of a citizen, can hardly be a good ruler of a whole state; but he should be the subordinate of other rulers. Wherefore, Cleinias and Megillus, let us consider whether we may not add to all the other laws which we have discussed this further one—that the nocturnal assembly of the magistrates, which has also shared in the whole scheme of education proposed by us, shall be a guard set according to law for the salvation of the state. Shall we propose this?

\par \textbf{CLEINIAS}
\par   Certainly, my good friend, we will if the thing is in any degree possible.

\par \textbf{ATHENIAN}
\par   Let us make a common effort to gain such an object; for I too will gladly share in the attempt. Of these matters I have had much experience, and have often considered them, and I dare say that I shall be able to find others who will also help.

\par \textbf{CLEINIAS}
\par   I agree, Stranger, that we should proceed along the road in which God is guiding us; and how we can proceed rightly has now to be investigated and explained.

\par \textbf{ATHENIAN}
\par   O Megillus and Cleinias, about these matters we cannot legislate further until the council is constituted; when that is done, then we will determine what authority they shall have of their own; but the explanation of how this is all to be ordered would only be given rightly in a long discourse.

\par \textbf{CLEINIAS}
\par   What do you mean, and what new thing is this?

\par \textbf{ATHENIAN}
\par   In the first place, a list would have to be made out of those who by their ages and studies and dispositions and habits are well fitted for the duty of a guardian. In the next place, it will not be easy for them to discover themselves what they ought to learn, or become the disciple of one who has already made the discovery. Furthermore, to write down the times at which, and during which, they ought to receive the several kinds of instruction, would be a vain thing; for the learners themselves do not know what is learned to advantage until the knowledge which is the result of learning has found a place in the soul of each. And so these details, although they could not be truly said to be secret, might be said to be incapable of being stated beforehand, because when stated they would have no meaning.

\par \textbf{CLEINIAS}
\par   What then are we to do, Stranger, under these circumstances?

\par \textbf{ATHENIAN}
\par   As the proverb says, the answer is no secret, but open to all of us:  We must risk the whole on the chance of throwing, as they say, thrice six or thrice ace, and I am willing to share with you the danger by stating and explaining to you my views about education and nurture, which is the question coming to the surface again. The danger is not a slight or ordinary one, and I would advise you, Cleinias, in particular, to see to the matter; for if you order rightly the city of the Magnetes, or whatever name God may give it, you will obtain the greatest glory; or at any rate you will be thought the most courageous of men in the estimation of posterity. Dear companions, if this our divine assembly can only be established, to them we will hand over the city; none of the present company of legislators, as I may call them, would hesitate about that. And the state will be perfected and become a waking reality, which a little while ago we attempted to create as a dream and in idea only, mingling together reason and mind in one image, in the hope that our citizens might be duly mingled and rightly educated; and being educated, and dwelling in the citadel of the land, might become perfect guardians, such as we have never seen in all our previous life, by reason of the saving virtue which is in them.

\par \textbf{MEGILLUS}
\par   Dear Cleinias, after all that has been said, either we must detain the Stranger, and by supplications and in all manner of ways make him share in the foundation of the city, or we must give up the undertaking.

\par \textbf{CLEINIAS}
\par   Very true, Megillus; and you must join with me in detaining him.

\par \textbf{MEGILLUS}
\par   I will.

\par 
 
\end{document}