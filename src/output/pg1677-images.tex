
\documentclass[11pt,letter]{article}


\begin{document}

\title{Alcibiades II\thanks{Source: https://www.gutenberg.org/files/1677/1677-h/1677-h.htm. License: http://gutenberg.org/license ds}}
\date{\today}
\author{Plato (spurious and doubtful works), 427? BCE-347? BCE\\ Translated by }
\maketitle

\setcounter{tocdepth}{1}
\tableofcontents
\renewcommand{\baselinestretch}{1.0}
\normalsize
\newpage

\section{
      APPENDIX II.
    }
\par  The two dialogues which are translated in the second appendix are not mentioned by Aristotle, or by any early authority, and have no claim to be ascribed to Plato. They are examples of Platonic dialogues to be assigned probably to the second or third generation after Plato, when his writings were well known at Athens and Alexandria. They exhibit considerable originality, and are remarkable for containing several thoughts of the sort which we suppose to be modern rather than ancient, and which therefore have a peculiar interest for us. The Second Alcibiades shows that the difficulties about prayer which have perplexed Christian theologians were not unknown among the followers of Plato. The Eryxias was doubted by the ancients themselves: yet it may claim the distinction of being, among all Greek or Roman writings, the one which anticipates in the most striking manner the modern science of political economy and gives an abstract form to some of its principal doctrines.

\par  For the translation of these two dialogues I am indebted to my friend and secretary, Mr. Knight.

\par  That the Dialogue which goes by the name of the Second Alcibiades is a genuine writing of Plato will not be maintained by any modern critic, and was hardly believed by the ancients themselves. The dialectic is poor and weak. There is no power over language, or beauty of style; and there is a certain abruptness and agroikia in the conversation, which is very un-Platonic. The best passage is probably that about the poets:—the remark that the poet, who is of a reserved disposition, is uncommonly difficult to understand, and the ridiculous interpretation of Homer, are entirely in the spirit of Plato (compare Protag; Ion; Apol.). The characters are ill-drawn. Socrates assumes the 'superior person' and preaches too much, while Alcibiades is stupid and heavy-in-hand. There are traces of Stoic influence in the general tone and phraseology of the Dialogue (compare opos melesei tis...kaka: oti pas aphron mainetai): and the writer seems to have been acquainted with the 'Laws' of Plato (compare Laws). An incident from the Symposium is rather clumsily introduced, and two somewhat hackneyed quotations (Symp., Gorg.) recur. The reference to the death of Archelaus as having occurred 'quite lately' is only a fiction, probably suggested by the Gorgias, where the story of Archelaus is told, and a similar phrase occurs;—ta gar echthes kai proen gegonota tauta, k.t.l. There are several passages which are either corrupt or extremely ill-expressed. But there is a modern interest in the subject of the dialogue; and it is a good example of a short spurious work, which may be attributed to the second or third century before Christ.

\par 
\section{
      ALCIBIADES II
    } 
\par \textbf{SOCRATES}
\par   Are you going, Alcibiades, to offer prayer to Zeus?

\par \textbf{ALCIBIADES}
\par   Yes, Socrates, I am.

\par \textbf{SOCRATES}
\par   you seem to be troubled and to cast your eyes on the ground, as though you were thinking about something.

\par \textbf{ALCIBIADES}
\par   Of what do you suppose that I am thinking?

\par \textbf{SOCRATES}
\par   Of the greatest of all things, as I believe. Tell me, do you not suppose that the Gods sometimes partly grant and partly reject the requests which we make in public and private, and favour some persons and not others?

\par \textbf{ALCIBIADES}
\par   Certainly.

\par \textbf{SOCRATES}
\par   Do you not imagine, then, that a man ought to be very careful, lest perchance without knowing it he implore great evils for himself, deeming that he is asking for good, especially if the Gods are in the mood to grant whatever he may request? There is the story of Oedipus, for instance, who prayed that his children might divide their inheritance between them by the sword:  he did not, as he might have done, beg that his present evils might be averted, but called down new ones. And was not his prayer accomplished, and did not many and terrible evils thence arise, upon which I need not dilate?

\par \textbf{ALCIBIADES}
\par   Yes, Socrates, but you are speaking of a madman:  surely you do not think that any one in his senses would venture to make such a prayer?

\par \textbf{SOCRATES}
\par   Madness, then, you consider to be the opposite of discretion?

\par \textbf{ALCIBIADES}
\par   Of course.

\par \textbf{SOCRATES}
\par   And some men seem to you to be discreet, and others the contrary?

\par \textbf{ALCIBIADES}
\par   They do.

\par \textbf{SOCRATES}
\par   Well, then, let us discuss who these are. We acknowledge that some are discreet, some foolish, and that some are mad?

\par \textbf{ALCIBIADES}
\par   Yes.

\par \textbf{SOCRATES}
\par   And again, there are some who are in health?

\par \textbf{ALCIBIADES}
\par   There are.

\par \textbf{SOCRATES}
\par   While others are ailing?

\par \textbf{ALCIBIADES}
\par   Yes.

\par \textbf{SOCRATES}
\par   And they are not the same?

\par \textbf{ALCIBIADES}
\par   Certainly not.

\par \textbf{SOCRATES}
\par   Nor are there any who are in neither state?

\par \textbf{ALCIBIADES}
\par   No.

\par \textbf{SOCRATES}
\par   A man must either be sick or be well?

\par \textbf{ALCIBIADES}
\par   That is my opinion.

\par \textbf{SOCRATES}
\par   Very good:  and do you think the same about discretion and want of discretion?

\par \textbf{ALCIBIADES}
\par   How do you mean?

\par \textbf{SOCRATES}
\par   Do you believe that a man must be either in or out of his senses; or is there some third or intermediate condition, in which he is neither one nor the other?

\par \textbf{ALCIBIADES}
\par   Decidedly not.

\par \textbf{SOCRATES}
\par   He must be either sane or insane?

\par \textbf{ALCIBIADES}
\par   So I suppose.

\par \textbf{SOCRATES}
\par   Did you not acknowledge that madness was the opposite of discretion?

\par \textbf{ALCIBIADES}
\par   Yes.

\par \textbf{SOCRATES}
\par   And that there is no third or middle term between discretion and indiscretion?

\par \textbf{ALCIBIADES}
\par   True.

\par \textbf{SOCRATES}
\par   And there cannot be two opposites to one thing?

\par \textbf{ALCIBIADES}
\par   There cannot.

\par \textbf{SOCRATES}
\par   Then madness and want of sense are the same?

\par \textbf{ALCIBIADES}
\par   That appears to be the case.

\par \textbf{SOCRATES}
\par   We shall be in the right, therefore, Alcibiades, if we say that all who are senseless are mad. For example, if among persons of your own age or older than yourself there are some who are senseless,—as there certainly are,—they are mad. For tell me, by heaven, do you not think that in the city the wise are few, while the foolish, whom you call mad, are many?

\par \textbf{ALCIBIADES}
\par   I do.

\par \textbf{SOCRATES}
\par   But how could we live in safety with so many crazy people? Should we not long since have paid the penalty at their hands, and have been struck and beaten and endured every other form of ill-usage which madmen are wont to inflict? Consider, my dear friend:  may it not be quite otherwise?

\par \textbf{ALCIBIADES}
\par   Why, Socrates, how is that possible? I must have been mistaken.

\par \textbf{SOCRATES}
\par   So it seems to me. But perhaps we may consider the matter thus: —

\par \textbf{ALCIBIADES}
\par   How?

\par \textbf{SOCRATES}
\par   I will tell you. We think that some are sick; do we not?

\par \textbf{ALCIBIADES}
\par   Yes.

\par \textbf{SOCRATES}
\par   And must every sick person either have the gout, or be in a fever, or suffer from ophthalmia? Or do you believe that a man may labour under some other disease, even although he has none of these complaints? Surely, they are not the only maladies which exist?

\par \textbf{ALCIBIADES}
\par   Certainly not.

\par \textbf{SOCRATES}
\par   And is every kind of ophthalmia a disease?

\par \textbf{ALCIBIADES}
\par   Yes.

\par \textbf{SOCRATES}
\par   And every disease ophthalmia?

\par \textbf{ALCIBIADES}
\par   Surely not. But I scarcely understand what I mean myself.

\par \textbf{SOCRATES}
\par   Perhaps, if you give me your best attention, 'two of us' looking together, we may find what we seek.

\par \textbf{ALCIBIADES}
\par   I am attending, Socrates, to the best of my power.

\par \textbf{SOCRATES}
\par   We are agreed, then, that every form of ophthalmia is a disease, but not every disease ophthalmia?

\par \textbf{ALCIBIADES}
\par   We are.

\par \textbf{SOCRATES}
\par   And so far we seem to be right. For every one who suffers from a fever is sick; but the sick, I conceive, do not all have fever or gout or ophthalmia, although each of these is a disease, which, according to those whom we call physicians, may require a different treatment. They are not all alike, nor do they produce the same result, but each has its own effect, and yet they are all diseases. May we not take an illustration from the artizans?

\par \textbf{ALCIBIADES}
\par   Certainly.

\par \textbf{SOCRATES}
\par   There are cobblers and carpenters and sculptors and others of all sorts and kinds, whom we need not stop to enumerate. All have their distinct employments and all are workmen, although they are not all of them cobblers or carpenters or sculptors.

\par \textbf{ALCIBIADES}
\par   No, indeed.

\par \textbf{SOCRATES}
\par   And in like manner men differ in regard to want of sense. Those who are most out of their wits we call 'madmen,' while we term those who are less far gone 'stupid' or 'idiotic,' or, if we prefer gentler language, describe them as 'romantic' or 'simple-minded,' or, again, as 'innocent' or 'inexperienced' or 'foolish.' You may even find other names, if you seek for them; but by all of them lack of sense is intended. They only differ as one art appeared to us to differ from another or one disease from another. Or what is your opinion?

\par \textbf{ALCIBIADES}
\par   I agree with you.

\par \textbf{SOCRATES}
\par   Then let us return to the point at which we digressed. We said at first that we should have to consider who were the wise and who the foolish. For we acknowledged that there are these two classes? Did we not?

\par \textbf{ALCIBIADES}
\par   To be sure.

\par \textbf{SOCRATES}
\par   And you regard those as sensible who know what ought to be done or said?

\par \textbf{ALCIBIADES}
\par   Yes.

\par \textbf{SOCRATES}
\par   The senseless are those who do not know this?

\par \textbf{ALCIBIADES}
\par   True.

\par \textbf{SOCRATES}
\par   The latter will say or do what they ought not without their own knowledge?

\par \textbf{ALCIBIADES}
\par   Exactly.

\par \textbf{SOCRATES}
\par   Oedipus, as I was saying, Alcibiades, was a person of this sort. And even now-a-days you will find many who (have offered inauspicious prayers), although, unlike him, they were not in anger nor thought that they were asking evil. He neither sought, nor supposed that he sought for good, but others have had quite the contrary notion. I believe that if the God whom you are about to consult should appear to you, and, in anticipation of your request, enquired whether you would be contented to become tyrant of Athens, and if this seemed in your eyes a small and mean thing, should add to it the dominion of all Hellas; and seeing that even then you would not be satisfied unless you were ruler of the whole of Europe, should promise, not only that, but, if you so desired, should proclaim to all mankind in one and the same day that Alcibiades, son of Cleinias, was tyrant: —in such a case, I imagine, you would depart full of joy, as one who had obtained the greatest of goods.

\par \textbf{ALCIBIADES}
\par   And not only I, Socrates, but any one else who should meet with such luck.

\par \textbf{SOCRATES}
\par   Yet you would not accept the dominion and lordship of all the Hellenes and all the barbarians in exchange for your life?

\par \textbf{ALCIBIADES}
\par   Certainly not:  for then what use could I make of them?

\par \textbf{SOCRATES}
\par   And would you accept them if you were likely to use them to a bad and mischievous end?

\par \textbf{ALCIBIADES}
\par   I would not.

\par \textbf{SOCRATES}
\par   You see that it is not safe for a man either rashly to accept whatever is offered him, or himself to request a thing, if he is likely to suffer thereby or immediately to lose his life. And yet we could tell of many who, having long desired and diligently laboured to obtain a tyranny, thinking that thus they would procure an advantage, have nevertheless fallen victims to designing enemies. You must have heard of what happened only the other day, how Archelaus of Macedonia was slain by his beloved (compare Aristotle, Pol. ), whose love for the tyranny was not less than that of Archelaus for him. The tyrannicide expected by his crime to become tyrant and afterwards to have a happy life; but when he had held the tyranny three or four days, he was in his turn conspired against and slain. Or look at certain of our own citizens,—and of their actions we have been not hearers, but eyewitnesses,—who have desired to obtain military command:  of those who have gained their object, some are even to this day exiles from the city, while others have lost their lives. And even they who seem to have fared best, have not only gone through many perils and terrors during their office, but after their return home they have been beset by informers worse than they once were by their foes, insomuch that several of them have wished that they had remained in a private station rather than have had the glories of command. If, indeed, such perils and terrors were of profit to the commonwealth, there would be reason in undergoing them; but the very contrary is the case. Again, you will find persons who have prayed for offspring, and when their prayers were heard, have fallen into the greatest pains and sufferings. For some have begotten children who were utterly bad, and have therefore passed all their days in misery, while the parents of good children have undergone the misfortune of losing them, and have been so little happier than the others that they would have preferred never to have had children rather than to have had them and lost them. And yet, although these and the like examples are manifest and known of all, it is rare to find any one who has refused what has been offered him, or, if he were likely to gain aught by prayer, has refrained from making his petition. The mass of mankind would not decline to accept a tyranny, or the command of an army, or any of the numerous things which cause more harm than good:  but rather, if they had them not, would have prayed to obtain them. And often in a short space of time they change their tone, and wish their old prayers unsaid. Wherefore also I suspect that men are entirely wrong when they blame the gods as the authors of the ills which befall them (compare Republic):  'their own presumption,' or folly (whichever is the right word)—

\par  'Has brought these unmeasured woes upon them.' (Homer. Odyss.)

\par  He must have been a wise poet, Alcibiades, who, seeing as I believe, his friends foolishly praying for and doing things which would not really profit them, offered up a common prayer in behalf of them all:—

\par  'King Zeus, grant us good whether prayed for or unsought by us; But that which we ask amiss, do thou avert.' (The author of these lines, which are probably of Pythagorean origin, is unknown. They are found also in the Anthology (Anth. Pal.).)

\par  In my opinion, I say, the poet spoke both well and prudently; but if you have anything to say in answer to him, speak out.

\par \textbf{ALCIBIADES}
\par   It is difficult, Socrates, to oppose what has been well said. And I perceive how many are the ills of which ignorance is the cause, since, as would appear, through ignorance we not only do, but what is worse, pray for the greatest evils. No man would imagine that he would do so; he would rather suppose that he was quite capable of praying for what was best:  to call down evils seems more like a curse than a prayer.

\par \textbf{SOCRATES}
\par   But perhaps, my good friend, some one who is wiser than either you or I will say that we have no right to blame ignorance thus rashly, unless we can add what ignorance we mean and of what, and also to whom and how it is respectively a good or an evil?

\par \textbf{ALCIBIADES}
\par   How do you mean? Can ignorance possibly be better than knowledge for any person in any conceivable case?

\par \textbf{SOCRATES}
\par   So I believe: —you do not think so?

\par \textbf{ALCIBIADES}
\par   Certainly not.

\par \textbf{SOCRATES}
\par   And yet surely I may not suppose that you would ever wish to act towards your mother as they say that Orestes and Alcmeon and others have done towards their parent.

\par \textbf{ALCIBIADES}
\par   Good words, Socrates, prithee.

\par \textbf{SOCRATES}
\par   You ought not to bid him use auspicious words, who says that you would not be willing to commit so horrible a deed, but rather him who affirms the contrary, if the act appear to you unfit even to be mentioned. Or do you think that Orestes, had he been in his senses and knew what was best for him to do, would ever have dared to venture on such a crime?

\par \textbf{ALCIBIADES}
\par   Certainly not.

\par \textbf{SOCRATES}
\par   Nor would any one else, I fancy?

\par \textbf{ALCIBIADES}
\par   No.

\par \textbf{SOCRATES}
\par   That ignorance is bad then, it would appear, which is of the best and does not know what is best?

\par \textbf{ALCIBIADES}
\par   So I think, at least.

\par \textbf{SOCRATES}
\par   And both to the person who is ignorant and everybody else?

\par \textbf{ALCIBIADES}
\par   Yes.

\par \textbf{SOCRATES}
\par   Let us take another case. Suppose that you were suddenly to get into your head that it would be a good thing to kill Pericles, your kinsman and guardian, and were to seize a sword and, going to the doors of his house, were to enquire if he were at home, meaning to slay only him and no one else: —the servants reply, 'Yes':  (Mind, I do not mean that you would really do such a thing; but there is nothing, you think, to prevent a man who is ignorant of the best, having occasionally the whim that what is worst is best?

\par \textbf{ALCIBIADES}
\par   No.)

\par \textbf{SOCRATES}
\par  —If, then, you went indoors, and seeing him, did not know him, but thought that he was some one else, would you venture to slay him?

\par \textbf{ALCIBIADES}
\par   Most decidedly not (it seems to me). (These words are omitted in several MSS.)

\par \textbf{SOCRATES}
\par   For you designed to kill, not the first who offered, but Pericles himself?

\par \textbf{ALCIBIADES}
\par   Certainly.

\par \textbf{SOCRATES}
\par   And if you made many attempts, and each time failed to recognize Pericles, you would never attack him?

\par \textbf{ALCIBIADES}
\par   Never.

\par \textbf{SOCRATES}
\par   Well, but if Orestes in like manner had not known his mother, do you think that he would ever have laid hands upon her?

\par \textbf{ALCIBIADES}
\par   No.

\par \textbf{SOCRATES}
\par   He did not intend to slay the first woman he came across, nor any one else's mother, but only his own?

\par \textbf{ALCIBIADES}
\par   True.

\par \textbf{SOCRATES}
\par   Ignorance, then, is better for those who are in such a frame of mind, and have such ideas?

\par \textbf{ALCIBIADES}
\par   Obviously.

\par \textbf{SOCRATES}
\par   You acknowledge that for some persons in certain cases the ignorance of some things is a good and not an evil, as you formerly supposed?

\par \textbf{ALCIBIADES}
\par   I do.

\par \textbf{SOCRATES}
\par   And there is still another case which will also perhaps appear strange to you, if you will consider it? (The reading is here uncertain.)

\par \textbf{ALCIBIADES}
\par   What is that, Socrates?

\par \textbf{SOCRATES}
\par   It may be, in short, that the possession of all the sciences, if unaccompanied by the knowledge of the best, will more often than not injure the possessor. Consider the matter thus: —Must we not, when we intend either to do or say anything, suppose that we know or ought to know that which we propose so confidently to do or say?

\par \textbf{ALCIBIADES}
\par   Yes, in my opinion.

\par \textbf{SOCRATES}
\par   We may take the orators for an example, who from time to time advise us about war and peace, or the building of walls and the construction of harbours, whether they understand the business in hand, or only think that they do. Whatever the city, in a word, does to another city, or in the management of her own affairs, all happens by the counsel of the orators.

\par \textbf{ALCIBIADES}
\par   True.

\par \textbf{SOCRATES}
\par   But now see what follows, if I can (make it clear to you). (Some words appear to have dropped out here.) You would distinguish the wise from the foolish?

\par \textbf{ALCIBIADES}
\par   Yes.

\par \textbf{SOCRATES}
\par   The many are foolish, the few wise?

\par \textbf{ALCIBIADES}
\par   Certainly.

\par \textbf{SOCRATES}
\par   And you use both the terms, 'wise' and 'foolish,' in reference to something?

\par \textbf{ALCIBIADES}
\par   I do.

\par \textbf{SOCRATES}
\par   Would you call a person wise who can give advice, but does not know whether or when it is better to carry out the advice?

\par \textbf{ALCIBIADES}
\par   Decidedly not.

\par \textbf{SOCRATES}
\par   Nor again, I suppose, a person who knows the art of war, but does not know whether it is better to go to war or for how long?

\par \textbf{ALCIBIADES}
\par   No.

\par \textbf{SOCRATES}
\par   Nor, once more, a person who knows how to kill another or to take away his property or to drive him from his native land, but not when it is better to do so or for whom it is better?

\par \textbf{ALCIBIADES}
\par   Certainly not.

\par \textbf{SOCRATES}
\par   But he who understands anything of the kind and has at the same time the knowledge of the best course of action: —and the best and the useful are surely the same?—

\par \textbf{ALCIBIADES}
\par   Yes.

\par \textbf{SOCRATES}
\par  —Such an one, I say, we should call wise and a useful adviser both of himself and of the city. What do you think?

\par \textbf{ALCIBIADES}
\par   I agree.

\par \textbf{SOCRATES}
\par   And if any one knows how to ride or to shoot with the bow or to box or to wrestle, or to engage in any other sort of contest or to do anything whatever which is in the nature of an art,—what do you call him who knows what is best according to that art? Do you not speak of one who knows what is best in riding as a good rider?

\par \textbf{ALCIBIADES}
\par   Yes.

\par \textbf{SOCRATES}
\par   And in a similar way you speak of a good boxer or a good flute-player or a good performer in any other art?

\par \textbf{ALCIBIADES}
\par   True.

\par \textbf{SOCRATES}
\par   But is it necessary that the man who is clever in any of these arts should be wise also in general? Or is there a difference between the clever artist and the wise man?

\par \textbf{ALCIBIADES}
\par   All the difference in the world.

\par \textbf{SOCRATES}
\par   And what sort of a state do you think that would be which was composed of good archers and flute-players and athletes and masters in other arts, and besides them of those others about whom we spoke, who knew how to go to war and how to kill, as well as of orators puffed up with political pride, but in which not one of them all had this knowledge of the best, and there was no one who could tell when it was better to apply any of these arts or in regard to whom?

\par \textbf{ALCIBIADES}
\par   I should call such a state bad, Socrates.

\par \textbf{SOCRATES}
\par   You certainly would when you saw each of them rivalling the other and esteeming that of the greatest importance in the state,

\par  'Wherein he himself most excelled.' (Euripides, Antiope.) —I mean that which was best in any art, while he was entirely ignorant of what was best for himself and for the state, because, as I think, he trusts to opinion which is devoid of intelligence. In such a case should we not be right if we said that the state would be full of anarchy and lawlessness?

\par \textbf{ALCIBIADES}
\par   Decidedly.

\par \textbf{SOCRATES}
\par   But ought we not then, think you, either to fancy that we know or really to know, what we confidently propose to do or say?

\par \textbf{ALCIBIADES}
\par   Yes.

\par \textbf{SOCRATES}
\par   And if a person does that which he knows or supposes that he knows, and the result is beneficial, he will act advantageously both for himself and for the state?

\par \textbf{ALCIBIADES}
\par   True.

\par \textbf{SOCRATES}
\par   And if he do the contrary, both he and the state will suffer?

\par \textbf{ALCIBIADES}
\par   Yes.

\par \textbf{SOCRATES}
\par   Well, and are you of the same mind, as before?

\par \textbf{ALCIBIADES}
\par   I am.

\par \textbf{SOCRATES}
\par   But were you not saying that you would call the many unwise and the few wise?

\par \textbf{ALCIBIADES}
\par   I was.

\par \textbf{SOCRATES}
\par   And have we not come back to our old assertion that the many fail to obtain the best because they trust to opinion which is devoid of intelligence?

\par \textbf{ALCIBIADES}
\par   That is the case.

\par \textbf{SOCRATES}
\par   It is good, then, for the many, if they particularly desire to do that which they know or suppose that they know, neither to know nor to suppose that they know, in cases where if they carry out their ideas in action they will be losers rather than gainers?

\par \textbf{ALCIBIADES}
\par   What you say is very true.

\par \textbf{SOCRATES}
\par   Do you not see that I was really speaking the truth when I affirmed that the possession of any other kind of knowledge was more likely to injure than to benefit the possessor, unless he had also the knowledge of the best?

\par \textbf{ALCIBIADES}
\par   I do now, if I did not before, Socrates.

\par \textbf{SOCRATES}
\par   The state or the soul, therefore, which wishes to have a right existence must hold firmly to this knowledge, just as the sick man clings to the physician, or the passenger depends for safety on the pilot. And if the soul does not set sail until she have obtained this she will be all the safer in the voyage through life. But when she rushes in pursuit of wealth or bodily strength or anything else, not having the knowledge of the best, so much the more is she likely to meet with misfortune. And he who has the love of learning (Or, reading polumatheian, 'abundant learning. '), and is skilful in many arts, and does not possess the knowledge of the best, but is under some other guidance, will make, as he deserves, a sorry voyage: —he will, I believe, hurry through the brief space of human life, pilotless in mid-ocean, and the words will apply to him in which the poet blamed his enemy: —

\par  '...Full many a thing he knew; But knew them all badly.' (A fragment from the pseudo-Homeric poem, 'Margites.')

\par \textbf{ALCIBIADES}
\par   How in the world, Socrates, do the words of the poet apply to him? They seem to me to have no bearing on the point whatever.

\par \textbf{SOCRATES}
\par   Quite the contrary, my sweet friend:  only the poet is talking in riddles after the fashion of his tribe. For all poetry has by nature an enigmatical character, and it is by no means everybody who can interpret it. And if, moreover, the spirit of poetry happen to seize on a man who is of a begrudging temper and does not care to manifest his wisdom but keeps it to himself as far as he can, it does indeed require an almost superhuman wisdom to discover what the poet would be at. You surely do not suppose that Homer, the wisest and most divine of poets, was unaware of the impossibility of knowing a thing badly:  for it was no less a person than he who said of Margites that 'he knew many things, but knew them all badly.' The solution of the riddle is this, I imagine: —By 'badly' Homer meant 'bad' and 'knew' stands for 'to know.' Put the words together;—the metre will suffer, but the poet's meaning is clear;—'Margites knew all these things, but it was bad for him to know them.' And, obviously, if it was bad for him to know so many things, he must have been a good-for-nothing, unless the argument has played us false.

\par \textbf{ALCIBIADES}
\par   But I do not think that it has, Socrates:  at least, if the argument is fallacious, it would be difficult for me to find another which I could trust.

\par \textbf{SOCRATES}
\par   And you are right in thinking so.

\par \textbf{ALCIBIADES}
\par   Well, that is my opinion.

\par \textbf{SOCRATES}
\par   But tell me, by Heaven: —you must see now the nature and greatness of the difficulty in which you, like others, have your part. For you change about in all directions, and never come to rest anywhere:  what you once most strongly inclined to suppose, you put aside again and quite alter your mind. If the God to whose shrine you are going should appear at this moment, and ask before you made your prayer, 'Whether you would desire to have one of the things which we mentioned at first, or whether he should leave you to make your own request: '—what in either case, think you, would be the best way to take advantage of the opportunity?

\par \textbf{ALCIBIADES}
\par   Indeed, Socrates, I could not answer you without consideration. It seems to me to be a wild thing (The Homeric word margos is said to be here employed in allusion to the quotation from the 'Margites' which Socrates has just made; but it is not used in the sense which it has in Homer.) to make such a request; a man must be very careful lest he pray for evil under the idea that he is asking for good, when shortly after he may have to recall his prayer, and, as you were saying, demand the opposite of what he at first requested.

\par \textbf{SOCRATES}
\par   And was not the poet whose words I originally quoted wiser than we are, when he bade us (pray God) to defend us from evil even though we asked for it?

\par \textbf{ALCIBIADES}
\par   I believe that you are right.

\par \textbf{SOCRATES}
\par   The Lacedaemonians, too, whether from admiration of the poet or because they have discovered the idea for themselves, are wont to offer the prayer alike in public and private, that the Gods will give unto them the beautiful as well as the good: —no one is likely to hear them make any further petition. And yet up to the present time they have not been less fortunate than other men; or if they have sometimes met with misfortune, the fault has not been due to their prayer. For surely, as I conceive, the Gods have power either to grant our requests, or to send us the contrary of what we ask.

\par  And now I will relate to you a story which I have heard from certain of our elders. It chanced that when the Athenians and Lacedaemonians were at war, our city lost every battle by land and sea and never gained a victory. The Athenians being annoyed and perplexed how to find a remedy for their troubles, decided to send and enquire at the shrine of Ammon. Their envoys were also to ask, 'Why the Gods always granted the victory to the Lacedaemonians?' 'We,' (they were to say,) 'offer them more and finer sacrifices than any other Hellenic state, and adorn their temples with gifts, as nobody else does; moreover, we make the most solemn and costly processions to them every year, and spend more money in their service than all the rest of the Hellenes put together. But the Lacedaemonians take no thought of such matters, and pay so little respect to the Gods that they have a habit of sacrificing blemished animals to them, and in various ways are less zealous than we are, although their wealth is quite equal to ours.' When they had thus spoken, and had made their request to know what remedy they could find against the evils which troubled them, the prophet made no direct answer,—clearly because he was not allowed by the God to do so;—but he summoned them to him and said: 'Thus saith Ammon to the Athenians: "The silent worship of the Lacedaemonians pleaseth me better than all the offerings of the other Hellenes."' Such were the words of the God, and nothing more. He seems to have meant by 'silent worship' the prayer of the Lacedaemonians, which is indeed widely different from the usual requests of the Hellenes. For they either bring to the altar bulls with gilded horns or make offerings to the Gods, and beg at random for what they need, good or bad. When, therefore, the Gods hear them using words of ill omen they reject these costly processions and sacrifices of theirs. And we ought, I think, to be very careful and consider well what we should say and what leave unsaid. Homer, too, will furnish us with similar stories. For he tells us how the Trojans in making their encampment,

\par  'Offered up whole hecatombs to the immortals,'

\par  and how the 'sweet savour' was borne 'to the heavens by the winds;
 
\par  So that it was in vain for them to sacrifice and offer gifts, seeing that they were hateful to the Gods, who are not, like vile usurers, to be gained over by bribes. And it is foolish for us to boast that we are superior to the Lacedaemonians by reason of our much worship. The idea is inconceivable that the Gods have regard, not to the justice and purity of our souls, but to costly processions and sacrifices, which men may celebrate year after year, although they have committed innumerable crimes against the Gods or against their fellow-men or the state. For the Gods, as Ammon and his prophet declare, are no receivers of gifts, and they scorn such unworthy service. Wherefore also it would seem that wisdom and justice are especially honoured both by the Gods and by men of sense; and they are the wisest and most just who know how to speak and act towards Gods and men. But I should like to hear what your opinion is about these matters.

\par \textbf{ALCIBIADES}
\par   I agree, Socrates, with you and with the God, whom, indeed, it would be unbecoming for me to oppose.

\par \textbf{SOCRATES}
\par   Do you not remember saying that you were in great perplexity, lest perchance you should ask for evil, supposing that you were asking for good?

\par \textbf{ALCIBIADES}
\par   I do.

\par \textbf{SOCRATES}
\par   You see, then, that there is a risk in your approaching the God in prayer, lest haply he should refuse your sacrifice when he hears the blasphemy which you utter, and make you partake of other evils as well. The wisest plan, therefore, seems to me that you should keep silence; for your 'highmindedness'—to use the mildest term which men apply to folly—will most likely prevent you from using the prayer of the Lacedaemonians. You had better wait until we find out how we should behave towards the Gods and towards men.

\par \textbf{ALCIBIADES}
\par   And how long must I wait, Socrates, and who will be my teacher? I should be very glad to see the man.

\par \textbf{SOCRATES}
\par   It is he who takes an especial interest in you. But first of all, I think, the darkness must be taken away in which your soul is now enveloped, just as Athene in Homer removes the mist from the eyes of Diomede that

\par  'He may distinguish between God and mortal man.'

\par  Afterwards the means may be given to you whereby you may distinguish between good and evil. At present, I fear, this is beyond your power.

\par \textbf{ALCIBIADES}
\par   Only let my instructor take away the impediment, whether it pleases him to call it mist or anything else! I care not who he is; but I am resolved to disobey none of his commands, if I am likely to be the better for them.

\par \textbf{SOCRATES}
\par   And surely he has a wondrous care for you.

\par \textbf{ALCIBIADES}
\par   It seems to be altogether advisable to put off the sacrifice until he is found.

\par \textbf{SOCRATES}
\par   You are right:  that will be safer than running such a tremendous risk.

\par \textbf{ALCIBIADES}
\par   But how shall we manage, Socrates?—At any rate I will set this crown of mine upon your head, as you have given me such excellent advice, and to the Gods we will offer crowns and perform the other customary rites when I see that day approaching:  nor will it be long hence, if they so will.

\par \textbf{SOCRATES}
\par   I accept your gift, and shall be ready and willing to receive whatever else you may proffer. Euripides makes Creon say in the play, when he beholds Teiresias with his crown and hears that he has gained it by his skill as the first-fruits of the spoil: —

\par  'An auspicious omen I deem thy victor's wreath: For well thou knowest that wave and storm oppress us.'

\par  And so I count your gift to be a token of good-fortune; for I am in no less stress than Creon, and would fain carry off the victory over your lovers.

\par 
 
\end{document}