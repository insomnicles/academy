
\documentclass[11pt,letter]{article}


\begin{document}

\title{Eryxias\thanks{Source: https://www.gutenberg.org/files/1681/1681-h/1681-h.htm. License: http://gutenberg.org/license ds}}
\date{\today}
\author{Plato (spurious and doubtful works), 427? BCE-347? BCE\\ Translated by Jowett, Benjamin, 1817-1893}
\maketitle

\setcounter{tocdepth}{1}
\tableofcontents
\renewcommand{\baselinestretch}{1.0}
\normalsize
\newpage

\section{
      APPENDIX II.
    }
\par  The two dialogues which are translated in the second appendix are not mentioned by Aristotle, or by any early authority, and have no claim to be ascribed to Plato. They are examples of Platonic dialogues to be assigned probably to the second or third generation after Plato, when his writings were well known at Athens and Alexandria. They exhibit considerable originality, and are remarkable for containing several thoughts of the sort which we suppose to be modern rather than ancient, and which therefore have a peculiar interest for us. The Second Alcibiades shows that the difficulties about prayer which have perplexed Christian theologians were not unknown among the followers of Plato. The Eryxias was doubted by the ancients themselves: yet it may claim the distinction of being, among all Greek or Roman writings, the one which anticipates in the most striking manner the modern science of political economy and gives an abstract form to some of its principal doctrines.

\par  For the translation of these two dialogues I am indebted to my friend and secretary, Mr. Knight.

\par  That the Dialogue which goes by the name of the Second Alcibiades is a genuine writing of Plato will not be maintained by any modern critic, and was hardly believed by the ancients themselves. The dialectic is poor and weak. There is no power over language, or beauty of style; and there is a certain abruptness and agroikia in the conversation, which is very un-Platonic. The best passage is probably that about the poets:—the remark that the poet, who is of a reserved disposition, is uncommonly difficult to understand, and the ridiculous interpretation of Homer, are entirely in the spirit of Plato (compare Protag; Ion; Apol.). The characters are ill-drawn. Socrates assumes the 'superior person' and preaches too much, while Alcibiades is stupid and heavy-in-hand. There are traces of Stoic influence in the general tone and phraseology of the Dialogue (compare opos melesei tis...kaka: oti pas aphron mainetai): and the writer seems to have been acquainted with the 'Laws' of Plato (compare Laws). An incident from the Symposium is rather clumsily introduced, and two somewhat hackneyed quotations (Symp., Gorg.) recur. The reference to the death of Archelaus as having occurred 'quite lately' is only a fiction, probably suggested by the Gorgias, where the story of Archelaus is told, and a similar phrase occurs;—ta gar echthes kai proen gegonota tauta, k.t.l. There are several passages which are either corrupt or extremely ill-expressed. But there is a modern interest in the subject of the dialogue; and it is a good example of a short spurious work, which may be attributed to the second or third century before Christ.

\par 
\section{
      INTRODUCTION.
    }
\par  Much cannot be said in praise of the style or conception of the Eryxias. It is frequently obscure; like the exercise of a student, it is full of small imitations of Plato:—Phaeax returning from an expedition to Sicily (compare Socrates in the Charmides from the army at Potidaea), the figure of the game at draughts, borrowed from the Republic, etc. It has also in many passages the ring of sophistry. On the other hand, the rather unhandsome treatment which is exhibited towards Prodicus is quite unlike the urbanity of Plato.

\par  Yet there are some points in the argument which are deserving of attention. (1) That wealth depends upon the need of it or demand for it, is the first anticipation in an abstract form of one of the great principles of modern political economy, and the nearest approach to it to be found in an ancient writer. (2) The resolution of wealth into its simplest implements going on to infinity is a subtle and refined thought. (3) That wealth is relative to circumstances is a sound conception. (4) That the arts and sciences which receive payment are likewise to be comprehended under the notion of wealth, also touches a question of modern political economy. (5) The distinction of post hoc and propter hoc, often lost sight of in modern as well as in ancient times. These metaphysical conceptions and distinctions show considerable power of thought in the writer, whatever we may think of his merits as an imitator of Plato.

\par 
\section{
      ERYXIAS
    }  
\par  It happened by chance that Eryxias the Steirian was walking with me in the Portico of Zeus the Deliverer, when there came up to us Critias and Erasistratus, the latter the son of Phaeax, who was the nephew of Erasistratus. Now Erasistratus had just arrived from Sicily and that part of the world. As they approached, he said, Hail, Socrates!

\par \textbf{SOCRATES}
\par   The same to you, I said; have you any good news from Sicily to tell us?

\par \textbf{ERASISTRATUS}
\par   Most excellent. But, if you please, let us first sit down; for I am tired with my yesterday's journey from Megara.

\par \textbf{SOCRATES}
\par   Gladly, if that is your desire.

\par \textbf{ERASISTRATUS}
\par   What would you wish to hear first? he said. What the Sicilians are doing, or how they are disposed towards our city? To my mind, they are very like wasps:  so long as you only cause them a little annoyance they are quite unmanageable; you must destroy their nests if you wish to get the better of them. And in a similar way, the Syracusans, unless we set to work in earnest, and go against them with a great expedition, will never submit to our rule. The petty injuries which we at present inflict merely irritate them enough to make them utterly intractable. And now they have sent ambassadors to Athens, and intend, I suspect, to play us some trick.—While we were talking, the Syracusan envoys chanced to go by, and Erasistratus, pointing to one of them, said to me, That, Socrates, is the richest man in all Italy and Sicily. For who has larger estates or more land at his disposal to cultivate if he please? And they are of a quality, too, finer than any other land in Hellas. Moreover, he has all the things which go to make up wealth, slaves and horses innumerable, gold and silver without end.

\par  I saw that he was inclined to expatiate on the riches of the man; so I asked him, Well, Erasistratus, and what sort of character does he bear in Sicily?

\par \textbf{ERASISTRATUS}
\par   He is esteemed to be, and really is, the wickedest of all the Sicilians and Italians, and even more wicked than he is rich; indeed, if you were to ask any Sicilian whom he thought to be the worst and the richest of mankind, you would never hear any one else named.

\par  I reflected that we were speaking, not of trivial matters, but about wealth and virtue, which are deemed to be of the greatest moment, and I asked Erasistratus whom he considered the wealthier,—he who was the possessor of a talent of silver or he who had a field worth two talents?

\par \textbf{ERASISTRATUS}
\par   The owner of the field.

\par \textbf{SOCRATES}
\par   And on the same principle he who had robes and bedding and such things which are of greater value to him than to a stranger would be richer than the stranger?

\par \textbf{ERASISTRATUS}
\par   True.

\par \textbf{SOCRATES}
\par   And if any one gave you a choice, which of these would you prefer?

\par \textbf{ERASISTRATUS}
\par   That which was most valuable.

\par \textbf{SOCRATES}
\par   In which way do you think you would be the richer?

\par \textbf{ERASISTRATUS}
\par   By choosing as I said.

\par \textbf{SOCRATES}
\par   And he appears to you to be the richest who has goods of the greatest value?

\par \textbf{ERASISTRATUS}
\par   He does.

\par \textbf{SOCRATES}
\par   And are not the healthy richer than the sick, since health is a possession more valuable than riches to the sick? Surely there is no one who would not prefer to be poor and well, rather than to have all the King of Persia's wealth and to be ill. And this proves that men set health above wealth, else they would never choose the one in preference to the other.

\par \textbf{ERASISTRATUS}
\par   True.

\par \textbf{SOCRATES}
\par   And if anything appeared to be more valuable than health, he would be the richest who possessed it?

\par \textbf{ERASISTRATUS}
\par   He would.

\par \textbf{SOCRATES}
\par   Suppose that some one came to us at this moment and were to ask, Well, Socrates and Eryxias and Erasistratus, can you tell me what is of the greatest value to men? Is it not that of which the possession will best enable a man to advise how his own and his friend's affairs should be administered?—What will be our reply?

\par \textbf{ERASISTRATUS}
\par   I should say, Socrates, that happiness was the most precious of human possessions.

\par \textbf{SOCRATES}
\par   Not a bad answer. But do we not deem those men who are most prosperous to be the happiest?

\par \textbf{ERASISTRATUS}
\par   That is my opinion.

\par \textbf{SOCRATES}
\par   And are they not most prosperous who commit the fewest errors in respect either of themselves or of other men?

\par \textbf{ERASISTRATUS}
\par   Certainly.

\par \textbf{SOCRATES}
\par   And they who know what is evil and what is good; what should be done and what should be left undone;—these behave the most wisely and make the fewest mistakes?

\par  Erasistratus agreed to this.

\par \textbf{SOCRATES}
\par   Then the wisest and those who do best and the most fortunate and the richest would appear to be all one and the same, if wisdom is really the most valuable of our possessions?

\par  Yes, said Eryxias, interposing, but what use would it be if a man had the wisdom of Nestor and wanted the necessaries of life, food and drink and clothes and the like? Where would be the advantage of wisdom then? Or how could he be the richest of men who might even have to go begging, because he had not wherewithal to live?

\par  I thought that what Eryxias was saying had some weight, and I replied, Would the wise man really suffer in this way, if he were so ill-provided; whereas if he had the house of Polytion, and the house were full of gold and silver, he would lack nothing?

\par \textbf{ERYXIAS}
\par   Yes; for then he might dispose of his property and obtain in exchange what he needed, or he might sell it for money with which he could supply his wants and in a moment procure abundance of everything.

\par \textbf{SOCRATES}
\par   True, if he could find some one who preferred such a house to the wisdom of Nestor. But if there are persons who set great store by wisdom like Nestor's and the advantages accruing from it, to sell these, if he were so disposed, would be easier still. Or is a house a most useful and necessary possession, and does it make a great difference in the comfort of life to have a mansion like Polytion's instead of living in a shabby little cottage, whereas wisdom is of small use and it is of no importance whether a man is wise or ignorant about the highest matters? Or is wisdom despised of men and can find no buyers, although cypress wood and marble of Pentelicus are eagerly bought by numerous purchasers? Surely the prudent pilot or the skilful physician, or the artist of any kind who is proficient in his art, is more worth than the things which are especially reckoned among riches; and he who can advise well and prudently for himself and others is able also to sell the product of his art, if he so desire.

\par  Eryxias looked askance, as if he had received some unfair treatment, and said, I believe, Socrates, that if you were forced to speak the truth, you would declare that you were richer than Callias the son of Hipponicus. And yet, although you claimed to be wiser about things of real importance, you would not any the more be richer than he.

\par  I dare say, Eryxias, I said, that you may regard these arguments of ours as a kind of game; you think that they have no relation to facts, but are like the pieces in the game of draughts which the player can move in such a way that his opponents are unable to make any countermove. (Compare Republic.) And perhaps, too, as regards riches you are of opinion that while facts remain the same, there are arguments, no matter whether true or false, which enable the user of them to prove that the wisest and the richest are one and the same, although he is in the wrong and his opponents are in the right. There would be nothing strange in this; it would be as if two persons were to dispute about letters, one declaring that the word Socrates began with an S, the other that it began with an A, and the latter could gain the victory over the former.

\par  Eryxias glanced at the audience, laughing and blushing at once, as if he had had nothing to do with what had just been said, and replied,—No, indeed, Socrates, I never supposed that our arguments should be of a kind which would never convince any one of those here present or be of advantage to them. For what man of sense could ever be persuaded that the wisest and the richest are the same? The truth is that we are discussing the subject of riches, and my notion is that we should argue respecting the honest and dishonest means of acquiring them, and, generally, whether they are a good thing or a bad.

\par  Very good, I said, and I am obliged to you for the hint: in future we will be more careful. But why do not you yourself, as you introduced the argument, and do not think that the former discussion touched the point at issue, tell us whether you consider riches to be a good or an evil?

\par  I am of opinion, he said, that they are a good. He was about to add something more, when Critias interrupted him:—Do you really suppose so, Eryxias?

\par  Certainly, replied Eryxias; I should be mad if I did not: and I do not fancy that you would find any one else of a contrary opinion.

\par  And I, retorted Critias, should say that there is no one whom I could not compel to admit that riches are bad for some men. But surely, if they were a good, they could not appear bad for any one?

\par  Here I interposed and said to them: If you two were having an argument about equitation and what was the best way of riding, supposing that I knew the art myself, I should try to bring you to an agreement. For I should be ashamed if I were present and did not do what I could to prevent your difference. And I should do the same if you were quarrelling about any other art and were likely, unless you agreed on the point in dispute, to part as enemies instead of as friends. But now, when we are contending about a thing of which the usefulness continues during the whole of life, and it makes an enormous difference whether we are to regard it as beneficial or not,—a thing, too, which is esteemed of the highest importance by the Hellenes:—(for parents, as soon as their children are, as they think, come to years of discretion, urge them to consider how wealth may be acquired, since by riches the value of a man is judged):—When, I say, we are thus in earnest, and you, who agree in other respects, fall to disputing about a matter of such moment, that is, about wealth, and not merely whether it is black or white, light or heavy, but whether it is a good or an evil, whereby, although you are now the dearest of friends and kinsmen, the most bitter hatred may arise betwixt you, I must hinder your dissension to the best of my power. If I could, I would tell you the truth, and so put an end to the dispute; but as I cannot do this, and each of you supposes that you can bring the other to an agreement, I am prepared, as far as my capacity admits, to help you in solving the question. Please, therefore, Critias, try to make us accept the doctrines which you yourself entertain.

\par \textbf{CRITIAS}
\par   I should like to follow up the argument, and will ask Eryxias whether he thinks that there are just and unjust men?

\par \textbf{ERYXIAS}
\par   Most decidedly.

\par \textbf{CRITIAS}
\par   And does injustice seem to you an evil or a good?

\par \textbf{ERYXIAS}
\par   An evil.

\par \textbf{CRITIAS}
\par   Do you consider that he who bribes his neighbour's wife and commits adultery with her, acts justly or unjustly, and this although both the state and the laws forbid?

\par \textbf{ERYXIAS}
\par   Unjustly.

\par \textbf{CRITIAS}
\par   And if the wicked man has wealth and is willing to spend it, he will carry out his evil purposes? whereas he who is short of means cannot do what he fain would, and therefore does not sin? In such a case, surely, it is better that a person should not be wealthy, if his poverty prevents the accomplishment of his desires, and his desires are evil? Or, again, should you call sickness a good or an evil?

\par \textbf{ERYXIAS}
\par   An evil.

\par \textbf{CRITIAS}
\par   Well, and do you think that some men are intemperate?

\par \textbf{ERYXIAS}
\par   Yes.

\par \textbf{CRITIAS}
\par   Then, if it is better for his health that the intemperate man should refrain from meat and drink and other pleasant things, but he cannot owing to his intemperance, will it not also be better that he should be too poor to gratify his lust rather than that he should have a superabundance of means? For thus he will not be able to sin, although he desire never so much.

\par  Critias appeared to be arguing so admirably that Eryxias, if he had not been ashamed of the bystanders, would probably have got up and struck him. For he thought that he had been robbed of a great possession when it became obvious to him that he had been wrong in his former opinion about wealth. I observed his vexation, and feared that they would proceed to abuse and quarrelling: so I said,—I heard that very argument used in the Lyceum yesterday by a wise man, Prodicus of Ceos; but the audience thought that he was talking mere nonsense, and no one could be persuaded that he was speaking the truth. And when at last a certain talkative young gentleman came in, and, taking his seat, began to laugh and jeer at Prodicus, tormenting him and demanding an explanation of his argument, he gained the ear of the audience far more than Prodicus.

\par  Can you repeat the discourse to us? Said Erasistratus.

\par \textbf{SOCRATES}
\par   If I can only remember it, I will. The youth began by asking Prodicus, In what way did he think that riches were a good and in what an evil? Prodicus answered, as you did just now, that they were a good to good men and to those who knew in what way they should be employed, while to the bad and the ignorant they were an evil. The same is true, he went on to say, of all other things; men make them to be what they are themselves. The saying of Archilochus is true: —

\par  'Men's thoughts correspond to the things which they meet with.'

\par  Well, then, replied the youth, if any one makes me wise in that wisdom whereby good men become wise, he must also make everything else good to me. Not that he concerns himself at all with these other things, but he has converted my ignorance into wisdom. If, for example, a person teach me grammar or music, he will at the same time teach me all that relates to grammar or music, and so when he makes me good, he makes things good to me.

\par  Prodicus did not altogether agree: still he consented to what was said.

\par  And do you think, said the youth, that doing good things is like building a house,—the work of human agency; or do things remain what they were at first, good or bad, for all time?

\par  Prodicus began to suspect, I fancy, the direction which the argument was likely to take, and did not wish to be put down by a mere stripling before all those present:—(if they two had been alone, he would not have minded):—so he answered, cleverly enough: I think that doing good things is a work of human agency.

\par  And is virtue in your opinion, Prodicus, innate or acquired by instruction?

\par  The latter, said Prodicus.

\par  Then you would consider him a simpleton who supposed that he could obtain by praying to the Gods the knowledge of grammar or music or any other art, which he must either learn from another or find out for himself?

\par  Prodicus agreed to this also.

\par  And when you pray to the Gods that you may do well and receive good, you mean by your prayer nothing else than that you desire to become good and wise:—if, at least, things are good to the good and wise and evil to the evil. But in that case, if virtue is acquired by instruction, it would appear that you only pray to be taught what you do not know.

\par  Hereupon I said to Prodicus that it was no misfortune to him if he had been proved to be in error in supposing that the Gods immediately granted to us whatever we asked:—if, I added, whenever you go up to the Acropolis you earnestly entreat the Gods to grant you good things, although you know not whether they can yield your request, it is as though you went to the doors of the grammarian and begged him, although you had never made a study of the art, to give you a knowledge of grammar which would enable you forthwith to do the business of a grammarian.

\par  While I was speaking, Prodicus was preparing to retaliate upon his youthful assailant, intending to employ the argument of which you have just made use; for he was annoyed to have it supposed that he offered a vain prayer to the Gods. But the master of the gymnasium came to him and begged him to leave because he was teaching the youths doctrines which were unsuited to them, and therefore bad for them.

\par  I have told you this because I want you to understand how men are circumstanced in regard to philosophy. Had Prodicus been present and said what you have said, the audience would have thought him raving, and he would have been ejected from the gymnasium. But you have argued so excellently well that you have not only persuaded your hearers, but have brought your opponent to an agreement. For just as in the law courts, if two witnesses testify to the same fact, one of whom seems to be an honest fellow and the other a rogue, the testimony of the rogue often has the contrary effect on the judges' minds to what he intended, while the same evidence if given by the honest man at once strikes them as perfectly true. And probably the audience have something of the same feeling about yourself and Prodicus; they think him a Sophist and a braggart, and regard you as a gentleman of courtesy and worth. For they do not pay attention to the argument so much as to the character of the speaker.

\par  But truly, Socrates, said Erasistratus, though you may be joking, Critias does seem to me to be saying something which is of weight.

\par \textbf{SOCRATES}
\par   I am in profound earnest, I assure you. But why, as you have begun your argument so prettily, do you not go on with the rest? There is still something lacking, now you have agreed that (wealth) is a good to some and an evil to others. It remains to enquire what constitutes wealth; for unless you know this, you cannot possibly come to an understanding as to whether it is a good or an evil. I am ready to assist you in the enquiry to the utmost of my power:  but first let him who affirms that riches are a good, tell us what, in his opinion, is wealth.

\par \textbf{ERASISTRATUS}
\par   Indeed, Socrates, I have no notion about wealth beyond that which men commonly have. I suppose that wealth is a quantity of money (compare Arist. Pol. ); and this, I imagine, would also be Critias' definition.

\par \textbf{SOCRATES}
\par   Then now we have to consider, What is money? Or else later on we shall be found to differ about the question. For instance, the Carthaginians use money of this sort. Something which is about the size of a stater is tied up in a small piece of leather:  what it is, no one knows but the makers. A seal is next set upon the leather, which then passes into circulation, and he who has the largest number of such pieces is esteemed the richest and best off. And yet if any one among us had a mass of such coins he would be no wealthier than if he had so many pebbles from the mountain. At Lacedaemon, again, they use iron by weight which has been rendered useless:  and he who has the greatest mass of such iron is thought to be the richest, although elsewhere it has no value. In Ethiopia engraved stones are employed, of which a Lacedaemonian could make no use. Once more, among the Nomad Scythians a man who owned the house of Polytion would not be thought richer than one who possessed Mount Lycabettus among ourselves. And clearly those things cannot all be regarded as possessions; for in some cases the possessors would appear none the richer thereby:  but, as I was saying, some one of them is thought in one place to be money, and the possessors of it are the wealthy, whereas in some other place it is not money, and the ownership of it does not confer wealth; just as the standard of morals varies, and what is honourable to some men is dishonourable to others. And if we wish to enquire why a house is valuable to us but not to the Scythians, or why the Carthaginians value leather which is worthless to us, or the Lacedaemonians find wealth in iron and we do not, can we not get an answer in some such way as this:  Would an Athenian, who had a thousand talents weight of the stones which lie about in the Agora and which we do not employ for any purpose, be thought to be any the richer?

\par \textbf{ERASISTRATUS}
\par   He certainly would not appear so to me.

\par \textbf{SOCRATES}
\par   But if he possessed a thousand talents weight of some precious stone, we should say that he was very rich?

\par \textbf{ERASISTRATUS}
\par   Of course.

\par \textbf{SOCRATES}
\par   The reason is that the one is useless and the other useful?

\par \textbf{ERASISTRATUS}
\par   Yes.

\par \textbf{SOCRATES}
\par   And in the same way among the Scythians a house has no value because they have no use for a house, nor would a Scythian set so much store on the finest house in the world as on a leather coat, because he could use the one and not the other. Or again, the Carthaginian coinage is not wealth in our eyes, for we could not employ it, as we can silver, to procure what we need, and therefore it is of no use to us.

\par \textbf{ERASISTRATUS}
\par   True.

\par \textbf{SOCRATES}
\par   What is useful to us, then, is wealth, and what is useless to us is not wealth?

\par  But how do you mean, Socrates? said Eryxias, interrupting. Do we not employ in our intercourse with one another speech and violence (?) and various other things? These are useful and yet they are not wealth.

\par \textbf{SOCRATES}
\par   Clearly we have not yet answered the question, What is wealth? That wealth must be useful, to be wealth at all,—thus much is acknowledged by every one. But what particular thing is wealth, if not all things? Let us pursue the argument in another way; and then we may perhaps find what we are seeking. What is the use of wealth, and for what purpose has the possession of riches been invented,—in the sense, I mean, in which drugs have been discovered for the cure of disease? Perhaps in this way we may throw some light on the question. It appears to be clear that whatever constitutes wealth must be useful, and that wealth is one class of useful things; and now we have to enquire, What is the use of those useful things which constitute wealth? For all things probably may be said to be useful which we use in production, just as all things which have life are animals, but there is a special kind of animal which we call 'man.' Now if any one were to ask us, What is that of which, if we were rid, we should not want medicine and the instruments of medicine, we might reply that this would be the case if disease were absent from our bodies and either never came to them at all or went away again as soon as it appeared; and we may therefore conclude that medicine is the science which is useful for getting rid of disease. But if we are further asked, What is that from which, if we were free, we should have no need of wealth? can we give an answer? If we have none, suppose that we restate the question thus: —If a man could live without food or drink, and yet suffer neither hunger nor thirst, would he want either money or anything else in order to supply his needs?

\par \textbf{ERYXIAS}
\par   He would not.

\par \textbf{SOCRATES}
\par   And does not this apply in other cases? If we did not want for the service of the body the things of which we now stand in need, and heat and cold and the other bodily sensations were unperceived by us, there would be no use in this so-called wealth, if no one, that is, had any necessity for those things which now make us wish for wealth in order that we may satisfy the desires and needs of the body in respect of our various wants. And therefore if the possession of wealth is useful in ministering to our bodily wants, and bodily wants were unknown to us, we should not need wealth, and possibly there would be no such thing as wealth.

\par \textbf{ERYXIAS}
\par   Clearly not.

\par \textbf{SOCRATES}
\par   Then our conclusion is, as would appear, that wealth is what is useful to this end?

\par  Eryxias once more gave his assent, but the small argument considerably troubled him.

\par \textbf{SOCRATES}
\par   And what is your opinion about another question: —Would you say that the same thing can be at one time useful and at another useless for the production of the same result?

\par \textbf{ERYXIAS}
\par   I cannot say more than that if we require the same thing to produce the same result, then it seems to me to be useful; if not, not.

\par \textbf{SOCRATES}
\par   Then if without the aid of fire we could make a brazen statue, we should not want fire for that purpose; and if we did not want it, it would be useless to us? And the argument applies equally in other cases.

\par \textbf{ERYXIAS}
\par   Clearly.

\par \textbf{SOCRATES}
\par   And therefore conditions which are not required for the existence of a thing are not useful for the production of it?

\par \textbf{ERYXIAS}
\par   Of course not.

\par \textbf{SOCRATES}
\par   And if without gold or silver or anything else which we do not use directly for the body in the way that we do food and drink and bedding and houses,—if without these we could satisfy the wants of the body, they would be of no use to us for that purpose?

\par \textbf{ERYXIAS}
\par   They would not.

\par \textbf{SOCRATES}
\par   They would no longer be regarded as wealth, because they are useless, whereas that would be wealth which enabled us to obtain what was useful to us?

\par \textbf{ERYXIAS}
\par   O Socrates, you will never be able to persuade me that gold and silver and similar things are not wealth. But I am very strongly of opinion that things which are useless to us are not wealth, and that the money which is useful for this purpose is of the greatest use; not that these things are not useful towards life, if by them we can procure wealth.

\par \textbf{SOCRATES}
\par   And how would you answer another question? There are persons, are there not, who teach music and grammar and other arts for pay, and thus procure those things of which they stand in need?

\par \textbf{ERYXIAS}
\par   There are.

\par \textbf{SOCRATES}
\par   And these men by the arts which they profess, and in exchange for them, obtain the necessities of life just as we do by means of gold and silver?

\par \textbf{ERYXIAS}
\par   True.

\par \textbf{SOCRATES}
\par   Then if they procure by this means what they want for the purposes of life, that art will be useful towards life? For do we not say that silver is useful because it enables us to supply our bodily needs?

\par \textbf{ERYXIAS}
\par   We do.

\par \textbf{SOCRATES}
\par   Then if these arts are reckoned among things useful, the arts are wealth for the same reason as gold and silver are, for, clearly, the possession of them gives wealth. Yet a little while ago we found it difficult to accept the argument which proved that the wisest are the wealthiest. But now there seems no escape from this conclusion. Suppose that we are asked, 'Is a horse useful to everybody?' will not our reply be, 'No, but only to those who know how to use a horse?'

\par \textbf{ERYXIAS}
\par   Certainly.

\par \textbf{SOCRATES}
\par   And so, too, physic is not useful to every one, but only to him who knows how to use it?

\par \textbf{ERYXIAS}
\par   True.

\par \textbf{SOCRATES}
\par   And the same is the case with everything else?

\par \textbf{ERYXIAS}
\par   Yes.

\par \textbf{SOCRATES}
\par   Then gold and silver and all the other elements which are supposed to make up wealth are only useful to the person who knows how to use them?

\par \textbf{ERYXIAS}
\par   Exactly.

\par \textbf{SOCRATES}
\par   And were we not saying before that it was the business of a good man and a gentleman to know where and how anything should be used?

\par \textbf{ERYXIAS}
\par   Yes.

\par \textbf{SOCRATES}
\par   The good and gentle, therefore will alone have profit from these things, supposing at least that they know how to use them. But if so, to them only will they seem to be wealth. It appears, however, that where a person is ignorant of riding, and has horses which are useless to him, if some one teaches him that art, he makes him also richer, for what was before useless has now become useful to him, and in giving him knowledge he has also conferred riches upon him.

\par \textbf{ERYXIAS}
\par   That is the case.

\par \textbf{SOCRATES}
\par   Yet I dare be sworn that Critias will not be moved a whit by the argument.

\par \textbf{CRITIAS}
\par   No, by heaven, I should be a madman if I were. But why do you not finish the argument which proves that gold and silver and other things which seem to be wealth are not real wealth? For I have been exceedingly delighted to hear the discourses which you have just been holding.

\par \textbf{SOCRATES}
\par   My argument, Critias (I said), appears to have given you the same kind of pleasure which you might have derived from some rhapsode's recitation of Homer; for you do not believe a word of what has been said. But come now, give me an answer to this question. Are not certain things useful to the builder when he is building a house?

\par \textbf{CRITIAS}
\par   They are.

\par \textbf{SOCRATES}
\par   And would you say that those things are useful which are employed in house building,—stones and bricks and beams and the like, and also the instruments with which the builder built the house, the beams and stones which they provided, and again the instruments by which these were obtained?

\par \textbf{CRITIAS}
\par   It seems to me that they are all useful for building.

\par \textbf{SOCRATES}
\par   And is it not true of every art, that not only the materials but the instruments by which we procure them and without which the work could not go on, are useful for that art?

\par \textbf{CRITIAS}
\par   Certainly.

\par \textbf{SOCRATES}
\par   And further, the instruments by which the instruments are procured, and so on, going back from stage to stage ad infinitum,—are not all these, in your opinion, necessary in order to carry out the work?

\par \textbf{CRITIAS}
\par   We may fairly suppose such to be the case.

\par \textbf{SOCRATES}
\par   And if a man has food and drink and clothes and the other things which are useful to the body, would he need gold or silver or any other means by which he could procure that which he now has?

\par \textbf{CRITIAS}
\par   I do not think so.

\par \textbf{SOCRATES}
\par   Then you consider that a man never wants any of these things for the use of the body?

\par \textbf{CRITIAS}
\par   Certainly not.

\par \textbf{SOCRATES}
\par   And if they appear useless to this end, ought they not always to appear useless? For we have already laid down the principle that things cannot be at one time useful and at another time not, in the same process.

\par \textbf{CRITIAS}
\par   But in that respect your argument and mine are the same. For you maintain if they are useful to a certain end, they can never become useless; whereas I say that in order to accomplish some results bad things are needed, and good for others.

\par \textbf{SOCRATES}
\par   But can a bad thing be used to carry out a good purpose?

\par \textbf{CRITIAS}
\par   I should say not.

\par \textbf{SOCRATES}
\par   And we call those actions good which a man does for the sake of virtue?

\par \textbf{CRITIAS}
\par   Yes.

\par \textbf{SOCRATES}
\par   But can a man learn any kind of knowledge which is imparted by word of mouth if he is wholly deprived of the sense of hearing?

\par \textbf{CRITIAS}
\par   Certainly not, I think.

\par \textbf{SOCRATES}
\par   And will not hearing be useful for virtue, if virtue is taught by hearing and we use the sense of hearing in giving instruction?

\par \textbf{CRITIAS}
\par   Yes.

\par \textbf{SOCRATES}
\par   And since medicine frees the sick man from his disease, that art too may sometimes appear useful in the acquisition of virtue, e.g. when hearing is procured by the aid of medicine.

\par \textbf{CRITIAS}
\par   Very likely.

\par \textbf{SOCRATES}
\par   But if, again, we obtain by wealth the aid of medicine, shall we not regard wealth as useful for virtue?

\par \textbf{CRITIAS}
\par   True.

\par \textbf{SOCRATES}
\par   And also the instruments by which wealth is procured?

\par \textbf{CRITIAS}
\par   Certainly.

\par \textbf{SOCRATES}
\par   Then you think that a man may gain wealth by bad and disgraceful means, and, having obtained the aid of medicine which enables him to acquire the power of hearing, may use that very faculty for the acquisition of virtue?

\par \textbf{CRITIAS}
\par   Yes, I do.

\par \textbf{SOCRATES}
\par   But can that which is evil be useful for virtue?

\par \textbf{CRITIAS}
\par   No.

\par \textbf{SOCRATES}
\par   It is not therefore necessary that the means by which we obtain what is useful for a certain object should always be useful for the same object:  for it seems that bad actions may sometimes serve good purposes? The matter will be still plainer if we look at it in this way: —If things are useful towards the several ends for which they exist, which ends would not come into existence without them, how would you regard them? Can ignorance, for instance, be useful for knowledge, or disease for health, or vice for virtue?

\par \textbf{CRITIAS}
\par   Never.

\par \textbf{SOCRATES}
\par   And yet we have already agreed—have we not?—that there can be no knowledge where there has not previously been ignorance, nor health where there has not been disease, nor virtue where there has not been vice?

\par \textbf{CRITIAS}
\par   I think that we have.

\par \textbf{SOCRATES}
\par   But then it would seem that the antecedents without which a thing cannot exist are not necessarily useful to it. Otherwise ignorance would appear useful for knowledge, disease for health, and vice for virtue.

\par  Critias still showed great reluctance to accept any argument which went to prove that all these things were useless. I saw that it was as difficult to persuade him as (according to the proverb) it is to boil a stone, so I said: Let us bid 'good-bye' to the discussion, since we cannot agree whether these things are useful and a part of wealth or not. But what shall we say to another question: Which is the happier and better man,—he who requires the greatest quantity of necessaries for body and diet, or he who requires only the fewest and least? The answer will perhaps become more obvious if we suppose some one, comparing the man himself at different times, to consider whether his condition is better when he is sick or when he is well?

\par \textbf{CRITIAS}
\par   That is not a question which needs much consideration.

\par \textbf{SOCRATES}
\par   Probably, I said, every one can understand that health is a better condition than disease. But when have we the greatest and the most various needs, when we are sick or when we are well?

\par \textbf{CRITIAS}
\par   When we are sick.

\par \textbf{SOCRATES}
\par   And when we are in the worst state we have the greatest and most especial need and desire of bodily pleasures?

\par \textbf{CRITIAS}
\par   True.

\par \textbf{SOCRATES}
\par   And seeing that a man is best off when he is least in need of such things, does not the same reasoning apply to the case of any two persons, of whom one has many and great wants and desires, and the other few and moderate? For instance, some men are gamblers, some drunkards, and some gluttons:  and gambling and the love of drink and greediness are all desires?

\par \textbf{CRITIAS}
\par   Certainly.

\par \textbf{SOCRATES}
\par   But desires are only the lack of something:  and those who have the greatest desires are in a worse condition than those who have none or very slight ones?

\par \textbf{CRITIAS}
\par   Certainly I consider that those who have such wants are bad, and that the greater their wants the worse they are.

\par \textbf{SOCRATES}
\par   And do we think it possible that a thing should be useful for a purpose unless we have need of it for that purpose?

\par \textbf{CRITIAS}
\par   No.

\par \textbf{SOCRATES}
\par   Then if these things are useful for supplying the needs of the body, we must want them for that purpose?

\par \textbf{CRITIAS}
\par   That is my opinion.

\par \textbf{SOCRATES}
\par   And he to whom the greatest number of things are useful for his purpose, will also want the greatest number of means of accomplishing it, supposing that we necessarily feel the want of all useful things?

\par \textbf{CRITIAS}
\par   It seems so.

\par \textbf{SOCRATES}
\par   The argument proves then that he who has great riches has likewise need of many things for the supply of the wants of the body; for wealth appears useful towards that end. And the richest must be in the worst condition, since they seem to be most in want of such things.

\par 
 
\end{document}