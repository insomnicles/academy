
\documentclass[11pt,letter]{article}


\begin{document}

\title{Theaetetus\thanks{Source: https://www.gutenberg.org/files/1726/1726-h/1726-h.htm. License: http://gutenberg.org/license ds}}
\date{\today}
\author{Plato, 427? BCE-347? BCE\\ Translated by Jowett, Benjamin, 1817-1893}
\maketitle

\setcounter{tocdepth}{1}
\tableofcontents
\renewcommand{\baselinestretch}{1.0}
\normalsize
\newpage

\section{
      INTRODUCTION AND ANALYSIS.
    }
\par  Some dialogues of Plato are of so various a character that their relation to the other dialogues cannot be determined with any degree of certainty. The Theaetetus, like the Parmenides, has points of similarity both with his earlier and his later writings. The perfection of style, the humour, the dramatic interest, the complexity of structure, the fertility of illustration, the shifting of the points of view, are characteristic of his best period of authorship. The vain search, the negative conclusion, the figure of the midwives, the constant profession of ignorance on the part of Socrates, also bear the stamp of the early dialogues, in which the original Socrates is not yet Platonized. Had we no other indications, we should be disposed to range the Theaetetus with the Apology and the Phaedrus, and perhaps even with the Protagoras and the Laches.

\par  But when we pass from the style to an examination of the subject, we trace a connection with the later rather than with the earlier dialogues. In the first place there is the connexion, indicated by Plato himself at the end of the dialogue, with the Sophist, to which in many respects the Theaetetus is so little akin. (1) The same persons reappear, including the younger Socrates, whose name is just mentioned in the Theaetetus; (2) the theory of rest, which Socrates has declined to consider, is resumed by the Eleatic Stranger; (3) there is a similar allusion in both dialogues to the meeting of Parmenides and Socrates (Theaet., Soph. ); and (4) the inquiry into not-being in the Sophist supplements the question of false opinion which is raised in the Theaetetus. (Compare also Theaet. and Soph. for parallel turns of thought.) Secondly, the later date of the dialogue is confirmed by the absence of the doctrine of recollection and of any doctrine of ideas except that which derives them from generalization and from reflection of the mind upon itself. The general character of the Theaetetus is dialectical, and there are traces of the same Megarian influences which appear in the Parmenides, and which later writers, in their matter of fact way, have explained by the residence of Plato at Megara. Socrates disclaims the character of a professional eristic, and also, with a sort of ironical admiration, expresses his inability to attain the Megarian precision in the use of terms. Yet he too employs a similar sophistical skill in overturning every conceivable theory of knowledge.

\par  The direct indications of a date amount to no more than this: the conversation is said to have taken place when Theaetetus was a youth, and shortly before the death of Socrates. At the time of his own death he is supposed to be a full-grown man. Allowing nine or ten years for the interval between youth and manhood, the dialogue could not have been written earlier than 390, when Plato was about thirty-nine years of age. No more definite date is indicated by the engagement in which Theaetetus is said to have fallen or to have been wounded, and which may have taken place any time during the Corinthian war, between the years 390-387. The later date which has been suggested, 369, when the Athenians and Lacedaemonians disputed the Isthmus with Epaminondas, would make the age of Theaetetus at his death forty-five or forty-six. This a little impairs the beauty of Socrates' remark, that 'he would be a great man if he lived.'

\par  In this uncertainty about the place of the Theaetetus, it seemed better, as in the case of the Republic, Timaeus, Critias, to retain the order in which Plato himself has arranged this and the two companion dialogues. We cannot exclude the possibility which has been already noticed in reference to other works of Plato, that the Theaetetus may not have been all written continuously; or the probability that the Sophist and Politicus, which differ greatly in style, were only appended after a long interval of time. The allusion to Parmenides compared with the Sophist, would probably imply that the dialogue which is called by his name was already in existence; unless, indeed, we suppose the passage in which the allusion occurs to have been inserted afterwards. Again, the Theaetetus may be connected with the Gorgias, either dialogue from different points of view containing an analysis of the real and apparent (Schleiermacher); and both may be brought into relation with the Apology as illustrating the personal life of Socrates. The Philebus, too, may with equal reason be placed either after or before what, in the language of Thrasyllus, may be called the Second Platonic Trilogy. Both the Parmenides and the Sophist, and still more the Theaetetus, have points of affinity with the Cratylus, in which the principles of rest and motion are again contrasted, and the Sophistical or Protagorean theory of language is opposed to that which is attributed to the disciple of Heracleitus, not to speak of lesser resemblances in thought and language. The Parmenides, again, has been thought by some to hold an intermediate position between the Theaetetus and the Sophist; upon this view, the Sophist may be regarded as the answer to the problems about One and Being which have been raised in the Parmenides. Any of these arrangements may suggest new views to the student of Plato; none of them can lay claim to an exclusive probability in its favour.

\par  The Theaetetus is one of the narrated dialogues of Plato, and is the only one which is supposed to have been written down. In a short introductory scene, Euclides and Terpsion are described as meeting before the door of Euclides' house in Megara. This may have been a spot familiar to Plato (for Megara was within a walk of Athens), but no importance can be attached to the accidental introduction of the founder of the Megarian philosophy. The real intention of the preface is to create an interest about the person of Theaetetus, who has just been carried up from the army at Corinth in a dying state. The expectation of his death recalls the promise of his youth, and especially the famous conversation which Socrates had with him when he was quite young, a few days before his own trial and death, as we are once more reminded at the end of the dialogue. Yet we may observe that Plato has himself forgotten this, when he represents Euclides as from time to time coming to Athens and correcting the copy from Socrates' own mouth. The narrative, having introduced Theaetetus, and having guaranteed the authenticity of the dialogue (compare Symposium, Phaedo, Parmenides), is then dropped. No further use is made of the device. As Plato himself remarks, who in this as in some other minute points is imitated by Cicero (De Amicitia), the interlocutory words are omitted.

\par  Theaetetus, the hero of the battle of Corinth and of the dialogue, is a disciple of Theodorus, the great geometrician, whose science is thus indicated to be the propaedeutic to philosophy. An interest has been already excited about him by his approaching death, and now he is introduced to us anew by the praises of his master Theodorus. He is a youthful Socrates, and exhibits the same contrast of the fair soul and the ungainly face and frame, the Silenus mask and the god within, which are described in the Symposium. The picture which Theodorus gives of his courage and patience and intelligence and modesty is verified in the course of the dialogue. His courage is shown by his behaviour in the battle, and his other qualities shine forth as the argument proceeds. Socrates takes an evident delight in 'the wise Theaetetus,' who has more in him than 'many bearded men'; he is quite inspired by his answers. At first the youth is lost in wonder, and is almost too modest to speak, but, encouraged by Socrates, he rises to the occasion, and grows full of interest and enthusiasm about the great question. Like a youth, he has not finally made up his mind, and is very ready to follow the lead of Socrates, and to enter into each successive phase of the discussion which turns up. His great dialectical talent is shown in his power of drawing distinctions, and of foreseeing the consequences of his own answers. The enquiry about the nature of knowledge is not new to him; long ago he has felt the 'pang of philosophy,' and has experienced the youthful intoxication which is depicted in the Philebus. But he has hitherto been unable to make the transition from mathematics to metaphysics. He can form a general conception of square and oblong numbers, but he is unable to attain a similar expression of knowledge in the abstract. Yet at length he begins to recognize that there are universal conceptions of being, likeness, sameness, number, which the mind contemplates in herself, and with the help of Socrates is conducted from a theory of sense to a theory of ideas.

\par  There is no reason to doubt that Theaetetus was a real person, whose name survived in the next generation. But neither can any importance be attached to the notices of him in Suidas and Proclus, which are probably based on the mention of him in Plato. According to a confused statement in Suidas, who mentions him twice over, first, as a pupil of Socrates, and then of Plato, he is said to have written the first work on the Five Solids. But no early authority cites the work, the invention of which may have been easily suggested by the division of roots, which Plato attributes to him, and the allusion to the backward state of solid geometry in the Republic. At any rate, there is no occasion to recall him to life again after the battle of Corinth, in order that we may allow time for the completion of such a work (Muller). We may also remark that such a supposition entirely destroys the pathetic interest of the introduction.

\par  Theodorus, the geometrician, had once been the friend and disciple of Protagoras, but he is very reluctant to leave his retirement and defend his old master. He is too old to learn Socrates' game of question and answer, and prefers the digressions to the main argument, because he finds them easier to follow. The mathematician, as Socrates says in the Republic, is not capable of giving a reason in the same manner as the dialectician, and Theodorus could not therefore have been appropriately introduced as the chief respondent. But he may be fairly appealed to, when the honour of his master is at stake. He is the 'guardian of his orphans,' although this is a responsibility which he wishes to throw upon Callias, the friend and patron of all Sophists, declaring that he himself had early 'run away' from philosophy, and was absorbed in mathematics. His extreme dislike to the Heraclitean fanatics, which may be compared with the dislike of Theaetetus to the materialists, and his ready acceptance of the noble words of Socrates, are noticeable traits of character.

\par  The Socrates of the Theaetetus is the same as the Socrates of the earlier dialogues. He is the invincible disputant, now advanced in years, of the Protagoras and Symposium; he is still pursuing his divine mission, his 'Herculean labours,' of which he has described the origin in the Apology; and he still hears the voice of his oracle, bidding him receive or not receive the truant souls. There he is supposed to have a mission to convict men of self-conceit; in the Theaetetus he has assigned to him by God the functions of a man-midwife, who delivers men of their thoughts, and under this character he is present throughout the dialogue. He is the true prophet who has an insight into the natures of men, and can divine their future; and he knows that sympathy is the secret power which unlocks their thoughts. The hit at Aristides, the son of Lysimachus, who was specially committed to his charge in the Laches, may be remarked by the way. The attempt to discover the definition of knowledge is in accordance with the character of Socrates as he is described in the Memorabilia, asking What is justice? what is temperance? and the like. But there is no reason to suppose that he would have analyzed the nature of perception, or traced the connexion of Protagoras and Heracleitus, or have raised the difficulty respecting false opinion. The humorous illustrations, as well as the serious thoughts, run through the dialogue. The snubnosedness of Theaetetus, a characteristic which he shares with Socrates, and the man-midwifery of Socrates, are not forgotten in the closing words. At the end of the dialogue, as in the Euthyphro, he is expecting to meet Meletus at the porch of the king Archon; but with the same indifference to the result which is everywhere displayed by him, he proposes that they shall reassemble on the following day at the same spot. The day comes, and in the Sophist the three friends again meet, but no further allusion is made to the trial, and the principal share in the argument is assigned, not to Socrates, but to an Eleatic stranger; the youthful Theaetetus also plays a different and less independent part. And there is no allusion in the Introduction to the second and third dialogues, which are afterwards appended. There seems, therefore, reason to think that there is a real change, both in the characters and in the design.

\par  The dialogue is an enquiry into the nature of knowledge, which is interrupted by two digressions. The first is the digression about the midwives, which is also a leading thought or continuous image, like the wave in the Republic, appearing and reappearing at intervals. Again and again we are reminded that the successive conceptions of knowledge are extracted from Theaetetus, who in his turn truly declares that Socrates has got a great deal more out of him than ever was in him. Socrates is never weary of working out the image in humorous details,—discerning the symptoms of labour, carrying the child round the hearth, fearing that Theaetetus will bite him, comparing his conceptions to wind-eggs, asserting an hereditary right to the occupation. There is also a serious side to the image, which is an apt similitude of the Socratic theory of education (compare Republic, Sophist), and accords with the ironical spirit in which the wisest of men delights to speak of himself.

\par  The other digression is the famous contrast of the lawyer and philosopher. This is a sort of landing-place or break in the middle of the dialogue. At the commencement of a great discussion, the reflection naturally arises, How happy are they who, like the philosopher, have time for such discussions (compare Republic)! There is no reason for the introduction of such a digression; nor is a reason always needed, any more than for the introduction of an episode in a poem, or of a topic in conversation. That which is given by Socrates is quite sufficient, viz. that the philosopher may talk and write as he pleases. But though not very closely connected, neither is the digression out of keeping with the rest of the dialogue. The philosopher naturally desires to pour forth the thoughts which are always present to him, and to discourse of the higher life. The idea of knowledge, although hard to be defined, is realised in the life of philosophy. And the contrast is the favourite antithesis between the world, in the various characters of sophist, lawyer, statesman, speaker, and the philosopher,—between opinion and knowledge,—between the conventional and the true.

\par  The greater part of the dialogue is devoted to setting up and throwing down definitions of science and knowledge. Proceeding from the lower to the higher by three stages, in which perception, opinion, reasoning are successively examined, we first get rid of the confusion of the idea of knowledge and specific kinds of knowledge,—a confusion which has been already noticed in the Lysis, Laches, Meno, and other dialogues. In the infancy of logic, a form of thought has to be invented before the content can be filled up. We cannot define knowledge until the nature of definition has been ascertained. Having succeeded in making his meaning plain, Socrates proceeds to analyze (1) the first definition which Theaetetus proposes: 'Knowledge is sensible perception.' This is speedily identified with the Protagorean saying, 'Man is the measure of all things;' and of this again the foundation is discovered in the perpetual flux of Heracleitus. The relativeness of sensation is then developed at length, and for a moment the definition appears to be accepted. But soon the Protagorean thesis is pronounced to be suicidal; for the adversaries of Protagoras are as good a measure as he is, and they deny his doctrine. He is then supposed to reply that the perception may be true at any given instant. But the reply is in the end shown to be inconsistent with the Heraclitean foundation, on which the doctrine has been affirmed to rest. For if the Heraclitean flux is extended to every sort of change in every instant of time, how can any thought or word be detained even for an instant? Sensible perception, like everything else, is tumbling to pieces. Nor can Protagoras himself maintain that one man is as good as another in his knowledge of the future; and 'the expedient,' if not 'the just and true,' belongs to the sphere of the future.

\par  And so we must ask again, What is knowledge? The comparison of sensations with one another implies a principle which is above sensation, and which resides in the mind itself. We are thus led to look for knowledge in a higher sphere, and accordingly Theaetetus, when again interrogated, replies (2) that 'knowledge is true opinion.' But how is false opinion possible? The Megarian or Eristic spirit within us revives the question, which has been already asked and indirectly answered in the Meno: 'How can a man be ignorant of that which he knows?' No answer is given to this not unanswerable question. The comparison of the mind to a block of wax, or to a decoy of birds, is found wanting.

\par  But are we not inverting the natural order in looking for opinion before we have found knowledge? And knowledge is not true opinion; for the Athenian dicasts have true opinion but not knowledge. What then is knowledge? We answer (3), 'True opinion, with definition or explanation.' But all the different ways in which this statement may be understood are set aside, like the definitions of courage in the Laches, or of friendship in the Lysis, or of temperance in the Charmides. At length we arrive at the conclusion, in which nothing is concluded.

\par  There are two special difficulties which beset the student of the Theaetetus: (1) he is uncertain how far he can trust Plato's account of the theory of Protagoras; and he is also uncertain (2) how far, and in what parts of the dialogue, Plato is expressing his own opinion. The dramatic character of the work renders the answer to both these questions difficult.

\par  1. In reply to the first, we have only probabilities to offer. Three main points have to be decided: (a) Would Protagoras have identified his own thesis, 'Man is the measure of all things,' with the other, 'All knowledge is sensible perception'? (b) Would he have based the relativity of knowledge on the Heraclitean flux? (c) Would he have asserted the absoluteness of sensation at each instant? Of the work of Protagoras on 'Truth' we know nothing, with the exception of the two famous fragments, which are cited in this dialogue, 'Man is the measure of all things,' and, 'Whether there are gods or not, I cannot tell.' Nor have we any other trustworthy evidence of the tenets of Protagoras, or of the sense in which his words are used. For later writers, including Aristotle in his Metaphysics, have mixed up the Protagoras of Plato, as they have the Socrates of Plato, with the real person.

\par  Returning then to the Theaetetus, as the only possible source from which an answer to these questions can be obtained, we may remark, that Plato had 'The Truth' of Protagoras before him, and frequently refers to the book. He seems to say expressly, that in this work the doctrine of the Heraclitean flux was not to be found; 'he told the real truth' (not in the book, which is so entitled, but) 'privately to his disciples,'—words which imply that the connexion between the doctrines of Protagoras and Heracleitus was not generally recognized in Greece, but was really discovered or invented by Plato. On the other hand, the doctrine that 'Man is the measure of all things,' is expressly identified by Socrates with the other statement, that 'What appears to each man is to him;' and a reference is made to the books in which the statement occurs;—this Theaetetus, who has 'often read the books,' is supposed to acknowledge (so Cratylus). And Protagoras, in the speech attributed to him, never says that he has been misunderstood: he rather seems to imply that the absoluteness of sensation at each instant was to be found in his words. He is only indignant at the 'reductio ad absurdum' devised by Socrates for his 'homo mensura,' which Theodorus also considers to be 'really too bad.'

\par  The question may be raised, how far Plato in the Theaetetus could have misrepresented Protagoras without violating the laws of dramatic probability. Could he have pretended to cite from a well-known writing what was not to be found there? But such a shadowy enquiry is not worth pursuing further. We need only remember that in the criticism which follows of the thesis of Protagoras, we are criticizing the Protagoras of Plato, and not attempting to draw a precise line between his real sentiments and those which Plato has attributed to him.

\par  2. The other difficulty is a more subtle, and also a more important one, because bearing on the general character of the Platonic dialogues. On a first reading of them, we are apt to imagine that the truth is only spoken by Socrates, who is never guilty of a fallacy himself, and is the great detector of the errors and fallacies of others. But this natural presumption is disturbed by the discovery that the Sophists are sometimes in the right and Socrates in the wrong. Like the hero of a novel, he is not to be supposed always to represent the sentiments of the author. There are few modern readers who do not side with Protagoras, rather than with Socrates, in the dialogue which is called by his name. The Cratylus presents a similar difficulty: in his etymologies, as in the number of the State, we cannot tell how far Socrates is serious; for the Socratic irony will not allow him to distinguish between his real and his assumed wisdom. No one is the superior of the invincible Socrates in argument (except in the first part of the Parmenides, where he is introduced as a youth); but he is by no means supposed to be in possession of the whole truth. Arguments are often put into his mouth (compare Introduction to the Gorgias) which must have seemed quite as untenable to Plato as to a modern writer. In this dialogue a great part of the answer of Protagoras is just and sound; remarks are made by him on verbal criticism, and on the importance of understanding an opponent's meaning, which are conceived in the true spirit of philosophy. And the distinction which he is supposed to draw between Eristic and Dialectic, is really a criticism of Plato on himself and his own criticism of Protagoras.

\par  The difficulty seems to arise from not attending to the dramatic character of the writings of Plato. There are two, or more, sides to questions; and these are parted among the different speakers. Sometimes one view or aspect of a question is made to predominate over the rest, as in the Gorgias or Sophist; but in other dialogues truth is divided, as in the Laches and Protagoras, and the interest of the piece consists in the contrast of opinions. The confusion caused by the irony of Socrates, who, if he is true to his character, cannot say anything of his own knowledge, is increased by the circumstance that in the Theaetetus and some other dialogues he is occasionally playing both parts himself, and even charging his own arguments with unfairness. In the Theaetetus he is designedly held back from arriving at a conclusion. For we cannot suppose that Plato conceived a definition of knowledge to be impossible. But this is his manner of approaching and surrounding a question. The lights which he throws on his subject are indirect, but they are not the less real for that. He has no intention of proving a thesis by a cut-and-dried argument; nor does he imagine that a great philosophical problem can be tied up within the limits of a definition. If he has analyzed a proposition or notion, even with the severity of an impossible logic, if half-truths have been compared by him with other half-truths, if he has cleared up or advanced popular ideas, or illustrated a new method, his aim has been sufficiently accomplished.

\par  The writings of Plato belong to an age in which the power of analysis had outrun the means of knowledge; and through a spurious use of dialectic, the distinctions which had been already 'won from the void and formless infinite,' seemed to be rapidly returning to their original chaos. The two great speculative philosophies, which a century earlier had so deeply impressed the mind of Hellas, were now degenerating into Eristic. The contemporaries of Plato and Socrates were vainly trying to find new combinations of them, or to transfer them from the object to the subject. The Megarians, in their first attempts to attain a severer logic, were making knowledge impossible (compare Theaet.). They were asserting 'the one good under many names,' and, like the Cynics, seem to have denied predication, while the Cynics themselves were depriving virtue of all which made virtue desirable in the eyes of Socrates and Plato. And besides these, we find mention in the later writings of Plato, especially in the Theaetetus, Sophist, and Laws, of certain impenetrable godless persons, who will not believe what they 'cannot hold in their hands'; and cannot be approached in argument, because they cannot argue (Theat; Soph.). No school of Greek philosophers exactly answers to these persons, in whom Plato may perhaps have blended some features of the Atomists with the vulgar materialistic tendencies of mankind in general (compare Introduction to the Sophist).

\par  And not only was there a conflict of opinions, but the stage which the mind had reached presented other difficulties hardly intelligible to us, who live in a different cycle of human thought. All times of mental progress are times of confusion; we only see, or rather seem to see things clearly, when they have been long fixed and defined. In the age of Plato, the limits of the world of imagination and of pure abstraction, of the old world and the new, were not yet fixed. The Greeks, in the fourth century before Christ, had no words for 'subject' and 'object,' and no distinct conception of them; yet they were always hovering about the question involved in them. The analysis of sense, and the analysis of thought, were equally difficult to them; and hopelessly confused by the attempt to solve them, not through an appeal to facts, but by the help of general theories respecting the nature of the universe.

\par  Plato, in his Theaetetus, gathers up the sceptical tendencies of his age, and compares them. But he does not seek to reconstruct out of them a theory of knowledge. The time at which such a theory could be framed had not yet arrived. For there was no measure of experience with which the ideas swarming in men's minds could be compared; the meaning of the word 'science' could scarcely be explained to them, except from the mathematical sciences, which alone offered the type of universality and certainty. Philosophy was becoming more and more vacant and abstract, and not only the Platonic Ideas and the Eleatic Being, but all abstractions seemed to be at variance with sense and at war with one another.

\par  The want of the Greek mind in the fourth century before Christ was not another theory of rest or motion, or Being or atoms, but rather a philosophy which could free the mind from the power of abstractions and alternatives, and show how far rest and how far motion, how far the universal principle of Being and the multitudinous principle of atoms, entered into the composition of the world; which could distinguish between the true and false analogy, and allow the negative as well as the positive a place in human thought. To such a philosophy Plato, in the Theaetetus, offers many contributions. He has followed philosophy into the region of mythology, and pointed out the similarities of opposing phases of thought. He has also shown that extreme abstractions are self-destructive, and, indeed, hardly distinguishable from one another. But his intention is not to unravel the whole subject of knowledge, if this had been possible; and several times in the course of the dialogue he rejects explanations of knowledge which have germs of truth in them; as, for example, 'the resolution of the compound into the simple;' or 'right opinion with a mark of difference.'

\par  ...

\par  Terpsion, who has come to Megara from the country, is described as having looked in vain for Euclides in the Agora; the latter explains that he has been down to the harbour, and on his way thither had met Theaetetus, who was being carried up from the army to Athens. He was scarcely alive, for he had been badly wounded at the battle of Corinth, and had taken the dysentery which prevailed in the camp. The mention of his condition suggests the reflection, 'What a loss he will be!' 'Yes, indeed,' replies Euclid; 'only just now I was hearing of his noble conduct in the battle.' 'That I should expect; but why did he not remain at Megara?' 'I wanted him to remain, but he would not; so I went with him as far as Erineum; and as I parted from him, I remembered that Socrates had seen him when he was a youth, and had a remarkable conversation with him, not long before his own death; and he then prophesied of him that he would be a great man if he lived.' 'How true that has been; how like all that Socrates said! And could you repeat the conversation?' 'Not from memory; but I took notes when I returned home, which I afterwards filled up at leisure, and got Socrates to correct them from time to time, when I came to Athens'...Terpsion had long intended to ask for a sight of this writing, of which he had already heard. They are both tired, and agree to rest and have the conversation read to them by a servant...'Here is the roll, Terpsion; I need only observe that I have omitted, for the sake of convenience, the interlocutory words, "said I," "said he"; and that Theaetetus, and Theodorus, the geometrician of Cyrene, are the persons with whom Socrates is conversing.'

\par  Socrates begins by asking Theodorus whether, in his visit to Athens, he has found any Athenian youth likely to attain distinction in science. 'Yes, Socrates, there is one very remarkable youth, with whom I have become acquainted. He is no beauty, and therefore you need not imagine that I am in love with him; and, to say the truth, he is very like you, for he has a snub nose, and projecting eyes, although these features are not so marked in him as in you. He combines the most various qualities, quickness, patience, courage; and he is gentle as well as wise, always silently flowing on, like a river of oil. Look! he is the middle one of those who are entering the palaestra.'

\par  Socrates, who does not know his name, recognizes him as the son of Euphronius, who was himself a good man and a rich. He is informed by Theodorus that the youth is named Theaetetus, but the property of his father has disappeared in the hands of trustees; this does not, however, prevent him from adding liberality to his other virtues. At the desire of Socrates he invites Theaetetus to sit by them.

\par  'Yes,' says Socrates, 'that I may see in you, Theaetetus, the image of my ugly self, as Theodorus declares. Not that his remark is of any importance; for though he is a philosopher, he is not a painter, and therefore he is no judge of our faces; but, as he is a man of science, he may be a judge of our intellects. And if he were to praise the mental endowments of either of us, in that case the hearer of the eulogy ought to examine into what he says, and the subject should not refuse to be examined.' Theaetetus consents, and is caught in a trap (compare the similar trap which is laid for Theodorus). 'Then, Theaetetus, you will have to be examined, for Theodorus has been praising you in a style of which I never heard the like.' 'He was only jesting.' 'Nay, that is not his way; and I cannot allow you, on that pretence, to retract the assent which you have already given, or I shall make Theodorus repeat your praises, and swear to them.' Theaetetus, in reply, professes that he is willing to be examined, and Socrates begins by asking him what he learns of Theodorus. He is himself anxious to learn anything of anybody; and now he has a little question to which he wants Theaetetus or Theodorus (or whichever of the company would not be 'donkey' to the rest) to find an answer. Without further preface, but at the same time apologizing for his eagerness, he asks, 'What is knowledge?' Theodorus is too old to answer questions, and begs him to interrogate Theaetetus, who has the advantage of youth.

\par  Theaetetus replies, that knowledge is what he learns of Theodorus, i.e. geometry and arithmetic; and that there are other kinds of knowledge—shoemaking, carpentering, and the like. But Socrates rejoins, that this answer contains too much and also too little. For although Theaetetus has enumerated several kinds of knowledge, he has not explained the common nature of them; as if he had been asked, 'What is clay?' and instead of saying 'Clay is moistened earth,' he had answered, 'There is one clay of image-makers, another of potters, another of oven-makers.' Theaetetus at once divines that Socrates means him to extend to all kinds of knowledge the same process of generalization which he has already learned to apply to arithmetic. For he has discovered a division of numbers into square numbers, 4, 9, 16, etc., which are composed of equal factors, and represent figures which have equal sides, and oblong numbers, 3, 5, 6, 7, etc., which are composed of unequal factors, and represent figures which have unequal sides. But he has never succeeded in attaining a similar conception of knowledge, though he has often tried; and, when this and similar questions were brought to him from Socrates, has been sorely distressed by them. Socrates explains to him that he is in labour. For men as well as women have pangs of labour; and both at times require the assistance of midwives. And he, Socrates, is a midwife, although this is a secret; he has inherited the art from his mother bold and bluff, and he ushers into light, not children, but the thoughts of men. Like the midwives, who are 'past bearing children,' he too can have no offspring—the God will not allow him to bring anything into the world of his own. He also reminds Theaetetus that the midwives are or ought to be the only matchmakers (this is the preparation for a biting jest); for those who reap the fruit are most likely to know on what soil the plants will grow. But respectable midwives avoid this department of practice—they do not want to be called procuresses. There are some other differences between the two sorts of pregnancy. For women do not bring into the world at one time real children and at another time idols which are with difficulty distinguished from them. 'At first,' says Socrates in his character of the man-midwife, 'my patients are barren and stolid, but after a while they "round apace," if the gods are propitious to them; and this is due not to me but to themselves; I and the god only assist in bringing their ideas to the birth. Many of them have left me too soon, and the result has been that they have produced abortions; or when I have delivered them of children they have lost them by an ill bringing up, and have ended by seeing themselves, as others see them, to be great fools. Aristides, the son of Lysimachus, is one of these, and there have been others. The truants often return to me and beg to be taken back; and then, if my familiar allows me, which is not always the case, I receive them, and they begin to grow again. There come to me also those who have nothing in them, and have no need of my art; and I am their matchmaker (see above), and marry them to Prodicus or some other inspired sage who is likely to suit them. I tell you this long story because I suspect that you are in labour. Come then to me, who am a midwife, and the son of a midwife, and I will deliver you. And do not bite me, as the women do, if I abstract your first-born; for I am acting out of good-will towards you; the God who is within me is the friend of man, though he will not allow me to dissemble the truth. Once more then, Theaetetus, I repeat my old question—"What is knowledge?" Take courage, and by the help of God you will discover an answer.' 'My answer is, that knowledge is perception.' 'That is the theory of Protagoras, who has another way of expressing the same thing when he says, "Man is the measure of all things." He was a very wise man, and we should try to understand him. In order to illustrate his meaning let me suppose that there is the same wind blowing in our faces, and one of us may be hot and the other cold. How is this? Protagoras will reply that the wind is hot to him who is cold, cold to him who is hot. And "is" means "appears," and when you say "appears to him," that means "he feels." Thus feeling, appearance, perception, coincide with being. I suspect, however, that this was only a "facon de parler," by which he imposed on the common herd like you and me; he told "the truth" (in allusion to the title of his book, which was called "The Truth") in secret to his disciples. For he was really a votary of that famous philosophy in which all things are said to be relative; nothing is great or small, or heavy or light, or one, but all is in motion and mixture and transition and flux and generation, not "being," as we ignorantly affirm, but "becoming." This has been the doctrine, not of Protagoras only, but of all philosophers, with the single exception of Parmenides; Empedocles, Heracleitus, and others, and all the poets, with Epicharmus, the king of Comedy, and Homer, the king of Tragedy, at their head, have said the same; the latter has these words—

\par  "Ocean, whence the gods sprang, and mother Tethys."

\par  And many arguments are used to show, that motion is the source of life, and rest of death: fire and warmth are produced by friction, and living creatures owe their origin to a similar cause; the bodily frame is preserved by exercise and destroyed by indolence; and if the sun ceased to move, "chaos would come again." Now apply this doctrine of "All is motion" to the senses, and first of all to the sense of sight. The colour of white, or any other colour, is neither in the eyes nor out of them, but ever in motion between the object and the eye, and varying in the case of every percipient. All is relative, and, as the followers of Protagoras remark, endless contradictions arise when we deny this; e.g. here are six dice; they are more than four and less than twelve; "more and also less," would you not say?' 'Yes.' 'But Protagoras will retort: "Can anything be more or less without addition or subtraction?"'

\par  'I should say "No" if I were not afraid of contradicting my former answer.'

\par  'And if you say "Yes," the tongue will escape conviction but not the mind, as Euripides would say?' 'True.' 'The thoroughbred Sophists, who know all that can be known, would have a sparring match over this, but you and I, who have no professional pride, want only to discover whether our ideas are clear and consistent. And we cannot be wrong in saying, first, that nothing can be greater or less while remaining equal; secondly, that there can be no becoming greater or less without addition or subtraction; thirdly, that what is and was not, cannot be without having become. But then how is this reconcilable with the case of the dice, and with similar examples?—that is the question.' 'I am often perplexed and amazed, Socrates, by these difficulties.' 'That is because you are a philosopher, for philosophy begins in wonder, and Iris is the child of Thaumas. Do you know the original principle on which the doctrine of Protagoras is based?' 'No.' 'Then I will tell you; but we must not let the uninitiated hear, and by the uninitiated I mean the obstinate people who believe in nothing which they cannot hold in their hands. The brethren whose mysteries I am about to unfold to you are far more ingenious. They maintain that all is motion; and that motion has two forms, action and passion, out of which endless phenomena are created, also in two forms—sense and the object of sense—which come to the birth together. There are two kinds of motions, a slow and a fast; the motions of the agent and the patient are slower, because they move and create in and about themselves, but the things which are born of them have a swifter motion, and pass rapidly from place to place. The eye and the appropriate object come together, and give birth to whiteness and the sensation of whiteness; the eye is filled with seeing, and becomes not sight but a seeing eye, and the object is filled with whiteness, and becomes not whiteness but white; and no other compound of either with another would have produced the same effect. All sensation is to be resolved into a similar combination of an agent and patient. Of either, taken separately, no idea can be formed; and the agent may become a patient, and the patient an agent. Hence there arises a general reflection that nothing is, but all things become; no name can detain or fix them. Are not these speculations charming, Theaetetus, and very good for a person in your interesting situation? I am offering you specimens of other men's wisdom, because I have no wisdom of my own, and I want to deliver you of something; and presently we will see whether you have brought forth wind or not. Tell me, then, what do you think of the notion that "All things are becoming"?'

\par  'When I hear your arguments, I am marvellously ready to assent.'

\par  'But I ought not to conceal from you that there is a serious objection which may be urged against this doctrine of Protagoras. For there are states, such as madness and dreaming, in which perception is false; and half our life is spent in dreaming; and who can say that at this instant we are not dreaming? Even the fancies of madmen are real at the time. But if knowledge is perception, how can we distinguish between the true and the false in such cases? Having stated the objection, I will now state the answer. Protagoras would deny the continuity of phenomena; he would say that what is different is entirely different, and whether active or passive has a different power. There are infinite agents and patients in the world, and these produce in every combination of them a different perception. Take myself as an instance:—Socrates may be ill or he may be well,—and remember that Socrates, with all his accidents, is spoken of. The wine which I drink when I am well is pleasant to me, but the same wine is unpleasant to me when I am ill. And there is nothing else from which I can receive the same impression, nor can another receive the same impression from the wine. Neither can I and the object of sense become separately what we become together. For the one in becoming is relative to the other, but they have no other relation; and the combination of them is absolute at each moment. (In modern language, the act of sensation is really indivisible, though capable of a mental analysis into subject and object.) My sensation alone is true, and true to me only. And therefore, as Protagoras says, "To myself I am the judge of what is and what is not." Thus the flux of Homer and Heracleitus, the great Protagorean saying that "Man is the measure of all things," the doctrine of Theaetetus that "Knowledge is perception," have all the same meaning. And this is thy new-born child, which by my art I have brought to light; and you must not be angry if instead of rearing your infant we expose him.'

\par  'Theaetetus will not be angry,' says Theodorus; 'he is very good-natured. But I should like to know, Socrates, whether you mean to say that all this is untrue?'

\par  'First reminding you that I am not the bag which contains the arguments, but that I extract them from Theaetetus, shall I tell you what amazes me in your friend Protagoras?'

\par  'What may that be?'

\par  'I like his doctrine that what appears is; but I wonder that he did not begin his great work on Truth with a declaration that a pig, or a dog-faced baboon, or any other monster which has sensation, is a measure of all things; then, while we were reverencing him as a god, he might have produced a magnificent effect by expounding to us that he was no wiser than a tadpole. For if sensations are always true, and one man's discernment is as good as another's, and every man is his own judge, and everything that he judges is right and true, then what need of Protagoras to be our instructor at a high figure; and why should we be less knowing than he is, or have to go to him, if every man is the measure of all things? My own art of midwifery, and all dialectic, is an enormous folly, if Protagoras' "Truth" be indeed truth, and the philosopher is not merely amusing himself by giving oracles out of his book.'

\par  Theodorus thinks that Socrates is unjust to his master, Protagoras; but he is too old and stiff to try a fall with him, and therefore refers him to Theaetetus, who is already driven out of his former opinion by the arguments of Socrates.

\par  Socrates then takes up the defence of Protagoras, who is supposed to reply in his own person—'Good people, you sit and declaim about the gods, of whose existence or non-existence I have nothing to say, or you discourse about man being reduced to the level of the brutes; but what proof have you of your statements? And yet surely you and Theodorus had better reflect whether probability is a safe guide. Theodorus would be a bad geometrician if he had nothing better to offer. '...Theaetetus is affected by the appeal to geometry, and Socrates is induced by him to put the question in a new form. He proceeds as follows:—'Should we say that we know what we see and hear,—e.g. the sound of words or the sight of letters in a foreign tongue?'

\par  'We should say that the figures of the letters, and the pitch of the voice in uttering them, were known to us, but not the meaning of them.'

\par  'Excellent; I want you to grow, and therefore I will leave that answer and ask another question: Is not seeing perceiving?' 'Very true.' 'And he who sees knows?' 'Yes.' 'And he who remembers, remembers that which he sees and knows?' 'Very true.' 'But if he closes his eyes, does he not remember?' 'He does.' 'Then he may remember and not see; and if seeing is knowing, he may remember and not know. Is not this a "reductio ad absurdum" of the hypothesis that knowledge is sensible perception? Yet perhaps we are crowing too soon; and if Protagoras, "the father of the myth," had been alive, the result might have been very different. But he is dead, and Theodorus, whom he left guardian of his "orphan," has not been very zealous in defending him.'

\par  Theodorus objects that Callias is the true guardian, but he hopes that Socrates will come to the rescue. Socrates prefaces his defence by resuming the attack. He asks whether a man can know and not know at the same time? 'Impossible.' Quite possible, if you maintain that seeing is knowing. The confident adversary, suiting the action to the word, shuts one of your eyes; and now, says he, you see and do not see, but do you know and not know? And a fresh opponent darts from his ambush, and transfers to knowledge the terms which are commonly applied to sight. He asks whether you can know near and not at a distance; whether you can have a sharp and also a dull knowledge. While you are wondering at his incomparable wisdom, he gets you into his power, and you will not escape until you have come to an understanding with him about the money which is to be paid for your release.

\par  But Protagoras has not yet made his defence; and already he may be heard contemptuously replying that he is not responsible for the admissions which were made by a boy, who could not foresee the coming move, and therefore had answered in a manner which enabled Socrates to raise a laugh against himself. 'But I cannot be fairly charged,' he will say, 'with an answer which I should not have given; for I never maintained that the memory of a feeling is the same as a feeling, or denied that a man might know and not know the same thing at the same time. Or, if you will have extreme precision, I say that man in different relations is many or rather infinite in number. And I challenge you, either to show that his perceptions are not individual, or that if they are, what appears to him is not what is. As to your pigs and baboons, you are yourself a pig, and you make my writings a sport of other swine. But I still affirm that man is the measure of all things, although I admit that one man may be a thousand times better than another, in proportion as he has better impressions. Neither do I deny the existence of wisdom or of the wise man. But I maintain that wisdom is a practical remedial power of turning evil into good, the bitterness of disease into the sweetness of health, and does not consist in any greater truth or superior knowledge. For the impressions of the sick are as true as the impressions of the healthy; and the sick are as wise as the healthy. Nor can any man be cured of a false opinion, for there is no such thing; but he may be cured of the evil habit which generates in him an evil opinion. This is effected in the body by the drugs of the physician, and in the soul by the words of the Sophist; and the new state or opinion is not truer, but only better than the old. And philosophers are not tadpoles, but physicians and husbandmen, who till the soil and infuse health into animals and plants, and make the good take the place of the evil, both in individuals and states. Wise and good rhetoricians make the good to appear just in states (for that is just which appears just to a state), and in return, they deserve to be well paid. And you, Socrates, whether you please or not, must continue to be a measure. This is my defence, and I must request you to meet me fairly. We are professing to reason, and not merely to dispute; and there is a great difference between reasoning and disputation. For the disputer is always seeking to trip up his opponent; and this is a mode of argument which disgusts men with philosophy as they grow older. But the reasoner is trying to understand him and to point out his errors to him, whether arising from his own or from his companion's fault; he does not argue from the customary use of names, which the vulgar pervert in all manner of ways. If you are gentle to an adversary he will follow and love you; and if defeated he will lay the blame on himself, and seek to escape from his own prejudices into philosophy. I would recommend you, Socrates, to adopt this humaner method, and to avoid captious and verbal criticisms.'

\par  Such, Theodorus, is the very slight help which I am able to afford to your friend; had he been alive, he would have helped himself in far better style.

\par  'You have made a most valorous defence.'

\par  Yes; but did you observe that Protagoras bade me be serious, and complained of our getting up a laugh against him with the aid of a boy? He meant to intimate that you must take the place of Theaetetus, who may be wiser than many bearded men, but not wiser than you, Theodorus.

\par  'The rule of the Spartan Palaestra is, Strip or depart; but you are like the giant Antaeus, and will not let me depart unless I try a fall with you.'

\par  Yes, that is the nature of my complaint. And many a Hercules, many a Theseus mighty in deeds and words has broken my head; but I am always at this rough game. Please, then, to favour me.

\par  'On the condition of not exceeding a single fall, I consent.'

\par  Socrates now resumes the argument. As he is very desirous of doing justice to Protagoras, he insists on citing his own words,—'What appears to each man is to him.' And how, asks Socrates, are these words reconcileable with the fact that all mankind are agreed in thinking themselves wiser than others in some respects, and inferior to them in others? In the hour of danger they are ready to fall down and worship any one who is their superior in wisdom as if he were a god. And the world is full of men who are asking to be taught and willing to be ruled, and of other men who are willing to rule and teach them. All which implies that men do judge of one another's impressions, and think some wise and others foolish. How will Protagoras answer this argument? For he cannot say that no one deems another ignorant or mistaken. If you form a judgment, thousands and tens of thousands are ready to maintain the opposite. The multitude may not and do not agree in Protagoras' own thesis that 'Man is the measure of all things;' and then who is to decide? Upon his own showing must not his 'truth' depend on the number of suffrages, and be more or less true in proportion as he has more or fewer of them? And he must acknowledge further, that they speak truly who deny him to speak truly, which is a famous jest. And if he admits that they speak truly who deny him to speak truly, he must admit that he himself does not speak truly. But his opponents will refuse to admit this of themselves, and he must allow that they are right in their refusal. The conclusion is, that all mankind, including Protagoras himself, will deny that he speaks truly; and his truth will be true neither to himself nor to anybody else.

\par  Theodorus is inclined to think that this is going too far. Socrates ironically replies, that he is not going beyond the truth. But if the old Protagoras could only pop his head out of the world below, he would doubtless give them both a sound castigation and be off to the shades in an instant. Seeing that he is not within call, we must examine the question for ourselves. It is clear that there are great differences in the understandings of men. Admitting, with Protagoras, that immediate sensations of hot, cold, and the like, are to each one such as they appear, yet this hypothesis cannot be extended to judgments or opinions. And even if we were to admit further,—and this is the view of some who are not thorough-going followers of Protagoras,—that right and wrong, holy and unholy, are to each state or individual such as they appear, still Protagoras will not venture to maintain that every man is equally the measure of expediency, or that the thing which seems is expedient to every one. But this begins a new question. 'Well, Socrates, we have plenty of leisure. Yes, we have, and, after the manner of philosophers, we are digressing; I have often observed how ridiculous this habit of theirs makes them when they appear in court. 'What do you mean?' I mean to say that a philosopher is a gentleman, but a lawyer is a servant. The one can have his talk out, and wander at will from one subject to another, as the fancy takes him; like ourselves, he may be long or short, as he pleases. But the lawyer is always in a hurry; there is the clepsydra limiting his time, and the brief limiting his topics, and his adversary is standing over him and exacting his rights. He is a servant disputing about a fellow-servant before his master, who holds the cause in his hands; the path never diverges, and often the race is for his life. Such experiences render him keen and shrewd; he learns the arts of flattery, and is perfect in the practice of crooked ways; dangers have come upon him too soon, when the tenderness of youth was unable to meet them with truth and honesty, and he has resorted to counter-acts of dishonesty and falsehood, and become warped and distorted; without any health or freedom or sincerity in him he has grown up to manhood, and is or esteems himself to be a master of cunning. Such are the lawyers; will you have the companion picture of philosophers? or will this be too much of a digression?

\par  'Nay, Socrates, the argument is our servant, and not our master. Who is the judge or where is the spectator, having a right to control us?'

\par  I will describe the leaders, then: for the inferior sort are not worth the trouble. The lords of philosophy have not learned the way to the dicastery or ecclesia; they neither see nor hear the laws and votes of the state, written or recited; societies, whether political or festive, clubs, and singing maidens do not enter even into their dreams. And the scandals of persons or their ancestors, male and female, they know no more than they can tell the number of pints in the ocean. Neither are they conscious of their own ignorance; for they do not practise singularity in order to gain reputation, but the truth is, that the outer form of them only is residing in the city; the inner man, as Pindar says, is going on a voyage of discovery, measuring as with line and rule the things which are under and in the earth, interrogating the whole of nature, only not condescending to notice what is near them.

\par  'What do you mean, Socrates?'

\par  I will illustrate my meaning by the jest of the witty maid-servant, who saw Thales tumbling into a well, and said of him, that he was so eager to know what was going on in heaven, that he could not see what was before his feet. This is applicable to all philosophers. The philosopher is unacquainted with the world; he hardly knows whether his neighbour is a man or an animal. For he is always searching into the essence of man, and enquiring what such a nature ought to do or suffer different from any other. Hence, on every occasion in private life and public, as I was saying, when he appears in a law-court or anywhere, he is the joke, not only of maid-servants, but of the general herd, falling into wells and every sort of disaster; he looks such an awkward, inexperienced creature, unable to say anything personal, when he is abused, in answer to his adversaries (for he knows no evil of any one); and when he hears the praises of others, he cannot help laughing from the bottom of his soul at their pretensions; and this also gives him a ridiculous appearance. A king or tyrant appears to him to be a kind of swine-herd or cow-herd, milking away at an animal who is much more troublesome and dangerous than cows or sheep; like the cow-herd, he has no time to be educated, and the pen in which he keeps his flock in the mountains is surrounded by a wall. When he hears of large landed properties of ten thousand acres or more, he thinks of the whole earth; or if he is told of the antiquity of a family, he remembers that every one has had myriads of progenitors, rich and poor, Greeks and barbarians, kings and slaves. And he who boasts of his descent from Amphitryon in the twenty-fifth generation, may, if he pleases, add as many more, and double that again, and our philosopher only laughs at his inability to do a larger sum. Such is the man at whom the vulgar scoff; he seems to them as if he could not mind his feet. 'That is very true, Socrates.' But when he tries to draw the quick-witted lawyer out of his pleas and rejoinders to the contemplation of absolute justice or injustice in their own nature, or from the popular praises of wealthy kings to the view of happiness and misery in themselves, or to the reasons why a man should seek after the one and avoid the other, then the situation is reversed; the little wretch turns giddy, and is ready to fall over the precipice; his utterance becomes thick, and he makes himself ridiculous, not to servant-maids, but to every man of liberal education. Such are the two pictures: the one of the philosopher and gentleman, who may be excused for not having learned how to make a bed, or cook up flatteries; the other, a serviceable knave, who hardly knows how to wear his cloak,—still less can he awaken harmonious thoughts or hymn virtue's praises.

\par  'If the world, Socrates, were as ready to receive your words as I am, there would be greater peace and less evil among mankind.'

\par  Evil, Theodorus, must ever remain in this world to be the antagonist of good, out of the way of the gods in heaven. Wherefore also we should fly away from ourselves to them; and to fly to them is to become like them; and to become like them is to become holy, just and true. But many live in the old wives' fable of appearances; they think that you should follow virtue in order that you may seem to be good. And yet the truth is, that God is righteous; and of men, he is most like him who is most righteous. To know this is wisdom; and in comparison of this the wisdom of the arts or the seeming wisdom of politicians is mean and common. The unrighteous man is apt to pride himself on his cunning; when others call him rogue, he says to himself: 'They only mean that I am one who deserves to live, and not a mere burden of the earth.' But he should reflect that his ignorance makes his condition worse than if he knew. For the penalty of injustice is not death or stripes, but the fatal necessity of becoming more and more unjust. Two patterns of life are set before him; the one blessed and divine, the other godless and wretched; and he is growing more and more like the one and unlike the other. He does not see that if he continues in his cunning, the place of innocence will not receive him after death. And yet if such a man has the courage to hear the argument out, he often becomes dissatisfied with himself, and has no more strength in him than a child.—But we have digressed enough.

\par  'For my part, Socrates, I like the digressions better than the argument, because I understand them better.'

\par  To return. When we left off, the Protagoreans and Heracliteans were maintaining that the ordinances of the State were just, while they lasted. But no one would maintain that the laws of the State were always good or expedient, although this may be the intention of them. For the expedient has to do with the future, about which we are liable to mistake. Now, would Protagoras maintain that man is the measure not only of the present and past, but of the future; and that there is no difference in the judgments of men about the future? Would an untrained man, for example, be as likely to know when he is going to have a fever, as the physician who attended him? And if they differ in opinion, which of them is likely to be right; or are they both right? Is not a vine-grower a better judge of a vintage which is not yet gathered, or a cook of a dinner which is in preparation, or Protagoras of the probable effect of a speech than an ordinary person? The last example speaks 'ad hominen.' For Protagoras would never have amassed a fortune if every man could judge of the future for himself. He is, therefore, compelled to admit that he is a measure; but I, who know nothing, am not equally convinced that I am. This is one way of refuting him; and he is refuted also by the authority which he attributes to the opinions of others, who deny his opinions. I am not equally sure that we can disprove the truth of immediate states of feeling. But this leads us to the doctrine of the universal flux, about which a battle-royal is always going on in the cities of Ionia. 'Yes; the Ephesians are downright mad about the flux; they cannot stop to argue with you, but are in perpetual motion, obedient to their text-books. Their restlessness is beyond expression, and if you ask any of them a question, they will not answer, but dart at you some unintelligible saying, and another and another, making no way either with themselves or with others; for nothing is fixed in them or their ideas,—they are at war with fixed principles.' I suppose, Theodorus, that you have never seen them in time of peace, when they discourse at leisure to their disciples? 'Disciples! they have none; they are a set of uneducated fanatics, and each of them says of the other that they have no knowledge. We must trust to ourselves, and not to them for the solution of the problem.' Well, the doctrine is old, being derived from the poets, who speak in a figure of Oceanus and Tethys; the truth was once concealed, but is now revealed by the superior wisdom of a later generation, and made intelligible to the cobbler, who, on hearing that all is in motion, and not some things only, as he ignorantly fancied, may be expected to fall down and worship his teachers. And the opposite doctrine must not be forgotten:—
 
\par  as Parmenides affirms. Thus we are in the midst of the fray; both parties are dragging us to their side; and we are not certain which of them are in the right; and if neither, then we shall be in a ridiculous position, having to set up our own opinion against ancient and famous men.

\par  Let us first approach the river-gods, or patrons of the flux.

\par  When they speak of motion, must they not include two kinds of motion, change of place and change of nature?—And all things must be supposed to have both kinds of motion; for if not, the same things would be at rest and in motion, which is contrary to their theory. And did we not say, that all sensations arise thus: they move about between the agent and patient together with a perception, and the patient ceases to be a perceiving power and becomes a percipient, and the agent a quale instead of a quality; but neither has any absolute existence? But now we make the further discovery, that neither white or whiteness, nor any sense or sensation, can be predicated of anything, for they are in a perpetual flux. And therefore we must modify the doctrine of Theaetetus and Protagoras, by asserting further that knowledge is and is not sensation; and of everything we must say equally, that this is and is not, or becomes or becomes not. And still the word 'this' is not quite correct, for language fails in the attempt to express their meaning.

\par  At the close of the discussion, Theodorus claims to be released from the argument, according to his agreement. But Theaetetus insists that they shall proceed to consider the doctrine of rest. This is declined by Socrates, who has too much reverence for the great Parmenides lightly to attack him. (We shall find that he returns to the doctrine of rest in the Sophist; but at present he does not wish to be diverted from his main purpose, which is, to deliver Theaetetus of his conception of knowledge.) He proceeds to interrogate him further. When he says that 'knowledge is in perception,' with what does he perceive? The first answer is, that he perceives sights with the eye, and sounds with the ear. This leads Socrates to make the reflection that nice distinctions of words are sometimes pedantic, but sometimes necessary; and he proposes in this case to substitute the word 'through' for 'with.' For the senses are not like the Trojan warriors in the horse, but have a common centre of perception, in which they all meet. This common principle is able to compare them with one another, and must therefore be distinct from them (compare Republic). And as there are facts of sense which are perceived through the organs of the body, there are also mathematical and other abstractions, such as sameness and difference, likeness and unlikeness, which the soul perceives by herself. Being is the most universal of these abstractions. The good and the beautiful are abstractions of another kind, which exist in relation and which above all others the mind perceives in herself, comparing within her past, present, and future. For example; we know a thing to be hard or soft by the touch, of which the perception is given at birth to men and animals. But the essence of hardness or softness, or the fact that this hardness is, and is the opposite of softness, is slowly learned by reflection and experience. Mere perception does not reach being, and therefore fails of truth; and therefore has no share in knowledge. But if so, knowledge is not perception. What then is knowledge? The mind, when occupied by herself with being, is said to have opinion—shall we say that 'Knowledge is true opinion'? But still an old difficulty recurs; we ask ourselves, 'How is false opinion possible?' This difficulty may be stated as follows:—

\par  Either we know or do not know a thing (for the intermediate processes of learning and forgetting need not at present be considered); and in thinking or having an opinion, we must either know or not know that which we think, and we cannot know and be ignorant at the same time; we cannot confuse one thing which we do not know, with another thing which we do not know; nor can we think that which we do not know to be that which we know, or that which we know to be that which we do not know. And what other case is conceivable, upon the supposition that we either know or do not know all things? Let us try another answer in the sphere of being: 'When a man thinks, and thinks that which is not.' But would this hold in any parallel case? Can a man see and see nothing? or hear and hear nothing? or touch and touch nothing? Must he not see, hear, or touch some one existing thing? For if he thinks about nothing he does not think, and not thinking he cannot think falsely. And so the path of being is closed against us, as well as the path of knowledge. But may there not be 'heterodoxy,' or transference of opinion;—I mean, may not one thing be supposed to be another? Theaetetus is confident that this must be 'the true falsehood,' when a man puts good for evil or evil for good. Socrates will not discourage him by attacking the paradoxical expression 'true falsehood,' but passes on. The new notion involves a process of thinking about two things, either together or alternately. And thinking is the conversing of the mind with herself, which is carried on in question and answer, until she no longer doubts, but determines and forms an opinion. And false opinion consists in saying to yourself, that one thing is another. But did you ever say to yourself, that good is evil, or evil good? Even in sleep, did you ever imagine that odd was even? Or did any man in his senses ever fancy that an ox was a horse, or that two are one? So that we can never think one thing to be another; for you must not meet me with the verbal quibble that one—eteron—is other—eteron (both 'one' and 'other' in Greek are called 'other'—eteron). He who has both the two things in his mind, cannot misplace them; and he who has only one of them in his mind, cannot misplace them—on either supposition transplacement is inconceivable.

\par  But perhaps there may still be a sense in which we can think that which we do not know to be that which we know: e.g. Theaetetus may know Socrates, but at a distance he may mistake another person for him. This process may be conceived by the help of an image. Let us suppose that every man has in his mind a block of wax of various qualities, the gift of Memory, the mother of the Muses; and on this he receives the seal or stamp of those sensations and perceptions which he wishes to remember. That which he succeeds in stamping is remembered and known by him as long as the impression lasts; but that, of which the impression is rubbed out or imperfectly made, is forgotten, and not known. No one can think one thing to be another, when he has the memorial or seal of both of these in his soul, and a sensible impression of neither; or when he knows one and does not know the other, and has no memorial or seal of the other; or when he knows neither; or when he perceives both, or one and not the other, or neither; or when he perceives and knows both, and identifies what he perceives with what he knows (this is still more impossible); or when he does not know one, and does not know and does not perceive the other; or does not perceive one, and does not know and does not perceive the other; or has no perception or knowledge of either—all these cases must be excluded. But he may err when he confuses what he knows or perceives, or what he perceives and does not know, with what he knows, or what he knows and perceives with what he knows and perceives.

\par  Theaetetus is unable to follow these distinctions; which Socrates proceeds to illustrate by examples, first of all remarking, that knowledge may exist without perception, and perception without knowledge. I may know Theodorus and Theaetetus and not see them; I may see them, and not know them. 'That I understand.' But I could not mistake one for the other if I knew you both, and had no perception of either; or if I knew one only, and perceived neither; or if I knew and perceived neither, or in any other of the excluded cases. The only possibility of error is: 1st, when knowing you and Theodorus, and having the impression of both of you on the waxen block, I, seeing you both imperfectly and at a distance, put the foot in the wrong shoe—that is to say, put the seal or stamp on the wrong object: or 2ndly, when knowing both of you I only see one; or when, seeing and knowing you both, I fail to identify the impression and the object. But there could be no error when perception and knowledge correspond.

\par  The waxen block in the heart of a man's soul, as I may say in the words of Homer, who played upon the words ker and keros, may be smooth and deep, and large enough, and then the signs are clearly marked and lasting, and do not get confused. But in the 'hairy heart,' as the all-wise poet sings, when the wax is muddy or hard or moist, there is a corresponding confusion and want of retentiveness; in the muddy and impure there is indistinctness, and still more in the hard, for there the impressions have no depth of wax, and in the moist they are too soon effaced. Yet greater is the indistinctness when they are all jolted together in a little soul, which is narrow and has no room. These are the sort of natures which have false opinion; from stupidity they see and hear and think amiss; and this is falsehood and ignorance. Error, then, is a confusion of thought and sense.

\par  Theaetetus is delighted with this explanation. But Socrates has no sooner found the new solution than he sinks into a fit of despondency. For an objection occurs to him:—May there not be errors where there is no confusion of mind and sense? e.g. in numbers. No one can confuse the man whom he has in his thoughts with the horse which he has in his thoughts, but he may err in the addition of five and seven. And observe that these are purely mental conceptions. Thus we are involved once more in the dilemma of saying, either that there is no such thing as false opinion, or that a man knows what he does not know.

\par  We are at our wit's end, and may therefore be excused for making a bold diversion. All this time we have been repeating the words 'know,' 'understand,' yet we do not know what knowledge is. 'Why, Socrates, how can you argue at all without using them?' Nay, but the true hero of dialectic would have forbidden me to use them until I had explained them. And I must explain them now. The verb 'to know' has two senses, to have and to possess knowledge, and I distinguish 'having' from 'possessing.' A man may possess a garment which he does not wear; or he may have wild birds in an aviary; these in one sense he possesses, and in another he has none of them. Let this aviary be an image of the mind, as the waxen block was; when we are young, the aviary is empty; after a time the birds are put in; for under this figure we may describe different forms of knowledge;—there are some of them in groups, and some single, which are flying about everywhere; and let us suppose a hunt after the science of odd and even, or some other science. The possession of the birds is clearly not the same as the having them in the hand. And the original chase of them is not the same as taking them in the hand when they are already caged.

\par  This distinction between use and possession saves us from the absurdity of supposing that we do not know what we know, because we may know in one sense, i.e. possess, what we do not know in another, i.e. use. But have we not escaped one difficulty only to encounter a greater? For how can the exchange of two kinds of knowledge ever become false opinion? As well might we suppose that ignorance could make a man know, or that blindness could make him see. Theaetetus suggests that in the aviary there may be flying about mock birds, or forms of ignorance, and we put forth our hands and grasp ignorance, when we are intending to grasp knowledge. But how can he who knows the forms of knowledge and the forms of ignorance imagine one to be the other? Is there some other form of knowledge which distinguishes them? and another, and another? Thus we go round and round in a circle and make no progress.

\par  All this confusion arises out of our attempt to explain false opinion without having explained knowledge. What then is knowledge? Theaetetus repeats that knowledge is true opinion. But this seems to be refuted by the instance of orators and judges. For surely the orator cannot convey a true knowledge of crimes at which the judges were not present; he can only persuade them, and the judge may form a true opinion and truly judge. But if true opinion were knowledge they could not have judged without knowledge.

\par  Once more. Theaetetus offers a definition which he has heard: Knowledge is true opinion accompanied by definition or explanation. Socrates has had a similar dream, and has further heard that the first elements are names only, and that definition or explanation begins when they are combined; the letters are unknown, the syllables or combinations are known. But this new hypothesis when tested by the letters of the alphabet is found to break down. The first syllable of Socrates' name is SO. But what is SO? Two letters, S and O, a sibilant and a vowel, of which no further explanation can be given. And how can any one be ignorant of either of them, and yet know both of them? There is, however, another alternative:—We may suppose that the syllable has a separate form or idea distinct from the letters or parts. The all of the parts may not be the whole. Theaetetus is very much inclined to adopt this suggestion, but when interrogated by Socrates he is unable to draw any distinction between the whole and all the parts. And if the syllables have no parts, then they are those original elements of which there is no explanation. But how can the syllable be known if the letter remains unknown? In learning to read as children, we are first taught the letters and then the syllables. And in music, the notes, which are the letters, have a much more distinct meaning to us than the combination of them.

\par  Once more, then, we must ask the meaning of the statement, that 'Knowledge is right opinion, accompanied by explanation or definition.' Explanation may mean, (1) the reflection or expression of a man's thoughts—but every man who is not deaf and dumb is able to express his thoughts—or (2) the enumeration of the elements of which anything is composed. A man may have a true opinion about a waggon, but then, and then only, has he knowledge of a waggon when he is able to enumerate the hundred planks of Hesiod. Or he may know the syllables of the name Theaetetus, but not the letters; yet not until he knows both can he be said to have knowledge as well as opinion. But on the other hand he may know the syllable 'The' in the name Theaetetus, yet he may be mistaken about the same syllable in the name Theodorus, and in learning to read we often make such mistakes. And even if he could write out all the letters and syllables of your name in order, still he would only have right opinion. Yet there may be a third meaning of the definition, besides the image or expression of the mind, and the enumeration of the elements, viz. (3) perception of difference.

\par  For example, I may see a man who has eyes, nose, and mouth;—that will not distinguish him from any other man. Or he may have a snub-nose and prominent eyes;—that will not distinguish him from myself and you and others who are like me. But when I see a certain kind of snub-nosedness, then I recognize Theaetetus. And having this sign of difference, I have knowledge. But have I knowledge or opinion of this difference; if I have only opinion I have not knowledge; if I have knowledge we assume a disputed term; for knowledge will have to be defined as right opinion with knowledge of difference.

\par  And so, Theaetetus, knowledge is neither perception nor true opinion, nor yet definition accompanying true opinion. And I have shown that the children of your brain are not worth rearing. Are you still in labour, or have you brought all you have to say about knowledge to the birth? If you have any more thoughts, you will be the better for having got rid of these; or if you have none, you will be the better for not fancying that you know what you do not know. Observe the limits of my art, which, like my mother's, is an art of midwifery; I do not pretend to compare with the good and wise of this and other ages.

\par  And now I go to meet Meletus at the porch of the King Archon; but to-morrow I shall hope to see you again, Theodorus, at this place.

\par  ...

\par  I. The saying of Theaetetus, that 'Knowledge is sensible perception,' may be assumed to be a current philosophical opinion of the age. 'The ancients,' as Aristotle (De Anim.) says, citing a verse of Empedocles, 'affirmed knowledge to be the same as perception.' We may now examine these words, first, with reference to their place in the history of philosophy, and secondly, in relation to modern speculations.

\par  (a) In the age of Socrates the mind was passing from the object to the subject. The same impulse which a century before had led men to form conceptions of the world, now led them to frame general notions of the human faculties and feelings, such as memory, opinion, and the like. The simplest of these is sensation, or sensible perception, by which Plato seems to mean the generalized notion of feelings and impressions of sense, without determining whether they are conscious or not.

\par  The theory that 'Knowledge is sensible perception' is the antithesis of that which derives knowledge from the mind (Theaet. ), or which assumes the existence of ideas independent of the mind (Parm.). Yet from their extreme abstraction these theories do not represent the opposite poles of thought in the same way that the corresponding differences would in modern philosophy. The most ideal and the most sensational have a tendency to pass into one another; Heracleitus, like his great successor Hegel, has both aspects. The Eleatic isolation of Being and the Megarian or Cynic isolation of individuals are placed in the same class by Plato (Soph. ); and the same principle which is the symbol of motion to one mind is the symbol of rest to another. The Atomists, who are sometimes regarded as the Materialists of Plato, denied the reality of sensation. And in the ancient as well as the modern world there were reactions from theory to experience, from ideas to sense. This is a point of view from which the philosophy of sensation presented great attraction to the ancient thinker. Amid the conflict of ideas and the variety of opinions, the impression of sense remained certain and uniform. Hardness, softness, cold, heat, etc. are not absolutely the same to different persons, but the art of measuring could at any rate reduce them all to definite natures (Republic). Thus the doctrine that knowledge is perception supplies or seems to supply a firm standing ground. Like the other notions of the earlier Greek philosophy, it was held in a very simple way, without much basis of reasoning, and without suggesting the questions which naturally arise in our own minds on the same subject.

\par  (b) The fixedness of impressions of sense furnishes a link of connexion between ancient and modern philosophy. The modern thinker often repeats the parallel axiom, 'All knowledge is experience.' He means to say that the outward and not the inward is both the original source and the final criterion of truth, because the outward can be observed and analyzed; the inward is only known by external results, and is dimly perceived by each man for himself. In what does this differ from the saying of Theaetetus? Chiefly in this—that the modern term 'experience,' while implying a point of departure in sense and a return to sense, also includes all the processes of reasoning and imagination which have intervened. The necessary connexion between them by no means affords a measure of the relative degree of importance which is to be ascribed to either element. For the inductive portion of any science may be small, as in mathematics or ethics, compared with that which the mind has attained by reasoning and reflection on a very few facts.

\par  II. The saying that 'All knowledge is sensation' is identified by Plato with the Protagorean thesis that 'Man is the measure of all things.' The interpretation which Protagoras himself is supposed to give of these latter words is: 'Things are to me as they appear to me, and to you as they appear to you.' But there remains still an ambiguity both in the text and in the explanation, which has to be cleared up. Did Protagoras merely mean to assert the relativity of knowledge to the human mind? Or did he mean to deny that there is an objective standard of truth?

\par  These two questions have not been always clearly distinguished; the relativity of knowledge has been sometimes confounded with uncertainty. The untutored mind is apt to suppose that objects exist independently of the human faculties, because they really exist independently of the faculties of any individual. In the same way, knowledge appears to be a body of truths stored up in books, which when once ascertained are independent of the discoverer. Further consideration shows us that these truths are not really independent of the mind; there is an adaptation of one to the other, of the eye to the object of sense, of the mind to the conception. There would be no world, if there neither were nor ever had been any one to perceive the world. A slight effort of reflection enables us to understand this; but no effort of reflection will enable us to pass beyond the limits of our own faculties, or to imagine the relation or adaptation of objects to the mind to be different from that of which we have experience. There are certain laws of language and logic to which we are compelled to conform, and to which our ideas naturally adapt themselves; and we can no more get rid of them than we can cease to be ourselves. The absolute and infinite, whether explained as self-existence, or as the totality of human thought, or as the Divine nature, if known to us at all, cannot escape from the category of relation.

\par  But because knowledge is subjective or relative to the mind, we are not to suppose that we are therefore deprived of any of the tests or criteria of truth. One man still remains wiser than another, a more accurate observer and relater of facts, a truer measure of the proportions of knowledge. The nature of testimony is not altered, nor the verification of causes by prescribed methods less certain. Again, the truth must often come to a man through others, according to the measure of his capacity and education. But neither does this affect the testimony, whether written or oral, which he knows by experience to be trustworthy. He cannot escape from the laws of his own mind; and he cannot escape from the further accident of being dependent for his knowledge on others. But still this is no reason why he should always be in doubt; of many personal, of many historical and scientific facts he may be absolutely assured. And having such a mass of acknowledged truth in the mathematical and physical, not to speak of the moral sciences, the moderns have certainly no reason to acquiesce in the statement that truth is appearance only, or that there is no difference between appearance and truth.

\par  The relativity of knowledge is a truism to us, but was a great psychological discovery in the fifth century before Christ. Of this discovery, the first distinct assertion is contained in the thesis of Protagoras. Probably he had no intention either of denying or affirming an objective standard of truth. He did not consider whether man in the higher or man in the lower sense was a 'measure of all things.' Like other great thinkers, he was absorbed with one idea, and that idea was the absoluteness of perception. Like Socrates, he seemed to see that philosophy must be brought back from 'nature' to 'truth,' from the world to man. But he did not stop to analyze whether he meant 'man' in the concrete or man in the abstract, any man or some men, 'quod semper quod ubique' or individual private judgment. Such an analysis lay beyond his sphere of thought; the age before Socrates had not arrived at these distinctions. Like the Cynics, again, he discarded knowledge in any higher sense than perception. For 'truer' or 'wiser' he substituted the word 'better,' and is not unwilling to admit that both states and individuals are capable of practical improvement. But this improvement does not arise from intellectual enlightenment, nor yet from the exertion of the will, but from a change of circumstances and impressions; and he who can effect this change in himself or others may be deemed a philosopher. In the mode of effecting it, while agreeing with Socrates and the Cynics in the importance which he attaches to practical life, he is at variance with both of them. To suppose that practice can be divorced from speculation, or that we may do good without caring about truth, is by no means singular, either in philosophy or life. The singularity of this, as of some other (so-called) sophistical doctrines, is the frankness with which they are avowed, instead of being veiled, as in modern times, under ambiguous and convenient phrases.

\par  Plato appears to treat Protagoras much as he himself is treated by Aristotle; that is to say, he does not attempt to understand him from his own point of view. But he entangles him in the meshes of a more advanced logic. To which Protagoras is supposed to reply by Megarian quibbles, which destroy logic, 'Not only man, but each man, and each man at each moment.' In the arguments about sight and memory there is a palpable unfairness which is worthy of the great 'brainless brothers,' Euthydemus and Dionysodorus, and may be compared with the egkekalummenos ('obvelatus') of Eubulides. For he who sees with one eye only cannot be truly said both to see and not to see; nor is memory, which is liable to forget, the immediate knowledge to which Protagoras applies the term. Theodorus justly charges Socrates with going beyond the truth; and Protagoras has equally right on his side when he protests against Socrates arguing from the common use of words, which 'the vulgar pervert in all manner of ways.'

\par  III. The theory of Protagoras is connected by Aristotle as well as Plato with the flux of Heracleitus. But Aristotle is only following Plato, and Plato, as we have already seen, did not mean to imply that such a connexion was admitted by Protagoras himself. His metaphysical genius saw or seemed to see a common tendency in them, just as the modern historian of ancient philosophy might perceive a parallelism between two thinkers of which they were probably unconscious themselves. We must remember throughout that Plato is not speaking of Heracleitus, but of the Heracliteans, who succeeded him; nor of the great original ideas of the master, but of the Eristic into which they had degenerated a hundred years later. There is nothing in the fragments of Heracleitus which at all justifies Plato's account of him. His philosophy may be resolved into two elements—first, change, secondly, law or measure pervading the change: these he saw everywhere, and often expressed in strange mythological symbols. But he has no analysis of sensible perception such as Plato attributes to him; nor is there any reason to suppose that he pushed his philosophy into that absolute negation in which Heracliteanism was sunk in the age of Plato. He never said that 'change means every sort of change;' and he expressly distinguished between 'the general and particular understanding.' Like a poet, he surveyed the elements of mythology, nature, thought, which lay before him, and sometimes by the light of genius he saw or seemed to see a mysterious principle working behind them. But as has been the case with other great philosophers, and with Plato and Aristotle themselves, what was really permanent and original could not be understood by the next generation, while a perverted logic carried out his chance expressions with an illogical consistency. His simple and noble thoughts, like those of the great Eleatic, soon degenerated into a mere strife of words. And when thus reduced to mere words, they seem to have exercised a far wider influence in the cities of Ionia (where the people 'were mad about them') than in the life-time of Heracleitus—a phenomenon which, though at first sight singular, is not without a parallel in the history of philosophy and theology.

\par  It is this perverted form of the Heraclitean philosophy which is supposed to effect the final overthrow of Protagorean sensationalism. For if all things are changing at every moment, in all sorts of ways, then there is nothing fixed or defined at all, and therefore no sensible perception, nor any true word by which that or anything else can be described. Of course Protagoras would not have admitted the justice of this argument any more than Heracleitus would have acknowledged the 'uneducated fanatics' who appealed to his writings. He might have said, 'The excellent Socrates has first confused me with Heracleitus, and Heracleitus with his Ephesian successors, and has then disproved the existence both of knowledge and sensation. But I am not responsible for what I never said, nor will I admit that my common-sense account of knowledge can be overthrown by unintelligible Heraclitean paradoxes.'

\par  IV. Still at the bottom of the arguments there remains a truth, that knowledge is something more than sensible perception;—this alone would not distinguish man from a tadpole. The absoluteness of sensations at each moment destroys the very consciousness of sensations (compare Phileb. ), or the power of comparing them. The senses are not mere holes in a 'Trojan horse,' but the organs of a presiding nature, in which they meet. A great advance has been made in psychology when the senses are recognized as organs of sense, and we are admitted to see or feel 'through them' and not 'by them,' a distinction of words which, as Socrates observes, is by no means pedantic. A still further step has been made when the most abstract notions, such as Being and Not-being, sameness and difference, unity and plurality, are acknowledged to be the creations of the mind herself, working upon the feelings or impressions of sense. In this manner Plato describes the process of acquiring them, in the words 'Knowledge consists not in the feelings or affections (pathemasi), but in the process of reasoning about them (sullogismo).' Here, is in the Parmenides, he means something not really different from generalization. As in the Sophist, he is laying the foundation of a rational psychology, which is to supersede the Platonic reminiscence of Ideas as well as the Eleatic Being and the individualism of Megarians and Cynics.

\par  V. Having rejected the doctrine that 'Knowledge is perception,' we now proceed to look for a definition of knowledge in the sphere of opinion. But here we are met by a singular difficulty: How is false opinion possible? For we must either know or not know that which is presented to the mind or to sense. We of course should answer at once: 'No; the alternative is not necessary, for there may be degrees of knowledge; and we may know and have forgotten, or we may be learning, or we may have a general but not a particular knowledge, or we may know but not be able to explain;' and many other ways may be imagined in which we know and do not know at the same time. But these answers belong to a later stage of metaphysical discussion; whereas the difficulty in question naturally arises owing to the childhood of the human mind, like the parallel difficulty respecting Not-being. Men had only recently arrived at the notion of opinion; they could not at once define the true and pass beyond into the false. The very word doxa was full of ambiguity, being sometimes, as in the Eleatic philosophy, applied to the sensible world, and again used in the more ordinary sense of opinion. There is no connexion between sensible appearance and probability, and yet both of them met in the word doxa, and could hardly be disengaged from one another in the mind of the Greek living in the fifth or fourth century B.C. To this was often added, as at the end of the fifth book of the Republic, the idea of relation, which is equally distinct from either of them; also a fourth notion, the conclusion of the dialectical process, the making up of the mind after she has been 'talking to herself' (Theat. ).

\par  We are not then surprised that the sphere of opinion and of Not-being should be a dusky, half-lighted place (Republic), belonging neither to the old world of sense and imagination, nor to the new world of reflection and reason. Plato attempts to clear up this darkness. In his accustomed manner he passes from the lower to the higher, without omitting the intermediate stages. This appears to be the reason why he seeks for the definition of knowledge first in the sphere of opinion. Hereafter we shall find that something more than opinion is required.

\par  False opinion is explained by Plato at first as a confusion of mind and sense, which arises when the impression on the mind does not correspond to the impression made on the senses. It is obvious that this explanation (supposing the distinction between impressions on the mind and impressions on the senses to be admitted) does not account for all forms of error; and Plato has excluded himself from the consideration of the greater number, by designedly omitting the intermediate processes of learning and forgetting; nor does he include fallacies in the use of language or erroneous inferences. But he is struck by one possibility of error, which is not covered by his theory, viz. errors in arithmetic. For in numbers and calculation there is no combination of thought and sense, and yet errors may often happen. Hence he is led to discard the explanation which might nevertheless have been supposed to hold good (for anything which he says to the contrary) as a rationale of error, in the case of facts derived from sense.

\par  Another attempt is made to explain false opinion by assigning to error a sort of positive existence. But error or ignorance is essentially negative—a not-knowing; if we knew an error, we should be no longer in error. We may veil our difficulty under figures of speech, but these, although telling arguments with the multitude, can never be the real foundation of a system of psychology. Only they lead us to dwell upon mental phenomena which if expressed in an abstract form would not be realized by us at all. The figure of the mind receiving impressions is one of those images which have rooted themselves for ever in language. It may or may not be a 'gracious aid' to thought; but it cannot be got rid of. The other figure of the enclosure is also remarkable as affording the first hint of universal all-pervading ideas,—a notion further carried out in the Sophist. This is implied in the birds, some in flocks, some solitary, which fly about anywhere and everywhere. Plato discards both figures, as not really solving the question which to us appears so simple: 'How do we make mistakes?' The failure of the enquiry seems to show that we should return to knowledge, and begin with that; and we may afterwards proceed, with a better hope of success, to the examination of opinion.

\par  But is true opinion really distinct from knowledge? The difference between these he seeks to establish by an argument, which to us appears singular and unsatisfactory. The existence of true opinion is proved by the rhetoric of the law courts, which cannot give knowledge, but may give true opinion. The rhetorician cannot put the judge or juror in possession of all the facts which prove an act of violence, but he may truly persuade them of the commission of such an act. Here the idea of true opinion seems to be a right conclusion from imperfect knowledge. But the correctness of such an opinion will be purely accidental; and is really the effect of one man, who has the means of knowing, persuading another who has not. Plato would have done better if he had said that true opinion was a contradiction in terms.

\par  Assuming the distinction between knowledge and opinion, Theaetetus, in answer to Socrates, proceeds to define knowledge as true opinion, with definite or rational explanation. This Socrates identifies with another and different theory, of those who assert that knowledge first begins with a proposition.

\par  The elements may be perceived by sense, but they are names, and cannot be defined. When we assign to them some predicate, they first begin to have a meaning (onomaton sumploke logou ousia). This seems equivalent to saying, that the individuals of sense become the subject of knowledge when they are regarded as they are in nature in relation to other individuals.

\par  Yet we feel a difficulty in following this new hypothesis. For must not opinion be equally expressed in a proposition? The difference between true and false opinion is not the difference between the particular and the universal, but between the true universal and the false. Thought may be as much at fault as sight. When we place individuals under a class, or assign to them attributes, this is not knowledge, but a very rudimentary process of thought; the first generalization of all, without which language would be impossible. And has Plato kept altogether clear of a confusion, which the analogous word logos tends to create, of a proposition and a definition? And is not the confusion increased by the use of the analogous term 'elements,' or 'letters'? For there is no real resemblance between the relation of letters to a syllable, and of the terms to a proposition.

\par  Plato, in the spirit of the Megarian philosophy, soon discovers a flaw in the explanation. For how can we know a compound of which the simple elements are unknown to us? Can two unknowns make a known? Can a whole be something different from the parts? The answer of experience is that they can; for we may know a compound, which we are unable to analyze into its elements; and all the parts, when united, may be more than all the parts separated: e.g. the number four, or any other number, is more than the units which are contained in it; any chemical compound is more than and different from the simple elements. But ancient philosophy in this, as in many other instances, proceeding by the path of mental analysis, was perplexed by doubts which warred against the plainest facts.

\par  Three attempts to explain the new definition of knowledge still remain to be considered. They all of them turn on the explanation of logos. The first account of the meaning of the word is the reflection of thought in speech—a sort of nominalism 'La science est une langue bien faite.' But anybody who is not dumb can say what he thinks; therefore mere speech cannot be knowledge. And yet we may observe, that there is in this explanation an element of truth which is not recognized by Plato; viz. that truth and thought are inseparable from language, although mere expression in words is not truth. The second explanation of logos is the enumeration of the elementary parts of the complex whole. But this is only definition accompanied with right opinion, and does not yet attain to the certainty of knowledge. Plato does not mention the greater objection, which is, that the enumeration of particulars is endless; such a definition would be based on no principle, and would not help us at all in gaining a common idea. The third is the best explanation,—the possession of a characteristic mark, which seems to answer to the logical definition by genus and difference. But this, again, is equally necessary for right opinion; and we have already determined, although not on very satisfactory grounds, that knowledge must be distinguished from opinion. A better distinction is drawn between them in the Timaeus. They might be opposed as philosophy and rhetoric, and as conversant respectively with necessary and contingent matter. But no true idea of the nature of either of them, or of their relation to one another, could be framed until science obtained a content. The ancient philosophers in the age of Plato thought of science only as pure abstraction, and to this opinion stood in no relation.

\par  Like Theaetetus, we have attained to no definite result. But an interesting phase of ancient philosophy has passed before us. And the negative result is not to be despised. For on certain subjects, and in certain states of knowledge, the work of negation or clearing the ground must go on, perhaps for a generation, before the new structure can begin to rise. Plato saw the necessity of combating the illogical logic of the Megarians and Eristics. For the completion of the edifice, he makes preparation in the Theaetetus, and crowns the work in the Sophist.

\par  Many (1) fine expressions, and (2) remarks full of wisdom, (3) also germs of a metaphysic of the future, are scattered up and down in the dialogue. Such, for example, as (1) the comparison of Theaetetus' progress in learning to the 'noiseless flow of a river of oil'; the satirical touch, 'flavouring a sauce or fawning speech'; or the remarkable expression, 'full of impure dialectic'; or the lively images under which the argument is described,—'the flood of arguments pouring in,' the fresh discussions 'bursting in like a band of revellers.' (2) As illustrations of the second head, may be cited the remark of Socrates, that 'distinctions of words, although sometimes pedantic, are also necessary'; or the fine touch in the character of the lawyer, that 'dangers came upon him when the tenderness of youth was unequal to them'; or the description of the manner in which the spirit is broken in a wicked man who listens to reproof until he becomes like a child; or the punishment of the wicked, which is not physical suffering, but the perpetual companionship of evil (compare Gorgias); or the saying, often repeated by Aristotle and others, that 'philosophy begins in wonder, for Iris is the child of Thaumas'; or the superb contempt with which the philosopher takes down the pride of wealthy landed proprietors by comparison of the whole earth. (3) Important metaphysical ideas are: a. the conception of thought, as the mind talking to herself; b. the notion of a common sense, developed further by Aristotle, and the explicit declaration, that the mind gains her conceptions of Being, sameness, number, and the like, from reflection on herself; c. the excellent distinction of Theaetetus (which Socrates, speaking with emphasis, 'leaves to grow') between seeing the forms or hearing the sounds of words in a foreign language, and understanding the meaning of them; and d. the distinction of Socrates himself between 'having' and 'possessing' knowledge, in which the answer to the whole discussion appears to be contained.

\par  ...

\par  There is a difference between ancient and modern psychology, and we have a difficulty in explaining one in the terms of the other. To us the inward and outward sense and the inward and outward worlds of which they are the organs are parted by a wall, and appear as if they could never be confounded. The mind is endued with faculties, habits, instincts, and a personality or consciousness in which they are bound together. Over against these are placed forms, colours, external bodies coming into contact with our own body. We speak of a subject which is ourselves, of an object which is all the rest. These are separable in thought, but united in any act of sensation, reflection, or volition. As there are various degrees in which the mind may enter into or be abstracted from the operations of sense, so there are various points at which this separation or union may be supposed to occur. And within the sphere of mind the analogy of sense reappears; and we distinguish not only external objects, but objects of will and of knowledge which we contrast with them. These again are comprehended in a higher object, which reunites with the subject. A multitude of abstractions are created by the efforts of successive thinkers which become logical determinations; and they have to be arranged in order, before the scheme of thought is complete. The framework of the human intellect is not the peculium of an individual, but the joint work of many who are of all ages and countries. What we are in mind is due, not merely to our physical, but to our mental antecedents which we trace in history, and more especially in the history of philosophy. Nor can mental phenomena be truly explained either by physiology or by the observation of consciousness apart from their history. They have a growth of their own, like the growth of a flower, a tree, a human being. They may be conceived as of themselves constituting a common mind, and having a sort of personal identity in which they coexist.

\par  So comprehensive is modern psychology, seeming to aim at constructing anew the entire world of thought. And prior to or simultaneously with this construction a negative process has to be carried on, a clearing away of useless abstractions which we have inherited from the past. Many erroneous conceptions of the mind derived from former philosophies have found their way into language, and we with difficulty disengage ourselves from them. Mere figures of speech have unconsciously influenced the minds of great thinkers. Also there are some distinctions, as, for example, that of the will and of the reason, and of the moral and intellectual faculties, which are carried further than is justified by experience. Any separation of things which we cannot see or exactly define, though it may be necessary, is a fertile source of error. The division of the mind into faculties or powers or virtues is too deeply rooted in language to be got rid of, but it gives a false impression. For if we reflect on ourselves we see that all our faculties easily pass into one another, and are bound together in a single mind or consciousness; but this mental unity is apt to be concealed from us by the distinctions of language.

\par  A profusion of words and ideas has obscured rather than enlightened mental science. It is hard to say how many fallacies have arisen from the representation of the mind as a box, as a 'tabula rasa,' a book, a mirror, and the like. It is remarkable how Plato in the Theaetetus, after having indulged in the figure of the waxen tablet and the decoy, afterwards discards them. The mind is also represented by another class of images, as the spring of a watch, a motive power, a breath, a stream, a succession of points or moments. As Plato remarks in the Cratylus, words expressive of motion as well as of rest are employed to describe the faculties and operations of the mind; and in these there is contained another store of fallacies. Some shadow or reflection of the body seems always to adhere to our thoughts about ourselves, and mental processes are hardly distinguished in language from bodily ones. To see or perceive are used indifferently of both; the words intuition, moral sense, common sense, the mind's eye, are figures of speech transferred from one to the other. And many other words used in early poetry or in sacred writings to express the works of mind have a materialistic sound; for old mythology was allied to sense, and the distinction of matter and mind had not as yet arisen. Thus materialism receives an illusive aid from language; and both in philosophy and religion the imaginary figure or association easily takes the place of real knowledge.

\par  Again, there is the illusion of looking into our own minds as if our thoughts or feelings were written down in a book. This is another figure of speech, which might be appropriately termed 'the fallacy of the looking-glass.' We cannot look at the mind unless we have the eye which sees, and we can only look, not into, but out of the mind at the thoughts, words, actions of ourselves and others. What we dimly recognize within us is not experience, but rather the suggestion of an experience, which we may gather, if we will, from the observation of the world. The memory has but a feeble recollection of what we were saying or doing a few weeks or a few months ago, and still less of what we were thinking or feeling. This is one among many reasons why there is so little self-knowledge among mankind; they do not carry with them the thought of what they are or have been. The so-called 'facts of consciousness' are equally evanescent; they are facts which nobody ever saw, and which can neither be defined nor described. Of the three laws of thought the first (All A = A) is an identical proposition—that is to say, a mere word or symbol claiming to be a proposition: the two others (Nothing can be A and not A, and Everything is either A or not A) are untrue, because they exclude degrees and also the mixed modes and double aspects under which truth is so often presented to us. To assert that man is man is unmeaning; to say that he is free or necessary and cannot be both is a half truth only. These are a few of the entanglements which impede the natural course of human thought. Lastly, there is the fallacy which lies still deeper, of regarding the individual mind apart from the universal, or either, as a self-existent entity apart from the ideas which are contained in them.

\par  In ancient philosophies the analysis of the mind is still rudimentary and imperfect. It naturally began with an effort to disengage the universal from sense—this was the first lifting up of the mist. It wavered between object and subject, passing imperceptibly from one or Being to mind and thought. Appearance in the outward object was for a time indistinguishable from opinion in the subject. At length mankind spoke of knowing as well as of opining or perceiving. But when the word 'knowledge' was found how was it to be explained or defined? It was not an error, it was a step in the right direction, when Protagoras said that 'Man is the measure of all things,' and that 'All knowledge is perception.' This was the subjective which corresponded to the objective 'All is flux.' But the thoughts of men deepened, and soon they began to be aware that knowledge was neither sense, nor yet opinion—with or without explanation; nor the expression of thought, nor the enumeration of parts, nor the addition of characteristic marks. Motion and rest were equally ill adapted to express its nature, although both must in some sense be attributed to it; it might be described more truly as the mind conversing with herself; the discourse of reason; the hymn of dialectic, the science of relations, of ideas, of the so-called arts and sciences, of the one, of the good, of the all:—this is the way along which Plato is leading us in his later dialogues. In its higher signification it was the knowledge, not of men, but of gods, perfect and all sufficing:—like other ideals always passing out of sight, and nevertheless present to the mind of Aristotle as well as Plato, and the reality to which they were both tending. For Aristotle as well as Plato would in modern phraseology have been termed a mystic; and like him would have defined the higher philosophy to be 'Knowledge of being or essence,'—words to which in our own day we have a difficulty in attaching a meaning.

\par  Yet, in spite of Plato and his followers, mankind have again and again returned to a sensational philosophy. As to some of the early thinkers, amid the fleetings of sensible objects, ideas alone seemed to be fixed, so to a later generation amid the fluctuation of philosophical opinions the only fixed points appeared to be outward objects. Any pretence of knowledge which went beyond them implied logical processes, of the correctness of which they had no assurance and which at best were only probable. The mind, tired of wandering, sought to rest on firm ground; when the idols of philosophy and language were stripped off, the perception of outward objects alone remained. The ancient Epicureans never asked whether the comparison of these with one another did not involve principles of another kind which were above and beyond them. In like manner the modern inductive philosophy forgot to enquire into the meaning of experience, and did not attempt to form a conception of outward objects apart from the mind, or of the mind apart from them. Soon objects of sense were merged in sensations and feelings, but feelings and sensations were still unanalyzed. At last we return to the doctrine attributed by Plato to Protagoras, that the mind is only a succession of momentary perceptions. At this point the modern philosophy of experience forms an alliance with ancient scepticism.

\par  The higher truths of philosophy and religion are very far removed from sense. Admitting that, like all other knowledge, they are derived from experience, and that experience is ultimately resolvable into facts which come to us through the eye and ear, still their origin is a mere accident which has nothing to do with their true nature. They are universal and unseen; they belong to all times—past, present, and future. Any worthy notion of mind or reason includes them. The proof of them is, 1st, their comprehensiveness and consistency with one another; 2ndly, their agreement with history and experience. But sensation is of the present only, is isolated, is and is not in successive moments. It takes the passing hour as it comes, following the lead of the eye or ear instead of the command of reason. It is a faculty which man has in common with the animals, and in which he is inferior to many of them. The importance of the senses in us is that they are the apertures of the mind, doors and windows through which we take in and make our own the materials of knowledge. Regarded in any other point of view sensation is of all mental acts the most trivial and superficial. Hence the term 'sensational' is rightly used to express what is shallow in thought and feeling.

\par  We propose in what follows, first of all, like Plato in the Theaetetus, to analyse sensation, and secondly to trace the connexion between theories of sensation and a sensational or Epicurean philosophy.

\par  Paragraph I. We, as well as the ancients, speak of the five senses, and of a sense, or common sense, which is the abstraction of them. The term 'sense' is also used metaphorically, both in ancient and modern philosophy, to express the operations of the mind which are immediate or intuitive. Of the five senses, two—the sight and the hearing—are of a more subtle and complex nature, while two others—the smell and the taste—seem to be only more refined varieties of touch. All of them are passive, and by this are distinguished from the active faculty of speech: they receive impressions, but do not produce them, except in so far as they are objects of sense themselves.

\par  Physiology speaks to us of the wonderful apparatus of nerves, muscles, tissues, by which the senses are enabled to fulfil their functions. It traces the connexion, though imperfectly, of the bodily organs with the operations of the mind. Of these latter, it seems rather to know the conditions than the causes. It can prove to us that without the brain we cannot think, and that without the eye we cannot see: and yet there is far more in thinking and seeing than is given by the brain and the eye. It observes the 'concomitant variations' of body and mind. Psychology, on the other hand, treats of the same subject regarded from another point of view. It speaks of the relation of the senses to one another; it shows how they meet the mind; it analyzes the transition from sense to thought. The one describes their nature as apparent to the outward eye; by the other they are regarded only as the instruments of the mind. It is in this latter point of view that we propose to consider them.

\par  The simplest sensation involves an unconscious or nascent operation of the mind; it implies objects of sense, and objects of sense have differences of form, number, colour. But the conception of an object without us, or the power of discriminating numbers, forms, colours, is not given by the sense, but by the mind. A mere sensation does not attain to distinctness: it is a confused impression, sugkechumenon ti, as Plato says (Republic), until number introduces light and order into the confusion. At what point confusion becomes distinctness is a question of degree which cannot be precisely determined. The distant object, the undefined notion, come out into relief as we approach them or attend to them. Or we may assist the analysis by attempting to imagine the world first dawning upon the eye of the infant or of a person newly restored to sight. Yet even with them the mind as well as the eye opens or enlarges. For all three are inseparably bound together—the object would be nowhere and nothing, if not perceived by the sense, and the sense would have no power of distinguishing without the mind.

\par  But prior to objects of sense there is a third nature in which they are contained—that is to say, space, which may be explained in various ways. It is the element which surrounds them; it is the vacuum or void which they leave or occupy when passing from one portion of space to another. It might be described in the language of ancient philosophy, as 'the Not-being' of objects. It is a negative idea which in the course of ages has become positive. It is originally derived from the contemplation of the world without us—the boundless earth or sea, the vacant heaven, and is therefore acquired chiefly through the sense of sight: to the blind the conception of space is feeble and inadequate, derived for the most part from touch or from the descriptions of others. At first it appears to be continuous; afterwards we perceive it to be capable of division by lines or points, real or imaginary. By the help of mathematics we form another idea of space, which is altogether independent of experience. Geometry teaches us that the innumerable lines and figures by which space is or may be intersected are absolutely true in all their combinations and consequences. New and unchangeable properties of space are thus developed, which are proved to us in a thousand ways by mathematical reasoning as well as by common experience. Through quantity and measure we are conducted to our simplest and purest notion of matter, which is to the cube or solid what space is to the square or surface. And all our applications of mathematics are applications of our ideas of space to matter. No wonder then that they seem to have a necessary existence to us. Being the simplest of our ideas, space is also the one of which we have the most difficulty in ridding ourselves. Neither can we set a limit to it, for wherever we fix a limit, space is springing up beyond. Neither can we conceive a smallest or indivisible portion of it; for within the smallest there is a smaller still; and even these inconceivable qualities of space, whether the infinite or the infinitesimal, may be made the subject of reasoning and have a certain truth to us.

\par  Whether space exists in the mind or out of it, is a question which has no meaning. We should rather say that without it the mind is incapable of conceiving the body, and therefore of conceiving itself. The mind may be indeed imagined to contain the body, in the same way that Aristotle (partly following Plato) supposes God to be the outer heaven or circle of the universe. But how can the individual mind carry about the universe of space packed up within, or how can separate minds have either a universe of their own or a common universe? In such conceptions there seems to be a confusion of the individual and the universal. To say that we can only have a true idea of ourselves when we deny the reality of that by which we have any idea of ourselves is an absurdity. The earth which is our habitation and 'the starry heaven above' and we ourselves are equally an illusion, if space is only a quality or condition of our minds.

\par  Again, we may compare the truths of space with other truths derived from experience, which seem to have a necessity to us in proportion to the frequency of their recurrence or the truth of the consequences which may be inferred from them. We are thus led to remark that the necessity in our ideas of space on which much stress has been laid, differs in a slight degree only from the necessity which appears to belong to other of our ideas, e.g. weight, motion, and the like. And there is another way in which this necessity may be explained. We have been taught it, and the truth which we were taught or which we inherited has never been contradicted in all our experience and is therefore confirmed by it. Who can resist an idea which is presented to him in a general form in every moment of his life and of which he finds no instance to the contrary? The greater part of what is sometimes regarded as the a priori intuition of space is really the conception of the various geometrical figures of which the properties have been revealed by mathematical analysis. And the certainty of these properties is immeasurably increased to us by our finding that they hold good not only in every instance, but in all the consequences which are supposed to flow from them.

\par  Neither must we forget that our idea of space, like our other ideas, has a history. The Homeric poems contain no word for it; even the later Greek philosophy has not the Kantian notion of space, but only the definite 'place' or 'the infinite.' To Plato, in the Timaeus, it is known only as the 'nurse of generation.' When therefore we speak of the necessity of our ideas of space we must remember that this is a necessity which has grown up with the growth of the human mind, and has been made by ourselves. We can free ourselves from the perplexities which are involved in it by ascending to a time in which they did not as yet exist. And when space or time are described as 'a priori forms or intuitions added to the matter given in sensation,' we should consider that such expressions belong really to the 'pre-historic study' of philosophy, i.e. to the eighteenth century, when men sought to explain the human mind without regard to history or language or the social nature of man.

\par  In every act of sense there is a latent perception of space, of which we only become conscious when objects are withdrawn from it. There are various ways in which we may trace the connexion between them. We may think of space as unresisting matter, and of matter as divided into objects; or of objects again as formed by abstraction into a collective notion of matter, and of matter as rarefied into space. And motion may be conceived as the union of there and not there in space, and force as the materializing or solidification of motion. Space again is the individual and universal in one; or, in other words, a perception and also a conception. So easily do what are sometimes called our simple ideas pass into one another, and differences of kind resolve themselves into differences of degree.

\par  Within or behind space there is another abstraction in many respects similar to it—time, the form of the inward, as space is the form of the outward. As we cannot think of outward objects of sense or of outward sensations without space, so neither can we think of a succession of sensations without time. It is the vacancy of thoughts or sensations, as space is the void of outward objects, and we can no more imagine the mind without the one than the world without the other. It is to arithmetic what space is to geometry; or, more strictly, arithmetic may be said to be equally applicable to both. It is defined in our minds, partly by the analogy of space and partly by the recollection of events which have happened to us, or the consciousness of feelings which we are experiencing. Like space, it is without limit, for whatever beginning or end of time we fix, there is a beginning and end before them, and so on without end. We speak of a past, present, and future, and again the analogy of space assists us in conceiving of them as coexistent. When the limit of time is removed there arises in our minds the idea of eternity, which at first, like time itself, is only negative, but gradually, when connected with the world and the divine nature, like the other negative infinity of space, becomes positive. Whether time is prior to the mind and to experience, or coeval with them, is (like the parallel question about space) unmeaning. Like space it has been realized gradually: in the Homeric poems, or even in the Hesiodic cosmogony, there is no more notion of time than of space. The conception of being is more general than either, and might therefore with greater plausibility be affirmed to be a condition or quality of the mind. The a priori intuitions of Kant would have been as unintelligible to Plato as his a priori synthetical propositions to Aristotle. The philosopher of Konigsberg supposed himself to be analyzing a necessary mode of thought: he was not aware that he was dealing with a mere abstraction. But now that we are able to trace the gradual developement of ideas through religion, through language, through abstractions, why should we interpose the fiction of time between ourselves and realities? Why should we single out one of these abstractions to be the a priori condition of all the others? It comes last and not first in the order of our thoughts, and is not the condition precedent of them, but the last generalization of them. Nor can any principle be imagined more suicidal to philosophy than to assume that all the truth which we are capable of attaining is seen only through an unreal medium. If all that exists in time is illusion, we may well ask with Plato, 'What becomes of the mind?'

\par  Leaving the a priori conditions of sensation we may proceed to consider acts of sense. These admit of various degrees of duration or intensity; they admit also of a greater or less extension from one object, which is perceived directly, to many which are perceived indirectly or in a less degree, and to the various associations of the object which are latent in the mind. In general the greater the intension the less the extension of them. The simplest sensation implies some relation of objects to one another, some position in space, some relation to a previous or subsequent sensation. The acts of seeing and hearing may be almost unconscious and may pass away unnoted; they may also leave an impression behind them or power of recalling them. If, after seeing an object we shut our eyes, the object remains dimly seen in the same or about the same place, but with form and lineaments half filled up. This is the simplest act of memory. And as we cannot see one thing without at the same time seeing another, different objects hang together in recollection, and when we call for one the other quickly follows. To think of the place in which we have last seen a thing is often the best way of recalling it to the mind. Hence memory is dependent on association. The act of recollection may be compared to the sight of an object at a great distance which we have previously seen near and seek to bring near to us in thought. Memory is to sense as dreaming is to waking; and like dreaming has a wayward and uncertain power of recalling impressions from the past.

\par  Thus begins the passage from the outward to the inward sense. But as yet there is no conception of a universal—the mind only remembers the individual object or objects, and is always attaching to them some colour or association of sense. The power of recollection seems to depend on the intensity or largeness of the perception, or on the strength of some emotion with which it is inseparably connected. This is the natural memory which is allied to sense, such as children appear to have and barbarians and animals. It is necessarily limited in range, and its limitation is its strength. In later life, when the mind has become crowded with names, acts, feelings, images innumerable, we acquire by education another memory of system and arrangement which is both stronger and weaker than the first—weaker in the recollection of sensible impressions as they are represented to us by eye or ear—stronger by the natural connexion of ideas with objects or with one another. And many of the notions which form a part of the train of our thoughts are hardly realized by us at the time, but, like numbers or algebraical symbols, are used as signs only, thus lightening the labour of recollection.

\par  And now we may suppose that numerous images present themselves to the mind, which begins to act upon them and to arrange them in various ways. Besides the impression of external objects present with us or just absent from us, we have a dimmer conception of other objects which have disappeared from our immediate recollection and yet continue to exist in us. The mind is full of fancies which are passing to and fro before it. Some feeling or association calls them up, and they are uttered by the lips. This is the first rudimentary imagination, which may be truly described in the language of Hobbes, as 'decaying sense,' an expression which may be applied with equal truth to memory as well. For memory and imagination, though we sometimes oppose them, are nearly allied; the difference between them seems chiefly to lie in the activity of the one compared with the passivity of the other. The sense decaying in memory receives a flash of light or life from imagination. Dreaming is a link of connexion between them; for in dreaming we feebly recollect and also feebly imagine at one and the same time. When reason is asleep the lower part of the mind wanders at will amid the images which have been received from without, the intelligent element retires, and the sensual or sensuous takes its place. And so in the first efforts of imagination reason is latent or set aside; and images, in part disorderly, but also having a unity (however imperfect) of their own, pour like a flood over the mind. And if we could penetrate into the heads of animals we should probably find that their intelligence, or the state of what in them is analogous to our intelligence, is of this nature.

\par  Thus far we have been speaking of men, rather in the points in which they resemble animals than in the points in which they differ from them. The animal too has memory in various degrees, and the elements of imagination, if, as appears to be the case, he dreams. How far their powers or instincts are educated by the circumstances of their lives or by intercourse with one another or with mankind, we cannot precisely tell. They, like ourselves, have the physical inheritance of form, scent, hearing, sight, and other qualities or instincts. But they have not the mental inheritance of thoughts and ideas handed down by tradition, 'the slow additions that build up the mind' of the human race. And language, which is the great educator of mankind, is wanting in them; whereas in us language is ever present—even in the infant the latent power of naming is almost immediately observable. And therefore the description which has been already given of the nascent power of the faculties is in reality an anticipation. For simultaneous with their growth in man a growth of language must be supposed. The child of two years old sees the fire once and again, and the feeble observation of the same recurring object is associated with the feeble utterance of the name by which he is taught to call it. Soon he learns to utter the name when the object is no longer there, but the desire or imagination of it is present to him. At first in every use of the word there is a colour of sense, an indistinct picture of the object which accompanies it. But in later years he sees in the name only the universal or class word, and the more abstract the notion becomes, the more vacant is the image which is presented to him. Henceforward all the operations of his mind, including the perceptions of sense, are a synthesis of sensations, words, conceptions. In seeing or hearing or looking or listening the sensible impression prevails over the conception and the word. In reflection the process is reversed—the outward object fades away into nothingness, the name or the conception or both together are everything. Language, like number, is intermediate between the two, partaking of the definiteness of the outer and of the universality of the inner world. For logic teaches us that every word is really a universal, and only condescends by the help of position or circumlocution to become the expression of individuals or particulars. And sometimes by using words as symbols we are able to give a 'local habitation and a name' to the infinite and inconceivable.

\par  Thus we see that no line can be drawn between the powers of sense and of reflection—they pass imperceptibly into one another. We may indeed distinguish between the seeing and the closed eye—between the sensation and the recollection of it. But this distinction carries us a very little way, for recollection is present in sight as well as sight in recollection. There is no impression of sense which does not simultaneously recall differences of form, number, colour, and the like. Neither is such a distinction applicable at all to our internal bodily sensations, which give no sign of themselves when unaccompanied with pain, and even when we are most conscious of them, have often no assignable place in the human frame. Who can divide the nerves or great nervous centres from the mind which uses them? Who can separate the pains and pleasures of the mind from the pains and pleasures of the body? The words 'inward and outward,' 'active and passive,' 'mind and body,' are best conceived by us as differences of degree passing into differences of kind, and at one time and under one aspect acting in harmony and then again opposed. They introduce a system and order into the knowledge of our being; and yet, like many other general terms, are often in advance of our actual analysis or observation.

\par  According to some writers the inward sense is only the fading away or imperfect realization of the outward. But this leaves out of sight one half of the phenomenon. For the mind is not only withdrawn from the world of sense but introduced to a higher world of thought and reflection, in which, like the outward sense, she is trained and educated. By use the outward sense becomes keener and more intense, especially when confined within narrow limits. The savage with little or no thought has a quicker discernment of the track than the civilised man; in like manner the dog, having the help of scent as well as of sight, is superior to the savage. By use again the inward thought becomes more defined and distinct; what was at first an effort is made easy by the natural instrumentality of language, and the mind learns to grasp universals with no more exertion than is required for the sight of an outward object. There is a natural connexion and arrangement of them, like the association of objects in a landscape. Just as a note or two of music suffices to recall a whole piece to the musician's or composer's mind, so a great principle or leading thought suggests and arranges a world of particulars. The power of reflection is not feebler than the faculty of sense, but of a higher and more comprehensive nature. It not only receives the universals of sense, but gives them a new content by comparing and combining them with one another. It withdraws from the seen that it may dwell in the unseen. The sense only presents us with a flat and impenetrable surface: the mind takes the world to pieces and puts it together on a new pattern. The universals which are detached from sense are reconstructed in science. They and not the mere impressions of sense are the truth of the world in which we live; and (as an argument to those who will only believe 'what they can hold in their hands') we may further observe that they are the source of our power over it. To say that the outward sense is stronger than the inward is like saying that the arm of the workman is stronger than the constructing or directing mind.

\par  Returning to the senses we may briefly consider two questions—first their relation to the mind, secondly, their relation to outward objects:—

\par  1. The senses are not merely 'holes set in a wooden horse' (Theaet. ), but instruments of the mind with which they are organically connected. There is no use of them without some use of words—some natural or latent logic—some previous experience or observation. Sensation, like all other mental processes, is complex and relative, though apparently simple. The senses mutually confirm and support one another; it is hard to say how much our impressions of hearing may be affected by those of sight, or how far our impressions of sight may be corrected by the touch, especially in infancy. The confirmation of them by one another cannot of course be given by any one of them. Many intuitions which are inseparable from the act of sense are really the result of complicated reasonings. The most cursory glance at objects enables the experienced eye to judge approximately of their relations and distance, although nothing is impressed upon the retina except colour, including gradations of light and shade. From these delicate and almost imperceptible differences we seem chiefly to derive our ideas of distance and position. By comparison of what is near with what is distant we learn that the tree, house, river, etc. which are a long way off are objects of a like nature with those which are seen by us in our immediate neighbourhood, although the actual impression made on the eye is very different in one case and in the other. This is a language of 'large and small letters' (Republic), slightly differing in form and exquisitely graduated by distance, which we are learning all our life long, and which we attain in various degrees according to our powers of sight or observation. There is nor the consideration. The greater or less strain upon the nerves of the eye or ear is communicated to the mind and silently informs the judgment. We have also the use not of one eye only, but of two, which give us a wider range, and help us to discern, by the greater or less acuteness of the angle which the rays of sight form, the distance of an object and its relation to other objects. But we are already passing beyond the limits of our actual knowledge on a subject which has given rise to many conjectures. More important than the addition of another conjecture is the observation, whether in the case of sight or of any other sense, of the great complexity of the causes and the great simplicity of the effect.

\par  The sympathy of the mind and the ear is no less striking than the sympathy of the mind and the eye. Do we not seem to perceive instinctively and as an act of sense the differences of articulate speech and of musical notes? Yet how small a part of speech or of music is produced by the impression of the ear compared with that which is furnished by the mind!

\par  Again: the more refined faculty of sense, as in animals so also in man, seems often to be transmitted by inheritance. Neither must we forget that in the use of the senses, as in his whole nature, man is a social being, who is always being educated by language, habit, and the teaching of other men as well as by his own observation. He knows distance because he is taught it by a more experienced judgment than his own; he distinguishes sounds because he is told to remark them by a person of a more discerning ear. And as we inherit from our parents or other ancestors peculiar powers of sense or feeling, so we improve and strengthen them, not only by regular teaching, but also by sympathy and communion with other persons.

\par  2. The second question, namely, that concerning the relation of the mind to external objects, is really a trifling one, though it has been made the subject of a famous philosophy. We may if we like, with Berkeley, resolve objects of sense into sensations; but the change is one of name only, and nothing is gained and something is lost by such a resolution or confusion of them. For we have not really made a single step towards idealism, and any arbitrary inversion of our ordinary modes of speech is disturbing to the mind. The youthful metaphysician is delighted at his marvellous discovery that nothing is, and that what we see or feel is our sensation only: for a day or two the world has a new interest to him; he alone knows the secret which has been communicated to him by the philosopher, that mind is all—when in fact he is going out of his mind in the first intoxication of a great thought. But he soon finds that all things remain as they were—the laws of motion, the properties of matter, the qualities of substances. After having inflicted his theories on any one who is willing to receive them 'first on his father and mother, secondly on some other patient listener, thirdly on his dog,' he finds that he only differs from the rest of mankind in the use of a word. He had once hoped that by getting rid of the solidity of matter he might open a passage to worlds beyond. He liked to think of the world as the representation of the divine nature, and delighted to imagine angels and spirits wandering through space, present in the room in which he is sitting without coming through the door, nowhere and everywhere at the same instant. At length he finds that he has been the victim of his own fancies; he has neither more nor less evidence of the supernatural than he had before. He himself has become unsettled, but the laws of the world remain fixed as at the beginning. He has discovered that his appeal to the fallibility of sense was really an illusion. For whatever uncertainty there may be in the appearances of nature, arises only out of the imperfection or variation of the human senses, or possibly from the deficiency of certain branches of knowledge; when science is able to apply her tests, the uncertainty is at an end. We are apt sometimes to think that moral and metaphysical philosophy are lowered by the influence which is exercised over them by physical science. But any interpretation of nature by physical science is far in advance of such idealism. The philosophy of Berkeley, while giving unbounded license to the imagination, is still grovelling on the level of sense.

\par  We may, if we please, carry this scepticism a step further, and deny, not only objects of sense, but the continuity of our sensations themselves. We may say with Protagoras and Hume that what is appears, and that what appears appears only to individuals, and to the same individual only at one instant. But then, as Plato asks,—and we must repeat the question,—What becomes of the mind? Experience tells us by a thousand proofs that our sensations of colour, taste, and the like, are the same as they were an instant ago—that the act which we are performing one minute is continued by us in the next—and also supplies abundant proof that the perceptions of other men are, speaking generally, the same or nearly the same with our own. After having slowly and laboriously in the course of ages gained a conception of a whole and parts, of the constitution of the mind, of the relation of man to God and nature, imperfect indeed, but the best we can, we are asked to return again to the 'beggarly elements' of ancient scepticism, and acknowledge only atoms and sensations devoid of life or unity. Why should we not go a step further still and doubt the existence of the senses of all things? We are but 'such stuff as dreams are made of;' for we have left ourselves no instruments of thought by which we can distinguish man from the animals, or conceive of the existence even of a mollusc. And observe, this extreme scepticism has been allowed to spring up among us, not, like the ancient scepticism, in an age when nature and language really seemed to be full of illusions, but in the eighteenth and nineteenth centuries, when men walk in the daylight of inductive science.

\par  The attractiveness of such speculations arises out of their true nature not being perceived. They are veiled in graceful language; they are not pushed to extremes; they stop where the human mind is disposed also to stop—short of a manifest absurdity. Their inconsistency is not observed by their authors or by mankind in general, who are equally inconsistent themselves. They leave on the mind a pleasing sense of wonder and novelty: in youth they seem to have a natural affinity to one class of persons as poetry has to another; but in later life either we drift back into common sense, or we make them the starting-points of a higher philosophy.

\par  We are often told that we should enquire into all things before we accept them;—with what limitations is this true? For we cannot use our senses without admitting that we have them, or think without presupposing that there is in us a power of thought, or affirm that all knowledge is derived from experience without implying that this first principle of knowledge is prior to experience. The truth seems to be that we begin with the natural use of the mind as of the body, and we seek to describe this as well as we can. We eat before we know the nature of digestion; we think before we know the nature of reflection. As our knowledge increases, our perception of the mind enlarges also. We cannot indeed get beyond facts, but neither can we draw any line which separates facts from ideas. And the mind is not something separate from them but included in them, and they in the mind, both having a distinctness and individuality of their own. To reduce our conception of mind to a succession of feelings and sensations is like the attempt to view a wide prospect by inches through a microscope, or to calculate a period of chronology by minutes. The mind ceases to exist when it loses its continuity, which though far from being its highest determination, is yet necessary to any conception of it. Even an inanimate nature cannot be adequately represented as an endless succession of states or conditions.

\par  Paragraph II. Another division of the subject has yet to be considered: Why should the doctrine that knowledge is sensation, in ancient times, or of sensationalism or materialism in modern times, be allied to the lower rather than to the higher view of ethical philosophy? At first sight the nature and origin of knowledge appear to be wholly disconnected from ethics and religion, nor can we deny that the ancient Stoics were materialists, or that the materialist doctrines prevalent in modern times have been associated with great virtues, or that both religious and philosophical idealism have not unfrequently parted company with practice. Still upon the whole it must be admitted that the higher standard of duty has gone hand in hand with the higher conception of knowledge. It is Protagoras who is seeking to adapt himself to the opinions of the world; it is Plato who rises above them: the one maintaining that all knowledge is sensation; the other basing the virtues on the idea of good. The reason of this phenomenon has now to be examined.

\par  By those who rest knowledge immediately upon sense, that explanation of human action is deemed to be the truest which is nearest to sense. As knowledge is reduced to sensation, so virtue is reduced to feeling, happiness or good to pleasure. The different virtues—the various characters which exist in the world—are the disguises of self-interest. Human nature is dried up; there is no place left for imagination, or in any higher sense for religion. Ideals of a whole, or of a state, or of a law of duty, or of a divine perfection, are out of place in an Epicurean philosophy. The very terms in which they are expressed are suspected of having no meaning. Man is to bring himself back as far as he is able to the condition of a rational beast. He is to limit himself to the pursuit of pleasure, but of this he is to make a far-sighted calculation;—he is to be rationalized, secularized, animalized: or he is to be an amiable sceptic, better than his own philosophy, and not falling below the opinions of the world.

\par  Imagination has been called that 'busy faculty' which is always intruding upon us in the search after truth. But imagination is also that higher power by which we rise above ourselves and the commonplaces of thought and life. The philosophical imagination is another name for reason finding an expression of herself in the outward world. To deprive life of ideals is to deprive it of all higher and comprehensive aims and of the power of imparting and communicating them to others. For men are taught, not by those who are on a level with them, but by those who rise above them, who see the distant hills, who soar into the empyrean. Like a bird in a cage, the mind confined to sense is always being brought back from the higher to the lower, from the wider to the narrower view of human knowledge. It seeks to fly but cannot: instead of aspiring towards perfection, 'it hovers about this lower world and the earthly nature.' It loses the religious sense which more than any other seems to take a man out of himself. Weary of asking 'What is truth?' it accepts the 'blind witness of eyes and ears;' it draws around itself the curtain of the physical world and is satisfied. The strength of a sensational philosophy lies in the ready accommodation of it to the minds of men; many who have been metaphysicians in their youth, as they advance in years are prone to acquiesce in things as they are, or rather appear to be. They are spectators, not thinkers, and the best philosophy is that which requires of them the least amount of mental effort.

\par  As a lower philosophy is easier to apprehend than a higher, so a lower way of life is easier to follow; and therefore such a philosophy seems to derive a support from the general practice of mankind. It appeals to principles which they all know and recognize: it gives back to them in a generalized form the results of their own experience. To the man of the world they are the quintessence of his own reflections upon life. To follow custom, to have no new ideas or opinions, not to be straining after impossibilities, to enjoy to-day with just so much forethought as is necessary to provide for the morrow, this is regarded by the greater part of the world as the natural way of passing through existence. And many who have lived thus have attained to a lower kind of happiness or equanimity. They have possessed their souls in peace without ever allowing them to wander into the region of religious or political controversy, and without any care for the higher interests of man. But nearly all the good (as well as some of the evil) which has ever been done in this world has been the work of another spirit, the work of enthusiasts and idealists, of apostles and martyrs. The leaders of mankind have not been of the gentle Epicurean type; they have personified ideas; they have sometimes also been the victims of them. But they have always been seeking after a truth or ideal of which they fell short; and have died in a manner disappointed of their hopes that they might lift the human race out of the slough in which they found them. They have done little compared with their own visions and aspirations; but they have done that little, only because they sought to do, and once perhaps thought that they were doing, a great deal more.

\par  The philosophies of Epicurus or Hume give no adequate or dignified conception of the mind. There is no organic unity in a succession of feeling or sensations; no comprehensiveness in an infinity of separate actions. The individual never reflects upon himself as a whole; he can hardly regard one act or part of his life as the cause or effect of any other act or part. Whether in practice or speculation, he is to himself only in successive instants. To such thinkers, whether in ancient or in modern times, the mind is only the poor recipient of impressions—not the heir of all the ages, or connected with all other minds. It begins again with its own modicum of experience having only such vague conceptions of the wisdom of the past as are inseparable from language and popular opinion. It seeks to explain from the experience of the individual what can only be learned from the history of the world. It has no conception of obligation, duty, conscience—these are to the Epicurean or Utilitarian philosopher only names which interfere with our natural perceptions of pleasure and pain.

\par  There seem then to be several answers to the question, Why the theory that all knowledge is sensation is allied to the lower rather than to the higher view of ethical philosophy:—1st, Because it is easier to understand and practise; 2ndly, Because it is fatal to the pursuit of ideals, moral, political, or religious; 3rdly, Because it deprives us of the means and instruments of higher thought, of any adequate conception of the mind, of knowledge, of conscience, of moral obligation.

\par  ...

\par  ON THE NATURE AND LIMITS Of PSYCHOLOGY.
 
\par  Since the above essay first appeared, many books on Psychology have been given to the world, partly based upon the views of Herbart and other German philosophers, partly independent of them. The subject has gained in bulk and extent; whether it has had any true growth is more doubtful. It begins to assume the language and claim the authority of a science; but it is only an hypothesis or outline, which may be filled up in many ways according to the fancy of individual thinkers. The basis of it is a precarious one,—consciousness of ourselves and a somewhat uncertain observation of the rest of mankind. Its relations to other sciences are not yet determined: they seem to be almost too complicated to be ascertained. It may be compared to an irregular building, run up hastily and not likely to last, because its foundations are weak, and in many places rest only on the surface of the ground. It has sought rather to put together scattered observations and to make them into a system than to describe or prove them. It has never severely drawn the line between facts and opinions. It has substituted a technical phraseology for the common use of language, being neither able to win acceptance for the one nor to get rid of the other.

\par  The system which has thus arisen appears to be a kind of metaphysic narrowed to the point of view of the individual mind, through which, as through some new optical instrument limiting the sphere of vision, the interior of thought and sensation is examined. But the individual mind in the abstract, as distinct from the mind of a particular individual and separated from the environment of circumstances, is a fiction only. Yet facts which are partly true gather around this fiction and are naturally described by the help of it. There is also a common type of the mind which is derived from the comparison of many minds with one another and with our own. The phenomena of which Psychology treats are familiar to us, but they are for the most part indefinite; they relate to a something inside the body, which seems also to overleap the limits of space. The operations of this something, when isolated, cannot be analyzed by us or subjected to observation and experiment. And there is another point to be considered. The mind, when thinking, cannot survey that part of itself which is used in thought. It can only be contemplated in the past, that is to say, in the history of the individual or of the world. This is the scientific method of studying the mind. But Psychology has also some other supports, specious rather than real. It is partly sustained by the false analogy of Physical Science and has great expectations from its near relationship to Physiology. We truly remark that there is an infinite complexity of the body corresponding to the infinite subtlety of the mind; we are conscious that they are very nearly connected. But in endeavouring to trace the nature of the connexion we are baffled and disappointed. In our knowledge of them the gulf remains the same: no microscope has ever seen into thought; no reflection on ourselves has supplied the missing link between mind and matter...These are the conditions of this very inexact science, and we shall only know less of it by pretending to know more, or by assigning to it a form or style to which it has not yet attained and is not really entitled.

\par  Experience shows that any system, however baseless and ineffectual, in our own or in any other age, may be accepted and continue to be studied, if it seeks to satisfy some unanswered question or is based upon some ancient tradition, especially if it takes the form and uses the language of inductive philosophy. The fact therefore that such a science exists and is popular, affords no evidence of its truth or value. Many who have pursued it far into detail have never examined the foundations on which it rests. The have been many imaginary subjects of knowledge of which enthusiastic persons have made a lifelong study, without ever asking themselves what is the evidence for them, what is the use of them, how long they will last? They may pass away, like the authors of them, and 'leave not a wrack behind;' or they may survive in fragments. Nor is it only in the Middle Ages, or in the literary desert of China or of India, that such systems have arisen; in our own enlightened age, growing up by the side of Physics, Ethics, and other really progressive sciences, there is a weary waste of knowledge, falsely so-called. There are sham sciences which no logic has ever put to the test, in which the desire for knowledge invents the materials of it.

\par  And therefore it is expedient once more to review the bases of Psychology, lest we should be imposed upon by its pretensions. The study of it may have done good service by awakening us to the sense of inveterate errors familiarized by language, yet it may have fallen into still greater ones; under the pretence of new investigations it may be wasting the lives of those who are engaged in it. It may also be found that the discussion of it will throw light upon some points in the Theaetetus of Plato,—the oldest work on Psychology which has come down to us. The imaginary science may be called, in the language of ancient philosophy, 'a shadow of a part of Dialectic or Metaphysic' (Gorg. ).

\par  In this postscript or appendix we propose to treat, first, of the true bases of Psychology; secondly, of the errors into which the students of it are most likely to fall; thirdly, of the principal subjects which are usually comprehended under it; fourthly, of the form which facts relating to the mind most naturally assume.

\par  We may preface the enquiry by two or three remarks:—

\par  (1) We do not claim for the popular Psychology the position of a science at all; it cannot, like the Physical Sciences, proceed by the Inductive Method: it has not the necessity of Mathematics: it does not, like Metaphysic, argue from abstract notions or from internal coherence. It is made up of scattered observations. A few of these, though they may sometimes appear to be truisms, are of the greatest value, and free from all doubt. We are conscious of them in ourselves; we observe them working in others; we are assured of them at all times. For example, we are absolutely certain, (a) of the influence exerted by the mind over the body or by the body over the mind: (b) of the power of association, by which the appearance of some person or the occurrence of some event recalls to mind, not always but often, other persons and events: (c) of the effect of habit, which is strongest when least disturbed by reflection, and is to the mind what the bones are to the body: (d) of the real, though not unlimited, freedom of the human will: (e) of the reference, more or less distinct, of our sensations, feelings, thoughts, actions, to ourselves, which is called consciousness, or, when in excess, self-consciousness: (f) of the distinction of the 'I' and 'Not I,' of ourselves and outward objects. But when we attempt to gather up these elements in a single system, we discover that the links by which we combine them are apt to be mere words. We are in a country which has never been cleared or surveyed; here and there only does a gleam of light come through the darkness of the forest.

\par  (2) These fragments, although they can never become science in the ordinary sense of the word, are a real part of knowledge and may be of great value in education. We may be able to add a good deal to them from our own experience, and we may verify them by it. Self-examination is one of those studies which a man can pursue alone, by attention to himself and the processes of his individual mind. He may learn much about his own character and about the character of others, if he will 'make his mind sit down' and look at itself in the glass. The great, if not the only use of such a study is a practical one,—to know, first, human nature, and, secondly, our own nature, as it truly is.

\par  (3) Hence it is important that we should conceive of the mind in the noblest and simplest manner. While acknowledging that language has been the greatest factor in the formation of human thought, we must endeavour to get rid of the disguises, oppositions, contradictions, which arise out of it. We must disengage ourselves from the ideas which the customary use of words has implanted in us. To avoid error as much as possible when we are speaking of things unseen, the principal terms which we use should be few, and we should not allow ourselves to be enslaved by them. Instead of seeking to frame a technical language, we should vary our forms of speech, lest they should degenerate into formulas. A difficult philosophical problem is better understood when translated into the vernacular.

\par  I.a. Psychology is inseparable from language, and early language contains the first impressions or the oldest experience of man respecting himself. These impressions are not accurate representations of the truth; they are the reflections of a rudimentary age of philosophy. The first and simplest forms of thought are rooted so deep in human nature that they can never be got rid of; but they have been perpetually enlarged and elevated, and the use of many words has been transferred from the body to the mind. The spiritual and intellectual have thus become separated from the material—there is a cleft between them; and the heart and the conscience of man rise above the dominion of the appetites and create a new language in which they too find expression. As the differences of actions begin to be perceived, more and more names are needed. This is the first analysis of the human mind; having a general foundation in popular experience, it is moulded to a certain extent by hierophants and philosophers. (See Introd. to Cratylus.)

\par  b. This primitive psychology is continually receiving additions from the first thinkers, who in return take a colour from the popular language of the time. The mind is regarded from new points of view, and becomes adapted to new conditions of knowledge. It seeks to isolate itself from matter and sense, and to assert its independence in thought. It recognizes that it is independent of the external world. It has five or six natural states or stages:—(1) sensation, in which it is almost latent or quiescent: (2) feeling, or inner sense, when the mind is just awakening: (3) memory, which is decaying sense, and from time to time, as with a spark or flash, has the power of recollecting or reanimating the buried past: (4) thought, in which images pass into abstract notions or are intermingled with them: (5) action, in which the mind moves forward, of itself, or under the impulse of want or desire or pain, to attain or avoid some end or consequence: and (6) there is the composition of these or the admixture or assimilation of them in various degrees. We never see these processes of the mind, nor can we tell the causes of them. But we know them by their results, and learn from other men that so far as we can describe to them or they to us the workings of the mind, their experience is the same or nearly the same with our own.

\par  c. But the knowledge of the mind is not to any great extent derived from the observation of the individual by himself. It is the growing consciousness of the human race, embodied in language, acknowledged by experience, and corrected from time to time by the influence of literature and philosophy. A great, perhaps the most important, part of it is to be found in early Greek thought. In the Theaetetus of Plato it has not yet become fixed: we are still stumbling on the threshold. In Aristotle the process is more nearly completed, and has gained innumerable abstractions, of which many have had to be thrown away because relative only to the controversies of the time. In the interval between Thales and Aristotle were realized the distinctions of mind and body, of universal and particular, of infinite and infinitesimal, of idea and phenomenon; the class conceptions of faculties and virtues, the antagonism of the appetites and the reason; and connected with this, at a higher stage of development, the opposition of moral and intellectual virtue; also the primitive conceptions of unity, being, rest, motion, and the like. These divisions were not really scientific, but rather based on popular experience. They were not held with the precision of modern thinkers, but taken all together they gave a new existence to the mind in thought, and greatly enlarged and more accurately defined man's knowledge of himself and of the world. The majority of them have been accepted by Christian and Western nations. Yet in modern times we have also drifted so far away from Aristotle, that if we were to frame a system on his lines we should be at war with ordinary language and untrue to our own consciousness. And there have been a few both in mediaeval times and since the Reformation who have rebelled against the Aristotelian point of view. Of these eccentric thinkers there have been various types, but they have all a family likeness. According to them, there has been too much analysis and too little synthesis, too much division of the mind into parts and too little conception of it as a whole or in its relation to God and the laws of the universe. They have thought that the elements of plurality and unity have not been duly adjusted. The tendency of such writers has been to allow the personality of man to be absorbed in the universal, or in the divine nature, and to deny the distinction between matter and mind, or to substitute one for the other. They have broken some of the idols of Psychology: they have challenged the received meaning of words: they have regarded the mind under many points of view. But though they may have shaken the old, they have not established the new; their views of philosophy, which seem like the echo of some voice from the East, have been alien to the mind of Europe.

\par  d. The Psychology which is found in common language is in some degree verified by experience, but not in such a manner as to give it the character of an exact science. We cannot say that words always correspond to facts. Common language represents the mind from different and even opposite points of view, which cannot be all of them equally true (compare Cratylus). Yet from diversity of statements and opinions may be obtained a nearer approach to the truth than is to be gained from any one of them. It also tends to correct itself, because it is gradually brought nearer to the common sense of mankind. There are some leading categories or classifications of thought, which, though unverified, must always remain the elements from which the science or study of the mind proceeds. For example, we must assume ideas before we can analyze them, and also a continuing mind to which they belong; the resolution of it into successive moments, which would say, with Protagoras, that the man is not the same person which he was a minute ago, is, as Plato implies in the Theaetetus, an absurdity.

\par  e. The growth of the mind, which may be traced in the histories of religions and philosophies and in the thoughts of nations, is one of the deepest and noblest modes of studying it. Here we are dealing with the reality, with the greater and, as it may be termed, the most sacred part of history. We study the mind of man as it begins to be inspired by a human or divine reason, as it is modified by circumstances, as it is distributed in nations, as it is renovated by great movements, which go beyond the limits of nations and affect human society on a scale still greater, as it is created or renewed by great minds, who, looking down from above, have a wider and more comprehensive vision. This is an ambitious study, of which most of us rather 'entertain conjecture' than arrive at any detailed or accurate knowledge. Later arises the reflection how these great ideas or movements of the world have been appropriated by the multitude and found a way to the minds of individuals. The real Psychology is that which shows how the increasing knowledge of nature and the increasing experience of life have always been slowly transforming the mind, how religions too have been modified in the course of ages 'that God may be all and in all.' E pollaplasion, eoe, to ergon e os nun zeteitai prostatteis.

\par  f. Lastly, though we speak of the study of mind in a special sense, it may also be said that there is no science which does not contribute to our knowledge of it. The methods of science and their analogies are new faculties, discovered by the few and imparted to the many. They are to the mind, what the senses are to the body; or better, they may be compared to instruments such as the telescope or microscope by which the discriminating power of the senses, or to other mechanical inventions, by which the strength and skill of the human body is so immeasurably increased.

\par  II. The new Psychology, whatever may be its claim to the authority of a science, has called attention to many facts and corrected many errors, which without it would have been unexamined. Yet it is also itself very liable to illusion. The evidence on which it rests is vague and indefinite. The field of consciousness is never seen by us as a whole, but only at particular points, which are always changing. The veil of language intercepts facts. Hence it is desirable that in making an approach to the study we should consider at the outset what are the kinds of error which most easily affect it, and note the differences which separate it from other branches of knowledge.

\par  a. First, we observe the mind by the mind. It would seem therefore that we are always in danger of leaving out the half of that which is the subject of our enquiry. We come at once upon the difficulty of what is the meaning of the word. Does it differ as subject and object in the same manner? Can we suppose one set of feelings or one part of the mind to interpret another? Is the introspecting thought the same with the thought which is introspected? Has the mind the power of surveying its whole domain at one and the same time?—No more than the eye can take in the whole human body at a glance. Yet there may be a glimpse round the corner, or a thought transferred in a moment from one point of view to another, which enables us to see nearly the whole, if not at once, at any rate in succession. Such glimpses will hardly enable us to contemplate from within the mind in its true proportions. Hence the firmer ground of Psychology is not the consciousness of inward feelings but the observation of external actions, being the actions not only of ourselves, but of the innumerable persons whom we come across in life.

\par  b. The error of supposing partial or occasional explanation of mental phenomena to be the only or complete ones. For example, we are disinclined to admit of the spontaneity or discontinuity of the mind—it seems to us like an effect without a cause, and therefore we suppose the train of our thoughts to be always called up by association. Yet it is probable, or indeed certain, that of many mental phenomena there are no mental antecedents, but only bodily ones.

\par  c. The false influence of language. We are apt to suppose that when there are two or more words describing faculties or processes of the mind, there are real differences corresponding to them. But this is not the case. Nor can we determine how far they do or do not exist, or by what degree or kind of difference they are distinguished. The same remark may be made about figures of speech. They fill up the vacancy of knowledge; they are to the mind what too much colour is to the eye; but the truth is rather concealed than revealed by them.

\par  d. The uncertain meaning of terms, such as Consciousness, Conscience, Will, Law, Knowledge, Internal and External Sense; these, in the language of Plato, 'we shamelessly use, without ever having taken the pains to analyze them.'

\par  e. A science such as Psychology is not merely an hypothesis, but an hypothesis which, unlike the hypotheses of Physics, can never be verified. It rests only on the general impressions of mankind, and there is little or no hope of adding in any considerable degree to our stock of mental facts.

\par  f. The parallelism of the Physical Sciences, which leads us to analyze the mind on the analogy of the body, and so to reduce mental operations to the level of bodily ones, or to confound one with the other.

\par  g. That the progress of Physiology may throw a new light on Psychology is a dream in which scientific men are always tempted to indulge. But however certain we may be of the connexion between mind and body, the explanation of the one by the other is a hidden place of nature which has hitherto been investigated with little or no success.

\par  h. The impossibility of distinguishing between mind and body. Neither in thought nor in experience can we separate them. They seem to act together; yet we feel that we are sometimes under the dominion of the one, sometimes of the other, and sometimes, both in the common use of language and in fact, they transform themselves, the one into the good principle, the other into the evil principle; and then again the 'I' comes in and mediates between them. It is also difficult to distinguish outward facts from the ideas of them in the mind, or to separate the external stimulus to a sensation from the activity of the organ, or this from the invisible agencies by which it reaches the mind, or any process of sense from its mental antecedent, or any mental energy from its nervous expression.

\par  i. The fact that mental divisions tend to run into one another, and that in speaking of the mind we cannot always distinguish differences of kind from differences of degree; nor have we any measure of the strength and intensity of our ideas or feelings.

\par  j. Although heredity has been always known to the ancients as well as ourselves to exercise a considerable influence on human character, yet we are unable to calculate what proportion this birth-influence bears to nurture and education. But this is the real question. We cannot pursue the mind into embryology: we can only trace how, after birth, it begins to grow. But how much is due to the soil, how much to the original latent seed, it is impossible to distinguish. And because we are certain that heredity exercises a considerable, but undefined influence, we must not increase the wonder by exaggerating it.

\par  k. The love of system is always tending to prevail over the historical investigation of the mind, which is our chief means of knowing it. It equally tends to hinder the other great source of our knowledge of the mind, the observation of its workings and processes which we can make for ourselves.

\par  l. The mind, when studied through the individual, is apt to be isolated—this is due to the very form of the enquiry; whereas, in truth, it is indistinguishable from circumstances, the very language which it uses being the result of the instincts of long-forgotten generations, and every word which a man utters being the answer to some other word spoken or suggested by somebody else.

\par  III. The tendency of the preceding remarks has been to show that Psychology is necessarily a fragment, and is not and cannot be a connected system. We cannot define or limit the mind, but we can describe it. We can collect information about it; we can enumerate the principal subjects which are included in the study of it. Thus we are able to rehabilitate Psychology to some extent, not as a branch of science, but as a collection of facts bearing on human life, as a part of the history of philosophy, as an aspect of Metaphysic. It is a fragment of a science only, which in all probability can never make any great progress or attain to much clearness or exactness. It is however a kind of knowledge which has a great interest for us and is always present to us, and of which we carry about the materials in our own bosoms. We can observe our minds and we can experiment upon them, and the knowledge thus acquired is not easily forgotten, and is a help to us in study as well as in conduct.

\par  The principal subjects of Psychology may be summed up as follows:—

\par  a. The relation of man to the world around him,—in what sense and within what limits can he withdraw from its laws or assert himself against them (Freedom and Necessity), and what is that which we suppose to be thus independent and which we call ourselves? How does the inward differ from the outward and what is the relation between them, and where do we draw the line by which we separate mind from matter, the soul from the body? Is the mind active or passive, or partly both? Are its movements identical with those of the body, or only preconcerted and coincident with them, or is one simply an aspect of the other?

\par  b. What are we to think of time and space? Time seems to have a nearer connexion with the mind, space with the body; yet time, as well as space, is necessary to our idea of either. We see also that they have an analogy with one another, and that in Mathematics they often interpenetrate. Space or place has been said by Kant to be the form of the outward, time of the inward sense. He regards them as parts or forms of the mind. But this is an unfortunate and inexpressive way of describing their relation to us. For of all the phenomena present to the human mind they seem to have most the character of objective existence. There is no use in asking what is beyond or behind them; we cannot get rid of them. And to throw the laws of external nature which to us are the type of the immutable into the subjective side of the antithesis seems to be equally inappropriate.

\par  c. When in imagination we enter into the closet of the mind and withdraw ourselves from the external world, we seem to find there more or less distinct processes which may be described by the words, 'I perceive,' 'I feel,' 'I think,' 'I want,' 'I wish,' 'I like,' 'I dislike,' 'I fear,' 'I know,' 'I remember,' 'I imagine,' 'I dream,' 'I act,' 'I endeavour,' 'I hope.' These processes would seem to have the same notions attached to them in the minds of all educated persons. They are distinguished from one another in thought, but they intermingle. It is possible to reflect upon them or to become conscious of them in a greater or less degree, or with a greater or less continuity or attention, and thus arise the intermittent phenomena of consciousness or self-consciousness. The use of all of them is possible to us at all times; and therefore in any operation of the mind the whole are latent. But we are able to characterise them sufficiently by that part of the complex action which is the most prominent. We have no difficulty in distinguishing an act of sight or an act of will from an act of thought, although thought is present in both of them. Hence the conception of different faculties or different virtues is precarious, because each of them is passing into the other, and they are all one in the mind itself; they appear and reappear, and may all be regarded as the ever-varying phases or aspects or differences of the same mind or person.

\par  d. Nearest the sense in the scale of the intellectual faculties is memory, which is a mode rather than a faculty of the mind, and accompanies all mental operations. There are two principal kinds of it, recollection and recognition,—recollection in which forgotten things are recalled or return to the mind, recognition in which the mind finds itself again among things once familiar. The simplest way in which we can represent the former to ourselves is by shutting our eyes and trying to recall in what we term the mind's eye the picture of the surrounding scene, or by laying down the book which we are reading and recapitulating what we can remember of it. But many times more powerful than recollection is recognition, perhaps because it is more assisted by association. We have known and forgotten, and after a long interval the thing which we have seen once is seen again by us, but with a different feeling, and comes back to us, not as new knowledge, but as a thing to which we ourselves impart a notion already present to us; in Plato's words, we set the stamp upon the wax. Every one is aware of the difference between the first and second sight of a place, between a scene clothed with associations or bare and divested of them. We say to ourselves on revisiting a spot after a long interval: How many things have happened since I last saw this! There is probably no impression ever received by us of which we can venture to say that the vestiges are altogether lost, or that we might not, under some circumstances, recover it. A long-forgotten knowledge may be easily renewed and therefore is very different from ignorance. Of the language learnt in childhood not a word may be remembered, and yet, when a new beginning is made, the old habit soon returns, the neglected organs come back into use, and the river of speech finds out the dried-up channel.

\par  e. 'Consciousness' is the most treacherous word which is employed in the study of the mind, for it is used in many senses, and has rarely, if ever, been minutely analyzed. Like memory, it accompanies all mental operations, but not always continuously, and it exists in various degrees. It may be imperceptible or hardly perceptible: it may be the living sense that our thoughts, actions, sufferings, are our own. It is a kind of attention which we pay to ourselves, and is intermittent rather than continuous. Its sphere has been exaggerated. It is sometimes said to assure us of our freedom; but this is an illusion: as there may be a real freedom without consciousness of it, so there may be a consciousness of freedom without the reality. It may be regarded as a higher degree of knowledge when we not only know but know that we know. Consciousness is opposed to habit, inattention, sleep, death. It may be illustrated by its derivative conscience, which speaks to men, not only of right and wrong in the abstract, but of right and wrong actions in reference to themselves and their circumstances.

\par  f. Association is another of the ever-present phenomena of the human mind. We speak of the laws of association, but this is an expression which is confusing, for the phenomenon itself is of the most capricious and uncertain sort. It may be briefly described as follows. The simplest case of association is that of sense. When we see or hear separately one of two things, which we have previously seen or heard together, the occurrence of the one has a tendency to suggest the other. So the sight or name of a house may recall to our minds the memory of those who once lived there. Like may recall like and everything its opposite. The parts of a whole, the terms of a series, objects lying near, words having a customary order stick together in the mind. A word may bring back a passage of poetry or a whole system of philosophy; from one end of the world or from one pole of knowledge we may travel to the other in an indivisible instant. The long train of association by which we pass from one point to the other, involving every sort of complex relation, so sudden, so accidental, is one of the greatest wonders of mind...This process however is not always continuous, but often intermittent: we can think of things in isolation as well as in association; we do not mean that they must all hang from one another. We can begin again after an interval of rest or vacancy, as a new train of thought suddenly arises, as, for example, when we wake of a morning or after violent exercise. Time, place, the same colour or sound or smell or taste, will often call up some thought or recollection either accidentally or naturally associated with them. But it is equally noticeable that the new thought may occur to us, we cannot tell how or why, by the spontaneous action of the mind itself or by the latent influence of the body. Both science and poetry are made up of associations or recollections, but we must observe also that the mind is not wholly dependent on them, having also the power of origination.

\par  There are other processes of the mind which it is good for us to study when we are at home and by ourselves,—the manner in which thought passes into act, the conflict of passion and reason in many stages, the transition from sensuality to love or sentiment and from earthly love to heavenly, the slow and silent influence of habit, which little by little changes the nature of men, the sudden change of the old nature of man into a new one, wrought by shame or by some other overwhelming impulse. These are the greater phenomena of mind, and he who has thought of them for himself will live and move in a better-ordered world, and will himself be a better-ordered man.

\par  At the other end of the 'globus intellectualis,' nearest, not to earth and sense, but to heaven and God, is the personality of man, by which he holds communion with the unseen world. Somehow, he knows not how, somewhere, he knows not where, under this higher aspect of his being he grasps the ideas of God, freedom and immortality; he sees the forms of truth, holiness and love, and is satisfied with them. No account of the mind can be complete which does not admit the reality or the possibility of another life. Whether regarded as an ideal or as a fact, the highest part of man's nature and that in which it seems most nearly to approach the divine, is a phenomenon which exists, and must therefore be included within the domain of Psychology.

\par  IV. We admit that there is no perfect or ideal Psychology. It is not a whole in the same sense in which Chemistry, Physiology, or Mathematics are wholes: that is to say, it is not a connected unity of knowledge. Compared with the wealth of other sciences, it rests upon a small number of facts; and when we go beyond these, we fall into conjectures and verbal discussions. The facts themselves are disjointed; the causes of them run up into other sciences, and we have no means of tracing them from one to the other. Yet it may be true of this, as of other beginnings of knowledge, that the attempt to put them together has tested the truth of them, and given a stimulus to the enquiry into them.

\par  Psychology should be natural, not technical. It should take the form which is the most intelligible to the common understanding, because it has to do with common things, which are familiar to us all. It should aim at no more than every reflecting man knows or can easily verify for himself. When simple and unpretentious, it is least obscured by words, least liable to fall under the influence of Physiology or Metaphysic. It should argue, not from exceptional, but from ordinary phenomena. It should be careful to distinguish the higher and the lower elements of human nature, and not allow one to be veiled in the disguise of the other, lest through the slippery nature of language we should pass imperceptibly from good to evil, from nature in the higher to nature in the neutral or lower sense. It should assert consistently the unity of the human faculties, the unity of knowledge, the unity of God and law. The difference between the will and the affections and between the reason and the passions should also be recognized by it.

\par  Its sphere is supposed to be narrowed to the individual soul; but it cannot be thus separated in fact. It goes back to the beginnings of things, to the first growth of language and philosophy, and to the whole science of man. There can be no truth or completeness in any study of the mind which is confined to the individual. The nature of language, though not the whole, is perhaps at present the most important element in our knowledge of it. It is not impossible that some numerical laws may be found to have a place in the relations of mind and matter, as in the rest of nature. The old Pythagorean fancy that the soul 'is or has in it harmony' may in some degree be realized. But the indications of such numerical harmonies are faint; either the secret of them lies deeper than we can discover, or nature may have rebelled against the use of them in the composition of men and animals. It is with qualitative rather than with quantitative differences that we are concerned in Psychology. The facts relating to the mind which we obtain from Physiology are negative rather than positive. They show us, not the processes of mental action, but the conditions of which when deprived the mind ceases to act. It would seem as if the time had not yet arrived when we can hope to add anything of much importance to our knowledge of the mind from the investigations of the microscope. The elements of Psychology can still only be learnt from reflections on ourselves, which interpret and are also interpreted by our experience of others. The history of language, of philosophy, and religion, the great thoughts or inventions or discoveries which move mankind, furnish the larger moulds or outlines in which the human mind has been cast. From these the individual derives so much as he is able to comprehend or has the opportunity of learning.

\par 
\section{
      THEAETETUS
    } 
\par  Euclid and Terpsion meet in front of Euclid's house in Megara; they enter the house, and the dialogue is read to them by a servant.

\par \textbf{EUCLID}
\par   Have you only just arrived from the country, Terpsion?

\par \textbf{TERPSION}
\par   No, I came some time ago:  and I have been in the Agora looking for you, and wondering that I could not find you.

\par \textbf{EUCLID}
\par   But I was not in the city.

\par \textbf{TERPSION}
\par   Where then?

\par \textbf{EUCLID}
\par   As I was going down to the harbour, I met Theaetetus—he was being carried up to Athens from the army at Corinth.

\par \textbf{TERPSION}
\par   Was he alive or dead?

\par \textbf{EUCLID}
\par   He was scarcely alive, for he has been badly wounded; but he was suffering even more from the sickness which has broken out in the army.

\par \textbf{TERPSION}
\par   The dysentery, you mean?

\par \textbf{EUCLID}
\par   Yes.

\par \textbf{TERPSION}
\par   Alas! what a loss he will be!

\par \textbf{EUCLID}
\par   Yes, Terpsion, he is a noble fellow; only to-day I heard some people highly praising his behaviour in this very battle.

\par \textbf{TERPSION}
\par   No wonder; I should rather be surprised at hearing anything else of him. But why did he go on, instead of stopping at Megara?

\par \textbf{EUCLID}
\par   He wanted to get home:  although I entreated and advised him to remain, he would not listen to me; so I set him on his way, and turned back, and then I remembered what Socrates had said of him, and thought how remarkably this, like all his predictions, had been fulfilled. I believe that he had seen him a little before his own death, when Theaetetus was a youth, and he had a memorable conversation with him, which he repeated to me when I came to Athens; he was full of admiration of his genius, and said that he would most certainly be a great man, if he lived.

\par \textbf{TERPSION}
\par   The prophecy has certainly been fulfilled; but what was the conversation? can you tell me?

\par \textbf{EUCLID}
\par   No, indeed, not offhand; but I took notes of it as soon as I got home; these I filled up from memory, writing them out at leisure; and whenever I went to Athens, I asked Socrates about any point which I had forgotten, and on my return I made corrections; thus I have nearly the whole conversation written down.

\par \textbf{TERPSION}
\par   I remember—you told me; and I have always been intending to ask you to show me the writing, but have put off doing so; and now, why should we not read it through?—having just come from the country, I should greatly like to rest.

\par \textbf{EUCLID}
\par   I too shall be very glad of a rest, for I went with Theaetetus as far as Erineum. Let us go in, then, and, while we are reposing, the servant shall read to us.

\par \textbf{TERPSION}
\par   Very good.

\par \textbf{EUCLID}
\par   Here is the roll, Terpsion; I may observe that I have introduced Socrates, not as narrating to me, but as actually conversing with the persons whom he mentioned—these were, Theodorus the geometrician (of Cyrene), and Theaetetus. I have omitted, for the sake of convenience, the interlocutory words 'I said,' 'I remarked,' which he used when he spoke of himself, and again, 'he agreed,' or 'disagreed,' in the answer, lest the repetition of them should be troublesome.

\par \textbf{TERPSION}
\par   Quite right, Euclid.

\par \textbf{EUCLID}
\par   And now, boy, you may take the roll and read.

\par  EUCLID'S SERVANT READS.

\par \textbf{SOCRATES}
\par   If I cared enough about the Cyrenians, Theodorus, I would ask you whether there are any rising geometricians or philosophers in that part of the world. But I am more interested in our own Athenian youth, and I would rather know who among them are likely to do well. I observe them as far as I can myself, and I enquire of any one whom they follow, and I see that a great many of them follow you, in which they are quite right, considering your eminence in geometry and in other ways. Tell me then, if you have met with any one who is good for anything.

\par \textbf{THEODORUS}
\par   Yes, Socrates, I have become acquainted with one very remarkable Athenian youth, whom I commend to you as well worthy of your attention. If he had been a beauty I should have been afraid to praise him, lest you should suppose that I was in love with him; but he is no beauty, and you must not be offended if I say that he is very like you; for he has a snub nose and projecting eyes, although these features are less marked in him than in you. Seeing, then, that he has no personal attractions, I may freely say, that in all my acquaintance, which is very large, I never knew any one who was his equal in natural gifts:  for he has a quickness of apprehension which is almost unrivalled, and he is exceedingly gentle, and also the most courageous of men; there is a union of qualities in him such as I have never seen in any other, and should scarcely have thought possible; for those who, like him, have quick and ready and retentive wits, have generally also quick tempers; they are ships without ballast, and go darting about, and are mad rather than courageous; and the steadier sort, when they have to face study, prove stupid and cannot remember. Whereas he moves surely and smoothly and successfully in the path of knowledge and enquiry; and he is full of gentleness, flowing on silently like a river of oil; at his age, it is wonderful.

\par \textbf{SOCRATES}
\par   That is good news; whose son is he?

\par \textbf{THEODORUS}
\par   The name of his father I have forgotten, but the youth himself is the middle one of those who are approaching us; he and his companions have been anointing themselves in the outer court, and now they seem to have finished, and are coming towards us. Look and see whether you know him.

\par \textbf{SOCRATES}
\par   I know the youth, but I do not know his name; he is the son of Euphronius the Sunian, who was himself an eminent man, and such another as his son is, according to your account of him; I believe that he left a considerable fortune.

\par \textbf{THEODORUS}
\par   Theaetetus, Socrates, is his name; but I rather think that the property disappeared in the hands of trustees; notwithstanding which he is wonderfully liberal.

\par \textbf{SOCRATES}
\par   He must be a fine fellow; tell him to come and sit by me.

\par \textbf{THEODORUS}
\par   I will. Come hither, Theaetetus, and sit by Socrates.

\par \textbf{SOCRATES}
\par   By all means, Theaetetus, in order that I may see the reflection of myself in your face, for Theodorus says that we are alike; and yet if each of us held in his hands a lyre, and he said that they were tuned alike, should we at once take his word, or should we ask whether he who said so was or was not a musician?

\par \textbf{THEAETETUS}
\par   We should ask.

\par \textbf{SOCRATES}
\par   And if we found that he was, we should take his word; and if not, not?

\par \textbf{THEAETETUS}
\par   True.

\par \textbf{SOCRATES}
\par   And if this supposed likeness of our faces is a matter of any interest to us, we should enquire whether he who says that we are alike is a painter or not?

\par \textbf{THEAETETUS}
\par   Certainly we should.

\par \textbf{SOCRATES}
\par   And is Theodorus a painter?

\par \textbf{THEAETETUS}
\par   I never heard that he was.

\par \textbf{SOCRATES}
\par   Is he a geometrician?

\par \textbf{THEAETETUS}
\par   Of course he is, Socrates.

\par \textbf{SOCRATES}
\par   And is he an astronomer and calculator and musician, and in general an educated man?

\par \textbf{THEAETETUS}
\par   I think so.

\par \textbf{SOCRATES}
\par   If, then, he remarks on a similarity in our persons, either by way of praise or blame, there is no particular reason why we should attend to him.

\par \textbf{THEAETETUS}
\par   I should say not.

\par \textbf{SOCRATES}
\par   But if he praises the virtue or wisdom which are the mental endowments of either of us, then he who hears the praises will naturally desire to examine him who is praised:  and he again should be willing to exhibit himself.

\par \textbf{THEAETETUS}
\par   Very true, Socrates.

\par \textbf{SOCRATES}
\par   Then now is the time, my dear Theaetetus, for me to examine, and for you to exhibit; since although Theodorus has praised many a citizen and stranger in my hearing, never did I hear him praise any one as he has been praising you.

\par \textbf{THEAETETUS}
\par   I am glad to hear it, Socrates; but what if he was only in jest?

\par \textbf{SOCRATES}
\par   Nay, Theodorus is not given to jesting; and I cannot allow you to retract your consent on any such pretence as that. If you do, he will have to swear to his words; and we are perfectly sure that no one will be found to impugn him. Do not be shy then, but stand to your word.

\par \textbf{THEAETETUS}
\par   I suppose I must, if you wish it.

\par \textbf{SOCRATES}
\par   In the first place, I should like to ask what you learn of Theodorus:  something of geometry, perhaps?

\par \textbf{THEAETETUS}
\par   Yes.

\par \textbf{SOCRATES}
\par   And astronomy and harmony and calculation?

\par \textbf{THEAETETUS}
\par   I do my best.

\par \textbf{SOCRATES}
\par   Yes, my boy, and so do I; and my desire is to learn of him, or of anybody who seems to understand these things. And I get on pretty well in general; but there is a little difficulty which I want you and the company to aid me in investigating. Will you answer me a question:  'Is not learning growing wiser about that which you learn?'

\par \textbf{THEAETETUS}
\par   Of course.

\par \textbf{SOCRATES}
\par   And by wisdom the wise are wise?

\par \textbf{THEAETETUS}
\par   Yes.

\par \textbf{SOCRATES}
\par   And is that different in any way from knowledge?

\par \textbf{THEAETETUS}
\par   What?

\par \textbf{SOCRATES}
\par   Wisdom; are not men wise in that which they know?

\par \textbf{THEAETETUS}
\par   Certainly they are.

\par \textbf{SOCRATES}
\par   Then wisdom and knowledge are the same?

\par \textbf{THEAETETUS}
\par   Yes.

\par \textbf{SOCRATES}
\par   Herein lies the difficulty which I can never solve to my satisfaction—What is knowledge? Can we answer that question? What say you? which of us will speak first? whoever misses shall sit down, as at a game of ball, and shall be donkey, as the boys say; he who lasts out his competitors in the game without missing, shall be our king, and shall have the right of putting to us any questions which he pleases...Why is there no reply? I hope, Theodorus, that I am not betrayed into rudeness by my love of conversation? I only want to make us talk and be friendly and sociable.

\par \textbf{THEODORUS}
\par   The reverse of rudeness, Socrates:  but I would rather that you would ask one of the young fellows; for the truth is, that I am unused to your game of question and answer, and I am too old to learn; the young will be more suitable, and they will improve more than I shall, for youth is always able to improve. And so having made a beginning with Theaetetus, I would advise you to go on with him and not let him off.

\par \textbf{SOCRATES}
\par   Do you hear, Theaetetus, what Theodorus says? The philosopher, whom you would not like to disobey, and whose word ought to be a command to a young man, bids me interrogate you. Take courage, then, and nobly say what you think that knowledge is.

\par \textbf{THEAETETUS}
\par   Well, Socrates, I will answer as you and he bid me; and if I make a mistake, you will doubtless correct me.

\par \textbf{SOCRATES}
\par   We will, if we can.

\par \textbf{THEAETETUS}
\par   Then, I think that the sciences which I learn from Theodorus—geometry, and those which you just now mentioned—are knowledge; and I would include the art of the cobbler and other craftsmen; these, each and all of, them, are knowledge.

\par \textbf{SOCRATES}
\par   Too much, Theaetetus, too much; the nobility and liberality of your nature make you give many and diverse things, when I am asking for one simple thing.

\par \textbf{THEAETETUS}
\par   What do you mean, Socrates?

\par \textbf{SOCRATES}
\par   Perhaps nothing. I will endeavour, however, to explain what I believe to be my meaning:  When you speak of cobbling, you mean the art or science of making shoes?

\par \textbf{THEAETETUS}
\par   Just so.

\par \textbf{SOCRATES}
\par   And when you speak of carpentering, you mean the art of making wooden implements?

\par \textbf{THEAETETUS}
\par   I do.

\par \textbf{SOCRATES}
\par   In both cases you define the subject matter of each of the two arts?

\par \textbf{THEAETETUS}
\par   True.

\par \textbf{SOCRATES}
\par   But that, Theaetetus, was not the point of my question:  we wanted to know not the subjects, nor yet the number of the arts or sciences, for we were not going to count them, but we wanted to know the nature of knowledge in the abstract. Am I not right?

\par \textbf{THEAETETUS}
\par   Perfectly right.

\par \textbf{SOCRATES}
\par   Let me offer an illustration:  Suppose that a person were to ask about some very trivial and obvious thing—for example, What is clay? and we were to reply, that there is a clay of potters, there is a clay of oven-makers, there is a clay of brick-makers; would not the answer be ridiculous?

\par \textbf{THEAETETUS}
\par   Truly.

\par \textbf{SOCRATES}
\par   In the first place, there would be an absurdity in assuming that he who asked the question would understand from our answer the nature of 'clay,' merely because we added 'of the image-makers,' or of any other workers. How can a man understand the name of anything, when he does not know the nature of it?

\par \textbf{THEAETETUS}
\par   He cannot.

\par \textbf{SOCRATES}
\par   Then he who does not know what science or knowledge is, has no knowledge of the art or science of making shoes?

\par \textbf{THEAETETUS}
\par   None.

\par \textbf{SOCRATES}
\par   Nor of any other science?

\par \textbf{THEAETETUS}
\par   No.

\par \textbf{SOCRATES}
\par   And when a man is asked what science or knowledge is, to give in answer the name of some art or science is ridiculous; for the question is, 'What is knowledge?' and he replies, 'A knowledge of this or that.'

\par \textbf{THEAETETUS}
\par   True.

\par \textbf{SOCRATES}
\par   Moreover, he might answer shortly and simply, but he makes an enormous circuit. For example, when asked about the clay, he might have said simply, that clay is moistened earth—what sort of clay is not to the point.

\par \textbf{THEAETETUS}
\par   Yes, Socrates, there is no difficulty as you put the question. You mean, if I am not mistaken, something like what occurred to me and to my friend here, your namesake Socrates, in a recent discussion.

\par \textbf{SOCRATES}
\par   What was that, Theaetetus?

\par \textbf{THEAETETUS}
\par   Theodorus was writing out for us something about roots, such as the roots of three or five, showing that they are incommensurable by the unit:  he selected other examples up to seventeen—there he stopped. Now as there are innumerable roots, the notion occurred to us of attempting to include them all under one name or class.

\par \textbf{SOCRATES}
\par   And did you find such a class?

\par \textbf{THEAETETUS}
\par   I think that we did; but I should like to have your opinion.

\par \textbf{SOCRATES}
\par   Let me hear.

\par \textbf{THEAETETUS}
\par   We divided all numbers into two classes:  those which are made up of equal factors multiplying into one another, which we compared to square figures and called square or equilateral numbers;—that was one class.

\par \textbf{SOCRATES}
\par   Very good.

\par \textbf{THEAETETUS}
\par   The intermediate numbers, such as three and five, and every other number which is made up of unequal factors, either of a greater multiplied by a less, or of a less multiplied by a greater, and when regarded as a figure, is contained in unequal sides;—all these we compared to oblong figures, and called them oblong numbers.

\par \textbf{SOCRATES}
\par   Capital; and what followed?

\par \textbf{THEAETETUS}
\par   The lines, or sides, which have for their squares the equilateral plane numbers, were called by us lengths or magnitudes; and the lines which are the roots of (or whose squares are equal to) the oblong numbers, were called powers or roots; the reason of this latter name being, that they are commensurable with the former [i.e., with the so-called lengths or magnitudes] not in linear measurement, but in the value of the superficial content of their squares; and the same about solids.

\par \textbf{SOCRATES}
\par   Excellent, my boys; I think that you fully justify the praises of Theodorus, and that he will not be found guilty of false witness.

\par \textbf{THEAETETUS}
\par   But I am unable, Socrates, to give you a similar answer about knowledge, which is what you appear to want; and therefore Theodorus is a deceiver after all.

\par \textbf{SOCRATES}
\par   Well, but if some one were to praise you for running, and to say that he never met your equal among boys, and afterwards you were beaten in a race by a grown-up man, who was a great runner—would the praise be any the less true?

\par \textbf{THEAETETUS}
\par   Certainly not.

\par \textbf{SOCRATES}
\par   And is the discovery of the nature of knowledge so small a matter, as just now said? Is it not one which would task the powers of men perfect in every way?

\par \textbf{THEAETETUS}
\par   By heaven, they should be the top of all perfection!

\par \textbf{SOCRATES}
\par   Well, then, be of good cheer; do not say that Theodorus was mistaken about you, but do your best to ascertain the true nature of knowledge, as well as of other things.

\par \textbf{THEAETETUS}
\par   I am eager enough, Socrates, if that would bring to light the truth.

\par \textbf{SOCRATES}
\par   Come, you made a good beginning just now; let your own answer about roots be your model, and as you comprehended them all in one class, try and bring the many sorts of knowledge under one definition.

\par \textbf{THEAETETUS}
\par   I can assure you, Socrates, that I have tried very often, when the report of questions asked by you was brought to me; but I can neither persuade myself that I have a satisfactory answer to give, nor hear of any one who answers as you would have him; and I cannot shake off a feeling of anxiety.

\par \textbf{SOCRATES}
\par   These are the pangs of labour, my dear Theaetetus; you have something within you which you are bringing to the birth.

\par \textbf{THEAETETUS}
\par   I do not know, Socrates; I only say what I feel.

\par \textbf{SOCRATES}
\par   And have you never heard, simpleton, that I am the son of a midwife, brave and burly, whose name was Phaenarete?

\par \textbf{THEAETETUS}
\par   Yes, I have.

\par \textbf{SOCRATES}
\par   And that I myself practise midwifery?

\par \textbf{THEAETETUS}
\par   No, never.

\par \textbf{SOCRATES}
\par   Let me tell you that I do though, my friend:  but you must not reveal the secret, as the world in general have not found me out; and therefore they only say of me, that I am the strangest of mortals and drive men to their wits' end. Did you ever hear that too?

\par \textbf{THEAETETUS}
\par   Yes.

\par \textbf{SOCRATES}
\par   Shall I tell you the reason?

\par \textbf{THEAETETUS}
\par   By all means.

\par \textbf{SOCRATES}
\par   Bear in mind the whole business of the midwives, and then you will see my meaning better: —No woman, as you are probably aware, who is still able to conceive and bear, attends other women, but only those who are past bearing.

\par \textbf{THEAETETUS}
\par   Yes, I know.

\par \textbf{SOCRATES}
\par   The reason of this is said to be that Artemis—the goddess of childbirth—is not a mother, and she honours those who are like herself; but she could not allow the barren to be midwives, because human nature cannot know the mystery of an art without experience; and therefore she assigned this office to those who are too old to bear.

\par \textbf{THEAETETUS}
\par   I dare say.

\par \textbf{SOCRATES}
\par   And I dare say too, or rather I am absolutely certain, that the midwives know better than others who is pregnant and who is not?

\par \textbf{THEAETETUS}
\par   Very true.

\par \textbf{SOCRATES}
\par   And by the use of potions and incantations they are able to arouse the pangs and to soothe them at will; they can make those bear who have a difficulty in bearing, and if they think fit they can smother the embryo in the womb.

\par \textbf{THEAETETUS}
\par   They can.

\par \textbf{SOCRATES}
\par   Did you ever remark that they are also most cunning matchmakers, and have a thorough knowledge of what unions are likely to produce a brave brood?

\par \textbf{THEAETETUS}
\par   No, never.

\par \textbf{SOCRATES}
\par   Then let me tell you that this is their greatest pride, more than cutting the umbilical cord. And if you reflect, you will see that the same art which cultivates and gathers in the fruits of the earth, will be most likely to know in what soils the several plants or seeds should be deposited.

\par \textbf{THEAETETUS}
\par   Yes, the same art.

\par \textbf{SOCRATES}
\par   And do you suppose that with women the case is otherwise?

\par \textbf{THEAETETUS}
\par   I should think not.

\par \textbf{SOCRATES}
\par   Certainly not; but midwives are respectable women who have a character to lose, and they avoid this department of their profession, because they are afraid of being called procuresses, which is a name given to those who join together man and woman in an unlawful and unscientific way; and yet the true midwife is also the true and only matchmaker.

\par \textbf{THEAETETUS}
\par   Clearly.

\par \textbf{SOCRATES}
\par   Such are the midwives, whose task is a very important one, but not so important as mine; for women do not bring into the world at one time real children, and at another time counterfeits which are with difficulty distinguished from them; if they did, then the discernment of the true and false birth would be the crowning achievement of the art of midwifery—you would think so?

\par \textbf{THEAETETUS}
\par   Indeed I should.

\par \textbf{SOCRATES}
\par   Well, my art of midwifery is in most respects like theirs; but differs, in that I attend men and not women; and look after their souls when they are in labour, and not after their bodies:  and the triumph of my art is in thoroughly examining whether the thought which the mind of the young man brings forth is a false idol or a noble and true birth. And like the midwives, I am barren, and the reproach which is often made against me, that I ask questions of others and have not the wit to answer them myself, is very just—the reason is, that the god compels me to be a midwife, but does not allow me to bring forth. And therefore I am not myself at all wise, nor have I anything to show which is the invention or birth of my own soul, but those who converse with me profit. Some of them appear dull enough at first, but afterwards, as our acquaintance ripens, if the god is gracious to them, they all make astonishing progress; and this in the opinion of others as well as in their own. It is quite clear that they never learned anything from me; the many fine discoveries to which they cling are of their own making. But to me and the god they owe their delivery. And the proof of my words is, that many of them in their ignorance, either in their self-conceit despising me, or falling under the influence of others, have gone away too soon; and have not only lost the children of whom I had previously delivered them by an ill bringing up, but have stifled whatever else they had in them by evil communications, being fonder of lies and shams than of the truth; and they have at last ended by seeing themselves, as others see them, to be great fools. Aristeides, the son of Lysimachus, is one of them, and there are many others. The truants often return to me, and beg that I would consort with them again—they are ready to go to me on their knees—and then, if my familiar allows, which is not always the case, I receive them, and they begin to grow again. Dire are the pangs which my art is able to arouse and to allay in those who consort with me, just like the pangs of women in childbirth; night and day they are full of perplexity and travail which is even worse than that of the women. So much for them. And there are others, Theaetetus, who come to me apparently having nothing in them; and as I know that they have no need of my art, I coax them into marrying some one, and by the grace of God I can generally tell who is likely to do them good. Many of them I have given away to Prodicus, and many to other inspired sages. I tell you this long story, friend Theaetetus, because I suspect, as indeed you seem to think yourself, that you are in labour—great with some conception. Come then to me, who am a midwife's son and myself a midwife, and do your best to answer the questions which I will ask you. And if I abstract and expose your first-born, because I discover upon inspection that the conception which you have formed is a vain shadow, do not quarrel with me on that account, as the manner of women is when their first children are taken from them. For I have actually known some who were ready to bite me when I deprived them of a darling folly; they did not perceive that I acted from goodwill, not knowing that no god is the enemy of man—that was not within the range of their ideas; neither am I their enemy in all this, but it would be wrong for me to admit falsehood, or to stifle the truth. Once more, then, Theaetetus, I repeat my old question, 'What is knowledge? '—and do not say that you cannot tell; but quit yourself like a man, and by the help of God you will be able to tell.

\par \textbf{THEAETETUS}
\par   At any rate, Socrates, after such an exhortation I should be ashamed of not trying to do my best. Now he who knows perceives what he knows, and, as far as I can see at present, knowledge is perception.

\par \textbf{SOCRATES}
\par   Bravely said, boy; that is the way in which you should express your opinion. And now, let us examine together this conception of yours, and see whether it is a true birth or a mere wind-egg: —You say that knowledge is perception?

\par \textbf{THEAETETUS}
\par   Yes.

\par \textbf{SOCRATES}
\par   Well, you have delivered yourself of a very important doctrine about knowledge; it is indeed the opinion of Protagoras, who has another way of expressing it. Man, he says, is the measure of all things, of the existence of things that are, and of the non-existence of things that are not: —You have read him?

\par \textbf{THEAETETUS}
\par   O yes, again and again.

\par \textbf{SOCRATES}
\par   Does he not say that things are to you such as they appear to you, and to me such as they appear to me, and that you and I are men?

\par \textbf{THEAETETUS}
\par   Yes, he says so.

\par \textbf{SOCRATES}
\par   A wise man is not likely to talk nonsense. Let us try to understand him:  the same wind is blowing, and yet one of us may be cold and the other not, or one may be slightly and the other very cold?

\par \textbf{THEAETETUS}
\par   Quite true.

\par \textbf{SOCRATES}
\par   Now is the wind, regarded not in relation to us but absolutely, cold or not; or are we to say, with Protagoras, that the wind is cold to him who is cold, and not to him who is not?

\par \textbf{THEAETETUS}
\par   I suppose the last.

\par \textbf{SOCRATES}
\par   Then it must appear so to each of them?

\par \textbf{THEAETETUS}
\par   Yes.

\par \textbf{SOCRATES}
\par   And 'appears to him' means the same as 'he perceives.'

\par \textbf{THEAETETUS}
\par   True.

\par \textbf{SOCRATES}
\par   Then appearing and perceiving coincide in the case of hot and cold, and in similar instances; for things appear, or may be supposed to be, to each one such as he perceives them?

\par \textbf{THEAETETUS}
\par   Yes.

\par \textbf{SOCRATES}
\par   Then perception is always of existence, and being the same as knowledge is unerring?

\par \textbf{THEAETETUS}
\par   Clearly.

\par \textbf{SOCRATES}
\par   In the name of the Graces, what an almighty wise man Protagoras must have been! He spoke these things in a parable to the common herd, like you and me, but told the truth, 'his Truth,' (In allusion to a book of Protagoras' which bore this title.) in secret to his own disciples.

\par \textbf{THEAETETUS}
\par   What do you mean, Socrates?

\par \textbf{SOCRATES}
\par   I am about to speak of a high argument, in which all things are said to be relative; you cannot rightly call anything by any name, such as great or small, heavy or light, for the great will be small and the heavy light—there is no single thing or quality, but out of motion and change and admixture all things are becoming relatively to one another, which 'becoming' is by us incorrectly called being, but is really becoming, for nothing ever is, but all things are becoming. Summon all philosophers—Protagoras, Heracleitus, Empedocles, and the rest of them, one after another, and with the exception of Parmenides they will agree with you in this. Summon the great masters of either kind of poetry—Epicharmus, the prince of Comedy, and Homer of Tragedy; when the latter sings of

\par  'Ocean whence sprang the gods, and mother Tethys,'

\par  does he not mean that all things are the offspring, of flux and motion?

\par \textbf{THEAETETUS}
\par   I think so.

\par \textbf{SOCRATES}
\par   And who could take up arms against such a great army having Homer for its general, and not appear ridiculous? (Compare Cratylus.)

\par \textbf{THEAETETUS}
\par   Who indeed, Socrates?

\par \textbf{SOCRATES}
\par   Yes, Theaetetus; and there are plenty of other proofs which will show that motion is the source of what is called being and becoming, and inactivity of not-being and destruction; for fire and warmth, which are supposed to be the parent and guardian of all other things, are born of movement and of friction, which is a kind of motion;—is not this the origin of fire?

\par \textbf{THEAETETUS}
\par   It is.

\par \textbf{SOCRATES}
\par   And the race of animals is generated in the same way?

\par \textbf{THEAETETUS}
\par   Certainly.

\par \textbf{SOCRATES}
\par   And is not the bodily habit spoiled by rest and idleness, but preserved for a long time by motion and exercise?

\par \textbf{THEAETETUS}
\par   True.

\par \textbf{SOCRATES}
\par   And what of the mental habit? Is not the soul informed, and improved, and preserved by study and attention, which are motions; but when at rest, which in the soul only means want of attention and study, is uninformed, and speedily forgets whatever she has learned?

\par \textbf{THEAETETUS}
\par   True.

\par \textbf{SOCRATES}
\par   Then motion is a good, and rest an evil, to the soul as well as to the body?

\par \textbf{THEAETETUS}
\par   Clearly.

\par \textbf{SOCRATES}
\par   I may add, that breathless calm, stillness and the like waste and impair, while wind and storm preserve; and the palmary argument of all, which I strongly urge, is the golden chain in Homer, by which he means the sun, thereby indicating that so long as the sun and the heavens go round in their orbits, all things human and divine are and are preserved, but if they were chained up and their motions ceased, then all things would be destroyed, and, as the saying is, turned upside down.

\par \textbf{THEAETETUS}
\par   I believe, Socrates, that you have truly explained his meaning.

\par \textbf{SOCRATES}
\par   Then now apply his doctrine to perception, my good friend, and first of all to vision; that which you call white colour is not in your eyes, and is not a distinct thing which exists out of them. And you must not assign any place to it:  for if it had position it would be, and be at rest, and there would be no process of becoming.

\par \textbf{THEAETETUS}
\par   Then what is colour?

\par \textbf{SOCRATES}
\par   Let us carry the principle which has just been affirmed, that nothing is self-existent, and then we shall see that white, black, and every other colour, arises out of the eye meeting the appropriate motion, and that what we call a colour is in each case neither the active nor the passive element, but something which passes between them, and is peculiar to each percipient; are you quite certain that the several colours appear to a dog or to any animal whatever as they appear to you?

\par \textbf{THEAETETUS}
\par   Far from it.

\par \textbf{SOCRATES}
\par   Or that anything appears the same to you as to another man? Are you so profoundly convinced of this? Rather would it not be true that it never appears exactly the same to you, because you are never exactly the same?

\par \textbf{THEAETETUS}
\par   The latter.

\par \textbf{SOCRATES}
\par   And if that with which I compare myself in size, or which I apprehend by touch, were great or white or hot, it could not become different by mere contact with another unless it actually changed; nor again, if the comparing or apprehending subject were great or white or hot, could this, when unchanged from within, become changed by any approximation or affection of any other thing. The fact is that in our ordinary way of speaking we allow ourselves to be driven into most ridiculous and wonderful contradictions, as Protagoras and all who take his line of argument would remark.

\par \textbf{THEAETETUS}
\par   How? and of what sort do you mean?

\par \textbf{SOCRATES}
\par   A little instance will sufficiently explain my meaning:  Here are six dice, which are more by a half when compared with four, and fewer by a half than twelve—they are more and also fewer. How can you or any one maintain the contrary?

\par \textbf{THEAETETUS}
\par   Very true.

\par \textbf{SOCRATES}
\par   Well, then, suppose that Protagoras or some one asks whether anything can become greater or more if not by increasing, how would you answer him, Theaetetus?

\par \textbf{THEAETETUS}
\par   I should say 'No,' Socrates, if I were to speak my mind in reference to this last question, and if I were not afraid of contradicting my former answer.

\par \textbf{SOCRATES}
\par   Capital! excellent! spoken like an oracle, my boy! And if you reply 'Yes,' there will be a case for Euripides; for our tongue will be unconvinced, but not our mind. (In allusion to the well-known line of Euripides, Hippol. :  e gloss omomoch e de thren anomotos.)

\par \textbf{THEAETETUS}
\par   Very true.

\par \textbf{SOCRATES}
\par   The thoroughbred Sophists, who know all that can be known about the mind, and argue only out of the superfluity of their wits, would have had a regular sparring-match over this, and would have knocked their arguments together finely. But you and I, who have no professional aims, only desire to see what is the mutual relation of these principles,—whether they are consistent with each or not.

\par \textbf{THEAETETUS}
\par   Yes, that would be my desire.

\par \textbf{SOCRATES}
\par   And mine too. But since this is our feeling, and there is plenty of time, why should we not calmly and patiently review our own thoughts, and thoroughly examine and see what these appearances in us really are? If I am not mistaken, they will be described by us as follows: —first, that nothing can become greater or less, either in number or magnitude, while remaining equal to itself—you would agree?

\par \textbf{THEAETETUS}
\par   Yes.

\par \textbf{SOCRATES}
\par   Secondly, that without addition or subtraction there is no increase or diminution of anything, but only equality.

\par \textbf{THEAETETUS}
\par   Quite true.

\par \textbf{SOCRATES}
\par   Thirdly, that what was not before cannot be afterwards, without becoming and having become.

\par \textbf{THEAETETUS}
\par   Yes, truly.

\par \textbf{SOCRATES}
\par   These three axioms, if I am not mistaken, are fighting with one another in our minds in the case of the dice, or, again, in such a case as this—if I were to say that I, who am of a certain height and taller than you, may within a year, without gaining or losing in height, be not so tall—not that I should have lost, but that you would have increased. In such a case, I am afterwards what I once was not, and yet I have not become; for I could not have become without becoming, neither could I have become less without losing somewhat of my height; and I could give you ten thousand examples of similar contradictions, if we admit them at all. I believe that you follow me, Theaetetus; for I suspect that you have thought of these questions before now.

\par \textbf{THEAETETUS}
\par   Yes, Socrates, and I am amazed when I think of them; by the Gods I am! and I want to know what on earth they mean; and there are times when my head quite swims with the contemplation of them.

\par \textbf{SOCRATES}
\par   I see, my dear Theaetetus, that Theodorus had a true insight into your nature when he said that you were a philosopher, for wonder is the feeling of a philosopher, and philosophy begins in wonder. He was not a bad genealogist who said that Iris (the messenger of heaven) is the child of Thaumas (wonder). But do you begin to see what is the explanation of this perplexity on the hypothesis which we attribute to Protagoras?

\par \textbf{THEAETETUS}
\par   Not as yet.

\par \textbf{SOCRATES}
\par   Then you will be obliged to me if I help you to unearth the hidden 'truth' of a famous man or school.

\par \textbf{THEAETETUS}
\par   To be sure, I shall be very much obliged.

\par \textbf{SOCRATES}
\par   Take a look round, then, and see that none of the uninitiated are listening. Now by the uninitiated I mean the people who believe in nothing but what they can grasp in their hands, and who will not allow that action or generation or anything invisible can have real existence.

\par \textbf{THEAETETUS}
\par   Yes, indeed, Socrates, they are very hard and impenetrable mortals.

\par \textbf{SOCRATES}
\par   Yes, my boy, outer barbarians. Far more ingenious are the brethren whose mysteries I am about to reveal to you. Their first principle is, that all is motion, and upon this all the affections of which we were just now speaking are supposed to depend:  there is nothing but motion, which has two forms, one active and the other passive, both in endless number; and out of the union and friction of them there is generated a progeny endless in number, having two forms, sense and the object of sense, which are ever breaking forth and coming to the birth at the same moment. The senses are variously named hearing, seeing, smelling; there is the sense of heat, cold, pleasure, pain, desire, fear, and many more which have names, as well as innumerable others which are without them; each has its kindred object,—each variety of colour has a corresponding variety of sight, and so with sound and hearing, and with the rest of the senses and the objects akin to them. Do you see, Theaetetus, the bearings of this tale on the preceding argument?

\par \textbf{THEAETETUS}
\par   Indeed I do not.

\par \textbf{SOCRATES}
\par   Then attend, and I will try to finish the story. The purport is that all these things are in motion, as I was saying, and that this motion is of two kinds, a slower and a quicker; and the slower elements have their motions in the same place and with reference to things near them, and so they beget; but what is begotten is swifter, for it is carried to fro, and moves from place to place. Apply this to sense: —When the eye and the appropriate object meet together and give birth to whiteness and the sensation connatural with it, which could not have been given by either of them going elsewhere, then, while the sight is flowing from the eye, whiteness proceeds from the object which combines in producing the colour; and so the eye is fulfilled with sight, and really sees, and becomes, not sight, but a seeing eye; and the object which combined to form the colour is fulfilled with whiteness, and becomes not whiteness but a white thing, whether wood or stone or whatever the object may be which happens to be coloured white. And this is true of all sensible objects, hard, warm, and the like, which are similarly to be regarded, as I was saying before, not as having any absolute existence, but as being all of them of whatever kind generated by motion in their intercourse with one another; for of the agent and patient, as existing in separation, no trustworthy conception, as they say, can be formed, for the agent has no existence until united with the patient, and the patient has no existence until united with the agent; and that which by uniting with something becomes an agent, by meeting with some other thing is converted into a patient. And from all these considerations, as I said at first, there arises a general reflection, that there is no one self-existent thing, but everything is becoming and in relation; and being must be altogether abolished, although from habit and ignorance we are compelled even in this discussion to retain the use of the term. But great philosophers tell us that we are not to allow either the word 'something,' or 'belonging to something,' or 'to me,' or 'this,' or 'that,' or any other detaining name to be used, in the language of nature all things are being created and destroyed, coming into being and passing into new forms; nor can any name fix or detain them; he who attempts to fix them is easily refuted. And this should be the way of speaking, not only of particulars but of aggregates; such aggregates as are expressed in the word 'man,' or 'stone,' or any name of an animal or of a class. O Theaetetus, are not these speculations sweet as honey? And do you not like the taste of them in the mouth?

\par \textbf{THEAETETUS}
\par   I do not know what to say, Socrates; for, indeed, I cannot make out whether you are giving your own opinion or only wanting to draw me out.

\par \textbf{SOCRATES}
\par   You forget, my friend, that I neither know, nor profess to know, anything of these matters; you are the person who is in labour, I am the barren midwife; and this is why I soothe you, and offer you one good thing after another, that you may taste them. And I hope that I may at last help to bring your own opinion into the light of day:  when this has been accomplished, then we will determine whether what you have brought forth is only a wind-egg or a real and genuine birth. Therefore, keep up your spirits, and answer like a man what you think.

\par \textbf{THEAETETUS}
\par   Ask me.

\par \textbf{SOCRATES}
\par   Then once more:  Is it your opinion that nothing is but what becomes?—the good and the noble, as well as all the other things which we were just now mentioning?

\par \textbf{THEAETETUS}
\par   When I hear you discoursing in this style, I think that there is a great deal in what you say, and I am very ready to assent.

\par \textbf{SOCRATES}
\par   Let us not leave the argument unfinished, then; for there still remains to be considered an objection which may be raised about dreams and diseases, in particular about madness, and the various illusions of hearing and sight, or of other senses. For you know that in all these cases the esse-percipi theory appears to be unmistakably refuted, since in dreams and illusions we certainly have false perceptions; and far from saying that everything is which appears, we should rather say that nothing is which appears.

\par \textbf{THEAETETUS}
\par   Very true, Socrates.

\par \textbf{SOCRATES}
\par   But then, my boy, how can any one contend that knowledge is perception, or that to every man what appears is?

\par \textbf{THEAETETUS}
\par   I am afraid to say, Socrates, that I have nothing to answer, because you rebuked me just now for making this excuse; but I certainly cannot undertake to argue that madmen or dreamers think truly, when they imagine, some of them that they are gods, and others that they can fly, and are flying in their sleep.

\par \textbf{SOCRATES}
\par   Do you see another question which can be raised about these phenomena, notably about dreaming and waking?

\par \textbf{THEAETETUS}
\par   What question?

\par \textbf{SOCRATES}
\par   A question which I think that you must often have heard persons ask: —How can you determine whether at this moment we are sleeping, and all our thoughts are a dream; or whether we are awake, and talking to one another in the waking state?

\par \textbf{THEAETETUS}
\par   Indeed, Socrates, I do not know how to prove the one any more than the other, for in both cases the facts precisely correspond;—and there is no difficulty in supposing that during all this discussion we have been talking to one another in a dream; and when in a dream we seem to be narrating dreams, the resemblance of the two states is quite astonishing.

\par \textbf{SOCRATES}
\par   You see, then, that a doubt about the reality of sense is easily raised, since there may even be a doubt whether we are awake or in a dream. And as our time is equally divided between sleeping and waking, in either sphere of existence the soul contends that the thoughts which are present to our minds at the time are true; and during one half of our lives we affirm the truth of the one, and, during the other half, of the other; and are equally confident of both.

\par \textbf{THEAETETUS}
\par   Most true.

\par \textbf{SOCRATES}
\par   And may not the same be said of madness and other disorders? the difference is only that the times are not equal.

\par \textbf{THEAETETUS}
\par   Certainly.

\par \textbf{SOCRATES}
\par   And is truth or falsehood to be determined by duration of time?

\par \textbf{THEAETETUS}
\par   That would be in many ways ridiculous.

\par \textbf{SOCRATES}
\par   But can you certainly determine by any other means which of these opinions is true?

\par \textbf{THEAETETUS}
\par   I do not think that I can.

\par \textbf{SOCRATES}
\par   Listen, then, to a statement of the other side of the argument, which is made by the champions of appearance. They would say, as I imagine—Can that which is wholly other than something, have the same quality as that from which it differs? and observe, Theaetetus, that the word 'other' means not 'partially,' but 'wholly other.'

\par \textbf{THEAETETUS}
\par   Certainly, putting the question as you do, that which is wholly other cannot either potentially or in any other way be the same.

\par \textbf{SOCRATES}
\par   And must therefore be admitted to be unlike?

\par \textbf{THEAETETUS}
\par   True.

\par \textbf{SOCRATES}
\par   If, then, anything happens to become like or unlike itself or another, when it becomes like we call it the same—when unlike, other?

\par \textbf{THEAETETUS}
\par   Certainly.

\par \textbf{SOCRATES}
\par   Were we not saying that there are agents many and infinite, and patients many and infinite?

\par \textbf{THEAETETUS}
\par   Yes.

\par \textbf{SOCRATES}
\par   And also that different combinations will produce results which are not the same, but different?

\par \textbf{THEAETETUS}
\par   Certainly.

\par \textbf{SOCRATES}
\par   Let us take you and me, or anything as an example: —There is Socrates in health, and Socrates sick—Are they like or unlike?

\par \textbf{THEAETETUS}
\par   You mean to compare Socrates in health as a whole, and Socrates in sickness as a whole?

\par \textbf{SOCRATES}
\par   Exactly; that is my meaning.

\par \textbf{THEAETETUS}
\par   I answer, they are unlike.

\par \textbf{SOCRATES}
\par   And if unlike, they are other?

\par \textbf{THEAETETUS}
\par   Certainly.

\par \textbf{SOCRATES}
\par   And would you not say the same of Socrates sleeping and waking, or in any of the states which we were mentioning?

\par \textbf{THEAETETUS}
\par   I should.

\par \textbf{SOCRATES}
\par   All agents have a different patient in Socrates, accordingly as he is well or ill.

\par \textbf{THEAETETUS}
\par   Of course.

\par \textbf{SOCRATES}
\par   And I who am the patient, and that which is the agent, will produce something different in each of the two cases?

\par \textbf{THEAETETUS}
\par   Certainly.

\par \textbf{SOCRATES}
\par   The wine which I drink when I am in health, appears sweet and pleasant to me?

\par \textbf{THEAETETUS}
\par   True.

\par \textbf{SOCRATES}
\par   For, as has been already acknowledged, the patient and agent meet together and produce sweetness and a perception of sweetness, which are in simultaneous motion, and the perception which comes from the patient makes the tongue percipient, and the quality of sweetness which arises out of and is moving about the wine, makes the wine both to be and to appear sweet to the healthy tongue.

\par \textbf{THEAETETUS}
\par   Certainly; that has been already acknowledged.

\par \textbf{SOCRATES}
\par   But when I am sick, the wine really acts upon another and a different person?

\par \textbf{THEAETETUS}
\par   Yes.

\par \textbf{SOCRATES}
\par   The combination of the draught of wine, and the Socrates who is sick, produces quite another result; which is the sensation of bitterness in the tongue, and the motion and creation of bitterness in and about the wine, which becomes not bitterness but something bitter; as I myself become not perception but percipient?

\par \textbf{THEAETETUS}
\par   True.

\par \textbf{SOCRATES}
\par   There is no other object of which I shall ever have the same perception, for another object would give another perception, and would make the percipient other and different; nor can that object which affects me, meeting another subject, produce the same, or become similar, for that too would produce another result from another subject, and become different.

\par \textbf{THEAETETUS}
\par   True.

\par \textbf{SOCRATES}
\par   Neither can I by myself, have this sensation, nor the object by itself, this quality.

\par \textbf{THEAETETUS}
\par   Certainly not.

\par \textbf{SOCRATES}
\par   When I perceive I must become percipient of something—there can be no such thing as perceiving and perceiving nothing; the object, whether it become sweet, bitter, or of any other quality, must have relation to a percipient; nothing can become sweet which is sweet to no one.

\par \textbf{THEAETETUS}
\par   Certainly not.

\par \textbf{SOCRATES}
\par   Then the inference is, that we (the agent and patient) are or become in relation to one another; there is a law which binds us one to the other, but not to any other existence, nor each of us to himself; and therefore we can only be bound to one another; so that whether a person says that a thing is or becomes, he must say that it is or becomes to or of or in relation to something else; but he must not say or allow any one else to say that anything is or becomes absolutely: —such is our conclusion.

\par \textbf{THEAETETUS}
\par   Very true, Socrates.

\par \textbf{SOCRATES}
\par   Then, if that which acts upon me has relation to me and to no other, I and no other am the percipient of it?

\par \textbf{THEAETETUS}
\par   Of course.

\par \textbf{SOCRATES}
\par   Then my perception is true to me, being inseparable from my own being; and, as Protagoras says, to myself I am judge of what is and what is not to me.

\par \textbf{THEAETETUS}
\par   I suppose so.

\par \textbf{SOCRATES}
\par   How then, if I never err, and if my mind never trips in the conception of being or becoming, can I fail of knowing that which I perceive?

\par \textbf{THEAETETUS}
\par   You cannot.

\par \textbf{SOCRATES}
\par   Then you were quite right in affirming that knowledge is only perception; and the meaning turns out to be the same, whether with Homer and Heracleitus, and all that company, you say that all is motion and flux, or with the great sage Protagoras, that man is the measure of all things; or with Theaetetus, that, given these premises, perception is knowledge. Am I not right, Theaetetus, and is not this your new-born child, of which I have delivered you? What say you?

\par \textbf{THEAETETUS}
\par   I cannot but agree, Socrates.

\par \textbf{SOCRATES}
\par   Then this is the child, however he may turn out, which you and I have with difficulty brought into the world. And now that he is born, we must run round the hearth with him, and see whether he is worth rearing, or is only a wind-egg and a sham. Is he to be reared in any case, and not exposed? or will you bear to see him rejected, and not get into a passion if I take away your first-born?

\par \textbf{THEODORUS}
\par   Theaetetus will not be angry, for he is very good-natured. But tell me, Socrates, in heaven's name, is this, after all, not the truth?

\par \textbf{SOCRATES}
\par   You, Theodorus, are a lover of theories, and now you innocently fancy that I am a bag full of them, and can easily pull one out which will overthrow its predecessor. But you do not see that in reality none of these theories come from me; they all come from him who talks with me. I only know just enough to extract them from the wisdom of another, and to receive them in a spirit of fairness. And now I shall say nothing myself, but shall endeavour to elicit something from our young friend.

\par \textbf{THEODORUS}
\par   Do as you say, Socrates; you are quite right.

\par \textbf{SOCRATES}
\par   Shall I tell you, Theodorus, what amazes me in your acquaintance Protagoras?

\par \textbf{THEODORUS}
\par   What is it?

\par \textbf{SOCRATES}
\par   I am charmed with his doctrine, that what appears is to each one, but I wonder that he did not begin his book on Truth with a declaration that a pig or a dog-faced baboon, or some other yet stranger monster which has sensation, is the measure of all things; then he might have shown a magnificent contempt for our opinion of him by informing us at the outset that while we were reverencing him like a God for his wisdom he was no better than a tadpole, not to speak of his fellow-men—would not this have produced an overpowering effect? For if truth is only sensation, and no man can discern another's feelings better than he, or has any superior right to determine whether his opinion is true or false, but each, as we have several times repeated, is to himself the sole judge, and everything that he judges is true and right, why, my friend, should Protagoras be preferred to the place of wisdom and instruction, and deserve to be well paid, and we poor ignoramuses have to go to him, if each one is the measure of his own wisdom? Must he not be talking 'ad captandum' in all this? I say nothing of the ridiculous predicament in which my own midwifery and the whole art of dialectic is placed; for the attempt to supervise or refute the notions or opinions of others would be a tedious and enormous piece of folly, if to each man his own are right; and this must be the case if Protagoras' Truth is the real truth, and the philosopher is not merely amusing himself by giving oracles out of the shrine of his book.

\par \textbf{THEODORUS}
\par   He was a friend of mine, Socrates, as you were saying, and therefore I cannot have him refuted by my lips, nor can I oppose you when I agree with you; please, then, to take Theaetetus again; he seemed to answer very nicely.

\par \textbf{SOCRATES}
\par   If you were to go into a Lacedaemonian palestra, Theodorus, would you have a right to look on at the naked wrestlers, some of them making a poor figure, if you did not strip and give them an opportunity of judging of your own person?

\par \textbf{THEODORUS}
\par   Why not, Socrates, if they would allow me, as I think you will, in consideration of my age and stiffness; let some more supple youth try a fall with you, and do not drag me into the gymnasium.

\par \textbf{SOCRATES}
\par   Your will is my will, Theodorus, as the proverbial philosophers say, and therefore I will return to the sage Theaetetus:  Tell me, Theaetetus, in reference to what I was saying, are you not lost in wonder, like myself, when you find that all of a sudden you are raised to the level of the wisest of men, or indeed of the gods?—for you would assume the measure of Protagoras to apply to the gods as well as men?

\par \textbf{THEAETETUS}
\par   Certainly I should, and I confess to you that I am lost in wonder. At first hearing, I was quite satisfied with the doctrine, that whatever appears is to each one, but now the face of things has changed.

\par \textbf{SOCRATES}
\par   Why, my dear boy, you are young, and therefore your ear is quickly caught and your mind influenced by popular arguments. Protagoras, or some one speaking on his behalf, will doubtless say in reply,—Good people, young and old, you meet and harangue, and bring in the gods, whose existence or non-existence I banish from writing and speech, or you talk about the reason of man being degraded to the level of the brutes, which is a telling argument with the multitude, but not one word of proof or demonstration do you offer. All is probability with you, and yet surely you and Theodorus had better reflect whether you are disposed to admit of probability and figures of speech in matters of such importance. He or any other mathematician who argued from probabilities and likelihoods in geometry, would not be worth an ace.

\par \textbf{THEAETETUS}
\par   But neither you nor we, Socrates, would be satisfied with such arguments.

\par \textbf{SOCRATES}
\par   Then you and Theodorus mean to say that we must look at the matter in some other way?

\par \textbf{THEAETETUS}
\par   Yes, in quite another way.

\par \textbf{SOCRATES}
\par   And the way will be to ask whether perception is or is not the same as knowledge; for this was the real point of our argument, and with a view to this we raised (did we not?) those many strange questions.

\par \textbf{THEAETETUS}
\par   Certainly.

\par \textbf{SOCRATES}
\par   Shall we say that we know every thing which we see and hear? for example, shall we say that not having learned, we do not hear the language of foreigners when they speak to us? or shall we say that we not only hear, but know what they are saying? Or again, if we see letters which we do not understand, shall we say that we do not see them? or shall we aver that, seeing them, we must know them?

\par \textbf{THEAETETUS}
\par   We shall say, Socrates, that we know what we actually see and hear of them—that is to say, we see and know the figure and colour of the letters, and we hear and know the elevation or depression of the sound of them; but we do not perceive by sight and hearing, or know, that which grammarians and interpreters teach about them.

\par \textbf{SOCRATES}
\par   Capital, Theaetetus; and about this there shall be no dispute, because I want you to grow; but there is another difficulty coming, which you will also have to repulse.

\par \textbf{THEAETETUS}
\par   What is it?

\par \textbf{SOCRATES}
\par   Some one will say, Can a man who has ever known anything, and still has and preserves a memory of that which he knows, not know that which he remembers at the time when he remembers? I have, I fear, a tedious way of putting a simple question, which is only, whether a man who has learned, and remembers, can fail to know?

\par \textbf{THEAETETUS}
\par   Impossible, Socrates; the supposition is monstrous.

\par \textbf{SOCRATES}
\par   Am I talking nonsense, then? Think:  is not seeing perceiving, and is not sight perception?

\par \textbf{THEAETETUS}
\par   True.

\par \textbf{SOCRATES}
\par   And if our recent definition holds, every man knows that which he has seen?

\par \textbf{THEAETETUS}
\par   Yes.

\par \textbf{SOCRATES}
\par   And you would admit that there is such a thing as memory?

\par \textbf{THEAETETUS}
\par   Yes.

\par \textbf{SOCRATES}
\par   And is memory of something or of nothing?

\par \textbf{THEAETETUS}
\par   Of something, surely.

\par \textbf{SOCRATES}
\par   Of things learned and perceived, that is?

\par \textbf{THEAETETUS}
\par   Certainly.

\par \textbf{SOCRATES}
\par   Often a man remembers that which he has seen?

\par \textbf{THEAETETUS}
\par   True.

\par \textbf{SOCRATES}
\par   And if he closed his eyes, would he forget?

\par \textbf{THEAETETUS}
\par   Who, Socrates, would dare to say so?

\par \textbf{SOCRATES}
\par   But we must say so, if the previous argument is to be maintained.

\par \textbf{THEAETETUS}
\par   What do you mean? I am not quite sure that I understand you, though I have a strong suspicion that you are right.

\par \textbf{SOCRATES}
\par   As thus:  he who sees knows, as we say, that which he sees; for perception and sight and knowledge are admitted to be the same.

\par \textbf{THEAETETUS}
\par   Certainly.

\par \textbf{SOCRATES}
\par   But he who saw, and has knowledge of that which he saw, remembers, when he closes his eyes, that which he no longer sees.

\par \textbf{THEAETETUS}
\par   True.

\par \textbf{SOCRATES}
\par   And seeing is knowing, and therefore not-seeing is not-knowing?

\par \textbf{THEAETETUS}
\par   Very true.

\par \textbf{SOCRATES}
\par   Then the inference is, that a man may have attained the knowledge of something, which he may remember and yet not know, because he does not see; and this has been affirmed by us to be a monstrous supposition.

\par \textbf{THEAETETUS}
\par   Most true.

\par \textbf{SOCRATES}
\par   Thus, then, the assertion that knowledge and perception are one, involves a manifest impossibility?

\par \textbf{THEAETETUS}
\par   Yes.

\par \textbf{SOCRATES}
\par   Then they must be distinguished?

\par \textbf{THEAETETUS}
\par   I suppose that they must.

\par \textbf{SOCRATES}
\par   Once more we shall have to begin, and ask 'What is knowledge?' and yet, Theaetetus, what are we going to do?

\par \textbf{THEAETETUS}
\par   About what?

\par \textbf{SOCRATES}
\par   Like a good-for-nothing cock, without having won the victory, we walk away from the argument and crow.

\par \textbf{THEAETETUS}
\par   How do you mean?

\par \textbf{SOCRATES}
\par   After the manner of disputers (Lys. ; Phaedo; Republic), we were satisfied with mere verbal consistency, and were well pleased if in this way we could gain an advantage. Although professing not to be mere Eristics, but philosophers, I suspect that we have unconsciously fallen into the error of that ingenious class of persons.

\par \textbf{THEAETETUS}
\par   I do not as yet understand you.

\par \textbf{SOCRATES}
\par   Then I will try to explain myself:  just now we asked the question, whether a man who had learned and remembered could fail to know, and we showed that a person who had seen might remember when he had his eyes shut and could not see, and then he would at the same time remember and not know. But this was an impossibility. And so the Protagorean fable came to nought, and yours also, who maintained that knowledge is the same as perception.

\par \textbf{THEAETETUS}
\par   True.

\par \textbf{SOCRATES}
\par   And yet, my friend, I rather suspect that the result would have been different if Protagoras, who was the father of the first of the two brats, had been alive; he would have had a great deal to say on their behalf. But he is dead, and we insult over his orphan child; and even the guardians whom he left, and of whom our friend Theodorus is one, are unwilling to give any help, and therefore I suppose that I must take up his cause myself, and see justice done?

\par \textbf{THEODORUS}
\par   Not I, Socrates, but rather Callias, the son of Hipponicus, is guardian of his orphans. I was too soon diverted from the abstractions of dialectic to geometry. Nevertheless, I shall be grateful to you if you assist him.

\par \textbf{SOCRATES}
\par   Very good, Theodorus; you shall see how I will come to the rescue. If a person does not attend to the meaning of terms as they are commonly used in argument, he may be involved even in greater paradoxes than these. Shall I explain this matter to you or to Theaetetus?

\par \textbf{THEODORUS}
\par   To both of us, and let the younger answer; he will incur less disgrace if he is discomfited.

\par \textbf{SOCRATES}
\par   Then now let me ask the awful question, which is this: —Can a man know and also not know that which he knows?

\par \textbf{THEODORUS}
\par   How shall we answer, Theaetetus?

\par \textbf{THEAETETUS}
\par   He cannot, I should say.

\par \textbf{SOCRATES}
\par   He can, if you maintain that seeing is knowing. When you are imprisoned in a well, as the saying is, and the self-assured adversary closes one of your eyes with his hand, and asks whether you can see his cloak with the eye which he has closed, how will you answer the inevitable man?

\par \textbf{THEAETETUS}
\par   I should answer, 'Not with that eye but with the other.'

\par \textbf{SOCRATES}
\par   Then you see and do not see the same thing at the same time.

\par \textbf{THEAETETUS}
\par   Yes, in a certain sense.

\par \textbf{SOCRATES}
\par   None of that, he will reply; I do not ask or bid you answer in what sense you know, but only whether you know that which you do not know. You have been proved to see that which you do not see; and you have already admitted that seeing is knowing, and that not-seeing is not-knowing:  I leave you to draw the inference.

\par \textbf{THEAETETUS}
\par   Yes; the inference is the contradictory of my assertion.

\par \textbf{SOCRATES}
\par   Yes, my marvel, and there might have been yet worse things in store for you, if an opponent had gone on to ask whether you can have a sharp and also a dull knowledge, and whether you can know near, but not at a distance, or know the same thing with more or less intensity, and so on without end. Such questions might have been put to you by a light-armed mercenary, who argued for pay. He would have lain in wait for you, and when you took up the position, that sense is knowledge, he would have made an assault upon hearing, smelling, and the other senses;—he would have shown you no mercy; and while you were lost in envy and admiration of his wisdom, he would have got you into his net, out of which you would not have escaped until you had come to an understanding about the sum to be paid for your release. Well, you ask, and how will Protagoras reinforce his position? Shall I answer for him?

\par \textbf{THEAETETUS}
\par   By all means.

\par \textbf{SOCRATES}
\par   He will repeat all those things which we have been urging on his behalf, and then he will close with us in disdain, and say: —The worthy Socrates asked a little boy, whether the same man could remember and not know the same thing, and the boy said No, because he was frightened, and could not see what was coming, and then Socrates made fun of poor me. The truth is, O slatternly Socrates, that when you ask questions about any assertion of mine, and the person asked is found tripping, if he has answered as I should have answered, then I am refuted, but if he answers something else, then he is refuted and not I. For do you really suppose that any one would admit the memory which a man has of an impression which has passed away to be the same with that which he experienced at the time? Assuredly not. Or would he hesitate to acknowledge that the same man may know and not know the same thing? Or, if he is afraid of making this admission, would he ever grant that one who has become unlike is the same as before he became unlike? Or would he admit that a man is one at all, and not rather many and infinite as the changes which take place in him? I speak by the card in order to avoid entanglements of words. But, O my good sir, he will say, come to the argument in a more generous spirit; and either show, if you can, that our sensations are not relative and individual, or, if you admit them to be so, prove that this does not involve the consequence that the appearance becomes, or, if you will have the word, is, to the individual only. As to your talk about pigs and baboons, you are yourself behaving like a pig, and you teach your hearers to make sport of my writings in the same ignorant manner; but this is not to your credit. For I declare that the truth is as I have written, and that each of us is a measure of existence and of non-existence. Yet one man may be a thousand times better than another in proportion as different things are and appear to him. And I am far from saying that wisdom and the wise man have no existence; but I say that the wise man is he who makes the evils which appear and are to a man, into goods which are and appear to him. And I would beg you not to press my words in the letter, but to take the meaning of them as I will explain them. Remember what has been already said,—that to the sick man his food appears to be and is bitter, and to the man in health the opposite of bitter. Now I cannot conceive that one of these men can be or ought to be made wiser than the other:  nor can you assert that the sick man because he has one impression is foolish, and the healthy man because he has another is wise; but the one state requires to be changed into the other, the worse into the better. As in education, a change of state has to be effected, and the sophist accomplishes by words the change which the physician works by the aid of drugs. Not that any one ever made another think truly, who previously thought falsely. For no one can think what is not, or, think anything different from that which he feels; and this is always true. But as the inferior habit of mind has thoughts of kindred nature, so I conceive that a good mind causes men to have good thoughts; and these which the inexperienced call true, I maintain to be only better, and not truer than others. And, O my dear Socrates, I do not call wise men tadpoles:  far from it; I say that they are the physicians of the human body, and the husbandmen of plants—for the husbandmen also take away the evil and disordered sensations of plants, and infuse into them good and healthy sensations—aye and true ones; and the wise and good rhetoricians make the good instead of the evil to seem just to states; for whatever appears to a state to be just and fair, so long as it is regarded as such, is just and fair to it; but the teacher of wisdom causes the good to take the place of the evil, both in appearance and in reality. And in like manner the Sophist who is able to train his pupils in this spirit is a wise man, and deserves to be well paid by them. And so one man is wiser than another; and no one thinks falsely, and you, whether you will or not, must endure to be a measure. On these foundations the argument stands firm, which you, Socrates, may, if you please, overthrow by an opposite argument, or if you like you may put questions to me—a method to which no intelligent person will object, quite the reverse. But I must beg you to put fair questions:  for there is great inconsistency in saying that you have a zeal for virtue, and then always behaving unfairly in argument. The unfairness of which I complain is that you do not distinguish between mere disputation and dialectic:  the disputer may trip up his opponent as often as he likes, and make fun; but the dialectician will be in earnest, and only correct his adversary when necessary, telling him the errors into which he has fallen through his own fault, or that of the company which he has previously kept. If you do so, your adversary will lay the blame of his own confusion and perplexity on himself, and not on you. He will follow and love you, and will hate himself, and escape from himself into philosophy, in order that he may become different from what he was. But the other mode of arguing, which is practised by the many, will have just the opposite effect upon him; and as he grows older, instead of turning philosopher, he will come to hate philosophy. I would recommend you, therefore, as I said before, not to encourage yourself in this polemical and controversial temper, but to find out, in a friendly and congenial spirit, what we really mean when we say that all things are in motion, and that to every individual and state what appears, is. In this manner you will consider whether knowledge and sensation are the same or different, but you will not argue, as you were just now doing, from the customary use of names and words, which the vulgar pervert in all sorts of ways, causing infinite perplexity to one another. Such, Theodorus, is the very slight help which I am able to offer to your old friend; had he been living, he would have helped himself in a far more gloriose style.

\par \textbf{THEODORUS}
\par   You are jesting, Socrates; indeed, your defence of him has been most valorous.

\par \textbf{SOCRATES}
\par   Thank you, friend; and I hope that you observed Protagoras bidding us be serious, as the text, 'Man is the measure of all things,' was a solemn one; and he reproached us with making a boy the medium of discourse, and said that the boy's timidity was made to tell against his argument; he also declared that we made a joke of him.

\par \textbf{THEODORUS}
\par   How could I fail to observe all that, Socrates?

\par \textbf{SOCRATES}
\par   Well, and shall we do as he says?

\par \textbf{THEODORUS}
\par   By all means.

\par \textbf{SOCRATES}
\par   But if his wishes are to be regarded, you and I must take up the argument, and in all seriousness, and ask and answer one another, for you see that the rest of us are nothing but boys. In no other way can we escape the imputation, that in our fresh analysis of his thesis we are making fun with boys.

\par \textbf{THEODORUS}
\par   Well, but is not Theaetetus better able to follow a philosophical enquiry than a great many men who have long beards?

\par \textbf{SOCRATES}
\par   Yes, Theodorus, but not better than you; and therefore please not to imagine that I am to defend by every means in my power your departed friend; and that you are to defend nothing and nobody. At any rate, my good man, do not sheer off until we know whether you are a true measure of diagrams, or whether all men are equally measures and sufficient for themselves in astronomy and geometry, and the other branches of knowledge in which you are supposed to excel them.

\par \textbf{THEODORUS}
\par   He who is sitting by you, Socrates, will not easily avoid being drawn into an argument; and when I said just now that you would excuse me, and not, like the Lacedaemonians, compel me to strip and fight, I was talking nonsense—I should rather compare you to Scirrhon, who threw travellers from the rocks; for the Lacedaemonian rule is 'strip or depart,' but you seem to go about your work more after the fashion of Antaeus:  you will not allow any one who approaches you to depart until you have stripped him, and he has been compelled to try a fall with you in argument.

\par \textbf{SOCRATES}
\par   There, Theodorus, you have hit off precisely the nature of my complaint; but I am even more pugnacious than the giants of old, for I have met with no end of heroes; many a Heracles, many a Theseus, mighty in words, has broken my head; nevertheless I am always at this rough exercise, which inspires me like a passion. Please, then, to try a fall with me, whereby you will do yourself good as well as me.

\par \textbf{THEODORUS}
\par   I consent; lead me whither you will, for I know that you are like destiny; no man can escape from any argument which you may weave for him. But I am not disposed to go further than you suggest.

\par \textbf{SOCRATES}
\par   Once will be enough; and now take particular care that we do not again unwittingly expose ourselves to the reproach of talking childishly.

\par \textbf{THEODORUS}
\par   I will do my best to avoid that error.

\par \textbf{SOCRATES}
\par   In the first place, let us return to our old objection, and see whether we were right in blaming and taking offence at Protagoras on the ground that he assumed all to be equal and sufficient in wisdom; although he admitted that there was a better and worse, and that in respect of this, some who as he said were the wise excelled others.

\par \textbf{THEODORUS}
\par   Very true.

\par \textbf{SOCRATES}
\par   Had Protagoras been living and answered for himself, instead of our answering for him, there would have been no need of our reviewing or reinforcing the argument. But as he is not here, and some one may accuse us of speaking without authority on his behalf, had we not better come to a clearer agreement about his meaning, for a great deal may be at stake?

\par \textbf{THEODORUS}
\par   True.

\par \textbf{SOCRATES}
\par   Then let us obtain, not through any third person, but from his own statement and in the fewest words possible, the basis of agreement.

\par \textbf{THEODORUS}
\par   In what way?

\par \textbf{SOCRATES}
\par   In this way: —His words are, 'What seems to a man, is to him.'

\par \textbf{THEODORUS}
\par   Yes, so he says.

\par \textbf{SOCRATES}
\par   And are not we, Protagoras, uttering the opinion of man, or rather of all mankind, when we say that every one thinks himself wiser than other men in some things, and their inferior in others? In the hour of danger, when they are in perils of war, or of the sea, or of sickness, do they not look up to their commanders as if they were gods, and expect salvation from them, only because they excel them in knowledge? Is not the world full of men in their several employments, who are looking for teachers and rulers of themselves and of the animals? and there are plenty who think that they are able to teach and able to rule. Now, in all this is implied that ignorance and wisdom exist among them, at least in their own opinion.

\par \textbf{THEODORUS}
\par   Certainly.

\par \textbf{SOCRATES}
\par   And wisdom is assumed by them to be true thought, and ignorance to be false opinion.

\par \textbf{THEODORUS}
\par   Exactly.

\par \textbf{SOCRATES}
\par   How then, Protagoras, would you have us treat the argument? Shall we say that the opinions of men are always true, or sometimes true and sometimes false? In either case, the result is the same, and their opinions are not always true, but sometimes true and sometimes false. For tell me, Theodorus, do you suppose that you yourself, or any other follower of Protagoras, would contend that no one deems another ignorant or mistaken in his opinion?

\par \textbf{THEODORUS}
\par   The thing is incredible, Socrates.

\par \textbf{SOCRATES}
\par   And yet that absurdity is necessarily involved in the thesis which declares man to be the measure of all things.

\par \textbf{THEODORUS}
\par   How so?

\par \textbf{SOCRATES}
\par   Why, suppose that you determine in your own mind something to be true, and declare your opinion to me; let us assume, as he argues, that this is true to you. Now, if so, you must either say that the rest of us are not the judges of this opinion or judgment of yours, or that we judge you always to have a true opinion? But are there not thousands upon thousands who, whenever you form a judgment, take up arms against you and are of an opposite judgment and opinion, deeming that you judge falsely?

\par \textbf{THEODORUS}
\par   Yes, indeed, Socrates, thousands and tens of thousands, as Homer says, who give me a world of trouble.

\par \textbf{SOCRATES}
\par   Well, but are we to assert that what you think is true to you and false to the ten thousand others?

\par \textbf{THEODORUS}
\par   No other inference seems to be possible.

\par \textbf{SOCRATES}
\par   And how about Protagoras himself? If neither he nor the multitude thought, as indeed they do not think, that man is the measure of all things, must it not follow that the truth of which Protagoras wrote would be true to no one? But if you suppose that he himself thought this, and that the multitude does not agree with him, you must begin by allowing that in whatever proportion the many are more than one, in that proportion his truth is more untrue than true.

\par \textbf{THEODORUS}
\par   That would follow if the truth is supposed to vary with individual opinion.

\par \textbf{SOCRATES}
\par   And the best of the joke is, that he acknowledges the truth of their opinion who believe his own opinion to be false; for he admits that the opinions of all men are true.

\par \textbf{THEODORUS}
\par   Certainly.

\par \textbf{SOCRATES}
\par   And does he not allow that his own opinion is false, if he admits that the opinion of those who think him false is true?

\par \textbf{THEODORUS}
\par   Of course.

\par \textbf{SOCRATES}
\par   Whereas the other side do not admit that they speak falsely?

\par \textbf{THEODORUS}
\par   They do not.

\par \textbf{SOCRATES}
\par   And he, as may be inferred from his writings, agrees that this opinion is also true.

\par \textbf{THEODORUS}
\par   Clearly.

\par \textbf{SOCRATES}
\par   Then all mankind, beginning with Protagoras, will contend, or rather, I should say that he will allow, when he concedes that his adversary has a true opinion—Protagoras, I say, will himself allow that neither a dog nor any ordinary man is the measure of anything which he has not learned—am I not right?

\par \textbf{THEODORUS}
\par   Yes.

\par \textbf{SOCRATES}
\par   And the truth of Protagoras being doubted by all, will be true neither to himself to any one else?

\par \textbf{THEODORUS}
\par   I think, Socrates, that we are running my old friend too hard.

\par \textbf{SOCRATES}
\par   But I do not know that we are going beyond the truth. Doubtless, as he is older, he may be expected to be wiser than we are. And if he could only just get his head out of the world below, he would have overthrown both of us again and again, me for talking nonsense and you for assenting to me, and have been off and underground in a trice. But as he is not within call, we must make the best use of our own faculties, such as they are, and speak out what appears to us to be true. And one thing which no one will deny is, that there are great differences in the understandings of men.

\par \textbf{THEODORUS}
\par   In that opinion I quite agree.

\par \textbf{SOCRATES}
\par   And is there not most likely to be firm ground in the distinction which we were indicating on behalf of Protagoras, viz. that most things, and all immediate sensations, such as hot, dry, sweet, are only such as they appear; if however difference of opinion is to be allowed at all, surely we must allow it in respect of health or disease? for every woman, child, or living creature has not such a knowledge of what conduces to health as to enable them to cure themselves.

\par \textbf{THEODORUS}
\par   I quite agree.

\par \textbf{SOCRATES}
\par   Or again, in politics, while affirming that just and unjust, honourable and disgraceful, holy and unholy, are in reality to each state such as the state thinks and makes lawful, and that in determining these matters no individual or state is wiser than another, still the followers of Protagoras will not deny that in determining what is or is not expedient for the community one state is wiser and one counsellor better than another—they will scarcely venture to maintain, that what a city enacts in the belief that it is expedient will always be really expedient. But in the other case, I mean when they speak of justice and injustice, piety and impiety, they are confident that in nature these have no existence or essence of their own—the truth is that which is agreed on at the time of the agreement, and as long as the agreement lasts; and this is the philosophy of many who do not altogether go along with Protagoras. Here arises a new question, Theodorus, which threatens to be more serious than the last.

\par \textbf{THEODORUS}
\par   Well, Socrates, we have plenty of leisure.

\par \textbf{SOCRATES}
\par   That is true, and your remark recalls to my mind an observation which I have often made, that those who have passed their days in the pursuit of philosophy are ridiculously at fault when they have to appear and speak in court. How natural is this!

\par \textbf{THEODORUS}
\par   What do you mean?

\par \textbf{SOCRATES}
\par   I mean to say, that those who have been trained in philosophy and liberal pursuits are as unlike those who from their youth upwards have been knocking about in the courts and such places, as a freeman is in breeding unlike a slave.

\par \textbf{THEODORUS}
\par   In what is the difference seen?

\par \textbf{SOCRATES}
\par   In the leisure spoken of by you, which a freeman can always command:  he has his talk out in peace, and, like ourselves, he wanders at will from one subject to another, and from a second to a third,—if the fancy takes him, he begins again, as we are doing now, caring not whether his words are many or few; his only aim is to attain the truth. But the lawyer is always in a hurry; there is the water of the clepsydra driving him on, and not allowing him to expatiate at will:  and there is his adversary standing over him, enforcing his rights; the indictment, which in their phraseology is termed the affidavit, is recited at the time:  and from this he must not deviate. He is a servant, and is continually disputing about a fellow-servant before his master, who is seated, and has the cause in his hands; the trial is never about some indifferent matter, but always concerns himself; and often the race is for his life. The consequence has been, that he has become keen and shrewd; he has learned how to flatter his master in word and indulge him in deed; but his soul is small and unrighteous. His condition, which has been that of a slave from his youth upwards, has deprived him of growth and uprightness and independence; dangers and fears, which were too much for his truth and honesty, came upon him in early years, when the tenderness of youth was unequal to them, and he has been driven into crooked ways; from the first he has practised deception and retaliation, and has become stunted and warped. And so he has passed out of youth into manhood, having no soundness in him; and is now, as he thinks, a master in wisdom. Such is the lawyer, Theodorus. Will you have the companion picture of the philosopher, who is of our brotherhood; or shall we return to the argument? Do not let us abuse the freedom of digression which we claim.

\par \textbf{THEODORUS}
\par   Nay, Socrates, not until we have finished what we are about; for you truly said that we belong to a brotherhood which is free, and are not the servants of the argument; but the argument is our servant, and must wait our leisure. Who is our judge? Or where is the spectator having any right to censure or control us, as he might the poets?

\par \textbf{SOCRATES}
\par   Then, as this is your wish, I will describe the leaders; for there is no use in talking about the inferior sort. In the first place, the lords of philosophy have never, from their youth upwards, known their way to the Agora, or the dicastery, or the council, or any other political assembly; they neither see nor hear the laws or decrees, as they are called, of the state written or recited; the eagerness of political societies in the attainment of offices—clubs, and banquets, and revels, and singing-maidens,—do not enter even into their dreams. Whether any event has turned out well or ill in the city, what disgrace may have descended to any one from his ancestors, male or female, are matters of which the philosopher no more knows than he can tell, as they say, how many pints are contained in the ocean. Neither is he conscious of his ignorance. For he does not hold aloof in order that he may gain a reputation; but the truth is, that the outer form of him only is in the city:  his mind, disdaining the littlenesses and nothingnesses of human things, is 'flying all abroad' as Pindar says, measuring earth and heaven and the things which are under and on the earth and above the heaven, interrogating the whole nature of each and all in their entirety, but not condescending to anything which is within reach.

\par \textbf{THEODORUS}
\par   What do you mean, Socrates?

\par \textbf{SOCRATES}
\par   I will illustrate my meaning, Theodorus, by the jest which the clever witty Thracian handmaid is said to have made about Thales, when he fell into a well as he was looking up at the stars. She said, that he was so eager to know what was going on in heaven, that he could not see what was before his feet. This is a jest which is equally applicable to all philosophers. For the philosopher is wholly unacquainted with his next-door neighbour; he is ignorant, not only of what he is doing, but he hardly knows whether he is a man or an animal; he is searching into the essence of man, and busy in enquiring what belongs to such a nature to do or suffer different from any other;—I think that you understand me, Theodorus?

\par \textbf{THEODORUS}
\par   I do, and what you say is true.

\par \textbf{SOCRATES}
\par   And thus, my friend, on every occasion, private as well as public, as I said at first, when he appears in a law-court, or in any place in which he has to speak of things which are at his feet and before his eyes, he is the jest, not only of Thracian handmaids but of the general herd, tumbling into wells and every sort of disaster through his inexperience. His awkwardness is fearful, and gives the impression of imbecility. When he is reviled, he has nothing personal to say in answer to the civilities of his adversaries, for he knows no scandals of any one, and they do not interest him; and therefore he is laughed at for his sheepishness; and when others are being praised and glorified, in the simplicity of his heart he cannot help going into fits of laughter, so that he seems to be a downright idiot. When he hears a tyrant or king eulogized, he fancies that he is listening to the praises of some keeper of cattle—a swineherd, or shepherd, or perhaps a cowherd, who is congratulated on the quantity of milk which he squeezes from them; and he remarks that the creature whom they tend, and out of whom they squeeze the wealth, is of a less tractable and more insidious nature. Then, again, he observes that the great man is of necessity as ill-mannered and uneducated as any shepherd—for he has no leisure, and he is surrounded by a wall, which is his mountain-pen. Hearing of enormous landed proprietors of ten thousand acres and more, our philosopher deems this to be a trifle, because he has been accustomed to think of the whole earth; and when they sing the praises of family, and say that some one is a gentleman because he can show seven generations of wealthy ancestors, he thinks that their sentiments only betray a dull and narrow vision in those who utter them, and who are not educated enough to look at the whole, nor to consider that every man has had thousands and ten thousands of progenitors, and among them have been rich and poor, kings and slaves, Hellenes and barbarians, innumerable. And when people pride themselves on having a pedigree of twenty-five ancestors, which goes back to Heracles, the son of Amphitryon, he cannot understand their poverty of ideas. Why are they unable to calculate that Amphitryon had a twenty-fifth ancestor, who might have been anybody, and was such as fortune made him, and he had a fiftieth, and so on? He amuses himself with the notion that they cannot count, and thinks that a little arithmetic would have got rid of their senseless vanity. Now, in all these cases our philosopher is derided by the vulgar, partly because he is thought to despise them, and also because he is ignorant of what is before him, and always at a loss.

\par \textbf{THEODORUS}
\par   That is very true, Socrates.

\par \textbf{SOCRATES}
\par   But, O my friend, when he draws the other into upper air, and gets him out of his pleas and rejoinders into the contemplation of justice and injustice in their own nature and in their difference from one another and from all other things; or from the commonplaces about the happiness of a king or of a rich man to the consideration of government, and of human happiness and misery in general—what they are, and how a man is to attain the one and avoid the other—when that narrow, keen, little legal mind is called to account about all this, he gives the philosopher his revenge; for dizzied by the height at which he is hanging, whence he looks down into space, which is a strange experience to him, he being dismayed, and lost, and stammering broken words, is laughed at, not by Thracian handmaidens or any other uneducated persons, for they have no eye for the situation, but by every man who has not been brought up a slave. Such are the two characters, Theodorus:  the one of the freeman, who has been trained in liberty and leisure, whom you call the philosopher,—him we cannot blame because he appears simple and of no account when he has to perform some menial task, such as packing up bed-clothes, or flavouring a sauce or fawning speech; the other character is that of the man who is able to do all this kind of service smartly and neatly, but knows not how to wear his cloak like a gentleman; still less with the music of discourse can he hymn the true life aright which is lived by immortals or men blessed of heaven.

\par \textbf{THEODORUS}
\par   If you could only persuade everybody, Socrates, as you do me, of the truth of your words, there would be more peace and fewer evils among men.

\par \textbf{SOCRATES}
\par   Evils, Theodorus, can never pass away; for there must always remain something which is antagonistic to good. Having no place among the gods in heaven, of necessity they hover around the mortal nature, and this earthly sphere. Wherefore we ought to fly away from earth to heaven as quickly as we can; and to fly away is to become like God, as far as this is possible; and to become like him, is to become holy, just, and wise. But, O my friend, you cannot easily convince mankind that they should pursue virtue or avoid vice, not merely in order that a man may seem to be good, which is the reason given by the world, and in my judgment is only a repetition of an old wives' fable. Whereas, the truth is that God is never in any way unrighteous—he is perfect righteousness; and he of us who is the most righteous is most like him. Herein is seen the true cleverness of a man, and also his nothingness and want of manhood. For to know this is true wisdom and virtue, and ignorance of this is manifest folly and vice. All other kinds of wisdom or cleverness, which seem only, such as the wisdom of politicians, or the wisdom of the arts, are coarse and vulgar. The unrighteous man, or the sayer and doer of unholy things, had far better not be encouraged in the illusion that his roguery is clever; for men glory in their shame—they fancy that they hear others saying of them, 'These are not mere good-for-nothing persons, mere burdens of the earth, but such as men should be who mean to dwell safely in a state.' Let us tell them that they are all the more truly what they do not think they are because they do not know it; for they do not know the penalty of injustice, which above all things they ought to know—not stripes and death, as they suppose, which evil-doers often escape, but a penalty which cannot be escaped.

\par \textbf{THEODORUS}
\par   What is that?

\par \textbf{SOCRATES}
\par   There are two patterns eternally set before them; the one blessed and divine, the other godless and wretched:  but they do not see them, or perceive that in their utter folly and infatuation they are growing like the one and unlike the other, by reason of their evil deeds; and the penalty is, that they lead a life answering to the pattern which they are growing like. And if we tell them, that unless they depart from their cunning, the place of innocence will not receive them after death; and that here on earth, they will live ever in the likeness of their own evil selves, and with evil friends—when they hear this they in their superior cunning will seem to be listening to the talk of idiots.

\par \textbf{THEODORUS}
\par   Very true, Socrates.

\par \textbf{SOCRATES}
\par   Too true, my friend, as I well know; there is, however, one peculiarity in their case:  when they begin to reason in private about their dislike of philosophy, if they have the courage to hear the argument out, and do not run away, they grow at last strangely discontented with themselves; their rhetoric fades away, and they become helpless as children. These however are digressions from which we must now desist, or they will overflow, and drown the original argument; to which, if you please, we will now return.

\par \textbf{THEODORUS}
\par   For my part, Socrates, I would rather have the digressions, for at my age I find them easier to follow; but if you wish, let us go back to the argument.

\par \textbf{SOCRATES}
\par   Had we not reached the point at which the partisans of the perpetual flux, who say that things are as they seem to each one, were confidently maintaining that the ordinances which the state commanded and thought just, were just to the state which imposed them, while they were in force; this was especially asserted of justice; but as to the good, no one had any longer the hardihood to contend of any ordinances which the state thought and enacted to be good that these, while they were in force, were really good;—he who said so would be playing with the name 'good,' and would not touch the real question—it would be a mockery, would it not?

\par \textbf{THEODORUS}
\par   Certainly it would.

\par \textbf{SOCRATES}
\par   He ought not to speak of the name, but of the thing which is contemplated under the name.

\par \textbf{THEODORUS}
\par   Right.

\par \textbf{SOCRATES}
\par   Whatever be the term used, the good or expedient is the aim of legislation, and as far as she has an opinion, the state imposes all laws with a view to the greatest expediency; can legislation have any other aim?

\par \textbf{THEODORUS}
\par   Certainly not.

\par \textbf{SOCRATES}
\par   But is the aim attained always? do not mistakes often happen?

\par \textbf{THEODORUS}
\par   Yes, I think that there are mistakes.

\par \textbf{SOCRATES}
\par   The possibility of error will be more distinctly recognised, if we put the question in reference to the whole class under which the good or expedient falls. That whole class has to do with the future, and laws are passed under the idea that they will be useful in after-time; which, in other words, is the future.

\par \textbf{THEODORUS}
\par   Very true.

\par \textbf{SOCRATES}
\par   Suppose now, that we ask Protagoras, or one of his disciples, a question: —O, Protagoras, we will say to him, Man is, as you declare, the measure of all things—white, heavy, light:  of all such things he is the judge; for he has the criterion of them in himself, and when he thinks that things are such as he experiences them to be, he thinks what is and is true to himself. Is it not so?

\par \textbf{THEODORUS}
\par   Yes.

\par \textbf{SOCRATES}
\par   And do you extend your doctrine, Protagoras (as we shall further say), to the future as well as to the present; and has he the criterion not only of what in his opinion is but of what will be, and do things always happen to him as he expected? For example, take the case of heat: —When an ordinary man thinks that he is going to have a fever, and that this kind of heat is coming on, and another person, who is a physician, thinks the contrary, whose opinion is likely to prove right? Or are they both right?—he will have a heat and fever in his own judgment, and not have a fever in the physician's judgment?

\par \textbf{THEODORUS}
\par   How ludicrous!

\par \textbf{SOCRATES}
\par   And the vinegrower, if I am not mistaken, is a better judge of the sweetness or dryness of the vintage which is not yet gathered than the harp-player?

\par \textbf{THEODORUS}
\par   Certainly.

\par \textbf{SOCRATES}
\par   And in musical composition the musician will know better than the training master what the training master himself will hereafter think harmonious or the reverse?

\par \textbf{THEODORUS}
\par   Of course.

\par \textbf{SOCRATES}
\par   And the cook will be a better judge than the guest, who is not a cook, of the pleasure to be derived from the dinner which is in preparation; for of present or past pleasure we are not as yet arguing; but can we say that every one will be to himself the best judge of the pleasure which will seem to be and will be to him in the future?—nay, would not you, Protagoras, better guess which arguments in a court would convince any one of us than the ordinary man?

\par \textbf{THEODORUS}
\par   Certainly, Socrates, he used to profess in the strongest manner that he was the superior of all men in this respect.

\par \textbf{SOCRATES}
\par   To be sure, friend:  who would have paid a large sum for the privilege of talking to him, if he had really persuaded his visitors that neither a prophet nor any other man was better able to judge what will be and seem to be in the future than every one could for himself?

\par \textbf{THEODORUS}
\par   Who indeed?

\par \textbf{SOCRATES}
\par   And legislation and expediency are all concerned with the future; and every one will admit that states, in passing laws, must often fail of their highest interests?

\par \textbf{THEODORUS}
\par   Quite true.

\par \textbf{SOCRATES}
\par   Then we may fairly argue against your master, that he must admit one man to be wiser than another, and that the wiser is a measure:  but I, who know nothing, am not at all obliged to accept the honour which the advocate of Protagoras was just now forcing upon me, whether I would or not, of being a measure of anything.

\par \textbf{THEODORUS}
\par   That is the best refutation of him, Socrates; although he is also caught when he ascribes truth to the opinions of others, who give the lie direct to his own opinion.

\par \textbf{SOCRATES}
\par   There are many ways, Theodorus, in which the doctrine that every opinion of every man is true may be refuted; but there is more difficulty in proving that states of feeling, which are present to a man, and out of which arise sensations and opinions in accordance with them, are also untrue. And very likely I have been talking nonsense about them; for they may be unassailable, and those who say that there is clear evidence of them, and that they are matters of knowledge, may probably be right; in which case our friend Theaetetus was not so far from the mark when he identified perception and knowledge. And therefore let us draw nearer, as the advocate of Protagoras desires; and give the truth of the universal flux a ring:  is the theory sound or not? at any rate, no small war is raging about it, and there are combination not a few.

\par \textbf{THEODORUS}
\par   No small, war, indeed, for in Ionia the sect makes rapid strides; the disciples of Heracleitus are most energetic upholders of the doctrine.

\par \textbf{SOCRATES}
\par   Then we are the more bound, my dear Theodorus, to examine the question from the foundation as it is set forth by themselves.

\par \textbf{THEODORUS}
\par   Certainly we are. About these speculations of Heracleitus, which, as you say, are as old as Homer, or even older still, the Ephesians themselves, who profess to know them, are downright mad, and you cannot talk with them on the subject. For, in accordance with their text-books, they are always in motion; but as for dwelling upon an argument or a question, and quietly asking and answering in turn, they can no more do so than they can fly; or rather, the determination of these fellows not to have a particle of rest in them is more than the utmost powers of negation can express. If you ask any of them a question, he will produce, as from a quiver, sayings brief and dark, and shoot them at you; and if you inquire the reason of what he has said, you will be hit by some other new-fangled word, and will make no way with any of them, nor they with one another; their great care is, not to allow of any settled principle either in their arguments or in their minds, conceiving, as I imagine, that any such principle would be stationary; for they are at war with the stationary, and do what they can to drive it out everywhere.

\par \textbf{SOCRATES}
\par   I suppose, Theodorus, that you have only seen them when they were fighting, and have never stayed with them in time of peace, for they are no friends of yours; and their peace doctrines are only communicated by them at leisure, as I imagine, to those disciples of theirs whom they want to make like themselves.

\par \textbf{THEODORUS}
\par   Disciples! my good sir, they have none; men of their sort are not one another's disciples, but they grow up at their own sweet will, and get their inspiration anywhere, each of them saying of his neighbour that he knows nothing. From these men, then, as I was going to remark, you will never get a reason, whether with their will or without their will; we must take the question out of their hands, and make the analysis ourselves, as if we were doing geometrical problem.

\par \textbf{SOCRATES}
\par   Quite right too; but as touching the aforesaid problem, have we not heard from the ancients, who concealed their wisdom from the many in poetical figures, that Oceanus and Tethys, the origin of all things, are streams, and that nothing is at rest? And now the moderns, in their superior wisdom, have declared the same openly, that the cobbler too may hear and learn of them, and no longer foolishly imagine that some things are at rest and others in motion—having learned that all is motion, he will duly honour his teachers. I had almost forgotten the opposite doctrine, Theodorus,
 
\par  This is the language of Parmenides, Melissus, and their followers, who stoutly maintain that all being is one and self-contained, and has no place in which to move. What shall we do, friend, with all these people; for, advancing step by step, we have imperceptibly got between the combatants, and, unless we can protect our retreat, we shall pay the penalty of our rashness—like the players in the palaestra who are caught upon the line, and are dragged different ways by the two parties. Therefore I think that we had better begin by considering those whom we first accosted, 'the river-gods,' and, if we find any truth in them, we will help them to pull us over, and try to get away from the others. But if the partisans of 'the whole' appear to speak more truly, we will fly off from the party which would move the immovable, to them. And if I find that neither of them have anything reasonable to say, we shall be in a ridiculous position, having so great a conceit of our own poor opinion and rejecting that of ancient and famous men. O Theodorus, do you think that there is any use in proceeding when the danger is so great?

\par \textbf{THEODORUS}
\par   Nay, Socrates, not to examine thoroughly what the two parties have to say would be quite intolerable.

\par \textbf{SOCRATES}
\par   Then examine we must, since you, who were so reluctant to begin, are so eager to proceed. The nature of motion appears to be the question with which we begin. What do they mean when they say that all things are in motion? Is there only one kind of motion, or, as I rather incline to think, two? I should like to have your opinion upon this point in addition to my own, that I may err, if I must err, in your company; tell me, then, when a thing changes from one place to another, or goes round in the same place, is not that what is called motion?

\par \textbf{THEODORUS}
\par   Yes.

\par \textbf{SOCRATES}
\par   Here then we have one kind of motion. But when a thing, remaining on the same spot, grows old, or becomes black from being white, or hard from being soft, or undergoes any other change, may not this be properly called motion of another kind?

\par \textbf{THEODORUS}
\par   I think so.

\par \textbf{SOCRATES}
\par   Say rather that it must be so. Of motion then there are these two kinds, 'change,' and 'motion in place.'

\par \textbf{THEODORUS}
\par   You are right.

\par \textbf{SOCRATES}
\par   And now, having made this distinction, let us address ourselves to those who say that all is motion, and ask them whether all things according to them have the two kinds of motion, and are changed as well as move in place, or is one thing moved in both ways, and another in one only?

\par \textbf{THEODORUS}
\par   Indeed, I do not know what to answer; but I think they would say that all things are moved in both ways.

\par \textbf{SOCRATES}
\par   Yes, comrade; for, if not, they would have to say that the same things are in motion and at rest, and there would be no more truth in saying that all things are in motion, than that all things are at rest.

\par \textbf{THEODORUS}
\par   To be sure.

\par \textbf{SOCRATES}
\par   And if they are to be in motion, and nothing is to be devoid of motion, all things must always have every sort of motion?

\par \textbf{THEODORUS}
\par   Most true.

\par \textbf{SOCRATES}
\par   Consider a further point:  did we not understand them to explain the generation of heat, whiteness, or anything else, in some such manner as the following: —were they not saying that each of them is moving between the agent and the patient, together with a perception, and that the patient ceases to be a perceiving power and becomes a percipient, and the agent a quale instead of a quality? I suspect that quality may appear a strange and uncouth term to you, and that you do not understand the abstract expression. Then I will take concrete instances:  I mean to say that the producing power or agent becomes neither heat nor whiteness but hot and white, and the like of other things. For I must repeat what I said before, that neither the agent nor patient have any absolute existence, but when they come together and generate sensations and their objects, the one becomes a thing of a certain quality, and the other a percipient. You remember?

\par \textbf{THEODORUS}
\par   Of course.

\par \textbf{SOCRATES}
\par   We may leave the details of their theory unexamined, but we must not forget to ask them the only question with which we are concerned:  Are all things in motion and flux?

\par \textbf{THEODORUS}
\par   Yes, they will reply.

\par \textbf{SOCRATES}
\par   And they are moved in both those ways which we distinguished, that is to say, they move in place and are also changed?

\par \textbf{THEODORUS}
\par   Of course, if the motion is to be perfect.

\par \textbf{SOCRATES}
\par   If they only moved in place and were not changed, we should be able to say what is the nature of the things which are in motion and flux?

\par \textbf{THEODORUS}
\par   Exactly.

\par \textbf{SOCRATES}
\par   But now, since not even white continues to flow white, and whiteness itself is a flux or change which is passing into another colour, and is never to be caught standing still, can the name of any colour be rightly used at all?

\par \textbf{THEODORUS}
\par   How is that possible, Socrates, either in the case of this or of any other quality—if while we are using the word the object is escaping in the flux?

\par \textbf{SOCRATES}
\par   And what would you say of perceptions, such as sight and hearing, or any other kind of perception? Is there any stopping in the act of seeing and hearing?

\par \textbf{THEODORUS}
\par   Certainly not, if all things are in motion.

\par \textbf{SOCRATES}
\par   Then we must not speak of seeing any more than of not-seeing, nor of any other perception more than of any non-perception, if all things partake of every kind of motion?

\par \textbf{THEODORUS}
\par   Certainly not.

\par \textbf{SOCRATES}
\par   Yet perception is knowledge:  so at least Theaetetus and I were saying.

\par \textbf{THEODORUS}
\par   Very true.

\par \textbf{SOCRATES}
\par   Then when we were asked what is knowledge, we no more answered what is knowledge than what is not knowledge?

\par \textbf{THEODORUS}
\par   I suppose not.

\par \textbf{SOCRATES}
\par   Here, then, is a fine result:  we corrected our first answer in our eagerness to prove that nothing is at rest. But if nothing is at rest, every answer upon whatever subject is equally right:  you may say that a thing is or is not thus; or, if you prefer, 'becomes' thus; and if we say 'becomes,' we shall not then hamper them with words expressive of rest.

\par \textbf{THEODORUS}
\par   Quite true.

\par \textbf{SOCRATES}
\par   Yes, Theodorus, except in saying 'thus' and 'not thus.' But you ought not to use the word 'thus,' for there is no motion in 'thus' or in 'not thus.' The maintainers of the doctrine have as yet no words in which to express themselves, and must get a new language. I know of no word that will suit them, except perhaps 'no how,' which is perfectly indefinite.

\par \textbf{THEODORUS}
\par   Yes, that is a manner of speaking in which they will be quite at home.

\par \textbf{SOCRATES}
\par   And so, Theodorus, we have got rid of your friend without assenting to his doctrine, that every man is the measure of all things—a wise man only is a measure; neither can we allow that knowledge is perception, certainly not on the hypothesis of a perpetual flux, unless perchance our friend Theaetetus is able to convince us that it is.

\par \textbf{THEODORUS}
\par   Very good, Socrates; and now that the argument about the doctrine of Protagoras has been completed, I am absolved from answering; for this was the agreement.

\par \textbf{THEAETETUS}
\par   Not, Theodorus, until you and Socrates have discussed the doctrine of those who say that all things are at rest, as you were proposing.

\par \textbf{THEODORUS}
\par   You, Theaetetus, who are a young rogue, must not instigate your elders to a breach of faith, but should prepare to answer Socrates in the remainder of the argument.

\par \textbf{THEAETETUS}
\par   Yes, if he wishes; but I would rather have heard about the doctrine of rest.

\par \textbf{THEODORUS}
\par   Invite Socrates to an argument—invite horsemen to the open plain; do but ask him, and he will answer.

\par \textbf{SOCRATES}
\par   Nevertheless, Theodorus, I am afraid that I shall not be able to comply with the request of Theaetetus.

\par \textbf{THEODORUS}
\par   Not comply! for what reason?

\par \textbf{SOCRATES}
\par   My reason is that I have a kind of reverence; not so much for Melissus and the others, who say that 'All is one and at rest,' as for the great leader himself, Parmenides, venerable and awful, as in Homeric language he may be called;—him I should be ashamed to approach in a spirit unworthy of him. I met him when he was an old man, and I was a mere youth, and he appeared to me to have a glorious depth of mind. And I am afraid that we may not understand his words, and may be still further from understanding his meaning; above all I fear that the nature of knowledge, which is the main subject of our discussion, may be thrust out of sight by the unbidden guests who will come pouring in upon our feast of discourse, if we let them in—besides, the question which is now stirring is of immense extent, and will be treated unfairly if only considered by the way; or if treated adequately and at length, will put into the shade the other question of knowledge. Neither the one nor the other can be allowed; but I must try by my art of midwifery to deliver Theaetetus of his conceptions about knowledge.

\par \textbf{THEAETETUS}
\par   Very well; do so if you will.

\par \textbf{SOCRATES}
\par   Then now, Theaetetus, take another view of the subject:  you answered that knowledge is perception?

\par \textbf{THEAETETUS}
\par   I did.

\par \textbf{SOCRATES}
\par   And if any one were to ask you:  With what does a man see black and white colours? and with what does he hear high and low sounds?—you would say, if I am not mistaken, 'With the eyes and with the ears.'

\par \textbf{THEAETETUS}
\par   I should.

\par \textbf{SOCRATES}
\par   The free use of words and phrases, rather than minute precision, is generally characteristic of a liberal education, and the opposite is pedantic; but sometimes precision is necessary, and I believe that the answer which you have just given is open to the charge of incorrectness; for which is more correct, to say that we see or hear with the eyes and with the ears, or through the eyes and through the ears.

\par \textbf{THEAETETUS}
\par   I should say 'through,' Socrates, rather than 'with.'

\par \textbf{SOCRATES}
\par   Yes, my boy, for no one can suppose that in each of us, as in a sort of Trojan horse, there are perched a number of unconnected senses, which do not all meet in some one nature, the mind, or whatever we please to call it, of which they are the instruments, and with which through them we perceive objects of sense.

\par \textbf{THEAETETUS}
\par   I agree with you in that opinion.

\par \textbf{SOCRATES}
\par   The reason why I am thus precise is, because I want to know whether, when we perceive black and white through the eyes, and again, other qualities through other organs, we do not perceive them with one and the same part of ourselves, and, if you were asked, you might refer all such perceptions to the body. Perhaps, however, I had better allow you to answer for yourself and not interfere. Tell me, then, are not the organs through which you perceive warm and hard and light and sweet, organs of the body?

\par \textbf{THEAETETUS}
\par   Of the body, certainly.

\par \textbf{SOCRATES}
\par   And you would admit that what you perceive through one faculty you cannot perceive through another; the objects of hearing, for example, cannot be perceived through sight, or the objects of sight through hearing?

\par \textbf{THEAETETUS}
\par   Of course not.

\par \textbf{SOCRATES}
\par   If you have any thought about both of them, this common perception cannot come to you, either through the one or the other organ?

\par \textbf{THEAETETUS}
\par   It cannot.

\par \textbf{SOCRATES}
\par   How about sounds and colours:  in the first place you would admit that they both exist?

\par \textbf{THEAETETUS}
\par   Yes.

\par \textbf{SOCRATES}
\par   And that either of them is different from the other, and the same with itself?

\par \textbf{THEAETETUS}
\par   Certainly.

\par \textbf{SOCRATES}
\par   And that both are two and each of them one?

\par \textbf{THEAETETUS}
\par   Yes.

\par \textbf{SOCRATES}
\par   You can further observe whether they are like or unlike one another?

\par \textbf{THEAETETUS}
\par   I dare say.

\par \textbf{SOCRATES}
\par   But through what do you perceive all this about them? for neither through hearing nor yet through seeing can you apprehend that which they have in common. Let me give you an illustration of the point at issue: —If there were any meaning in asking whether sounds and colours are saline or not, you would be able to tell me what faculty would consider the question. It would not be sight or hearing, but some other.

\par \textbf{THEAETETUS}
\par   Certainly; the faculty of taste.

\par \textbf{SOCRATES}
\par   Very good; and now tell me what is the power which discerns, not only in sensible objects, but in all things, universal notions, such as those which are called being and not-being, and those others about which we were just asking—what organs will you assign for the perception of these notions?

\par \textbf{THEAETETUS}
\par   You are thinking of being and not being, likeness and unlikeness, sameness and difference, and also of unity and other numbers which are applied to objects of sense; and you mean to ask, through what bodily organ the soul perceives odd and even numbers and other arithmetical conceptions.

\par \textbf{SOCRATES}
\par   You follow me excellently, Theaetetus; that is precisely what I am asking.

\par \textbf{THEAETETUS}
\par   Indeed, Socrates, I cannot answer; my only notion is, that these, unlike objects of sense, have no separate organ, but that the mind, by a power of her own, contemplates the universals in all things.

\par \textbf{SOCRATES}
\par   You are a beauty, Theaetetus, and not ugly, as Theodorus was saying; for he who utters the beautiful is himself beautiful and good. And besides being beautiful, you have done me a kindness in releasing me from a very long discussion, if you are clear that the soul views some things by herself and others through the bodily organs. For that was my own opinion, and I wanted you to agree with me.

\par \textbf{THEAETETUS}
\par   I am quite clear.

\par \textbf{SOCRATES}
\par   And to which class would you refer being or essence; for this, of all our notions, is the most universal?

\par \textbf{THEAETETUS}
\par   I should say, to that class which the soul aspires to know of herself.

\par \textbf{SOCRATES}
\par   And would you say this also of like and unlike, same and other?

\par \textbf{THEAETETUS}
\par   Yes.

\par \textbf{SOCRATES}
\par   And would you say the same of the noble and base, and of good and evil?

\par \textbf{THEAETETUS}
\par   These I conceive to be notions which are essentially relative, and which the soul also perceives by comparing in herself things past and present with the future.

\par \textbf{SOCRATES}
\par   And does she not perceive the hardness of that which is hard by the touch, and the softness of that which is soft equally by the touch?

\par \textbf{THEAETETUS}
\par   Yes.

\par \textbf{SOCRATES}
\par   But their essence and what they are, and their opposition to one another, and the essential nature of this opposition, the soul herself endeavours to decide for us by the review and comparison of them?

\par \textbf{THEAETETUS}
\par   Certainly.

\par \textbf{SOCRATES}
\par   The simple sensations which reach the soul through the body are given at birth to men and animals by nature, but their reflections on the being and use of them are slowly and hardly gained, if they are ever gained, by education and long experience.

\par \textbf{THEAETETUS}
\par   Assuredly.

\par \textbf{SOCRATES}
\par   And can a man attain truth who fails of attaining being?

\par \textbf{THEAETETUS}
\par   Impossible.

\par \textbf{SOCRATES}
\par   And can he who misses the truth of anything, have a knowledge of that thing?

\par \textbf{THEAETETUS}
\par   He cannot.

\par \textbf{SOCRATES}
\par   Then knowledge does not consist in impressions of sense, but in reasoning about them; in that only, and not in the mere impression, truth and being can be attained?

\par \textbf{THEAETETUS}
\par   Clearly.

\par \textbf{SOCRATES}
\par   And would you call the two processes by the same name, when there is so great a difference between them?

\par \textbf{THEAETETUS}
\par   That would certainly not be right.

\par \textbf{SOCRATES}
\par   And what name would you give to seeing, hearing, smelling, being cold and being hot?

\par \textbf{THEAETETUS}
\par   I should call all of them perceiving—what other name could be given to them?

\par \textbf{SOCRATES}
\par   Perception would be the collective name of them?

\par \textbf{THEAETETUS}
\par   Certainly.

\par \textbf{SOCRATES}
\par   Which, as we say, has no part in the attainment of truth any more than of being?

\par \textbf{THEAETETUS}
\par   Certainly not.

\par \textbf{SOCRATES}
\par   And therefore not in science or knowledge?

\par \textbf{THEAETETUS}
\par   No.

\par \textbf{SOCRATES}
\par   Then perception, Theaetetus, can never be the same as knowledge or science?

\par \textbf{THEAETETUS}
\par   Clearly not, Socrates; and knowledge has now been most distinctly proved to be different from perception.

\par \textbf{SOCRATES}
\par   But the original aim of our discussion was to find out rather what knowledge is than what it is not; at the same time we have made some progress, for we no longer seek for knowledge in perception at all, but in that other process, however called, in which the mind is alone and engaged with being.

\par \textbf{THEAETETUS}
\par   You mean, Socrates, if I am not mistaken, what is called thinking or opining.

\par \textbf{SOCRATES}
\par   You conceive truly. And now, my friend, please to begin again at this point; and having wiped out of your memory all that has preceded, see if you have arrived at any clearer view, and once more say what is knowledge.

\par \textbf{THEAETETUS}
\par   I cannot say, Socrates, that all opinion is knowledge, because there may be a false opinion; but I will venture to assert, that knowledge is true opinion:  let this then be my reply; and if this is hereafter disproved, I must try to find another.

\par \textbf{SOCRATES}
\par   That is the way in which you ought to answer, Theaetetus, and not in your former hesitating strain, for if we are bold we shall gain one of two advantages; either we shall find what we seek, or we shall be less likely to think that we know what we do not know—in either case we shall be richly rewarded. And now, what are you saying?—Are there two sorts of opinion, one true and the other false; and do you define knowledge to be the true?

\par \textbf{THEAETETUS}
\par   Yes, according to my present view.

\par \textbf{SOCRATES}
\par   Is it still worth our while to resume the discussion touching opinion?

\par \textbf{THEAETETUS}
\par   To what are you alluding?

\par \textbf{SOCRATES}
\par   There is a point which often troubles me, and is a great perplexity to me, both in regard to myself and others. I cannot make out the nature or origin of the mental experience to which I refer.

\par \textbf{THEAETETUS}
\par   Pray what is it?

\par \textbf{SOCRATES}
\par   How there can be false opinion—that difficulty still troubles the eye of my mind; and I am uncertain whether I shall leave the question, or begin over again in a new way.

\par \textbf{THEAETETUS}
\par   Begin again, Socrates,—at least if you think that there is the slightest necessity for doing so. Were not you and Theodorus just now remarking very truly, that in discussions of this kind we may take our own time?

\par \textbf{SOCRATES}
\par   You are quite right, and perhaps there will be no harm in retracing our steps and beginning again. Better a little which is well done, than a great deal imperfectly.

\par \textbf{THEAETETUS}
\par   Certainly.

\par \textbf{SOCRATES}
\par   Well, and what is the difficulty? Do we not speak of false opinion, and say that one man holds a false and another a true opinion, as though there were some natural distinction between them?

\par \textbf{THEAETETUS}
\par   We certainly say so.

\par \textbf{SOCRATES}
\par   All things and everything are either known or not known. I leave out of view the intermediate conceptions of learning and forgetting, because they have nothing to do with our present question.

\par \textbf{THEAETETUS}
\par   There can be no doubt, Socrates, if you exclude these, that there is no other alternative but knowing or not knowing a thing.

\par \textbf{SOCRATES}
\par   That point being now determined, must we not say that he who has an opinion, must have an opinion about something which he knows or does not know?

\par \textbf{THEAETETUS}
\par   He must.

\par \textbf{SOCRATES}
\par   He who knows, cannot but know; and he who does not know, cannot know?

\par \textbf{THEAETETUS}
\par   Of course.

\par \textbf{SOCRATES}
\par   What shall we say then? When a man has a false opinion does he think that which he knows to be some other thing which he knows, and knowing both, is he at the same time ignorant of both?

\par \textbf{THEAETETUS}
\par   That, Socrates, is impossible.

\par \textbf{SOCRATES}
\par   But perhaps he thinks of something which he does not know as some other thing which he does not know; for example, he knows neither Theaetetus nor Socrates, and yet he fancies that Theaetetus is Socrates, or Socrates Theaetetus?

\par \textbf{THEAETETUS}
\par   How can he?

\par \textbf{SOCRATES}
\par   But surely he cannot suppose what he knows to be what he does not know, or what he does not know to be what he knows?

\par \textbf{THEAETETUS}
\par   That would be monstrous.

\par \textbf{SOCRATES}
\par   Where, then, is false opinion? For if all things are either known or unknown, there can be no opinion which is not comprehended under this alternative, and so false opinion is excluded.

\par \textbf{THEAETETUS}
\par   Most true.

\par \textbf{SOCRATES}
\par   Suppose that we remove the question out of the sphere of knowing or not knowing, into that of being and not-being.

\par \textbf{THEAETETUS}
\par   What do you mean?

\par \textbf{SOCRATES}
\par   May we not suspect the simple truth to be that he who thinks about anything, that which is not, will necessarily think what is false, whatever in other respects may be the state of his mind?

\par \textbf{THEAETETUS}
\par   That, again, is not unlikely, Socrates.

\par \textbf{SOCRATES}
\par   Then suppose some one to say to us, Theaetetus: —Is it possible for any man to think that which is not, either as a self-existent substance or as a predicate of something else? And suppose that we answer, 'Yes, he can, when he thinks what is not true. '—That will be our answer?

\par \textbf{THEAETETUS}
\par   Yes.

\par \textbf{SOCRATES}
\par   But is there any parallel to this?

\par \textbf{THEAETETUS}
\par   What do you mean?

\par \textbf{SOCRATES}
\par   Can a man see something and yet see nothing?

\par \textbf{THEAETETUS}
\par   Impossible.

\par \textbf{SOCRATES}
\par   But if he sees any one thing, he sees something that exists. Do you suppose that what is one is ever to be found among non-existing things?

\par \textbf{THEAETETUS}
\par   I do not.

\par \textbf{SOCRATES}
\par   He then who sees some one thing, sees something which is?

\par \textbf{THEAETETUS}
\par   Clearly.

\par \textbf{SOCRATES}
\par   And he who hears anything, hears some one thing, and hears that which is?

\par \textbf{THEAETETUS}
\par   Yes.

\par \textbf{SOCRATES}
\par   And he who touches anything, touches something which is one and therefore is?

\par \textbf{THEAETETUS}
\par   That again is true.

\par \textbf{SOCRATES}
\par   And does not he who thinks, think some one thing?

\par \textbf{THEAETETUS}
\par   Certainly.

\par \textbf{SOCRATES}
\par   And does not he who thinks some one thing, think something which is?

\par \textbf{THEAETETUS}
\par   I agree.

\par \textbf{SOCRATES}
\par   Then he who thinks of that which is not, thinks of nothing?

\par \textbf{THEAETETUS}
\par   Clearly.

\par \textbf{SOCRATES}
\par   And he who thinks of nothing, does not think at all?

\par \textbf{THEAETETUS}
\par   Obviously.

\par \textbf{SOCRATES}
\par   Then no one can think that which is not, either as a self-existent substance or as a predicate of something else?

\par \textbf{THEAETETUS}
\par   Clearly not.

\par \textbf{SOCRATES}
\par   Then to think falsely is different from thinking that which is not?

\par \textbf{THEAETETUS}
\par   It would seem so.

\par \textbf{SOCRATES}
\par   Then false opinion has no existence in us, either in the sphere of being or of knowledge?

\par \textbf{THEAETETUS}
\par   Certainly not.

\par \textbf{SOCRATES}
\par   But may not the following be the description of what we express by this name?

\par \textbf{THEAETETUS}
\par   What?

\par \textbf{SOCRATES}
\par   May we not suppose that false opinion or thought is a sort of heterodoxy; a person may make an exchange in his mind, and say that one real object is another real object. For thus he always thinks that which is, but he puts one thing in place of another; and missing the aim of his thoughts, he may be truly said to have false opinion.

\par \textbf{THEAETETUS}
\par   Now you appear to me to have spoken the exact truth:  when a man puts the base in the place of the noble, or the noble in the place of the base, then he has truly false opinion.

\par \textbf{SOCRATES}
\par   I see, Theaetetus, that your fear has disappeared, and that you are beginning to despise me.

\par \textbf{THEAETETUS}
\par   What makes you say so?

\par \textbf{SOCRATES}
\par   You think, if I am not mistaken, that your 'truly false' is safe from censure, and that I shall never ask whether there can be a swift which is slow, or a heavy which is light, or any other self-contradictory thing, which works, not according to its own nature, but according to that of its opposite. But I will not insist upon this, for I do not wish needlessly to discourage you. And so you are satisfied that false opinion is heterodoxy, or the thought of something else?

\par \textbf{THEAETETUS}
\par   I am.

\par \textbf{SOCRATES}
\par   It is possible then upon your view for the mind to conceive of one thing as another?

\par \textbf{THEAETETUS}
\par   True.

\par \textbf{SOCRATES}
\par   But must not the mind, or thinking power, which misplaces them, have a conception either of both objects or of one of them?

\par \textbf{THEAETETUS}
\par   Certainly.

\par \textbf{SOCRATES}
\par   Either together or in succession?

\par \textbf{THEAETETUS}
\par   Very good.

\par \textbf{SOCRATES}
\par   And do you mean by conceiving, the same which I mean?

\par \textbf{THEAETETUS}
\par   What is that?

\par \textbf{SOCRATES}
\par   I mean the conversation which the soul holds with herself in considering of anything. I speak of what I scarcely understand; but the soul when thinking appears to me to be just talking—asking questions of herself and answering them, affirming and denying. And when she has arrived at a decision, either gradually or by a sudden impulse, and has at last agreed, and does not doubt, this is called her opinion. I say, then, that to form an opinion is to speak, and opinion is a word spoken,—I mean, to oneself and in silence, not aloud or to another:  What think you?

\par \textbf{THEAETETUS}
\par   I agree.

\par \textbf{SOCRATES}
\par   Then when any one thinks of one thing as another, he is saying to himself that one thing is another?

\par \textbf{THEAETETUS}
\par   Yes.

\par \textbf{SOCRATES}
\par   But do you ever remember saying to yourself that the noble is certainly base, or the unjust just; or, best of all—have you ever attempted to convince yourself that one thing is another? Nay, not even in sleep, did you ever venture to say to yourself that odd is even, or anything of the kind?

\par \textbf{THEAETETUS}
\par   Never.

\par \textbf{SOCRATES}
\par   And do you suppose that any other man, either in his senses or out of them, ever seriously tried to persuade himself that an ox is a horse, or that two are one?

\par \textbf{THEAETETUS}
\par   Certainly not.

\par \textbf{SOCRATES}
\par   But if thinking is talking to oneself, no one speaking and thinking of two objects, and apprehending them both in his soul, will say and think that the one is the other of them, and I must add, that even you, lover of dispute as you are, had better let the word 'other' alone (i.e. not insist that 'one' and 'other' are the same (Both words in Greek are called eteron:  compare Parmen. ; Euthyd.)). I mean to say, that no one thinks the noble to be base, or anything of the kind.

\par \textbf{THEAETETUS}
\par   I will give up the word 'other,' Socrates; and I agree to what you say.

\par \textbf{SOCRATES}
\par   If a man has both of them in his thoughts, he cannot think that the one of them is the other?

\par \textbf{THEAETETUS}
\par   True.

\par \textbf{SOCRATES}
\par   Neither, if he has one of them only in his mind and not the other, can he think that one is the other?

\par \textbf{THEAETETUS}
\par   True; for we should have to suppose that he apprehends that which is not in his thoughts at all.

\par \textbf{SOCRATES}
\par   Then no one who has either both or only one of the two objects in his mind can think that the one is the other. And therefore, he who maintains that false opinion is heterodoxy is talking nonsense; for neither in this, any more than in the previous way, can false opinion exist in us.

\par \textbf{THEAETETUS}
\par   No.

\par \textbf{SOCRATES}
\par   But if, Theaetetus, this is not admitted, we shall be driven into many absurdities.

\par \textbf{THEAETETUS}
\par   What are they?

\par \textbf{SOCRATES}
\par   I will not tell you until I have endeavoured to consider the matter from every point of view. For I should be ashamed of us if we were driven in our perplexity to admit the absurd consequences of which I speak. But if we find the solution, and get away from them, we may regard them only as the difficulties of others, and the ridicule will not attach to us. On the other hand, if we utterly fail, I suppose that we must be humble, and allow the argument to trample us under foot, as the sea-sick passenger is trampled upon by the sailor, and to do anything to us. Listen, then, while I tell you how I hope to find a way out of our difficulty.

\par \textbf{THEAETETUS}
\par   Let me hear.

\par \textbf{SOCRATES}
\par   I think that we were wrong in denying that a man could think what he knew to be what he did not know; and that there is a way in which such a deception is possible.

\par \textbf{THEAETETUS}
\par   You mean to say, as I suspected at the time, that I may know Socrates, and at a distance see some one who is unknown to me, and whom I mistake for him—then the deception will occur?

\par \textbf{SOCRATES}
\par   But has not that position been relinquished by us, because involving the absurdity that we should know and not know the things which we know?

\par \textbf{THEAETETUS}
\par   True.

\par \textbf{SOCRATES}
\par   Let us make the assertion in another form, which may or may not have a favourable issue; but as we are in a great strait, every argument should be turned over and tested. Tell me, then, whether I am right in saying that you may learn a thing which at one time you did not know?

\par \textbf{THEAETETUS}
\par   Certainly you may.

\par \textbf{SOCRATES}
\par   And another and another?

\par \textbf{THEAETETUS}
\par   Yes.

\par \textbf{SOCRATES}
\par   I would have you imagine, then, that there exists in the mind of man a block of wax, which is of different sizes in different men; harder, moister, and having more or less of purity in one than another, and in some of an intermediate quality.

\par \textbf{THEAETETUS}
\par   I see.

\par \textbf{SOCRATES}
\par   Let us say that this tablet is a gift of Memory, the mother of the Muses; and that when we wish to remember anything which we have seen, or heard, or thought in our own minds, we hold the wax to the perceptions and thoughts, and in that material receive the impression of them as from the seal of a ring; and that we remember and know what is imprinted as long as the image lasts; but when the image is effaced, or cannot be taken, then we forget and do not know.

\par \textbf{THEAETETUS}
\par   Very good.

\par \textbf{SOCRATES}
\par   Now, when a person has this knowledge, and is considering something which he sees or hears, may not false opinion arise in the following manner?

\par \textbf{THEAETETUS}
\par   In what manner?

\par \textbf{SOCRATES}
\par   When he thinks what he knows, sometimes to be what he knows, and sometimes to be what he does not know. We were wrong before in denying the possibility of this.

\par \textbf{THEAETETUS}
\par   And how would you amend the former statement?

\par \textbf{SOCRATES}
\par   I should begin by making a list of the impossible cases which must be excluded. (1) No one can think one thing to be another when he does not perceive either of them, but has the memorial or seal of both of them in his mind; nor can any mistaking of one thing for another occur, when he only knows one, and does not know, and has no impression of the other; nor can he think that one thing which he does not know is another thing which he does not know, or that what he does not know is what he knows; nor (2) that one thing which he perceives is another thing which he perceives, or that something which he perceives is something which he does not perceive; or that something which he does not perceive is something else which he does not perceive; or that something which he does not perceive is something which he perceives; nor again (3) can he think that something which he knows and perceives, and of which he has the impression coinciding with sense, is something else which he knows and perceives, and of which he has the impression coinciding with sense;—this last case, if possible, is still more inconceivable than the others; nor (4) can he think that something which he knows and perceives, and of which he has the memorial coinciding with sense, is something else which he knows; nor so long as these agree, can he think that a thing which he knows and perceives is another thing which he perceives; or that a thing which he does not know and does not perceive, is the same as another thing which he does not know and does not perceive;—nor again, can he suppose that a thing which he does not know and does not perceive is the same as another thing which he does not know; or that a thing which he does not know and does not perceive is another thing which he does not perceive: —All these utterly and absolutely exclude the possibility of false opinion. The only cases, if any, which remain, are the following.

\par \textbf{THEAETETUS}
\par   What are they? If you tell me, I may perhaps understand you better; but at present I am unable to follow you.

\par \textbf{SOCRATES}
\par   A person may think that some things which he knows, or which he perceives and does not know, are some other things which he knows and perceives; or that some things which he knows and perceives, are other things which he knows and perceives.

\par \textbf{THEAETETUS}
\par   I understand you less than ever now.

\par \textbf{SOCRATES}
\par   Hear me once more, then: —I, knowing Theodorus, and remembering in my own mind what sort of person he is, and also what sort of person Theaetetus is, at one time see them, and at another time do not see them, and sometimes I touch them, and at another time not, or at one time I may hear them or perceive them in some other way, and at another time not perceive them, but still I remember them, and know them in my own mind.

\par \textbf{THEAETETUS}
\par   Very true.

\par \textbf{SOCRATES}
\par   Then, first of all, I want you to understand that a man may or may not perceive sensibly that which he knows.

\par \textbf{THEAETETUS}
\par   True.

\par \textbf{SOCRATES}
\par   And that which he does not know will sometimes not be perceived by him and sometimes will be perceived and only perceived?

\par \textbf{THEAETETUS}
\par   That is also true.

\par \textbf{SOCRATES}
\par   See whether you can follow me better now:  Socrates can recognize Theodorus and Theaetetus, but he sees neither of them, nor does he perceive them in any other way; he cannot then by any possibility imagine in his own mind that Theaetetus is Theodorus. Am I not right?

\par \textbf{THEAETETUS}
\par   You are quite right.

\par \textbf{SOCRATES}
\par   Then that was the first case of which I spoke.

\par \textbf{THEAETETUS}
\par   Yes.

\par \textbf{SOCRATES}
\par   The second case was, that I, knowing one of you and not knowing the other, and perceiving neither, can never think him whom I know to be him whom I do not know.

\par \textbf{THEAETETUS}
\par   True.

\par \textbf{SOCRATES}
\par   In the third case, not knowing and not perceiving either of you, I cannot think that one of you whom I do not know is the other whom I do not know. I need not again go over the catalogue of excluded cases, in which I cannot form a false opinion about you and Theodorus, either when I know both or when I am in ignorance of both, or when I know one and not the other. And the same of perceiving:  do you understand me?

\par \textbf{THEAETETUS}
\par   I do.

\par \textbf{SOCRATES}
\par   The only possibility of erroneous opinion is, when knowing you and Theodorus, and having on the waxen block the impression of both of you given as by a seal, but seeing you imperfectly and at a distance, I try to assign the right impression of memory to the right visual impression, and to fit this into its own print:  if I succeed, recognition will take place; but if I fail and transpose them, putting the foot into the wrong shoe—that is to say, putting the vision of either of you on to the wrong impression, or if my mind, like the sight in a mirror, which is transferred from right to left, err by reason of some similar affection, then 'heterodoxy' and false opinion ensues.

\par \textbf{THEAETETUS}
\par   Yes, Socrates, you have described the nature of opinion with wonderful exactness.

\par \textbf{SOCRATES}
\par   Or again, when I know both of you, and perceive as well as know one of you, but not the other, and my knowledge of him does not accord with perception—that was the case put by me just now which you did not understand.

\par \textbf{THEAETETUS}
\par   No, I did not.

\par \textbf{SOCRATES}
\par   I meant to say, that when a person knows and perceives one of you, his knowledge coincides with his perception, he will never think him to be some other person, whom he knows and perceives, and the knowledge of whom coincides with his perception—for that also was a case supposed.

\par \textbf{THEAETETUS}
\par   True.

\par \textbf{SOCRATES}
\par   But there was an omission of the further case, in which, as we now say, false opinion may arise, when knowing both, and seeing, or having some other sensible perception of both, I fail in holding the seal over against the corresponding sensation; like a bad archer, I miss and fall wide of the mark—and this is called falsehood.

\par \textbf{THEAETETUS}
\par   Yes; it is rightly so called.

\par \textbf{SOCRATES}
\par   When, therefore, perception is present to one of the seals or impressions but not to the other, and the mind fits the seal of the absent perception on the one which is present, in any case of this sort the mind is deceived; in a word, if our view is sound, there can be no error or deception about things which a man does not know and has never perceived, but only in things which are known and perceived; in these alone opinion turns and twists about, and becomes alternately true and false;—true when the seals and impressions of sense meet straight and opposite—false when they go awry and crooked.

\par \textbf{THEAETETUS}
\par   And is not that, Socrates, nobly said?

\par \textbf{SOCRATES}
\par   Nobly! yes; but wait a little and hear the explanation, and then you will say so with more reason; for to think truly is noble and to be deceived is base.

\par \textbf{THEAETETUS}
\par   Undoubtedly.

\par \textbf{SOCRATES}
\par   And the origin of truth and error is as follows: —When the wax in the soul of any one is deep and abundant, and smooth and perfectly tempered, then the impressions which pass through the senses and sink into the heart of the soul, as Homer says in a parable, meaning to indicate the likeness of the soul to wax (Kerh Kerhos); these, I say, being pure and clear, and having a sufficient depth of wax, are also lasting, and minds, such as these, easily learn and easily retain, and are not liable to confusion, but have true thoughts, for they have plenty of room, and having clear impressions of things, as we term them, quickly distribute them into their proper places on the block. And such men are called wise. Do you agree?

\par \textbf{THEAETETUS}
\par   Entirely.

\par \textbf{SOCRATES}
\par   But when the heart of any one is shaggy—a quality which the all-wise poet commends, or muddy and of impure wax, or very soft, or very hard, then there is a corresponding defect in the mind—the soft are good at learning, but apt to forget; and the hard are the reverse; the shaggy and rugged and gritty, or those who have an admixture of earth or dung in their composition, have the impressions indistinct, as also the hard, for there is no depth in them; and the soft too are indistinct, for their impressions are easily confused and effaced. Yet greater is the indistinctness when they are all jostled together in a little soul, which has no room. These are the natures which have false opinion; for when they see or hear or think of anything, they are slow in assigning the right objects to the right impressions—in their stupidity they confuse them, and are apt to see and hear and think amiss—and such men are said to be deceived in their knowledge of objects, and ignorant.

\par \textbf{THEAETETUS}
\par   No man, Socrates, can say anything truer than that.

\par \textbf{SOCRATES}
\par   Then now we may admit the existence of false opinion in us?

\par \textbf{THEAETETUS}
\par   Certainly.

\par \textbf{SOCRATES}
\par   And of true opinion also?

\par \textbf{THEAETETUS}
\par   Yes.

\par \textbf{SOCRATES}
\par   We have at length satisfactorily proven beyond a doubt there are these two sorts of opinion?

\par \textbf{THEAETETUS}
\par   Undoubtedly.

\par \textbf{SOCRATES}
\par   Alas, Theaetetus, what a tiresome creature is a man who is fond of talking!

\par \textbf{THEAETETUS}
\par   What makes you say so?

\par \textbf{SOCRATES}
\par   Because I am disheartened at my own stupidity and tiresome garrulity; for what other term will describe the habit of a man who is always arguing on all sides of a question; whose dulness cannot be convinced, and who will never leave off?

\par \textbf{THEAETETUS}
\par   But what puts you out of heart?

\par \textbf{SOCRATES}
\par   I am not only out of heart, but in positive despair; for I do not know what to answer if any one were to ask me: —O Socrates, have you indeed discovered that false opinion arises neither in the comparison of perceptions with one another nor yet in thought, but in union of thought and perception? Yes, I shall say, with the complacence of one who thinks that he has made a noble discovery.

\par \textbf{THEAETETUS}
\par   I see no reason why we should be ashamed of our demonstration, Socrates.

\par \textbf{SOCRATES}
\par   He will say:  You mean to argue that the man whom we only think of and do not see, cannot be confused with the horse which we do not see or touch, but only think of and do not perceive? That I believe to be my meaning, I shall reply.

\par \textbf{THEAETETUS}
\par   Quite right.

\par \textbf{SOCRATES}
\par   Well, then, he will say, according to that argument, the number eleven, which is only thought, can never be mistaken for twelve, which is only thought:  How would you answer him?

\par \textbf{THEAETETUS}
\par   I should say that a mistake may very likely arise between the eleven or twelve which are seen or handled, but that no similar mistake can arise between the eleven and twelve which are in the mind.

\par \textbf{SOCRATES}
\par   Well, but do you think that no one ever put before his own mind five and seven,—I do not mean five or seven men or horses, but five or seven in the abstract, which, as we say, are recorded on the waxen block, and in which false opinion is held to be impossible; did no man ever ask himself how many these numbers make when added together, and answer that they are eleven, while another thinks that they are twelve, or would all agree in thinking and saying that they are twelve?

\par \textbf{THEAETETUS}
\par   Certainly not; many would think that they are eleven, and in the higher numbers the chance of error is greater still; for I assume you to be speaking of numbers in general.

\par \textbf{SOCRATES}
\par   Exactly; and I want you to consider whether this does not imply that the twelve in the waxen block are supposed to be eleven?

\par \textbf{THEAETETUS}
\par   Yes, that seems to be the case.

\par \textbf{SOCRATES}
\par   Then do we not come back to the old difficulty? For he who makes such a mistake does think one thing which he knows to be another thing which he knows; but this, as we said, was impossible, and afforded an irresistible proof of the non-existence of false opinion, because otherwise the same person would inevitably know and not know the same thing at the same time.

\par \textbf{THEAETETUS}
\par   Most true.

\par \textbf{SOCRATES}
\par   Then false opinion cannot be explained as a confusion of thought and sense, for in that case we could not have been mistaken about pure conceptions of thought; and thus we are obliged to say, either that false opinion does not exist, or that a man may not know that which he knows;—which alternative do you prefer?

\par \textbf{THEAETETUS}
\par   It is hard to determine, Socrates.

\par \textbf{SOCRATES}
\par   And yet the argument will scarcely admit of both. But, as we are at our wits' end, suppose that we do a shameless thing?

\par \textbf{THEAETETUS}
\par   What is it?

\par \textbf{SOCRATES}
\par   Let us attempt to explain the verb 'to know.'

\par \textbf{THEAETETUS}
\par   And why should that be shameless?

\par \textbf{SOCRATES}
\par   You seem not to be aware that the whole of our discussion from the very beginning has been a search after knowledge, of which we are assumed not to know the nature.

\par \textbf{THEAETETUS}
\par   Nay, but I am well aware.

\par \textbf{SOCRATES}
\par   And is it not shameless when we do not know what knowledge is, to be explaining the verb 'to know'? The truth is, Theaetetus, that we have long been infected with logical impurity. Thousands of times have we repeated the words 'we know,' and 'do not know,' and 'we have or have not science or knowledge,' as if we could understand what we are saying to one another, so long as we remain ignorant about knowledge; and at this moment we are using the words 'we understand,' 'we are ignorant,' as though we could still employ them when deprived of knowledge or science.

\par \textbf{THEAETETUS}
\par   But if you avoid these expressions, Socrates, how will you ever argue at all?

\par \textbf{SOCRATES}
\par   I could not, being the man I am. The case would be different if I were a true hero of dialectic:  and O that such an one were present! for he would have told us to avoid the use of these terms; at the same time he would not have spared in you and me the faults which I have noted. But, seeing that we are no great wits, shall I venture to say what knowing is? for I think that the attempt may be worth making.

\par \textbf{THEAETETUS}
\par   Then by all means venture, and no one shall find fault with you for using the forbidden terms.

\par \textbf{SOCRATES}
\par   You have heard the common explanation of the verb 'to know'?

\par \textbf{THEAETETUS}
\par   I think so, but I do not remember it at the moment.

\par \textbf{SOCRATES}
\par   They explain the word 'to know' as meaning 'to have knowledge.'

\par \textbf{THEAETETUS}
\par   True.

\par \textbf{SOCRATES}
\par   I should like to make a slight change, and say 'to possess' knowledge.

\par \textbf{THEAETETUS}
\par   How do the two expressions differ?

\par \textbf{SOCRATES}
\par   Perhaps there may be no difference; but still I should like you to hear my view, that you may help me to test it.

\par \textbf{THEAETETUS}
\par   I will, if I can.

\par \textbf{SOCRATES}
\par   I should distinguish 'having' from 'possessing':  for example, a man may buy and keep under his control a garment which he does not wear; and then we should say, not that he has, but that he possesses the garment.

\par \textbf{THEAETETUS}
\par   It would be the correct expression.

\par \textbf{SOCRATES}
\par   Well, may not a man 'possess' and yet not 'have' knowledge in the sense of which I am speaking? As you may suppose a man to have caught wild birds—doves or any other birds—and to be keeping them in an aviary which he has constructed at home; we might say of him in one sense, that he always has them because he possesses them, might we not?

\par \textbf{THEAETETUS}
\par   Yes.

\par \textbf{SOCRATES}
\par   And yet, in another sense, he has none of them; but they are in his power, and he has got them under his hand in an enclosure of his own, and can take and have them whenever he likes;—he can catch any which he likes, and let the bird go again, and he may do so as often as he pleases.

\par \textbf{THEAETETUS}
\par   True.

\par \textbf{SOCRATES}
\par   Once more, then, as in what preceded we made a sort of waxen figment in the mind, so let us now suppose that in the mind of each man there is an aviary of all sorts of birds—some flocking together apart from the rest, others in small groups, others solitary, flying anywhere and everywhere.

\par \textbf{THEAETETUS}
\par   Let us imagine such an aviary—and what is to follow?

\par \textbf{SOCRATES}
\par   We may suppose that the birds are kinds of knowledge, and that when we were children, this receptacle was empty; whenever a man has gotten and detained in the enclosure a kind of knowledge, he may be said to have learned or discovered the thing which is the subject of the knowledge:  and this is to know.

\par \textbf{THEAETETUS}
\par   Granted.

\par \textbf{SOCRATES}
\par   And further, when any one wishes to catch any of these knowledges or sciences, and having taken, to hold it, and again to let them go, how will he express himself?—will he describe the 'catching' of them and the original 'possession' in the same words? I will make my meaning clearer by an example: —You admit that there is an art of arithmetic?

\par \textbf{THEAETETUS}
\par   To be sure.

\par \textbf{SOCRATES}
\par   Conceive this under the form of a hunt after the science of odd and even in general.

\par \textbf{THEAETETUS}
\par   I follow.

\par \textbf{SOCRATES}
\par   Having the use of the art, the arithmetician, if I am not mistaken, has the conceptions of number under his hand, and can transmit them to another.

\par \textbf{THEAETETUS}
\par   Yes.

\par \textbf{SOCRATES}
\par   And when transmitting them he may be said to teach them, and when receiving to learn them, and when receiving to learn them, and when having them in possession in the aforesaid aviary he may be said to know them.

\par \textbf{THEAETETUS}
\par   Exactly.

\par \textbf{SOCRATES}
\par   Attend to what follows:  must not the perfect arithmetician know all numbers, for he has the science of all numbers in his mind?

\par \textbf{THEAETETUS}
\par   True.

\par \textbf{SOCRATES}
\par   And he can reckon abstract numbers in his head, or things about him which are numerable?

\par \textbf{THEAETETUS}
\par   Of course he can.

\par \textbf{SOCRATES}
\par   And to reckon is simply to consider how much such and such a number amounts to?

\par \textbf{THEAETETUS}
\par   Very true.

\par \textbf{SOCRATES}
\par   And so he appears to be searching into something which he knows, as if he did not know it, for we have already admitted that he knows all numbers;—you have heard these perplexing questions raised?

\par \textbf{THEAETETUS}
\par   I have.

\par \textbf{SOCRATES}
\par   May we not pursue the image of the doves, and say that the chase after knowledge is of two kinds? one kind is prior to possession and for the sake of possession, and the other for the sake of taking and holding in the hands that which is possessed already. And thus, when a man has learned and known something long ago, he may resume and get hold of the knowledge which he has long possessed, but has not at hand in his mind.

\par \textbf{THEAETETUS}
\par   True.

\par \textbf{SOCRATES}
\par   That was my reason for asking how we ought to speak when an arithmetician sets about numbering, or a grammarian about reading? Shall we say, that although he knows, he comes back to himself to learn what he already knows?

\par \textbf{THEAETETUS}
\par   It would be too absurd, Socrates.

\par \textbf{SOCRATES}
\par   Shall we say then that he is going to read or number what he does not know, although we have admitted that he knows all letters and all numbers?

\par \textbf{THEAETETUS}
\par   That, again, would be an absurdity.

\par \textbf{SOCRATES}
\par   Then shall we say that about names we care nothing?—any one may twist and turn the words 'knowing' and 'learning' in any way which he likes, but since we have determined that the possession of knowledge is not the having or using it, we do assert that a man cannot not possess that which he possesses; and, therefore, in no case can a man not know that which he knows, but he may get a false opinion about it; for he may have the knowledge, not of this particular thing, but of some other;—when the various numbers and forms of knowledge are flying about in the aviary, and wishing to capture a certain sort of knowledge out of the general store, he takes the wrong one by mistake, that is to say, when he thought eleven to be twelve, he got hold of the ring-dove which he had in his mind, when he wanted the pigeon.

\par \textbf{THEAETETUS}
\par   A very rational explanation.

\par \textbf{SOCRATES}
\par   But when he catches the one which he wants, then he is not deceived, and has an opinion of what is, and thus false and true opinion may exist, and the difficulties which were previously raised disappear. I dare say that you agree with me, do you not?

\par \textbf{THEAETETUS}
\par   Yes.

\par \textbf{SOCRATES}
\par   And so we are rid of the difficulty of a man's not knowing what he knows, for we are not driven to the inference that he does not possess what he possesses, whether he be or be not deceived. And yet I fear that a greater difficulty is looking in at the window.

\par \textbf{THEAETETUS}
\par   What is it?

\par \textbf{SOCRATES}
\par   How can the exchange of one knowledge for another ever become false opinion?

\par \textbf{THEAETETUS}
\par   What do you mean?

\par \textbf{SOCRATES}
\par   In the first place, how can a man who has the knowledge of anything be ignorant of that which he knows, not by reason of ignorance, but by reason of his own knowledge? And, again, is it not an extreme absurdity that he should suppose another thing to be this, and this to be another thing;—that, having knowledge present with him in his mind, he should still know nothing and be ignorant of all things?—you might as well argue that ignorance may make a man know, and blindness make him see, as that knowledge can make him ignorant.

\par \textbf{THEAETETUS}
\par   Perhaps, Socrates, we may have been wrong in making only forms of knowledge our birds:  whereas there ought to have been forms of ignorance as well, flying about together in the mind, and then he who sought to take one of them might sometimes catch a form of knowledge, and sometimes a form of ignorance; and thus he would have a false opinion from ignorance, but a true one from knowledge, about the same thing.

\par \textbf{SOCRATES}
\par   I cannot help praising you, Theaetetus, and yet I must beg you to reconsider your words. Let us grant what you say—then, according to you, he who takes ignorance will have a false opinion—am I right?

\par \textbf{THEAETETUS}
\par   Yes.

\par \textbf{SOCRATES}
\par   He will certainly not think that he has a false opinion?

\par \textbf{THEAETETUS}
\par   Of course not.

\par \textbf{SOCRATES}
\par   He will think that his opinion is true, and he will fancy that he knows the things about which he has been deceived?

\par \textbf{THEAETETUS}
\par   Certainly.

\par \textbf{SOCRATES}
\par   Then he will think that he has captured knowledge and not ignorance?

\par \textbf{THEAETETUS}
\par   Clearly.

\par \textbf{SOCRATES}
\par   And thus, after going a long way round, we are once more face to face with our original difficulty. The hero of dialectic will retort upon us: —'O my excellent friends, he will say, laughing, if a man knows the form of ignorance and the form of knowledge, can he think that one of them which he knows is the other which he knows? or, if he knows neither of them, can he think that the one which he knows not is another which he knows not? or, if he knows one and not the other, can he think the one which he knows to be the one which he does not know? or the one which he does not know to be the one which he knows? or will you tell me that there are other forms of knowledge which distinguish the right and wrong birds, and which the owner keeps in some other aviaries or graven on waxen blocks according to your foolish images, and which he may be said to know while he possesses them, even though he have them not at hand in his mind? And thus, in a perpetual circle, you will be compelled to go round and round, and you will make no progress.' What are we to say in reply, Theaetetus?

\par \textbf{THEAETETUS}
\par   Indeed, Socrates, I do not know what we are to say.

\par \textbf{SOCRATES}
\par   Are not his reproaches just, and does not the argument truly show that we are wrong in seeking for false opinion until we know what knowledge is; that must be first ascertained; then, the nature of false opinion?

\par \textbf{THEAETETUS}
\par   I cannot but agree with you, Socrates, so far as we have yet gone.

\par \textbf{SOCRATES}
\par   Then, once more, what shall we say that knowledge is?—for we are not going to lose heart as yet.

\par \textbf{THEAETETUS}
\par   Certainly, I shall not lose heart, if you do not.

\par \textbf{SOCRATES}
\par   What definition will be most consistent with our former views?

\par \textbf{THEAETETUS}
\par   I cannot think of any but our old one, Socrates.

\par \textbf{SOCRATES}
\par   What was it?

\par \textbf{THEAETETUS}
\par   Knowledge was said by us to be true opinion; and true opinion is surely unerring, and the results which follow from it are all noble and good.

\par \textbf{SOCRATES}
\par   He who led the way into the river, Theaetetus, said 'The experiment will show;' and perhaps if we go forward in the search, we may stumble upon the thing which we are looking for; but if we stay where we are, nothing will come to light.

\par \textbf{THEAETETUS}
\par   Very true; let us go forward and try.

\par \textbf{SOCRATES}
\par   The trail soon comes to an end, for a whole profession is against us.

\par \textbf{THEAETETUS}
\par   How is that, and what profession do you mean?

\par \textbf{SOCRATES}
\par   The profession of the great wise ones who are called orators and lawyers; for these persuade men by their art and make them think whatever they like, but they do not teach them. Do you imagine that there are any teachers in the world so clever as to be able to convince others of the truth about acts of robbery or violence, of which they were not eye-witnesses, while a little water is flowing in the clepsydra?

\par \textbf{THEAETETUS}
\par   Certainly not, they can only persuade them.

\par \textbf{SOCRATES}
\par   And would you not say that persuading them is making them have an opinion?

\par \textbf{THEAETETUS}
\par   To be sure.

\par \textbf{SOCRATES}
\par   When, therefore, judges are justly persuaded about matters which you can know only by seeing them, and not in any other way, and when thus judging of them from report they attain a true opinion about them, they judge without knowledge, and yet are rightly persuaded, if they have judged well.

\par \textbf{THEAETETUS}
\par   Certainly.

\par \textbf{SOCRATES}
\par   And yet, O my friend, if true opinion in law courts and knowledge are the same, the perfect judge could not have judged rightly without knowledge; and therefore I must infer that they are not the same.

\par \textbf{THEAETETUS}
\par   That is a distinction, Socrates, which I have heard made by some one else, but I had forgotten it. He said that true opinion, combined with reason, was knowledge, but that the opinion which had no reason was out of the sphere of knowledge; and that things of which there is no rational account are not knowable—such was the singular expression which he used—and that things which have a reason or explanation are knowable.

\par \textbf{SOCRATES}
\par   Excellent; but then, how did he distinguish between things which are and are not 'knowable'? I wish that you would repeat to me what he said, and then I shall know whether you and I have heard the same tale.

\par \textbf{THEAETETUS}
\par   I do not know whether I can recall it; but if another person would tell me, I think that I could follow him.

\par \textbf{SOCRATES}
\par   Let me give you, then, a dream in return for a dream: —Methought that I too had a dream, and I heard in my dream that the primeval letters or elements out of which you and I and all other things are compounded, have no reason or explanation; you can only name them, but no predicate can be either affirmed or denied of them, for in the one case existence, in the other non-existence is already implied, neither of which must be added, if you mean to speak of this or that thing by itself alone. It should not be called itself, or that, or each, or alone, or this, or the like; for these go about everywhere and are applied to all things, but are distinct from them; whereas, if the first elements could be described, and had a definition of their own, they would be spoken of apart from all else. But none of these primeval elements can be defined; they can only be named, for they have nothing but a name, and the things which are compounded of them, as they are complex, are expressed by a combination of names, for the combination of names is the essence of a definition. Thus, then, the elements or letters are only objects of perception, and cannot be defined or known; but the syllables or combinations of them are known and expressed, and are apprehended by true opinion. When, therefore, any one forms the true opinion of anything without rational explanation, you may say that his mind is truly exercised, but has no knowledge; for he who cannot give and receive a reason for a thing, has no knowledge of that thing; but when he adds rational explanation, then, he is perfected in knowledge and may be all that I have been denying of him. Was that the form in which the dream appeared to you?

\par \textbf{THEAETETUS}
\par   Precisely.

\par \textbf{SOCRATES}
\par   And you allow and maintain that true opinion, combined with definition or rational explanation, is knowledge?

\par \textbf{THEAETETUS}
\par   Exactly.

\par \textbf{SOCRATES}
\par   Then may we assume, Theaetetus, that to-day, and in this casual manner, we have found a truth which in former times many wise men have grown old and have not found?

\par \textbf{THEAETETUS}
\par   At any rate, Socrates, I am satisfied with the present statement.

\par \textbf{SOCRATES}
\par   Which is probably correct—for how can there be knowledge apart from definition and true opinion? And yet there is one point in what has been said which does not quite satisfy me.

\par \textbf{THEAETETUS}
\par   What was it?

\par \textbf{SOCRATES}
\par   What might seem to be the most ingenious notion of all: —That the elements or letters are unknown, but the combination or syllables known.

\par \textbf{THEAETETUS}
\par   And was that wrong?

\par \textbf{SOCRATES}
\par   We shall soon know; for we have as hostages the instances which the author of the argument himself used.

\par \textbf{THEAETETUS}
\par   What hostages?

\par \textbf{SOCRATES}
\par   The letters, which are the clements; and the syllables, which are the combinations;—he reasoned, did he not, from the letters of the alphabet?

\par \textbf{THEAETETUS}
\par   Yes; he did.

\par \textbf{SOCRATES}
\par   Let us take them and put them to the test, or rather, test ourselves: —What was the way in which we learned letters? and, first of all, are we right in saying that syllables have a definition, but that letters have no definition?

\par \textbf{THEAETETUS}
\par   I think so.

\par \textbf{SOCRATES}
\par   I think so too; for, suppose that some one asks you to spell the first syllable of my name: —Theaetetus, he says, what is SO?

\par \textbf{THEAETETUS}
\par   I should reply S and O.

\par \textbf{SOCRATES}
\par   That is the definition which you would give of the syllable?

\par \textbf{THEAETETUS}
\par   I should.

\par \textbf{SOCRATES}
\par   I wish that you would give me a similar definition of the S.

\par \textbf{THEAETETUS}
\par   But how can any one, Socrates, tell the elements of an element? I can only reply, that S is a consonant, a mere noise, as of the tongue hissing; B, and most other letters, again, are neither vowel-sounds nor noises. Thus letters may be most truly said to be undefined; for even the most distinct of them, which are the seven vowels, have a sound only, but no definition at all.

\par \textbf{SOCRATES}
\par   Then, I suppose, my friend, that we have been so far right in our idea about knowledge?

\par \textbf{THEAETETUS}
\par   Yes; I think that we have.

\par \textbf{SOCRATES}
\par   Well, but have we been right in maintaining that the syllables can be known, but not the letters?

\par \textbf{THEAETETUS}
\par   I think so.

\par \textbf{SOCRATES}
\par   And do we mean by a syllable two letters, or if there are more, all of them, or a single idea which arises out of the combination of them?

\par \textbf{THEAETETUS}
\par   I should say that we mean all the letters.

\par \textbf{SOCRATES}
\par   Take the case of the two letters S and O, which form the first syllable of my own name; must not he who knows the syllable, know both of them?

\par \textbf{THEAETETUS}
\par   Certainly.

\par \textbf{SOCRATES}
\par   He knows, that is, the S and O?

\par \textbf{THEAETETUS}
\par   Yes.

\par \textbf{SOCRATES}
\par   But can he be ignorant of either singly and yet know both together?

\par \textbf{THEAETETUS}
\par   Such a supposition, Socrates, is monstrous and unmeaning.

\par \textbf{SOCRATES}
\par   But if he cannot know both without knowing each, then if he is ever to know the syllable, he must know the letters first; and thus the fine theory has again taken wings and departed.

\par \textbf{THEAETETUS}
\par   Yes, with wonderful celerity.

\par \textbf{SOCRATES}
\par   Yes, we did not keep watch properly. Perhaps we ought to have maintained that a syllable is not the letters, but rather one single idea framed out of them, having a separate form distinct from them.

\par \textbf{THEAETETUS}
\par   Very true; and a more likely notion than the other.

\par \textbf{SOCRATES}
\par   Take care; let us not be cowards and betray a great and imposing theory.

\par \textbf{THEAETETUS}
\par   No, indeed.

\par \textbf{SOCRATES}
\par   Let us assume then, as we now say, that the syllable is a simple form arising out of the several combinations of harmonious elements—of letters or of any other elements.

\par \textbf{THEAETETUS}
\par   Very good.

\par \textbf{SOCRATES}
\par   And it must have no parts.

\par \textbf{THEAETETUS}
\par   Why?

\par \textbf{SOCRATES}
\par   Because that which has parts must be a whole of all the parts. Or would you say that a whole, although formed out of the parts, is a single notion different from all the parts?

\par \textbf{THEAETETUS}
\par   I should.

\par \textbf{SOCRATES}
\par   And would you say that all and the whole are the same, or different?

\par \textbf{THEAETETUS}
\par   I am not certain; but, as you like me to answer at once, I shall hazard the reply, that they are different.

\par \textbf{SOCRATES}
\par   I approve of your readiness, Theaetetus, but I must take time to think whether I equally approve of your answer.

\par \textbf{THEAETETUS}
\par   Yes; the answer is the point.

\par \textbf{SOCRATES}
\par   According to this new view, the whole is supposed to differ from all?

\par \textbf{THEAETETUS}
\par   Yes.

\par \textbf{SOCRATES}
\par   Well, but is there any difference between all (in the plural) and the all (in the singular)? Take the case of number: —When we say one, two, three, four, five, six; or when we say twice three, or three times two, or four and two, or three and two and one, are we speaking of the same or of different numbers?

\par \textbf{THEAETETUS}
\par   Of the same.

\par \textbf{SOCRATES}
\par   That is of six?

\par \textbf{THEAETETUS}
\par   Yes.

\par \textbf{SOCRATES}
\par   And in each form of expression we spoke of all the six?

\par \textbf{THEAETETUS}
\par   True.

\par \textbf{SOCRATES}
\par   Again, in speaking of all (in the plural) is there not one thing which we express?

\par \textbf{THEAETETUS}
\par   Of course there is.

\par \textbf{SOCRATES}
\par   And that is six?

\par \textbf{THEAETETUS}
\par   Yes.

\par \textbf{SOCRATES}
\par   Then in predicating the word 'all' of things measured by number, we predicate at the same time a singular and a plural?

\par \textbf{THEAETETUS}
\par   Clearly we do.

\par \textbf{SOCRATES}
\par   Again, the number of the acre and the acre are the same; are they not?

\par \textbf{THEAETETUS}
\par   Yes.

\par \textbf{SOCRATES}
\par   And the number of the stadium in like manner is the stadium?

\par \textbf{THEAETETUS}
\par   Yes.

\par \textbf{SOCRATES}
\par   And the army is the number of the army; and in all similar cases, the entire number of anything is the entire thing?

\par \textbf{THEAETETUS}
\par   True.

\par \textbf{SOCRATES}
\par   And the number of each is the parts of each?

\par \textbf{THEAETETUS}
\par   Exactly.

\par \textbf{SOCRATES}
\par   Then as many things as have parts are made up of parts?

\par \textbf{THEAETETUS}
\par   Clearly.

\par \textbf{SOCRATES}
\par   But all the parts are admitted to be the all, if the entire number is the all?

\par \textbf{THEAETETUS}
\par   True.

\par \textbf{SOCRATES}
\par   Then the whole is not made up of parts, for it would be the all, if consisting of all the parts?

\par \textbf{THEAETETUS}
\par   That is the inference.

\par \textbf{SOCRATES}
\par   But is a part a part of anything but the whole?

\par \textbf{THEAETETUS}
\par   Yes, of the all.

\par \textbf{SOCRATES}
\par   You make a valiant defence, Theaetetus. And yet is not the all that of which nothing is wanting?

\par \textbf{THEAETETUS}
\par   Certainly.

\par \textbf{SOCRATES}
\par   And is not a whole likewise that from which nothing is absent? but that from which anything is absent is neither a whole nor all;—if wanting in anything, both equally lose their entirety of nature.

\par \textbf{THEAETETUS}
\par   I now think that there is no difference between a whole and all.

\par \textbf{SOCRATES}
\par   But were we not saying that when a thing has parts, all the parts will be a whole and all?

\par \textbf{THEAETETUS}
\par   Certainly.

\par \textbf{SOCRATES}
\par   Then, as I was saying before, must not the alternative be that either the syllable is not the letters, and then the letters are not parts of the syllable, or that the syllable will be the same with the letters, and will therefore be equally known with them?

\par \textbf{THEAETETUS}
\par   You are right.

\par \textbf{SOCRATES}
\par   And, in order to avoid this, we suppose it to be different from them?

\par \textbf{THEAETETUS}
\par   Yes.

\par \textbf{SOCRATES}
\par   But if letters are not parts of syllables, can you tell me of any other parts of syllables, which are not letters?

\par \textbf{THEAETETUS}
\par   No, indeed, Socrates; for if I admit the existence of parts in a syllable, it would be ridiculous in me to give up letters and seek for other parts.

\par \textbf{SOCRATES}
\par   Quite true, Theaetetus, and therefore, according to our present view, a syllable must surely be some indivisible form?

\par \textbf{THEAETETUS}
\par   True.

\par \textbf{SOCRATES}
\par   But do you remember, my friend, that only a little while ago we admitted and approved the statement, that of the first elements out of which all other things are compounded there could be no definition, because each of them when taken by itself is uncompounded; nor can one rightly attribute to them the words 'being' or 'this,' because they are alien and inappropriate words, and for this reason the letters or elements were indefinable and unknown?

\par \textbf{THEAETETUS}
\par   I remember.

\par \textbf{SOCRATES}
\par   And is not this also the reason why they are simple and indivisible? I can see no other.

\par \textbf{THEAETETUS}
\par   No other reason can be given.

\par \textbf{SOCRATES}
\par   Then is not the syllable in the same case as the elements or letters, if it has no parts and is one form?

\par \textbf{THEAETETUS}
\par   To be sure.

\par \textbf{SOCRATES}
\par   If, then, a syllable is a whole, and has many parts or letters, the letters as well as the syllable must be intelligible and expressible, since all the parts are acknowledged to be the same as the whole?

\par \textbf{THEAETETUS}
\par   True.

\par \textbf{SOCRATES}
\par   But if it be one and indivisible, then the syllables and the letters are alike undefined and unknown, and for the same reason?

\par \textbf{THEAETETUS}
\par   I cannot deny that.

\par \textbf{SOCRATES}
\par   We cannot, therefore, agree in the opinion of him who says that the syllable can be known and expressed, but not the letters.

\par \textbf{THEAETETUS}
\par   Certainly not; if we may trust the argument.

\par \textbf{SOCRATES}
\par   Well, but will you not be equally inclined to disagree with him, when you remember your own experience in learning to read?

\par \textbf{THEAETETUS}
\par   What experience?

\par \textbf{SOCRATES}
\par   Why, that in learning you were kept trying to distinguish the separate letters both by the eye and by the ear, in order that, when you heard them spoken or saw them written, you might not be confused by their position.

\par \textbf{THEAETETUS}
\par   Very true.

\par \textbf{SOCRATES}
\par   And is the education of the harp-player complete unless he can tell what string answers to a particular note; the notes, as every one would allow, are the elements or letters of music?

\par \textbf{THEAETETUS}
\par   Exactly.

\par \textbf{SOCRATES}
\par   Then, if we argue from the letters and syllables which we know to other simples and compounds, we shall say that the letters or simple elements as a class are much more certainly known than the syllables, and much more indispensable to a perfect knowledge of any subject; and if some one says that the syllable is known and the letter unknown, we shall consider that either intentionally or unintentionally he is talking nonsense?

\par \textbf{THEAETETUS}
\par   Exactly.

\par \textbf{SOCRATES}
\par   And there might be given other proofs of this belief, if I am not mistaken. But do not let us in looking for them lose sight of the question before us, which is the meaning of the statement, that right opinion with rational definition or explanation is the most perfect form of knowledge.

\par \textbf{THEAETETUS}
\par   We must not.

\par \textbf{SOCRATES}
\par   Well, and what is the meaning of the term 'explanation'? I think that we have a choice of three meanings.

\par \textbf{THEAETETUS}
\par   What are they?

\par \textbf{SOCRATES}
\par   In the first place, the meaning may be, manifesting one's thought by the voice with verbs and nouns, imaging an opinion in the stream which flows from the lips, as in a mirror or water. Does not explanation appear to be of this nature?

\par \textbf{THEAETETUS}
\par   Certainly; he who so manifests his thought, is said to explain himself.

\par \textbf{SOCRATES}
\par   And every one who is not born deaf or dumb is able sooner or later to manifest what he thinks of anything; and if so, all those who have a right opinion about anything will also have right explanation; nor will right opinion be anywhere found to exist apart from knowledge.

\par \textbf{THEAETETUS}
\par   True.

\par \textbf{SOCRATES}
\par   Let us not, therefore, hastily charge him who gave this account of knowledge with uttering an unmeaning word; for perhaps he only intended to say, that when a person was asked what was the nature of anything, he should be able to answer his questioner by giving the elements of the thing.

\par \textbf{THEAETETUS}
\par   As for example, Socrates...?

\par \textbf{SOCRATES}
\par   As, for example, when Hesiod says that a waggon is made up of a hundred planks. Now, neither you nor I could describe all of them individually; but if any one asked what is a waggon, we should be content to answer, that a waggon consists of wheels, axle, body, rims, yoke.

\par \textbf{THEAETETUS}
\par   Certainly.

\par \textbf{SOCRATES}
\par   And our opponent will probably laugh at us, just as he would if we professed to be grammarians and to give a grammatical account of the name of Theaetetus, and yet could only tell the syllables and not the letters of your name—that would be true opinion, and not knowledge; for knowledge, as has been already remarked, is not attained until, combined with true opinion, there is an enumeration of the elements out of which anything is composed.

\par \textbf{THEAETETUS}
\par   Yes.

\par \textbf{SOCRATES}
\par   In the same general way, we might also have true opinion about a waggon; but he who can describe its essence by an enumeration of the hundred planks, adds rational explanation to true opinion, and instead of opinion has art and knowledge of the nature of a waggon, in that he attains to the whole through the elements.

\par \textbf{THEAETETUS}
\par   And do you not agree in that view, Socrates?

\par \textbf{SOCRATES}
\par   If you do, my friend; but I want to know first, whether you admit the resolution of all things into their elements to be a rational explanation of them, and the consideration of them in syllables or larger combinations of them to be irrational—is this your view?

\par \textbf{THEAETETUS}
\par   Precisely.

\par \textbf{SOCRATES}
\par   Well, and do you conceive that a man has knowledge of any element who at one time affirms and at another time denies that element of something, or thinks that the same thing is composed of different elements at different times?

\par \textbf{THEAETETUS}
\par   Assuredly not.

\par \textbf{SOCRATES}
\par   And do you not remember that in your case and in that of others this often occurred in the process of learning to read?

\par \textbf{THEAETETUS}
\par   You mean that I mistook the letters and misspelt the syllables?

\par \textbf{SOCRATES}
\par   Yes.

\par \textbf{THEAETETUS}
\par   To be sure; I perfectly remember, and I am very far from supposing that they who are in this condition have knowledge.

\par \textbf{SOCRATES}
\par   When a person at the time of learning writes the name of Theaetetus, and thinks that he ought to write and does write Th and e; but, again, meaning to write the name of Theododorus, thinks that he ought to write and does write T and e—can we suppose that he knows the first syllables of your two names?

\par \textbf{THEAETETUS}
\par   We have already admitted that such a one has not yet attained knowledge.

\par \textbf{SOCRATES}
\par   And in like manner be may enumerate without knowing them the second and third and fourth syllables of your name?

\par \textbf{THEAETETUS}
\par   He may.

\par \textbf{SOCRATES}
\par   And in that case, when he knows the order of the letters and can write them out correctly, he has right opinion?

\par \textbf{THEAETETUS}
\par   Clearly.

\par \textbf{SOCRATES}
\par   But although we admit that he has right opinion, he will still be without knowledge?

\par \textbf{THEAETETUS}
\par   Yes.

\par \textbf{SOCRATES}
\par   And yet he will have explanation, as well as right opinion, for he knew the order of the letters when he wrote; and this we admit to be explanation.

\par \textbf{THEAETETUS}
\par   True.

\par \textbf{SOCRATES}
\par   Then, my friend, there is such a thing as right opinion united with definition or explanation, which does not as yet attain to the exactness of knowledge.

\par \textbf{THEAETETUS}
\par   It would seem so.

\par \textbf{SOCRATES}
\par   And what we fancied to be a perfect definition of knowledge is a dream only. But perhaps we had better not say so as yet, for were there not three explanations of knowledge, one of which must, as we said, be adopted by him who maintains knowledge to be true opinion combined with rational explanation? And very likely there may be found some one who will not prefer this but the third.

\par \textbf{THEAETETUS}
\par   You are quite right; there is still one remaining. The first was the image or expression of the mind in speech; the second, which has just been mentioned, is a way of reaching the whole by an enumeration of the elements. But what is the third definition?

\par \textbf{SOCRATES}
\par   There is, further, the popular notion of telling the mark or sign of difference which distinguishes the thing in question from all others.

\par \textbf{THEAETETUS}
\par   Can you give me any example of such a definition?

\par \textbf{SOCRATES}
\par   As, for example, in the case of the sun, I think that you would be contented with the statement that the sun is the brightest of the heavenly bodies which revolve about the earth.

\par \textbf{THEAETETUS}
\par   Certainly.

\par \textbf{SOCRATES}
\par   Understand why: —the reason is, as I was just now saying, that if you get at the difference and distinguishing characteristic of each thing, then, as many persons affirm, you will get at the definition or explanation of it; but while you lay hold only of the common and not of the characteristic notion, you will only have the definition of those things to which this common quality belongs.

\par \textbf{THEAETETUS}
\par   I understand you, and your account of definition is in my judgment correct.

\par \textbf{SOCRATES}
\par   But he, who having right opinion about anything, can find out the difference which distinguishes it from other things will know that of which before he had only an opinion.

\par \textbf{THEAETETUS}
\par   Yes; that is what we are maintaining.

\par \textbf{SOCRATES}
\par   Nevertheless, Theaetetus, on a nearer view, I find myself quite disappointed; the picture, which at a distance was not so bad, has now become altogether unintelligible.

\par \textbf{THEAETETUS}
\par   What do you mean?

\par \textbf{SOCRATES}
\par   I will endeavour to explain:  I will suppose myself to have true opinion of you, and if to this I add your definition, then I have knowledge, but if not, opinion only.

\par \textbf{THEAETETUS}
\par   Yes.

\par \textbf{SOCRATES}
\par   The definition was assumed to be the interpretation of your difference.

\par \textbf{THEAETETUS}
\par   True.

\par \textbf{SOCRATES}
\par   But when I had only opinion, I had no conception of your distinguishing characteristics.

\par \textbf{THEAETETUS}
\par   I suppose not.

\par \textbf{SOCRATES}
\par   Then I must have conceived of some general or common nature which no more belonged to you than to another.

\par \textbf{THEAETETUS}
\par   True.

\par \textbf{SOCRATES}
\par   Tell me, now—How in that case could I have formed a judgment of you any more than of any one else? Suppose that I imagine Theaetetus to be a man who has nose, eyes, and mouth, and every other member complete; how would that enable me to distinguish Theaetetus from Theodorus, or from some outer barbarian?

\par \textbf{THEAETETUS}
\par   How could it?

\par \textbf{SOCRATES}
\par   Or if I had further conceived of you, not only as having nose and eyes, but as having a snub nose and prominent eyes, should I have any more notion of you than of myself and others who resemble me?

\par \textbf{THEAETETUS}
\par   Certainly not.

\par \textbf{SOCRATES}
\par   Surely I can have no conception of Theaetetus until your snub-nosedness has left an impression on my mind different from the snub-nosedness of all others whom I have ever seen, and until your other peculiarities have a like distinctness; and so when I meet you to-morrow the right opinion will be re-called?

\par \textbf{THEAETETUS}
\par   Most true.

\par \textbf{SOCRATES}
\par   Then right opinion implies the perception of differences?

\par \textbf{THEAETETUS}
\par   Clearly.

\par \textbf{SOCRATES}
\par   What, then, shall we say of adding reason or explanation to right opinion? If the meaning is, that we should form an opinion of the way in which something differs from another thing, the proposal is ridiculous.

\par \textbf{THEAETETUS}
\par   How so?

\par \textbf{SOCRATES}
\par   We are supposed to acquire a right opinion of the differences which distinguish one thing from another when we have already a right opinion of them, and so we go round and round: —the revolution of the scytal, or pestle, or any other rotatory machine, in the same circles, is as nothing compared with such a requirement; and we may be truly described as the blind directing the blind; for to add those things which we already have, in order that we may learn what we already think, is like a soul utterly benighted.

\par \textbf{THEAETETUS}
\par   Tell me; what were you going to say just now, when you asked the question?

\par \textbf{SOCRATES}
\par   If, my boy, the argument, in speaking of adding the definition, had used the word to 'know,' and not merely 'have an opinion' of the difference, this which is the most promising of all the definitions of knowledge would have come to a pretty end, for to know is surely to acquire knowledge.

\par \textbf{THEAETETUS}
\par   True.

\par \textbf{SOCRATES}
\par   And so, when the question is asked, What is knowledge? this fair argument will answer 'Right opinion with knowledge,'—knowledge, that is, of difference, for this, as the said argument maintains, is adding the definition.

\par \textbf{THEAETETUS}
\par   That seems to be true.

\par \textbf{SOCRATES}
\par   But how utterly foolish, when we are asking what is knowledge, that the reply should only be, right opinion with knowledge of difference or of anything! And so, Theaetetus, knowledge is neither sensation nor true opinion, nor yet definition and explanation accompanying and added to true opinion?

\par \textbf{THEAETETUS}
\par   I suppose not.

\par \textbf{SOCRATES}
\par   And are you still in labour and travail, my dear friend, or have you brought all that you have to say about knowledge to the birth?

\par \textbf{THEAETETUS}
\par   I am sure, Socrates, that you have elicited from me a good deal more than ever was in me.

\par \textbf{SOCRATES}
\par   And does not my art show that you have brought forth wind, and that the offspring of your brain are not worth bringing up?

\par \textbf{THEAETETUS}
\par   Very true.

\par \textbf{SOCRATES}
\par   But if, Theaetetus, you should ever conceive afresh, you will be all the better for the present investigation, and if not, you will be soberer and humbler and gentler to other men, and will be too modest to fancy that you know what you do not know. These are the limits of my art; I can no further go, nor do I know aught of the things which great and famous men know or have known in this or former ages. The office of a midwife I, like my mother, have received from God; she delivered women, I deliver men; but they must be young and noble and fair.

\par  And now I have to go to the porch of the King Archon, where I am to meet Meletus and his indictment. To-morrow morning, Theodorus, I shall hope to see you again at this place.

\par 
 
\end{document}