
\documentclass[11pt,letter]{article}


\begin{document}

\title{Charmides\thanks{Source: https://www.gutenberg.org/files/1580/1580-h/1580-h.htm. License: http://gutenberg.org/license ds}}
\date{\today}
\author{Plato, 427? BCE-347? BCE\\ Translated by Jowett, Benjamin, 1817-1893}
\maketitle

\setcounter{tocdepth}{1}
\tableofcontents
\renewcommand{\baselinestretch}{1.0}
\normalsize
\newpage

\section{
      PREFACE TO THE FIRST EDITION.
    }
\par  The Text which has been mostly followed in this Translation of Plato is the latest 8vo. edition of Stallbaum; the principal deviations are noted at the bottom of the page.

\par  I have to acknowledge many obligations to old friends and pupils. These are:—Mr. John Purves, Fellow of Balliol College, with whom I have revised about half of the entire Translation; the Rev. Professor Campbell, of St. Andrews, who has helped me in the revision of several parts of the work, especially of the Theaetetus, Sophist, and Politicus; Mr. Robinson Ellis, Fellow of Trinity College, and Mr. Alfred Robinson, Fellow of New College, who read with me the Cratylus and the Gorgias; Mr. Paravicini, Student of Christ Church, who assisted me in the Symposium; Mr. Raper, Fellow of Queen's College, Mr. Monro, Fellow of Oriel College, and Mr. Shadwell, Student of Christ Church, who gave me similar assistance in the Laws. Dr. Greenhill, of Hastings, has also kindly sent me remarks on the physiological part of the Timaeus, which I have inserted as corrections under the head of errata at the end of the Introduction. The degree of accuracy which I have been enabled to attain is in great measure due to these gentlemen, and I heartily thank them for the pains and time which they have bestowed on my work.

\par  I have further to explain how far I have received help from other labourers in the same field. The books which I have found of most use are Steinhart and Muller's German Translation of Plato with Introductions; Zeller's 'Philosophie der Griechen,' and 'Platonische Studien;' Susemihl's 'Genetische Entwickelung der Paltonischen Philosophie;' Hermann's 'Geschichte der Platonischen Philosophie;' Bonitz, 'Platonische Studien;' Stallbaum's Notes and Introductions; Professor Campbell's editions of the 'Theaetetus,' the 'Sophist,' and the 'Politicus;' Professor Thompson's 'Phaedrus;' Th. Martin's 'Etudes sur le Timee;' Mr. Poste's edition and translation of the 'Philebus;' the Translation of the 'Republic,' by Messrs. Davies and Vaughan, and the Translation of the 'Gorgias,' by Mr. Cope.

\par  I have also derived much assistance from the great work of Mr. Grote, which contains excellent analyses of the Dialogues, and is rich in original thoughts and observations. I agree with him in rejecting as futile the attempt of Schleiermacher and others to arrange the Dialogues of Plato into a harmonious whole. Any such arrangement appears to me not only to be unsupported by evidence, but to involve an anachronism in the history of philosophy. There is a common spirit in the writings of Plato, but not a unity of design in the whole, nor perhaps a perfect unity in any single Dialogue. The hypothesis of a general plan which is worked out in the successive Dialogues is an after-thought of the critics who have attributed a system to writings belonging to an age when system had not as yet taken possession of philosophy.

\par  If Mr. Grote should do me the honour to read any portion of this work he will probably remark that I have endeavoured to approach Plato from a point of view which is opposed to his own. The aim of the Introductions in these volumes has been to represent Plato as the father of Idealism, who is not to be measured by the standard of utilitarianism or any other modern philosophical system. He is the poet or maker of ideas, satisfying the wants of his own age, providing the instruments of thought for future generations. He is no dreamer, but a great philosophical genius struggling with the unequal conditions of light and knowledge under which he is living. He may be illustrated by the writings of moderns, but he must be interpreted by his own, and by his place in the history of philosophy. We are not concerned to determine what is the residuum of truth which remains for ourselves. His truth may not be our truth, and nevertheless may have an extraordinary value and interest for us.

\par  I cannot agree with Mr. Grote in admitting as genuine all the writings commonly attributed to Plato in antiquity, any more than with Schaarschmidt and some other German critics who reject nearly half of them. The German critics, to whom I refer, proceed chiefly on grounds of internal evidence; they appear to me to lay too much stress on the variety of doctrine and style, which must be equally acknowledged as a fact, even in the Dialogues regarded by Schaarschmidt as genuine, e.g. in the Phaedrus, or Symposium, when compared with the Laws. He who admits works so different in style and matter to have been the composition of the same author, need have no difficulty in admitting the Sophist or the Politicus. (The negative argument adduced by the same school of critics, which is based on the silence of Aristotle, is not worthy of much consideration. For why should Aristotle, because he has quoted several Dialogues of Plato, have quoted them all? Something must be allowed to chance, and to the nature of the subjects treated of in them.) On the other hand, Mr. Grote trusts mainly to the Alexandrian Canon. But I hardly think that we are justified in attributing much weight to the authority of the Alexandrian librarians in an age when there was no regular publication of books, and every temptation to forge them; and in which the writings of a school were naturally attributed to the founder of the school. And even without intentional fraud, there was an inclination to believe rather than to enquire. Would Mr. Grote accept as genuine all the writings which he finds in the lists of learned ancients attributed to Hippocrates, to Xenophon, to Aristotle? The Alexandrian Canon of the Platonic writings is deprived of credit by the admission of the Epistles, which are not only unworthy of Plato, and in several passages plagiarized from him, but flagrantly at variance with historical fact. It will be seen also that I do not agree with Mr. Grote's views about the Sophists; nor with the low estimate which he has formed of Plato's Laws; nor with his opinion respecting Plato's doctrine of the rotation of the earth. But I 'am not going to lay hands on my father Parmenides' (Soph. ), who will, I hope, forgive me for differing from him on these points. I cannot close this Preface without expressing my deep respect for his noble and gentle character, and the great services which he has rendered to Greek Literature.

\par  Balliol College, January, 1871.

\par 
\section{
      PREFACE TO THE SECOND AND THIRD EDITIONS.
    }
\par  In publishing a Second Edition (1875) of the Dialogues of Plato in English, I had to acknowledge the assistance of several friends: of the Rev. G.G. Bradley, Master of University College, now Dean of Westminster, who sent me some valuable remarks on the Phaedo; of Dr. Greenhill, who had again revised a portion of the Timaeus; of Mr. R.L. Nettleship, Fellow and Tutor of Balliol College, to whom I was indebted for an excellent criticism of the Parmenides; and, above all, of the Rev. Professor Campbell of St. Andrews, and Mr. Paravicini, late Student of Christ Church and Tutor of Balliol College, with whom I had read over the greater part of the translation. I was also indebted to Mr. Evelyn Abbott, Fellow and Tutor of Balliol College, for a complete and accurate index.

\par  In this, the Third Edition, I am under very great obligations to Mr. Matthew Knight, who has not only favoured me with valuable suggestions throughout the work, but has largely extended the Index (from 61 to 175 pages) and translated the Eryxias and Second Alcibiades; and to Mr Frank Fletcher, of Balliol College, my Secretary. I am also considerably indebted to Mr. J.W. Mackail, late Fellow of Balliol College, who read over the Republic in the Second Edition and noted several inaccuracies.

\par  In both editions the Introductions to the Dialogues have been enlarged, and essays on subjects having an affinity to the Platonic Dialogues have been introduced into several of them. The analyses have been corrected, and innumerable alterations have been made in the Text. There have been added also, in the Third Edition, headings to the pages and a marginal analysis to the text of each dialogue.

\par  At the end of a long task, the translator may without impropriety point out the difficulties which he has had to encounter. These have been far greater than he would have anticipated; nor is he at all sanguine that he has succeeded in overcoming them. Experience has made him feel that a translation, like a picture, is dependent for its effect on very minute touches; and that it is a work of infinite pains, to be returned to in many moods and viewed in different lights.

\par  I. An English translation ought to be idiomatic and interesting, not only to the scholar, but to the unlearned reader. Its object should not simply be to render the words of one language into the words of another or to preserve the construction and order of the original;—this is the ambition of a schoolboy, who wishes to show that he has made a good use of his Dictionary and Grammar; but is quite unworthy of the translator, who seeks to produce on his reader an impression similar or nearly similar to that produced by the original. To him the feeling should be more important than the exact word. He should remember Dryden's quaint admonition not to 'lacquey by the side of his author, but to mount up behind him.' (Dedication to the Aeneis.) He must carry in his mind a comprehensive view of the whole work, of what has preceded and of what is to follow,—as well as of the meaning of particular passages. His version should be based, in the first instance, on an intimate knowledge of the text; but the precise order and arrangement of the words may be left to fade out of sight, when the translation begins to take shape. He must form a general idea of the two languages, and reduce the one to the terms of the other. His work should be rhythmical and varied, the right admixture of words and syllables, and even of letters, should be carefully attended to; above all, it should be equable in style. There must also be quantity, which is necessary in prose as well as in verse: clauses, sentences, paragraphs, must be in due proportion. Metre and even rhyme may be rarely admitted; though neither is a legitimate element of prose writing, they may help to lighten a cumbrous expression (Symp.). The translation should retain as far as possible the characteristic qualities of the ancient writer—his freedom, grace, simplicity, stateliness, weight, precision; or the best part of him will be lost to the English reader. It should read as an original work, and should also be the most faithful transcript which can be made of the language from which the translation is taken, consistently with the first requirement of all, that it be English. Further, the translation being English, it should also be perfectly intelligible in itself without reference to the Greek, the English being really the more lucid and exact of the two languages. In some respects it may be maintained that ordinary English writing, such as the newspaper article, is superior to Plato: at any rate it is couched in language which is very rarely obscure. On the other hand, the greatest writers of Greece, Thucydides, Plato, Aeschylus, Sophocles, Pindar, Demosthenes, are generally those which are found to be most difficult and to diverge most widely from the English idiom. The translator will often have to convert the more abstract Greek into the more concrete English, or vice versa, and he ought not to force upon one language the character of another. In some cases, where the order is confused, the expression feeble, the emphasis misplaced, or the sense somewhat faulty, he will not strive in his rendering to reproduce these characteristics, but will re-write the passage as his author would have written it at first, had he not been 'nodding'; and he will not hesitate to supply anything which, owing to the genius of the language or some accident of composition, is omitted in the Greek, but is necessary to make the English clear and consecutive.

\par  It is difficult to harmonize all these conflicting elements. In a translation of Plato what may be termed the interests of the Greek and English are often at war with one another. In framing the English sentence we are insensibly diverted from the exact meaning of the Greek; when we return to the Greek we are apt to cramp and overlay the English. We substitute, we compromise, we give and take, we add a little here and leave out a little there. The translator may sometimes be allowed to sacrifice minute accuracy for the sake of clearness and sense. But he is not therefore at liberty to omit words and turns of expression which the English language is quite capable of supplying. He must be patient and self-controlled; he must not be easily run away with. Let him never allow the attraction of a favourite expression, or a sonorous cadence, to overpower his better judgment, or think much of an ornament which is out of keeping with the general character of his work. He must ever be casting his eyes upwards from the copy to the original, and down again from the original to the copy (Rep.). His calling is not held in much honour by the world of scholars; yet he himself may be excused for thinking it a kind of glory to have lived so many years in the companionship of one of the greatest of human intelligences, and in some degree, more perhaps than others, to have had the privilege of understanding him (Sir Joshua Reynolds' Lectures: Disc. xv. ).

\par  There are fundamental differences in Greek and English, of which some may be managed while others remain intractable. (1). The structure of the Greek language is partly adversative and alternative, and partly inferential; that is to say, the members of a sentence are either opposed to one another, or one of them expresses the cause or effect or condition or reason of another. The two tendencies may be called the horizontal and perpendicular lines of the language; and the opposition or inference is often much more one of words than of ideas. But modern languages have rubbed off this adversative and inferential form: they have fewer links of connection, there is less mortar in the interstices, and they are content to place sentences side by side, leaving their relation to one another to be gathered from their position or from the context. The difficulty of preserving the effect of the Greek is increased by the want of adversative and inferential particles in English, and by the nice sense of tautology which characterizes all modern languages. We cannot have two 'buts' or two 'fors' in the same sentence where the Greek repeats (Greek). There is a similar want of particles expressing the various gradations of objective and subjective thought—(Greek) and the like, which are so thickly scattered over the Greek page. Further, we can only realize to a very imperfect degree the common distinction between (Greek), and the combination of the two suggests a subtle shade of negation which cannot be expressed in English. And while English is more dependent than Greek upon the apposition of clauses and sentences, yet there is a difficulty in using this form of construction owing to the want of case endings. For the same reason there cannot be an equal variety in the order of words or an equal nicety of emphasis in English as in Greek.

\par  (2) The formation of the sentence and of the paragraph greatly differs in Greek and English. The lines by which they are divided are generally much more marked in modern languages than in ancient. Both sentences and paragraphs are more precise and definite—they do not run into one another. They are also more regularly developed from within. The sentence marks another step in an argument or a narrative or a statement; in reading a paragraph we silently turn over the page and arrive at some new view or aspect of the subject. Whereas in Plato we are not always certain where a sentence begins and ends; and paragraphs are few and far between. The language is distributed in a different way, and less articulated than in English. For it was long before the true use of the period was attained by the classical writers both in poetry or prose; it was (Greek). The balance of sentences and the introduction of paragraphs at suitable intervals must not be neglected if the harmony of the English language is to be preserved. And still a caution has to be added on the other side, that we must avoid giving it a numerical or mechanical character.

\par  (3) This, however, is not one of the greatest difficulties of the translator; much greater is that which arises from the restriction of the use of the genders. Men and women in English are masculine and feminine, and there is a similar distinction of sex in the words denoting animals; but all things else, whether outward objects or abstract ideas, are relegated to the class of neuters. Hardly in some flight of poetry do we ever endue any of them with the characteristics of a sentient being, and then only by speaking of them in the feminine gender. The virtues may be pictured in female forms, but they are not so described in language; a ship is humorously supposed to be the sailor's bride; more doubtful are the personifications of church and country as females. Now the genius of the Greek language is the opposite of this. The same tendency to personification which is seen in the Greek mythology is common also in the language; and genders are attributed to things as well as persons according to their various degrees of strength and weakness; or from fanciful resemblances to the male or female form, or some analogy too subtle to be discovered. When the gender of any object was once fixed, a similar gender was naturally assigned to similar objects, or to words of similar formation. This use of genders in the denotation of objects or ideas not only affects the words to which genders are attributed, but the words with which they are construed or connected, and passes into the general character of the style. Hence arises a difficulty in translating Greek into English which cannot altogether be overcome. Shall we speak of the soul and its qualities, of virtue, power, wisdom, and the like, as feminine or neuter? The usage of the English language does not admit of the former, and yet the life and beauty of the style are impaired by the latter. Often the translator will have recourse to the repetition of the word, or to the ambiguous 'they,' 'their,' etc. ; for fear of spoiling the effect of the sentence by introducing 'it.' Collective nouns in Greek and English create a similar but lesser awkwardness.

\par  (4) To use of relation is far more extended in Greek than in English. Partly the greater variety of genders and cases makes the connexion of relative and antecedent less ambiguous: partly also the greater number of demonstrative and relative pronouns, and the use of the article, make the correlation of ideas simpler and more natural. The Greek appears to have had an ear or intelligence for a long and complicated sentence which is rarely to be found in modern nations; and in order to bring the Greek down to the level of the modern, we must break up the long sentence into two or more short ones. Neither is the same precision required in Greek as in Latin or English, nor in earlier Greek as in later; there was nothing shocking to the contemporary of Thucydides and Plato in anacolutha and repetitions. In such cases the genius of the English language requires that the translation should be more intelligible than the Greek. The want of more distinctions between the demonstrative pronouns is also greatly felt. Two genitives dependent on one another, unless familiarised by idiom, have an awkward effect in English. Frequently the noun has to take the place of the pronoun. 'This' and 'that' are found repeating themselves to weariness in the rough draft of a translation. As in the previous case, while the feeling of the modern language is more opposed to tautology, there is also a greater difficulty in avoiding it.

\par  (5) Though no precise rule can be laid down about the repetition of words, there seems to be a kind of impertinence in presenting to the reader the same thought in the same words, repeated twice over in the same passage without any new aspect or modification of it. And the evasion of tautology—that is, the substitution of one word of precisely the same meaning for another—is resented by us equally with the repetition of words. Yet on the other hand the least difference of meaning or the least change of form from a substantive to an adjective, or from a participle to a verb, will often remedy the unpleasant effect. Rarely and only for the sake of emphasis or clearness can we allow an important word to be used twice over in two successive sentences or even in the same paragraph. The particles and pronouns, as they are of most frequent occurrence, are also the most troublesome. Strictly speaking, except a few of the commonest of them, 'and,' 'the,' etc., they ought not to occur twice in the same sentence. But the Greek has no such precise rules; and hence any literal translation of a Greek author is full of tautology. The tendency of modern languages is to become more correct as well as more perspicuous than ancient. And, therefore, while the English translator is limited in the power of expressing relation or connexion, by the law of his own language increased precision and also increased clearness are required of him. The familiar use of logic, and the progress of science, have in these two respects raised the standard. But modern languages, while they have become more exacting in their demands, are in many ways not so well furnished with powers of expression as the ancient classical ones.

\par  Such are a few of the difficulties which have to be overcome in the work of translation; and we are far from having exhausted the list. (6) The excellence of a translation will consist, not merely in the faithful rendering of words, or in the composition of a sentence only, or yet of a single paragraph, but in the colour and style of the whole work. Equability of tone is best attained by the exclusive use of familiar and idiomatic words. But great care must be taken; for an idiomatic phrase, if an exception to the general style, is of itself a disturbing element. No word, however expressive and exact, should be employed, which makes the reader stop to think, or unduly attracts attention by difficulty and peculiarity, or disturbs the effect of the surrounding language. In general the style of one author is not appropriate to another; as in society, so in letters, we expect every man to have 'a good coat of his own,' and not to dress himself out in the rags of another. (a) Archaic expressions are therefore to be avoided. Equivalents may be occasionally drawn from Shakspere, who is the common property of us all; but they must be used sparingly. For, like some other men of genius of the Elizabethan and Jacobean age, he outdid the capabilities of the language, and many of the expressions which he introduced have been laid aside and have dropped out of use. (b) A similar principle should be observed in the employment of Scripture. Having a greater force and beauty than other language, and a religious association, it disturbs the even flow of the style. It may be used to reproduce in the translation the quaint effect of some antique phrase in the original, but rarely; and when adopted, it should have a certain freshness and a suitable 'entourage.' It is strange to observe that the most effective use of Scripture phraseology arises out of the application of it in a sense not intended by the author. (c) Another caution: metaphors differ in different languages, and the translator will often be compelled to substitute one for another, or to paraphrase them, not giving word for word, but diffusing over several words the more concentrated thought of the original. The Greek of Plato often goes beyond the English in its imagery: compare Laws, (Greek); Rep.; etc. Or again the modern word, which in substance is the nearest equivalent to the Greek, may be found to include associations alien to Greek life: e.g. (Greek), 'jurymen,' (Greek), 'the bourgeoisie.' (d) The translator has also to provide expressions for philosophical terms of very indefinite meaning in the more definite language of modern philosophy. And he must not allow discordant elements to enter into the work. For example, in translating Plato, it would equally be an anachronism to intrude on him the feeling and spirit of the Jewish or Christian Scriptures or the technical terms of the Hegelian or Darwinian philosophy.

\par  (7) As no two words are precise equivalents (just as no two leaves of the forest are exactly similar), it is a mistaken attempt at precision always to translate the same Greek word by the same English word. There is no reason why in the New Testament (Greek) should always be rendered 'righteousness,' or (Greek) 'covenant.' In such cases the translator may be allowed to employ two words—sometimes when the two meanings occur in the same passage, varying them by an 'or'—e.g. (Greek), 'science' or 'knowledge,' (Greek), 'idea' or 'class,' (Greek), 'temperance' or 'prudence,'—at the point where the change of meaning occurs. If translations are intended not for the Greek scholar but for the general reader, their worst fault will be that they sacrifice the general effect and meaning to the over-precise rendering of words and forms of speech.

\par  (8) There is no kind of literature in English which corresponds to the Greek Dialogue; nor is the English language easily adapted to it. The rapidity and abruptness of question and answer, the constant repetition of (Greek), etc., which Cicero avoided in Latin (de Amicit), the frequent occurrence of expletives, would, if reproduced in a translation, give offence to the reader. Greek has a freer and more frequent use of the Interrogative, and is of a more passionate and emotional character, and therefore lends itself with greater readiness to the dialogue form. Most of the so-called English Dialogues are but poor imitations of Plato, which fall very far short of the original. The breath of conversation, the subtle adjustment of question and answer, the lively play of fancy, the power of drawing characters, are wanting in them. But the Platonic dialogue is a drama as well as a dialogue, of which Socrates is the central figure, and there are lesser performers as well:—the insolence of Thrasymachus, the anger of Callicles and Anytus, the patronizing style of Protagoras, the self-consciousness of Prodicus and Hippias, are all part of the entertainment. To reproduce this living image the same sort of effort is required as in translating poetry. The language, too, is of a finer quality; the mere prose English is slow in lending itself to the form of question and answer, and so the ease of conversation is lost, and at the same time the dialectical precision with which the steps of the argument are drawn out is apt to be impaired.

\par  II. In the Introductions to the Dialogues there have been added some essays on modern philosophy, and on political and social life. The chief subjects discussed in these are Utility, Communism, the Kantian and Hegelian philosophies, Psychology, and the Origin of Language. (There have been added also in the Third Edition remarks on other subjects. A list of the most important of these additions is given at the end of this Preface.)

\par  Ancient and modern philosophy throw a light upon one another: but they should be compared, not confounded. Although the connexion between them is sometimes accidental, it is often real. The same questions are discussed by them under different conditions of language and civilization; but in some cases a mere word has survived, while nothing or hardly anything of the pre-Socratic, Platonic, or Aristotelian meaning is retained. There are other questions familiar to the moderns, which have no place in ancient philosophy. The world has grown older in two thousand years, and has enlarged its stock of ideas and methods of reasoning. Yet the germ of modern thought is found in ancient, and we may claim to have inherited, notwithstanding many accidents of time and place, the spirit of Greek philosophy. There is, however, no continuous growth of the one into the other, but a new beginning, partly artificial, partly arising out of the questionings of the mind itself, and also receiving a stimulus from the study of ancient writings.

\par  Considering the great and fundamental differences which exist in ancient and modern philosophy, it seems best that we should at first study them separately, and seek for the interpretation of either, especially of the ancient, from itself only, comparing the same author with himself and with his contemporaries, and with the general state of thought and feeling prevalent in his age. Afterwards comes the remoter light which they cast on one another. We begin to feel that the ancients had the same thoughts as ourselves, the same difficulties which characterize all periods of transition, almost the same opposition between science and religion. Although we cannot maintain that ancient and modern philosophy are one and continuous (as has been affirmed with more truth respecting ancient and modern history), for they are separated by an interval of a thousand years, yet they seem to recur in a sort of cycle, and we are surprised to find that the new is ever old, and that the teaching of the past has still a meaning for us.

\par  III. In the preface to the first edition I expressed a strong opinion at variance with Mr. Grote's, that the so-called Epistles of Plato were spurious. His friend and editor, Professor Bain, thinks that I ought to give the reasons why I differ from so eminent an authority. Reserving the fuller discussion of the question for another place, I will shortly defend my opinion by the following arguments:—

\par  (a) Because almost all epistles purporting to be of the classical age of Greek literature are forgeries. (Compare Bentley's Works (Dyce's Edition).) Of all documents this class are the least likely to be preserved and the most likely to be invented. The ancient world swarmed with them; the great libraries stimulated the demand for them; and at a time when there was no regular publication of books, they easily crept into the world.

\par  (b) When one epistle out of a number is spurious, the remainder of the series cannot be admitted to be genuine, unless there be some independent ground for thinking them so: when all but one are spurious, overwhelming evidence is required of the genuineness of the one: when they are all similar in style or motive, like witnesses who agree in the same tale, they stand or fall together. But no one, not even Mr. Grote, would maintain that all the Epistles of Plato are genuine, and very few critics think that more than one of them is so. And they are clearly all written from the same motive, whether serious or only literary. Nor is there an example in Greek antiquity of a series of Epistles, continuous and yet coinciding with a succession of events extending over a great number of years.

\par  The external probability therefore against them is enormous, and the internal probability is not less: for they are trivial and unmeaning, devoid of delicacy and subtlety, wanting in a single fine expression. And even if this be matter of dispute, there can be no dispute that there are found in them many plagiarisms, inappropriately borrowed, which is a common note of forgery. They imitate Plato, who never imitates either himself or any one else; reminiscences of the Republic and the Laws are continually recurring in them; they are too like him and also too unlike him, to be genuine (see especially Karsten, Commentio Critica de Platonis quae feruntur Epistolis). They are full of egotism, self-assertion, affectation, faults which of all writers Plato was most careful to avoid, and into which he was least likely to fall. They abound in obscurities, irrelevancies, solecisms, pleonasms, inconsistencies, awkwardnesses of construction, wrong uses of words. They also contain historical blunders, such as the statement respecting Hipparinus and Nysaeus, the nephews of Dion, who are said to 'have been well inclined to philosophy, and well able to dispose the mind of their brother Dionysius in the same course,' at a time when they could not have been more than six or seven years of age—also foolish allusions, such as the comparison of the Athenian empire to the empire of Darius, which show a spirit very different from that of Plato; and mistakes of fact, as e.g. about the Thirty Tyrants, whom the writer of the letters seems to have confused with certain inferior magistrates, making them in all fifty-one. These palpable errors and absurdities are absolutely irreconcilable with their genuineness. And as they appear to have a common parentage, the more they are studied, the more they will be found to furnish evidence against themselves. The Seventh, which is thought to be the most important of these Epistles, has affinities with the Third and the Eighth, and is quite as impossible and inconsistent as the rest. It is therefore involved in the same condemnation.—The final conclusion is that neither the Seventh nor any other of them, when carefully analyzed, can be imagined to have proceeded from the hand or mind of Plato. The other testimonies to the voyages of Plato to Sicily and the court of Dionysius are all of them later by several centuries than the events to which they refer. No extant writer mentions them older than Cicero and Cornelius Nepos. It does not seem impossible that so attractive a theme as the meeting of a philosopher and a tyrant, once imagined by the genius of a Sophist, may have passed into a romance which became famous in Hellas and the world. It may have created one of the mists of history, like the Trojan war or the legend of Arthur, which we are unable to penetrate. In the age of Cicero, and still more in that of Diogenes Laertius and Appuleius, many other legends had gathered around the personality of Plato,—more voyages, more journeys to visit tyrants and Pythagorean philosophers. But if, as we agree with Karsten in supposing, they are the forgery of some rhetorician or sophist, we cannot agree with him in also supposing that they are of any historical value, the rather as there is no early independent testimony by which they are supported or with which they can be compared.

\par  IV. There is another subject to which I must briefly call attention, lest I should seem to have overlooked it. Dr. Henry Jackson, of Trinity College, Cambridge, in a series of articles which he has contributed to the Journal of Philology, has put forward an entirely new explanation of the Platonic 'Ideas.' He supposes that in the mind of Plato they took, at different times in his life, two essentially different forms:—an earlier one which is found chiefly in the Republic and the Phaedo, and a later, which appears in the Theaetetus, Philebus, Sophist, Politicus, Parmenides, Timaeus. In the first stage of his philosophy Plato attributed Ideas to all things, at any rate to all things which have classes or common notions: these he supposed to exist only by participation in them. In the later Dialogues he no longer included in them manufactured articles and ideas of relation, but restricted them to 'types of nature,' and having become convinced that the many cannot be parts of the one, for the idea of participation in them he substituted imitation of them. To quote Dr. Jackson's own expressions,—'whereas in the period of the Republic and the Phaedo, it was proposed to pass through ontology to the sciences, in the period of the Parmenides and the Philebus, it is proposed to pass through the sciences to ontology': or, as he repeats in nearly the same words,—'whereas in the Republic and in the Phaedo he had dreamt of passing through ontology to the sciences, he is now content to pass through the sciences to ontology.'

\par  This theory is supposed to be based on Aristotle's Metaphysics, a passage containing an account of the ideas, which hitherto scholars have found impossible to reconcile with the statements of Plato himself. The preparations for the new departure are discovered in the Parmenides and in the Theaetetus; and it is said to be expressed under a different form by the (Greek) and the (Greek) of the Philebus. The (Greek) of the Philebus is the principle which gives form and measure to the (Greek); and in the 'Later Theory' is held to be the (Greek) or (Greek) which converts the Infinite or Indefinite into ideas. They are neither (Greek) nor (Greek), but belong to the (Greek) which partakes of both.

\par  With great respect for the learning and ability of Dr. Jackson, I find myself unable to agree in this newly fashioned doctrine of the Ideas, which he ascribes to Plato. I have not the space to go into the question fully; but I will briefly state some objections which are, I think, fatal to it.

\par  (1) First, the foundation of his argument is laid in the Metaphysics of Aristotle. But we cannot argue, either from the Metaphysics, or from any other of the philosophical treatises of Aristotle, to the dialogues of Plato until we have ascertained the relation in which his so-called works stand to the philosopher himself. There is of course no doubt of the great influence exercised upon Greece and upon the world by Aristotle and his philosophy. But on the other hand almost every one who is capable of understanding the subject acknowledges that his writings have not come down to us in an authentic form like most of the dialogues of Plato. How much of them is to be ascribed to Aristotle's own hand, how much is due to his successors in the Peripatetic School, is a question which has never been determined, and probably never can be, because the solution of it depends upon internal evidence only. To 'the height of this great argument' I do not propose to ascend. But one little fact, not irrelevant to the present discussion, will show how hopeless is the attempt to explain Plato out of the writings of Aristotle. In the chapter of the Metaphysics quoted by Dr. Jackson, about two octavo pages in length, there occur no less than seven or eight references to Plato, although nothing really corresponding to them can be found in his extant writings:—a small matter truly; but what a light does it throw on the character of the entire book in which they occur! We can hardly escape from the conclusion that they are not statements of Aristotle respecting Plato, but of a later generation of Aristotelians respecting a later generation of Platonists. (Compare the striking remark of the great Scaliger respecting the Magna Moralia:—Haec non sunt Aristotelis, tamen utitur auctor Aristotelis nomine tanquam suo.)

\par  (2) There is no hint in Plato's own writings that he was conscious of having made any change in the Doctrine of Ideas such as Dr. Jackson attributes to him, although in the Republic the platonic Socrates speaks of 'a longer and a shorter way', and of a way in which his disciple Glaucon 'will be unable to follow him'; also of a way of Ideas, to which he still holds fast, although it has often deserted him (Philebus, Phaedo), and although in the later dialogues and in the Laws the reference to Ideas disappears, and Mind claims her own (Phil. ; Laws). No hint is given of what Plato meant by the 'longer way' (Rep.), or 'the way in which Glaucon was unable to follow'; or of the relation of Mind to the Ideas. It might be said with truth that the conception of the Idea predominates in the first half of the Dialogues, which, according to the order adopted in this work, ends with the Republic, the 'conception of Mind' and a way of speaking more in agreement with modern terminology, in the latter half. But there is no reason to suppose that Plato's theory, or, rather, his various theories, of the Ideas underwent any definite change during his period of authorship. They are substantially the same in the twelfth Book of the Laws as in the Meno and Phaedo; and since the Laws were written in the last decade of his life, there is no time to which this change of opinions can be ascribed. It is true that the theory of Ideas takes several different forms, not merely an earlier and a later one, in the various Dialogues. They are personal and impersonal, ideals and ideas, existing by participation or by imitation, one and many, in different parts of his writings or even in the same passage. They are the universal definitions of Socrates, and at the same time 'of more than mortal knowledge' (Rep.). But they are always the negations of sense, of matter, of generation, of the particular: they are always the subjects of knowledge and not of opinion; and they tend, not to diversity, but to unity. Other entities or intelligences are akin to them, but not the same with them, such as mind, measure, limit, eternity, essence (Philebus; Timaeus): these and similar terms appear to express the same truths from a different point of view, and to belong to the same sphere with them. But we are not justified, therefore, in attempting to identify them, any more than in wholly opposing them. The great oppositions of the sensible and intellectual, the unchangeable and the transient, in whatever form of words expressed, are always maintained in Plato. But the lesser logical distinctions, as we should call them, whether of ontology or predication, which troubled the pre-Socratic philosophy and came to the front in Aristotle, are variously discussed and explained. Thus far we admit inconsistency in Plato, but no further. He lived in an age before logic and system had wholly permeated language, and therefore we must not always expect to find in him systematic arrangement or logical precision:—'poema magis putandum.' But he is always true to his own context, the careful study of which is of more value to the interpreter than all the commentators and scholiasts put together.

\par  (3) The conclusions at which Dr. Jackson has arrived are such as might be expected to follow from his method of procedure. For he takes words without regard to their connection, and pieces together different parts of dialogues in a purely arbitrary manner, although there is no indication that the author intended the two passages to be so combined, or that when he appears to be experimenting on the different points of view from which a subject of philosophy may be regarded, he is secretly elaborating a system. By such a use of language any premises may be made to lead to any conclusion. I am not one of those who believe Plato to have been a mystic or to have had hidden meanings; nor do I agree with Dr. Jackson in thinking that 'when he is precise and dogmatic, he generally contrives to introduce an element of obscurity into the expostion' (J. of Philol.). The great master of language wrote as clearly as he could in an age when the minds of men were clouded by controversy, and philosophical terms had not yet acquired a fixed meaning. I have just said that Plato is to be interpreted by his context; and I do not deny that in some passages, especially in the Republic and Laws, the context is at a greater distance than would be allowable in a modern writer. But we are not therefore justified in connecting passages from different parts of his writings, or even from the same work, which he has not himself joined. We cannot argue from the Parmenides to the Philebus, or from either to the Sophist, or assume that the Parmenides, the Philebus, and the Timaeus were 'written simultaneously,' or 'were intended to be studied in the order in which they are here named (J. of Philol.) We have no right to connect statements which are only accidentally similar. Nor is it safe for the author of a theory about ancient philosophy to argue from what will happen if his statements are rejected. For those consequences may never have entered into the mind of the ancient writer himself; and they are very likely to be modern consequences which would not have been understood by him. 'I cannot think,' says Dr. Jackson, 'that Plato would have changed his opinions, but have nowhere explained the nature of the change.' But is it not much more improbable that he should have changed his opinions, and not stated in an unmistakable manner that the most essential principle of his philosophy had been reversed? It is true that a few of the dialogues, such as the Republic and the Timaeus, or the Theaetetus and the Sophist, or the Meno and the Apology, contain allusions to one another. But these allusions are superficial and, except in the case of the Republic and the Laws, have no philosophical importance. They do not affect the substance of the work. It may be remarked further that several of the dialogues, such as the Phaedrus, the Sophist, and the Parmenides, have more than one subject. But it does not therefore follow that Plato intended one dialogue to succeed another, or that he begins anew in one dialogue a subject which he has left unfinished in another, or that even in the same dialogue he always intended the two parts to be connected with each other. We cannot argue from a casual statement found in the Parmenides to other statements which occur in the Philebus. Much more truly is his own manner described by himself when he says that 'words are more plastic than wax' (Rep.), and 'whither the wind blows, the argument follows'. The dialogues of Plato are like poems, isolated and separate works, except where they are indicated by the author himself to have an intentional sequence.

\par  It is this method of taking passages out of their context and placing them in a new connexion when they seem to confirm a preconceived theory, which is the defect of Dr. Jackson's procedure. It may be compared, though not wholly the same with it, to that method which the Fathers practised, sometimes called 'the mystical interpretation of Scripture,' in which isolated words are separated from their context, and receive any sense which the fancy of the interpreter may suggest. It is akin to the method employed by Schleiermacher of arranging the dialogues of Plato in chronological order according to what he deems the true arrangement of the ideas contained in them. (Dr. Jackson is also inclined, having constructed a theory, to make the chronology of Plato's writings dependent upon it (See J. of Philol. and elsewhere.) It may likewise be illustrated by the ingenuity of those who employ symbols to find in Shakespeare a hidden meaning. In the three cases the error is nearly the same:—words are taken out of their natural context, and thus become destitute of any real meaning.

\par  (4) According to Dr. Jackson's 'Later Theory,' Plato's Ideas, which were once regarded as the summa genera of all things, are now to be explained as Forms or Types of some things only,—that is to say, of natural objects: these we conceive imperfectly, but are always seeking in vain to have a more perfect notion of them. He says (J. of Philol.) that 'Plato hoped by the study of a series of hypothetical or provisional classifications to arrive at one in which nature's distribution of kinds is approximately represented, and so to attain approximately to the knowledge of the ideas. But whereas in the Republic, and even in the Phaedo, though less hopefully, he had sought to convert his provisional definitions into final ones by tracing their connexion with the summum genus, the (Greek), in the Parmenides his aspirations are less ambitious,' and so on. But where does Dr. Jackson find any such notion as this in Plato or anywhere in ancient philosophy? Is it not an anachronism, gracious to the modern physical philosopher, and the more acceptable because it seems to form a link between ancient and modern philosophy, and between physical and metaphysical science; but really unmeaning?

\par  (5) To this 'Later Theory' of Plato's Ideas I oppose the authority of Professor Zeller, who affirms that none of the passages to which Dr. Jackson appeals (Theaet. ; Phil. ; Tim. ; Parm.) 'in the smallest degree prove his point'; and that in the second class of dialogues, in which the 'Later Theory of Ideas' is supposed to be found, quite as clearly as in the first, are admitted Ideas, not only of natural objects, but of properties, relations, works of art, negative notions (Theaet. ; Parm. ; Soph. ); and that what Dr. Jackson distinguishes as the first class of dialogues from the second equally assert or imply that the relation of things to the Ideas, is one of participation in them as well as of imitation of them (Prof. Zeller's summary of his own review of Dr. Jackson, Archiv fur Geschichte der Philosophie.)

\par  In conclusion I may remark that in Plato's writings there is both unity, and also growth and development; but that we must not intrude upon him either a system or a technical language.

\par  Balliol College, October, 1891.

\par 
\section{
      NOTE
    }
\par  The chief additions to the Introductions in the Third Edition consist of Essays on the following subjects:—

\par  1. Language.

\par  2. The decline of Greek Literature.

\par  3. The 'Ideas' of Plato and Modern Philosophy.

\par  4. The myths of Plato.

\par  5. The relation of the Republic, Statesman and Laws.

\par  6. The legend of Atlantis.

\par  7. Psychology.

\par  8. Comparison of the Laws of Plato with Spartan and Athenian Laws and Institutions.

\par  CHARMIDES.
\section{
      INTRODUCTION.
    }
\par  The subject of the Charmides is Temperance or (Greek), a peculiarly Greek notion, which may also be rendered Moderation (Compare Cic. Tusc. '(Greek), quam soleo equidem tum temperantiam, tum moderationem appellare, nonnunquam etiam modestiam. '), Modesty, Discretion, Wisdom, without completely exhausting by all these terms the various associations of the word. It may be described as 'mens sana in corpore sano,' the harmony or due proportion of the higher and lower elements of human nature which 'makes a man his own master,' according to the definition of the Republic. In the accompanying translation the word has been rendered in different places either Temperance or Wisdom, as the connection seemed to require: for in the philosophy of Plato (Greek) still retains an intellectual element (as Socrates is also said to have identified (Greek) with (Greek): Xen. Mem.) and is not yet relegated to the sphere of moral virtue, as in the Nicomachean Ethics of Aristotle.

\par  The beautiful youth, Charmides, who is also the most temperate of human beings, is asked by Socrates, 'What is Temperance?' He answers characteristically, (1) 'Quietness.' 'But Temperance is a fine and noble thing; and quietness in many or most cases is not so fine a thing as quickness.' He tries again and says (2) that temperance is modesty. But this again is set aside by a sophistical application of Homer: for temperance is good as well as noble, and Homer has declared that 'modesty is not good for a needy man.' (3) Once more Charmides makes the attempt. This time he gives a definition which he has heard, and of which Socrates conjectures that Critias must be the author: 'Temperance is doing one's own business.' But the artisan who makes another man's shoes may be temperate, and yet he is not doing his own business; and temperance defined thus would be opposed to the division of labour which exists in every temperate or well-ordered state. How is this riddle to be explained?

\par  Critias, who takes the place of Charmides, distinguishes in his answer between 'making' and 'doing,' and with the help of a misapplied quotation from Hesiod assigns to the words 'doing' and 'work' an exclusively good sense: Temperance is doing one's own business;—(4) is doing good.

\par  Still an element of knowledge is wanting which Critias is readily induced to admit at the suggestion of Socrates; and, in the spirit of Socrates and of Greek life generally, proposes as a fifth definition, (5) Temperance is self-knowledge. But all sciences have a subject: number is the subject of arithmetic, health of medicine—what is the subject of temperance or wisdom? The answer is that (6) Temperance is the knowledge of what a man knows and of what he does not know. But this is contrary to analogy; there is no vision of vision, but only of visible things; no love of loves, but only of beautiful things; how then can there be a knowledge of knowledge? That which is older, heavier, lighter, is older, heavier, and lighter than something else, not than itself, and this seems to be true of all relative notions—the object of relation is outside of them; at any rate they can only have relation to themselves in the form of that object. Whether there are any such cases of reflex relation or not, and whether that sort of knowledge which we term Temperance is of this reflex nature, has yet to be determined by the great metaphysician. But even if knowledge can know itself, how does the knowledge of what we know imply the knowledge of what we do not know? Besides, knowledge is an abstraction only, and will not inform us of any particular subject, such as medicine, building, and the like. It may tell us that we or other men know something, but can never tell us what we know.

\par  Admitting that there is a knowledge of what we know and of what we do not know, which would supply a rule and measure of all things, still there would be no good in this; and the knowledge which temperance gives must be of a kind which will do us good; for temperance is a good. But this universal knowledge does not tend to our happiness and good: the only kind of knowledge which brings happiness is the knowledge of good and evil. To this Critias replies that the science or knowledge of good and evil, and all the other sciences, are regulated by the higher science or knowledge of knowledge. Socrates replies by again dividing the abstract from the concrete, and asks how this knowledge conduces to happiness in the same definite way in which medicine conduces to health.

\par  And now, after making all these concessions, which are really inadmissible, we are still as far as ever from ascertaining the nature of temperance, which Charmides has already discovered, and had therefore better rest in the knowledge that the more temperate he is the happier he will be, and not trouble himself with the speculations of Socrates.

\par  In this Dialogue may be noted (1) The Greek ideal of beauty and goodness, the vision of the fair soul in the fair body, realised in the beautiful Charmides; (2) The true conception of medicine as a science of the whole as well as the parts, and of the mind as well as the body, which is playfully intimated in the story of the Thracian; (3) The tendency of the age to verbal distinctions, which here, as in the Protagoras and Cratylus, are ascribed to the ingenuity of Prodicus; and to interpretations or rather parodies of Homer or Hesiod, which are eminently characteristic of Plato and his contemporaries; (4) The germ of an ethical principle contained in the notion that temperance is 'doing one's own business,' which in the Republic (such is the shifting character of the Platonic philosophy) is given as the definition, not of temperance, but of justice; (5) The impatience which is exhibited by Socrates of any definition of temperance in which an element of science or knowledge is not included; (6) The beginning of metaphysics and logic implied in the two questions: whether there can be a science of science, and whether the knowledge of what you know is the same as the knowledge of what you do not know; and also in the distinction between 'what you know' and 'that you know,' (Greek;) here too is the first conception of an absolute self-determined science (the claims of which, however, are disputed by Socrates, who asks cui bono?) as well as the first suggestion of the difficulty of the abstract and concrete, and one of the earliest anticipations of the relation of subject and object, and of the subjective element in knowledge—a 'rich banquet' of metaphysical questions in which we 'taste of many things.' (7) And still the mind of Plato, having snatched for a moment at these shadows of the future, quickly rejects them: thus early has he reached the conclusion that there can be no science which is a 'science of nothing' (Parmen.). (8) The conception of a science of good and evil also first occurs here, an anticipation of the Philebus and Republic as well as of moral philosophy in later ages.

\par  The dramatic interest of the Dialogue chiefly centres in the youth Charmides, with whom Socrates talks in the kindly spirit of an elder. His childlike simplicity and ingenuousness are contrasted with the dialectical and rhetorical arts of Critias, who is the grown-up man of the world, having a tincture of philosophy. No hint is given, either here or in the Timaeus, of the infamy which attaches to the name of the latter in Athenian history. He is simply a cultivated person who, like his kinsman Plato, is ennobled by the connection of his family with Solon (Tim. ), and had been the follower, if not the disciple, both of Socrates and of the Sophists. In the argument he is not unfair, if allowance is made for a slight rhetorical tendency, and for a natural desire to save his reputation with the company; he is sometimes nearer the truth than Socrates. Nothing in his language or behaviour is unbecoming the guardian of the beautiful Charmides. His love of reputation is characteristically Greek, and contrasts with the humility of Socrates. Nor in Charmides himself do we find any resemblance to the Charmides of history, except, perhaps, the modest and retiring nature which, according to Xenophon, at one time of his life prevented him from speaking in the Assembly (Mem. ); and we are surprised to hear that, like Critias, he afterwards became one of the thirty tyrants. In the Dialogue he is a pattern of virtue, and is therefore in no need of the charm which Socrates is unable to apply. With youthful naivete, keeping his secret and entering into the spirit of Socrates, he enjoys the detection of his elder and guardian Critias, who is easily seen to be the author of the definition which he has so great an interest in maintaining. The preceding definition, 'Temperance is doing one's own business,' is assumed to have been borrowed by Charmides from another; and when the enquiry becomes more abstract he is superseded by Critias (Theaet. ; Euthyd.). Socrates preserves his accustomed irony to the end; he is in the neighbourhood of several great truths, which he views in various lights, but always either by bringing them to the test of common sense, or by demanding too great exactness in the use of words, turns aside from them and comes at last to no conclusion.

\par  The definitions of temperance proceed in regular order from the popular to the philosophical. The first two are simple enough and partially true, like the first thoughts of an intelligent youth; the third, which is a real contribution to ethical philosophy, is perverted by the ingenuity of Socrates, and hardly rescued by an equal perversion on the part of Critias. The remaining definitions have a higher aim, which is to introduce the element of knowledge, and at last to unite good and truth in a single science. But the time has not yet arrived for the realization of this vision of metaphysical philosophy; and such a science when brought nearer to us in the Philebus and the Republic will not be called by the name of (Greek). Hence we see with surprise that Plato, who in his other writings identifies good and knowledge, here opposes them, and asks, almost in the spirit of Aristotle, how can there be a knowledge of knowledge, and even if attainable, how can such a knowledge be of any use?

\par  The difficulty of the Charmides arises chiefly from the two senses of the word (Greek), or temperance. From the ethical notion of temperance, which is variously defined to be quietness, modesty, doing our own business, the doing of good actions, the dialogue passes onto the intellectual conception of (Greek), which is declared also to be the science of self-knowledge, or of the knowledge of what we know and do not know, or of the knowledge of good and evil. The dialogue represents a stage in the history of philosophy in which knowledge and action were not yet distinguished. Hence the confusion between them, and the easy transition from one to the other. The definitions which are offered are all rejected, but it is to be observed that they all tend to throw a light on the nature of temperance, and that, unlike the distinction of Critias between (Greek), none of them are merely verbal quibbles, it is implied that this question, although it has not yet received a solution in theory, has been already answered by Charmides himself, who has learned to practise the virtue of self-knowledge which philosophers are vainly trying to define in words. In a similar spirit we might say to a young man who is disturbed by theological difficulties, 'Do not trouble yourself about such matters, but only lead a good life;' and yet in either case it is not to be denied that right ideas of truth may contribute greatly to the improvement of character.

\par  The reasons why the Charmides, Lysis, Laches have been placed together and first in the series of Platonic dialogues, are: (i) Their shortness and simplicity. The Charmides and the Lysis, if not the Laches, are of the same 'quality' as the Phaedrus and Symposium: and it is probable, though far from certain, that the slighter effort preceded the greater one. (ii) Their eristic, or rather Socratic character; they belong to the class called dialogues of search (Greek), which have no conclusion. (iii) The absence in them of certain favourite notions of Plato, such as the doctrine of recollection and of the Platonic ideas; the questions, whether virtue can be taught; whether the virtues are one or many. (iv) They have a want of depth, when compared with the dialogues of the middle and later period; and a youthful beauty and grace which is wanting in the later ones. (v) Their resemblance to one another; in all the three boyhood has a great part. These reasons have various degrees of weight in determining their place in the catalogue of the Platonic writings, though they are not conclusive. No arrangement of the Platonic dialogues can be strictly chronological. The order which has been adopted is intended mainly for the convenience of the reader; at the same time, indications of the date supplied either by Plato himself or allusions found in the dialogues have not been lost sight of. Much may be said about this subject, but the results can only be probable; there are no materials which would enable us to attain to anything like certainty.

\par  The relations of knowledge and virtue are again brought forward in the companion dialogues of the Lysis and Laches; and also in the Protagoras and Euthydemus. The opposition of abstract and particular knowledge in this dialogue may be compared with a similar opposition of ideas and phenomena which occurs in the Prologues to the Parmenides, but seems rather to belong to a later stage of the philosophy of Plato.

\par 
\section{
      CHARMIDES,   OR TEMPERANCE
    }
\par 
  
\par  Yesterday evening I returned from the army at Potidaea, and having been a good while away, I thought that I should like to go and look at my old haunts. So I went into the palaestra of Taureas, which is over against the temple adjoining the porch of the King Archon, and there I found a number of persons, most of whom I knew, but not all. My visit was unexpected, and no sooner did they see me entering than they saluted me from afar on all sides; and Chaerephon, who is a kind of madman, started up and ran to me, seizing my hand, and saying, How did you escape, Socrates?—(I should explain that an engagement had taken place at Potidaea not long before we came away, of which the news had only just reached Athens.)

\par  You see, I replied, that here I am.

\par  There was a report, he said, that the engagement was very severe, and that many of our acquaintance had fallen.

\par  That, I replied, was not far from the truth.

\par  I suppose, he said, that you were present.

\par  I was.

\par  Then sit down, and tell us the whole story, which as yet we have only heard imperfectly.

\par  I took the place which he assigned to me, by the side of Critias the son of Callaeschrus, and when I had saluted him and the rest of the company, I told them the news from the army, and answered their several enquiries.

\par  Then, when there had been enough of this, I, in my turn, began to make enquiries about matters at home—about the present state of philosophy, and about the youth. I asked whether any of them were remarkable for wisdom or beauty, or both. Critias, glancing at the door, invited my attention to some youths who were coming in, and talking noisily to one another, followed by a crowd. Of the beauties, Socrates, he said, I fancy that you will soon be able to form a judgment. For those who are just entering are the advanced guard of the great beauty, as he is thought to be, of the day, and he is likely to be not far off himself.

\par  Who is he, I said; and who is his father?

\par  Charmides, he replied, is his name; he is my cousin, and the son of my uncle Glaucon: I rather think that you know him too, although he was not grown up at the time of your departure.

\par  Certainly, I know him, I said, for he was remarkable even then when he was still a child, and I should imagine that by this time he must be almost a young man.

\par  You will see, he said, in a moment what progress he has made and what he is like. He had scarcely said the word, when Charmides entered.

\par  Now you know, my friend, that I cannot measure anything, and of the beautiful, I am simply such a measure as a white line is of chalk; for almost all young persons appear to be beautiful in my eyes. But at that moment, when I saw him coming in, I confess that I was quite astonished at his beauty and stature; all the world seemed to be enamoured of him; amazement and confusion reigned when he entered; and a troop of lovers followed him. That grown-up men like ourselves should have been affected in this way was not surprising, but I observed that there was the same feeling among the boys; all of them, down to the very least child, turned and looked at him, as if he had been a statue.

\par  Chaerephon called me and said: What do you think of him, Socrates? Has he not a beautiful face?

\par  Most beautiful, I said.

\par  But you would think nothing of his face, he replied, if you could see his naked form: he is absolutely perfect.

\par  And to this they all agreed.

\par  By Heracles, I said, there never was such a paragon, if he has only one other slight addition.

\par  What is that? said Critias.

\par  If he has a noble soul; and being of your house, Critias, he may be expected to have this.

\par  He is as fair and good within, as he is without, replied Critias.

\par  Then, before we see his body, should we not ask him to show us his soul, naked and undisguised? he is just of an age at which he will like to talk.

\par  That he will, said Critias, and I can tell you that he is a philosopher already, and also a considerable poet, not in his own opinion only, but in that of others.

\par  That, my dear Critias, I replied, is a distinction which has long been in your family, and is inherited by you from Solon. But why do you not call him, and show him to us? for even if he were younger than he is, there could be no impropriety in his talking to us in the presence of you, who are his guardian and cousin.

\par  Very well, he said; then I will call him; and turning to the attendant, he said, Call Charmides, and tell him that I want him to come and see a physician about the illness of which he spoke to me the day before yesterday. Then again addressing me, he added: He has been complaining lately of having a headache when he rises in the morning: now why should you not make him believe that you know a cure for the headache?

\par  Why not, I said; but will he come?

\par  He will be sure to come, he replied.

\par  He came as he was bidden, and sat down between Critias and me. Great amusement was occasioned by every one pushing with might and main at his neighbour in order to make a place for him next to themselves, until at the two ends of the row one had to get up and the other was rolled over sideways. Now I, my friend, was beginning to feel awkward; my former bold belief in my powers of conversing with him had vanished. And when Critias told him that I was the person who had the cure, he looked at me in such an indescribable manner, and was just going to ask a question. And at that moment all the people in the palaestra crowded about us, and, O rare! I caught a sight of the inwards of his garment, and took the flame. Then I could no longer contain myself. I thought how well Cydias understood the nature of love, when, in speaking of a fair youth, he warns some one 'not to bring the fawn in the sight of the lion to be devoured by him,' for I felt that I had been overcome by a sort of wild-beast appetite. But I controlled myself, and when he asked me if I knew the cure of the headache, I answered, but with an effort, that I did know.

\par  And what is it? he said.

\par  I replied that it was a kind of leaf, which required to be accompanied by a charm, and if a person would repeat the charm at the same time that he used the cure, he would be made whole; but that without the charm the leaf would be of no avail.

\par  Then I will write out the charm from your dictation, he said.

\par  With my consent? I said, or without my consent?

\par  With your consent, Socrates, he said, laughing.

\par  Very good, I said; and are you quite sure that you know my name?

\par  I ought to know you, he replied, for there is a great deal said about you among my companions; and I remember when I was a child seeing you in company with my cousin Critias.

\par  I am glad to find that you remember me, I said; for I shall now be more at home with you and shall be better able to explain the nature of the charm, about which I felt a difficulty before. For the charm will do more, Charmides, than only cure the headache. I dare say that you have heard eminent physicians say to a patient who comes to them with bad eyes, that they cannot cure his eyes by themselves, but that if his eyes are to be cured, his head must be treated; and then again they say that to think of curing the head alone, and not the rest of the body also, is the height of folly. And arguing in this way they apply their methods to the whole body, and try to treat and heal the whole and the part together. Did you ever observe that this is what they say?

\par  Yes, he said.

\par  And they are right, and you would agree with them?

\par  Yes, he said, certainly I should.

\par  His approving answers reassured me, and I began by degrees to regain confidence, and the vital heat returned. Such, Charmides, I said, is the nature of the charm, which I learned when serving with the army from one of the physicians of the Thracian king Zamolxis, who are said to be so skilful that they can even give immortality. This Thracian told me that in these notions of theirs, which I was just now mentioning, the Greek physicians are quite right as far as they go; but Zamolxis, he added, our king, who is also a god, says further, 'that as you ought not to attempt to cure the eyes without the head, or the head without the body, so neither ought you to attempt to cure the body without the soul; and this,' he said, 'is the reason why the cure of many diseases is unknown to the physicians of Hellas, because they are ignorant of the whole, which ought to be studied also; for the part can never be well unless the whole is well.' For all good and evil, whether in the body or in human nature, originates, as he declared, in the soul, and overflows from thence, as if from the head into the eyes. And therefore if the head and body are to be well, you must begin by curing the soul; that is the first thing. And the cure, my dear youth, has to be effected by the use of certain charms, and these charms are fair words; and by them temperance is implanted in the soul, and where temperance is, there health is speedily imparted, not only to the head, but to the whole body. And he who taught me the cure and the charm at the same time added a special direction: 'Let no one,' he said, 'persuade you to cure the head, until he has first given you his soul to be cured by the charm. For this,' he said, 'is the great error of our day in the treatment of the human body, that physicians separate the soul from the body.' And he added with emphasis, at the same time making me swear to his words, 'Let no one, however rich, or noble, or fair, persuade you to give him the cure, without the charm.' Now I have sworn, and I must keep my oath, and therefore if you will allow me to apply the Thracian charm first to your soul, as the stranger directed, I will afterwards proceed to apply the cure to your head. But if not, I do not know what I am to do with you, my dear Charmides.

\par  Critias, when he heard this, said: The headache will be an unexpected gain to my young relation, if the pain in his head compels him to improve his mind: and I can tell you, Socrates, that Charmides is not only pre-eminent in beauty among his equals, but also in that quality which is given by the charm; and this, as you say, is temperance?

\par  Yes, I said.

\par  Then let me tell you that he is the most temperate of human beings, and for his age inferior to none in any quality.

\par  Yes, I said, Charmides; and indeed I think that you ought to excel others in all good qualities; for if I am not mistaken there is no one present who could easily point out two Athenian houses, whose union would be likely to produce a better or nobler scion than the two from which you are sprung. There is your father's house, which is descended from Critias the son of Dropidas, whose family has been commemorated in the panegyrical verses of Anacreon, Solon, and many other poets, as famous for beauty and virtue and all other high fortune: and your mother's house is equally distinguished; for your maternal uncle, Pyrilampes, is reputed never to have found his equal, in Persia at the court of the great king, or on the continent of Asia, in all the places to which he went as ambassador, for stature and beauty; that whole family is not a whit inferior to the other. Having such ancestors you ought to be first in all things, and, sweet son of Glaucon, your outward form is no dishonour to any of them. If to beauty you add temperance, and if in other respects you are what Critias declares you to be, then, dear Charmides, blessed art thou, in being the son of thy mother. And here lies the point; for if, as he declares, you have this gift of temperance already, and are temperate enough, in that case you have no need of any charms, whether of Zamolxis or of Abaris the Hyperborean, and I may as well let you have the cure of the head at once; but if you have not yet acquired this quality, I must use the charm before I give you the medicine. Please, therefore, to inform me whether you admit the truth of what Critias has been saying;—have you or have you not this quality of temperance?

\par  Charmides blushed, and the blush heightened his beauty, for modesty is becoming in youth; he then said very ingenuously, that he really could not at once answer, either yes, or no, to the question which I had asked: For, said he, if I affirm that I am not temperate, that would be a strange thing for me to say of myself, and also I should give the lie to Critias, and many others who think as he tells you, that I am temperate: but, on the other hand, if I say that I am, I shall have to praise myself, which would be ill manners; and therefore I do not know how to answer you.

\par  I said to him: That is a natural reply, Charmides, and I think that you and I ought together to enquire whether you have this quality about which I am asking or not; and then you will not be compelled to say what you do not like; neither shall I be a rash practitioner of medicine: therefore, if you please, I will share the enquiry with you, but I will not press you if you would rather not.

\par  There is nothing which I should like better, he said; and as far as I am concerned you may proceed in the way which you think best.

\par  I think, I said, that I had better begin by asking you a question; for if temperance abides in you, you must have an opinion about her; she must give some intimation of her nature and qualities, which may enable you to form a notion of her. Is not that true?

\par  Yes, he said, that I think is true.

\par  You know your native language, I said, and therefore you must be able to tell what you feel about this.

\par  Certainly, he said.

\par  In order, then, that I may form a conjecture whether you have temperance abiding in you or not, tell me, I said, what, in your opinion, is Temperance?

\par  At first he hesitated, and was very unwilling to answer: then he said that he thought temperance was doing things orderly and quietly, such things for example as walking in the streets, and talking, or anything else of that nature. In a word, he said, I should answer that, in my opinion, temperance is quietness.

\par  Are you right, Charmides? I said. No doubt some would affirm that the quiet are the temperate; but let us see whether these words have any meaning; and first tell me whether you would not acknowledge temperance to be of the class of the noble and good?

\par  Yes.

\par  But which is best when you are at the writing-master's, to write the same letters quickly or quietly?

\par  Quickly.

\par  And to read quickly or slowly?

\par  Quickly again.

\par  And in playing the lyre, or wrestling, quickness or sharpness are far better than quietness and slowness?

\par  Yes.

\par  And the same holds in boxing and in the pancratium?

\par  Certainly.

\par  And in leaping and running and in bodily exercises generally, quickness and agility are good; slowness, and inactivity, and quietness, are bad?

\par  That is evident.

\par  Then, I said, in all bodily actions, not quietness, but the greatest agility and quickness, is noblest and best?

\par  Yes, certainly.

\par  And is temperance a good?

\par  Yes.

\par  Then, in reference to the body, not quietness, but quickness will be the higher degree of temperance, if temperance is a good?

\par  True, he said.

\par  And which, I said, is better—facility in learning, or difficulty in learning?

\par  Facility.

\par  Yes, I said; and facility in learning is learning quickly, and difficulty in learning is learning quietly and slowly?

\par  True.

\par  And is it not better to teach another quickly and energetically, rather than quietly and slowly?

\par  Yes.

\par  And which is better, to call to mind, and to remember, quickly and readily, or quietly and slowly?

\par  The former.

\par  And is not shrewdness a quickness or cleverness of the soul, and not a quietness?

\par  True.

\par  And is it not best to understand what is said, whether at the writing-master's or the music-master's, or anywhere else, not as quietly as possible, but as quickly as possible?

\par  Yes.

\par  And in the searchings or deliberations of the soul, not the quietest, as I imagine, and he who with difficulty deliberates and discovers, is thought worthy of praise, but he who does so most easily and quickly?

\par  Quite true, he said.

\par  And in all that concerns either body or soul, swiftness and activity are clearly better than slowness and quietness?

\par  Clearly they are.

\par  Then temperance is not quietness, nor is the temperate life quiet,—certainly not upon this view; for the life which is temperate is supposed to be the good. And of two things, one is true,—either never, or very seldom, do the quiet actions in life appear to be better than the quick and energetic ones; or supposing that of the nobler actions, there are as many quiet, as quick and vehement: still, even if we grant this, temperance will not be acting quietly any more than acting quickly and energetically, either in walking or talking or in anything else; nor will the quiet life be more temperate than the unquiet, seeing that temperance is admitted by us to be a good and noble thing, and the quick have been shown to be as good as the quiet.

\par  I think, he said, Socrates, that you are right.

\par  Then once more, Charmides, I said, fix your attention, and look within; consider the effect which temperance has upon yourself, and the nature of that which has the effect. Think over all this, and, like a brave youth, tell me—What is temperance?

\par  After a moment's pause, in which he made a real manly effort to think, he said: My opinion is, Socrates, that temperance makes a man ashamed or modest, and that temperance is the same as modesty.

\par  Very good, I said; and did you not admit, just now, that temperance is noble?

\par  Yes, certainly, he said.

\par  And the temperate are also good?

\par  Yes.

\par  And can that be good which does not make men good?

\par  Certainly not.

\par  And you would infer that temperance is not only noble, but also good?

\par  That is my opinion.

\par  Well, I said; but surely you would agree with Homer when he says,

\par  'Modesty is not good for a needy man'?

\par  Yes, he said; I agree.

\par  Then I suppose that modesty is and is not good?

\par  Clearly.

\par  But temperance, whose presence makes men only good, and not bad, is always good?

\par  That appears to me to be as you say.

\par  And the inference is that temperance cannot be modesty—if temperance is a good, and if modesty is as much an evil as a good?

\par  All that, Socrates, appears to me to be true; but I should like to know what you think about another definition of temperance, which I just now remember to have heard from some one, who said, 'That temperance is doing our own business.' Was he right who affirmed that?

\par  You monster! I said; this is what Critias, or some philosopher has told you.

\par  Some one else, then, said Critias; for certainly I have not.

\par  But what matter, said Charmides, from whom I heard this?

\par  No matter at all, I replied; for the point is not who said the words, but whether they are true or not.

\par  There you are in the right, Socrates, he replied.

\par  To be sure, I said; yet I doubt whether we shall ever be able to discover their truth or falsehood; for they are a kind of riddle.

\par  What makes you think so? he said.

\par  Because, I said, he who uttered them seems to me to have meant one thing, and said another. Is the scribe, for example, to be regarded as doing nothing when he reads or writes?

\par  I should rather think that he was doing something.

\par  And does the scribe write or read, or teach you boys to write or read, your own names only, or did you write your enemies' names as well as your own and your friends'?

\par  As much one as the other.

\par  And was there anything meddling or intemperate in this?

\par  Certainly not.

\par  And yet if reading and writing are the same as doing, you were doing what was not your own business?

\par  But they are the same as doing.

\par  And the healing art, my friend, and building, and weaving, and doing anything whatever which is done by art,—these all clearly come under the head of doing?

\par  Certainly.

\par  And do you think that a state would be well ordered by a law which compelled every man to weave and wash his own coat, and make his own shoes, and his own flask and strigil, and other implements, on this principle of every one doing and performing his own, and abstaining from what is not his own?

\par  I think not, he said.

\par  But, I said, a temperate state will be a well-ordered state.

\par  Of course, he replied.

\par  Then temperance, I said, will not be doing one's own business; not at least in this way, or doing things of this sort?

\par  Clearly not.

\par  Then, as I was just now saying, he who declared that temperance is a man doing his own business had another and a hidden meaning; for I do not think that he could have been such a fool as to mean this. Was he a fool who told you, Charmides?

\par  Nay, he replied, I certainly thought him a very wise man.

\par  Then I am quite certain that he put forth his definition as a riddle, thinking that no one would know the meaning of the words 'doing his own business.'

\par  I dare say, he replied.

\par  And what is the meaning of a man doing his own business? Can you tell me?

\par  Indeed, I cannot; and I should not wonder if the man himself who used this phrase did not understand what he was saying. Whereupon he laughed slyly, and looked at Critias.

\par  Critias had long been showing uneasiness, for he felt that he had a reputation to maintain with Charmides and the rest of the company. He had, however, hitherto managed to restrain himself; but now he could no longer forbear, and I am convinced of the truth of the suspicion which I entertained at the time, that Charmides had heard this answer about temperance from Critias. And Charmides, who did not want to answer himself, but to make Critias answer, tried to stir him up. He went on pointing out that he had been refuted, at which Critias grew angry, and appeared, as I thought, inclined to quarrel with him; just as a poet might quarrel with an actor who spoiled his poems in repeating them; so he looked hard at him and said—

\par  Do you imagine, Charmides, that the author of this definition of temperance did not understand the meaning of his own words, because you do not understand them?

\par  Why, at his age, I said, most excellent Critias, he can hardly be expected to understand; but you, who are older, and have studied, may well be assumed to know the meaning of them; and therefore, if you agree with him, and accept his definition of temperance, I would much rather argue with you than with him about the truth or falsehood of the definition.

\par  I entirely agree, said Critias, and accept the definition.

\par  Very good, I said; and now let me repeat my question—Do you admit, as I was just now saying, that all craftsmen make or do something?

\par  I do.

\par  And do they make or do their own business only, or that of others also?

\par  They make or do that of others also.

\par  And are they temperate, seeing that they make not for themselves or their own business only?

\par  Why not? he said.

\par  No objection on my part, I said, but there may be a difficulty on his who proposes as a definition of temperance, 'doing one's own business,' and then says that there is no reason why those who do the business of others should not be temperate.

\par  Nay (The English reader has to observe that the word 'make' (Greek), in Greek, has also the sense of 'do' (Greek). ), said he; did I ever acknowledge that those who do the business of others are temperate? I said, those who make, not those who do.

\par  What! I asked; do you mean to say that doing and making are not the same?

\par  No more, he replied, than making or working are the same; thus much I have learned from Hesiod, who says that 'work is no disgrace.' Now do you imagine that if he had meant by working and doing such things as you were describing, he would have said that there was no disgrace in them—for example, in the manufacture of shoes, or in selling pickles, or sitting for hire in a house of ill-fame? That, Socrates, is not to be supposed: but I conceive him to have distinguished making from doing and work; and, while admitting that the making anything might sometimes become a disgrace, when the employment was not honourable, to have thought that work was never any disgrace at all. For things nobly and usefully made he called works; and such makings he called workings, and doings; and he must be supposed to have called such things only man's proper business, and what is hurtful, not his business: and in that sense Hesiod, and any other wise man, may be reasonably supposed to call him wise who does his own work.

\par  O Critias, I said, no sooner had you opened your mouth, than I pretty well knew that you would call that which is proper to a man, and that which is his own, good; and that the makings (Greek) of the good you would call doings (Greek), for I am no stranger to the endless distinctions which Prodicus draws about names. Now I have no objection to your giving names any signification which you please, if you will only tell me what you mean by them. Please then to begin again, and be a little plainer. Do you mean that this doing or making, or whatever is the word which you would use, of good actions, is temperance?

\par  I do, he said.

\par  Then not he who does evil, but he who does good, is temperate?

\par  Yes, he said; and you, friend, would agree.

\par  No matter whether I should or not; just now, not what I think, but what you are saying, is the point at issue.

\par  Well, he answered; I mean to say, that he who does evil, and not good, is not temperate; and that he is temperate who does good, and not evil: for temperance I define in plain words to be the doing of good actions.

\par  And you may be very likely right in what you are saying; but I am curious to know whether you imagine that temperate men are ignorant of their own temperance?

\par  I do not think so, he said.

\par  And yet were you not saying, just now, that craftsmen might be temperate in doing another's work, as well as in doing their own?

\par  I was, he replied; but what is your drift?

\par  I have no particular drift, but I wish that you would tell me whether a physician who cures a patient may do good to himself and good to another also?

\par  I think that he may.

\par  And he who does so does his duty?

\par  Yes.

\par  And does not he who does his duty act temperately or wisely?

\par  Yes, he acts wisely.

\par  But must the physician necessarily know when his treatment is likely to prove beneficial, and when not? or must the craftsman necessarily know when he is likely to be benefited, and when not to be benefited, by the work which he is doing?

\par  I suppose not.

\par  Then, I said, he may sometimes do good or harm, and not know what he is himself doing, and yet, in doing good, as you say, he has done temperately or wisely. Was not that your statement?

\par  Yes.

\par  Then, as would seem, in doing good, he may act wisely or temperately, and be wise or temperate, but not know his own wisdom or temperance?

\par  But that, Socrates, he said, is impossible; and therefore if this is, as you imply, the necessary consequence of any of my previous admissions, I will withdraw them, rather than admit that a man can be temperate or wise who does not know himself; and I am not ashamed to confess that I was in error. For self-knowledge would certainly be maintained by me to be the very essence of knowledge, and in this I agree with him who dedicated the inscription, 'Know thyself!' at Delphi. That word, if I am not mistaken, is put there as a sort of salutation which the god addresses to those who enter the temple; as much as to say that the ordinary salutation of 'Hail!' is not right, and that the exhortation 'Be temperate!' would be a far better way of saluting one another. The notion of him who dedicated the inscription was, as I believe, that the god speaks to those who enter his temple, not as men speak; but, when a worshipper enters, the first word which he hears is 'Be temperate!' This, however, like a prophet he expresses in a sort of riddle, for 'Know thyself!' and 'Be temperate!' are the same, as I maintain, and as the letters imply (Greek), and yet they may be easily misunderstood; and succeeding sages who added 'Never too much,' or, 'Give a pledge, and evil is nigh at hand,' would appear to have so misunderstood them; for they imagined that 'Know thyself!' was a piece of advice which the god gave, and not his salutation of the worshippers at their first coming in; and they dedicated their own inscription under the idea that they too would give equally useful pieces of advice. Shall I tell you, Socrates, why I say all this? My object is to leave the previous discussion (in which I know not whether you or I are more right, but, at any rate, no clear result was attained), and to raise a new one in which I will attempt to prove, if you deny, that temperance is self-knowledge.

\par  Yes, I said, Critias; but you come to me as though I professed to know about the questions which I ask, and as though I could, if I only would, agree with you. Whereas the fact is that I enquire with you into the truth of that which is advanced from time to time, just because I do not know; and when I have enquired, I will say whether I agree with you or not. Please then to allow me time to reflect.

\par  Reflect, he said.

\par  I am reflecting, I replied, and discover that temperance, or wisdom, if implying a knowledge of anything, must be a science, and a science of something.

\par  Yes, he said; the science of itself.

\par  Is not medicine, I said, the science of health?

\par  True.

\par  And suppose, I said, that I were asked by you what is the use or effect of medicine, which is this science of health, I should answer that medicine is of very great use in producing health, which, as you will admit, is an excellent effect.

\par  Granted.

\par  And if you were to ask me, what is the result or effect of architecture, which is the science of building, I should say houses, and so of other arts, which all have their different results. Now I want you, Critias, to answer a similar question about temperance, or wisdom, which, according to you, is the science of itself. Admitting this view, I ask of you, what good work, worthy of the name wise, does temperance or wisdom, which is the science of itself, effect? Answer me.

\par  That is not the true way of pursuing the enquiry, Socrates, he said; for wisdom is not like the other sciences, any more than they are like one another: but you proceed as if they were alike. For tell me, he said, what result is there of computation or geometry, in the same sense as a house is the result of building, or a garment of weaving, or any other work of any other art? Can you show me any such result of them? You cannot.

\par  That is true, I said; but still each of these sciences has a subject which is different from the science. I can show you that the art of computation has to do with odd and even numbers in their numerical relations to themselves and to each other. Is not that true?

\par  Yes, he said.

\par  And the odd and even numbers are not the same with the art of computation?

\par  They are not.

\par  The art of weighing, again, has to do with lighter and heavier; but the art of weighing is one thing, and the heavy and the light another. Do you admit that?

\par  Yes.

\par  Now, I want to know, what is that which is not wisdom, and of which wisdom is the science?

\par  You are just falling into the old error, Socrates, he said. You come asking in what wisdom or temperance differs from the other sciences, and then you try to discover some respect in which they are alike; but they are not, for all the other sciences are of something else, and not of themselves; wisdom alone is a science of other sciences, and of itself. And of this, as I believe, you are very well aware: and that you are only doing what you denied that you were doing just now, trying to refute me, instead of pursuing the argument.

\par  And what if I am? How can you think that I have any other motive in refuting you but what I should have in examining into myself? which motive would be just a fear of my unconsciously fancying that I knew something of which I was ignorant. And at this moment I pursue the argument chiefly for my own sake, and perhaps in some degree also for the sake of my other friends. For is not the discovery of things as they truly are, a good common to all mankind?

\par  Yes, certainly, Socrates, he said.

\par  Then, I said, be cheerful, sweet sir, and give your opinion in answer to the question which I asked, never minding whether Critias or Socrates is the person refuted; attend only to the argument, and see what will come of the refutation.

\par  I think that you are right, he replied; and I will do as you say.

\par  Tell me, then, I said, what you mean to affirm about wisdom.

\par  I mean to say that wisdom is the only science which is the science of itself as well as of the other sciences.

\par  But the science of science, I said, will also be the science of the absence of science.

\par  Very true, he said.

\par  Then the wise or temperate man, and he only, will know himself, and be able to examine what he knows or does not know, and to see what others know and think that they know and do really know; and what they do not know, and fancy that they know, when they do not. No other person will be able to do this. And this is wisdom and temperance and self-knowledge—for a man to know what he knows, and what he does not know. That is your meaning?

\par  Yes, he said.

\par  Now then, I said, making an offering of the third or last argument to Zeus the Saviour, let us begin again, and ask, in the first place, whether it is or is not possible for a person to know that he knows and does not know what he knows and does not know; and in the second place, whether, if perfectly possible, such knowledge is of any use.

\par  That is what we have to consider, he said.

\par  And here, Critias, I said, I hope that you will find a way out of a difficulty into which I have got myself. Shall I tell you the nature of the difficulty?

\par  By all means, he replied.

\par  Does not what you have been saying, if true, amount to this: that there must be a single science which is wholly a science of itself and of other sciences, and that the same is also the science of the absence of science?

\par  Yes.

\par  But consider how monstrous this proposition is, my friend: in any parallel case, the impossibility will be transparent to you.

\par  How is that? and in what cases do you mean?

\par  In such cases as this: Suppose that there is a kind of vision which is not like ordinary vision, but a vision of itself and of other sorts of vision, and of the defect of them, which in seeing sees no colour, but only itself and other sorts of vision: Do you think that there is such a kind of vision?

\par  Certainly not.

\par  Or is there a kind of hearing which hears no sound at all, but only itself and other sorts of hearing, or the defects of them?

\par  There is not.

\par  Or take all the senses: can you imagine that there is any sense of itself and of other senses, but which is incapable of perceiving the objects of the senses?

\par  I think not.

\par  Could there be any desire which is not the desire of any pleasure, but of itself, and of all other desires?

\par  Certainly not.

\par  Or can you imagine a wish which wishes for no good, but only for itself and all other wishes?

\par  I should answer, No.

\par  Or would you say that there is a love which is not the love of beauty, but of itself and of other loves?

\par  I should not.

\par  Or did you ever know of a fear which fears itself or other fears, but has no object of fear?

\par  I never did, he said.

\par  Or of an opinion which is an opinion of itself and of other opinions, and which has no opinion on the subjects of opinion in general?

\par  Certainly not.

\par  But surely we are assuming a science of this kind, which, having no subject-matter, is a science of itself and of the other sciences?

\par  Yes, that is what is affirmed.

\par  But how strange is this, if it be indeed true: we must not however as yet absolutely deny the possibility of such a science; let us rather consider the matter.

\par  You are quite right.

\par  Well then, this science of which we are speaking is a science of something, and is of a nature to be a science of something?

\par  Yes.

\par  Just as that which is greater is of a nature to be greater than something else? (Socrates is intending to show that science differs from the object of science, as any other relative differs from the object of relation. But where there is comparison—greater, less, heavier, lighter, and the like—a relation to self as well as to other things involves an absolute contradiction; and in other cases, as in the case of the senses, is hardly conceivable. The use of the genitive after the comparative in Greek, (Greek), creates an unavoidable obscurity in the translation.)

\par  Yes.

\par  Which is less, if the other is conceived to be greater?

\par  To be sure.

\par  And if we could find something which is at once greater than itself, and greater than other great things, but not greater than those things in comparison of which the others are greater, then that thing would have the property of being greater and also less than itself?

\par  That, Socrates, he said, is the inevitable inference.

\par  Or if there be a double which is double of itself and of other doubles, these will be halves; for the double is relative to the half?

\par  That is true.

\par  And that which is greater than itself will also be less, and that which is heavier will also be lighter, and that which is older will also be younger: and the same of other things; that which has a nature relative to self will retain also the nature of its object: I mean to say, for example, that hearing is, as we say, of sound or voice. Is that true?

\par  Yes.

\par  Then if hearing hears itself, it must hear a voice; for there is no other way of hearing.

\par  Certainly.

\par  And sight also, my excellent friend, if it sees itself must see a colour, for sight cannot see that which has no colour.

\par  No.

\par  Do you remark, Critias, that in several of the examples which have been recited the notion of a relation to self is altogether inadmissible, and in other cases hardly credible—inadmissible, for example, in the case of magnitudes, numbers, and the like?

\par  Very true.

\par  But in the case of hearing and sight, or in the power of self-motion, and the power of heat to burn, this relation to self will be regarded as incredible by some, but perhaps not by others. And some great man, my friend, is wanted, who will satisfactorily determine for us, whether there is nothing which has an inherent property of relation to self, or some things only and not others; and whether in this class of self-related things, if there be such a class, that science which is called wisdom or temperance is included. I altogether distrust my own power of determining these matters: I am not certain whether there is such a science of science at all; and even if there be, I should not acknowledge this to be wisdom or temperance, until I can also see whether such a science would or would not do us any good; for I have an impression that temperance is a benefit and a good. And therefore, O son of Callaeschrus, as you maintain that temperance or wisdom is a science of science, and also of the absence of science, I will request you to show in the first place, as I was saying before, the possibility, and in the second place, the advantage, of such a science; and then perhaps you may satisfy me that you are right in your view of temperance.

\par  Critias heard me say this, and saw that I was in a difficulty; and as one person when another yawns in his presence catches the infection of yawning from him, so did he seem to be driven into a difficulty by my difficulty. But as he had a reputation to maintain, he was ashamed to admit before the company that he could not answer my challenge or determine the question at issue; and he made an unintelligible attempt to hide his perplexity. In order that the argument might proceed, I said to him, Well then Critias, if you like, let us assume that there is this science of science; whether the assumption is right or wrong may hereafter be investigated. Admitting the existence of it, will you tell me how such a science enables us to distinguish what we know or do not know, which, as we were saying, is self-knowledge or wisdom: so we were saying?

\par  Yes, Socrates, he said; and that I think is certainly true: for he who has this science or knowledge which knows itself will become like the knowledge which he has, in the same way that he who has swiftness will be swift, and he who has beauty will be beautiful, and he who has knowledge will know. In the same way he who has that knowledge which is self-knowing, will know himself.

\par  I do not doubt, I said, that a man will know himself, when he possesses that which has self-knowledge: but what necessity is there that, having this, he should know what he knows and what he does not know?

\par  Because, Socrates, they are the same.

\par  Very likely, I said; but I remain as stupid as ever; for still I fail to comprehend how this knowing what you know and do not know is the same as the knowledge of self.

\par  What do you mean? he said.

\par  This is what I mean, I replied: I will admit that there is a science of science;—can this do more than determine that of two things one is and the other is not science or knowledge?

\par  No, just that.

\par  But is knowledge or want of knowledge of health the same as knowledge or want of knowledge of justice?

\par  Certainly not.

\par  The one is medicine, and the other is politics; whereas that of which we are speaking is knowledge pure and simple.

\par  Very true.

\par  And if a man knows only, and has only knowledge of knowledge, and has no further knowledge of health and justice, the probability is that he will only know that he knows something, and has a certain knowledge, whether concerning himself or other men.

\par  True.

\par  Then how will this knowledge or science teach him to know what he knows? Say that he knows health;—not wisdom or temperance, but the art of medicine has taught it to him;—and he has learned harmony from the art of music, and building from the art of building,—neither, from wisdom or temperance: and the same of other things.

\par  That is evident.

\par  How will wisdom, regarded only as a knowledge of knowledge or science of science, ever teach him that he knows health, or that he knows building?

\par  It is impossible.

\par  Then he who is ignorant of these things will only know that he knows, but not what he knows?

\par  True.

\par  Then wisdom or being wise appears to be not the knowledge of the things which we do or do not know, but only the knowledge that we know or do not know?

\par  That is the inference.

\par  Then he who has this knowledge will not be able to examine whether a pretender knows or does not know that which he says that he knows: he will only know that he has a knowledge of some kind; but wisdom will not show him of what the knowledge is?

\par  Plainly not.

\par  Neither will he be able to distinguish the pretender in medicine from the true physician, nor between any other true and false professor of knowledge. Let us consider the matter in this way: If the wise man or any other man wants to distinguish the true physician from the false, how will he proceed? He will not talk to him about medicine; and that, as we were saying, is the only thing which the physician understands.

\par  True.

\par  And, on the other hand, the physician knows nothing of science, for this has been assumed to be the province of wisdom.

\par  True.

\par  And further, since medicine is science, we must infer that he does not know anything of medicine.

\par  Exactly.

\par  Then the wise man may indeed know that the physician has some kind of science or knowledge; but when he wants to discover the nature of this he will ask, What is the subject-matter? For the several sciences are distinguished not by the mere fact that they are sciences, but by the nature of their subjects. Is not that true?

\par  Quite true.

\par  And medicine is distinguished from other sciences as having the subject-matter of health and disease?

\par  Yes.

\par  And he who would enquire into the nature of medicine must pursue the enquiry into health and disease, and not into what is extraneous?

\par  True.

\par  And he who judges rightly will judge of the physician as a physician in what relates to these?

\par  He will.

\par  He will consider whether what he says is true, and whether what he does is right, in relation to health and disease?

\par  He will.

\par  But can any one attain the knowledge of either unless he have a knowledge of medicine?

\par  He cannot.

\par  No one at all, it would seem, except the physician can have this knowledge; and therefore not the wise man; he would have to be a physician as well as a wise man.

\par  Very true.

\par  Then, assuredly, wisdom or temperance, if only a science of science, and of the absence of science or knowledge, will not be able to distinguish the physician who knows from one who does not know but pretends or thinks that he knows, or any other professor of anything at all; like any other artist, he will only know his fellow in art or wisdom, and no one else.

\par  That is evident, he said.

\par  But then what profit, Critias, I said, is there any longer in wisdom or temperance which yet remains, if this is wisdom? If, indeed, as we were supposing at first, the wise man had been able to distinguish what he knew and did not know, and that he knew the one and did not know the other, and to recognize a similar faculty of discernment in others, there would certainly have been a great advantage in being wise; for then we should never have made a mistake, but have passed through life the unerring guides of ourselves and of those who are under us; and we should not have attempted to do what we did not know, but we should have found out those who knew, and have handed the business over to them and trusted in them; nor should we have allowed those who were under us to do anything which they were not likely to do well; and they would be likely to do well just that of which they had knowledge; and the house or state which was ordered or administered under the guidance of wisdom, and everything else of which wisdom was the lord, would have been well ordered; for truth guiding, and error having been eliminated, in all their doings, men would have done well, and would have been happy. Was not this, Critias, what we spoke of as the great advantage of wisdom—to know what is known and what is unknown to us?

\par  Very true, he said.

\par  And now you perceive, I said, that no such science is to be found anywhere.

\par  I perceive, he said.

\par  May we assume then, I said, that wisdom, viewed in this new light merely as a knowledge of knowledge and ignorance, has this advantage:—that he who possesses such knowledge will more easily learn anything which he learns; and that everything will be clearer to him, because, in addition to the knowledge of individuals, he sees the science, and this also will better enable him to test the knowledge which others have of what he knows himself; whereas the enquirer who is without this knowledge may be supposed to have a feebler and weaker insight? Are not these, my friend, the real advantages which are to be gained from wisdom? And are not we looking and seeking after something more than is to be found in her?

\par  That is very likely, he said.

\par  That is very likely, I said; and very likely, too, we have been enquiring to no purpose; as I am led to infer, because I observe that if this is wisdom, some strange consequences would follow. Let us, if you please, assume the possibility of this science of sciences, and further admit and allow, as was originally suggested, that wisdom is the knowledge of what we know and do not know. Assuming all this, still, upon further consideration, I am doubtful, Critias, whether wisdom, such as this, would do us much good. For we were wrong, I think, in supposing, as we were saying just now, that such wisdom ordering the government of house or state would be a great benefit.

\par  How so? he said.

\par  Why, I said, we were far too ready to admit the great benefits which mankind would obtain from their severally doing the things which they knew, and committing the things of which they are ignorant to those who were better acquainted with them.

\par  Were we not right in making that admission?

\par  I think not.

\par  How very strange, Socrates!

\par  By the dog of Egypt, I said, there I agree with you; and I was thinking as much just now when I said that strange consequences would follow, and that I was afraid we were on the wrong track; for however ready we may be to admit that this is wisdom, I certainly cannot make out what good this sort of thing does to us.

\par  What do you mean? he said; I wish that you could make me understand what you mean.

\par  I dare say that what I am saying is nonsense, I replied; and yet if a man has any feeling of what is due to himself, he cannot let the thought which comes into his mind pass away unheeded and unexamined.

\par  I like that, he said.

\par  Hear, then, I said, my own dream; whether coming through the horn or the ivory gate, I cannot tell. The dream is this: Let us suppose that wisdom is such as we are now defining, and that she has absolute sway over us; then each action will be done according to the arts or sciences, and no one professing to be a pilot when he is not, or any physician or general, or any one else pretending to know matters of which he is ignorant, will deceive or elude us; our health will be improved; our safety at sea, and also in battle, will be assured; our coats and shoes, and all other instruments and implements will be skilfully made, because the workmen will be good and true. Aye, and if you please, you may suppose that prophecy, which is the knowledge of the future, will be under the control of wisdom, and that she will deter deceivers and set up the true prophets in their place as the revealers of the future. Now I quite agree that mankind, thus provided, would live and act according to knowledge, for wisdom would watch and prevent ignorance from intruding on us. But whether by acting according to knowledge we shall act well and be happy, my dear Critias,—this is a point which we have not yet been able to determine.

\par  Yet I think, he replied, that if you discard knowledge, you will hardly find the crown of happiness in anything else.

\par  But of what is this knowledge? I said. Just answer me that small question. Do you mean a knowledge of shoemaking?

\par  God forbid.

\par  Or of working in brass?

\par  Certainly not.

\par  Or in wool, or wood, or anything of that sort?

\par  No, I do not.

\par  Then, I said, we are giving up the doctrine that he who lives according to knowledge is happy, for these live according to knowledge, and yet they are not allowed by you to be happy; but I think that you mean to confine happiness to particular individuals who live according to knowledge, such for example as the prophet, who, as I was saying, knows the future. Is it of him you are speaking or of some one else?

\par  Yes, I mean him, but there are others as well.

\par  Yes, I said, some one who knows the past and present as well as the future, and is ignorant of nothing. Let us suppose that there is such a person, and if there is, you will allow that he is the most knowing of all living men.

\par  Certainly he is.

\par  Yet I should like to know one thing more: which of the different kinds of knowledge makes him happy? or do all equally make him happy?

\par  Not all equally, he replied.

\par  But which most tends to make him happy? the knowledge of what past, present, or future thing? May I infer this to be the knowledge of the game of draughts?

\par  Nonsense about the game of draughts.

\par  Or of computation?

\par  No.

\par  Or of health?

\par  That is nearer the truth, he said.

\par  And that knowledge which is nearest of all, I said, is the knowledge of what?

\par  The knowledge with which he discerns good and evil.

\par  Monster! I said; you have been carrying me round in a circle, and all this time hiding from me the fact that the life according to knowledge is not that which makes men act rightly and be happy, not even if knowledge include all the sciences, but one science only, that of good and evil. For, let me ask you, Critias, whether, if you take away this, medicine will not equally give health, and shoemaking equally produce shoes, and the art of the weaver clothes?—whether the art of the pilot will not equally save our lives at sea, and the art of the general in war?

\par  Quite so.

\par  And yet, my dear Critias, none of these things will be well or beneficially done, if the science of the good be wanting.

\par  True.

\par  But that science is not wisdom or temperance, but a science of human advantage; not a science of other sciences, or of ignorance, but of good and evil: and if this be of use, then wisdom or temperance will not be of use.

\par  And why, he replied, will not wisdom be of use? For, however much we assume that wisdom is a science of sciences, and has a sway over other sciences, surely she will have this particular science of the good under her control, and in this way will benefit us.

\par  And will wisdom give health? I said; is not this rather the effect of medicine? Or does wisdom do the work of any of the other arts,—do they not each of them do their own work? Have we not long ago asseverated that wisdom is only the knowledge of knowledge and of ignorance, and of nothing else?

\par  That is obvious.

\par  Then wisdom will not be the producer of health.

\par  Certainly not.

\par  The art of health is different.

\par  Yes, different.

\par  Nor does wisdom give advantage, my good friend; for that again we have just now been attributing to another art.

\par  Very true.

\par  How then can wisdom be advantageous, when giving no advantage?

\par  That, Socrates, is certainly inconceivable.

\par  You see then, Critias, that I was not far wrong in fearing that I could have no sound notion about wisdom; I was quite right in depreciating myself; for that which is admitted to be the best of all things would never have seemed to us useless, if I had been good for anything at an enquiry. But now I have been utterly defeated, and have failed to discover what that is to which the imposer of names gave this name of temperance or wisdom. And yet many more admissions were made by us than could be fairly granted; for we admitted that there was a science of science, although the argument said No, and protested against us; and we admitted further, that this science knew the works of the other sciences (although this too was denied by the argument), because we wanted to show that the wise man had knowledge of what he knew and did not know; also we nobly disregarded, and never even considered, the impossibility of a man knowing in a sort of way that which he does not know at all; for our assumption was, that he knows that which he does not know; than which nothing, as I think, can be more irrational. And yet, after finding us so easy and good-natured, the enquiry is still unable to discover the truth; but mocks us to a degree, and has gone out of its way to prove the inutility of that which we admitted only by a sort of supposition and fiction to be the true definition of temperance or wisdom: which result, as far as I am concerned, is not so much to be lamented, I said. But for your sake, Charmides, I am very sorry—that you, having such beauty and such wisdom and temperance of soul, should have no profit or good in life from your wisdom and temperance. And still more am I grieved about the charm which I learned with so much pain, and to so little profit, from the Thracian, for the sake of a thing which is nothing worth. I think indeed that there is a mistake, and that I must be a bad enquirer, for wisdom or temperance I believe to be really a great good; and happy are you, Charmides, if you certainly possess it. Wherefore examine yourself, and see whether you have this gift and can do without the charm; for if you can, I would rather advise you to regard me simply as a fool who is never able to reason out anything; and to rest assured that the more wise and temperate you are, the happier you will be.

\par  Charmides said: I am sure that I do not know, Socrates, whether I have or have not this gift of wisdom and temperance; for how can I know whether I have a thing, of which even you and Critias are, as you say, unable to discover the nature?—(not that I believe you.) And further, I am sure, Socrates, that I do need the charm, and as far as I am concerned, I shall be willing to be charmed by you daily, until you say that I have had enough.

\par  Very good, Charmides, said Critias; if you do this I shall have a proof of your temperance, that is, if you allow yourself to be charmed by Socrates, and never desert him at all.

\par  You may depend on my following and not deserting him, said Charmides: if you who are my guardian command me, I should be very wrong not to obey you.

\par  And I do command you, he said.

\par  Then I will do as you say, and begin this very day.

\par  You sirs, I said, what are you conspiring about?

\par  We are not conspiring, said Charmides, we have conspired already.

\par  And are you about to use violence, without even going through the forms of justice?

\par  Yes, I shall use violence, he replied, since he orders me; and therefore you had better consider well.

\par  But the time for consideration has passed, I said, when violence is employed; and you, when you are determined on anything, and in the mood of violence, are irresistible.

\par  Do not you resist me then, he said.

\par  I will not resist you, I replied.

\par 
 
\end{document}