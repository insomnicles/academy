
\documentclass[11pt,letter]{article}


\begin{document}

\title{Statesman\thanks{Source: https://www.gutenberg.org/files/1738/1738-h/1738-h.htm. License: http://gutenberg.org/license ds}}
\date{\today}
\author{Plato, 427? BCE-347? BCE\\ Translated by Jowett, Benjamin, 1817-1893}
\maketitle

\setcounter{tocdepth}{1}
\tableofcontents
\renewcommand{\baselinestretch}{1.0}
\normalsize
\newpage

\section{
      INTRODUCTION AND ANALYSIS.
    }
\par  In the Phaedrus, the Republic, the Philebus, the Parmenides, and the Sophist, we may observe the tendency of Plato to combine two or more subjects or different aspects of the same subject in a single dialogue. In the Sophist and Statesman especially we note that the discussion is partly regarded as an illustration of method, and that analogies are brought from afar which throw light on the main subject. And in his later writings generally we further remark a decline of style, and of dramatic power; the characters excite little or no interest, and the digressions are apt to overlay the main thesis; there is not the 'callida junctura' of an artistic whole. Both the serious discussions and the jests are sometimes out of place. The invincible Socrates is withdrawn from view; and new foes begin to appear under old names. Plato is now chiefly concerned, not with the original Sophist, but with the sophistry of the schools of philosophy, which are making reasoning impossible; and is driven by them out of the regions of transcendental speculation back into the path of common sense. A logical or psychological phase takes the place of the doctrine of Ideas in his mind. He is constantly dwelling on the importance of regular classification, and of not putting words in the place of things. He has banished the poets, and is beginning to use a technical language. He is bitter and satirical, and seems to be sadly conscious of the realities of human life. Yet the ideal glory of the Platonic philosophy is not extinguished. He is still looking for a city in which kings are either philosophers or gods (compare Laws).

\par  The Statesman has lost the grace and beauty of the earlier dialogues. The mind of the writer seems to be so overpowered in the effort of thought as to impair his style; at least his gift of expression does not keep up with the increasing difficulty of his theme. The idea of the king or statesman and the illustration of method are connected, not like the love and rhetoric of the Phaedrus, by 'little invisible pegs,' but in a confused and inartistic manner, which fails to produce any impression of a whole on the mind of the reader. Plato apologizes for his tediousness, and acknowledges that the improvement of his audience has been his only aim in some of his digressions. His own image may be used as a motto of his style: like an inexpert statuary he has made the figure or outline too large, and is unable to give the proper colours or proportions to his work. He makes mistakes only to correct them—this seems to be his way of drawing attention to common dialectical errors. The Eleatic stranger, here, as in the Sophist, has no appropriate character, and appears only as the expositor of a political ideal, in the delineation of which he is frequently interrupted by purely logical illustrations. The younger Socrates resembles his namesake in nothing but a name. The dramatic character is so completely forgotten, that a special reference is twice made to discussions in the Sophist; and this, perhaps, is the strongest ground which can be urged for doubting the genuineness of the work. But, when we remember that a similar allusion is made in the Laws to the Republic, we see that the entire disregard of dramatic propriety is not always a sufficient reason for doubting the genuineness of a Platonic writing.

\par  The search after the Statesman, which is carried on, like that for the Sophist, by the method of dichotomy, gives an opportunity for many humorous and satirical remarks. Several of the jests are mannered and laboured: for example, the turn of words with which the dialogue opens; or the clumsy joke about man being an animal, who has a power of two-feet—both which are suggested by the presence of Theodorus, the geometrician. There is political as well as logical insight in refusing to admit the division of mankind into Hellenes and Barbarians: 'if a crane could speak, he would in like manner oppose men and all other animals to cranes.' The pride of the Hellene is further humbled, by being compared to a Phrygian or Lydian. Plato glories in this impartiality of the dialectical method, which places birds in juxtaposition with men, and the king side by side with the bird-catcher; king or vermin-destroyer are objects of equal interest to science (compare Parmen.). There are other passages which show that the irony of Socrates was a lesson which Plato was not slow in learning—as, for example, the passing remark, that 'the kings and statesmen of our day are in their breeding and education very like their subjects;' or the anticipation that the rivals of the king will be found in the class of servants; or the imposing attitude of the priests, who are the established interpreters of the will of heaven, authorized by law. Nothing is more bitter in all his writings than his comparison of the contemporary politicians to lions, centaurs, satyrs, and other animals of a feebler sort, who are ever changing their forms and natures. But, as in the later dialogues generally, the play of humour and the charm of poetry have departed, never to return.

\par  Still the Politicus contains a higher and more ideal conception of politics than any other of Plato's writings. The city of which there is a pattern in heaven (Republic), is here described as a Paradisiacal state of human society. In the truest sense of all, the ruler is not man but God; and such a government existed in a former cycle of human history, and may again exist when the gods resume their care of mankind. In a secondary sense, the true form of government is that which has scientific rulers, who are irresponsible to their subjects. Not power but knowledge is the characteristic of a king or royal person. And the rule of a man is better and higher than law, because he is more able to deal with the infinite complexity of human affairs. But mankind, in despair of finding a true ruler, are willing to acquiesce in any law or custom which will save them from the caprice of individuals. They are ready to accept any of the six forms of government which prevail in the world. To the Greek, nomos was a sacred word, but the political idealism of Plato soars into a region beyond; for the laws he would substitute the intelligent will of the legislator. Education is originally to implant in men's minds a sense of truth and justice, which is the divine bond of states, and the legislator is to contrive human bonds, by which dissimilar natures may be united in marriage and supply the deficiencies of one another. As in the Republic, the government of philosophers, the causes of the perversion of states, the regulation of marriages, are still the political problems with which Plato's mind is occupied. He treats them more slightly, partly because the dialogue is shorter, and also because the discussion of them is perpetually crossed by the other interest of dialectic, which has begun to absorb him.

\par  The plan of the Politicus or Statesman may be briefly sketched as follows: (1) By a process of division and subdivision we discover the true herdsman or king of men. But before we can rightly distinguish him from his rivals, we must view him, (2) as he is presented to us in a famous ancient tale: the tale will also enable us to distinguish the divine from the human herdsman or shepherd: (3) and besides our fable, we must have an example; for our example we will select the art of weaving, which will have to be distinguished from the kindred arts; and then, following this pattern, we will separate the king from his subordinates or competitors. (4) But are we not exceeding all due limits; and is there not a measure of all arts and sciences, to which the art of discourse must conform? There is; but before we can apply this measure, we must know what is the aim of discourse: and our discourse only aims at the dialectical improvement of ourselves and others.—Having made our apology, we return once more to the king or statesman, and proceed to contrast him with pretenders in the same line with him, under their various forms of government. (5) His characteristic is, that he alone has science, which is superior to law and written enactments; these do but spring out of the necessities of mankind, when they are in despair of finding the true king. (6) The sciences which are most akin to the royal are the sciences of the general, the judge, the orator, which minister to him, but even these are subordinate to him. (7) Fixed principles are implanted by education, and the king or statesman completes the political web by marrying together dissimilar natures, the courageous and the temperate, the bold and the gentle, who are the warp and the woof of society.

\par  The outline may be filled up as follows:—

\par \textbf{SOCRATES}
\par   I have reason to thank you, Theodorus, for the acquaintance of Theaetetus and the Stranger.

\par \textbf{THEODORUS}
\par   And you will have three times as much reason to thank me when they have delineated the Statesman and Philosopher, as well as the Sophist.

\par \textbf{SOCRATES}
\par   Does the great geometrician apply the same measure to all three? Are they not divided by an interval which no geometrical ratio can express?

\par \textbf{THEODORUS}
\par   By the god Ammon, Socrates, you are right; and I am glad to see that you have not forgotten your geometry. But before I retaliate on you, I must request the Stranger to finish the argument...

\par  The Stranger suggests that Theaetetus shall be allowed to rest, and that Socrates the younger shall respond in his place; Theodorus agrees to the suggestion, and Socrates remarks that the name of the one and the face of the other give him a right to claim relationship with both of them. They propose to take the Statesman after the Sophist; his path they must determine, and part off all other ways, stamping upon them a single negative form (compare Soph. ).

\par  The Stranger begins the enquiry by making a division of the arts and sciences into theoretical and practical—the one kind concerned with knowledge exclusively, and the other with action; arithmetic and the mathematical sciences are examples of the former, and carpentering and handicraft arts of the latter (compare Philebus). Under which of the two shall we place the Statesman? Or rather, shall we not first ask, whether the king, statesman, master, householder, practise one art or many? As the adviser of a physician may be said to have medical science and to be a physician, so the adviser of a king has royal science and is a king. And the master of a large household may be compared to the ruler of a small state. Hence we conclude that the science of the king, statesman, and householder is one and the same. And this science is akin to knowledge rather than to action. For a king rules with his mind, and not with his hands.

\par  But theoretical science may be a science either of judging, like arithmetic, or of ruling and superintending, like that of the architect or master-builder. And the science of the king is of the latter nature; but the power which he exercises is underived and uncontrolled,—a characteristic which distinguishes him from heralds, prophets, and other inferior officers. He is the wholesale dealer in command, and the herald, or other officer, retails his commands to others. Again, a ruler is concerned with the production of some object, and objects may be divided into living and lifeless, and rulers into the rulers of living and lifeless objects. And the king is not like the master-builder, concerned with lifeless matter, but has the task of managing living animals. And the tending of living animals may be either a tending of individuals, or a managing of herds. And the Statesman is not a groom, but a herdsman, and his art may be called either the art of managing a herd, or the art of collective management:—Which do you prefer? 'No matter.' Very good, Socrates, and if you are not too particular about words you will be all the richer some day in true wisdom. But how would you subdivide the herdsman's art? 'I should say, that there is one management of men, and another of beasts.' Very good, but you are in too great a hurry to get to man. All divisions which are rightly made should cut through the middle; if you attend to this rule, you will be more likely to arrive at classes. 'I do not understand the nature of my mistake.' Your division was like a division of the human race into Hellenes and Barbarians, or into Lydians or Phrygians and all other nations, instead of into male and female; or like a division of number into ten thousand and all other numbers, instead of into odd and even. And I should like you to observe further, that though I maintain a class to be a part, there is no similar necessity for a part to be a class. But to return to your division, you spoke of men and other animals as two classes—the second of which you comprehended under the general name of beasts. This is the sort of division which an intelligent crane would make: he would put cranes into a class by themselves for their special glory, and jumble together all others, including man, in the class of beasts. An error of this kind can only be avoided by a more regular subdivision. Just now we divided the whole class of animals into gregarious and non-gregarious, omitting the previous division into tame and wild. We forgot this in our hurry to arrive at man, and found by experience, as the proverb says, that 'the more haste the worse speed.'

\par  And now let us begin again at the art of managing herds. You have probably heard of the fish-preserves in the Nile and in the ponds of the Great King, and of the nurseries of geese and cranes in Thessaly. These suggest a new division into the rearing or management of land-herds and of water-herds:—I need not say with which the king is concerned. And land-herds may be divided into walking and flying; and every idiot knows that the political animal is a pedestrian. At this point we may take a longer or a shorter road, and as we are already near the end, I see no harm in taking the longer, which is the way of mesotomy, and accords with the principle which we were laying down. The tame, walking, herding animal, may be divided into two classes—the horned and the hornless, and the king is concerned with the hornless; and these again may be subdivided into animals having or not having cloven feet, or mixing or not mixing the breed; and the king or statesman has the care of animals which have not cloven feet, and which do not mix the breed. And now, if we omit dogs, who can hardly be said to herd, I think that we have only two species left which remain undivided: and how are we to distinguish them? To geometricians, like you and Theaetetus, I can have no difficulty in explaining that man is a diameter, having a power of two feet; and the power of four-legged creatures, being the double of two feet, is the diameter of our diameter. There is another excellent jest which I spy in the two remaining species. Men and birds are both bipeds, and human beings are running a race with the airiest and freest of creation, in which they are far behind their competitors;—this is a great joke, and there is a still better in the juxtaposition of the bird-taker and the king, who may be seen scampering after them. For, as we remarked in discussing the Sophist, the dialectical method is no respecter of persons. But we might have proceeded, as I was saying, by another and a shorter road. In that case we should have begun by dividing land animals into bipeds and quadrupeds, and bipeds into winged and wingless; we should than have taken the Statesman and set him over the 'bipes implume,' and put the reins of government into his hands.

\par  Here let us sum up:—The science of pure knowledge had a part which was the science of command, and this had a part which was a science of wholesale command; and this was divided into the management of animals, and was again parted off into the management of herds of animals, and again of land animals, and these into hornless, and these into bipeds; and so at last we arrived at man, and found the political and royal science. And yet we have not clearly distinguished the political shepherd from his rivals. No one would think of usurping the prerogatives of the ordinary shepherd, who on all hands is admitted to be the trainer, matchmaker, doctor, musician of his flock. But the royal shepherd has numberless competitors, from whom he must be distinguished; there are merchants, husbandmen, physicians, who will all dispute his right to manage the flock. I think that we can best distinguish him by having recourse to a famous old tradition, which may amuse as well as instruct us; the narrative is perfectly true, although the scepticism of mankind is prone to doubt the tales of old. You have heard what happened in the quarrel of Atreus and Thyestes? 'You mean about the golden lamb?' No, not that; but another part of the story, which tells how the sun and stars once arose in the west and set in the east, and that the god reversed their motion, as a witness to the right of Atreus. 'There is such a story.' And no doubt you have heard of the empire of Cronos, and of the earthborn men? The origin of these and the like stories is to be found in the tale which I am about to narrate.

\par  There was a time when God directed the revolutions of the world, but at the completion of a certain cycle he let go; and the world, by a necessity of its nature, turned back, and went round the other way. For divine things alone are unchangeable; but the earth and heavens, although endowed with many glories, have a body, and are therefore liable to perturbation. In the case of the world, the perturbation is very slight, and amounts only to a reversal of motion. For the lord of moving things is alone self-moved; neither can piety allow that he goes at one time in one direction and at another time in another; or that God has given the universe opposite motions; or that there are two gods, one turning it in one direction, another in another. But the truth is, that there are two cycles of the world, and in one of them it is governed by an immediate Providence, and receives life and immortality, and in the other is let go again, and has a reverse action during infinite ages. This new action is spontaneous, and is due to exquisite perfection of balance, to the vast size of the universe, and to the smallness of the pivot upon which it turns. All changes in the heaven affect the animal world, and this being the greatest of them, is most destructive to men and animals. At the beginning of the cycle before our own very few of them had survived; and on these a mighty change passed. For their life was reversed like the motion of the world, and first of all coming to a stand then quickly returned to youth and beauty. The white locks of the aged became black; the cheeks of the bearded man were restored to their youth and fineness; the young men grew softer and smaller, and, being reduced to the condition of children in mind as well as body, began to vanish away; and the bodies of those who had died by violence, in a few moments underwent a parallel change and disappeared. In that cycle of existence there was no such thing as the procreation of animals from one another, but they were born of the earth, and of this our ancestors, who came into being immediately after the end of the last cycle and at the beginning of this, have preserved the recollection. Such traditions are often now unduly discredited, and yet they may be proved by internal evidence. For observe how consistent the narrative is; as the old returned to youth, so the dead returned to life; the wheel of their existence having been reversed, they rose again from the earth: a few only were reserved by God for another destiny. Such was the origin of the earthborn men.

\par  'And is this cycle, of which you are speaking, the reign of Cronos, or our present state of existence?' No, Socrates, that blessed and spontaneous life belongs not to this, but to the previous state, in which God was the governor of the whole world, and other gods subject to him ruled over parts of the world, as is still the case in certain places. They were shepherds of men and animals, each of them sufficing for those of whom he had the care. And there was no violence among them, or war, or devouring of one another. Their life was spontaneous, because in those days God ruled over man; and he was to man what man is now to the animals. Under his government there were no estates, or private possessions, or families; but the earth produced a sufficiency of all things, and men were born out of the earth, having no traditions of the past; and as the temperature of the seasons was mild, they took no thought for raiment, and had no beds, but lived and dwelt in the open air.

\par  Such was the age of Cronos, and the age of Zeus is our own. Tell me, which is the happier of the two? Or rather, shall I tell you that the happiness of these children of Cronos must have depended on how they used their time? If having boundless leisure, and the power of discoursing not only with one another but with the animals, they had employed these advantages with a view to philosophy, gathering from every nature some addition to their store of knowledge;—or again, if they had merely eaten and drunk, and told stories to one another, and to the beasts;—in either case, I say, there would be no difficulty in answering the question. But as nobody knows which they did, the question must remain unanswered. And here is the point of my tale. In the fulness of time, when the earthborn men had all passed away, the ruler of the universe let go the helm, and became a spectator; and destiny and natural impulse swayed the world. At the same instant all the inferior deities gave up their hold; the whole universe rebounded, and there was a great earthquake, and utter ruin of all manner of animals. After a while the tumult ceased, and the universal creature settled down in his accustomed course, having authority over all other creatures, and following the instructions of his God and Father, at first more precisely, afterwards with less exactness. The reason of the falling off was the disengagement of a former chaos; 'a muddy vesture of decay' was a part of his original nature, out of which he was brought by his Creator, under whose immediate guidance, while he remained in that former cycle, the evil was minimized and the good increased to the utmost. And in the beginning of the new cycle all was well enough, but as time went on, discord entered in; at length the good was minimized and the evil everywhere diffused, and there was a danger of universal ruin. Then the Creator, seeing the world in great straits, and fearing that chaos and infinity would come again, in his tender care again placed himself at the helm and restored order, and made the world immortal and imperishable. Once more the cycle of life and generation was reversed; the infants grew into young men, and the young men became greyheaded; no longer did the animals spring out of the earth; as the whole world was now lord of its own progress, so the parts were to be self-created and self-nourished. At first the case of men was very helpless and pitiable; for they were alone among the wild beasts, and had to carry on the struggle for existence without arts or knowledge, and had no food, and did not know how to get any. That was the time when Prometheus brought them fire, Hephaestus and Athene taught them arts, and other gods gave them seeds and plants. Out of these human life was framed; for mankind were left to themselves, and ordered their own ways, living, like the universe, in one cycle after one manner, and in another cycle after another manner.

\par  Enough of the myth, which may show us two errors of which we were guilty in our account of the king. The first and grand error was in choosing for our king a god, who belongs to the other cycle, instead of a man from our own; there was a lesser error also in our failure to define the nature of the royal functions. The myth gave us only the image of a divine shepherd, whereas the statesmen and kings of our own day very much resemble their subjects in education and breeding. On retracing our steps we find that we gave too narrow a designation to the art which was concerned with command-for-self over living creatures, when we called it the 'feeding' of animals in flocks. This would apply to all shepherds, with the exception of the Statesman; but if we say 'managing' or 'tending' animals, the term would include him as well. Having remodelled the name, we may subdivide as before, first separating the human from the divine shepherd or manager. Then we may subdivide the human art of governing into the government of willing and unwilling subjects—royalty and tyranny—which are the extreme opposites of one another, although we in our simplicity have hitherto confounded them.

\par  And yet the figure of the king is still defective. We have taken up a lump of fable, and have used more than we needed. Like statuaries, we have made some of the features out of proportion, and shall lose time in reducing them. Or our mythus may be compared to a picture, which is well drawn in outline, but is not yet enlivened by colour. And to intelligent persons language is, or ought to be, a better instrument of description than any picture. 'But what, Stranger, is the deficiency of which you speak?' No higher truth can be made clear without an example; every man seems to know all things in a dream, and to know nothing when he is awake. And the nature of example can only be illustrated by an example. Children are taught to read by being made to compare cases in which they do not know a certain letter with cases in which they know it, until they learn to recognize it in all its combinations. Example comes into use when we identify something unknown with that which is known, and form a common notion of both of them. Like the child who is learning his letters, the soul recognizes some of the first elements of things; and then again is at fault and unable to recognize them when they are translated into the difficult language of facts. Let us, then, take an example, which will illustrate the nature of example, and will also assist us in characterizing the political science, and in separating the true king from his rivals.

\par  I will select the example of weaving, or, more precisely, weaving of wool. In the first place, all possessions are either productive or preventive; of the preventive sort are spells and antidotes, divine and human, and also defences, and defences are either arms or screens, and screens are veils and also shields against heat and cold, and shields against heat and cold are shelters and coverings, and coverings are blankets or garments, and garments are in one piece or have many parts; and of these latter, some are stitched and others are fastened, and of these again some are made of fibres of plants and some of hair, and of these some are cemented with water and earth, and some are fastened with their own material; the latter are called clothes, and are made by the art of clothing, from which the art of weaving differs only in name, as the political differs from the royal science. Thus we have drawn several distinctions, but as yet have not distinguished the weaving of garments from the kindred and co-operative arts. For the first process to which the material is subjected is the opposite of weaving—I mean carding. And the art of carding, and the whole art of the fuller and the mender, are concerned with the treatment and production of clothes, as well as the art of weaving. Again, there are the arts which make the weaver's tools. And if we say that the weaver's art is the greatest and noblest of those which have to do with woollen garments,—this, although true, is not sufficiently distinct; because these other arts require to be first cleared away. Let us proceed, then, by regular steps:—There are causal or principal, and co-operative or subordinate arts. To the causal class belong the arts of washing and mending, of carding and spinning the threads, and the other arts of working in wool; these are chiefly of two kinds, falling under the two great categories of composition and division. Carding is of the latter sort. But our concern is chiefly with that part of the art of wool-working which composes, and of which one kind twists and the other interlaces the threads, whether the firmer texture of the warp or the looser texture of the woof. These are adapted to each other, and the orderly composition of them forms a woollen garment. And the art which presides over these operations is the art of weaving.

\par  But why did we go through this circuitous process, instead of saying at once that weaving is the art of entwining the warp and the woof? In order that our labour may not seem to be lost, I must explain the whole nature of excess and defect. There are two arts of measuring—one is concerned with relative size, and the other has reference to a mean or standard of what is meet. The difference between good and evil is the difference between a mean or measure and excess or defect. All things require to be compared, not only with one another, but with the mean, without which there would be no beauty and no art, whether the art of the statesman or the art of weaving or any other; for all the arts guard against excess or defect, which are real evils. This we must endeavour to show, if the arts are to exist; and the proof of this will be a harder piece of work than the demonstration of the existence of not-being which we proved in our discussion about the Sophist. At present I am content with the indirect proof that the existence of such a standard is necessary to the existence of the arts. The standard or measure, which we are now only applying to the arts, may be some day required with a view to the demonstration of absolute truth.

\par  We may now divide this art of measurement into two parts; placing in the one part all the arts which measure the relative size or number of objects, and in the other all those which depend upon a mean or standard. Many accomplished men say that the art of measurement has to do with all things, but these persons, although in this notion of theirs they may very likely be right, are apt to fail in seeing the differences of classes—they jumble together in one the 'more' and the 'too much,' which are very different things. Whereas the right way is to find the differences of classes, and to comprehend the things which have any affinity under the same class.

\par  I will make one more observation by the way. When a pupil at a school is asked the letters which make up a particular word, is he not asked with a view to his knowing the same letters in all words? And our enquiry about the Statesman in like manner is intended not only to improve our knowledge of politics, but our reasoning powers generally. Still less would any one analyze the nature of weaving for its own sake. There is no difficulty in exhibiting sensible images, but the greatest and noblest truths have no outward form adapted to the eye of sense, and are only revealed in thought. And all that we are now saying is said for the sake of them. I make these remarks, because I want you to get rid of any impression that our discussion about weaving and about the reversal of the universe, and the other discussion about the Sophist and not-being, were tedious and irrelevant. Please to observe that they can only be fairly judged when compared with what is meet; and yet not with what is meet for producing pleasure, nor even meet for making discoveries, but for the great end of developing the dialectical method and sharpening the wits of the auditors. He who censures us, should prove that, if our words had been fewer, they would have been better calculated to make men dialecticians.

\par  And now let us return to our king or statesman, and transfer to him the example of weaving. The royal art has been separated from that of other herdsmen, but not from the causal and co-operative arts which exist in states; these do not admit of dichotomy, and therefore they must be carved neatly, like the limbs of a victim, not into more parts than are necessary. And first (1) we have the large class of instruments, which includes almost everything in the world; from these may be parted off (2) vessels which are framed for the preservation of things, moist or dry, prepared in the fire or out of the fire. The royal or political art has nothing to do with either of these, any more than with the arts of making (3) vehicles, or (4) defences, whether dresses, or arms, or walls, or (5) with the art of making ornaments, whether pictures or other playthings, as they may be fitly called, for they have no serious use. Then (6) there are the arts which furnish gold, silver, wood, bark, and other materials, which should have been put first; these, again, have no concern with the kingly science; any more than the arts (7) which provide food and nourishment for the human body, and which furnish occupation to the husbandman, huntsman, doctor, cook, and the like, but not to the king or statesman. Further, there are small things, such as coins, seals, stamps, which may with a little violence be comprehended in one of the above-mentioned classes. Thus they will embrace every species of property with the exception of animals,—but these have been already included in the art of tending herds. There remains only the class of slaves or ministers, among whom I expect that the real rivals of the king will be discovered. I am not speaking of the veritable slave bought with money, nor of the hireling who lets himself out for service, nor of the trader or merchant, who at best can only lay claim to economical and not to royal science. Nor am I referring to government officials, such as heralds and scribes, for these are only the servants of the rulers, and not the rulers themselves. I admit that there may be something strange in any servants pretending to be masters, but I hardly think that I could have been wrong in supposing that the principal claimants to the throne will be of this class. Let us try once more: There are diviners and priests, who are full of pride and prerogative; these, as the law declares, know how to give acceptable gifts to the gods, and in many parts of Hellas the duty of performing solemn sacrifices is assigned to the chief magistrate, as at Athens to the King Archon. At last, then, we have found a trace of those whom we were seeking. But still they are only servants and ministers.

\par  And who are these who next come into view in various forms of men and animals and other monsters appearing—lions and centaurs and satyrs—who are these? I did not know them at first, for every one looks strange when he is unexpected. But now I recognize the politician and his troop, the chief of Sophists, the prince of charlatans, the most accomplished of wizards, who must be carefully distinguished from the true king or statesman. And here I will interpose a question: What are the true forms of government? Are they not three—monarchy, oligarchy, and democracy? and the distinctions of freedom and compulsion, law and no law, poverty and riches expand these three into six. Monarchy may be divided into royalty and tyranny; oligarchy into aristocracy and plutocracy; and democracy may observe the law or may not observe it. But are any of these governments worthy of the name? Is not government a science, and are we to suppose that scientific government is secured by the rulers being many or few, rich or poor, or by the rule being compulsory or voluntary? Can the many attain to science? In no Hellenic city are there fifty good draught players, and certainly there are not as many kings, for by kings we mean all those who are possessed of the political science. A true government must therefore be the government of one, or of a few. And they may govern us either with or without law, and whether they are poor or rich, and however they govern, provided they govern on some scientific principle,—it makes no difference. And as the physician may cure us with our will, or against our will, and by any mode of treatment, burning, bleeding, lowering, fattening, if he only proceeds scientifically: so the true governor may reduce or fatten or bleed the body corporate, while he acts according to the rules of his art, and with a view to the good of the state, whether according to law or without law.

\par  'I do not like the notion, that there can be good government without law.'

\par  I must explain: Law-making certainly is the business of a king; and yet the best thing of all is, not that the law should rule, but that the king should rule, for the varieties of circumstances are endless, and no simple or universal rule can suit them all, or last for ever. The law is just an ignorant brute of a tyrant, who insists always on his commands being fulfilled under all circumstances. 'Then why have we laws at all?' I will answer that question by asking you whether the training master gives a different discipline to each of his pupils, or whether he has a general rule of diet and exercise which is suited to the constitutions of the majority? 'The latter.' The legislator, too, is obliged to lay down general laws, and cannot enact what is precisely suitable to each particular case. He cannot be sitting at every man's side all his life, and prescribe for him the minute particulars of his duty, and therefore he is compelled to impose on himself and others the restriction of a written law. Let me suppose now, that a physician or trainer, having left directions for his patients or pupils, goes into a far country, and comes back sooner than he intended; owing to some unexpected change in the weather, the patient or pupil seems to require a different mode of treatment: Would he persist in his old commands, under the idea that all others are noxious and heterodox? Viewed in the light of science, would not the continuance of such regulations be ridiculous? And if the legislator, or another like him, comes back from a far country, is he to be prohibited from altering his own laws? The common people say: Let a man persuade the city first, and then let him impose new laws. But is a physician only to cure his patients by persuasion, and not by force? Is he a worse physician who uses a little gentle violence in effecting the cure? Or shall we say, that the violence is just, if exercised by a rich man, and unjust, if by a poor man? May not any man, rich or poor, with or without law, and whether the citizens like or not, do what is for their good? The pilot saves the lives of the crew, not by laying down rules, but by making his art a law, and, like him, the true governor has a strength of art which is superior to the law. This is scientific government, and all others are imitations only. Yet no great number of persons can attain to this science. And hence follows an important result. The true political principle is to assert the inviolability of the law, which, though not the best thing possible, is best for the imperfect condition of man.

\par  I will explain my meaning by an illustration:—Suppose that mankind, indignant at the rogueries and caprices of physicians and pilots, call together an assembly, in which all who like may speak, the skilled as well as the unskilled, and that in their assembly they make decrees for regulating the practice of navigation and medicine which are to be binding on these professions for all time. Suppose that they elect annually by vote or lot those to whom authority in either department is to be delegated. And let us further imagine, that when the term of their magistracy has expired, the magistrates appointed by them are summoned before an ignorant and unprofessional court, and may be condemned and punished for breaking the regulations. They even go a step further, and enact, that he who is found enquiring into the truth of navigation and medicine, and is seeking to be wise above what is written, shall be called not an artist, but a dreamer, a prating Sophist and a corruptor of youth; and if he try to persuade others to investigate those sciences in a manner contrary to the law, he shall be punished with the utmost severity. And like rules might be extended to any art or science. But what would be the consequence?

\par  'The arts would utterly perish, and human life, which is bad enough already, would become intolerable.'

\par  But suppose, once more, that we were to appoint some one as the guardian of the law, who was both ignorant and interested, and who perverted the law: would not this be a still worse evil than the other? 'Certainly.' For the laws are based on some experience and wisdom. Hence the wiser course is, that they should be observed, although this is not the best thing of all, but only the second best. And whoever, having skill, should try to improve them, would act in the spirit of the law-giver. But then, as we have seen, no great number of men, whether poor or rich, can be makers of laws. And so, the nearest approach to true government is, when men do nothing contrary to their own written laws and national customs. When the rich preserve their customs and maintain the law, this is called aristocracy, or if they neglect the law, oligarchy. When an individual rules according to law, whether by the help of science or opinion, this is called monarchy; and when he has royal science he is a king, whether he be so in fact or not; but when he rules in spite of law, and is blind with ignorance and passion, he is called a tyrant. These forms of government exist, because men despair of the true king ever appearing among them; if he were to appear, they would joyfully hand over to him the reins of government. But, as there is no natural ruler of the hive, they meet together and make laws. And do we wonder, when the foundation of politics is in the letter only, at the miseries of states? Ought we not rather to admire the strength of the political bond? For cities have endured the worst of evils time out of mind; many cities have been shipwrecked, and some are like ships foundering, because their pilots are absolutely ignorant of the science which they profess.

\par  Let us next ask, which of these untrue forms of government is the least bad, and which of them is the worst? I said at the beginning, that each of the three forms of government, royalty, aristocracy, and democracy, might be divided into two, so that the whole number of them, including the best, will be seven. Under monarchy we have already distinguished royalty and tyranny; of oligarchy there were two kinds, aristocracy and plutocracy; and democracy may also be divided, for there is a democracy which observes, and a democracy which neglects, the laws. The government of one is the best and the worst—the government of a few is less bad and less good—the government of the many is the least bad and least good of them all, being the best of all lawless governments, and the worst of all lawful ones. But the rulers of all these states, unless they have knowledge, are maintainers of idols, and themselves idols—wizards, and also Sophists; for, after many windings, the term 'Sophist' comes home to them.

\par  And now enough of centaurs and satyrs: the play is ended, and they may quit the political stage. Still there remain some other and better elements, which adhere to the royal science, and must be drawn off in the refiner's fire before the gold can become quite pure. The arts of the general, the judge, and the orator, will have to be separated from the royal art; when the separation has been made, the nature of the king will be unalloyed. Now there are inferior sciences, such as music and others; and there is a superior science, which determines whether music is to be learnt or not, and this is different from them, and the governor of them. The science which determines whether we are to use persuasion, or not, is higher than the art of persuasion; the science which determines whether we are to go to war, is higher than the art of the general. The science which makes the laws, is higher than that which only administers them. And the science which has this authority over the rest, is the science of the king or statesman.

\par  Once more we will endeavour to view this royal science by the light of our example. We may compare the state to a web, and I will show you how the different threads are drawn into one. You would admit—would you not?—that there are parts of virtue (although this position is sometimes assailed by Eristics), and one part of virtue is temperance, and another courage. These are two principles which are in a manner antagonistic to one another; and they pervade all nature; the whole class of the good and beautiful is included under them. The beautiful may be subdivided into two lesser classes: one of these is described by us in terms expressive of motion or energy, and the other in terms expressive of rest and quietness. We say, how manly! how vigorous! how ready! and we say also, how calm! how temperate! how dignified! This opposition of terms is extended by us to all actions, to the tones of the voice, the notes of music, the workings of the mind, the characters of men. The two classes both have their exaggerations; and the exaggerations of the one are termed 'hardness,' 'violence,' 'madness;' of the other 'cowardliness,' or 'sluggishness.' And if we pursue the enquiry, we find that these opposite characters are naturally at variance, and can hardly be reconciled. In lesser matters the antagonism between them is ludicrous, but in the State may be the occasion of grave disorders, and may disturb the whole course of human life. For the orderly class are always wanting to be at peace, and hence they pass imperceptibly into the condition of slaves; and the courageous sort are always wanting to go to war, even when the odds are against them, and are soon destroyed by their enemies. But the true art of government, first preparing the material by education, weaves the two elements into one, maintaining authority over the carders of the wool, and selecting the proper subsidiary arts which are necessary for making the web. The royal science is queen of educators, and begins by choosing the natures which she is to train, punishing with death and exterminating those who are violently carried away to atheism and injustice, and enslaving those who are wallowing in the mire of ignorance. The rest of the citizens she blends into one, combining the stronger element of courage, which we may call the warp, with the softer element of temperance, which we may imagine to be the woof. These she binds together, first taking the eternal elements of the honourable, the good, and the just, and fastening them with a divine cord in a heaven-born nature, and then fastening the animal elements with a human cord. The good legislator can implant by education the higher principles; and where they exist there is no difficulty in inserting the lesser human bonds, by which the State is held together; these are the laws of intermarriage, and of union for the sake of offspring. Most persons in their marriages seek after wealth or power; or they are clannish, and choose those who are like themselves,—the temperate marrying the temperate, and the courageous the courageous. The two classes thrive and flourish at first, but they soon degenerate; the one become mad, and the other feeble and useless. This would not have been the case, if they had both originally held the same notions about the honourable and the good; for then they never would have allowed the temperate natures to be separated from the courageous, but they would have bound them together by common honours and reputations, by intermarriages, and by the choice of rulers who combine both qualities. The temperate are careful and just, but are wanting in the power of action; the courageous fall short of them in justice, but in action are superior to them: and no state can prosper in which either of these qualities is wanting. The noblest and best of all webs or states is that which the royal science weaves, combining the two sorts of natures in a single texture, and in this enfolding freeman and slave and every other social element, and presiding over them all.

\par  'Your picture, Stranger, of the king and statesman, no less than of the Sophist, is quite perfect.'

\par  ...

\par  The principal subjects in the Statesman may be conveniently embraced under six or seven heads:—(1) the myth; (2) the dialectical interest; (3) the political aspects of the dialogue; (4) the satirical and paradoxical vein; (5) the necessary imperfection of law; (6) the relation of the work to the other writings of Plato; lastly (7), we may briefly consider the genuineness of the Sophist and Statesman, which can hardly be assumed without proof, since the two dialogues have been questioned by three such eminent Platonic scholars as Socher, Schaarschmidt, and Ueberweg.

\par  I. The hand of the master is clearly visible in the myth. First in the connection with mythology;—he wins a kind of verisimilitude for this as for his other myths, by adopting received traditions, of which he pretends to find an explanation in his own larger conception (compare Introduction to Critias). The young Socrates has heard of the sun rising in the west and setting in the east, and of the earth-born men; but he has never heard the origin of these remarkable phenomena. Nor is Plato, here or elsewhere, wanting in denunciations of the incredulity of 'this latter age,' on which the lovers of the marvellous have always delighted to enlarge. And he is not without express testimony to the truth of his narrative;—such testimony as, in the Timaeus, the first men gave of the names of the gods ('They must surely have known their own ancestors'). For the first generation of the new cycle, who lived near the time, are supposed to have preserved a recollection of a previous one. He also appeals to internal evidence, viz. the perfect coherence of the tale, though he is very well aware, as he says in the Cratylus, that there may be consistency in error as well as in truth. The gravity and minuteness with which some particulars are related also lend an artful aid. The profound interest and ready assent of the young Socrates, who is not too old to be amused 'with a tale which a child would love to hear,' are a further assistance. To those who were naturally inclined to believe that the fortunes of mankind are influenced by the stars, or who maintained that some one principle, like the principle of the Same and the Other in the Timaeus, pervades all things in the world, the reversal of the motion of the heavens seemed necessarily to produce a reversal of the order of human life. The spheres of knowledge, which to us appear wide asunder as the poles, astronomy and medicine, were naturally connected in the minds of early thinkers, because there was little or nothing in the space between them. Thus there is a basis of philosophy, on which the improbabilities of the tale may be said to rest. These are some of the devices by which Plato, like a modern novelist, seeks to familiarize the marvellous.

\par  The myth, like that of the Timaeus and Critias, is rather historical than poetical, in this respect corresponding to the general change in the later writings of Plato, when compared with the earlier ones. It is hardly a myth in the sense in which the term might be applied to the myth of the Phaedrus, the Republic, the Phaedo, or the Gorgias, but may be more aptly compared with the didactic tale in which Protagoras describes the fortunes of primitive man, or with the description of the gradual rise of a new society in the Third Book of the Laws. Some discrepancies may be observed between the mythology of the Statesman and the Timaeus, and between the Timaeus and the Republic. But there is no reason to expect that all Plato's visions of a former, any more than of a future, state of existence, should conform exactly to the same pattern. We do not find perfect consistency in his philosophy; and still less have we any right to demand this of him in his use of mythology and figures of speech. And we observe that while employing all the resources of a writer of fiction to give credibility to his tales, he is not disposed to insist upon their literal truth. Rather, as in the Phaedo, he says, 'Something of the kind is true;' or, as in the Gorgias, 'This you will think to be an old wife's tale, but you can think of nothing truer;' or, as in the Statesman, he describes his work as a 'mass of mythology,' which was introduced in order to teach certain lessons; or, as in the Phaedrus, he secretly laughs at such stories while refusing to disturb the popular belief in them.

\par  The greater interest of the myth consists in the philosophical lessons which Plato presents to us in this veiled form. Here, as in the tale of Er, the son of Armenius, he touches upon the question of freedom and necessity, both in relation to God and nature. For at first the universe is governed by the immediate providence of God,—this is the golden age,—but after a while the wheel is reversed, and man is left to himself. Like other theologians and philosophers, Plato relegates his explanation of the problem to a transcendental world; he speaks of what in modern language might be termed 'impossibilities in the nature of things,' hindering God from continuing immanent in the world. But there is some inconsistency; for the 'letting go' is spoken of as a divine act, and is at the same time attributed to the necessary imperfection of matter; there is also a numerical necessity for the successive births of souls. At first, man and the world retain their divine instincts, but gradually degenerate. As in the Book of Genesis, the first fall of man is succeeded by a second; the misery and wickedness of the world increase continually. The reason of this further decline is supposed to be the disorganisation of matter: the latent seeds of a former chaos are disengaged, and envelope all things. The condition of man becomes more and more miserable; he is perpetually waging an unequal warfare with the beasts. At length he obtains such a measure of education and help as is necessary for his existence. Though deprived of God's help, he is not left wholly destitute; he has received from Athene and Hephaestus a knowledge of the arts; other gods give him seeds and plants; and out of these human life is reconstructed. He now eats bread in the sweat of his brow, and has dominion over the animals, subjected to the conditions of his nature, and yet able to cope with them by divine help. Thus Plato may be said to represent in a figure—(1) the state of innocence; (2) the fall of man; (3) the still deeper decline into barbarism; (4) the restoration of man by the partial interference of God, and the natural growth of the arts and of civilised society. Two lesser features of this description should not pass unnoticed:—(1) the primitive men are supposed to be created out of the earth, and not after the ordinary manner of human generation—half the causes of moral evil are in this way removed; (2) the arts are attributed to a divine revelation: and so the greatest difficulty in the history of pre-historic man is solved. Though no one knew better than Plato that the introduction of the gods is not a reason, but an excuse for not giving a reason (Cratylus), yet, considering that more than two thousand years later mankind are still discussing these problems, we may be satisfied to find in Plato a statement of the difficulties which arise in conceiving the relation of man to God and nature, without expecting to obtain from him a solution of them. In such a tale, as in the Phaedrus, various aspects of the Ideas were doubtless indicated to Plato's own mind, as the corresponding theological problems are to us. The immanence of things in the Ideas, or the partial separation of them, and the self-motion of the supreme Idea, are probably the forms in which he would have interpreted his own parable.

\par  He touches upon another question of great interest—the consciousness of evil—what in the Jewish Scriptures is called 'eating of the tree of the knowledge of good and evil.' At the end of the narrative, the Eleatic asks his companion whether this life of innocence, or that which men live at present, is the better of the two. He wants to distinguish between the mere animal life of innocence, the 'city of pigs,' as it is comically termed by Glaucon in the Republic, and the higher life of reason and philosophy. But as no one can determine the state of man in the world before the Fall, 'the question must remain unanswered.' Similar questions have occupied the minds of theologians in later ages; but they can hardly be said to have found an answer. Professor Campbell well observes, that the general spirit of the myth may be summed up in the words of the Lysis: 'If evil were to perish, should we hunger any more, or thirst any more, or have any similar sensations? Yet perhaps the question what will or will not be is a foolish one, for who can tell?' As in the Theaetetus, evil is supposed to continue,—here, as the consequence of a former state of the world, a sort of mephitic vapour exhaling from some ancient chaos,—there, as involved in the possibility of good, and incident to the mixed state of man.

\par  Once more—and this is the point of connexion with the rest of the dialogue—the myth is intended to bring out the difference between the ideal and the actual state of man. In all ages of the world men have dreamed of a state of perfection, which has been, and is to be, but never is, and seems to disappear under the necessary conditions of human society. The uselessness, the danger, the true value of such political ideals have often been discussed; youth is too ready to believe in them; age to disparage them. Plato's 'prudens quaestio' respecting the comparative happiness of men in this and in a former cycle of existence is intended to elicit this contrast between the golden age and 'the life under Zeus' which is our own. To confuse the divine and human, or hastily apply one to the other, is a 'tremendous error.' Of the ideal or divine government of the world we can form no true or adequate conception; and this our mixed state of life, in which we are partly left to ourselves, but not wholly deserted by the gods, may contain some higher elements of good and knowledge than could have existed in the days of innocence under the rule of Cronos. So we may venture slightly to enlarge a Platonic thought which admits of a further application to Christian theology. Here are suggested also the distinctions between God causing and permitting evil, and between his more and less immediate government of the world.

\par  II. The dialectical interest of the Statesman seems to contend in Plato's mind with the political; the dialogue might have been designated by two equally descriptive titles—either the 'Statesman,' or 'Concerning Method.' Dialectic, which in the earlier writings of Plato is a revival of the Socratic question and answer applied to definition, is now occupied with classification; there is nothing in which he takes greater delight than in processes of division (compare Phaedr. ); he pursues them to a length out of proportion to his main subject, and appears to value them as a dialectical exercise, and for their own sake. A poetical vision of some order or hierarchy of ideas or sciences has already been floating before us in the Symposium and the Republic. And in the Phaedrus this aspect of dialectic is further sketched out, and the art of rhetoric is based on the division of the characters of mankind into their several classes. The same love of divisions is apparent in the Gorgias. But in a well-known passage of the Philebus occurs the first criticism on the nature of classification. There we are exhorted not to fall into the common error of passing from unity to infinity, but to find the intermediate classes; and we are reminded that in any process of generalization, there may be more than one class to which individuals may be referred, and that we must carry on the process of division until we have arrived at the infima species.

\par  These precepts are not forgotten, either in the Sophist or in the Statesman. The Sophist contains four examples of division, carried on by regular steps, until in four different lines of descent we detect the Sophist. In the Statesman the king or statesman is discovered by a similar process; and we have a summary, probably made for the first time, of possessions appropriated by the labour of man, which are distributed into seven classes. We are warned against preferring the shorter to the longer method;—if we divide in the middle, we are most likely to light upon species; at the same time, the important remark is made, that 'a part is not to be confounded with a class.' Having discovered the genus under which the king falls, we proceed to distinguish him from the collateral species. To assist our imagination in making this separation, we require an example. The higher ideas, of which we have a dreamy knowledge, can only be represented by images taken from the external world. But, first of all, the nature of example is explained by an example. The child is taught to read by comparing the letters in words which he knows with the same letters in unknown combinations; and this is the sort of process which we are about to attempt. As a parallel to the king we select the worker in wool, and compare the art of weaving with the royal science, trying to separate either of them from the inferior classes to which they are akin. This has the incidental advantage, that weaving and the web furnish us with a figure of speech, which we can afterwards transfer to the State.

\par  There are two uses of examples or images—in the first place, they suggest thoughts—secondly, they give them a distinct form. In the infancy of philosophy, as in childhood, the language of pictures is natural to man: truth in the abstract is hardly won, and only by use familiarized to the mind. Examples are akin to analogies, and have a reflex influence on thought; they people the vacant mind, and may often originate new directions of enquiry. Plato seems to be conscious of the suggestiveness of imagery; the general analogy of the arts is constantly employed by him as well as the comparison of particular arts—weaving, the refining of gold, the learning to read, music, statuary, painting, medicine, the art of the pilot—all of which occur in this dialogue alone: though he is also aware that 'comparisons are slippery things,' and may often give a false clearness to ideas. We shall find, in the Philebus, a division of sciences into practical and speculative, and into more or less speculative: here we have the idea of master-arts, or sciences which control inferior ones. Besides the supreme science of dialectic, 'which will forget us, if we forget her,' another master-science for the first time appears in view—the science of government, which fixes the limits of all the rest. This conception of the political or royal science as, from another point of view, the science of sciences, which holds sway over the rest, is not originally found in Aristotle, but in Plato.

\par  The doctrine that virtue and art are in a mean, which is familiarized to us by the study of the Nicomachean Ethics, is also first distinctly asserted in the Statesman of Plato. The too much and the too little are in restless motion: they must be fixed by a mean, which is also a standard external to them. The art of measuring or finding a mean between excess and defect, like the principle of division in the Phaedrus, receives a particular application to the art of discourse. The excessive length of a discourse may be blamed; but who can say what is excess, unless he is furnished with a measure or standard? Measure is the life of the arts, and may some day be discovered to be the single ultimate principle in which all the sciences are contained. Other forms of thought may be noted—the distinction between causal and co-operative arts, which may be compared with the distinction between primary and co-operative causes in the Timaeus; or between cause and condition in the Phaedo; the passing mention of economical science; the opposition of rest and motion, which is found in all nature; the general conception of two great arts of composition and division, in which are contained weaving, politics, dialectic; and in connexion with the conception of a mean, the two arts of measuring.

\par  In the Theaetetus, Plato remarks that precision in the use of terms, though sometimes pedantic, is sometimes necessary. Here he makes the opposite reflection, that there may be a philosophical disregard of words. The evil of mere verbal oppositions, the requirement of an impossible accuracy in the use of terms, the error of supposing that philosophy was to be found in language, the danger of word-catching, have frequently been discussed by him in the previous dialogues, but nowhere has the spirit of modern inductive philosophy been more happily indicated than in the words of the Statesman:—'If you think more about things, and less about words, you will be richer in wisdom as you grow older.' A similar spirit is discernible in the remarkable expressions, 'the long and difficult language of facts;' and 'the interrogation of every nature, in order to obtain the particular contribution of each to the store of knowledge.' Who has described 'the feeble intelligence of all things; given by metaphysics better than the Eleatic Stranger in the words—'The higher ideas can hardly be set forth except through the medium of examples; every man seems to know all things in a kind of dream, and then again nothing when he is awake?' Or where is the value of metaphysical pursuits more truly expressed than in the words,—'The greatest and noblest things have no outward image of themselves visible to man: therefore we should learn to give a rational account of them?'

\par  III. The political aspects of the dialogue are closely connected with the dialectical. As in the Cratylus, the legislator has 'the dialectician standing on his right hand;' so in the Statesman, the king or statesman is the dialectician, who, although he may be in a private station, is still a king. Whether he has the power or not, is a mere accident; or rather he has the power, for what ought to be is ('Was ist vernunftig, das ist wirklich'); and he ought to be and is the true governor of mankind. There is a reflection in this idealism of the Socratic 'Virtue is knowledge;' and, without idealism, we may remark that knowledge is a great part of power. Plato does not trouble himself to construct a machinery by which 'philosophers shall be made kings,' as in the Republic: he merely holds up the ideal, and affirms that in some sense science is really supreme over human life.

\par  He is struck by the observation 'quam parva sapientia regitur mundus,' and is touched with a feeling of the ills which afflict states. The condition of Megara before and during the Peloponnesian War, of Athens under the Thirty and afterwards, of Syracuse and the other Sicilian cities in their alternations of democratic excess and tyranny, might naturally suggest such reflections. Some states he sees already shipwrecked, others foundering for want of a pilot; and he wonders not at their destruction, but at their endurance. For they ought to have perished long ago, if they had depended on the wisdom of their rulers. The mingled pathos and satire of this remark is characteristic of Plato's later style.

\par  The king is the personification of political science. And yet he is something more than this,—the perfectly good and wise tyrant of the Laws, whose will is better than any law. He is the special providence who is always interfering with and regulating all things. Such a conception has sometimes been entertained by modern theologians, and by Plato himself, of the Supreme Being. But whether applied to Divine or to human governors the conception is faulty for two reasons, neither of which are noticed by Plato:—first, because all good government supposes a degree of co-operation in the ruler and his subjects,—an 'education in politics' as well as in moral virtue; secondly, because government, whether Divine or human, implies that the subject has a previous knowledge of the rules under which he is living. There is a fallacy, too, in comparing unchangeable laws with a personal governor. For the law need not necessarily be an 'ignorant and brutal tyrant,' but gentle and humane, capable of being altered in the spirit of the legislator, and of being administered so as to meet the cases of individuals. Not only in fact, but in idea, both elements must remain—the fixed law and the living will; the written word and the spirit; the principles of obligation and of freedom; and their applications whether made by law or equity in particular cases.

\par  There are two sides from which positive laws may be attacked:—either from the side of nature, which rises up and rebels against them in the spirit of Callicles in the Gorgias; or from the side of idealism, which attempts to soar above them,—and this is the spirit of Plato in the Statesman. But he soon falls, like Icarus, and is content to walk instead of flying; that is, to accommodate himself to the actual state of human things. Mankind have long been in despair of finding the true ruler; and therefore are ready to acquiesce in any of the five or six received forms of government as better than none. And the best thing which they can do (though only the second best in reality), is to reduce the ideal state to the conditions of actual life. Thus in the Statesman, as in the Laws, we have three forms of government, which we may venture to term, (1) the ideal, (2) the practical, (3) the sophistical—what ought to be, what might be, what is. And thus Plato seems to stumble, almost by accident, on the notion of a constitutional monarchy, or of a monarchy ruling by laws.

\par  The divine foundations of a State are to be laid deep in education (Republic), and at the same time some little violence may be used in exterminating natures which are incapable of education (compare Laws). Plato is strongly of opinion that the legislator, like the physician, may do men good against their will (compare Gorgias). The human bonds of states are formed by the inter-marriage of dispositions adapted to supply the defects of each other. As in the Republic, Plato has observed that there are opposite natures in the world, the strong and the gentle, the courageous and the temperate, which, borrowing an expression derived from the image of weaving, he calls the warp and the woof of human society. To interlace these is the crowning achievement of political science. In the Protagoras, Socrates was maintaining that there was only one virtue, and not many: now Plato is inclined to think that there are not only parallel, but opposite virtues, and seems to see a similar opposition pervading all art and nature. But he is satisfied with laying down the principle, and does not inform us by what further steps the union of opposites is to be effected.

\par  In the loose framework of a single dialogue Plato has thus combined two distinct subjects—politics and method. Yet they are not so far apart as they appear: in his own mind there was a secret link of connexion between them. For the philosopher or dialectician is also the only true king or statesman. In the execution of his plan Plato has invented or distinguished several important forms of thought, and made incidentally many valuable remarks. Questions of interest both in ancient and modern politics also arise in the course of the dialogue, which may with advantage be further considered by us:—

\par  a. The imaginary ruler, whether God or man, is above the law, and is a law to himself and to others. Among the Greeks as among the Jews, law was a sacred name, the gift of God, the bond of states. But in the Statesman of Plato, as in the New Testament, the word has also become the symbol of an imperfect good, which is almost an evil. The law sacrifices the individual to the universal, and is the tyranny of the many over the few (compare Republic). It has fixed rules which are the props of order, and will not swerve or bend in extreme cases. It is the beginning of political society, but there is something higher—an intelligent ruler, whether God or man, who is able to adapt himself to the endless varieties of circumstances. Plato is fond of picturing the advantages which would result from the union of the tyrant who has power with the legislator who has wisdom: he regards this as the best and speediest way of reforming mankind. But institutions cannot thus be artificially created, nor can the external authority of a ruler impose laws for which a nation is unprepared. The greatest power, the highest wisdom, can only proceed one or two steps in advance of public opinion. In all stages of civilization human nature, after all our efforts, remains intractable,—not like clay in the hands of the potter, or marble under the chisel of the sculptor. Great changes occur in the history of nations, but they are brought about slowly, like the changes in the frame of nature, upon which the puny arm of man hardly makes an impression. And, speaking generally, the slowest growths, both in nature and in politics, are the most permanent.

\par  b. Whether the best form of the ideal is a person or a law may fairly be doubted. The former is more akin to us: it clothes itself in poetry and art, and appeals to reason more in the form of feeling: in the latter there is less danger of allowing ourselves to be deluded by a figure of speech. The ideal of the Greek state found an expression in the deification of law: the ancient Stoic spoke of a wise man perfect in virtue, who was fancifully said to be a king; but neither they nor Plato had arrived at the conception of a person who was also a law. Nor is it easy for the Christian to think of God as wisdom, truth, holiness, and also as the wise, true, and holy one. He is always wanting to break through the abstraction and interrupt the law, in order that he may present to himself the more familiar image of a divine friend. While the impersonal has too slender a hold upon the affections to be made the basis of religion, the conception of a person on the other hand tends to degenerate into a new kind of idolatry. Neither criticism nor experience allows us to suppose that there are interferences with the laws of nature; the idea is inconceivable to us and at variance with facts. The philosopher or theologian who could realize to mankind that a person is a law, that the higher rule has no exception, that goodness, like knowledge, is also power, would breathe a new religious life into the world.

\par  c. Besides the imaginary rule of a philosopher or a God, the actual forms of government have to be considered. In the infancy of political science, men naturally ask whether the rule of the many or of the few is to be preferred. If by 'the few' we mean 'the good' and by 'the many,' 'the bad,' there can be but one reply: 'The rule of one good man is better than the rule of all the rest, if they are bad.' For, as Heracleitus says, 'One is ten thousand if he be the best.' If, however, we mean by the rule of the few the rule of a class neither better nor worse than other classes, not devoid of a feeling of right, but guided mostly by a sense of their own interests, and by the rule of the many the rule of all classes, similarly under the influence of mixed motives, no one would hesitate to answer—'The rule of all rather than one, because all classes are more likely to take care of all than one of another; and the government has greater power and stability when resting on a wider basis.' Both in ancient and modern times the best balanced form of government has been held to be the best; and yet it should not be so nicely balanced as to make action and movement impossible.

\par  The statesman who builds his hope upon the aristocracy, upon the middle classes, upon the people, will probably, if he have sufficient experience of them, conclude that all classes are much alike, and that one is as good as another, and that the liberties of no class are safe in the hands of the rest. The higher ranks have the advantage in education and manners, the middle and lower in industry and self-denial; in every class, to a certain extent, a natural sense of right prevails, sometimes communicated from the lower to the higher, sometimes from the higher to the lower, which is too strong for class interests. There have been crises in the history of nations, as at the time of the Crusades or the Reformation, or the French Revolution, when the same inspiration has taken hold of whole peoples, and permanently raised the sense of freedom and justice among mankind.

\par  But even supposing the different classes of a nation, when viewed impartially, to be on a level with each other in moral virtue, there remain two considerations of opposite kinds which enter into the problem of government. Admitting of course that the upper and lower classes are equal in the eye of God and of the law, yet the one may be by nature fitted to govern and the other to be governed. A ruling caste does not soon altogether lose the governing qualities, nor a subject class easily acquire them. Hence the phenomenon so often observed in the old Greek revolutions, and not without parallel in modern times, that the leaders of the democracy have been themselves of aristocratic origin. The people are expecting to be governed by representatives of their own, but the true man of the people either never appears, or is quickly altered by circumstances. Their real wishes hardly make themselves felt, although their lower interests and prejudices may sometimes be flattered and yielded to for the sake of ulterior objects by those who have political power. They will often learn by experience that the democracy has become a plutocracy. The influence of wealth, though not the enjoyment of it, has become diffused among the poor as well as among the rich; and society, instead of being safer, is more at the mercy of the tyrant, who, when things are at the worst, obtains a guard—that is, an army—and announces himself as the saviour.

\par  The other consideration is of an opposite kind. Admitting that a few wise men are likely to be better governors than the unwise many, yet it is not in their power to fashion an entire people according to their behest. When with the best intentions the benevolent despot begins his regime, he finds the world hard to move. A succession of good kings has at the end of a century left the people an inert and unchanged mass. The Roman world was not permanently improved by the hundred years of Hadrian and the Antonines. The kings of Spain during the last century were at least equal to any contemporary sovereigns in virtue and ability. In certain states of the world the means are wanting to render a benevolent power effectual. These means are not a mere external organisation of posts or telegraphs, hardly the introduction of new laws or modes of industry. A change must be made in the spirit of a people as well as in their externals. The ancient legislator did not really take a blank tablet and inscribe upon it the rules which reflection and experience had taught him to be for a nation's interest; no one would have obeyed him if he had. But he took the customs which he found already existing in a half-civilised state of society: these he reduced to form and inscribed on pillars; he defined what had before been undefined, and gave certainty to what was uncertain. No legislation ever sprang, like Athene, in full power out of the head either of God or man.

\par  Plato and Aristotle are sensible of the difficulty of combining the wisdom of the few with the power of the many. According to Plato, he is a physician who has the knowledge of a physician, and he is a king who has the knowledge of a king. But how the king, one or more, is to obtain the required power, is hardly at all considered by him. He presents the idea of a perfect government, but except the regulation for mixing different tempers in marriage, he never makes any provision for the attainment of it. Aristotle, casting aside ideals, would place the government in a middle class of citizens, sufficiently numerous for stability, without admitting the populace; and such appears to have been the constitution which actually prevailed for a short time at Athens—the rule of the Five Thousand—characterized by Thucydides as the best government of Athens which he had known. It may however be doubted how far, either in a Greek or modern state, such a limitation is practicable or desirable; for those who are left outside the pale will always be dangerous to those who are within, while on the other hand the leaven of the mob can hardly affect the representation of a great country. There is reason for the argument in favour of a property qualification; there is reason also in the arguments of those who would include all and so exhaust the political situation.

\par  The true answer to the question is relative to the circumstances of nations. How can we get the greatest intelligence combined with the greatest power? The ancient legislator would have found this question more easy than we do. For he would have required that all persons who had a share of government should have received their education from the state and have borne her burdens, and should have served in her fleets and armies. But though we sometimes hear the cry that we must 'educate the masses, for they are our masters,' who would listen to a proposal that the franchise should be confined to the educated or to those who fulfil political duties? Then again, we know that the masses are not our masters, and that they are more likely to become so if we educate them. In modern politics so many interests have to be consulted that we are compelled to do, not what is best, but what is possible.

\par  d. Law is the first principle of society, but it cannot supply all the wants of society, and may easily cause more evils than it cures. Plato is aware of the imperfection of law in failing to meet the varieties of circumstances: he is also aware that human life would be intolerable if every detail of it were placed under legal regulation. It may be a great evil that physicians should kill their patients or captains cast away their ships, but it would be a far greater evil if each particular in the practice of medicine or seamanship were regulated by law. Much has been said in modern times about the duty of leaving men to themselves, which is supposed to be the best way of taking care of them. The question is often asked, What are the limits of legislation in relation to morals? And the answer is to the same effect, that morals must take care of themselves. There is a one-sided truth in these answers, if they are regarded as condemnations of the interference with commerce in the last century or of clerical persecution in the Middle Ages. But 'laissez-faire' is not the best but only the second best. What the best is, Plato does not attempt to determine; he only contrasts the imperfection of law with the wisdom of the perfect ruler.

\par  Laws should be just, but they must also be certain, and we are obliged to sacrifice something of their justice to their certainty. Suppose a wise and good judge, who paying little or no regard to the law, attempted to decide with perfect justice the cases that were brought before him. To the uneducated person he would appear to be the ideal of a judge. Such justice has been often exercised in primitive times, or at the present day among eastern rulers. But in the first place it depends entirely on the personal character of the judge. He may be honest, but there is no check upon his dishonesty, and his opinion can only be overruled, not by any principle of law, but by the opinion of another judging like himself without law. In the second place, even if he be ever so honest, his mode of deciding questions would introduce an element of uncertainty into human life; no one would know beforehand what would happen to him, or would seek to conform in his conduct to any rule of law. For the compact which the law makes with men, that they shall be protected if they observe the law in their dealings with one another, would have to be substituted another principle of a more general character, that they shall be protected by the law if they act rightly in their dealings with one another. The complexity of human actions and also the uncertainty of their effects would be increased tenfold. For one of the principal advantages of law is not merely that it enforces honesty, but that it makes men act in the same way, and requires them to produce the same evidence of their acts. Too many laws may be the sign of a corrupt and overcivilized state of society, too few are the sign of an uncivilized one; as soon as commerce begins to grow, men make themselves customs which have the validity of laws. Even equity, which is the exception to the law, conforms to fixed rules and lies for the most part within the limits of previous decisions.

\par  IV. The bitterness of the Statesman is characteristic of Plato's later style, in which the thoughts of youth and love have fled away, and we are no longer tended by the Muses or the Graces. We do not venture to say that Plato was soured by old age, but certainly the kindliness and courtesy of the earlier dialogues have disappeared. He sees the world under a harder and grimmer aspect: he is dealing with the reality of things, not with visions or pictures of them: he is seeking by the aid of dialectic only, to arrive at truth. He is deeply impressed with the importance of classification: in this alone he finds the true measure of human things; and very often in the process of division curious results are obtained. For the dialectical art is no respecter of persons: king and vermin-taker are all alike to the philosopher. There may have been a time when the king was a god, but he now is pretty much on a level with his subjects in breeding and education. Man should be well advised that he is only one of the animals, and the Hellene in particular should be aware that he himself was the author of the distinction between Hellene and Barbarian, and that the Phrygian would equally divide mankind into Phrygians and Barbarians, and that some intelligent animal, like a crane, might go a step further, and divide the animal world into cranes and all other animals. Plato cannot help laughing (compare Theaet.) when he thinks of the king running after his subjects, like the pig-driver or the bird-taker. He would seriously have him consider how many competitors there are to his throne, chiefly among the class of serving-men. A good deal of meaning is lurking in the expression—'There is no art of feeding mankind worthy the name.' There is a similar depth in the remark,—'The wonder about states is not that they are short-lived, but that they last so long in spite of the badness of their rulers.'

\par  V. There is also a paradoxical element in the Statesman which delights in reversing the accustomed use of words. The law which to the Greek was the highest object of reverence is an ignorant and brutal tyrant—the tyrant is converted into a beneficent king. The sophist too is no longer, as in the earlier dialogues, the rival of the statesman, but assumes his form. Plato sees that the ideal of the state in his own day is more and more severed from the actual. From such ideals as he had once formed, he turns away to contemplate the decline of the Greek cities which were far worse now in his old age than they had been in his youth, and were to become worse and worse in the ages which followed. He cannot contain his disgust at the contemporary statesmen, sophists who had turned politicians, in various forms of men and animals, appearing, some like lions and centaurs, others like satyrs and monkeys. In this new disguise the Sophists make their last appearance on the scene: in the Laws Plato appears to have forgotten them, or at any rate makes only a slight allusion to them in a single passage (Laws).

\par  VI. The Statesman is naturally connected with the Sophist. At first sight we are surprised to find that the Eleatic Stranger discourses to us, not only concerning the nature of Being and Not-being, but concerning the king and statesman. We perceive, however, that there is no inappropriateness in his maintaining the character of chief speaker, when we remember the close connexion which is assumed by Plato to exist between politics and dialectic. In both dialogues the Proteus Sophist is exhibited, first, in the disguise of an Eristic, secondly, of a false statesman. There are several lesser features which the two dialogues have in common. The styles and the situations of the speakers are very similar; there is the same love of division, and in both of them the mind of the writer is greatly occupied about method, to which he had probably intended to return in the projected 'Philosopher.'

\par  The Statesman stands midway between the Republic and the Laws, and is also related to the Timaeus. The mythical or cosmical element reminds us of the Timaeus, the ideal of the Republic. A previous chaos in which the elements as yet were not, is hinted at both in the Timaeus and Statesman. The same ingenious arts of giving verisimilitude to a fiction are practised in both dialogues, and in both, as well as in the myth at the end of the Republic, Plato touches on the subject of necessity and free-will. The words in which he describes the miseries of states seem to be an amplification of the 'Cities will never cease from ill' of the Republic. The point of view in both is the same; and the differences not really important, e.g. in the myth, or in the account of the different kinds of states. But the treatment of the subject in the Statesman is fragmentary, and the shorter and later work, as might be expected, is less finished, and less worked out in detail. The idea of measure and the arrangement of the sciences supply connecting links both with the Republic and the Philebus.

\par  More than any of the preceding dialogues, the Statesman seems to approximate in thought and language to the Laws. There is the same decline and tendency to monotony in style, the same self-consciousness, awkwardness, and over-civility; and in the Laws is contained the pattern of that second best form of government, which, after all, is admitted to be the only attainable one in this world. The 'gentle violence,' the marriage of dissimilar natures, the figure of the warp and the woof, are also found in the Laws. Both expressly recognize the conception of a first or ideal state, which has receded into an invisible heaven. Nor does the account of the origin and growth of society really differ in them, if we make allowance for the mythic character of the narrative in the Statesman. The virtuous tyrant is common to both of them; and the Eleatic Stranger takes up a position similar to that of the Athenian Stranger in the Laws.

\par  VII. There would have been little disposition to doubt the genuineness of the Sophist and Statesman, if they had been compared with the Laws rather than with the Republic, and the Laws had been received, as they ought to be, on the authority of Aristotle and on the ground of their intrinsic excellence, as an undoubted work of Plato. The detailed consideration of the genuineness and order of the Platonic dialogues has been reserved for another place: a few of the reasons for defending the Sophist and Statesman may be given here.

\par  1. The excellence, importance, and metaphysical originality of the two dialogues: no works at once so good and of such length are known to have proceeded from the hands of a forger.

\par  2. The resemblances in them to other dialogues of Plato are such as might be expected to be found in works of the same author, and not in those of an imitator, being too subtle and minute to have been invented by another. The similar passages and turns of thought are generally inferior to the parallel passages in his earlier writings; and we might a priori have expected that, if altered, they would have been improved. But the comparison of the Laws proves that this repetition of his own thoughts and words in an inferior form is characteristic of Plato's later style.

\par  3. The close connexion of them with the Theaetetus, Parmenides, and Philebus, involves the fate of these dialogues, as well as of the two suspected ones.

\par  4. The suspicion of them seems mainly to rest on a presumption that in Plato's writings we may expect to find an uniform type of doctrine and opinion. But however we arrange the order, or narrow the circle of the dialogues, we must admit that they exhibit a growth and progress in the mind of Plato. And the appearance of change or progress is not to be regarded as impugning the genuineness of any particular writings, but may be even an argument in their favour. If we suppose the Sophist and Politicus to stand halfway between the Republic and the Laws, and in near connexion with the Theaetetus, the Parmenides, the Philebus, the arguments against them derived from differences of thought and style disappear or may be said without paradox in some degree to confirm their genuineness. There is no such interval between the Republic or Phaedrus and the two suspected dialogues, as that which separates all the earlier writings of Plato from the Laws. And the Theaetetus, Parmenides, and Philebus, supply links, by which, however different from them, they may be reunited with the great body of the Platonic writings.

\par 
\section{
      STATESMAN
    } 
\par \textbf{SOCRATES}
\par   I owe you many thanks, indeed, Theodorus, for the acquaintance both of Theaetetus and of the Stranger.

\par \textbf{THEODORUS}
\par   And in a little while, Socrates, you will owe me three times as many, when they have completed for you the delineation of the Statesman and of the Philosopher, as well as of the Sophist.

\par \textbf{SOCRATES}
\par   Sophist, statesman, philosopher! O my dear Theodorus, do my ears truly witness that this is the estimate formed of them by the great calculator and geometrician?

\par \textbf{THEODORUS}
\par   What do you mean, Socrates?

\par \textbf{SOCRATES}
\par   I mean that you rate them all at the same value, whereas they are really separated by an interval, which no geometrical ratio can express.

\par \textbf{THEODORUS}
\par   By Ammon, the god of Cyrene, Socrates, that is a very fair hit; and shows that you have not forgotten your geometry. I will retaliate on you at some other time, but I must now ask the Stranger, who will not, I hope, tire of his goodness to us, to proceed either with the Statesman or with the Philosopher, whichever he prefers.

\par \textbf{STRANGER}
\par   That is my duty, Theodorus; having begun I must go on, and not leave the work unfinished. But what shall be done with Theaetetus?

\par \textbf{THEODORUS}
\par   In what respect?

\par \textbf{STRANGER}
\par   Shall we relieve him, and take his companion, the Young Socrates, instead of him? What do you advise?

\par \textbf{THEODORUS}
\par   Yes, give the other a turn, as you propose. The young always do better when they have intervals of rest.

\par \textbf{SOCRATES}
\par   I think, Stranger, that both of them may be said to be in some way related to me; for the one, as you affirm, has the cut of my ugly face (compare Theaet. ), the other is called by my name. And we should always be on the look-out to recognize a kinsman by the style of his conversation. I myself was discoursing with Theaetetus yesterday, and I have just been listening to his answers; my namesake I have not yet examined, but I must. Another time will do for me; to-day let him answer you.

\par \textbf{STRANGER}
\par   Very good. Young Socrates, do you hear what the elder Socrates is proposing?

\par \textbf{YOUNG SOCRATES}
\par   I do.

\par \textbf{STRANGER}
\par   And do you agree to his proposal?

\par \textbf{YOUNG SOCRATES}
\par   Certainly.

\par \textbf{STRANGER}
\par   As you do not object, still less can I. After the Sophist, then, I think that the Statesman naturally follows next in the order of enquiry. And please to say, whether he, too, should be ranked among those who have science.

\par \textbf{YOUNG SOCRATES}
\par   Yes.

\par \textbf{STRANGER}
\par   Then the sciences must be divided as before?

\par \textbf{YOUNG SOCRATES}
\par   I dare say.

\par \textbf{STRANGER}
\par   But yet the division will not be the same?

\par \textbf{YOUNG SOCRATES}
\par   How then?

\par \textbf{STRANGER}
\par   They will be divided at some other point.

\par \textbf{YOUNG SOCRATES}
\par   Yes.

\par \textbf{STRANGER}
\par   Where shall we discover the path of the Statesman? We must find and separate off, and set our seal upon this, and we will set the mark of another class upon all diverging paths. Thus the soul will conceive of all kinds of knowledge under two classes.

\par \textbf{YOUNG SOCRATES}
\par   To find the path is your business, Stranger, and not mine.

\par \textbf{STRANGER}
\par   Yes, Socrates, but the discovery, when once made, must be yours as well as mine.

\par \textbf{YOUNG SOCRATES}
\par   Very good.

\par \textbf{STRANGER}
\par   Well, and are not arithmetic and certain other kindred arts, merely abstract knowledge, wholly separated from action?

\par \textbf{YOUNG SOCRATES}
\par   True.

\par \textbf{STRANGER}
\par   But in the art of carpentering and all other handicrafts, the knowledge of the workman is merged in his work; he not only knows, but he also makes things which previously did not exist.

\par \textbf{YOUNG SOCRATES}
\par   Certainly.

\par \textbf{STRANGER}
\par   Then let us divide sciences in general into those which are practical and those which are purely intellectual.

\par \textbf{YOUNG SOCRATES}
\par   Let us assume these two divisions of science, which is one whole.

\par \textbf{STRANGER}
\par   And are 'statesman,' 'king,' 'master,' or 'householder,' one and the same; or is there a science or art answering to each of these names? Or rather, allow me to put the matter in another way.

\par \textbf{YOUNG SOCRATES}
\par   Let me hear.

\par \textbf{STRANGER}
\par   If any one who is in a private station has the skill to advise one of the public physicians, must not he also be called a physician?

\par \textbf{YOUNG SOCRATES}
\par   Yes.

\par \textbf{STRANGER}
\par   And if any one who is in a private station is able to advise the ruler of a country, may not he be said to have the knowledge which the ruler himself ought to have?

\par \textbf{YOUNG SOCRATES}
\par   True.

\par \textbf{STRANGER}
\par   But surely the science of a true king is royal science?

\par \textbf{YOUNG SOCRATES}
\par   Yes.

\par \textbf{STRANGER}
\par   And will not he who possesses this knowledge, whether he happens to be a ruler or a private man, when regarded only in reference to his art, be truly called 'royal'?

\par \textbf{YOUNG SOCRATES}
\par   He certainly ought to be.

\par \textbf{STRANGER}
\par   And the householder and master are the same?

\par \textbf{YOUNG SOCRATES}
\par   Of course.

\par \textbf{STRANGER}
\par   Again, a large household may be compared to a small state: —will they differ at all, as far as government is concerned?

\par \textbf{YOUNG SOCRATES}
\par   They will not.

\par \textbf{STRANGER}
\par   Then, returning to the point which we were just now discussing, do we not clearly see that there is one science of all of them; and this science may be called either royal or political or economical; we will not quarrel with any one about the name.

\par \textbf{YOUNG SOCRATES}
\par   Certainly not.

\par \textbf{STRANGER}
\par   This too, is evident, that the king cannot do much with his hands, or with his whole body, towards the maintenance of his empire, compared with what he does by the intelligence and strength of his mind.

\par \textbf{YOUNG SOCRATES}
\par   Clearly not.

\par \textbf{STRANGER}
\par   Then, shall we say that the king has a greater affinity to knowledge than to manual arts and to practical life in general?

\par \textbf{YOUNG SOCRATES}
\par   Certainly he has.

\par \textbf{STRANGER}
\par   Then we may put all together as one and the same—statesmanship and the statesman—the kingly science and the king.

\par \textbf{YOUNG SOCRATES}
\par   Clearly.

\par \textbf{STRANGER}
\par   And now we shall only be proceeding in due order if we go on to divide the sphere of knowledge?

\par \textbf{YOUNG SOCRATES}
\par   Very good.

\par \textbf{STRANGER}
\par   Think whether you can find any joint or parting in knowledge.

\par \textbf{YOUNG SOCRATES}
\par   Tell me of what sort.

\par \textbf{STRANGER}
\par   Such as this:  You may remember that we made an art of calculation?

\par \textbf{YOUNG SOCRATES}
\par   Yes.

\par \textbf{STRANGER}
\par   Which was, unmistakeably, one of the arts of knowledge?

\par \textbf{YOUNG SOCRATES}
\par   Certainly.

\par \textbf{STRANGER}
\par   And to this art of calculation which discerns the differences of numbers shall we assign any other function except to pass judgment on their differences?

\par \textbf{YOUNG SOCRATES}
\par   How could we?

\par \textbf{STRANGER}
\par   You know that the master-builder does not work himself, but is the ruler of workmen?

\par \textbf{YOUNG SOCRATES}
\par   Yes.

\par \textbf{STRANGER}
\par   He contributes knowledge, not manual labour?

\par \textbf{YOUNG SOCRATES}
\par   True.

\par \textbf{STRANGER}
\par   And may therefore be justly said to share in theoretical science?

\par \textbf{YOUNG SOCRATES}
\par   Quite true.

\par \textbf{STRANGER}
\par   But he ought not, like the calculator, to regard his functions as at an end when he has formed a judgment;—he must assign to the individual workmen their appropriate task until they have completed the work.

\par \textbf{YOUNG SOCRATES}
\par   True.

\par \textbf{STRANGER}
\par   Are not all such sciences, no less than arithmetic and the like, subjects of pure knowledge; and is not the difference between the two classes, that the one sort has the power of judging only, and the other of ruling as well?

\par \textbf{YOUNG SOCRATES}
\par   That is evident.

\par \textbf{STRANGER}
\par   May we not very properly say, that of all knowledge, there are two divisions—one which rules, and the other which judges?

\par \textbf{YOUNG SOCRATES}
\par   I should think so.

\par \textbf{STRANGER}
\par   And when men have anything to do in common, that they should be of one mind is surely a desirable thing?

\par \textbf{YOUNG SOCRATES}
\par   Very true.

\par \textbf{STRANGER}
\par   Then while we are at unity among ourselves, we need not mind about the fancies of others?

\par \textbf{YOUNG SOCRATES}
\par   Certainly not.

\par \textbf{STRANGER}
\par   And now, in which of these divisions shall we place the king?—Is he a judge and a kind of spectator? Or shall we assign to him the art of command—for he is a ruler?

\par \textbf{YOUNG SOCRATES}
\par   The latter, clearly.

\par \textbf{STRANGER}
\par   Then we must see whether there is any mark of division in the art of command too. I am inclined to think that there is a distinction similar to that of manufacturer and retail dealer, which parts off the king from the herald.

\par \textbf{YOUNG SOCRATES}
\par   How is this?

\par \textbf{STRANGER}
\par   Why, does not the retailer receive and sell over again the productions of others, which have been sold before?

\par \textbf{YOUNG SOCRATES}
\par   Certainly he does.

\par \textbf{STRANGER}
\par   And is not the herald under command, and does he not receive orders, and in his turn give them to others?

\par \textbf{YOUNG SOCRATES}
\par   Very true.

\par \textbf{STRANGER}
\par   Then shall we mingle the kingly art in the same class with the art of the herald, the interpreter, the boatswain, the prophet, and the numerous kindred arts which exercise command; or, as in the preceding comparison we spoke of manufacturers, or sellers for themselves, and of retailers,—seeing, too, that the class of supreme rulers, or rulers for themselves, is almost nameless—shall we make a word following the same analogy, and refer kings to a supreme or ruling-for-self science, leaving the rest to receive a name from some one else? For we are seeking the ruler; and our enquiry is not concerned with him who is not a ruler.

\par \textbf{YOUNG SOCRATES}
\par   Very good.

\par \textbf{STRANGER}
\par   Thus a very fair distinction has been attained between the man who gives his own commands, and him who gives another's. And now let us see if the supreme power allows of any further division.

\par \textbf{YOUNG SOCRATES}
\par   By all means.

\par \textbf{STRANGER}
\par   I think that it does; and please to assist me in making the division.

\par \textbf{YOUNG SOCRATES}
\par   At what point?

\par \textbf{STRANGER}
\par   May not all rulers be supposed to command for the sake of producing something?

\par \textbf{YOUNG SOCRATES}
\par   Certainly.

\par \textbf{STRANGER}
\par   Nor is there any difficulty in dividing the things produced into two classes.

\par \textbf{YOUNG SOCRATES}
\par   How would you divide them?

\par \textbf{STRANGER}
\par   Of the whole class, some have life and some are without life.

\par \textbf{YOUNG SOCRATES}
\par   True.

\par \textbf{STRANGER}
\par   And by the help of this distinction we may make, if we please, a subdivision of the section of knowledge which commands.

\par \textbf{YOUNG SOCRATES}
\par   At what point?

\par \textbf{STRANGER}
\par   One part may be set over the production of lifeless, the other of living objects; and in this way the whole will be divided.

\par \textbf{YOUNG SOCRATES}
\par   Certainly.

\par \textbf{STRANGER}
\par   That division, then, is complete; and now we may leave one half, and take up the other; which may also be divided into two.

\par \textbf{YOUNG SOCRATES}
\par   Which of the two halves do you mean?

\par \textbf{STRANGER}
\par   Of course that which exercises command about animals. For, surely, the royal science is not like that of a master-workman, a science presiding over lifeless objects;—the king has a nobler function, which is the management and control of living beings.

\par \textbf{YOUNG SOCRATES}
\par   True.

\par \textbf{STRANGER}
\par   And the breeding and tending of living beings may be observed to be sometimes a tending of the individual; in other cases, a common care of creatures in flocks?

\par \textbf{YOUNG SOCRATES}
\par   True.

\par \textbf{STRANGER}
\par   But the statesman is not a tender of individuals—not like the driver or groom of a single ox or horse; he is rather to be compared with the keeper of a drove of horses or oxen.

\par \textbf{YOUNG SOCRATES}
\par   Yes, I see, thanks to you.

\par \textbf{STRANGER}
\par   Shall we call this art of tending many animals together, the art of managing a herd, or the art of collective management?

\par \textbf{YOUNG SOCRATES}
\par   No matter;—whichever suggests itself to us in the course of conversation.

\par \textbf{STRANGER}
\par   Very good, Socrates; and, if you continue to be not too particular about names, you will be all the richer in wisdom when you are an old man. And now, as you say, leaving the discussion of the name,—can you see a way in which a person, by showing the art of herding to be of two kinds, may cause that which is now sought amongst twice the number of things, to be then sought amongst half that number?

\par \textbf{YOUNG SOCRATES}
\par   I will try;—there appears to me to be one management of men and another of beasts.

\par \textbf{STRANGER}
\par   You have certainly divided them in a most straightforward and manly style; but you have fallen into an error which hereafter I think that we had better avoid.

\par \textbf{YOUNG SOCRATES}
\par   What is the error?

\par \textbf{STRANGER}
\par   I think that we had better not cut off a single small portion which is not a species, from many larger portions; the part should be a species. To separate off at once the subject of investigation, is a most excellent plan, if only the separation be rightly made; and you were under the impression that you were right, because you saw that you would come to man; and this led you to hasten the steps. But you should not chip off too small a piece, my friend; the safer way is to cut through the middle; which is also the more likely way of finding classes. Attention to this principle makes all the difference in a process of enquiry.

\par \textbf{YOUNG SOCRATES}
\par   What do you mean, Stranger?

\par \textbf{STRANGER}
\par   I will endeavour to speak more plainly out of love to your good parts, Socrates; and, although I cannot at present entirely explain myself, I will try, as we proceed, to make my meaning a little clearer.

\par \textbf{YOUNG SOCRATES}
\par   What was the error of which, as you say, we were guilty in our recent division?

\par \textbf{STRANGER}
\par   The error was just as if some one who wanted to divide the human race, were to divide them after the fashion which prevails in this part of the world; here they cut off the Hellenes as one species, and all the other species of mankind, which are innumerable, and have no ties or common language, they include under the single name of 'barbarians,' and because they have one name they are supposed to be of one species also. Or suppose that in dividing numbers you were to cut off ten thousand from all the rest, and make of it one species, comprehending the rest under another separate name, you might say that here too was a single class, because you had given it a single name. Whereas you would make a much better and more equal and logical classification of numbers, if you divided them into odd and even; or of the human species, if you divided them into male and female; and only separated off Lydians or Phrygians, or any other tribe, and arrayed them against the rest of the world, when you could no longer make a division into parts which were also classes.

\par \textbf{YOUNG SOCRATES}
\par   Very true; but I wish that this distinction between a part and a class could still be made somewhat plainer.

\par \textbf{STRANGER}
\par   O Socrates, best of men, you are imposing upon me a very difficult task. We have already digressed further from our original intention than we ought, and you would have us wander still further away. But we must now return to our subject; and hereafter, when there is a leisure hour, we will follow up the other track; at the same time, I wish you to guard against imagining that you ever heard me declare—

\par \textbf{YOUNG SOCRATES}
\par   What?

\par \textbf{STRANGER}
\par   That a class and a part are distinct.

\par \textbf{YOUNG SOCRATES}
\par   What did I hear, then?

\par \textbf{STRANGER}
\par   That a class is necessarily a part, but there is no similar necessity that a part should be a class; that is the view which I should always wish you to attribute to me, Socrates.

\par \textbf{YOUNG SOCRATES}
\par   So be it.

\par \textbf{STRANGER}
\par   There is another thing which I should like to know.

\par \textbf{YOUNG SOCRATES}
\par   What is it?

\par \textbf{STRANGER}
\par   The point at which we digressed; for, if I am not mistaken, the exact place was at the question, Where you would divide the management of herds. To this you appeared rather too ready to answer that there were two species of animals; man being one, and all brutes making up the other.

\par \textbf{YOUNG SOCRATES}
\par   True.

\par \textbf{STRANGER}
\par   I thought that in taking away a part, you imagined that the remainder formed a class, because you were able to call them by the common name of brutes.

\par \textbf{YOUNG SOCRATES}
\par   That again is true.

\par \textbf{STRANGER}
\par   Suppose now, O most courageous of dialecticians, that some wise and understanding creature, such as a crane is reputed to be, were, in imitation of you, to make a similar division, and set up cranes against all other animals to their own special glorification, at the same time jumbling together all the others, including man, under the appellation of brutes,—here would be the sort of error which we must try to avoid.

\par \textbf{YOUNG SOCRATES}
\par   How can we be safe?

\par \textbf{STRANGER}
\par   If we do not divide the whole class of animals, we shall be less likely to fall into that error.

\par \textbf{YOUNG SOCRATES}
\par   We had better not take the whole?

\par \textbf{STRANGER}
\par   Yes, there lay the source of error in our former division.

\par \textbf{YOUNG SOCRATES}
\par   How?

\par \textbf{STRANGER}
\par   You remember how that part of the art of knowledge which was concerned with command, had to do with the rearing of living creatures,—I mean, with animals in herds?

\par \textbf{YOUNG SOCRATES}
\par   Yes.

\par \textbf{STRANGER}
\par   In that case, there was already implied a division of all animals into tame and wild; those whose nature can be tamed are called tame, and those which cannot be tamed are called wild.

\par \textbf{YOUNG SOCRATES}
\par   True.

\par \textbf{STRANGER}
\par   And the political science of which we are in search, is and ever was concerned with tame animals, and is also confined to gregarious animals.

\par \textbf{YOUNG SOCRATES}
\par   Yes.

\par \textbf{STRANGER}
\par   But then we ought not to divide, as we did, taking the whole class at once. Neither let us be in too great haste to arrive quickly at the political science; for this mistake has already brought upon us the misfortune of which the proverb speaks.

\par \textbf{YOUNG SOCRATES}
\par   What misfortune?

\par \textbf{STRANGER}
\par   The misfortune of too much haste, which is too little speed.

\par \textbf{YOUNG SOCRATES}
\par   And all the better, Stranger;—we got what we deserved.

\par \textbf{STRANGER}
\par   Very well:  Let us then begin again, and endeavour to divide the collective rearing of animals; for probably the completion of the argument will best show what you are so anxious to know. Tell me, then—

\par \textbf{YOUNG SOCRATES}
\par   What?

\par \textbf{STRANGER}
\par   Have you ever heard, as you very likely may—for I do not suppose that you ever actually visited them—of the preserves of fishes in the Nile, and in the ponds of the Great King; or you may have seen similar preserves in wells at home?

\par \textbf{YOUNG SOCRATES}
\par   Yes, to be sure, I have seen them, and I have often heard the others described.

\par \textbf{STRANGER}
\par   And you may have heard also, and may have been assured by report, although you have not travelled in those regions, of nurseries of geese and cranes in the plains of Thessaly?

\par \textbf{YOUNG SOCRATES}
\par   Certainly.

\par \textbf{STRANGER}
\par   I asked you, because here is a new division of the management of herds, into the management of land and of water herds.

\par \textbf{YOUNG SOCRATES}
\par   There is.

\par \textbf{STRANGER}
\par   And do you agree that we ought to divide the collective rearing of herds into two corresponding parts, the one the rearing of water, and the other the rearing of land herds?

\par \textbf{YOUNG SOCRATES}
\par   Yes.

\par \textbf{STRANGER}
\par   There is surely no need to ask which of these two contains the royal art, for it is evident to everybody.

\par \textbf{YOUNG SOCRATES}
\par   Certainly.

\par \textbf{STRANGER}
\par   Any one can divide the herds which feed on dry land?

\par \textbf{YOUNG SOCRATES}
\par   How would you divide them?

\par \textbf{STRANGER}
\par   I should distinguish between those which fly and those which walk.

\par \textbf{YOUNG SOCRATES}
\par   Most true.

\par \textbf{STRANGER}
\par   And where shall we look for the political animal? Might not an idiot, so to speak, know that he is a pedestrian?

\par \textbf{YOUNG SOCRATES}
\par   Certainly.

\par \textbf{STRANGER}
\par   The art of managing the walking animal has to be further divided, just as you might halve an even number.

\par \textbf{YOUNG SOCRATES}
\par   Clearly.

\par \textbf{STRANGER}
\par   Let me note that here appear in view two ways to that part or class which the argument aims at reaching,—the one a speedier way, which cuts off a small portion and leaves a large; the other agrees better with the principle which we were laying down, that as far as we can we should divide in the middle; but it is longer. We can take either of them, whichever we please.

\par \textbf{YOUNG SOCRATES}
\par   Cannot we have both ways?

\par \textbf{STRANGER}
\par   Together? What a thing to ask! but, if you take them in turn, you clearly may.

\par \textbf{YOUNG SOCRATES}
\par   Then I should like to have them in turn.

\par \textbf{STRANGER}
\par   There will be no difficulty, as we are near the end; if we had been at the beginning, or in the middle, I should have demurred to your request; but now, in accordance with your desire, let us begin with the longer way; while we are fresh, we shall get on better. And now attend to the division.

\par \textbf{YOUNG SOCRATES}
\par   Let me hear.

\par \textbf{STRANGER}
\par   The tame walking herding animals are distributed by nature into two classes.

\par \textbf{YOUNG SOCRATES}
\par   Upon what principle?

\par \textbf{STRANGER}
\par   The one grows horns; and the other is without horns.

\par \textbf{YOUNG SOCRATES}
\par   Clearly.

\par \textbf{STRANGER}
\par   Suppose that you divide the science which manages pedestrian animals into two corresponding parts, and define them; for if you try to invent names for them, you will find the intricacy too great.

\par \textbf{YOUNG SOCRATES}
\par   How must I speak of them, then?

\par \textbf{STRANGER}
\par   In this way:  let the science of managing pedestrian animals be divided into two parts, and one part assigned to the horned herd, and the other to the herd that has no horns.

\par \textbf{YOUNG SOCRATES}
\par   All that you say has been abundantly proved, and may therefore be assumed.

\par \textbf{STRANGER}
\par   The king is clearly the shepherd of a polled herd, who have no horns.

\par \textbf{YOUNG SOCRATES}
\par   That is evident.

\par \textbf{STRANGER}
\par   Shall we break up this hornless herd into sections, and endeavour to assign to him what is his?

\par \textbf{YOUNG SOCRATES}
\par   By all means.

\par \textbf{STRANGER}
\par   Shall we distinguish them by their having or not having cloven feet, or by their mixing or not mixing the breed? You know what I mean.

\par \textbf{YOUNG SOCRATES}
\par   What?

\par \textbf{STRANGER}
\par   I mean that horses and asses naturally breed from one another.

\par \textbf{YOUNG SOCRATES}
\par   Yes.

\par \textbf{STRANGER}
\par   But the remainder of the hornless herd of tame animals will not mix the breed.

\par \textbf{YOUNG SOCRATES}
\par   Very true.

\par \textbf{STRANGER}
\par   And of which has the Statesman charge,—of the mixed or of the unmixed race?

\par \textbf{YOUNG SOCRATES}
\par   Clearly of the unmixed.

\par \textbf{STRANGER}
\par   I suppose that we must divide this again as before.

\par \textbf{YOUNG SOCRATES}
\par   We must.

\par \textbf{STRANGER}
\par   Every tame and herding animal has now been split up, with the exception of two species; for I hardly think that dogs should be reckoned among gregarious animals.

\par \textbf{YOUNG SOCRATES}
\par   Certainly not; but how shall we divide the two remaining species?

\par \textbf{STRANGER}
\par   There is a measure of difference which may be appropriately employed by you and Theaetetus, who are students of geometry.

\par \textbf{YOUNG SOCRATES}
\par   What is that?

\par \textbf{STRANGER}
\par   The diameter; and, again, the diameter of a diameter. (Compare Meno.)

\par \textbf{YOUNG SOCRATES}
\par   What do you mean?

\par \textbf{STRANGER}
\par   How does man walk, but as a diameter whose power is two feet?

\par \textbf{YOUNG SOCRATES}
\par   Just so.

\par \textbf{STRANGER}
\par   And the power of the remaining kind, being the power of twice two feet, may be said to be the diameter of our diameter.

\par \textbf{YOUNG SOCRATES}
\par   Certainly; and now I think that I pretty nearly understand you.

\par \textbf{STRANGER}
\par   In these divisions, Socrates, I descry what would make another famous jest.

\par \textbf{YOUNG SOCRATES}
\par   What is it?

\par \textbf{STRANGER}
\par   Human beings have come out in the same class with the freest and airiest of creation, and have been running a race with them.

\par \textbf{YOUNG SOCRATES}
\par   I remark that very singular coincidence.

\par \textbf{STRANGER}
\par   And would you not expect the slowest to arrive last?

\par \textbf{YOUNG SOCRATES}
\par   Indeed I should.

\par \textbf{STRANGER}
\par   And there is a still more ridiculous consequence, that the king is found running about with the herd and in close competition with the bird-catcher, who of all mankind is most of an adept at the airy life. (Plato is here introducing a new subdivision, i.e. that of bipeds into men and birds. Others however refer the passage to the division into quadrupeds and bipeds, making pigs compete with human beings and the pig-driver with the king. According to this explanation we must translate the words above, 'freest and airiest of creation,' 'worthiest and laziest of creation.')

\par \textbf{YOUNG SOCRATES}
\par   Certainly.

\par \textbf{STRANGER}
\par   Then here, Socrates, is still clearer evidence of the truth of what was said in the enquiry about the Sophist? (Compare Sophist.)

\par \textbf{YOUNG SOCRATES}
\par   What?

\par \textbf{STRANGER}
\par   That the dialectical method is no respecter of persons, and does not set the great above the small, but always arrives in her own way at the truest result.

\par \textbf{YOUNG SOCRATES}
\par   Clearly.

\par \textbf{STRANGER}
\par   And now, I will not wait for you to ask, but will of my own accord take you by the shorter road to the definition of a king.

\par \textbf{YOUNG SOCRATES}
\par   By all means.

\par \textbf{STRANGER}
\par   I say that we should have begun at first by dividing land animals into biped and quadruped; and since the winged herd, and that alone, comes out in the same class with man, we should divide bipeds into those which have feathers and those which have not, and when they have been divided, and the art of the management of mankind is brought to light, the time will have come to produce our Statesman and ruler, and set him like a charioteer in his place, and hand over to him the reins of state, for that too is a vocation which belongs to him.

\par \textbf{YOUNG SOCRATES}
\par   Very good; you have paid me the debt,—I mean, that you have completed the argument, and I suppose that you added the digression by way of interest. (Compare Republic.)

\par \textbf{STRANGER}
\par   Then now, let us go back to the beginning, and join the links, which together make the definition of the name of the Statesman's art.

\par \textbf{YOUNG SOCRATES}
\par   By all means.

\par \textbf{STRANGER}
\par   The science of pure knowledge had, as we said originally, a part which was the science of rule or command, and from this was derived another part, which was called command-for-self, on the analogy of selling-for-self; an important section of this was the management of living animals, and this again was further limited to the management of them in herds; and again in herds of pedestrian animals. The chief division of the latter was the art of managing pedestrian animals which are without horns; this again has a part which can only be comprehended under one term by joining together three names—shepherding pure-bred animals. The only further subdivision is the art of man-herding,—this has to do with bipeds, and is what we were seeking after, and have now found, being at once the royal and political.

\par \textbf{YOUNG SOCRATES}
\par   To be sure.

\par \textbf{STRANGER}
\par   And do you think, Socrates, that we really have done as you say?

\par \textbf{YOUNG SOCRATES}
\par   What?

\par \textbf{STRANGER}
\par   Do you think, I mean, that we have really fulfilled our intention?—There has been a sort of discussion, and yet the investigation seems to me not to be perfectly worked out:  this is where the enquiry fails.

\par \textbf{YOUNG SOCRATES}
\par   I do not understand.

\par \textbf{STRANGER}
\par   I will try to make the thought, which is at this moment present in my mind, clearer to us both.

\par \textbf{YOUNG SOCRATES}
\par   Let me hear.

\par \textbf{STRANGER}
\par   There were many arts of shepherding, and one of them was the political, which had the charge of one particular herd?

\par \textbf{YOUNG SOCRATES}
\par   Yes.

\par \textbf{STRANGER}
\par   And this the argument defined to be the art of rearing, not horses or other brutes, but the art of rearing man collectively?

\par \textbf{YOUNG SOCRATES}
\par   True.

\par \textbf{STRANGER}
\par   Note, however, a difference which distinguishes the king from all other shepherds.

\par \textbf{YOUNG SOCRATES}
\par   To what do you refer?

\par \textbf{STRANGER}
\par   I want to ask, whether any one of the other herdsmen has a rival who professes and claims to share with him in the management of the herd?

\par \textbf{YOUNG SOCRATES}
\par   What do you mean?

\par \textbf{STRANGER}
\par   I mean to say that merchants, husbandmen, providers of food, and also training-masters and physicians, will all contend with the herdsmen of humanity, whom we call Statesmen, declaring that they themselves have the care of rearing or managing mankind, and that they rear not only the common herd, but also the rulers themselves.

\par \textbf{YOUNG SOCRATES}
\par   Are they not right in saying so?

\par \textbf{STRANGER}
\par   Very likely they may be, and we will consider their claim. But we are certain of this,—that no one will raise a similar claim as against the herdsman, who is allowed on all hands to be the sole and only feeder and physician of his herd; he is also their match-maker and accoucheur; no one else knows that department of science. And he is their merry-maker and musician, as far as their nature is susceptible of such influences, and no one can console and soothe his own herd better than he can, either with the natural tones of his voice or with instruments. And the same may be said of tenders of animals in general.

\par \textbf{YOUNG SOCRATES}
\par   Very true.

\par \textbf{STRANGER}
\par   But if this is as you say, can our argument about the king be true and unimpeachable? Were we right in selecting him out of ten thousand other claimants to be the shepherd and rearer of the human flock?

\par \textbf{YOUNG SOCRATES}
\par   Surely not.

\par \textbf{STRANGER}
\par   Had we not reason just to now to apprehend, that although we may have described a sort of royal form, we have not as yet accurately worked out the true image of the Statesman? and that we cannot reveal him as he truly is in his own nature, until we have disengaged and separated him from those who hang about him and claim to share in his prerogatives?

\par \textbf{YOUNG SOCRATES}
\par   Very true.

\par \textbf{STRANGER}
\par   And that, Socrates, is what we must do, if we do not mean to bring disgrace upon the argument at its close.

\par \textbf{YOUNG SOCRATES}
\par   We must certainly avoid that.

\par \textbf{STRANGER}
\par   Then let us make a new beginning, and travel by a different road.

\par \textbf{YOUNG SOCRATES}
\par   What road?

\par \textbf{STRANGER}
\par   I think that we may have a little amusement; there is a famous tale, of which a good portion may with advantage be interwoven, and then we may resume our series of divisions, and proceed in the old path until we arrive at the desired summit. Shall we do as I say?

\par \textbf{YOUNG SOCRATES}
\par   By all means.

\par \textbf{STRANGER}
\par   Listen, then, to a tale which a child would love to hear; and you are not too old for childish amusement.

\par \textbf{YOUNG SOCRATES}
\par   Let me hear.

\par \textbf{STRANGER}
\par   There did really happen, and will again happen, like many other events of which ancient tradition has preserved the record, the portent which is traditionally said to have occurred in the quarrel of Atreus and Thyestes. You have heard, no doubt, and remember what they say happened at that time?

\par \textbf{YOUNG SOCRATES}
\par   I suppose you to mean the token of the birth of the golden lamb.

\par \textbf{STRANGER}
\par   No, not that; but another part of the story, which tells how the sun and the stars once rose in the west, and set in the east, and that the god reversed their motion, and gave them that which they now have as a testimony to the right of Atreus.

\par \textbf{YOUNG SOCRATES}
\par   Yes; there is that legend also.

\par \textbf{STRANGER}
\par   Again, we have been often told of the reign of Cronos.

\par \textbf{YOUNG SOCRATES}
\par   Yes, very often.

\par \textbf{STRANGER}
\par   Did you ever hear that the men of former times were earth-born, and not begotten of one another?

\par \textbf{YOUNG SOCRATES}
\par   Yes, that is another old tradition.

\par \textbf{STRANGER}
\par   All these stories, and ten thousand others which are still more wonderful, have a common origin; many of them have been lost in the lapse of ages, or are repeated only in a disconnected form; but the origin of them is what no one has told, and may as well be told now; for the tale is suited to throw light on the nature of the king.

\par \textbf{YOUNG SOCRATES}
\par   Very good; and I hope that you will give the whole story, and leave out nothing.

\par \textbf{STRANGER}
\par   Listen, then. There is a time when God himself guides and helps to roll the world in its course; and there is a time, on the completion of a certain cycle, when he lets go, and the world being a living creature, and having originally received intelligence from its author and creator, turns about and by an inherent necessity revolves in the opposite direction.

\par \textbf{YOUNG SOCRATES}
\par   Why is that?

\par \textbf{STRANGER}
\par   Why, because only the most divine things of all remain ever unchanged and the same, and body is not included in this class. Heaven and the universe, as we have termed them, although they have been endowed by the Creator with many glories, partake of a bodily nature, and therefore cannot be entirely free from perturbation. But their motion is, as far as possible, single and in the same place, and of the same kind; and is therefore only subject to a reversal, which is the least alteration possible. For the lord of all moving things is alone able to move of himself; and to think that he moves them at one time in one direction and at another time in another is blasphemy. Hence we must not say that the world is either self-moved always, or all made to go round by God in two opposite courses; or that two Gods, having opposite purposes, make it move round. But as I have already said (and this is the only remaining alternative) the world is guided at one time by an external power which is divine and receives fresh life and immortality from the renewing hand of the Creator, and again, when let go, moves spontaneously, being set free at such a time as to have, during infinite cycles of years, a reverse movement:  this is due to its perfect balance, to its vast size, and to the fact that it turns on the smallest pivot.

\par \textbf{YOUNG SOCRATES}
\par   Your account of the world seems to be very reasonable indeed.

\par \textbf{STRANGER}
\par   Let us now reflect and try to gather from what has been said the nature of the phenomenon which we affirmed to be the cause of all these wonders. It is this.

\par \textbf{YOUNG SOCRATES}
\par   What?

\par \textbf{STRANGER}
\par   The reversal which takes place from time to time of the motion of the universe.

\par \textbf{YOUNG SOCRATES}
\par   How is that the cause?

\par \textbf{STRANGER}
\par   Of all changes of the heavenly motions, we may consider this to be the greatest and most complete.

\par \textbf{YOUNG SOCRATES}
\par   I should imagine so.

\par \textbf{STRANGER}
\par   And it may be supposed to result in the greatest changes to the human beings who are the inhabitants of the world at the time.

\par \textbf{YOUNG SOCRATES}
\par   Such changes would naturally occur.

\par \textbf{STRANGER}
\par   And animals, as we know, survive with difficulty great and serious changes of many different kinds when they come upon them at once.

\par \textbf{YOUNG SOCRATES}
\par   Very true.

\par \textbf{STRANGER}
\par   Hence there necessarily occurs a great destruction of them, which extends also to the life of man; few survivors of the race are left, and those who remain become the subjects of several novel and remarkable phenomena, and of one in particular, which takes place at the time when the transition is made to the cycle opposite to that in which we are now living.

\par \textbf{YOUNG SOCRATES}
\par   What is it?

\par \textbf{STRANGER}
\par   The life of all animals first came to a standstill, and the mortal nature ceased to be or look older, and was then reversed and grew young and delicate; the white locks of the aged darkened again, and the cheeks the bearded man became smooth, and recovered their former bloom; the bodies of youths in their prime grew softer and smaller, continually by day and night returning and becoming assimilated to the nature of a newly-born child in mind as well as body; in the succeeding stage they wasted away and wholly disappeared. And the bodies of those who died by violence at that time quickly passed through the like changes, and in a few days were no more seen.

\par \textbf{YOUNG SOCRATES}
\par   Then how, Stranger, were the animals created in those days; and in what way were they begotten of one another?

\par \textbf{STRANGER}
\par   It is evident, Socrates, that there was no such thing in the then order of nature as the procreation of animals from one another; the earth-born race, of which we hear in story, was the one which existed in those days—they rose again from the ground; and of this tradition, which is now-a-days often unduly discredited, our ancestors, who were nearest in point of time to the end of the last period and came into being at the beginning of this, are to us the heralds. And mark how consistent the sequel of the tale is; after the return of age to youth, follows the return of the dead, who are lying in the earth, to life; simultaneously with the reversal of the world the wheel of their generation has been turned back, and they are put together and rise and live in the opposite order, unless God has carried any of them away to some other lot. According to this tradition they of necessity sprang from the earth and have the name of earth-born, and so the above legend clings to them.

\par \textbf{YOUNG SOCRATES}
\par   Certainly that is quite consistent with what has preceded; but tell me, was the life which you said existed in the reign of Cronos in that cycle of the world, or in this? For the change in the course of the stars and the sun must have occurred in both.

\par \textbf{STRANGER}
\par   I see that you enter into my meaning;—no, that blessed and spontaneous life does not belong to the present cycle of the world, but to the previous one, in which God superintended the whole revolution of the universe; and the several parts the universe were distributed under the rule of certain inferior deities, as is the way in some places still. There were demigods, who were the shepherds of the various species and herds of animals, and each one was in all respects sufficient for those of whom he was the shepherd; neither was there any violence, or devouring of one another, or war or quarrel among them; and I might tell of ten thousand other blessings, which belonged to that dispensation. The reason why the life of man was, as tradition says, spontaneous, is as follows:  In those days God himself was their shepherd, and ruled over them, just as man, who is by comparison a divine being, still rules over the lower animals. Under him there were no forms of government or separate possession of women and children; for all men rose again from the earth, having no memory of the past. And although they had nothing of this sort, the earth gave them fruits in abundance, which grew on trees and shrubs unbidden, and were not planted by the hand of man. And they dwelt naked, and mostly in the open air, for the temperature of their seasons was mild; and they had no beds, but lay on soft couches of grass, which grew plentifully out of the earth. Such was the life of man in the days of Cronos, Socrates; the character of our present life, which is said to be under Zeus, you know from your own experience. Can you, and will you, determine which of them you deem the happier?

\par \textbf{YOUNG SOCRATES}
\par   Impossible.

\par \textbf{STRANGER}
\par   Then shall I determine for you as well as I can?

\par \textbf{YOUNG SOCRATES}
\par   By all means.

\par \textbf{STRANGER}
\par   Suppose that the nurslings of Cronos, having this boundless leisure, and the power of holding intercourse, not only with men, but with the brute creation, had used all these advantages with a view to philosophy, conversing with the brutes as well as with one another, and learning of every nature which was gifted with any special power, and was able to contribute some special experience to the store of wisdom, there would be no difficulty in deciding that they would be a thousand times happier than the men of our own day. Or, again, if they had merely eaten and drunk until they were full, and told stories to one another and to the animals—such stories as are now attributed to them—in this case also, as I should imagine, the answer would be easy. But until some satisfactory witness can be found of the love of that age for knowledge and discussion, we had better let the matter drop, and give the reason why we have unearthed this tale, and then we shall be able to get on. In the fulness of time, when the change was to take place, and the earth-born race had all perished, and every soul had completed its proper cycle of births and been sown in the earth her appointed number of times, the pilot of the universe let the helm go, and retired to his place of view; and then Fate and innate desire reversed the motion of the world. Then also all the inferior deities who share the rule of the supreme power, being informed of what was happening, let go the parts of the world which were under their control. And the world turning round with a sudden shock, being impelled in an opposite direction from beginning to end, was shaken by a mighty earthquake, which wrought a new destruction of all manner of animals. Afterwards, when sufficient time had elapsed, the tumult and confusion and earthquake ceased, and the universal creature, once more at peace, attained to a calm, and settled down into his own orderly and accustomed course, having the charge and rule of himself and of all the creatures which are contained in him, and executing, as far as he remembered them, the instructions of his Father and Creator, more precisely at first, but afterwords with less exactness. The reason of the falling off was the admixture of matter in him; this was inherent in the primal nature, which was full of disorder, until attaining to the present order. From God, the constructor, the world received all that is good in him, but from a previous state came elements of evil and unrighteousness, which, thence derived, first of all passed into the world, and were then transmitted to the animals. While the world was aided by the pilot in nurturing the animals, the evil was small, and great the good which he produced, but after the separation, when the world was let go, at first all proceeded well enough; but, as time went on, there was more and more forgetting, and the old discord again held sway and burst forth in full glory; and at last small was the good, and great was the admixture of evil, and there was a danger of universal ruin to the world, and to the things contained in him. Wherefore God, the orderer of all, in his tender care, seeing that the world was in great straits, and fearing that all might be dissolved in the storm and disappear in infinite chaos, again seated himself at the helm; and bringing back the elements which had fallen into dissolution and disorder to the motion which had prevailed under his dispensation, he set them in order and restored them, and made the world imperishable and immortal. And this is the whole tale, of which the first part will suffice to illustrate the nature of the king. For when the world turned towards the present cycle of generation, the age of man again stood still, and a change opposite to the previous one was the result. The small creatures which had almost disappeared grew in and stature, and the newly-born children of the earth became grey and died and sank into the earth again. All things changed, imitating and following the condition of the universe, and of necessity agreeing with that in their mode of conception and generation and nurture; for no animal was any longer allowed to come into being in the earth through the agency of other creative beings, but as the world was ordained to be the lord of his own progress, in like manner the parts were ordained to grow and generate and give nourishment, as far as they could, of themselves, impelled by a similar movement. And so we have arrived at the real end of this discourse; for although there might be much to tell of the lower animals, and of the condition out of which they changed and of the causes of the change, about men there is not much, and that little is more to the purpose. Deprived of the care of God, who had possessed and tended them, they were left helpless and defenceless, and were torn in pieces by the beasts, who were naturally fierce and had now grown wild. And in the first ages they were still without skill or resource; the food which once grew spontaneously had failed, and as yet they knew not how to procure it, because they had never felt the pressure of necessity. For all these reasons they were in a great strait; wherefore also the gifts spoken of in the old tradition were imparted to man by the gods, together with so much teaching and education as was indispensable; fire was given to them by Prometheus, the arts by Hephaestus and his fellow-worker, Athene, seeds and plants by others. From these is derived all that has helped to frame human life; since the care of the Gods, as I was saying, had now failed men, and they had to order their course of life for themselves, and were their own masters, just like the universal creature whom they imitate and follow, ever changing, as he changes, and ever living and growing, at one time in one manner, and at another time in another. Enough of the story, which may be of use in showing us how greatly we erred in the delineation of the king and the statesman in our previous discourse.

\par \textbf{YOUNG SOCRATES}
\par   What was this great error of which you speak?

\par \textbf{STRANGER}
\par   There were two; the first a lesser one, the other was an error on a much larger and grander scale.

\par \textbf{YOUNG SOCRATES}
\par   What do you mean?

\par \textbf{STRANGER}
\par   I mean to say that when we were asked about a king and statesman of the present cycle and generation, we told of a shepherd of a human flock who belonged to the other cycle, and of one who was a god when he ought to have been a man; and this a great error. Again, we declared him to be the ruler of the entire State, without explaining how:  this was not the whole truth, nor very intelligible; but still it was true, and therefore the second error was not so great as the first.

\par \textbf{YOUNG SOCRATES}
\par   Very good.

\par \textbf{STRANGER}
\par   Before we can expect to have a perfect description of the statesman we must define the nature of his office.

\par \textbf{YOUNG SOCRATES}
\par   Certainly.

\par \textbf{STRANGER}
\par   And the myth was introduced in order to show, not only that all others are rivals of the true shepherd who is the object of our search, but in order that we might have a clearer view of him who is alone worthy to receive this appellation, because he alone of shepherds and herdsmen, according to the image which we have employed, has the care of human beings.

\par \textbf{YOUNG SOCRATES}
\par   Very true.

\par \textbf{STRANGER}
\par   And I cannot help thinking, Socrates, that the form of the divine shepherd is even higher than that of a king; whereas the statesmen who are now on earth seem to be much more like their subjects in character, and much more nearly to partake of their breeding and education.

\par \textbf{YOUNG SOCRATES}
\par   Certainly.

\par \textbf{STRANGER}
\par   Still they must be investigated all the same, to see whether, like the divine shepherd, they are above their subjects or on a level with them.

\par \textbf{YOUNG SOCRATES}
\par   Of course.

\par \textbf{STRANGER}
\par   To resume: —Do you remember that we spoke of a command-for-self exercised over animals, not singly but collectively, which we called the art of rearing a herd?

\par \textbf{YOUNG SOCRATES}
\par   Yes, I remember.

\par \textbf{STRANGER}
\par   There, somewhere, lay our error; for we never included or mentioned the Statesman; and we did not observe that he had no place in our nomenclature.

\par \textbf{YOUNG SOCRATES}
\par   How was that?

\par \textbf{STRANGER}
\par   All other herdsmen 'rear' their herds, but this is not a suitable term to apply to the Statesman; we should use a name which is common to them all.

\par \textbf{YOUNG SOCRATES}
\par   True, if there be such a name.

\par \textbf{STRANGER}
\par   Why, is not 'care' of herds applicable to all? For this implies no feeding, or any special duty; if we say either 'tending' the herds, or 'managing' the herds, or 'having the care' of them, the same word will include all, and then we may wrap up the Statesman with the rest, as the argument seems to require.

\par \textbf{YOUNG SOCRATES}
\par   Quite right; but how shall we take the next step in the division?

\par \textbf{STRANGER}
\par   As before we divided the art of 'rearing' herds accordingly as they were land or water herds, winged and wingless, mixing or not mixing the breed, horned and hornless, so we may divide by these same differences the 'tending' of herds, comprehending in our definition the kingship of to-day and the rule of Cronos.

\par \textbf{YOUNG SOCRATES}
\par   That is clear; but I still ask, what is to follow.

\par \textbf{STRANGER}
\par   If the word had been 'managing' herds, instead of feeding or rearing them, no one would have argued that there was no care of men in the case of the politician, although it was justly contended, that there was no human art of feeding them which was worthy of the name, or at least, if there were, many a man had a prior and greater right to share in such an art than any king.

\par \textbf{YOUNG SOCRATES}
\par   True.

\par \textbf{STRANGER}
\par   But no other art or science will have a prior or better right than the royal science to care for human society and to rule over men in general.

\par \textbf{YOUNG SOCRATES}
\par   Quite true.

\par \textbf{STRANGER}
\par   In the next place, Socrates, we must surely notice that a great error was committed at the end of our analysis.

\par \textbf{YOUNG SOCRATES}
\par   What was it?

\par \textbf{STRANGER}
\par   Why, supposing we were ever so sure that there is such an art as the art of rearing or feeding bipeds, there was no reason why we should call this the royal or political art, as though there were no more to be said.

\par \textbf{YOUNG SOCRATES}
\par   Certainly not.

\par \textbf{STRANGER}
\par   Our first duty, as we were saying, was to remodel the name, so as to have the notion of care rather than of feeding, and then to divide, for there may be still considerable divisions.

\par \textbf{YOUNG SOCRATES}
\par   How can they be made?

\par \textbf{STRANGER}
\par   First, by separating the divine shepherd from the human guardian or manager.

\par \textbf{YOUNG SOCRATES}
\par   True.

\par \textbf{STRANGER}
\par   And the art of management which is assigned to man would again have to be subdivided.

\par \textbf{YOUNG SOCRATES}
\par   On what principle?

\par \textbf{STRANGER}
\par   On the principle of voluntary and compulsory.

\par \textbf{YOUNG SOCRATES}
\par   Why?

\par \textbf{STRANGER}
\par   Because, if I am not mistaken, there has been an error here; for our simplicity led us to rank king and tyrant together, whereas they are utterly distinct, like their modes of government.

\par \textbf{YOUNG SOCRATES}
\par   True.

\par \textbf{STRANGER}
\par   Then, now, as I said, let us make the correction and divide human care into two parts, on the principle of voluntary and compulsory.

\par \textbf{YOUNG SOCRATES}
\par   Certainly.

\par \textbf{STRANGER}
\par   And if we call the management of violent rulers tyranny, and the voluntary management of herds of voluntary bipeds politics, may we not further assert that he who has this latter art of management is the true king and statesman?

\par \textbf{YOUNG SOCRATES}
\par   I think, Stranger, that we have now completed the account of the Statesman.

\par \textbf{STRANGER}
\par   Would that we had, Socrates, but I have to satisfy myself as well as you; and in my judgment the figure of the king is not yet perfected; like statuaries who, in their too great haste, having overdone the several parts of their work, lose time in cutting them down, so too we, partly out of haste, partly out of a magnanimous desire to expose our former error, and also because we imagined that a king required grand illustrations, have taken up a marvellous lump of fable, and have been obliged to use more than was necessary. This made us discourse at large, and, nevertheless, the story never came to an end. And our discussion might be compared to a picture of some living being which had been fairly drawn in outline, but had not yet attained the life and clearness which is given by the blending of colours. Now to intelligent persons a living being had better be delineated by language and discourse than by any painting or work of art:  to the duller sort by works of art.

\par \textbf{YOUNG SOCRATES}
\par   Very true; but what is the imperfection which still remains? I wish that you would tell me.

\par \textbf{STRANGER}
\par   The higher ideas, my dear friend, can hardly be set forth except through the medium of examples; every man seems to know all things in a dreamy sort of way, and then again to wake up and to know nothing.

\par \textbf{YOUNG SOCRATES}
\par   What do you mean?

\par \textbf{STRANGER}
\par   I fear that I have been unfortunate in raising a question about our experience of knowledge.

\par \textbf{YOUNG SOCRATES}
\par   Why so?

\par \textbf{STRANGER}
\par   Why, because my 'example' requires the assistance of another example.

\par \textbf{YOUNG SOCRATES}
\par   Proceed; you need not fear that I shall tire.

\par \textbf{STRANGER}
\par   I will proceed, finding, as I do, such a ready listener in you:  when children are beginning to know their letters—

\par \textbf{YOUNG SOCRATES}
\par   What are you going to say?

\par \textbf{STRANGER}
\par   That they distinguish the several letters well enough in very short and easy syllables, and are able to tell them correctly.

\par \textbf{YOUNG SOCRATES}
\par   Certainly.

\par \textbf{STRANGER}
\par   Whereas in other syllables they do not recognize them, and think and speak falsely of them.

\par \textbf{YOUNG SOCRATES}
\par   Very true.

\par \textbf{STRANGER}
\par   Will not the best and easiest way of bringing them to a knowledge of what they do not as yet know be—

\par \textbf{YOUNG SOCRATES}
\par   Be what?

\par \textbf{STRANGER}
\par   To refer them first of all to cases in which they judge correctly about the letters in question, and then to compare these with the cases in which they do not as yet know, and to show them that the letters are the same, and have the same character in both combinations, until all cases in which they are right have been placed side by side with all cases in which they are wrong. In this way they have examples, and are made to learn that each letter in every combination is always the same and not another, and is always called by the same name.

\par \textbf{YOUNG SOCRATES}
\par   Certainly.

\par \textbf{STRANGER}
\par   Are not examples formed in this manner? We take a thing and compare it with another distinct instance of the same thing, of which we have a right conception, and out of the comparison there arises one true notion, which includes both of them.

\par \textbf{YOUNG SOCRATES}
\par   Exactly.

\par \textbf{STRANGER}
\par   Can we wonder, then, that the soul has the same uncertainty about the alphabet of things, and sometimes and in some cases is firmly fixed by the truth in each particular, and then, again, in other cases is altogether at sea; having somehow or other a correct notion of combinations; but when the elements are transferred into the long and difficult language (syllables) of facts, is again ignorant of them?

\par \textbf{YOUNG SOCRATES}
\par   There is nothing wonderful in that.

\par \textbf{STRANGER}
\par   Could any one, my friend, who began with false opinion ever expect to arrive even at a small portion of truth and to attain wisdom?

\par \textbf{YOUNG SOCRATES}
\par   Hardly.

\par \textbf{STRANGER}
\par   Then you and I will not be far wrong in trying to see the nature of example in general in a small and particular instance; afterwards from lesser things we intend to pass to the royal class, which is the highest form of the same nature, and endeavour to discover by rules of art what the management of cities is; and then the dream will become a reality to us.

\par \textbf{YOUNG SOCRATES}
\par   Very true.

\par \textbf{STRANGER}
\par   Then, once more, let us resume the previous argument, and as there were innumerable rivals of the royal race who claim to have the care of states, let us part them all off, and leave him alone; and, as I was saying, a model or example of this process has first to be framed.

\par \textbf{YOUNG SOCRATES}
\par   Exactly.

\par \textbf{STRANGER}
\par   What model is there which is small, and yet has any analogy with the political occupation? Suppose, Socrates, that if we have no other example at hand, we choose weaving, or, more precisely, weaving of wool—this will be quite enough, without taking the whole of weaving, to illustrate our meaning?

\par \textbf{YOUNG SOCRATES}
\par   Certainly.

\par \textbf{STRANGER}
\par   Why should we not apply to weaving the same processes of division and subdivision which we have already applied to other classes; going once more as rapidly as we can through all the steps until we come to that which is needed for our purpose?

\par \textbf{YOUNG SOCRATES}
\par   How do you mean?

\par \textbf{STRANGER}
\par   I shall reply by actually performing the process.

\par \textbf{YOUNG SOCRATES}
\par   Very good.

\par \textbf{STRANGER}
\par   All things which we make or acquire are either creative or preventive; of the preventive class are antidotes, divine and human, and also defences; and defences are either military weapons or protections; and protections are veils, and also shields against heat and cold, and shields against heat and cold are shelters and coverings; and coverings are blankets and garments; and garments are some of them in one piece, and others of them are made in several parts; and of these latter some are stitched, others are fastened and not stitched; and of the not stitched, some are made of the sinews of plants, and some of hair; and of these, again, some are cemented with water and earth, and others are fastened together by themselves. And these last defences and coverings which are fastened together by themselves are called clothes, and the art which superintends them we may call, from the nature of the operation, the art of clothing, just as before the art of the Statesman was derived from the State; and may we not say that the art of weaving, at least that largest portion of it which was concerned with the making of clothes, differs only in name from this art of clothing, in the same way that, in the previous case, the royal science differed from the political?

\par \textbf{YOUNG SOCRATES}
\par   Most true.

\par \textbf{STRANGER}
\par   In the next place, let us make the reflection, that the art of weaving clothes, which an incompetent person might fancy to have been sufficiently described, has been separated off from several others which are of the same family, but not from the co-operative arts.

\par \textbf{YOUNG SOCRATES}
\par   And which are the kindred arts?

\par \textbf{STRANGER}
\par   I see that I have not taken you with me. So I think that we had better go backwards, starting from the end. We just now parted off from the weaving of clothes, the making of blankets, which differ from each other in that one is put under and the other is put around:  and these are what I termed kindred arts.

\par \textbf{YOUNG SOCRATES}
\par   I understand.

\par \textbf{STRANGER}
\par   And we have subtracted the manufacture of all articles made of flax and cords, and all that we just now metaphorically termed the sinews of plants, and we have also separated off the process of felting and the putting together of materials by stitching and sewing, of which the most important part is the cobbler's art.

\par \textbf{YOUNG SOCRATES}
\par   Precisely.

\par \textbf{STRANGER}
\par   Then we separated off the currier's art, which prepared coverings in entire pieces, and the art of sheltering, and subtracted the various arts of making water-tight which are employed in building, and in general in carpentering, and in other crafts, and all such arts as furnish impediments to thieving and acts of violence, and are concerned with making the lids of boxes and the fixing of doors, being divisions of the art of joining; and we also cut off the manufacture of arms, which is a section of the great and manifold art of making defences; and we originally began by parting off the whole of the magic art which is concerned with antidotes, and have left, as would appear, the very art of which we were in search, the art of protection against winter cold, which fabricates woollen defences, and has the name of weaving.

\par \textbf{YOUNG SOCRATES}
\par   Very true.

\par \textbf{STRANGER}
\par   Yes, my boy, but that is not all; for the first process to which the material is subjected is the opposite of weaving.

\par \textbf{YOUNG SOCRATES}
\par   How so?

\par \textbf{STRANGER}
\par   Weaving is a sort of uniting?

\par \textbf{YOUNG SOCRATES}
\par   Yes.

\par \textbf{STRANGER}
\par   But the first process is a separation of the clotted and matted fibres?

\par \textbf{YOUNG SOCRATES}
\par   What do you mean?

\par \textbf{STRANGER}
\par   I mean the work of the carder's art; for we cannot say that carding is weaving, or that the carder is a weaver.

\par \textbf{YOUNG SOCRATES}
\par   Certainly not.

\par \textbf{STRANGER}
\par   Again, if a person were to say that the art of making the warp and the woof was the art of weaving, he would say what was paradoxical and false.

\par \textbf{YOUNG SOCRATES}
\par   To be sure.

\par \textbf{STRANGER}
\par   Shall we say that the whole art of the fuller or of the mender has nothing to do with the care and treatment of clothes, or are we to regard all these as arts of weaving?

\par \textbf{YOUNG SOCRATES}
\par   Certainly not.

\par \textbf{STRANGER}
\par   And yet surely all these arts will maintain that they are concerned with the treatment and production of clothes; they will dispute the exclusive prerogative of weaving, and though assigning a larger sphere to that, will still reserve a considerable field for themselves.

\par \textbf{YOUNG SOCRATES}
\par   Very true.

\par \textbf{STRANGER}
\par   Besides these, there are the arts which make tools and instruments of weaving, and which will claim at least to be co-operative causes in every work of the weaver.

\par \textbf{YOUNG SOCRATES}
\par   Most true.

\par \textbf{STRANGER}
\par   Well, then, suppose that we define weaving, or rather that part of it which has been selected by us, to be the greatest and noblest of arts which are concerned with woollen garments—shall we be right? Is not the definition, although true, wanting in clearness and completeness; for do not all those other arts require to be first cleared away?

\par \textbf{YOUNG SOCRATES}
\par   True.

\par \textbf{STRANGER}
\par   Then the next thing will be to separate them, in order that the argument may proceed in a regular manner?

\par \textbf{YOUNG SOCRATES}
\par   By all means.

\par \textbf{STRANGER}
\par   Let us consider, in the first place, that there are two kinds of arts entering into everything which we do.

\par \textbf{YOUNG SOCRATES}
\par   What are they?

\par \textbf{STRANGER}
\par   The one kind is the conditional or co-operative, the other the principal cause.

\par \textbf{YOUNG SOCRATES}
\par   What do you mean?

\par \textbf{STRANGER}
\par   The arts which do not manufacture the actual thing, but which furnish the necessary tools for the manufacture, without which the several arts could not fulfil their appointed work, are co-operative; but those which make the things themselves are causal.

\par \textbf{YOUNG SOCRATES}
\par   A very reasonable distinction.

\par \textbf{STRANGER}
\par   Thus the arts which make spindles, combs, and other instruments of the production of clothes, may be called co-operative, and those which treat and fabricate the things themselves, causal.

\par \textbf{YOUNG SOCRATES}
\par   Very true.

\par \textbf{STRANGER}
\par   The arts of washing and mending, and the other preparatory arts which belong to the causal class, and form a division of the great art of adornment, may be all comprehended under what we call the fuller's art.

\par \textbf{YOUNG SOCRATES}
\par   Very good.

\par \textbf{STRANGER}
\par   Carding and spinning threads and all the parts of the process which are concerned with the actual manufacture of a woollen garment form a single art, which is one of those universally acknowledged,—the art of working in wool.

\par \textbf{YOUNG SOCRATES}
\par   To be sure.

\par \textbf{STRANGER}
\par   Of working in wool, again, there are two divisions, and both these are parts of two arts at once.

\par \textbf{YOUNG SOCRATES}
\par   How is that?

\par \textbf{STRANGER}
\par   Carding and one half of the use of the comb, and the other processes of wool-working which separate the composite, may be classed together as belonging both to the art of wool-working, and also to one of the two great arts which are of universal application—the art of composition and the art of division.

\par \textbf{YOUNG SOCRATES}
\par   Yes.

\par \textbf{STRANGER}
\par   To the latter belong carding and the other processes of which I was just now speaking; the art of discernment or division in wool and yarn, which is effected in one manner with the comb and in another with the hands, is variously described under all the names which I just now mentioned.

\par \textbf{YOUNG SOCRATES}
\par   Very true.

\par \textbf{STRANGER}
\par   Again, let us take some process of wool-working which is also a portion of the art of composition, and, dismissing the elements of division which we found there, make two halves, one on the principle of composition, and the other on the principle of division.

\par \textbf{YOUNG SOCRATES}
\par   Let that be done.

\par \textbf{STRANGER}
\par   And once more, Socrates, we must divide the part which belongs at once both to wool-working and composition, if we are ever to discover satisfactorily the aforesaid art of weaving.

\par \textbf{YOUNG SOCRATES}
\par   We must.

\par \textbf{STRANGER}
\par   Yes, certainly, and let us call one part of the art the art of twisting threads, the other the art of combining them.

\par \textbf{YOUNG SOCRATES}
\par   Do I understand you, in speaking of twisting, to be referring to manufacture of the warp?

\par \textbf{STRANGER}
\par   Yes, and of the woof too; how, if not by twisting, is the woof made?

\par \textbf{YOUNG SOCRATES}
\par   There is no other way.

\par \textbf{STRANGER}
\par   Then suppose that you define the warp and the woof, for I think that the definition will be of use to you.

\par \textbf{YOUNG SOCRATES}
\par   How shall I define them?

\par \textbf{STRANGER}
\par   As thus:  A piece of carded wool which is drawn out lengthwise and breadthwise is said to be pulled out.

\par \textbf{YOUNG SOCRATES}
\par   Yes.

\par \textbf{STRANGER}
\par   And the wool thus prepared, when twisted by the spindle, and made into a firm thread, is called the warp, and the art which regulates these operations the art of spinning the warp.

\par \textbf{YOUNG SOCRATES}
\par   True.

\par \textbf{STRANGER}
\par   And the threads which are more loosely spun, having a softness proportioned to the intertexture of the warp and to the degree of force used in dressing the cloth,—the threads which are thus spun are called the woof, and the art which is set over them may be called the art of spinning the woof.

\par \textbf{YOUNG SOCRATES}
\par   Very true.

\par \textbf{STRANGER}
\par   And, now, there can be no mistake about the nature of the part of weaving which we have undertaken to define. For when that part of the art of composition which is employed in the working of wool forms a web by the regular intertexture of warp and woof, the entire woven substance is called by us a woollen garment, and the art which presides over this is the art of weaving.

\par \textbf{YOUNG SOCRATES}
\par   Very true.

\par \textbf{STRANGER}
\par   But why did we not say at once that weaving is the art of entwining warp and woof, instead of making a long and useless circuit?

\par \textbf{YOUNG SOCRATES}
\par   I thought, Stranger, that there was nothing useless in what was said.

\par \textbf{STRANGER}
\par   Very likely, but you may not always think so, my sweet friend; and in case any feeling of dissatisfaction should hereafter arise in your mind, as it very well may, let me lay down a principle which will apply to arguments in general.

\par \textbf{YOUNG SOCRATES}
\par   Proceed.

\par \textbf{STRANGER}
\par   Let us begin by considering the whole nature of excess and defect, and then we shall have a rational ground on which we may praise or blame too much length or too much shortness in discussions of this kind.

\par \textbf{YOUNG SOCRATES}
\par   Let us do so.

\par \textbf{STRANGER}
\par   The points on which I think that we ought to dwell are the following: —

\par \textbf{YOUNG SOCRATES}
\par   What?

\par \textbf{STRANGER}
\par   Length and shortness, excess and defect; with all of these the art of measurement is conversant.

\par \textbf{YOUNG SOCRATES}
\par   Yes.

\par \textbf{STRANGER}
\par   And the art of measurement has to be divided into two parts, with a view to our present purpose.

\par \textbf{YOUNG SOCRATES}
\par   Where would you make the division?

\par \textbf{STRANGER}
\par   As thus:  I would make two parts, one having regard to the relativity of greatness and smallness to each other; and there is another, without which the existence of production would be impossible.

\par \textbf{YOUNG SOCRATES}
\par   What do you mean?

\par \textbf{STRANGER}
\par   Do you not think that it is only natural for the greater to be called greater with reference to the less alone, and the less less with reference to the greater alone?

\par \textbf{YOUNG SOCRATES}
\par   Yes.

\par \textbf{STRANGER}
\par   Well, but is there not also something exceeding and exceeded by the principle of the mean, both in speech and action, and is not this a reality, and the chief mark of difference between good and bad men?

\par \textbf{YOUNG SOCRATES}
\par   Plainly.

\par \textbf{STRANGER}
\par   Then we must suppose that the great and small exist and are discerned in both these ways, and not, as we were saying before, only relatively to one another, but there must also be another comparison of them with the mean or ideal standard; would you like to hear the reason why?

\par \textbf{YOUNG SOCRATES}
\par   Certainly.

\par \textbf{STRANGER}
\par   If we assume the greater to exist only in relation to the less, there will never be any comparison of either with the mean.

\par \textbf{YOUNG SOCRATES}
\par   True.

\par \textbf{STRANGER}
\par   And would not this doctrine be the ruin of all the arts and their creations; would not the art of the Statesman and the aforesaid art of weaving disappear? For all these arts are on the watch against excess and defect, not as unrealities, but as real evils, which occasion a difficulty in action; and the excellence or beauty of every work of art is due to this observance of measure.

\par \textbf{YOUNG SOCRATES}
\par   Certainly.

\par \textbf{STRANGER}
\par   But if the science of the Statesman disappears, the search for the royal science will be impossible.

\par \textbf{YOUNG SOCRATES}
\par   Very true.

\par \textbf{STRANGER}
\par   Well, then, as in the case of the Sophist we extorted the inference that not-being had an existence, because here was the point at which the argument eluded our grasp, so in this we must endeavour to show that the greater and less are not only to be measured with one another, but also have to do with the production of the mean; for if this is not admitted, neither a statesman nor any other man of action can be an undisputed master of his science.

\par \textbf{YOUNG SOCRATES}
\par   Yes, we must certainly do again what we did then.

\par \textbf{STRANGER}
\par   But this, Socrates, is a greater work than the other, of which we only too well remember the length. I think, however, that we may fairly assume something of this sort—

\par \textbf{YOUNG SOCRATES}
\par   What?

\par \textbf{STRANGER}
\par   That we shall some day require this notion of a mean with a view to the demonstration of absolute truth; meanwhile, the argument that the very existence of the arts must be held to depend on the possibility of measuring more or less, not only with one another, but also with a view to the attainment of the mean, seems to afford a grand support and satisfactory proof of the doctrine which we are maintaining; for if there are arts, there is a standard of measure, and if there is a standard of measure, there are arts; but if either is wanting, there is neither.

\par \textbf{YOUNG SOCRATES}
\par   True; and what is the next step?

\par \textbf{STRANGER}
\par   The next step clearly is to divide the art of measurement into two parts, as we have said already, and to place in the one part all the arts which measure number, length, depth, breadth, swiftness with their opposites; and to have another part in which they are measured with the mean, and the fit, and the opportune, and the due, and with all those words, in short, which denote a mean or standard removed from the extremes.

\par \textbf{YOUNG SOCRATES}
\par   Here are two vast divisions, embracing two very different spheres.

\par \textbf{STRANGER}
\par   There are many accomplished men, Socrates, who say, believing themselves to speak wisely, that the art of measurement is universal, and has to do with all things. And this means what we are now saying; for all things which come within the province of art do certainly in some sense partake of measure. But these persons, because they are not accustomed to distinguish classes according to real forms, jumble together two widely different things, relation to one another, and to a standard, under the idea that they are the same, and also fall into the converse error of dividing other things not according to their real parts. Whereas the right way is, if a man has first seen the unity of things, to go on with the enquiry and not desist until he has found all the differences contained in it which form distinct classes; nor again should he be able to rest contented with the manifold diversities which are seen in a multitude of things until he has comprehended all of them that have any affinity within the bounds of one similarity and embraced them within the reality of a single kind. But we have said enough on this head, and also of excess and defect; we have only to bear in mind that two divisions of the art of measurement have been discovered which are concerned with them, and not forget what they are.

\par \textbf{YOUNG SOCRATES}
\par   We will not forget.

\par \textbf{STRANGER}
\par   And now that this discussion is completed, let us go on to consider another question, which concerns not this argument only but the conduct of such arguments in general.

\par \textbf{YOUNG SOCRATES}
\par   What is this new question?

\par \textbf{STRANGER}
\par   Take the case of a child who is engaged in learning his letters:  when he is asked what letters make up a word, should we say that the question is intended to improve his grammatical knowledge of that particular word, or of all words?

\par \textbf{YOUNG SOCRATES}
\par   Clearly, in order that he may have a better knowledge of all words.

\par \textbf{STRANGER}
\par   And is our enquiry about the Statesman intended only to improve our knowledge of politics, or our power of reasoning generally?

\par \textbf{YOUNG SOCRATES}
\par   Clearly, as in the former example, the purpose is general.

\par \textbf{STRANGER}
\par   Still less would any rational man seek to analyse the notion of weaving for its own sake. But people seem to forget that some things have sensible images, which are readily known, and can be easily pointed out when any one desires to answer an enquirer without any trouble or argument; whereas the greatest and highest truths have no outward image of themselves visible to man, which he who wishes to satisfy the soul of the enquirer can adapt to the eye of sense (compare Phaedr. ), and therefore we ought to train ourselves to give and accept a rational account of them; for immaterial things, which are the noblest and greatest, are shown only in thought and idea, and in no other way, and all that we are now saying is said for the sake of them. Moreover, there is always less difficulty in fixing the mind on small matters than on great.

\par \textbf{YOUNG SOCRATES}
\par   Very good.

\par \textbf{STRANGER}
\par   Let us call to mind the bearing of all this.

\par \textbf{YOUNG SOCRATES}
\par   What is it?

\par \textbf{STRANGER}
\par   I wanted to get rid of any impression of tediousness which we may have experienced in the discussion about weaving, and the reversal of the universe, and in the discussion concerning the Sophist and the being of not-being. I know that they were felt to be too long, and I reproached myself with this, fearing that they might be not only tedious but irrelevant; and all that I have now said is only designed to prevent the recurrence of any such disagreeables for the future.

\par \textbf{YOUNG SOCRATES}
\par   Very good. Will you proceed?

\par \textbf{STRANGER}
\par   Then I would like to observe that you and I, remembering what has been said, should praise or blame the length or shortness of discussions, not by comparing them with one another, but with what is fitting, having regard to the part of measurement, which, as we said, was to be borne in mind.

\par \textbf{YOUNG SOCRATES}
\par   Very true.

\par \textbf{STRANGER}
\par   And yet, not everything is to be judged even with a view to what is fitting; for we should only want such a length as is suited to give pleasure, if at all, as a secondary matter; and reason tells us, that we should be contented to make the ease or rapidity of an enquiry, not our first, but our second object; the first and highest of all being to assert the great method of division according to species—whether the discourse be shorter or longer is not to the point. No offence should be taken at length, but the longer and shorter are to be employed indifferently, according as either of them is better calculated to sharpen the wits of the auditors. Reason would also say to him who censures the length of discourses on such occasions and cannot away with their circumlocution, that he should not be in such a hurry to have done with them, when he can only complain that they are tedious, but he should prove that if they had been shorter they would have made those who took part in them better dialecticians, and more capable of expressing the truth of things; about any other praise and blame, he need not trouble himself—he should pretend not to hear them. But we have had enough of this, as you will probably agree with me in thinking. Let us return to our Statesman, and apply to his case the aforesaid example of weaving.

\par \textbf{YOUNG SOCRATES}
\par   Very good;—let us do as you say.

\par \textbf{STRANGER}
\par   The art of the king has been separated from the similar arts of shepherds, and, indeed, from all those which have to do with herds at all. There still remain, however, of the causal and co-operative arts those which are immediately concerned with States, and which must first be distinguished from one another.

\par \textbf{YOUNG SOCRATES}
\par   Very good.

\par \textbf{STRANGER}
\par   You know that these arts cannot easily be divided into two halves; the reason will be very evident as we proceed.

\par \textbf{YOUNG SOCRATES}
\par   Then we had better do so.

\par \textbf{STRANGER}
\par   We must carve them like a victim into members or limbs, since we cannot bisect them. (Compare Phaedr.) For we certainly should divide everything into as few parts as possible.

\par \textbf{YOUNG SOCRATES}
\par   What is to be done in this case?

\par \textbf{STRANGER}
\par   What we did in the example of weaving—all those arts which furnish the tools were regarded by us as co-operative.

\par \textbf{YOUNG SOCRATES}
\par   Yes.

\par \textbf{STRANGER}
\par   So now, and with still more reason, all arts which make any implement in a State, whether great or small, may be regarded by us as co-operative, for without them neither State nor Statesmanship would be possible; and yet we are not inclined to say that any of them is a product of the kingly art.

\par \textbf{YOUNG SOCRATES}
\par   No, indeed.

\par \textbf{STRANGER}
\par   The task of separating this class from others is not an easy one; for there is plausibility in saying that anything in the world is the instrument of doing something. But there is another class of possessions in a city, of which I have a word to say.

\par \textbf{YOUNG SOCRATES}
\par   What class do you mean?

\par \textbf{STRANGER}
\par   A class which may be described as not having this power; that is to say, not like an instrument, framed for production, but designed for the preservation of that which is produced.

\par \textbf{YOUNG SOCRATES}
\par   To what do you refer?

\par \textbf{STRANGER}
\par   To the class of vessels, as they are comprehensively termed, which are constructed for the preservation of things moist and dry, of things prepared in the fire or out of the fire; this is a very large class, and has, if I am not mistaken, literally nothing to do with the royal art of which we are in search.

\par \textbf{YOUNG SOCRATES}
\par   Certainly not.

\par \textbf{STRANGER}
\par   There is also a third class of possessions to be noted, different from these and very extensive, moving or resting on land or water, honourable and also dishonourable. The whole of this class has one name, because it is intended to be sat upon, being always a seat for something.

\par \textbf{YOUNG SOCRATES}
\par   What is it?

\par \textbf{STRANGER}
\par   A vehicle, which is certainly not the work of the Statesman, but of the carpenter, potter, and coppersmith.

\par \textbf{YOUNG SOCRATES}
\par   I understand.

\par \textbf{STRANGER}
\par   And is there not a fourth class which is again different, and in which most of the things formerly mentioned are contained,—every kind of dress, most sorts of arms, walls and enclosures, whether of earth or stone, and ten thousand other things? all of which being made for the sake of defence, may be truly called defences, and are for the most part to be regarded as the work of the builder or of the weaver, rather than of the Statesman.

\par \textbf{YOUNG SOCRATES}
\par   Certainly.

\par \textbf{STRANGER}
\par   Shall we add a fifth class, of ornamentation and drawing, and of the imitations produced by drawing and music, which are designed for amusement only, and may be fairly comprehended under one name?

\par \textbf{YOUNG SOCRATES}
\par   What is it?

\par \textbf{STRANGER}
\par   Plaything is the name.

\par \textbf{YOUNG SOCRATES}
\par   Certainly.

\par \textbf{STRANGER}
\par   That one name may be fitly predicated of all of them, for none of these things have a serious purpose—amusement is their sole aim.

\par \textbf{YOUNG SOCRATES}
\par   That again I understand.

\par \textbf{STRANGER}
\par   Then there is a class which provides materials for all these, out of which and in which the arts already mentioned fabricate their works;—this manifold class, I say, which is the creation and offspring of many other arts, may I not rank sixth?

\par \textbf{YOUNG SOCRATES}
\par   What do you mean?

\par \textbf{STRANGER}
\par   I am referring to gold, silver, and other metals, and all that wood-cutting and shearing of every sort provides for the art of carpentry and plaiting; and there is the process of barking and stripping the cuticle of plants, and the currier's art, which strips off the skins of animals, and other similar arts which manufacture corks and papyri and cords, and provide for the manufacture of composite species out of simple kinds—the whole class may be termed the primitive and simple possession of man, and with this the kingly science has no concern at all.

\par \textbf{YOUNG SOCRATES}
\par   True.

\par \textbf{STRANGER}
\par   The provision of food and of all other things which mingle their particles with the particles of the human body, and minister to the body, will form a seventh class, which may be called by the general term of nourishment, unless you have any better name to offer. This, however, appertains rather to the husbandman, huntsman, trainer, doctor, cook, and is not to be assigned to the Statesman's art.

\par \textbf{YOUNG SOCRATES}
\par   Certainly not.

\par \textbf{STRANGER}
\par   These seven classes include nearly every description of property, with the exception of tame animals. Consider;—there was the original material, which ought to have been placed first; next come instruments, vessels, vehicles, defences, playthings, nourishment; small things, which may be included under one of these—as for example, coins, seals and stamps, are omitted, for they have not in them the character of any larger kind which includes them; but some of them may, with a little forcing, be placed among ornaments, and others may be made to harmonize with the class of implements. The art of herding, which has been already divided into parts, will include all property in tame animals, except slaves.

\par \textbf{YOUNG SOCRATES}
\par   Very true.

\par \textbf{STRANGER}
\par   The class of slaves and ministers only remains, and I suspect that in this the real aspirants for the throne, who are the rivals of the king in the formation of the political web, will be discovered; just as spinners, carders, and the rest of them, were the rivals of the weaver. All the others, who were termed co-operators, have been got rid of among the occupations already mentioned, and separated from the royal and political science.

\par \textbf{YOUNG SOCRATES}
\par   I agree.

\par \textbf{STRANGER}
\par   Let us go a little nearer, in order that we may be more certain of the complexion of this remaining class.

\par \textbf{YOUNG SOCRATES}
\par   Let us do so.

\par \textbf{STRANGER}
\par   We shall find from our present point of view that the greatest servants are in a case and condition which is the reverse of what we anticipated.

\par \textbf{YOUNG SOCRATES}
\par   Who are they?

\par \textbf{STRANGER}
\par   Those who have been purchased, and have so become possessions; these are unmistakably slaves, and certainly do not claim royal science.

\par \textbf{YOUNG SOCRATES}
\par   Certainly not.

\par \textbf{STRANGER}
\par   Again, freemen who of their own accord become the servants of the other classes in a State, and who exchange and equalise the products of husbandry and the other arts, some sitting in the market-place, others going from city to city by land or sea, and giving money in exchange for money or for other productions—the money-changer, the merchant, the ship-owner, the retailer, will not put in any claim to statecraft or politics?

\par \textbf{YOUNG SOCRATES}
\par   No; unless, indeed, to the politics of commerce.

\par \textbf{STRANGER}
\par   But surely men whom we see acting as hirelings and serfs, and too happy to turn their hand to anything, will not profess to share in royal science?

\par \textbf{YOUNG SOCRATES}
\par   Certainly not.

\par \textbf{STRANGER}
\par   But what would you say of some other serviceable officials?

\par \textbf{YOUNG SOCRATES}
\par   Who are they, and what services do they perform?

\par \textbf{STRANGER}
\par   There are heralds, and scribes perfected by practice, and divers others who have great skill in various sorts of business connected with the government of states—what shall we call them?

\par \textbf{YOUNG SOCRATES}
\par   They are the officials, and servants of the rulers, as you just now called them, but not themselves rulers.

\par \textbf{STRANGER}
\par   There may be something strange in any servant pretending to be a ruler, and yet I do not think that I could have been dreaming when I imagined that the principal claimants to political science would be found somewhere in this neighbourhood.

\par \textbf{YOUNG SOCRATES}
\par   Very true.

\par \textbf{STRANGER}
\par   Well, let us draw nearer, and try the claims of some who have not yet been tested:  in the first place, there are diviners, who have a portion of servile or ministerial science, and are thought to be the interpreters of the gods to men.

\par \textbf{YOUNG SOCRATES}
\par   True.

\par \textbf{STRANGER}
\par   There is also the priestly class, who, as the law declares, know how to give the gods gifts from men in the form of sacrifices which are acceptable to them, and to ask on our behalf blessings in return from them. Now both these are branches of the servile or ministerial art.

\par \textbf{YOUNG SOCRATES}
\par   Yes, clearly.

\par \textbf{STRANGER}
\par   And here I think that we seem to be getting on the right track; for the priest and the diviner are swollen with pride and prerogative, and they create an awful impression of themselves by the magnitude of their enterprises; in Egypt, the king himself is not allowed to reign, unless he have priestly powers, and if he should be of another class and has thrust himself in, he must get enrolled in the priesthood. In many parts of Hellas, the duty of offering the most solemn propitiatory sacrifices is assigned to the highest magistracies, and here, at Athens, the most solemn and national of the ancient sacrifices are supposed to be celebrated by him who has been chosen by lot to be the King Archon.

\par \textbf{YOUNG SOCRATES}
\par   Precisely.

\par \textbf{STRANGER}
\par   But who are these other kings and priests elected by lot who now come into view followed by their retainers and a vast throng, as the former class disappears and the scene changes?

\par \textbf{YOUNG SOCRATES}
\par   Whom can you mean?

\par \textbf{STRANGER}
\par   They are a strange crew.

\par \textbf{YOUNG SOCRATES}
\par   Why strange?

\par \textbf{STRANGER}
\par   A minute ago I thought that they were animals of every tribe; for many of them are like lions and centaurs, and many more like satyrs and such weak and shifty creatures;—Protean shapes quickly changing into one another's forms and natures; and now, Socrates, I begin to see who they are.

\par \textbf{YOUNG SOCRATES}
\par   Who are they? You seem to be gazing on some strange vision.

\par \textbf{STRANGER}
\par   Yes; every one looks strange when you do not know him; and just now I myself fell into this mistake—at first sight, coming suddenly upon him, I did not recognize the politician and his troop.

\par \textbf{YOUNG SOCRATES}
\par   Who is he?

\par \textbf{STRANGER}
\par   The chief of Sophists and most accomplished of wizards, who must at any cost be separated from the true king or Statesman, if we are ever to see daylight in the present enquiry.

\par \textbf{YOUNG SOCRATES}
\par   That is a hope not lightly to be renounced.

\par \textbf{STRANGER}
\par   Never, if I can help it; and, first, let me ask you a question.

\par \textbf{YOUNG SOCRATES}
\par   What?

\par \textbf{STRANGER}
\par   Is not monarchy a recognized form of government?

\par \textbf{YOUNG SOCRATES}
\par   Yes.

\par \textbf{STRANGER}
\par   And, after monarchy, next in order comes the government of the few?

\par \textbf{YOUNG SOCRATES}
\par   Of course.

\par \textbf{STRANGER}
\par   Is not the third form of government the rule of the multitude, which is called by the name of democracy?

\par \textbf{YOUNG SOCRATES}
\par   Certainly.

\par \textbf{STRANGER}
\par   And do not these three expand in a manner into five, producing out of themselves two other names?

\par \textbf{YOUNG SOCRATES}
\par   What are they?

\par \textbf{YOUNG SOCRATES}
\par   What are they?

\par \textbf{STRANGER}
\par   There is a criterion of voluntary and involuntary, poverty and riches, law and the absence of law, which men now-a-days apply to them; the two first they subdivide accordingly, and ascribe to monarchy two forms and two corresponding names, royalty and tyranny.

\par \textbf{YOUNG SOCRATES}
\par   Very true.

\par \textbf{STRANGER}
\par   And the government of the few they distinguish by the names of aristocracy and oligarchy.

\par \textbf{YOUNG SOCRATES}
\par   Certainly.

\par \textbf{STRANGER}
\par   Democracy alone, whether rigidly observing the laws or not, and whether the multitude rule over the men of property with their consent or against their consent, always in ordinary language has the same name.

\par \textbf{YOUNG SOCRATES}
\par   True.

\par \textbf{STRANGER}
\par   But do you suppose that any form of government which is defined by these characteristics of the one, the few, or the many, of poverty or wealth, of voluntary or compulsory submission, of written law or the absence of law, can be a right one?

\par \textbf{YOUNG SOCRATES}
\par   Why not?

\par \textbf{STRANGER}
\par   Reflect; and follow me.

\par \textbf{YOUNG SOCRATES}
\par   In what direction?

\par \textbf{STRANGER}
\par   Shall we abide by what we said at first, or shall we retract our words?

\par \textbf{YOUNG SOCRATES}
\par   To what do you refer?

\par \textbf{STRANGER}
\par   If I am not mistaken, we said that royal power was a science?

\par \textbf{YOUNG SOCRATES}
\par   Yes.

\par \textbf{STRANGER}
\par   And a science of a peculiar kind, which was selected out of the rest as having a character which is at once judicial and authoritative?

\par \textbf{YOUNG SOCRATES}
\par   Yes.

\par \textbf{STRANGER}
\par   And there was one kind of authority over lifeless things and another other living animals; and so we proceeded in the division step by step up to this point, not losing the idea of science, but unable as yet to determine the nature of the particular science?

\par \textbf{YOUNG SOCRATES}
\par   True.

\par \textbf{STRANGER}
\par   Hence we are led to observe that the distinguishing principle of the State cannot be the few or many, the voluntary or involuntary, poverty or riches; but some notion of science must enter into it, if we are to be consistent with what has preceded.

\par \textbf{YOUNG SOCRATES}
\par   And we must be consistent.

\par \textbf{STRANGER}
\par   Well, then, in which of these various forms of States may the science of government, which is among the greatest of all sciences and most difficult to acquire, be supposed to reside? That we must discover, and then we shall see who are the false politicians who pretend to be politicians but are not, although they persuade many, and shall separate them from the wise king.

\par \textbf{YOUNG SOCRATES}
\par   That, as the argument has already intimated, will be our duty.

\par \textbf{STRANGER}
\par   Do you think that the multitude in a State can attain political science?

\par \textbf{YOUNG SOCRATES}
\par   Impossible.

\par \textbf{STRANGER}
\par   But, perhaps, in a city of a thousand men, there would be a hundred, or say fifty, who could?

\par \textbf{YOUNG SOCRATES}
\par   In that case political science would certainly be the easiest of all sciences; there could not be found in a city of that number as many really first-rate draught-players, if judged by the standard of the rest of Hellas, and there would certainly not be as many kings. For kings we may truly call those who possess royal science, whether they rule or not, as was shown in the previous argument.

\par \textbf{STRANGER}
\par   Thank you for reminding me; and the consequence is that any true form of government can only be supposed to be the government of one, two, or, at any rate, of a few.

\par \textbf{YOUNG SOCRATES}
\par   Certainly.

\par \textbf{STRANGER}
\par   And these, whether they rule with the will, or against the will, of their subjects, with written laws or without written laws, and whether they are poor or rich, and whatever be the nature of their rule, must be supposed, according to our present view, to rule on some scientific principle; just as the physician, whether he cures us against our will or with our will, and whatever be his mode of treatment,—incision, burning, or the infliction of some other pain,—whether he practises out of a book or not out of a book, and whether he be rich or poor, whether he purges or reduces in some other way, or even fattens his patients, is a physician all the same, so long as he exercises authority over them according to rules of art, if he only does them good and heals and saves them. And this we lay down to be the only proper test of the art of medicine, or of any other art of command.

\par \textbf{YOUNG SOCRATES}
\par   Quite true.

\par \textbf{STRANGER}
\par   Then that can be the only true form of government in which the governors are really found to possess science, and are not mere pretenders, whether they rule according to law or without law, over willing or unwilling subjects, and are rich or poor themselves—none of these things can with any propriety be included in the notion of the ruler.

\par \textbf{YOUNG SOCRATES}
\par   True.

\par \textbf{STRANGER}
\par   And whether with a view to the public good they purge the State by killing some, or exiling some; whether they reduce the size of the body corporate by sending out from the hive swarms of citizens, or, by introducing persons from without, increase it; while they act according to the rules of wisdom and justice, and use their power with a view to the general security and improvement, the city over which they rule, and which has these characteristics, may be described as the only true State. All other governments are not genuine or real; but only imitations of this, and some of them are better and some of them are worse; the better are said to be well governed, but they are mere imitations like the others.

\par \textbf{YOUNG SOCRATES}
\par   I agree, Stranger, in the greater part of what you say; but as to their ruling without laws—the expression has a harsh sound.

\par \textbf{STRANGER}
\par   You have been too quick for me, Socrates; I was just going to ask you whether you objected to any of my statements. And now I see that we shall have to consider this notion of there being good government without laws.

\par \textbf{YOUNG SOCRATES}
\par   Certainly.

\par \textbf{STRANGER}
\par   There can be no doubt that legislation is in a manner the business of a king, and yet the best thing of all is not that the law should rule, but that a man should rule supposing him to have wisdom and royal power. Do you see why this is?

\par \textbf{YOUNG SOCRATES}
\par   Why?

\par \textbf{STRANGER}
\par   Because the law does not perfectly comprehend what is noblest and most just for all and therefore cannot enforce what is best. The differences of men and actions, and the endless irregular movements of human things, do not admit of any universal and simple rule. And no art whatsoever can lay down a rule which will last for all time.

\par \textbf{YOUNG SOCRATES}
\par   Of course not.

\par \textbf{STRANGER}
\par   But the law is always striving to make one;—like an obstinate and ignorant tyrant, who will not allow anything to be done contrary to his appointment, or any question to be asked—not even in sudden changes of circumstances, when something happens to be better than what he commanded for some one.

\par \textbf{YOUNG SOCRATES}
\par   Certainly; the law treats us all precisely in the manner which you describe.

\par \textbf{STRANGER}
\par   A perfectly simple principle can never be applied to a state of things which is the reverse of simple.

\par \textbf{YOUNG SOCRATES}
\par   True.

\par \textbf{STRANGER}
\par   Then if the law is not the perfection of right, why are we compelled to make laws at all? The reason of this has next to be investigated.

\par \textbf{YOUNG SOCRATES}
\par   Certainly.

\par \textbf{STRANGER}
\par   Let me ask, whether you have not meetings for gymnastic contests in your city, such as there are in other cities, at which men compete in running, wrestling, and the like?

\par \textbf{YOUNG SOCRATES}
\par   Yes; they are very common among us.

\par \textbf{STRANGER}
\par   And what are the rules which are enforced on their pupils by professional trainers or by others having similar authority? Can you remember?

\par \textbf{YOUNG SOCRATES}
\par   To what do you refer?

\par \textbf{STRANGER}
\par   The training-masters do not issue minute rules for individuals, or give every individual what is exactly suited to his constitution; they think that they ought to go more roughly to work, and to prescribe generally the regimen which will benefit the majority.

\par \textbf{YOUNG SOCRATES}
\par   Very true.

\par \textbf{STRANGER}
\par   And therefore they assign equal amounts of exercise to them all; they send them forth together, and let them rest together from their running, wrestling, or whatever the form of bodily exercise may be.

\par \textbf{YOUNG SOCRATES}
\par   True.

\par \textbf{STRANGER}
\par   And now observe that the legislator who has to preside over the herd, and to enforce justice in their dealings with one another, will not be able, in enacting for the general good, to provide exactly what is suitable for each particular case.

\par \textbf{YOUNG SOCRATES}
\par   He cannot be expected to do so.

\par \textbf{STRANGER}
\par   He will lay down laws in a general form for the majority, roughly meeting the cases of individuals; and some of them he will deliver in writing, and others will be unwritten; and these last will be traditional customs of the country.

\par \textbf{YOUNG SOCRATES}
\par   He will be right.

\par \textbf{STRANGER}
\par   Yes, quite right; for how can he sit at every man's side all through his life, prescribing for him the exact particulars of his duty? Who, Socrates, would be equal to such a task? No one who really had the royal science, if he had been able to do this, would have imposed upon himself the restriction of a written law.

\par \textbf{YOUNG SOCRATES}
\par   So I should infer from what has now been said.

\par \textbf{STRANGER}
\par   Or rather, my good friend, from what is going to be said.

\par \textbf{YOUNG SOCRATES}
\par   And what is that?

\par \textbf{STRANGER}
\par   Let us put to ourselves the case of a physician, or trainer, who is about to go into a far country, and is expecting to be a long time away from his patients—thinking that his instructions will not be remembered unless they are written down, he will leave notes of them for the use of his pupils or patients.

\par \textbf{YOUNG SOCRATES}
\par   True.

\par \textbf{STRANGER}
\par   But what would you say, if he came back sooner than he had intended, and, owing to an unexpected change of the winds or other celestial influences, something else happened to be better for them,—would he not venture to suggest this new remedy, although not contemplated in his former prescription? Would he persist in observing the original law, neither himself giving any new commandments, nor the patient daring to do otherwise than was prescribed, under the idea that this course only was healthy and medicinal, all others noxious and heterodox? Viewed in the light of science and true art, would not all such enactments be utterly ridiculous?

\par \textbf{YOUNG SOCRATES}
\par   Utterly.

\par \textbf{STRANGER}
\par   And if he who gave laws, written or unwritten, determining what was good or bad, honourable or dishonourable, just or unjust, to the tribes of men who flock together in their several cities, and are governed in accordance with them; if, I say, the wise legislator were suddenly to come again, or another like to him, is he to be prohibited from changing them?—would not this prohibition be in reality quite as ridiculous as the other?

\par \textbf{YOUNG SOCRATES}
\par   Certainly.

\par \textbf{STRANGER}
\par   Do you know a plausible saying of the common people which is in point?

\par \textbf{YOUNG SOCRATES}
\par   I do not recall what you mean at the moment.

\par \textbf{STRANGER}
\par   They say that if any one knows how the ancient laws may be improved, he must first persuade his own State of the improvement, and then he may legislate, but not otherwise.

\par \textbf{YOUNG SOCRATES}
\par   And are they not right?

\par \textbf{STRANGER}
\par   I dare say. But supposing that he does use some gentle violence for their good, what is this violence to be called? Or rather, before you answer, let me ask the same question in reference to our previous instances.

\par \textbf{YOUNG SOCRATES}
\par   What do you mean?

\par \textbf{STRANGER}
\par   Suppose that a skilful physician has a patient, of whatever sex or age, whom he compels against his will to do something for his good which is contrary to the written rules; what is this compulsion to be called? Would you ever dream of calling it a violation of the art, or a breach of the laws of health? Nothing could be more unjust than for the patient to whom such violence is applied, to charge the physician who practises the violence with wanting skill or aggravating his disease.

\par \textbf{YOUNG SOCRATES}
\par   Most true.

\par \textbf{STRANGER}
\par   In the political art error is not called disease, but evil, or disgrace, or injustice.

\par \textbf{YOUNG SOCRATES}
\par   Quite true.

\par \textbf{STRANGER}
\par   And when the citizen, contrary to law and custom, is compelled to do what is juster and better and nobler than he did before, the last and most absurd thing which he could say about such violence is that he has incurred disgrace or evil or injustice at the hands of those who compelled him.

\par \textbf{YOUNG SOCRATES}
\par   Very true.

\par \textbf{STRANGER}
\par   And shall we say that the violence, if exercised by a rich man, is just, and if by a poor man, unjust? May not any man, rich or poor, with or without laws, with the will of the citizens or against the will of the citizens, do what is for their interest? Is not this the true principle of government, according to which the wise and good man will order the affairs of his subjects? As the pilot, by watching continually over the interests of the ship and of the crew,—not by laying down rules, but by making his art a law,—preserves the lives of his fellow-sailors, even so, and in the self-same way, may there not be a true form of polity created by those who are able to govern in a similar spirit, and who show a strength of art which is superior to the law? Nor can wise rulers ever err while they observing the one great rule of distributing justice to the citizens with intelligence and skill, are able to preserve them, and, as far as may be, to make them better from being worse.

\par \textbf{YOUNG SOCRATES}
\par   No one can deny what has been now said.

\par \textbf{STRANGER}
\par   Neither, if you consider, can any one deny the other statement.

\par \textbf{YOUNG SOCRATES}
\par   What was it?

\par \textbf{STRANGER}
\par   We said that no great number of persons, whoever they may be, can attain political knowledge, or order a State wisely, but that the true government is to be found in a small body, or in an individual, and that other States are but imitations of this, as we said a little while ago, some for the better and some for the worse.

\par \textbf{YOUNG SOCRATES}
\par   What do you mean? I cannot have understood your previous remark about imitations.

\par \textbf{STRANGER}
\par   And yet the mere suggestion which I hastily threw out is highly important, even if we leave the question where it is, and do not seek by the discussion of it to expose the error which prevails in this matter.

\par \textbf{YOUNG SOCRATES}
\par   What do you mean?

\par \textbf{STRANGER}
\par   The idea which has to be grasped by us is not easy or familiar; but we may attempt to express it thus: —Supposing the government of which I have been speaking to be the only true model, then the others must use the written laws of this—in no other way can they be saved; they will have to do what is now generally approved, although not the best thing in the world.

\par \textbf{YOUNG SOCRATES}
\par   What is this?

\par \textbf{STRANGER}
\par   No citizen should do anything contrary to the laws, and any infringement of them should be punished with death and the most extreme penalties; and this is very right and good when regarded as the second best thing, if you set aside the first, of which I was just now speaking. Shall I explain the nature of what I call the second best?

\par \textbf{YOUNG SOCRATES}
\par   By all means.

\par \textbf{STRANGER}
\par   I must again have recourse to my favourite images; through them, and them alone, can I describe kings and rulers.

\par \textbf{YOUNG SOCRATES}
\par   What images?

\par \textbf{STRANGER}
\par   The noble pilot and the wise physician, who 'is worth many another man'—in the similitude of these let us endeavour to discover some image of the king.

\par \textbf{YOUNG SOCRATES}
\par   What sort of an image?

\par \textbf{STRANGER}
\par   Well, such as this: —Every man will reflect that he suffers strange things at the hands of both of them; the physician saves any whom he wishes to save, and any whom he wishes to maltreat he maltreats—cutting or burning them; and at the same time requiring them to bring him payments, which are a sort of tribute, of which little or nothing is spent upon the sick man, and the greater part is consumed by him and his domestics; and the finale is that he receives money from the relations of the sick man or from some enemy of his, and puts him out of the way. And the pilots of ships are guilty of numberless evil deeds of the same kind; they intentionally play false and leave you ashore when the hour of sailing arrives; or they cause mishaps at sea and cast away their freight; and are guilty of other rogueries. Now suppose that we, bearing all this in mind, were to determine, after consideration, that neither of these arts shall any longer be allowed to exercise absolute control either over freemen or over slaves, but that we will summon an assembly either of all the people, or of the rich only, that anybody who likes, whatever may be his calling, or even if he have no calling, may offer an opinion either about seamanship or about diseases—whether as to the manner in which physic or surgical instruments are to be applied to the patient, or again about the vessels and the nautical implements which are required in navigation, and how to meet the dangers of winds and waves which are incidental to the voyage, how to behave when encountering pirates, and what is to be done with the old-fashioned galleys, if they have to fight with others of a similar build—and that, whatever shall be decreed by the multitude on these points, upon the advice of persons skilled or unskilled, shall be written down on triangular tablets and columns, or enacted although unwritten to be national customs; and that in all future time vessels shall be navigated and remedies administered to the patient after this fashion.

\par \textbf{YOUNG SOCRATES}
\par   What a strange notion!

\par \textbf{STRANGER}
\par   Suppose further, that the pilots and physicians are appointed annually, either out of the rich, or out of the whole people, and that they are elected by lot; and that after their election they navigate vessels and heal the sick according to the written rules.

\par \textbf{YOUNG SOCRATES}
\par   Worse and worse.

\par \textbf{STRANGER}
\par   But hear what follows: —When the year of office has expired, the pilot or physician has to come before a court of review, in which the judges are either selected from the wealthy classes or chosen by lot out of the whole people; and anybody who pleases may be their accuser, and may lay to their charge, that during the past year they have not navigated their vessels or healed their patients according to the letter of the law and the ancient customs of their ancestors; and if either of them is condemned, some of the judges must fix what he is to suffer or pay.

\par \textbf{YOUNG SOCRATES}
\par   He who is willing to take a command under such conditions, deserves to suffer any penalty.

\par \textbf{STRANGER}
\par   Yet once more, we shall have to enact that if any one is detected enquiring into piloting and navigation, or into health and the true nature of medicine, or about the winds, or other conditions of the atmosphere, contrary to the written rules, and has any ingenious notions about such matters, he is not to be called a pilot or physician, but a cloudy prating sophist;—further, on the ground that he is a corrupter of the young, who would persuade them to follow the art of medicine or piloting in an unlawful manner, and to exercise an arbitrary rule over their patients or ships, any one who is qualified by law may inform against him, and indict him in some court, and then if he is found to be persuading any, whether young or old, to act contrary to the written law, he is to be punished with the utmost rigour; for no one should presume to be wiser than the laws; and as touching healing and health and piloting and navigation, the nature of them is known to all, for anybody may learn the written laws and the national customs. If such were the mode of procedure, Socrates, about these sciences and about generalship, and any branch of hunting, or about painting or imitation in general, or carpentry, or any sort of handicraft, or husbandry, or planting, or if we were to see an art of rearing horses, or tending herds, or divination, or any ministerial service, or draught-playing, or any science conversant with number, whether simple or square or cube, or comprising motion,—I say, if all these things were done in this way according to written regulations, and not according to art, what would be the result?

\par \textbf{YOUNG SOCRATES}
\par   All the arts would utterly perish, and could never be recovered, because enquiry would be unlawful. And human life, which is bad enough already, would then become utterly unendurable.

\par \textbf{STRANGER}
\par   But what, if while compelling all these operations to be regulated by written law, we were to appoint as the guardian of the laws some one elected by a show of hands, or by lot, and he caring nothing about the laws, were to act contrary to them from motives of interest or favour, and without knowledge,—would not this be a still worse evil than the former?

\par \textbf{YOUNG SOCRATES}
\par   Very true.

\par \textbf{STRANGER}
\par   To go against the laws, which are based upon long experience, and the wisdom of counsellors who have graciously recommended them and persuaded the multitude to pass them, would be a far greater and more ruinous error than any adherence to written law?

\par \textbf{YOUNG SOCRATES}
\par   Certainly.

\par \textbf{STRANGER}
\par   Therefore, as there is a danger of this, the next best thing in legislating is not to allow either the individual or the multitude to break the law in any respect whatever.

\par \textbf{YOUNG SOCRATES}
\par   True.

\par \textbf{STRANGER}
\par   The laws would be copies of the true particulars of action as far as they admit of being written down from the lips of those who have knowledge?

\par \textbf{YOUNG SOCRATES}
\par   Certainly they would.

\par \textbf{STRANGER}
\par   And, as we were saying, he who has knowledge and is a true Statesman, will do many things within his own sphere of action by his art without regard to the laws, when he is of opinion that something other than that which he has written down and enjoined to be observed during his absence would be better.

\par \textbf{YOUNG SOCRATES}
\par   Yes, we said so.

\par \textbf{STRANGER}
\par   And any individual or any number of men, having fixed laws, in acting contrary to them with a view to something better, would only be acting, as far as they are able, like the true Statesman?

\par \textbf{YOUNG SOCRATES}
\par   Certainly.

\par \textbf{STRANGER}
\par   If they had no knowledge of what they were doing, they would imitate the truth, and they would always imitate ill; but if they had knowledge, the imitation would be the perfect truth, and an imitation no longer.

\par \textbf{YOUNG SOCRATES}
\par   Quite true.

\par \textbf{STRANGER}
\par   And the principle that no great number of men are able to acquire a knowledge of any art has been already admitted by us.

\par \textbf{YOUNG SOCRATES}
\par   Yes, it has.

\par \textbf{STRANGER}
\par   Then the royal or political art, if there be such an art, will never be attained either by the wealthy or by the other mob.

\par \textbf{YOUNG SOCRATES}
\par   Impossible.

\par \textbf{STRANGER}
\par   Then the nearest approach which these lower forms of government can ever make to the true government of the one scientific ruler, is to do nothing contrary to their own written laws and national customs.

\par \textbf{YOUNG SOCRATES}
\par   Very good.

\par \textbf{STRANGER}
\par   When the rich imitate the true form, such a government is called aristocracy; and when they are regardless of the laws, oligarchy.

\par \textbf{YOUNG SOCRATES}
\par   True.

\par \textbf{STRANGER}
\par   Or again, when an individual rules according to law in imitation of him who knows, we call him a king; and if he rules according to law, we give him the same name, whether he rules with opinion or with knowledge.

\par \textbf{YOUNG SOCRATES}
\par   To be sure.

\par \textbf{STRANGER}
\par   And when an individual truly possessing knowledge rules, his name will surely be the same—he will be called a king; and thus the five names of governments, as they are now reckoned, become one.

\par \textbf{YOUNG SOCRATES}
\par   That is true.

\par \textbf{STRANGER}
\par   And when an individual ruler governs neither by law nor by custom, but following in the steps of the true man of science pretends that he can only act for the best by violating the laws, while in reality appetite and ignorance are the motives of the imitation, may not such an one be called a tyrant?

\par \textbf{YOUNG SOCRATES}
\par   Certainly.

\par \textbf{STRANGER}
\par   And this we believe to be the origin of the tyrant and the king, of oligarchies, and aristocracies, and democracies,—because men are offended at the one monarch, and can never be made to believe that any one can be worthy of such authority, or is able and willing in the spirit of virtue and knowledge to act justly and holily to all; they fancy that he will be a despot who will wrong and harm and slay whom he pleases of us; for if there could be such a despot as we describe, they would acknowledge that we ought to be too glad to have him, and that he alone would be the happy ruler of a true and perfect State.

\par \textbf{YOUNG SOCRATES}
\par   To be sure.

\par \textbf{STRANGER}
\par   But then, as the State is not like a beehive, and has no natural head who is at once recognized to be the superior both in body and in mind, mankind are obliged to meet and make laws, and endeavour to approach as nearly as they can to the true form of government.

\par \textbf{YOUNG SOCRATES}
\par   True.

\par \textbf{STRANGER}
\par   And when the foundation of politics is in the letter only and in custom, and knowledge is divorced from action, can we wonder, Socrates, at the miseries which there are, and always will be, in States? Any other art, built on such a foundation and thus conducted, would ruin all that it touched. Ought we not rather to wonder at the natural strength of the political bond? For States have endured all this, time out of mind, and yet some of them still remain and are not overthrown, though many of them, like ships at sea, founder from time to time, and perish and have perished and will hereafter perish, through the badness of their pilots and crews, who have the worst sort of ignorance of the highest truths—I mean to say, that they are wholly unaquainted with politics, of which, above all other sciences, they believe themselves to have acquired the most perfect knowledge.

\par \textbf{YOUNG SOCRATES}
\par   Very true.

\par \textbf{STRANGER}
\par   Then the question arises: —which of these untrue forms of government is the least oppressive to their subjects, though they are all oppressive; and which is the worst of them? Here is a consideration which is beside our present purpose, and yet having regard to the whole it seems to influence all our actions:  we must examine it.

\par \textbf{YOUNG SOCRATES}
\par   Yes, we must.

\par \textbf{STRANGER}
\par   You may say that of the three forms, the same is at once the hardest and the easiest.

\par \textbf{YOUNG SOCRATES}
\par   What do you mean?

\par \textbf{STRANGER}
\par   I am speaking of the three forms of government, which I mentioned at the beginning of this discussion—monarchy, the rule of the few, and the rule of the many.

\par \textbf{YOUNG SOCRATES}
\par   True.

\par \textbf{STRANGER}
\par   If we divide each of these we shall have six, from which the true one may be distinguished as a seventh.

\par \textbf{YOUNG SOCRATES}
\par   How would you make the division?

\par \textbf{STRANGER}
\par   Monarchy divides into royalty and tyranny; the rule of the few into aristocracy, which has an auspicious name, and oligarchy; and democracy or the rule of the many, which before was one, must now be divided.

\par \textbf{YOUNG SOCRATES}
\par   On what principle of division?

\par \textbf{STRANGER}
\par   On the same principle as before, although the name is now discovered to have a twofold meaning. For the distinction of ruling with law or without law, applies to this as well as to the rest.

\par \textbf{YOUNG SOCRATES}
\par   Yes.

\par \textbf{STRANGER}
\par   The division made no difference when we were looking for the perfect State, as we showed before. But now that this has been separated off, and, as we said, the others alone are left for us, the principle of law and the absence of law will bisect them all.

\par \textbf{YOUNG SOCRATES}
\par   That would seem to follow, from what has been said.

\par \textbf{STRANGER}
\par   Then monarchy, when bound by good prescriptions or laws, is the best of all the six, and when lawless is the most bitter and oppressive to the subject.

\par \textbf{YOUNG SOCRATES}
\par   True.

\par \textbf{STRANGER}
\par   The government of the few, which is intermediate between that of the one and many, is also intermediate in good and evil; but the government of the many is in every respect weak and unable to do either any great good or any great evil, when compared with the others, because the offices are too minutely subdivided and too many hold them. And this therefore is the worst of all lawful governments, and the best of all lawless ones. If they are all without the restraints of law, democracy is the form in which to live is best; if they are well ordered, then this is the last which you should choose, as royalty, the first form, is the best, with the exception of the seventh, for that excels them all, and is among States what God is among men.

\par \textbf{YOUNG SOCRATES}
\par   You are quite right, and we should choose that above all.

\par \textbf{STRANGER}
\par   The members of all these States, with the exception of the one which has knowledge, may be set aside as being not Statesmen but partisans,—upholders of the most monstrous idols, and themselves idols; and, being the greatest imitators and magicians, they are also the greatest of Sophists.

\par \textbf{YOUNG SOCRATES}
\par   The name of Sophist after many windings in the argument appears to have been most justly fixed upon the politicians, as they are termed.

\par \textbf{STRANGER}
\par   And so our satyric drama has been played out; and the troop of Centaurs and Satyrs, however unwilling to leave the stage, have at last been separated from the political science.

\par \textbf{YOUNG SOCRATES}
\par   So I perceive.

\par \textbf{STRANGER}
\par   There remain, however, natures still more troublesome, because they are more nearly akin to the king, and more difficult to discern; the examination of them may be compared to the process of refining gold.

\par \textbf{YOUNG SOCRATES}
\par   What is your meaning?

\par \textbf{STRANGER}
\par   The workmen begin by sifting away the earth and stones and the like; there remain in a confused mass the valuable elements akin to gold, which can only be separated by fire,—copper, silver, and other precious metal; these are at last refined away by the use of tests, until the gold is left quite pure.

\par \textbf{YOUNG SOCRATES}
\par   Yes, that is the way in which these things are said to be done.

\par \textbf{STRANGER}
\par   In like manner, all alien and uncongenial matter has been separated from political science, and what is precious and of a kindred nature has been left; there remain the nobler arts of the general and the judge, and the higher sort of oratory which is an ally of the royal art, and persuades men to do justice, and assists in guiding the helm of States: —How can we best clear away all these, leaving him whom we seek alone and unalloyed?

\par \textbf{YOUNG SOCRATES}
\par   That is obviously what has in some way to be attempted.

\par \textbf{STRANGER}
\par   If the attempt is all that is wanting, he shall certainly be brought to light; and I think that the illustration of music may assist in exhibiting him. Please to answer me a question.

\par \textbf{YOUNG SOCRATES}
\par   What question?

\par \textbf{STRANGER}
\par   There is such a thing as learning music or handicraft arts in general?

\par \textbf{YOUNG SOCRATES}
\par   There is.

\par \textbf{STRANGER}
\par   And is there any higher art or science, having power to decide which of these arts are and are not to be learned;—what do you say?

\par \textbf{YOUNG SOCRATES}
\par   I should answer that there is.

\par \textbf{STRANGER}
\par   And do we acknowledge this science to be different from the others?

\par \textbf{YOUNG SOCRATES}
\par   Yes.

\par \textbf{STRANGER}
\par   And ought the other sciences to be superior to this, or no single science to any other? Or ought this science to be the overseer and governor of all the others?

\par \textbf{YOUNG SOCRATES}
\par   The latter.

\par \textbf{STRANGER}
\par   You mean to say that the science which judges whether we ought to learn or not, must be superior to the science which is learned or which teaches?

\par \textbf{YOUNG SOCRATES}
\par   Far superior.

\par \textbf{STRANGER}
\par   And the science which determines whether we ought to persuade or not, must be superior to the science which is able to persuade?

\par \textbf{YOUNG SOCRATES}
\par   Of course.

\par \textbf{STRANGER}
\par   Very good; and to what science do we assign the power of persuading a multitude by a pleasing tale and not by teaching?

\par \textbf{YOUNG SOCRATES}
\par   That power, I think, must clearly be assigned to rhetoric.

\par \textbf{STRANGER}
\par   And to what science do we give the power of determining whether we are to employ persuasion or force towards any one, or to refrain altogether?

\par \textbf{YOUNG SOCRATES}
\par   To that science which governs the arts of speech and persuasion.

\par \textbf{STRANGER}
\par   Which, if I am not mistaken, will be politics?

\par \textbf{YOUNG SOCRATES}
\par   Very good.

\par \textbf{STRANGER}
\par   Rhetoric seems to be quickly distinguished from politics, being a different species, yet ministering to it.

\par \textbf{YOUNG SOCRATES}
\par   Yes.

\par \textbf{STRANGER}
\par   But what would you think of another sort of power or science?

\par \textbf{YOUNG SOCRATES}
\par   What science?

\par \textbf{STRANGER}
\par   The science which has to do with military operations against our enemies—is that to be regarded as a science or not?

\par \textbf{YOUNG SOCRATES}
\par   How can generalship and military tactics be regarded as other than a science?

\par \textbf{STRANGER}
\par   And is the art which is able and knows how to advise when we are to go to war, or to make peace, the same as this or different?

\par \textbf{YOUNG SOCRATES}
\par   If we are to be consistent, we must say different.

\par \textbf{STRANGER}
\par   And we must also suppose that this rules the other, if we are not to give up our former notion?

\par \textbf{YOUNG SOCRATES}
\par   True.

\par \textbf{STRANGER}
\par   And, considering how great and terrible the whole art of war is, can we imagine any which is superior to it but the truly royal?

\par \textbf{YOUNG SOCRATES}
\par   No other.

\par \textbf{STRANGER}
\par   The art of the general is only ministerial, and therefore not political?

\par \textbf{YOUNG SOCRATES}
\par   Exactly.

\par \textbf{STRANGER}
\par   Once more let us consider the nature of the righteous judge.

\par \textbf{YOUNG SOCRATES}
\par   Very good.

\par \textbf{STRANGER}
\par   Does he do anything but decide the dealings of men with one another to be just or unjust in accordance with the standard which he receives from the king and legislator,—showing his own peculiar virtue only in this, that he is not perverted by gifts, or fears, or pity, or by any sort of favour or enmity, into deciding the suits of men with one another contrary to the appointment of the legislator?

\par \textbf{YOUNG SOCRATES}
\par   No; his office is such as you describe.

\par \textbf{STRANGER}
\par   Then the inference is that the power of the judge is not royal, but only the power of a guardian of the law which ministers to the royal power?

\par \textbf{YOUNG SOCRATES}
\par   True.

\par \textbf{STRANGER}
\par   The review of all these sciences shows that none of them is political or royal. For the truly royal ought not itself to act, but to rule over those who are able to act; the king ought to know what is and what is not a fitting opportunity for taking the initiative in matters of the greatest importance, whilst others should execute his orders.

\par \textbf{YOUNG SOCRATES}
\par   True.

\par \textbf{STRANGER}
\par   And, therefore, the arts which we have described, as they have no authority over themselves or one another, but are each of them concerned with some special action of their own, have, as they ought to have, special names corresponding to their several actions.

\par \textbf{YOUNG SOCRATES}
\par   I agree.

\par \textbf{STRANGER}
\par   And the science which is over them all, and has charge of the laws, and of all matters affecting the State, and truly weaves them all into one, if we would describe under a name characteristic of their common nature, most truly we may call politics.

\par \textbf{YOUNG SOCRATES}
\par   Exactly so.

\par \textbf{STRANGER}
\par   Then, now that we have discovered the various classes in a State, shall I analyse politics after the pattern which weaving supplied?

\par \textbf{YOUNG SOCRATES}
\par   I greatly wish that you would.

\par \textbf{STRANGER}
\par   Then I must describe the nature of the royal web, and show how the various threads are woven into one piece.

\par \textbf{YOUNG SOCRATES}
\par   Clearly.

\par \textbf{STRANGER}
\par   A task has to be accomplished, which, although difficult, appears to be necessary.

\par \textbf{YOUNG SOCRATES}
\par   Certainly the attempt must be made.

\par \textbf{STRANGER}
\par   To assume that one part of virtue differs in kind from another, is a position easily assailable by contentious disputants, who appeal to popular opinion.

\par \textbf{YOUNG SOCRATES}
\par   I do not understand.

\par \textbf{STRANGER}
\par   Let me put the matter in another way:  I suppose that you would consider courage to be a part of virtue?

\par \textbf{YOUNG SOCRATES}
\par   Certainly I should.

\par \textbf{STRANGER}
\par   And you would think temperance to be different from courage; and likewise to be a part of virtue?

\par \textbf{YOUNG SOCRATES}
\par   True.

\par \textbf{STRANGER}
\par   I shall venture to put forward a strange theory about them.

\par \textbf{YOUNG SOCRATES}
\par   What is it?

\par \textbf{STRANGER}
\par   That they are two principles which thoroughly hate one another and are antagonistic throughout a great part of nature.

\par \textbf{YOUNG SOCRATES}
\par   How singular!

\par \textbf{STRANGER}
\par   Yes, very—for all the parts of virtue are commonly said to be friendly to one another.

\par \textbf{YOUNG SOCRATES}
\par   Yes.

\par \textbf{STRANGER}
\par   Then let us carefully investigate whether this is universally true, or whether there are not parts of virtue which are at war with their kindred in some respect.

\par \textbf{YOUNG SOCRATES}
\par   Tell me how we shall consider that question.

\par \textbf{STRANGER}
\par   We must extend our enquiry to all those things which we consider beautiful and at the same time place in two opposite classes.

\par \textbf{YOUNG SOCRATES}
\par   Explain; what are they?

\par \textbf{STRANGER}
\par   Acuteness and quickness, whether in body or soul or in the movement of sound, and the imitations of them which painting and music supply, you must have praised yourself before now, or been present when others praised them.

\par \textbf{YOUNG SOCRATES}
\par   Certainly.

\par \textbf{STRANGER}
\par   And do you remember the terms in which they are praised?

\par \textbf{YOUNG SOCRATES}
\par   I do not.

\par \textbf{STRANGER}
\par   I wonder whether I can explain to you in words the thought which is passing in my mind.

\par \textbf{YOUNG SOCRATES}
\par   Why not?

\par \textbf{STRANGER}
\par   You fancy that this is all so easy:  Well, let us consider these notions with reference to the opposite classes of action under which they fall. When we praise quickness and energy and acuteness, whether of mind or body or sound, we express our praise of the quality which we admire by one word, and that one word is manliness or courage.

\par \textbf{YOUNG SOCRATES}
\par   How?

\par \textbf{STRANGER}
\par   We speak of an action as energetic and brave, quick and manly, and vigorous too; and when we apply the name of which I speak as the common attribute of all these natures, we certainly praise them.

\par \textbf{YOUNG SOCRATES}
\par   True.

\par \textbf{STRANGER}
\par   And do we not often praise the quiet strain of action also?

\par \textbf{YOUNG SOCRATES}
\par   To be sure.

\par \textbf{STRANGER}
\par   And do we not then say the opposite of what we said of the other?

\par \textbf{YOUNG SOCRATES}
\par   How do you mean?

\par \textbf{STRANGER}
\par   We exclaim How calm! How temperate! in admiration of the slow and quiet working of the intellect, and of steadiness and gentleness in action, of smoothness and depth of voice, and of all rhythmical movement and of music in general, when these have a proper solemnity. Of all such actions we predicate not courage, but a name indicative of order.

\par \textbf{YOUNG SOCRATES}
\par   Very true.

\par \textbf{STRANGER}
\par   But when, on the other hand, either of these is out of place, the names of either are changed into terms of censure.

\par \textbf{YOUNG SOCRATES}
\par   How so?

\par \textbf{STRANGER}
\par   Too great sharpness or quickness or hardness is termed violence or madness; too great slowness or gentleness is called cowardice or sluggishness; and we may observe, that for the most part these qualities, and the temperance and manliness of the opposite characters, are arrayed as enemies on opposite sides, and do not mingle with one another in their respective actions; and if we pursue the enquiry, we shall find that men who have these different qualities of mind differ from one another.

\par \textbf{YOUNG SOCRATES}
\par   In what respect?

\par \textbf{STRANGER}
\par   In respect of all the qualities which I mentioned, and very likely of many others. According to their respective affinities to either class of actions they distribute praise and blame,—praise to the actions which are akin to their own, blame to those of the opposite party—and out of this many quarrels and occasions of quarrel arise among them.

\par \textbf{YOUNG SOCRATES}
\par   True.

\par \textbf{STRANGER}
\par   The difference between the two classes is often a trivial concern; but in a state, and when affecting really important matters, becomes of all disorders the most hateful.

\par \textbf{YOUNG SOCRATES}
\par   To what do you refer?

\par \textbf{STRANGER}
\par   To nothing short of the whole regulation of human life. For the orderly class are always ready to lead a peaceful life, quietly doing their own business; this is their manner of behaving with all men at home, and they are equally ready to find some way of keeping the peace with foreign States. And on account of this fondness of theirs for peace, which is often out of season where their influence prevails, they become by degrees unwarlike, and bring up their young men to be like themselves; they are at the mercy of their enemies; whence in a few years they and their children and the whole city often pass imperceptibly from the condition of freemen into that of slaves.

\par \textbf{YOUNG SOCRATES}
\par   What a cruel fate!

\par \textbf{STRANGER}
\par   And now think of what happens with the more courageous natures. Are they not always inciting their country to go to war, owing to their excessive love of the military life? they raise up enemies against themselves many and mighty, and either utterly ruin their native-land or enslave and subject it to its foes?

\par \textbf{YOUNG SOCRATES}
\par   That, again, is true.

\par \textbf{STRANGER}
\par   Must we not admit, then, that where these two classes exist, they always feel the greatest antipathy and antagonism towards one another?

\par \textbf{YOUNG SOCRATES}
\par   We cannot deny it.

\par \textbf{STRANGER}
\par   And returning to the enquiry with which we began, have we not found that considerable portions of virtue are at variance with one another, and give rise to a similar opposition in the characters who are endowed with them?

\par \textbf{YOUNG SOCRATES}
\par   True.

\par \textbf{STRANGER}
\par   Let us consider a further point.

\par \textbf{YOUNG SOCRATES}
\par   What is it?

\par \textbf{STRANGER}
\par   I want to know, whether any constructive art will make any, even the most trivial thing, out of bad and good materials indifferently, if this can be helped? does not all art rather reject the bad as far as possible, and accept the good and fit materials, and from these elements, whether like or unlike, gathering them all into one, work out some nature or idea?

\par \textbf{YOUNG SOCRATES}
\par   To, be sure.

\par \textbf{STRANGER}
\par   Then the true and natural art of statesmanship will never allow any State to be formed by a combination of good and bad men, if this can be avoided; but will begin by testing human natures in play, and after testing them, will entrust them to proper teachers who are the ministers of her purposes—she will herself give orders, and maintain authority; just as the art of weaving continually gives orders and maintains authority over the carders and all the others who prepare the material for the work, commanding the subsidiary arts to execute the works which she deems necessary for making the web.

\par \textbf{YOUNG SOCRATES}
\par   Quite true.

\par \textbf{STRANGER}
\par   In like manner, the royal science appears to me to be the mistress of all lawful educators and instructors, and having this queenly power, will not permit them to train men in what will produce characters unsuited to the political constitution which she desires to create, but only in what will produce such as are suitable. Those which have no share of manliness and temperance, or any other virtuous inclination, and, from the necessity of an evil nature, are violently carried away to godlessness and insolence and injustice, she gets rid of by death and exile, and punishes them with the greatest of disgraces.

\par \textbf{YOUNG SOCRATES}
\par   That is commonly said.

\par \textbf{STRANGER}
\par   But those who are wallowing in ignorance and baseness she bows under the yoke of slavery.

\par \textbf{YOUNG SOCRATES}
\par   Quite right.

\par \textbf{STRANGER}
\par   The rest of the citizens, out of whom, if they have education, something noble may be made, and who are capable of being united by the statesman, the kingly art blends and weaves together; taking on the one hand those whose natures tend rather to courage, which is the stronger element and may be regarded as the warp, and on the other hand those which incline to order and gentleness, and which are represented in the figure as spun thick and soft, after the manner of the woof—these, which are naturally opposed, she seeks to bind and weave together in the following manner:

\par \textbf{YOUNG SOCRATES}
\par   In what manner?

\par \textbf{STRANGER}
\par   First of all, she takes the eternal element of the soul and binds it with a divine cord, to which it is akin, and then the animal nature, and binds that with human cords.

\par \textbf{YOUNG SOCRATES}
\par   I do not understand what you mean.

\par \textbf{STRANGER}
\par   The meaning is, that the opinion about the honourable and the just and good and their opposites, which is true and confirmed by reason, is a divine principle, and when implanted in the soul, is implanted, as I maintain, in a nature of heavenly birth.

\par \textbf{YOUNG SOCRATES}
\par   Yes; what else should it be?

\par \textbf{STRANGER}
\par   Only the Statesman and the good legislator, having the inspiration of the royal muse, can implant this opinion, and he, only in the rightly educated, whom we were just now describing.

\par \textbf{YOUNG SOCRATES}
\par   Likely enough.

\par \textbf{STRANGER}
\par   But him who cannot, we will not designate by any of the names which are the subject of the present enquiry.

\par \textbf{YOUNG SOCRATES}
\par   Very right.

\par \textbf{STRANGER}
\par   The courageous soul when attaining this truth becomes civilized, and rendered more capable of partaking of justice; but when not partaking, is inclined to brutality. Is not that true?

\par \textbf{YOUNG SOCRATES}
\par   Certainly.

\par \textbf{STRANGER}
\par   And again, the peaceful and orderly nature, if sharing in these opinions, becomes temperate and wise, as far as this may be in a State, but if not, deservedly obtains the ignominious name of silliness.

\par \textbf{YOUNG SOCRATES}
\par   Quite true.

\par \textbf{STRANGER}
\par   Can we say that such a connexion as this will lastingly unite the evil with one another or with the good, or that any science would seriously think of using a bond of this kind to join such materials?

\par \textbf{YOUNG SOCRATES}
\par   Impossible.

\par \textbf{STRANGER}
\par   But in those who were originally of a noble nature, and who have been nurtured in noble ways, and in those only, may we not say that union is implanted by law, and that this is the medicine which art prescribes for them, and of all the bonds which unite the dissimilar and contrary parts of virtue is not this, as I was saying, the divinest?

\par \textbf{YOUNG SOCRATES}
\par   Very true.

\par \textbf{STRANGER}
\par   Where this divine bond exists there is no difficulty in imagining, or when you have imagined, in creating the other bonds, which are human only.

\par \textbf{YOUNG SOCRATES}
\par   How is that, and what bonds do you mean?

\par \textbf{STRANGER}
\par   Rights of intermarriage, and ties which are formed between States by giving and taking children in marriage, or between individuals by private betrothals and espousals. For most persons form marriage connexions without due regard to what is best for the procreation of children.

\par \textbf{YOUNG SOCRATES}
\par   In what way?

\par \textbf{STRANGER}
\par   They seek after wealth and power, which in matrimony are objects not worthy even of a serious censure.

\par \textbf{YOUNG SOCRATES}
\par   There is no need to consider them at all.

\par \textbf{STRANGER}
\par   More reason is there to consider the practice of those who make family their chief aim, and to indicate their error.

\par \textbf{YOUNG SOCRATES}
\par   Quite true.

\par \textbf{STRANGER}
\par   They act on no true principle at all; they seek their ease and receive with open arms those who are like themselves, and hate those who are unlike them, being too much influenced by feelings of dislike.

\par \textbf{YOUNG SOCRATES}
\par   How so?

\par \textbf{STRANGER}
\par   The quiet orderly class seek for natures like their own, and as far as they can they marry and give in marriage exclusively in this class, and the courageous do the same; they seek natures like their own, whereas they should both do precisely the opposite.

\par \textbf{YOUNG SOCRATES}
\par   How and why is that?

\par \textbf{STRANGER}
\par   Because courage, when untempered by the gentler nature during many generations, may at first bloom and strengthen, but at last bursts forth into downright madness.

\par \textbf{YOUNG SOCRATES}
\par   Like enough.

\par \textbf{STRANGER}
\par   And then, again, the soul which is over-full of modesty and has no element of courage in many successive generations, is apt to grow too indolent, and at last to become utterly paralyzed and useless.

\par \textbf{YOUNG SOCRATES}
\par   That, again, is quite likely.

\par \textbf{STRANGER}
\par   It was of these bonds I said that there would be no difficulty in creating them, if only both classes originally held the same opinion about the honourable and good;—indeed, in this single work, the whole process of royal weaving is comprised—never to allow temperate natures to be separated from the brave, but to weave them together, like the warp and the woof, by common sentiments and honours and reputation, and by the giving of pledges to one another; and out of them forming one smooth and even web, to entrust to them the offices of State.

\par \textbf{YOUNG SOCRATES}
\par   How do you mean?

\par \textbf{STRANGER}
\par   Where one officer only is needed, you must choose a ruler who has both these qualities—when many, you must mingle some of each, for the temperate ruler is very careful and just and safe, but is wanting in thoroughness and go.

\par \textbf{YOUNG SOCRATES}
\par   Certainly, that is very true.

\par \textbf{STRANGER}
\par   The character of the courageous, on the other hand, falls short of the former in justice and caution, but has the power of action in a remarkable degree, and where either of these two qualities is wanting, there cities cannot altogether prosper either in their public or private life.

\par \textbf{YOUNG SOCRATES}
\par   Certainly they cannot.

\par \textbf{STRANGER}
\par   This then we declare to be the completion of the web of political action, which is created by a direct intertexture of the brave and temperate natures, whenever the royal science has drawn the two minds into communion with one another by unanimity and friendship, and having perfected the noblest and best of all the webs which political life admits, and enfolding therein all other inhabitants of cities, whether slaves or freemen, binds them in one fabric and governs and presides over them, and, in so far as to be happy is vouchsafed to a city, in no particular fails to secure their happiness.

\par \textbf{YOUNG SOCRATES}
\par   Your picture, Stranger, of the king and statesman, no less than of the Sophist, is quite perfect.

\par 
 
\end{document}