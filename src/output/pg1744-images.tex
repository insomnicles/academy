
\documentclass[11pt,letter]{article}


\begin{document}

\title{Philebus\thanks{Source: https://www.gutenberg.org/files/1744/1744-h/1744-h.htm. License: http://gutenberg.org/license ds}}
\date{\today}
\author{Plato, 427? BCE-347? BCE\\ Translated by Jowett, Benjamin, 1817-1893}
\maketitle

\setcounter{tocdepth}{1}
\tableofcontents
\renewcommand{\baselinestretch}{1.0}
\normalsize
\newpage

\section{
      INTRODUCTION AND ANALYSIS.
    }
\par  The Philebus appears to be one of the later writings of Plato, in which the style has begun to alter, and the dramatic and poetical element has become subordinate to the speculative and philosophical. In the development of abstract thought great advances have been made on the Protagoras or the Phaedrus, and even on the Republic. But there is a corresponding diminution of artistic skill, a want of character in the persons, a laboured march in the dialogue, and a degree of confusion and incompleteness in the general design. As in the speeches of Thucydides, the multiplication of ideas seems to interfere with the power of expression. Instead of the equally diffused grace and ease of the earlier dialogues there occur two or three highly-wrought passages; instead of the ever-flowing play of humour, now appearing, now concealed, but always present, are inserted a good many bad jests, as we may venture to term them. We may observe an attempt at artificial ornament, and far-fetched modes of expression; also clamorous demands on the part of his companions, that Socrates shall answer his own questions, as well as other defects of style, which remind us of the Laws. The connection is often abrupt and inharmonious, and far from clear. Many points require further explanation; e.g. the reference of pleasure to the indefinite class, compared with the assertion which almost immediately follows, that pleasure and pain naturally have their seat in the third or mixed class: these two statements are unreconciled. In like manner, the table of goods does not distinguish between the two heads of measure and symmetry; and though a hint is given that the divine mind has the first place, nothing is said of this in the final summing up. The relation of the goods to the sciences does not appear; though dialectic may be thought to correspond to the highest good, the sciences and arts and true opinions are enumerated in the fourth class. We seem to have an intimation of a further discussion, in which some topics lightly passed over were to receive a fuller consideration. The various uses of the word 'mixed,' for the mixed life, the mixed class of elements, the mixture of pleasures, or of pleasure and pain, are a further source of perplexity. Our ignorance of the opinions which Plato is attacking is also an element of obscurity. Many things in a controversy might seem relevant, if we knew to what they were intended to refer. But no conjecture will enable us to supply what Plato has not told us; or to explain, from our fragmentary knowledge of them, the relation in which his doctrine stood to the Eleatic Being or the Megarian good, or to the theories of Aristippus or Antisthenes respecting pleasure. Nor are we able to say how far Plato in the Philebus conceives the finite and infinite (which occur both in the fragments of Philolaus and in the Pythagorean table of opposites) in the same manner as contemporary Pythagoreans.

\par  There is little in the characters which is worthy of remark. The Socrates of the Philebus is devoid of any touch of Socratic irony, though here, as in the Phaedrus, he twice attributes the flow of his ideas to a sudden inspiration. The interlocutor Protarchus, the son of Callias, who has been a hearer of Gorgias, is supposed to begin as a disciple of the partisans of pleasure, but is drawn over to the opposite side by the arguments of Socrates. The instincts of ingenuous youth are easily induced to take the better part. Philebus, who has withdrawn from the argument, is several times brought back again, that he may support pleasure, of which he remains to the end the uncompromising advocate. On the other hand, the youthful group of listeners by whom he is surrounded, 'Philebus' boys' as they are termed, whose presence is several times intimated, are described as all of them at last convinced by the arguments of Socrates. They bear a very faded resemblance to the interested audiences of the Charmides, Lysis, or Protagoras. Other signs of relation to external life in the dialogue, or references to contemporary things and persons, with the single exception of the allusions to the anonymous enemies of pleasure, and the teachers of the flux, there are none.

\par  The omission of the doctrine of recollection, derived from a previous state of existence, is a note of progress in the philosophy of Plato. The transcendental theory of pre-existent ideas, which is chiefly discussed by him in the Meno, the Phaedo, and the Phaedrus, has given way to a psychological one. The omission is rendered more significant by his having occasion to speak of memory as the basis of desire. Of the ideas he treats in the same sceptical spirit which appears in his criticism of them in the Parmenides. He touches on the same difficulties and he gives no answer to them. His mode of speaking of the analytical and synthetical processes may be compared with his discussion of the same subject in the Phaedrus; here he dwells on the importance of dividing the genera into all the species, while in the Phaedrus he conveys the same truth in a figure, when he speaks of carving the whole, which is described under the image of a victim, into parts or members, 'according to their natural articulation, without breaking any of them.' There is also a difference, which may be noted, between the two dialogues. For whereas in the Phaedrus, and also in the Symposium, the dialectician is described as a sort of enthusiast or lover, in the Philebus, as in all the later writings of Plato, the element of love is wanting; the topic is only introduced, as in the Republic, by way of illustration. On other subjects of which they treat in common, such as the nature and kinds of pleasure, true and false opinion, the nature of the good, the order and relation of the sciences, the Republic is less advanced than the Philebus, which contains, perhaps, more metaphysical truth more obscurely expressed than any other Platonic dialogue. Here, as Plato expressly tells us, he is 'forging weapons of another make,' i.e. new categories and modes of conception, though 'some of the old ones might do again.'

\par  But if superior in thought and dialectical power, the Philebus falls very far short of the Republic in fancy and feeling. The development of the reason undisturbed by the emotions seems to be the ideal at which Plato aims in his later dialogues. There is no mystic enthusiasm or rapturous contemplation of ideas. Whether we attribute this change to the greater feebleness of age, or to the development of the quarrel between philosophy and poetry in Plato's own mind, or perhaps, in some degree, to a carelessness about artistic effect, when he was absorbed in abstract ideas, we can hardly be wrong in assuming, amid such a variety of indications, derived from style as well as subject, that the Philebus belongs to the later period of his life and authorship. But in this, as in all the later writings of Plato, there are not wanting thoughts and expressions in which he rises to his highest level.

\par  The plan is complicated, or rather, perhaps, the want of plan renders the progress of the dialogue difficult to follow. A few leading ideas seem to emerge: the relation of the one and many, the four original elements, the kinds of pleasure, the kinds of knowledge, the scale of goods. These are only partially connected with one another. The dialogue is not rightly entitled 'Concerning pleasure' or 'Concerning good,' but should rather be described as treating of the relations of pleasure and knowledge, after they have been duly analyzed, to the good. (1) The question is asked, whether pleasure or wisdom is the chief good, or some nature higher than either; and if the latter, how pleasure and wisdom are related to this higher good. (2) Before we can reply with exactness, we must know the kinds of pleasure and the kinds of knowledge. (3) But still we may affirm generally, that the combined life of pleasure and wisdom or knowledge has more of the character of the good than either of them when isolated. (4) to determine which of them partakes most of the higher nature, we must know under which of the four unities or elements they respectively fall. These are, first, the infinite; secondly, the finite; thirdly, the union of the two; fourthly, the cause of the union. Pleasure is of the first, wisdom or knowledge of the third class, while reason or mind is akin to the fourth or highest.

\par  (5) Pleasures are of two kinds, the mixed and unmixed. Of mixed pleasures there are three classes—(a) those in which both the pleasures and pains are corporeal, as in eating and hunger; (b) those in which there is a pain of the body and pleasure of the mind, as when you are hungry and are looking forward to a feast; (c) those in which the pleasure and pain are both mental. Of unmixed pleasures there are four kinds: those of sight, hearing, smell, knowledge.

\par  (6) The sciences are likewise divided into two classes, theoretical and productive: of the latter, one part is pure, the other impure. The pure part consists of arithmetic, mensuration, and weighing. Arts like carpentering, which have an exact measure, are to be regarded as higher than music, which for the most part is mere guess-work. But there is also a higher arithmetic, and a higher mensuration, which is exclusively theoretical; and a dialectical science, which is higher still and the truest and purest knowledge.

\par  (7) We are now able to determine the composition of the perfect life. First, we admit the pure pleasures and the pure sciences; secondly, the impure sciences, but not the impure pleasures. We have next to discover what element of goodness is contained in this mixture. There are three criteria of goodness—beauty, symmetry, truth. These are clearly more akin to reason than to pleasure, and will enable us to fix the places of both of them in the scale of good. First in the scale is measure; the second place is assigned to symmetry; the third, to reason and wisdom; the fourth, to knowledge and true opinion; the fifth, to pure pleasures; and here the Muse says 'Enough.'

\par  'Bidding farewell to Philebus and Socrates,' we may now consider the metaphysical conceptions which are presented to us. These are (I) the paradox of unity and plurality; (II) the table of categories or elements; (III) the kinds of pleasure; (IV) the kinds of knowledge; (V) the conception of the good. We may then proceed to examine (VI) the relation of the Philebus to the Republic, and to other dialogues.

\par  I. The paradox of the one and many originated in the restless dialectic of Zeno, who sought to prove the absolute existence of the one by showing the contradictions that are involved in admitting the existence of the many (compare Parm.). Zeno illustrated the contradiction by well-known examples taken from outward objects. But Socrates seems to intimate that the time had arrived for discarding these hackneyed illustrations; such difficulties had long been solved by common sense ('solvitur ambulando'); the fact of the co-existence of opposites was a sufficient answer to them. He will leave them to Cynics and Eristics; the youth of Athens may discourse of them to their parents. To no rational man could the circumstance that the body is one, but has many members, be any longer a stumbling-block.

\par  Plato's difficulty seems to begin in the region of ideas. He cannot understand how an absolute unity, such as the Eleatic Being, can be broken up into a number of individuals, or be in and out of them at once. Philosophy had so deepened or intensified the nature of one or Being, by the thoughts of successive generations, that the mind could no longer imagine 'Being' as in a state of change or division. To say that the verb of existence is the copula, or that unity is a mere unit, is to us easy; but to the Greek in a particular stage of thought such an analysis involved the same kind of difficulty as the conception of God existing both in and out of the world would to ourselves. Nor was he assisted by the analogy of sensible objects. The sphere of mind was dark and mysterious to him; but instead of being illustrated by sense, the greatest light appeared to be thrown on the nature of ideas when they were contrasted with sense.

\par  Both here and in the Parmenides, where similar difficulties are raised, Plato seems prepared to desert his ancient ground. He cannot tell the relation in which abstract ideas stand to one another, and therefore he transfers the one and many out of his transcendental world, and proceeds to lay down practical rules for their application to different branches of knowledge. As in the Republic he supposes the philosopher to proceed by regular steps, until he arrives at the idea of good; as in the Sophist and Politicus he insists that in dividing the whole into its parts we should bisect in the middle in the hope of finding species; as in the Phaedrus (see above) he would have 'no limb broken' of the organism of knowledge;—so in the Philebus he urges the necessity of filling up all the intermediate links which occur (compare Bacon's 'media axiomata') in the passage from unity to infinity. With him the idea of science may be said to anticipate science; at a time when the sciences were not yet divided, he wants to impress upon us the importance of classification; neither neglecting the many individuals, nor attempting to count them all, but finding the genera and species under which they naturally fall. Here, then, and in the parallel passages of the Phaedrus and of the Sophist, is found the germ of the most fruitful notion of modern science.

\par  Plato describes with ludicrous exaggeration the influence exerted by the one and many on the minds of young men in their first fervour of metaphysical enthusiasm (compare Republic). But they are none the less an everlasting quality of reason or reasoning which never grows old in us. At first we have but a confused conception of them, analogous to the eyes blinking at the light in the Republic. To this Plato opposes the revelation from Heaven of the real relations of them, which some Prometheus, who gave the true fire from heaven, is supposed to have imparted to us. Plato is speaking of two things—(1) the crude notion of the one and many, which powerfully affects the ordinary mind when first beginning to think; (2) the same notion when cleared up by the help of dialectic.

\par  To us the problem of the one and many has lost its chief interest and perplexity. We readily acknowledge that a whole has many parts, that the continuous is also the divisible, that in all objects of sense there is a one and many, and that a like principle may be applied to analogy to purely intellectual conceptions. If we attend to the meaning of the words, we are compelled to admit that two contradictory statements are true. But the antinomy is so familiar as to be scarcely observed by us. Our sense of the contradiction, like Plato's, only begins in a higher sphere, when we speak of necessity and free-will, of mind and body, of Three Persons and One Substance, and the like. The world of knowledge is always dividing more and more; every truth is at first the enemy of every other truth. Yet without this division there can be no truth; nor any complete truth without the reunion of the parts into a whole. And hence the coexistence of opposites in the unity of the idea is regarded by Hegel as the supreme principle of philosophy; and the law of contradiction, which is affirmed by logicians to be an ultimate principle of the human mind, is displaced by another law, which asserts the coexistence of contradictories as imperfect and divided elements of the truth. Without entering further into the depths of Hegelianism, we may remark that this and all similar attempts to reconcile antinomies have their origin in the old Platonic problem of the 'One and Many.'

\par  II. 1. The first of Plato's categories or elements is the infinite. This is the negative of measure or limit; the unthinkable, the unknowable; of which nothing can be affirmed; the mixture or chaos which preceded distinct kinds in the creation of the world; the first vague impression of sense; the more or less which refuses to be reduced to rule, having certain affinities with evil, with pleasure, with ignorance, and which in the scale of being is farthest removed from the beautiful and good. To a Greek of the age of Plato, the idea of an infinite mind would have been an absurdity. He would have insisted that 'the good is of the nature of the finite,' and that the infinite is a mere negative, which is on the level of sensation, and not of thought. He was aware that there was a distinction between the infinitely great and the infinitely small, but he would have equally denied the claim of either to true existence. Of that positive infinity, or infinite reality, which we attribute to God, he had no conception.

\par  The Greek conception of the infinite would be more truly described, in our way of speaking, as the indefinite. To us, the notion of infinity is subsequent rather than prior to the finite, expressing not absolute vacancy or negation, but only the removal of limit or restraint, which we suppose to exist not before but after we have already set bounds to thought and matter, and divided them after their kinds. From different points of view, either the finite or infinite may be looked upon respectively both as positive and negative (compare 'Omnis determinatio est negatio')' and the conception of the one determines that of the other. The Greeks and the moderns seem to be nearly at the opposite poles in their manner of regarding them. And both are surprised when they make the discovery, as Plato has done in the Sophist, how large an element negation forms in the framework of their thoughts.

\par  2, 3. The finite element which mingles with and regulates the infinite is best expressed to us by the word 'law.' It is that which measures all things and assigns to them their limit; which preserves them in their natural state, and brings them within the sphere of human cognition. This is described by the terms harmony, health, order, perfection, and the like. All things, in as far as they are good, even pleasures, which are for the most part indefinite, partake of this element. We should be wrong in attributing to Plato the conception of laws of nature derived from observation and experiment. And yet he has as intense a conviction as any modern philosopher that nature does not proceed by chance. But observing that the wonderful construction of number and figure, which he had within himself, and which seemed to be prior to himself, explained a part of the phenomena of the external world, he extended their principles to the whole, finding in them the true type both of human life and of the order of nature.

\par  Two other points may be noticed respecting the third class. First, that Plato seems to be unconscious of any interval or chasm which separates the finite from the infinite. The one is in various ways and degrees working in the other. Hence he has implicitly answered the difficulty with which he started, of how the one could remain one and yet be divided among many individuals, or 'how ideas could be in and out of themselves,' and the like. Secondly, that in this mixed class we find the idea of beauty. Good, when exhibited under the aspect of measure or symmetry, becomes beauty. And if we translate his language into corresponding modern terms, we shall not be far wrong in saying that here, as well as in the Republic, Plato conceives beauty under the idea of proportion.

\par  4. Last and highest in the list of principles or elements is the cause of the union of the finite and infinite, to which Plato ascribes the order of the world. Reasoning from man to the universe, he argues that as there is a mind in the one, there must be a mind in the other, which he identifies with the royal mind of Zeus. This is the first cause of which 'our ancestors spoke,' as he says, appealing to tradition, in the Philebus as well as in the Timaeus. The 'one and many' is also supposed to have been revealed by tradition. For the mythical element has not altogether disappeared.

\par  Some characteristic differences may here be noted, which distinguish the ancient from the modern mode of conceiving God.

\par  a. To Plato, the idea of God or mind is both personal and impersonal. Nor in ascribing, as appears to us, both these attributes to him, and in speaking of God both in the masculine and neuter gender, did he seem to himself inconsistent. For the difference between the personal and impersonal was not marked to him as to ourselves. We make a fundamental distinction between a thing and a person, while to Plato, by the help of various intermediate abstractions, such as end, good, cause, they appear almost to meet in one, or to be two aspects of the same. Hence, without any reconciliation or even remark, in the Republic he speaks at one time of God or Gods, and at another time of the Good. So in the Phaedrus he seems to pass unconsciously from the concrete to the abstract conception of the Ideas in the same dialogue. Nor in the Philebus is he careful to show in what relation the idea of the divine mind stands to the supreme principle of measure.

\par  b. Again, to us there is a strongly-marked distinction between a first cause and a final cause. And we should commonly identify a first cause with God, and the final cause with the world, which is His work. But Plato, though not a Pantheist, and very far from confounding God with the world, tends to identify the first with the final cause. The cause of the union of the finite and infinite might be described as a higher law; the final measure which is the highest expression of the good may also be described as the supreme law. Both these conceptions are realized chiefly by the help of the material world; and therefore when we pass into the sphere of ideas can hardly be distinguished.

\par  The four principles are required for the determination of the relative places of pleasure and wisdom. Plato has been saying that we should proceed by regular steps from the one to the many. Accordingly, before assigning the precedence either to good or pleasure, he must first find out and arrange in order the general principles of things. Mind is ascertained to be akin to the nature of the cause, while pleasure is found in the infinite or indefinite class. We may now proceed to divide pleasure and knowledge after their kinds.

\par  III. 1. Plato speaks of pleasure as indefinite, as relative, as a generation, and in all these points of view as in a category distinct from good. For again we must repeat, that to the Greek 'the good is of the nature of the finite,' and, like virtue, either is, or is nearly allied to, knowledge. The modern philosopher would remark that the indefinite is equally real with the definite. Health and mental qualities are in the concrete undefined; they are nevertheless real goods, and Plato rightly regards them as falling under the finite class. Again, we are able to define objects or ideas, not in so far as they are in the mind, but in so far as they are manifested externally, and can therefore be reduced to rule and measure. And if we adopt the test of definiteness, the pleasures of the body are more capable of being defined than any other pleasures. As in art and knowledge generally, we proceed from without inwards, beginning with facts of sense, and passing to the more ideal conceptions of mental pleasure, happiness, and the like.

\par  2. Pleasure is depreciated as relative, while good is exalted as absolute. But this distinction seems to arise from an unfair mode of regarding them; the abstract idea of the one is compared with the concrete experience of the other. For all pleasure and all knowledge may be viewed either abstracted from the mind, or in relation to the mind (compare Aristot. Nic. Ethics). The first is an idea only, which may be conceived as absolute and unchangeable, and then the abstract idea of pleasure will be equally unchangeable with that of knowledge. But when we come to view either as phenomena of consciousness, the same defects are for the most part incident to both of them. Our hold upon them is equally transient and uncertain; the mind cannot be always in a state of intellectual tension, any more than capable of feeling pleasure always. The knowledge which is at one time clear and distinct, at another seems to fade away, just as the pleasure of health after sickness, or of eating after hunger, soon passes into a neutral state of unconsciousness and indifference. Change and alternation are necessary for the mind as well as for the body; and in this is to be acknowledged, not an element of evil, but rather a law of nature. The chief difference between subjective pleasure and subjective knowledge in respect of permanence is that the latter, when our feeble faculties are able to grasp it, still conveys to us an idea of unchangeableness which cannot be got rid of.

\par  3. In the language of ancient philosophy, the relative character of pleasure is described as becoming or generation. This is relative to Being or Essence, and from one point of view may be regarded as the Heraclitean flux in contrast with the Eleatic Being; from another, as the transient enjoyment of eating and drinking compared with the supposed permanence of intellectual pleasures. But to us the distinction is unmeaning, and belongs to a stage of philosophy which has passed away. Plato himself seems to have suspected that the continuance or life of things is quite as much to be attributed to a principle of rest as of motion (compare Charm. Cratyl.). A later view of pleasure is found in Aristotle, who agrees with Plato in many points, e.g. in his view of pleasure as a restoration to nature, in his distinction between bodily and mental, between necessary and non-necessary pleasures. But he is also in advance of Plato; for he affirms that pleasure is not in the body at all; and hence not even the bodily pleasures are to be spoken of as generations, but only as accompanied by generation (Nic. Eth. ).

\par  4. Plato attempts to identify vicious pleasures with some form of error, and insists that the term false may be applied to them: in this he appears to be carrying out in a confused manner the Socratic doctrine, that virtue is knowledge, vice ignorance. He will allow of no distinction between the pleasures and the erroneous opinions on which they are founded, whether arising out of the illusion of distance or not. But to this we naturally reply with Protarchus, that the pleasure is what it is, although the calculation may be false, or the after-effects painful. It is difficult to acquit Plato, to use his own language, of being a 'tyro in dialectics,' when he overlooks such a distinction. Yet, on the other hand, we are hardly fair judges of confusions of thought in those who view things differently from ourselves.

\par  5. There appears also to be an incorrectness in the notion which occurs both here and in the Gorgias, of the simultaneousness of merely bodily pleasures and pains. We may, perhaps, admit, though even this is not free from doubt, that the feeling of pleasureable hope or recollection is, or rather may be, simultaneous with acute bodily suffering. But there is no such coexistence of the pain of thirst with the pleasures of drinking; they are not really simultaneous, for the one expels the other. Nor does Plato seem to have considered that the bodily pleasures, except in certain extreme cases, are unattended with pain. Few philosophers will deny that a degree of pleasure attends eating and drinking; and yet surely we might as well speak of the pains of digestion which follow, as of the pains of hunger and thirst which precede them. Plato's conception is derived partly from the extreme case of a man suffering pain from hunger or thirst, partly from the image of a full and empty vessel. But the truth is rather, that while the gratification of our bodily desires constantly affords some degree of pleasure, the antecedent pains are scarcely perceived by us, being almost done away with by use and regularity.

\par  6. The desire to classify pleasures as accompanied or not accompanied by antecedent pains, has led Plato to place under one head the pleasures of smell and sight, as well as those derived from sounds of music and from knowledge. He would have done better to make a separate class of the pleasures of smell, having no association of mind, or perhaps to have divided them into natural and artificial. The pleasures of sight and sound might then have been regarded as being the expression of ideas. But this higher and truer point of view never appears to have occurred to Plato. Nor has he any distinction between the fine arts and the mechanical; and, neither here nor anywhere, an adequate conception of the beautiful in external things.

\par  7. Plato agrees partially with certain 'surly or fastidious' philosophers, as he terms them, who defined pleasure to be the absence of pain. They are also described as eminent in physics. There is unfortunately no school of Greek philosophy known to us which combined these two characteristics. Antisthenes, who was an enemy of pleasure, was not a physical philosopher; the atomists, who were physical philosophers, were not enemies of pleasure. Yet such a combination of opinions is far from being impossible. Plato's omission to mention them by name has created the same uncertainty respecting them which also occurs respecting the 'friends of the ideas' and the 'materialists' in the Sophist.

\par  On the whole, this discussion is one of the least satisfactory in the dialogues of Plato. While the ethical nature of pleasure is scarcely considered, and the merely physical phenomenon imperfectly analysed, too much weight is given to ideas of measure and number, as the sole principle of good. The comparison of pleasure and knowledge is really a comparison of two elements, which have no common measure, and which cannot be excluded from each other. Feeling is not opposed to knowledge, and in all consciousness there is an element of both. The most abstract kinds of knowledge are inseparable from some pleasure or pain, which accompanies the acquisition or possession of them: the student is liable to grow weary of them, and soon discovers that continuous mental energy is not granted to men. The most sensual pleasure, on the other hand, is inseparable from the consciousness of pleasure; no man can be happy who, to borrow Plato's illustration, is leading the life of an oyster. Hence (by his own confession) the main thesis is not worth determining; the real interest lies in the incidental discussion. We can no more separate pleasure from knowledge in the Philebus than we can separate justice from happiness in the Republic.

\par  IV. An interesting account is given in the Philebus of the rank and order of the sciences or arts, which agrees generally with the scheme of knowledge in the Sixth Book of the Republic. The chief difference is, that the position of the arts is more exactly defined. They are divided into an empirical part and a scientific part, of which the first is mere guess-work, the second is determined by rule and measure. Of the more empirical arts, music is given as an example; this, although affirmed to be necessary to human life, is depreciated. Music is regarded from a point of view entirely opposite to that of the Republic, not as a sublime science, coordinate with astronomy, but as full of doubt and conjecture. According to the standard of accuracy which is here adopted, it is rightly placed lower in the scale than carpentering, because the latter is more capable of being reduced to measure.

\par  The theoretical element of the arts may also become a purely abstract science, when separated from matter, and is then said to be pure and unmixed. The distinction which Plato here makes seems to be the same as that between pure and applied mathematics, and may be expressed in the modern formula—science is art theoretical, art is science practical. In the reason which he gives for the superiority of the pure science of number over the mixed or applied, we can only agree with him in part. He says that the numbers which the philosopher employs are always the same, whereas the numbers which are used in practice represent different sizes or quantities. He does not see that this power of expressing different quantities by the same symbol is the characteristic and not the defect of numbers, and is due to their abstract nature;—although we admit of course what Plato seems to feel in his distinctions between pure and impure knowledge, that the imperfection of matter enters into the applications of them.

\par  Above the other sciences, as in the Republic, towers dialectic, which is the science of eternal Being, apprehended by the purest mind and reason. The lower sciences, including the mathematical, are akin to opinion rather than to reason, and are placed together in the fourth class of goods. The relation in which they stand to dialectic is obscure in the Republic, and is not cleared up in the Philebus.

\par  V. Thus far we have only attained to the vestibule or ante-chamber of the good; for there is a good exceeding knowledge, exceeding essence, which, like Glaucon in the Republic, we find a difficulty in apprehending. This good is now to be exhibited to us under various aspects and gradations. The relative dignity of pleasure and knowledge has been determined; but they have not yet received their exact position in the scale of goods. Some difficulties occur to us in the enumeration: First, how are we to distinguish the first from the second class of goods, or the second from the third? Secondly, why is there no mention of the supreme mind? Thirdly, the nature of the fourth class. Fourthly, the meaning of the allusion to a sixth class, which is not further investigated.

\par  (I) Plato seems to proceed in his table of goods, from the more abstract to the less abstract; from the subjective to the objective; until at the lower end of the scale we fairly descend into the region of human action and feeling. To him, the greater the abstraction the greater the truth, and he is always tending to see abstractions within abstractions; which, like the ideas in the Parmenides, are always appearing one behind another. Hence we find a difficulty in following him into the sphere of thought which he is seeking to attain. First in his scale of goods he places measure, in which he finds the eternal nature: this would be more naturally expressed in modern language as eternal law, and seems to be akin both to the finite and to the mind or cause, which were two of the elements in the former table. Like the supreme nature in the Timaeus, like the ideal beauty in the Symposium or the Phaedrus, or like the ideal good in the Republic, this is the absolute and unapproachable being. But this being is manifested in symmetry and beauty everywhere, in the order of nature and of mind, in the relations of men to one another. For the word 'measure' he now substitutes the word 'symmetry,' as if intending to express measure conceived as relation. He then proceeds to regard the good no longer in an objective form, but as the human reason seeking to attain truth by the aid of dialectic; such at least we naturally infer to be his meaning, when we consider that both here and in the Republic the sphere of nous or mind is assigned to dialectic. (2) It is remarkable (see above) that this personal conception of mind is confined to the human mind, and not extended to the divine. (3) If we may be allowed to interpret one dialogue of Plato by another, the sciences of figure and number are probably classed with the arts and true opinions, because they proceed from hypotheses (compare Republic). (4) The sixth class, if a sixth class is to be added, is playfully set aside by a quotation from Orpheus: Plato means to say that a sixth class, if there be such a class, is not worth considering, because pleasure, having only gained the fifth place in the scale of goods, is already out of the running.

\par  VI. We may now endeavour to ascertain the relation of the Philebus to the other dialogues. Here Plato shows the same indifference to his own doctrine of Ideas which he has already manifested in the Parmenides and the Sophist. The principle of the one and many of which he here speaks, is illustrated by examples in the Sophist and Statesman. Notwithstanding the differences of style, many resemblances may be noticed between the Philebus and Gorgias. The theory of the simultaneousness of pleasure and pain is common to both of them (Phil. Gorg. ); there is also a common tendency in them to take up arms against pleasure, although the view of the Philebus, which is probably the later of the two dialogues, is the more moderate. There seems to be an allusion to the passage in the Gorgias, in which Socrates dilates on the pleasures of itching and scratching. Nor is there any real discrepancy in the manner in which Gorgias and his art are spoken of in the two dialogues. For Socrates is far from implying that the art of rhetoric has a real sphere of practical usefulness: he only means that the refutation of the claims of Gorgias is not necessary for his present purpose. He is saying in effect: 'Admit, if you please, that rhetoric is the greatest and usefullest of sciences:—this does not prove that dialectic is not the purest and most exact.' From the Sophist and Statesman we know that his hostility towards the sophists and rhetoricians was not mitigated in later life; although both in the Statesman and Laws he admits of a higher use of rhetoric.

\par  Reasons have been already given for assigning a late date to the Philebus. That the date is probably later than that of the Republic, may be further argued on the following grounds:—1. The general resemblance to the later dialogues and to the Laws: 2. The more complete account of the nature of good and pleasure: 3. The distinction between perception, memory, recollection, and opinion which indicates a great progress in psychology; also between understanding and imagination, which is described under the figure of the scribe and the painter. A superficial notion may arise that Plato probably wrote shorter dialogues, such as the Philebus, the Sophist, and the Statesman, as studies or preparations for longer ones. This view may be natural; but on further reflection is seen to be fallacious, because these three dialogues are found to make an advance upon the metaphysical conceptions of the Republic. And we can more easily suppose that Plato composed shorter writings after longer ones, than suppose that he lost hold of further points of view which he had once attained.

\par  It is more easy to find traces of the Pythagoreans, Eleatics, Megarians, Cynics, Cyrenaics and of the ideas of Anaxagoras, in the Philebus, than to say how much is due to each of them. Had we fuller records of those old philosophers, we should probably find Plato in the midst of the fray attempting to combine Eleatic and Pythagorean doctrines, and seeking to find a truth beyond either Being or number; setting up his own concrete conception of good against the abstract practical good of the Cynics, or the abstract intellectual good of the Megarians, and his own idea of classification against the denial of plurality in unity which is also attributed to them; warring against the Eristics as destructive of truth, as he had formerly fought against the Sophists; taking up a middle position between the Cynics and Cyrenaics in his doctrine of pleasure; asserting with more consistency than Anaxagoras the existence of an intelligent mind and cause. Of the Heracliteans, whom he is said by Aristotle to have cultivated in his youth, he speaks in the Philebus, as in the Theaetetus and Cratylus, with irony and contempt. But we have not the knowledge which would enable us to pursue further the line of reflection here indicated; nor can we expect to find perfect clearness or order in the first efforts of mankind to understand the working of their own minds. The ideas which they are attempting to analyse, they are also in process of creating; the abstract universals of which they are seeking to adjust the relations have been already excluded by them from the category of relation.

\par  ...

\par  The Philebus, like the Cratylus, is supposed to be the continuation of a previous discussion. An argument respecting the comparative claims of pleasure and wisdom to rank as the chief good has been already carried on between Philebus and Socrates. The argument is now transferred to Protarchus, the son of Callias, a noble Athenian youth, sprung from a family which had spent 'a world of money' on the Sophists (compare Apol. ; Crat. ; Protag.). Philebus, who appears to be the teacher, or elder friend, and perhaps the lover, of Protarchus, takes no further part in the discussion beyond asserting in the strongest manner his adherence, under all circumstances, to the cause of pleasure.

\par  Socrates suggests that they shall have a first and second palm of victory. For there may be a good higher than either pleasure or wisdom, and then neither of them will gain the first prize, but whichever of the two is more akin to this higher good will have a right to the second. They agree, and Socrates opens the game by enlarging on the diversity and opposition which exists among pleasures. For there are pleasures of all kinds, good and bad, wise and foolish—pleasures of the temperate as well as of the intemperate. Protarchus replies that although pleasures may be opposed in so far as they spring from opposite sources, nevertheless as pleasures they are alike. Yes, retorts Socrates, pleasure is like pleasure, as figure is like figure and colour like colour; yet we all know that there is great variety among figures and colours. Protarchus does not see the drift of this remark; and Socrates proceeds to ask how he can have a right to attribute a new predicate (i.e. 'good') to pleasures in general, when he cannot deny that they are different? What common property in all of them does he mean to indicate by the term 'good'? If he continues to assert that there is some trivial sense in which pleasure is one, Socrates may retort by saying that knowledge is one, but the result will be that such merely verbal and trivial conceptions, whether of knowledge or pleasure, will spoil the discussion, and will prove the incapacity of the two disputants. In order to avoid this danger, he proposes that they shall beat a retreat, and, before they proceed, come to an understanding about the 'high argument' of the one and the many.

\par  Protarchus agrees to the proposal, but he is under the impression that Socrates means to discuss the common question—how a sensible object can be one, and yet have opposite attributes, such as 'great' and 'small,' 'light' and 'heavy,' or how there can be many members in one body, and the like wonders. Socrates has long ceased to see any wonder in these phenomena; his difficulties begin with the application of number to abstract unities (e.g. 'man,' 'good') and with the attempt to divide them. For have these unities of idea any real existence? How, if imperishable, can they enter into the world of generation? How, as units, can they be divided and dispersed among different objects? Or do they exist in their entirety in each object? These difficulties are but imperfectly answered by Socrates in what follows.

\par  We speak of a one and many, which is ever flowing in and out of all things, concerning which a young man often runs wild in his first metaphysical enthusiasm, talking about analysis and synthesis to his father and mother and the neighbours, hardly sparing even his dog. This 'one in many' is a revelation of the order of the world, which some Prometheus first made known to our ancestors; and they, who were better men and nearer the gods than we are, have handed it down to us. To know how to proceed by regular steps from one to many, and from many to one, is just what makes the difference between eristic and dialectic. And the right way of proceeding is to look for one idea or class in all things, and when you have found one to look for more than one, and for all that there are, and when you have found them all and regularly divided a particular field of knowledge into classes, you may leave the further consideration of individuals. But you must not pass at once either from unity to infinity, or from infinity to unity. In music, for example, you may begin with the most general notion, but this alone will not make you a musician: you must know also the number and nature of the intervals, and the systems which are framed out of them, and the rhythms of the dance which correspond to them. And when you have a similar knowledge of any other subject, you may be said to know that subject. In speech again there are infinite varieties of sound, and some one who was a wise man, or more than man, comprehended them all in the classes of mutes, vowels, and semivowels, and gave to each of them a name, and assigned them to the art of grammar.

\par  'But whither, Socrates, are you going? And what has this to do with the comparative eligibility of pleasure and wisdom:' Socrates replies, that before we can adjust their respective claims, we want to know the number and kinds of both of them. What are they? He is requested to answer the question himself. That he will, if he may be allowed to make one or two preliminary remarks. In the first place he has a dreamy recollection of hearing that neither pleasure nor knowledge is the highest good, for the good should be perfect and sufficient. But is the life of pleasure perfect and sufficient, when deprived of memory, consciousness, anticipation? Is not this the life of an oyster? Or is the life of mind sufficient, if devoid of any particle of pleasure? Must not the union of the two be higher and more eligible than either separately? And is not the element which makes this mixed life eligible more akin to mind than to pleasure? Thus pleasure is rejected and mind is rejected. And yet there may be a life of mind, not human but divine, which conquers still.

\par  But, if we are to pursue this argument further, we shall require some new weapons; and by this, I mean a new classification of existence. (1) There is a finite element of existence, and (2) an infinite, and (3) the union of the two, and (4) the cause of the union. More may be added if they are wanted, but at present we can do without them. And first of the infinite or indefinite:—That is the class which is denoted by the terms more or less, and is always in a state of comparison. All words or ideas to which the words 'gently,' 'extremely,' and other comparative expressions are applied, fall under this class. The infinite would be no longer infinite, if limited or reduced to measure by number and quantity. The opposite class is the limited or finite, and includes all things which have number and quantity. And there is a third class of generation into essence by the union of the finite and infinite, in which the finite gives law to the infinite;—under this are comprehended health, strength, temperate seasons, harmony, beauty, and the like. The goddess of beauty saw the universal wantonness of all things, and gave law and order to be the salvation of the soul. But no effect can be generated without a cause, and therefore there must be a fourth class, which is the cause of generation; for the cause or agent is not the same as the patient or effect.

\par  And now, having obtained our classes, we may determine in which our conqueror life is to be placed: Clearly in the third or mixed class, in which the finite gives law to the infinite. And in which is pleasure to find a place? As clearly in the infinite or indefinite, which alone, as Protarchus thinks (who seems to confuse the infinite with the superlative), gives to pleasure the character of the absolute good. Yes, retorts Socrates, and also to pain the character of absolute evil. And therefore the infinite cannot be that which imparts to pleasure the nature of the good. But where shall we place mind? That is a very serious and awful question, which may be prefaced by another. Is mind or chance the lord of the universe? All philosophers will say the first, and yet, perhaps, they may be only magnifying themselves. And for this reason I should like to consider the matter a little more deeply, even though some lovers of disorder in the world should ridicule my attempt.

\par  Now the elements earth, air, fire, water, exist in us, and they exist in the cosmos; but they are purer and fairer in the cosmos than they are in us, and they come to us from thence. And as we have a soul as well as a body, in like manner the elements of the finite, the infinite, the union of the two, and the cause, are found to exist in us. And if they, like the elements, exist in us, and the three first exist in the world, must not the fourth or cause which is the noblest of them, exist in the world? And this cause is wisdom or mind, the royal mind of Zeus, who is the king of all, as there are other gods who have other noble attributes. Observe how well this agrees with the testimony of men of old, who affirmed mind to be the ruler of the universe. And remember that mind belongs to the class which we term the cause, and pleasure to the infinite or indefinite class. We will examine the place and origin of both.

\par  What is the origin of pleasure? Her natural seat is the mixed class, in which health and harmony were placed. Pain is the violation, and pleasure the restoration of limit. There is a natural union of finite and infinite, which in hunger, thirst, heat, cold, is impaired—this is painful, but the return to nature, in which the elements are restored to their normal proportions, is pleasant. Here is our first class of pleasures. And another class of pleasures and pains are hopes and fears; these are in the mind only. And inasmuch as the pleasures are unalloyed by pains and the pains by pleasures, the examination of them may show us whether all pleasure is to be desired, or whether this entire desirableness is not rather the attribute of another class. But if pleasures and pains consist in the violation and restoration of limit, may there not be a neutral state, in which there is neither dissolution nor restoration? That is a further question, and admitting, as we must, the possibility of such a state, there seems to be no reason why the life of wisdom should not exist in this neutral state, which is, moreover, the state of the gods, who cannot, without indecency, be supposed to feel either joy or sorrow.

\par  The second class of pleasures involves memory. There are affections which are extinguished before they reach the soul, and of these there is no consciousness, and therefore no memory. And there are affections which the body and soul feel together, and this feeling is termed consciousness. And memory is the preservation of consciousness, and reminiscence is the recovery of consciousness. Now the memory of pleasure, when a man is in pain, is the memory of the opposite of his actual bodily state, and is therefore not in the body, but in the mind. And there may be an intermediate state, in which a person is balanced between pleasure and pain; in his body there is want which is a cause of pain, but in his mind a sure hope of replenishment, which is pleasant. (But if the hope be converted into despair, he has two pains and not a balance of pain and pleasure.) Another question is raised: May not pleasures, like opinions, be true and false? In the sense of being real, both must be admitted to be true: nor can we deny that to both of them qualities may be attributed; for pleasures as well as opinions may be described as good or bad. And though we do not all of us allow that there are true and false pleasures, we all acknowledge that there are some pleasures associated with right opinion, and others with falsehood and ignorance. Let us endeavour to analyze the nature of this association.

\par  Opinion is based on perception, which may be correct or mistaken. You may see a figure at a distance, and say first of all, 'This is a man,' and then say, 'No, this is an image made by the shepherds.' And you may affirm this in a proposition to your companion, or make the remark mentally to yourself. Whether the words are actually spoken or not, on such occasions there is a scribe within who registers them, and a painter who paints the images of the things which the scribe has written down in the soul,—at least that is my own notion of the process; and the words and images which are inscribed by them may be either true or false; and they may represent either past, present, or future. And, representing the future, they must also represent the pleasures and pains of anticipation—the visions of gold and other fancies which are never wanting in the mind of man. Now these hopes, as they are termed, are propositions, which are sometimes true, and sometimes false; for the good, who are the friends of the gods, see true pictures of the future, and the bad false ones. And as there may be opinion about things which are not, were not, and will not be, which is opinion still, so there may be pleasure about things which are not, were not, and will not be, which is pleasure still,—that is to say, false pleasure; and only when false, can pleasure, like opinion, be vicious. Against this conclusion Protarchus reclaims.

\par  Leaving his denial for the present, Socrates proceeds to show that some pleasures are false from another point of view. In desire, as we admitted, the body is divided from the soul, and hence pleasures and pains are often simultaneous. And we further admitted that both of them belonged to the infinite class. How, then, can we compare them? Are we not liable, or rather certain, as in the case of sight, to be deceived by distance and relation? In this case the pleasures and pains are not false because based upon false opinion, but are themselves false. And there is another illusion: pain has often been said by us to arise out of the derangement—pleasure out of the restoration—of our nature. But in passing from one to the other, do we not experience neutral states, which although they appear pleasureable or painful are really neither? For even if we admit, with the wise man whom Protarchus loves (and only a wise man could have ever entertained such a notion), that all things are in a perpetual flux, still these changes are often unconscious, and devoid either of pleasure or pain. We assume, then, that there are three states—pleasureable, painful, neutral; we may embellish a little by calling them gold, silver, and that which is neither.

\par  But there are certain natural philosophers who will not admit a third state. Their instinctive dislike to pleasure leads them to affirm that pleasure is only the absence of pain. They are noble fellows, and, although we do not agree with them, we may use them as diviners who will indicate to us the right track. They will say, that the nature of anything is best known from the examination of extreme cases, e.g. the nature of hardness from the examination of the hardest things; and that the nature of pleasure will be best understood from an examination of the most intense pleasures. Now these are the pleasures of the body, not of the mind; the pleasures of disease and not of health, the pleasures of the intemperate and not of the temperate. I am speaking, not of the frequency or continuance, but only of the intensity of such pleasures, and this is given them by contrast with the pain or sickness of body which precedes them. Their morbid nature is illustrated by the lesser instances of itching and scratching, respecting which I swear that I cannot tell whether they are a pleasure or a pain. (1) Some of these arise out of a transition from one state of the body to another, as from cold to hot; (2) others are caused by the contrast of an internal pain and an external pleasure in the body: sometimes the feeling of pain predominates, as in itching and tingling, when they are relieved by scratching; sometimes the feeling of pleasure: or the pleasure which they give may be quite overpowering, and is then accompanied by all sorts of unutterable feelings which have a death of delights in them. But there are also mixed pleasures which are in the mind only. For are not love and sorrow as well as anger 'sweeter than honey,' and also full of pain? Is there not a mixture of feelings in the spectator of tragedy? and of comedy also? 'I do not understand that last.' Well, then, with the view of lighting up the obscurity of these mixed feelings, let me ask whether envy is painful. 'Yes.' And yet the envious man finds something pleasing in the misfortunes of others? 'True.' And ignorance is a misfortune? 'Certainly.' And one form of ignorance is self-conceit—a man may fancy himself richer, fairer, better, wiser than he is? 'Yes.' And he who thus deceives himself may be strong or weak? 'He may.' And if he is strong we fear him, and if he is weak we laugh at him, which is a pleasure, and yet we envy him, which is a pain? These mixed feelings are the rationale of tragedy and comedy, and equally the rationale of the greater drama of human life. (There appears to be some confusion in this passage. There is no difficulty in seeing that in comedy, as in tragedy, the spectator may view the performance with mixed feelings of pain as well as of pleasure; nor is there any difficulty in understanding that envy is a mixed feeling, which rejoices not without pain at the misfortunes of others, and laughs at their ignorance of themselves. But Plato seems to think further that he has explained the feeling of the spectator in comedy sufficiently by a theory which only applies to comedy in so far as in comedy we laugh at the conceit or weakness of others. He has certainly given a very partial explanation of the ridiculous.) Having shown how sorrow, anger, envy are feelings of a mixed nature, I will reserve the consideration of the remainder for another occasion.

\par  Next follow the unmixed pleasures; which, unlike the philosophers of whom I was speaking, I believe to be real. These unmixed pleasures are: (1) The pleasures derived from beauty of form, colour, sound, smell, which are absolutely pure; and in general those which are unalloyed with pain: (2) The pleasures derived from the acquisition of knowledge, which in themselves are pure, but may be attended by an accidental pain of forgetting; this, however, arises from a subsequent act of reflection, of which we need take no account. At the same time, we admit that the latter pleasures are the property of a very few. To these pure and unmixed pleasures we ascribe measure, whereas all others belong to the class of the infinite, and are liable to every species of excess. And here several questions arise for consideration:—What is the meaning of pure and impure, of moderate and immoderate? We may answer the question by an illustration: Purity of white paint consists in the clearness or quality of the white, and this is distinct from the quantity or amount of white paint; a little pure white is fairer than a great deal which is impure. But there is another question:—Pleasure is affirmed by ingenious philosophers to be a generation; they say that there are two natures—one self-existent, the other dependent; the one noble and majestic, the other failing in both these qualities. 'I do not understand.' There are lovers and there are loves. 'Yes, I know, but what is the application?' The argument is in play, and desires to intimate that there are relatives and there are absolutes, and that the relative is for the sake of the absolute; and generation is for the sake of essence. Under relatives I class all things done with a view to generation; and essence is of the class of good. But if essence is of the class of good, generation must be of some other class; and our friends, who affirm that pleasure is a generation, would laugh at the notion that pleasure is a good; and at that other notion, that pleasure is produced by generation, which is only the alternative of destruction. Who would prefer such an alternation to the equable life of pure thought? Here is one absurdity, and not the only one, to which the friends of pleasure are reduced. For is there not also an absurdity in affirming that good is of the soul only; or in declaring that the best of men, if he be in pain, is bad?

\par  And now, from the consideration of pleasure, we pass to that of knowledge. Let us reflect that there are two kinds of knowledge—the one creative or productive, and the other educational and philosophical. Of the creative arts, there is one part purer or more akin to knowledge than the other. There is an element of guess-work and an element of number and measure in them. In music, for example, especially in flute-playing, the conjectural element prevails; while in carpentering there is more application of rule and measure. Of the creative arts, then, we may make two classes—the less exact and the more exact. And the exacter part of all of them is really arithmetic and mensuration. But arithmetic and mensuration again may be subdivided with reference either to their use in the concrete, or to their nature in the abstract—as they are regarded popularly in building and binding, or theoretically by philosophers. And, borrowing the analogy of pleasure, we may say that the philosophical use of them is purer than the other. Thus we have two arts of arithmetic, and two of mensuration. And truest of all in the estimation of every rational man is dialectic, or the science of being, which will forget and disown us, if we forget and disown her.

\par  'But, Socrates, I have heard Gorgias say that rhetoric is the greatest and usefullest of arts; and I should not like to quarrel either with him or you.' Neither is there any inconsistency, Protarchus, with his statement in what I am now saying; for I am not maintaining that dialectic is the greatest or usefullest, but only that she is the truest of arts; my remark is not quantitative but qualitative, and refers not to the advantage or repetition of either, but to the degree of truth which they attain—here Gorgias will not care to compete; this is what we affirm to be possessed in the highest degree by dialectic. And do not let us appeal to Gorgias or Philebus or Socrates, but ask, on behalf of the argument, what are the highest truths which the soul has the power of attaining. And is not this the science which has a firmer grasp of them than any other? For the arts generally are only occupied with matters of opinion, and with the production and action and passion of this sensible world. But the highest truth is that which is eternal and unchangeable. And reason and wisdom are concerned with the eternal; and these are the very claimants, if not for the first, at least for the second place, whom I propose as rivals to pleasure.

\par  And now, having the materials, we may proceed to mix them—first recapitulating the question at issue.

\par  Philebus affirmed pleasure to be the good, and assumed them to be one nature; I affirmed that they were two natures, and declared that knowledge was more akin to the good than pleasure. I said that the two together were more eligible than either taken singly; and to this we adhere. Reason intimates, as at first, that we should seek the good not in the unmixed life, but in the mixed.

\par  The cup is ready, waiting to be mingled, and here are two fountains, one of honey, the other of pure water, out of which to make the fairest possible mixture. There are pure and impure pleasures—pure and impure sciences. Let us consider the sections of each which have the most of purity and truth; to admit them all indiscriminately would be dangerous. First we will take the pure sciences; but shall we mingle the impure—the art which uses the false rule and the false measure? That we must, if we are any of us to find our way home; man cannot live upon pure mathematics alone. And must I include music, which is admitted to be guess-work? 'Yes, you must, if human life is to have any humanity.' Well, then, I will open the door and let them all in; they shall mingle in an Homeric 'meeting of the waters.' And now we turn to the pleasures; shall I admit them? 'Admit first of all the pure pleasures; secondly, the necessary.' And what shall we say about the rest? First, ask the pleasures—they will be too happy to dwell with wisdom. Secondly, ask the arts and sciences—they reply that the excesses of intemperance are the ruin of them; and that they would rather only have the pleasures of health and temperance, which are the handmaidens of virtue. But still we want truth? That is now added; and so the argument is complete, and may be compared to an incorporeal law, which is to hold fair rule over a living body. And now we are at the vestibule of the good, in which there are three chief elements—truth, symmetry, and beauty. These will be the criterion of the comparative claims of pleasure and wisdom.

\par  Which has the greater share of truth? Surely wisdom; for pleasure is the veriest impostor in the world, and the perjuries of lovers have passed into a proverb.

\par  Which of symmetry? Wisdom again; for nothing is more immoderate than pleasure.

\par  Which of beauty? Once more, wisdom; for pleasure is often unseemly, and the greatest pleasures are put out of sight.

\par  Not pleasure, then, ranks first in the scale of good, but measure, and eternal harmony.

\par  Second comes the symmetrical and beautiful and perfect.

\par  Third, mind and wisdom.

\par  Fourth, sciences and arts and true opinions.

\par  Fifth, painless pleasures.

\par  Of a sixth class, I have no more to say. Thus, pleasure and mind may both renounce the claim to the first place. But mind is ten thousand times nearer to the chief good than pleasure. Pleasure ranks fifth and not first, even though all the animals in the world assert the contrary.

\par  ...

\par  From the days of Aristippus and Epicurus to our own times the nature of pleasure has occupied the attention of philosophers. 'Is pleasure an evil? a good? the only good?' are the simple forms which the enquiry assumed among the Socratic schools. But at an early stage of the controversy another question was asked: 'Do pleasures differ in kind? and are some bad, some good, and some neither bad nor good?' There are bodily and there are mental pleasures, which were at first confused but afterwards distinguished. A distinction was also made between necessary and unnecessary pleasures; and again between pleasures which had or had not corresponding pains. The ancient philosophers were fond of asking, in the language of their age, 'Is pleasure a "becoming" only, and therefore transient and relative, or do some pleasures partake of truth and Being?' To these ancient speculations the moderns have added a further question:—'Whose pleasure? The pleasure of yourself, or of your neighbour,—of the individual, or of the world?' This little addition has changed the whole aspect of the discussion: the same word is now supposed to include two principles as widely different as benevolence and self-love. Some modern writers have also distinguished between pleasure the test, and pleasure the motive of actions. For the universal test of right actions (how I know them) may not always be the highest or best motive of them (why I do them).

\par  Socrates, as we learn from the Memorabilia of Xenophon, first drew attention to the consequences of actions. Mankind were said by him to act rightly when they knew what they were doing, or, in the language of the Gorgias, 'did what they would.' He seems to have been the first who maintained that the good was the useful (Mem.). In his eagerness for generalization, seeking, as Aristotle says, for the universal in Ethics (Metaph. ), he took the most obvious intellectual aspect of human action which occurred to him. He meant to emphasize, not pleasure, but the calculation of pleasure; neither is he arguing that pleasure is the chief good, but that we should have a principle of choice. He did not intend to oppose 'the useful' to some higher conception, such as the Platonic ideal, but to chance and caprice. The Platonic Socrates pursues the same vein of thought in the Protagoras, where he argues against the so-called sophist that pleasure and pain are the final standards and motives of good and evil, and that the salvation of human life depends upon a right estimate of pleasures greater or less when seen near and at a distance. The testimony of Xenophon is thus confirmed by that of Plato, and we are therefore justified in calling Socrates the first utilitarian; as indeed there is no side or aspect of philosophy which may not with reason be ascribed to him—he is Cynic and Cyrenaic, Platonist and Aristotelian in one. But in the Phaedo the Socratic has already passed into a more ideal point of view; and he, or rather Plato speaking in his person, expressly repudiates the notion that the exchange of a less pleasure for a greater can be an exchange of virtue. Such virtue is the virtue of ordinary men who live in the world of appearance; they are temperate only that they may enjoy the pleasures of intemperance, and courageous from fear of danger. Whereas the philosopher is seeking after wisdom and not after pleasure, whether near or distant: he is the mystic, the initiated, who has learnt to despise the body and is yearning all his life long for a truth which will hereafter be revealed to him. In the Republic the pleasures of knowledge are affirmed to be superior to other pleasures, because the philosopher so estimates them; and he alone has had experience of both kinds. (Compare a similar argument urged by one of the latest defenders of Utilitarianism, Mill's Utilitarianism). In the Philebus, Plato, although he regards the enemies of pleasure with complacency, still further modifies the transcendentalism of the Phaedo. For he is compelled to confess, rather reluctantly, perhaps, that some pleasures, i.e. those which have no antecedent pains, claim a place in the scale of goods.

\par  There have been many reasons why not only Plato but mankind in general have been unwilling to acknowledge that 'pleasure is the chief good.' Either they have heard a voice calling to them out of another world; or the life and example of some great teacher has cast their thoughts of right and wrong in another mould; or the word 'pleasure' has been associated in their mind with merely animal enjoyment. They could not believe that what they were always striving to overcome, and the power or principle in them which overcame, were of the same nature. The pleasure of doing good to others and of bodily self-indulgence, the pleasures of intellect and the pleasures of sense, are so different:—Why then should they be called by a common name? Or, if the equivocal or metaphorical use of the word is justified by custom (like the use of other words which at first referred only to the body, and then by a figure have been transferred to the mind), still, why should we make an ambiguous word the corner-stone of moral philosophy? To the higher thinker the Utilitarian or hedonist mode of speaking has been at variance with religion and with any higher conception both of politics and of morals. It has not satisfied their imagination; it has offended their taste. To elevate pleasure, 'the most fleeting of all things,' into a general idea seems to such men a contradiction. They do not desire to bring down their theory to the level of their practice. The simplicity of the 'greatest happiness' principle has been acceptable to philosophers, but the better part of the world has been slow to receive it.

\par  Before proceeding, we may make a few admissions which will narrow the field of dispute; and we may as well leave behind a few prejudices, which intelligent opponents of Utilitarianism have by this time 'agreed to discard'. We admit that Utility is coextensive with right, and that no action can be right which does not tend to the happiness of mankind; we acknowledge that a large class of actions are made right or wrong by their consequences only; we say further that mankind are not too mindful, but that they are far too regardless of consequences, and that they need to have the doctrine of utility habitually inculcated on them. We recognize the value of a principle which can supply a connecting link between Ethics and Politics, and under which all human actions are or may be included. The desire to promote happiness is no mean preference of expediency to right, but one of the highest and noblest motives by which human nature can be animated. Neither in referring actions to the test of utility have we to make a laborious calculation, any more than in trying them by other standards of morals. For long ago they have been classified sufficiently for all practical purposes by the thinker, by the legislator, by the opinion of the world. Whatever may be the hypothesis on which they are explained, or which in doubtful cases may be applied to the regulation of them, we are very rarely, if ever, called upon at the moment of performing them to determine their effect upon the happiness of mankind.

\par  There is a theory which has been contrasted with Utility by Paley and others—the theory of a moral sense: Are our ideas of right and wrong innate or derived from experience? This, perhaps, is another of those speculations which intelligent men might 'agree to discard.' For it has been worn threadbare; and either alternative is equally consistent with a transcendental or with an eudaemonistic system of ethics, with a greatest happiness principle or with Kant's law of duty. Yet to avoid misconception, what appears to be the truth about the origin of our moral ideas may be shortly summed up as follows:—To each of us individually our moral ideas come first of all in childhood through the medium of education, from parents and teachers, assisted by the unconscious influence of language; they are impressed upon a mind which at first is like a waxen tablet, adapted to receive them; but they soon become fixed or set, and in after life are strengthened, or perhaps weakened by the force of public opinion. They may be corrected and enlarged by experience, they may be reasoned about, they may be brought home to us by the circumstances of our lives, they may be intensified by imagination, by reflection, by a course of action likely to confirm them. Under the influence of religious feeling or by an effort of thought, any one beginning with the ordinary rules of morality may create out of them for himself ideals of holiness and virtue. They slumber in the minds of most men, yet in all of us there remains some tincture of affection, some desire of good, some sense of truth, some fear of the law. Of some such state or process each individual is conscious in himself, and if he compares his own experience with that of others he will find the witness of their consciences to coincide with that of his own. All of us have entered into an inheritance which we have the power of appropriating and making use of. No great effort of mind is required on our part; we learn morals, as we learn to talk, instinctively, from conversing with others, in an enlightened age, in a civilized country, in a good home. A well-educated child of ten years old already knows the essentials of morals: 'Thou shalt not steal,' 'thou shalt speak the truth,' 'thou shalt love thy parents,' 'thou shalt fear God.' What more does he want?

\par  But whence comes this common inheritance or stock of moral ideas? Their beginning, like all other beginnings of human things, is obscure, and is the least important part of them. Imagine, if you will, that Society originated in the herding of brutes, in their parental instincts, in their rude attempts at self-preservation:—Man is not man in that he resembles, but in that he differs from them. We must pass into another cycle of existence, before we can discover in him by any evidence accessible to us even the germs of our moral ideas. In the history of the world, which viewed from within is the history of the human mind, they have been slowly created by religion, by poetry, by law, having their foundation in the natural affections and in the necessity of some degree of truth and justice in a social state; they have been deepened and enlarged by the efforts of great thinkers who have idealized and connected them—by the lives of saints and prophets who have taught and exemplified them. The schools of ancient philosophy which seem so far from us—Socrates, Plato, Aristotle, the Stoics, the Epicureans, and a few modern teachers, such as Kant and Bentham, have each of them supplied 'moments' of thought to the world. The life of Christ has embodied a divine love, wisdom, patience, reasonableness. For his image, however imperfectly handed down to us, the modern world has received a standard more perfect in idea than the societies of ancient times, but also further removed from practice. For there is certainly a greater interval between the theory and practice of Christians than between the theory and practice of the Greeks and Romans; the ideal is more above us, and the aspiration after good has often lent a strange power to evil. And sometimes, as at the Reformation, or French Revolution, when the upper classes of a so-called Christian country have become corrupted by priestcraft, by casuistry, by licentiousness, by despotism, the lower have risen up and re-asserted the natural sense of religion and right.

\par  We may further remark that our moral ideas, as the world grows older, perhaps as we grow older ourselves, unless they have been undermined in us by false philosophy or the practice of mental analysis, or infected by the corruption of society or by some moral disorder in the individual, are constantly assuming a more natural and necessary character. The habit of the mind, the opinion of the world, familiarizes them to us; and they take more and more the form of immediate intuition. The moral sense comes last and not first in the order of their development, and is the instinct which we have inherited or acquired, not the nobler effort of reflection which created them and which keeps them alive. We do not stop to reason about common honesty. Whenever we are not blinded by self-deceit, as for example in judging the actions of others, we have no hesitation in determining what is right and wrong. The principles of morality, when not at variance with some desire or worldly interest of our own, or with the opinion of the public, are hardly perceived by us; but in the conflict of reason and passion they assert their authority and are not overcome without remorse.

\par  Such is a brief outline of the history of our moral ideas. We have to distinguish, first of all, the manner in which they have grown up in the world from the manner in which they have been communicated to each of us. We may represent them to ourselves as flowing out of the boundless ocean of language and thought in little rills, which convey them to the heart and brain of each individual. But neither must we confound the theories or aspects of morality with the origin of our moral ideas. These are not the roots or 'origines' of morals, but the latest efforts of reflection, the lights in which the whole moral world has been regarded by different thinkers and successive generations of men. If we ask: Which of these many theories is the true one? we may answer: All of them—moral sense, innate ideas, a priori, a posteriori notions, the philosophy of experience, the philosophy of intuition—all of them have added something to our conception of Ethics; no one of them is the whole truth. But to decide how far our ideas of morality are derived from one source or another; to determine what history, what philosophy has contributed to them; to distinguish the original, simple elements from the manifold and complex applications of them, would be a long enquiry too far removed from the question which we are now pursuing.

\par  Bearing in mind the distinction which we have been seeking to establish between our earliest and our most mature ideas of morality, we may now proceed to state the theory of Utility, not exactly in the words, but in the spirit of one of its ablest and most moderate supporters (Mill's Utilitarianism):—'That which alone makes actions either right or desirable is their utility, or tendency to promote the happiness of mankind, or, in other words, to increase the sum of pleasure in the world. But all pleasures are not the same: they differ in quality as well as in quantity, and the pleasure which is superior in quality is incommensurable with the inferior. Neither is the pleasure or happiness, which we seek, our own pleasure, but that of others,—of our family, of our country, of mankind. The desire of this, and even the sacrifice of our own interest to that of other men, may become a passion to a rightly educated nature. The Utilitarian finds a place in his system for this virtue and for every other.'

\par  Good or happiness or pleasure is thus regarded as the true and only end of human life. To this all our desires will be found to tend, and in accordance with this all the virtues, including justice, may be explained. Admitting that men rest for a time in inferior ends, and do not cast their eyes beyond them, these ends are really dependent on the greater end of happiness, and would not be pursued, unless in general they had been found to lead to it. The existence of such an end is proved, as in Aristotle's time, so in our own, by the universal fact that men desire it. The obligation to promote it is based upon the social nature of man; this sense of duty is shared by all of us in some degree, and is capable of being greatly fostered and strengthened. So far from being inconsistent with religion, the greatest happiness principle is in the highest degree agreeable to it. For what can be more reasonable than that God should will the happiness of all his creatures? and in working out their happiness we may be said to be 'working together with him.' Nor is it inconceivable that a new enthusiasm of the future, far stronger than any old religion, may be based upon such a conception.

\par  But then for the familiar phrase of the 'greatest happiness principle,' it seems as if we ought now to read 'the noblest happiness principle,' 'the happiness of others principle'—the principle not of the greatest, but of the highest pleasure, pursued with no more regard to our own immediate interest than is required by the law of self-preservation. Transfer the thought of happiness to another life, dropping the external circumstances which form so large a part of our idea of happiness in this, and the meaning of the word becomes indistinguishable from holiness, harmony, wisdom, love. By the slight addition 'of others,' all the associations of the word are altered; we seem to have passed over from one theory of morals to the opposite. For allowing that the happiness of others is reflected on ourselves, and also that every man must live before he can do good to others, still the last limitation is a very trifling exception, and the happiness of another is very far from compensating for the loss of our own. According to Mr. Mill, he would best carry out the principle of utility who sacrificed his own pleasure most to that of his fellow-men. But if so, Hobbes and Butler, Shaftesbury and Hume, are not so far apart as they and their followers imagine. The thought of self and the thought of others are alike superseded in the more general notion of the happiness of mankind at large. But in this composite good, until society becomes perfected, the friend of man himself has generally the least share, and may be a great sufferer.

\par  And now what objection have we to urge against a system of moral philosophy so beneficent, so enlightened, so ideal, and at the same time so practical,—so Christian, as we may say without exaggeration,—and which has the further advantage of resting morality on a principle intelligible to all capacities? Have we not found that which Socrates and Plato 'grew old in seeking'? Are we not desirous of happiness, at any rate for ourselves and our friends, if not for all mankind? If, as is natural, we begin by thinking of ourselves first, we are easily led on to think of others; for we cannot help acknowledging that what is right for us is the right and inheritance of others. We feel the advantage of an abstract principle wide enough and strong enough to override all the particularisms of mankind; which acknowledges a universal good, truth, right; which is capable of inspiring men like a passion, and is the symbol of a cause for which they are ready to contend to their life's end.

\par  And if we test this principle by the lives of its professors, it would certainly appear inferior to none as a rule of action. From the days of Eudoxus (Arist. Ethics) and Epicurus to our own, the votaries of pleasure have gained belief for their principles by their practice. Two of the noblest and most disinterested men who have lived in this century, Bentham and J. S. Mill, whose lives were a long devotion to the service of their fellows, have been among the most enthusiastic supporters of utility; while among their contemporaries, some who were of a more mystical turn of mind, have ended rather in aspiration than in action, and have been found unequal to the duties of life. Looking back on them now that they are removed from the scene, we feel that mankind has been the better for them. The world was against them while they lived; but this is rather a reason for admiring than for depreciating them. Nor can any one doubt that the influence of their philosophy on politics—especially on foreign politics, on law, on social life, has been upon the whole beneficial. Nevertheless, they will never have justice done to them, for they do not agree either with the better feeling of the multitude or with the idealism of more refined thinkers. Without Bentham, a great word in the history of philosophy would have remained unspoken. Yet to this day it is rare to hear his name received with any mark of respect such as would be freely granted to the ambiguous memory of some father of the Church. The odium which attached to him when alive has not been removed by his death. For he shocked his contemporaries by egotism and want of taste; and this generation which has reaped the benefit of his labours has inherited the feeling of the last. He was before his own age, and is hardly remembered in this.

\par  While acknowledging the benefits which the greatest happiness principle has conferred upon mankind, the time appears to have arrived, not for denying its claims, but for criticizing them and comparing them with other principles which equally claim to lie at the foundation of ethics. Any one who adds a general principle to knowledge has been a benefactor to the world. But there is a danger that, in his first enthusiasm, he may not recognize the proportions or limitations to which his truth is subjected; he does not see how far he has given birth to a truism, or how that which is a truth to him is a truism to the rest of the world; or may degenerate in the next generation. He believes that to be the whole which is only a part,—to be the necessary foundation which is really only a valuable aspect of the truth. The systems of all philosophers require the criticism of 'the morrow,' when the heat of imagination which forged them has cooled, and they are seen in the temperate light of day. All of them have contributed to enrich the mind of the civilized world; none of them occupy that supreme or exclusive place which their authors would have assigned to them.

\par  We may preface the criticism with a few preliminary remarks:—

\par  Mr. Mill, Mr. Austin, and others, in their eagerness to maintain the doctrine of utility, are fond of repeating that we are in a lamentable state of uncertainty about morals. While other branches of knowledge have made extraordinary progress, in moral philosophy we are supposed by them to be no better than children, and with few exceptions—that is to say, Bentham and his followers—to be no further advanced than men were in the age of Socrates and Plato, who, in their turn, are deemed to be as backward in ethics as they necessarily were in physics. But this, though often asserted, is recanted almost in a breath by the same writers who speak thus depreciatingly of our modern ethical philosophy. For they are the first to acknowledge that we have not now to begin classifying actions under the head of utility; they would not deny that about the general conceptions of morals there is a practical agreement. There is no more doubt that falsehood is wrong than that a stone falls to the ground, although the first does not admit of the same ocular proof as the second. There is no greater uncertainty about the duty of obedience to parents and to the law of the land than about the properties of triangles. Unless we are looking for a new moral world which has no marrying and giving in marriage, there is no greater disagreement in theory about the right relations of the sexes than about the composition of water. These and a few other simple principles, as they have endless applications in practice, so also may be developed in theory into counsels of perfection.

\par  To what then is to be attributed this opinion which has been often entertained about the uncertainty of morals? Chiefly to this,—that philosophers have not always distinguished the theoretical and the casuistical uncertainty of morals from the practical certainty. There is an uncertainty about details,—whether, for example, under given circumstances such and such a moral principle is to be enforced, or whether in some cases there may not be a conflict of duties: these are the exceptions to the ordinary rules of morality, important, indeed, but not extending to the one thousandth or one ten-thousandth part of human actions. This is the domain of casuistry. Secondly, the aspects under which the most general principles of morals may be presented to us are many and various. The mind of man has been more than usually active in thinking about man. The conceptions of harmony, happiness, right, freedom, benevolence, self-love, have all of them seemed to some philosopher or other the truest and most comprehensive expression of morality. There is no difference, or at any rate no great difference, of opinion about the right and wrong of actions, but only about the general notion which furnishes the best explanation or gives the most comprehensive view of them. This, in the language of Kant, is the sphere of the metaphysic of ethics. But these two uncertainties at either end, en tois malista katholou and en tois kath ekasta, leave space enough for an intermediate principle which is practically certain.

\par  The rule of human life is not dependent on the theories of philosophers: we know what our duties are for the most part before we speculate about them. And the use of speculation is not to teach us what we already know, but to inspire in our minds an interest about morals in general, to strengthen our conception of the virtues by showing that they confirm one another, to prove to us, as Socrates would have said, that they are not many, but one. There is the same kind of pleasure and use in reducing morals, as in reducing physics, to a few very simple truths. And not unfrequently the more general principle may correct prejudices and misconceptions, and enable us to regard our fellow-men in a larger and more generous spirit.

\par  The two qualities which seem to be most required in first principles of ethics are, (1) that they should afford a real explanation of the facts, (2) that they should inspire the mind,—should harmonize, strengthen, settle us. We can hardly estimate the influence which a simple principle such as 'Act so as to promote the happiness of mankind,' or 'Act so that the rule on which thou actest may be adopted as a law by all rational beings,' may exercise on the mind of an individual. They will often seem to open a new world to him, like the religious conceptions of faith or the spirit of God. The difficulties of ethics disappear when we do not suffer ourselves to be distracted between different points of view. But to maintain their hold on us, the general principles must also be psychologically true—they must agree with our experience, they must accord with the habits of our minds.

\par  When we are told that actions are right or wrong only in so far as they tend towards happiness, we naturally ask what is meant by 'happiness.' For the term in the common use of language is only to a certain extent commensurate with moral good and evil. We should hardly say that a good man could be utterly miserable (Arist. Ethics), or place a bad man in the first rank of happiness. But yet, from various circumstances, the measure of a man's happiness may be out of all proportion to his desert. And if we insist on calling the good man alone happy, we shall be using the term in some new and transcendental sense, as synonymous with well-being. We have already seen that happiness includes the happiness of others as well as our own; we must now comprehend unconscious as well as conscious happiness under the same word. There is no harm in this extension of the meaning, but a word which admits of such an extension can hardly be made the basis of a philosophical system. The exactness which is required in philosophy will not allow us to comprehend under the same term two ideas so different as the subjective feeling of pleasure or happiness and the objective reality of a state which receives our moral approval.

\par  Like Protarchus in the Philebus, we can give no answer to the question, 'What is that common quality which in all states of human life we call happiness? which includes the lower and the higher kind of happiness, and is the aim of the noblest, as well as of the meanest of mankind?' If we say 'Not pleasure, not virtue, not wisdom, nor yet any quality which we can abstract from these'—what then? After seeming to hover for a time on the verge of a great truth, we have gained only a truism.

\par  Let us ask the question in another form. What is that which constitutes happiness, over and above the several ingredients of health, wealth, pleasure, virtue, knowledge, which are included under it? Perhaps we answer, 'The subjective feeling of them.' But this is very far from being coextensive with right. Or we may reply that happiness is the whole of which the above-mentioned are the parts. Still the question recurs, 'In what does the whole differ from all the parts?' And if we are unable to distinguish them, happiness will be the mere aggregate of the goods of life.

\par  Again, while admitting that in all right action there is an element of happiness, we cannot help seeing that the utilitarian theory supplies a much easier explanation of some virtues than of others. Of many patriotic or benevolent actions we can give a straightforward account by their tendency to promote happiness. For the explanation of justice, on the other hand, we have to go a long way round. No man is indignant with a thief because he has not promoted the greatest happiness of the greatest number, but because he has done him a wrong. There is an immeasurable interval between a crime against property or life, and the omission of an act of charity or benevolence. Yet of this interval the utilitarian theory takes no cognizance. The greatest happiness principle strengthens our sense of positive duties towards others, but weakens our recognition of their rights. To promote in every way possible the happiness of others may be a counsel of perfection, but hardly seems to offer any ground for a theory of obligation. For admitting that our ideas of obligation are partly derived from religion and custom, yet they seem also to contain other essential elements which cannot be explained by the tendency of actions to promote happiness. Whence comes the necessity of them? Why are some actions rather than others which equally tend to the happiness of mankind imposed upon us with the authority of law? 'You ought' and 'you had better' are fundamental distinctions in human thought; and having such distinctions, why should we seek to efface and unsettle them?

\par  Bentham and Mr. Mill are earnest in maintaining that happiness includes the happiness of others as well as of ourselves. But what two notions can be more opposed in many cases than these? Granting that in a perfect state of the world my own happiness and that of all other men would coincide, in the imperfect state they often diverge, and I cannot truly bridge over the difficulty by saying that men will always find pleasure in sacrificing themselves or in suffering for others. Upon the greatest happiness principle it is admitted that I am to have a share, and in consistency I should pursue my own happiness as impartially as that of my neighbour. But who can decide what proportion should be mine and what his, except on the principle that I am most likely to be deceived in my own favour, and had therefore better give the larger share, if not all, to him?

\par  Further, it is admitted that utility and right coincide, not in particular instances, but in classes of actions. But is it not distracting to the conscience of a man to be told that in the particular case they are opposed? Happiness is said to be the ground of moral obligation, yet he must not do what clearly conduces to his own happiness if it is at variance with the good of the whole. Nay, further, he will be taught that when utility and right are in apparent conflict any amount of utility does not alter by a hair's-breadth the morality of actions, which cannot be allowed to deviate from established law or usage; and that the non-detection of an immoral act, say of telling a lie, which may often make the greatest difference in the consequences, not only to himself, but to all the world, makes none whatever in the act itself.

\par  Again, if we are concerned not with particular actions but with classes of actions, is the tendency of actions to happiness a principle upon which we can classify them? There is a universal law which imperatively declares certain acts to be right or wrong:—can there be any universality in the law which measures actions by their tendencies towards happiness? For an act which is the cause of happiness to one person may be the cause of unhappiness to another; or an act which if performed by one person may increase the happiness of mankind may have the opposite effect if performed by another. Right can never be wrong, or wrong right, that there are no actions which tend to the happiness of mankind which may not under other circumstances tend to their unhappiness. Unless we say not only that all right actions tend to happiness, but that they tend to happiness in the same degree in which they are right (and in that case the word 'right' is plainer), we weaken the absoluteness of our moral standard; we reduce differences in kind to differences in degree; we obliterate the stamp which the authority of ages has set upon vice and crime.

\par  Once more: turning from theory to practice we feel the importance of retaining the received distinctions of morality. Words such as truth, justice, honesty, virtue, love, have a simple meaning; they have become sacred to us,—'the word of God' written on the human heart: to no other words can the same associations be attached. We cannot explain them adequately on principles of utility; in attempting to do so we rob them of their true character. We give them a meaning often paradoxical and distorted, and generally weaker than their signification in common language. And as words influence men's thoughts, we fear that the hold of morality may also be weakened, and the sense of duty impaired, if virtue and vice are explained only as the qualities which do or do not contribute to the pleasure of the world. In that very expression we seem to detect a false ring, for pleasure is individual not universal; we speak of eternal and immutable justice, but not of eternal and immutable pleasure; nor by any refinement can we avoid some taint of bodily sense adhering to the meaning of the word.

\par  Again: the higher the view which men take of life, the more they lose sight of their own pleasure or interest. True religion is not working for a reward only, but is ready to work equally without a reward. It is not 'doing the will of God for the sake of eternal happiness,' but doing the will of God because it is best, whether rewarded or unrewarded. And this applies to others as well as to ourselves. For he who sacrifices himself for the good of others, does not sacrifice himself that they may be saved from the persecution which he endures for their sakes, but rather that they in their turn may be able to undergo similar sufferings, and like him stand fast in the truth. To promote their happiness is not his first object, but to elevate their moral nature. Both in his own case and that of others there may be happiness in the distance, but if there were no happiness he would equally act as he does. We are speaking of the highest and noblest natures; and a passing thought naturally arises in our minds, 'Whether that can be the first principle of morals which is hardly regarded in their own case by the greatest benefactors of mankind?'

\par  The admissions that pleasures differ in kind, and that actions are already classified; the acknowledgment that happiness includes the happiness of others, as well as of ourselves; the confusion (not made by Aristotle) between conscious and unconscious happiness, or between happiness the energy and happiness the result of the energy, introduce uncertainty and inconsistency into the whole enquiry. We reason readily and cheerfully from a greatest happiness principle. But we find that utilitarians do not agree among themselves about the meaning of the word. Still less can they impart to others a common conception or conviction of the nature of happiness. The meaning of the word is always insensibly slipping away from us, into pleasure, out of pleasure, now appearing as the motive, now as the test of actions, and sometimes varying in successive sentences. And as in a mathematical demonstration an error in the original number disturbs the whole calculation which follows, this fundamental uncertainty about the word vitiates all the applications of it. Must we not admit that a notion so uncertain in meaning, so void of content, so at variance with common language and opinion, does not comply adequately with either of our two requirements? It can neither strike the imaginative faculty, nor give an explanation of phenomena which is in accordance with our individual experience. It is indefinite; it supplies only a partial account of human actions: it is one among many theories of philosophers. It may be compared with other notions, such as the chief good of Plato, which may be best expressed to us under the form of a harmony, or with Kant's obedience to law, which may be summed up under the word 'duty,' or with the Stoical 'Follow nature,' and seems to have no advantage over them. All of these present a certain aspect of moral truth. None of them are, or indeed profess to be, the only principle of morals.

\par  And this brings us to speak of the most serious objection to the utilitarian system—its exclusiveness. There is no place for Kant or Hegel, for Plato and Aristotle alongside of it. They do not reject the greatest happiness principle, but it rejects them. Now the phenomena of moral action differ, and some are best explained upon one principle and some upon another: the virtue of justice seems to be naturally connected with one theory of morals, the virtues of temperance and benevolence with another. The characters of men also differ; and some are more attracted by one aspect of the truth, some by another. The firm stoical nature will conceive virtue under the conception of law, the philanthropist under that of doing good, the quietist under that of resignation, the enthusiast under that of faith or love. The upright man of the world will desire above all things that morality should be plain and fixed, and should use language in its ordinary sense. Persons of an imaginative temperament will generally be dissatisfied with the words 'utility' or 'pleasure': their principle of right is of a far higher character—what or where to be found they cannot always distinctly tell;—deduced from the laws of human nature, says one; resting on the will of God, says another; based upon some transcendental idea which animates more worlds than one, says a third:
 
\par  To satisfy an imaginative nature in any degree, the doctrine of utility must be so transfigured that it becomes altogether different and loses all simplicity.

\par  But why, since there are different characters among men, should we not allow them to envisage morality accordingly, and be thankful to the great men who have provided for all of us modes and instruments of thought? Would the world have been better if there had been no Stoics or Kantists, no Platonists or Cartesians? No more than if the other pole of moral philosophy had been excluded. All men have principles which are above their practice; they admit premises which, if carried to their conclusions, are a sufficient basis of morals. In asserting liberty of speculation we are not encouraging individuals to make right or wrong for themselves, but only conceding that they may choose the form under which they prefer to contemplate them. Nor do we say that one of these aspects is as true and good as another; but that they all of them, if they are not mere sophisms and illusions, define and bring into relief some part of the truth which would have been obscure without their light. Why should we endeavour to bind all men within the limits of a single metaphysical conception? The necessary imperfection of language seems to require that we should view the same truth under more than one aspect.

\par  We are living in the second age of utilitarianism, when the charm of novelty and the fervour of the first disciples has passed away. The doctrine is no longer stated in the forcible paradoxical manner of Bentham, but has to be adapted to meet objections; its corners are rubbed off, and the meaning of its most characteristic expressions is softened. The array of the enemy melts away when we approach him. The greatest happiness of the greatest number was a great original idea when enunciated by Bentham, which leavened a generation and has left its mark on thought and civilization in all succeeding times. His grasp of it had the intensity of genius. In the spirit of an ancient philosopher he would have denied that pleasures differed in kind, or that by happiness he meant anything but pleasure. He would perhaps have revolted us by his thoroughness. The 'guardianship of his doctrine' has passed into other hands; and now we seem to see its weak points, its ambiguities, its want of exactness while assuming the highest exactness, its one-sidedness, its paradoxical explanation of several of the virtues. No philosophy has ever stood this criticism of the next generation, though the founders of all of them have imagined that they were built upon a rock. And the utilitarian system, like others, has yielded to the inevitable analysis. Even in the opinion of 'her admirers she has been terribly damaged' (Phil. ), and is no longer the only moral philosophy, but one among many which have contributed in various degrees to the intellectual progress of mankind.

\par  But because the utilitarian philosophy can no longer claim 'the prize,' we must not refuse to acknowledge the great benefits conferred by it on the world. All philosophies are refuted in their turn, says the sceptic, and he looks forward to all future systems sharing the fate of the past. All philosophies remain, says the thinker; they have done a great work in their own day, and they supply posterity with aspects of the truth and with instruments of thought. Though they may be shorn of their glory, they retain their place in the organism of knowledge.

\par  And still there remain many rules of morals which are better explained and more forcibly inculcated on the principle of utility than on any other. The question Will such and such an action promote the happiness of myself, my family, my country, the world? may check the rising feeling of pride or honour which would cause a quarrel, an estrangement, a war. 'How can I contribute to the greatest happiness of others?' is another form of the question which will be more attractive to the minds of many than a deduction of the duty of benevolence from a priori principles. In politics especially hardly any other argument can be allowed to have weight except the happiness of a people. All parties alike profess to aim at this, which though often used only as the disguise of self-interest has a great and real influence on the minds of statesmen. In religion, again, nothing can more tend to mitigate superstition than the belief that the good of man is also the will of God. This is an easy test to which the prejudices and superstitions of men may be brought:—whatever does not tend to the good of men is not of God. And the ideal of the greatest happiness of mankind, especially if believed to be the will of God, when compared with the actual fact, will be one of the strongest motives to do good to others.

\par  On the other hand, when the temptation is to speak falsely, to be dishonest or unjust, or in any way to interfere with the rights of others, the argument that these actions regarded as a class will not conduce to the happiness of mankind, though true enough, seems to have less force than the feeling which is already implanted in the mind by conscience and authority. To resolve this feeling into the greatest happiness principle takes away from its sacred and authoritative character. The martyr will not go to the stake in order that he may promote the happiness of mankind, but for the sake of the truth: neither will the soldier advance to the cannon's mouth merely because he believes military discipline to be for the good of mankind. It is better for him to know that he will be shot, that he will be disgraced, if he runs away—he has no need to look beyond military honour, patriotism, 'England expects every man to do his duty.' These are stronger motives than the greatest happiness of the greatest number, which is the thesis of a philosopher, not the watchword of an army. For in human actions men do not always require broad principles; duties often come home to us more when they are limited and defined, and sanctioned by custom and public opinion.

\par  Lastly, if we turn to the history of ethics, we shall find that our moral ideas have originated not in utility but in religion, in law, in conceptions of nature, of an ideal good, and the like. And many may be inclined to think that this conclusively disproves the claim of utility to be the basis of morals. But the utilitarian will fairly reply (see above) that we must distinguish the origin of ethics from the principles of them—the historical germ from the later growth of reflection. And he may also truly add that for two thousand years and more, utility, if not the originating, has been the great corrective principle in law, in politics, in religion, leading men to ask how evil may be diminished and good increased—by what course of policy the public interest may be promoted, and to understand that God wills the happiness, not of some of his creatures and in this world only, but of all of them and in every stage of their existence.

\par  'What is the place of happiness or utility in a system of moral philosophy?' is analogous to the question asked in the Philebus, 'What rank does pleasure hold in the scale of goods?' Admitting the greatest happiness principle to be true and valuable, and the necessary foundation of that part of morals which relates to the consequences of actions, we still have to consider whether this or some other general notion is the highest principle of human life. We may try them in this comparison by three tests—definiteness, comprehensiveness, and motive power.

\par  There are three subjective principles of morals,—sympathy, benevolence, self-love. But sympathy seems to rest morality on feelings which differ widely even in good men; benevolence and self-love torture one half of our virtuous actions into the likeness of the other. The greatest happiness principle, which includes both, has the advantage over all these in comprehensiveness, but the advantage is purchased at the expense of definiteness.

\par  Again, there are the legal and political principles of morals—freedom, equality, rights of persons; 'Every man to count for one and no man for more than one,' 'Every man equal in the eye of the law and of the legislator.' There is also the other sort of political morality, which if not beginning with 'Might is right,' at any rate seeks to deduce our ideas of justice from the necessities of the state and of society. According to this view the greatest good of men is obedience to law: the best human government is a rational despotism, and the best idea which we can form of a divine being is that of a despot acting not wholly without regard to law and order. To such a view the present mixed state of the world, not wholly evil or wholly good, is supposed to be a witness. More we might desire to have, but are not permitted. Though a human tyrant would be intolerable, a divine tyrant is a very tolerable governor of the universe. This is the doctrine of Thrasymachus adapted to the public opinion of modern times.

\par  There is yet a third view which combines the two:—freedom is obedience to the law, and the greatest order is also the greatest freedom; 'Act so that thy action may be the law of every intelligent being.' This view is noble and elevating; but it seems to err, like other transcendental principles of ethics, in being too abstract. For there is the same difficulty in connecting the idea of duty with particular duties as in bridging the gulf between phainomena and onta; and when, as in the system of Kant, this universal idea or law is held to be independent of space and time, such a mataion eidos becomes almost unmeaning.

\par  Once more there are the religious principles of morals:—the will of God revealed in Scripture and in nature. No philosophy has supplied a sanction equal in authority to this, or a motive equal in strength to the belief in another life. Yet about these too we must ask What will of God? how revealed to us, and by what proofs? Religion, like happiness, is a word which has great influence apart from any consideration of its content: it may be for great good or for great evil. But true religion is the synthesis of religion and morality, beginning with divine perfection in which all human perfection is embodied. It moves among ideas of holiness, justice, love, wisdom, truth; these are to God, in whom they are personified, what the Platonic ideas are to the idea of good. It is the consciousness of the will of God that all men should be as he is. It lives in this world and is known to us only through the phenomena of this world, but it extends to worlds beyond. Ordinary religion which is alloyed with motives of this world may easily be in excess, may be fanatical, may be interested, may be the mask of ambition, may be perverted in a thousand ways. But of that religion which combines the will of God with our highest ideas of truth and right there can never be too much. This impossibility of excess is the note of divine moderation.

\par  So then, having briefly passed in review the various principles of moral philosophy, we may now arrange our goods in order, though, like the reader of the Philebus, we have a difficulty in distinguishing the different aspects of them from one another, or defining the point at which the human passes into the divine.

\par  First, the eternal will of God in this world and in another,—justice, holiness, wisdom, love, without succession of acts (ouch e genesis prosestin), which is known to us in part only, and reverenced by us as divine perfection.

\par  Secondly, human perfection, or the fulfilment of the will of God in this world, and co-operation with his laws revealed to us by reason and experience, in nature, history, and in our own minds.

\par  Thirdly, the elements of human perfection,—virtue, knowledge, and right opinion.

\par  Fourthly, the external conditions of perfection,—health and the goods of life.

\par  Fifthly, beauty and happiness,—the inward enjoyment of that which is best and fairest in this world and in the human soul.

\par  ...

\par  The Philebus is probably the latest in time of the writings of Plato with the exception of the Laws. We have in it therefore the last development of his philosophy. The extreme and one-sided doctrines of the Cynics and Cyrenaics are included in a larger whole; the relations of pleasure and knowledge to each other and to the good are authoritatively determined; the Eleatic Being and the Heraclitean Flux no longer divide the empire of thought; the Mind of Anaxagoras has become the Mind of God and of the World. The great distinction between pure and applied science for the first time has a place in philosophy; the natural claim of dialectic to be the Queen of the Sciences is once more affirmed. This latter is the bond of union which pervades the whole or nearly the whole of the Platonic writings. And here as in several other dialogues (Phaedrus, Republic, etc.) it is presented to us in a manner playful yet also serious, and sometimes as if the thought of it were too great for human utterance and came down from heaven direct. It is the organization of knowledge wonderful to think of at a time when knowledge itself could hardly be said to exist. It is this more than any other element which distinguishes Plato, not only from the presocratic philosophers, but from Socrates himself.

\par  We have not yet reached the confines of Aristotle, but we make a somewhat nearer approach to him in the Philebus than in the earlier Platonic writings. The germs of logic are beginning to appear, but they are not collected into a whole, or made a separate science or system. Many thinkers of many different schools have to be interposed between the Parmenides or Philebus of Plato, and the Physics or Metaphysics of Aristotle. It is this interval upon which we have to fix our minds if we would rightly understand the character of the transition from one to the other. Plato and Aristotle do not dovetail into one another; nor does the one begin where the other ends; there is a gulf between them not to be measured by time, which in the fragmentary state of our knowledge it is impossible to bridge over. It follows that the one cannot be interpreted by the other. At any rate, it is not Plato who is to be interpreted by Aristotle, but Aristotle by Plato. Of all philosophy and of all art the true understanding is to be sought not in the afterthoughts of posterity, but in the elements out of which they have arisen. For the previous stage is a tendency towards the ideal at which they are aiming; the later is a declination or deviation from them, or even a perversion of them. No man's thoughts were ever so well expressed by his disciples as by himself.

\par  But although Plato in the Philebus does not come into any close connexion with Aristotle, he is now a long way from himself and from the beginnings of his own philosophy. At the time of his death he left his system still incomplete; or he may be more truly said to have had no system, but to have lived in the successive stages or moments of metaphysical thought which presented themselves from time to time. The earlier discussions about universal ideas and definitions seem to have died away; the correlation of ideas has taken their place. The flowers of rhetoric and poetry have lost their freshness and charm; and a technical language has begun to supersede and overgrow them. But the power of thinking tends to increase with age, and the experience of life to widen and deepen. The good is summed up under categories which are not summa genera, but heads or gradations of thought. The question of pleasure and the relation of bodily pleasures to mental, which is hardly treated of elsewhere in Plato, is here analysed with great subtlety. The mean or measure is now made the first principle of good. Some of these questions reappear in Aristotle, as does also the distinction between metaphysics and mathematics. But there are many things in Plato which have been lost in Aristotle; and many things in Aristotle not to be found in Plato. The most remarkable deficiency in Aristotle is the disappearance of the Platonic dialectic, which in the Aristotelian school is only used in a comparatively unimportant and trivial sense. The most remarkable additions are the invention of the Syllogism, the conception of happiness as the foundation of morals, the reference of human actions to the standard of the better mind of the world, or of the one 'sensible man' or 'superior person.' His conception of ousia, or essence, is not an advance upon Plato, but a return to the poor and meagre abstractions of the Eleatic philosophy. The dry attempt to reduce the presocratic philosophy by his own rather arbitrary standard of the four causes, contrasts unfavourably with Plato's general discussion of the same subject (Sophist). To attempt further to sum up the differences between the two great philosophers would be out of place here. Any real discussion of their relation to one another must be preceded by an examination into the nature and character of the Aristotelian writings and the form in which they have come down to us. This enquiry is not really separable from an investigation of Theophrastus as well as Aristotle and of the remains of other schools of philosophy as well as of the Peripatetics. But, without entering on this wide field, even a superficial consideration of the logical and metaphysical works which pass under the name of Aristotle, whether we suppose them to have come directly from his hand or to be the tradition of his school, is sufficient to show how great was the mental activity which prevailed in the latter half of the fourth century B.C. ; what eddies and whirlpools of controversies were surging in the chaos of thought, what transformations of the old philosophies were taking place everywhere, what eclecticisms and syncretisms and realisms and nominalisms were affecting the mind of Hellas. The decline of philosophy during this period is no less remarkable than the loss of freedom; and the two are not unconnected with each other. But of the multitudinous sea of opinions which were current in the age of Aristotle we have no exact account. We know of them from allusions only. And we cannot with advantage fill up the void of our knowledge by conjecture: we can only make allowance for our ignorance.

\par  There are several passages in the Philebus which are very characteristic of Plato, and which we shall do well to consider not only in their connexion, but apart from their connexion as inspired sayings or oracles which receive their full interpretation only from the history of philosophy in later ages. The more serious attacks on traditional beliefs which are often veiled under an unusual simplicity or irony are of this kind. Such, for example, is the excessive and more than human awe which Socrates expresses about the names of the gods, which may be not unaptly compared with the importance attached by mankind to theological terms in other ages; for this also may be comprehended under the satire of Socrates. Let us observe the religious and intellectual enthusiasm which shines forth in the following, 'The power and faculty of loving the truth, and of doing all things for the sake of the truth': or, again, the singular acknowledgment which may be regarded as the anticipation of a new logic, that 'In going to war for mind I must have weapons of a different make from those which I used before, although some of the old ones may do again.' Let us pause awhile to reflect on a sentence which is full of meaning to reformers of religion or to the original thinker of all ages: 'Shall we then agree with them of old time, and merely reassert the notions of others without risk to ourselves; or shall we venture also to share in the risk and bear the reproach which will await us': i.e. if we assert mind to be the author of nature. Let us note the remarkable words, 'That in the divine nature of Zeus there is the soul and mind of a King, because there is in him the power of the cause,' a saying in which theology and philosophy are blended and reconciled; not omitting to observe the deep insight into human nature which is shown by the repetition of the same thought 'All philosophers are agreed that mind is the king of heaven and earth' with the ironical addition, 'in this way truly they magnify themselves.' Nor let us pass unheeded the indignation felt by the generous youth at the 'blasphemy' of those who say that Chaos and Chance Medley created the world; or the significance of the words 'those who said of old time that mind rules the universe'; or the pregnant observation that 'we are not always conscious of what we are doing or of what happens to us,' a chance expression to which if philosophers had attended they would have escaped many errors in psychology. We may contrast the contempt which is poured upon the verbal difficulty of the one and many, and the seriousness with the unity of opposites is regarded from the higher point of view of abstract ideas: or compare the simple manner in which the question of cause and effect and their mutual dependence is regarded by Plato (to which modern science has returned in Mill and Bacon), and the cumbrous fourfold division of causes in the Physics and Metaphysics of Aristotle, for which it has puzzled the world to find a use in so many centuries. When we consider the backwardness of knowledge in the age of Plato, the boldness with which he looks forward into the distance, the many questions of modern philosophy which are anticipated in his writings, may we not truly describe him in his own words as a 'spectator of all time and of all existence'?

\par 
\section{
      PHILEBUS
    } 
\par \textbf{SOCRATES}
\par   Observe, Protarchus, the nature of the position which you are now going to take from Philebus, and what the other position is which I maintain, and which, if you do not approve of it, is to be controverted by you. Shall you and I sum up the two sides?

\par \textbf{PROTARCHUS}
\par   By all means.

\par \textbf{SOCRATES}
\par   Philebus was saying that enjoyment and pleasure and delight, and the class of feelings akin to them, are a good to every living being, whereas I contend, that not these, but wisdom and intelligence and memory, and their kindred, right opinion and true reasoning, are better and more desirable than pleasure for all who are able to partake of them, and that to all such who are or ever will be they are the most advantageous of all things. Have I not given, Philebus, a fair statement of the two sides of the argument?

\par \textbf{PHILEBUS}
\par   Nothing could be fairer, Socrates.

\par \textbf{SOCRATES}
\par   And do you, Protarchus, accept the position which is assigned to you?

\par \textbf{PROTARCHUS}
\par   I cannot do otherwise, since our excellent Philebus has left the field.

\par \textbf{SOCRATES}
\par   Surely the truth about these matters ought, by all means, to be ascertained.

\par \textbf{PROTARCHUS}
\par   Certainly.

\par \textbf{SOCRATES}
\par   Shall we further agree—

\par \textbf{PROTARCHUS}
\par   To what?

\par \textbf{SOCRATES}
\par   That you and I must now try to indicate some state and disposition of the soul, which has the property of making all men happy.

\par \textbf{PROTARCHUS}
\par   Yes, by all means.

\par \textbf{SOCRATES}
\par   And you say that pleasure, and I say that wisdom, is such a state?

\par \textbf{PROTARCHUS}
\par   True.

\par \textbf{SOCRATES}
\par   And what if there be a third state, which is better than either? Then both of us are vanquished—are we not? But if this life, which really has the power of making men happy, turn out to be more akin to pleasure than to wisdom, the life of pleasure may still have the advantage over the life of wisdom.

\par \textbf{PROTARCHUS}
\par   True.

\par \textbf{SOCRATES}
\par   Or suppose that the better life is more nearly allied to wisdom, then wisdom conquers, and pleasure is defeated;—do you agree?

\par \textbf{PROTARCHUS}
\par   Certainly.

\par \textbf{SOCRATES}
\par   And what do you say, Philebus?

\par \textbf{PHILEBUS}
\par   I say, and shall always say, that pleasure is easily the conqueror; but you must decide for yourself, Protarchus.

\par \textbf{PROTARCHUS}
\par   You, Philebus, have handed over the argument to me, and have no longer a voice in the matter?

\par \textbf{PHILEBUS}
\par   True enough. Nevertheless I would clear myself and deliver my soul of you; and I call the goddess herself to witness that I now do so.

\par \textbf{PROTARCHUS}
\par   You may appeal to us; we too will be the witnesses of your words. And now, Socrates, whether Philebus is pleased or displeased, we will proceed with the argument.

\par \textbf{SOCRATES}
\par   Then let us begin with the goddess herself, of whom Philebus says that she is called Aphrodite, but that her real name is Pleasure.

\par \textbf{PROTARCHUS}
\par   Very good.

\par \textbf{SOCRATES}
\par   The awe which I always feel, Protarchus, about the names of the gods is more than human—it exceeds all other fears. And now I would not sin against Aphrodite by naming her amiss; let her be called what she pleases. But Pleasure I know to be manifold, and with her, as I was just now saying, we must begin, and consider what her nature is. She has one name, and therefore you would imagine that she is one; and yet surely she takes the most varied and even unlike forms. For do we not say that the intemperate has pleasure, and that the temperate has pleasure in his very temperance,—that the fool is pleased when he is full of foolish fancies and hopes, and that the wise man has pleasure in his wisdom? and how foolish would any one be who affirmed that all these opposite pleasures are severally alike!

\par \textbf{PROTARCHUS}
\par   Why, Socrates, they are opposed in so far as they spring from opposite sources, but they are not in themselves opposite. For must not pleasure be of all things most absolutely like pleasure,—that is, like itself?

\par \textbf{SOCRATES}
\par   Yes, my good friend, just as colour is like colour;—in so far as colours are colours, there is no difference between them; and yet we all know that black is not only unlike, but even absolutely opposed to white:  or again, as figure is like figure, for all figures are comprehended under one class; and yet particular figures may be absolutely opposed to one another, and there is an infinite diversity of them. And we might find similar examples in many other things; therefore do not rely upon this argument, which would go to prove the unity of the most extreme opposites. And I suspect that we shall find a similar opposition among pleasures.

\par \textbf{PROTARCHUS}
\par   Very likely; but how will this invalidate the argument?

\par \textbf{SOCRATES}
\par   Why, I shall reply, that dissimilar as they are, you apply to them a new predicate, for you say that all pleasant things are good; now although no one can argue that pleasure is not pleasure, he may argue, as we are doing, that pleasures are oftener bad than good; but you call them all good, and at the same time are compelled, if you are pressed, to acknowledge that they are unlike. And so you must tell us what is the identical quality existing alike in good and bad pleasures, which makes you designate all of them as good.

\par \textbf{PROTARCHUS}
\par   What do you mean, Socrates? Do you think that any one who asserts pleasure to be the good, will tolerate the notion that some pleasures are good and others bad?

\par \textbf{SOCRATES}
\par   And yet you will acknowledge that they are different from one another, and sometimes opposed?

\par \textbf{PROTARCHUS}
\par   Not in so far as they are pleasures.

\par \textbf{SOCRATES}
\par   That is a return to the old position, Protarchus, and so we are to say (are we?) that there is no difference in pleasures, but that they are all alike; and the examples which have just been cited do not pierce our dull minds, but we go on arguing all the same, like the weakest and most inexperienced reasoners? (Probably corrupt.)

\par \textbf{PROTARCHUS}
\par   What do you mean?

\par \textbf{SOCRATES}
\par   Why, I mean to say, that in self-defence I may, if I like, follow your example, and assert boldly that the two things most unlike are most absolutely alike; and the result will be that you and I will prove ourselves to be very tyros in the art of disputing; and the argument will be blown away and lost. Suppose that we put back, and return to the old position; then perhaps we may come to an understanding with one another.

\par \textbf{PROTARCHUS}
\par   How do you mean?

\par \textbf{SOCRATES}
\par   Shall I, Protarchus, have my own question asked of me by you?

\par \textbf{PROTARCHUS}
\par   What question?

\par \textbf{SOCRATES}
\par   Ask me whether wisdom and science and mind, and those other qualities which I, when asked by you at first what is the nature of the good, affirmed to be good, are not in the same case with the pleasures of which you spoke.

\par \textbf{PROTARCHUS}
\par   What do you mean?

\par \textbf{SOCRATES}
\par   The sciences are a numerous class, and will be found to present great differences. But even admitting that, like the pleasures, they are opposite as well as different, should I be worthy of the name of dialectician if, in order to avoid this difficulty, I were to say (as you are saying of pleasure) that there is no difference between one science and another;—would not the argument founder and disappear like an idle tale, although we might ourselves escape drowning by clinging to a fallacy?

\par \textbf{PROTARCHUS}
\par   May none of this befal us, except the deliverance! Yet I like the even-handed justice which is applied to both our arguments. Let us assume, then, that there are many and diverse pleasures, and many and different sciences.

\par \textbf{SOCRATES}
\par   And let us have no concealment, Protarchus, of the differences between my good and yours; but let us bring them to the light in the hope that, in the process of testing them, they may show whether pleasure is to be called the good, or wisdom, or some third quality; for surely we are not now simply contending in order that my view or that yours may prevail, but I presume that we ought both of us to be fighting for the truth.

\par \textbf{PROTARCHUS}
\par   Certainly we ought.

\par \textbf{SOCRATES}
\par   Then let us have a more definite understanding and establish the principle on which the argument rests.

\par \textbf{PROTARCHUS}
\par   What principle?

\par \textbf{SOCRATES}
\par   A principle about which all men are always in a difficulty, and some men sometimes against their will.

\par \textbf{PROTARCHUS}
\par   Speak plainer.

\par \textbf{SOCRATES}
\par   The principle which has just turned up, which is a marvel of nature; for that one should be many or many one, are wonderful propositions; and he who affirms either is very open to attack.

\par \textbf{PROTARCHUS}
\par   Do you mean, when a person says that I, Protarchus, am by nature one and also many, dividing the single 'me' into many 'me's,' and even opposing them as great and small, light and heavy, and in ten thousand other ways?

\par \textbf{SOCRATES}
\par   Those, Protarchus, are the common and acknowledged paradoxes about the one and many, which I may say that everybody has by this time agreed to dismiss as childish and obvious and detrimental to the true course of thought; and no more favour is shown to that other puzzle, in which a person proves the members and parts of anything to be divided, and then confessing that they are all one, says laughingly in disproof of his own words:  Why, here is a miracle, the one is many and infinite, and the many are only one.

\par \textbf{PROTARCHUS}
\par   But what, Socrates, are those other marvels connected with this subject which, as you imply, have not yet become common and acknowledged?

\par \textbf{SOCRATES}
\par   When, my boy, the one does not belong to the class of things that are born and perish, as in the instances which we were giving, for in those cases, and when unity is of this concrete nature, there is, as I was saying, a universal consent that no refutation is needed; but when the assertion is made that man is one, or ox is one, or beauty one, or the good one, then the interest which attaches to these and similar unities and the attempt which is made to divide them gives birth to a controversy.

\par \textbf{PROTARCHUS}
\par   Of what nature?

\par \textbf{SOCRATES}
\par   In the first place, as to whether these unities have a real existence; and then how each individual unity, being always the same, and incapable either of generation or of destruction, but retaining a permanent individuality, can be conceived either as dispersed and multiplied in the infinity of the world of generation, or as still entire and yet divided from itself, which latter would seem to be the greatest impossibility of all, for how can one and the same thing be at the same time in one and in many things? These, Protarchus, are the real difficulties, and this is the one and many to which they relate; they are the source of great perplexity if ill decided, and the right determination of them is very helpful.

\par \textbf{PROTARCHUS}
\par   Then, Socrates, let us begin by clearing up these questions.

\par \textbf{SOCRATES}
\par   That is what I should wish.

\par \textbf{PROTARCHUS}
\par   And I am sure that all my other friends will be glad to hear them discussed; Philebus, fortunately for us, is not disposed to move, and we had better not stir him up with questions.

\par \textbf{SOCRATES}
\par   Good; and where shall we begin this great and multifarious battle, in which such various points are at issue? Shall we begin thus?

\par \textbf{PROTARCHUS}
\par   How?

\par \textbf{SOCRATES}
\par   We say that the one and many become identified by thought, and that now, as in time past, they run about together, in and out of every word which is uttered, and that this union of them will never cease, and is not now beginning, but is, as I believe, an everlasting quality of thought itself, which never grows old. Any young man, when he first tastes these subtleties, is delighted, and fancies that he has found a treasure of wisdom; in the first enthusiasm of his joy he leaves no stone, or rather no thought unturned, now rolling up the many into the one, and kneading them together, now unfolding and dividing them; he puzzles himself first and above all, and then he proceeds to puzzle his neighbours, whether they are older or younger, or of his own age—that makes no difference; neither father nor mother does he spare; no human being who has ears is safe from him, hardly even his dog, and a barbarian would have no chance of escaping him, if an interpreter could only be found.

\par \textbf{PROTARCHUS}
\par   Considering, Socrates, how many we are, and that all of us are young men, is there not a danger that we and Philebus may all set upon you, if you abuse us? We understand what you mean; but is there no charm by which we may dispel all this confusion, no more excellent way of arriving at the truth? If there is, we hope that you will guide us into that way, and we will do our best to follow, for the enquiry in which we are engaged, Socrates, is not unimportant.

\par \textbf{SOCRATES}
\par   The reverse of unimportant, my boys, as Philebus calls you, and there neither is nor ever will be a better than my own favourite way, which has nevertheless already often deserted me and left me helpless in the hour of need.

\par \textbf{PROTARCHUS}
\par   Tell us what that is.

\par \textbf{SOCRATES}
\par   One which may be easily pointed out, but is by no means easy of application; it is the parent of all the discoveries in the arts.

\par \textbf{PROTARCHUS}
\par   Tell us what it is.

\par \textbf{SOCRATES}
\par   A gift of heaven, which, as I conceive, the gods tossed among men by the hands of a new Prometheus, and therewith a blaze of light; and the ancients, who were our betters and nearer the gods than we are, handed down the tradition, that whatever things are said to be are composed of one and many, and have the finite and infinite implanted in them:  seeing, then, that such is the order of the world, we too ought in every enquiry to begin by laying down one idea of that which is the subject of enquiry; this unity we shall find in everything. Having found it, we may next proceed to look for two, if there be two, or, if not, then for three or some other number, subdividing each of these units, until at last the unity with which we began is seen not only to be one and many and infinite, but also a definite number; the infinite must not be suffered to approach the many until the entire number of the species intermediate between unity and infinity has been discovered,—then, and not till then, we may rest from division, and without further troubling ourselves about the endless individuals may allow them to drop into infinity. This, as I was saying, is the way of considering and learning and teaching one another, which the gods have handed down to us. But the wise men of our time are either too quick or too slow in conceiving plurality in unity. Having no method, they make their one and many anyhow, and from unity pass at once to infinity; the intermediate steps never occur to them. And this, I repeat, is what makes the difference between the mere art of disputation and true dialectic.

\par \textbf{PROTARCHUS}
\par   I think that I partly understand you Socrates, but I should like to have a clearer notion of what you are saying.

\par \textbf{SOCRATES}
\par   I may illustrate my meaning by the letters of the alphabet, Protarchus, which you were made to learn as a child.

\par \textbf{PROTARCHUS}
\par   How do they afford an illustration?

\par \textbf{SOCRATES}
\par   The sound which passes through the lips whether of an individual or of all men is one and yet infinite.

\par \textbf{PROTARCHUS}
\par   Very true.

\par \textbf{SOCRATES}
\par   And yet not by knowing either that sound is one or that sound is infinite are we perfect in the art of speech, but the knowledge of the number and nature of sounds is what makes a man a grammarian.

\par \textbf{PROTARCHUS}
\par   Very true.

\par \textbf{SOCRATES}
\par   And the knowledge which makes a man a musician is of the same kind.

\par \textbf{PROTARCHUS}
\par   How so?

\par \textbf{SOCRATES}
\par   Sound is one in music as well as in grammar?

\par \textbf{PROTARCHUS}
\par   Certainly.

\par \textbf{SOCRATES}
\par   And there is a higher note and a lower note, and a note of equal pitch: —may we affirm so much?

\par \textbf{PROTARCHUS}
\par   Yes.

\par \textbf{SOCRATES}
\par   But you would not be a real musician if this was all that you knew; though if you did not know this you would know almost nothing of music.

\par \textbf{PROTARCHUS}
\par   Nothing.

\par \textbf{SOCRATES}
\par   But when you have learned what sounds are high and what low, and the number and nature of the intervals and their limits or proportions, and the systems compounded out of them, which our fathers discovered, and have handed down to us who are their descendants under the name of harmonies; and the affections corresponding to them in the movements of the human body, which when measured by numbers ought, as they say, to be called rhythms and measures; and they tell us that the same principle should be applied to every one and many;—when, I say, you have learned all this, then, my dear friend, you are perfect; and you may be said to understand any other subject, when you have a similar grasp of it. But the infinity of kinds and the infinity of individuals which there is in each of them, when not classified, creates in every one of us a state of infinite ignorance; and he who never looks for number in anything, will not himself be looked for in the number of famous men.

\par \textbf{PROTARCHUS}
\par   I think that what Socrates is now saying is excellent, Philebus.

\par \textbf{PHILEBUS}
\par   I think so too, but how do his words bear upon us and upon the argument?

\par \textbf{SOCRATES}
\par   Philebus is right in asking that question of us, Protarchus.

\par \textbf{PROTARCHUS}
\par   Indeed he is, and you must answer him.

\par \textbf{SOCRATES}
\par   I will; but you must let me make one little remark first about these matters; I was saying, that he who begins with any individual unity, should proceed from that, not to infinity, but to a definite number, and now I say conversely, that he who has to begin with infinity should not jump to unity, but he should look about for some number representing a certain quantity, and thus out of all end in one. And now let us return for an illustration of our principle to the case of letters.

\par \textbf{PROTARCHUS}
\par   What do you mean?

\par \textbf{SOCRATES}
\par   Some god or divine man, who in the Egyptian legend is said to have been Theuth, observing that the human voice was infinite, first distinguished in this infinity a certain number of vowels, and then other letters which had sound, but were not pure vowels (i.e., the semivowels); these too exist in a definite number; and lastly, he distinguished a third class of letters which we now call mutes, without voice and without sound, and divided these, and likewise the two other classes of vowels and semivowels, into the individual sounds, and told the number of them, and gave to each and all of them the name of letters; and observing that none of us could learn any one of them and not learn them all, and in consideration of this common bond which in a manner united them, he assigned to them all a single art, and this he called the art of grammar or letters.

\par \textbf{PHILEBUS}
\par   The illustration, Protarchus, has assisted me in understanding the original statement, but I still feel the defect of which I just now complained.

\par \textbf{SOCRATES}
\par   Are you going to ask, Philebus, what this has to do with the argument?

\par \textbf{PHILEBUS}
\par   Yes, that is a question which Protarchus and I have been long asking.

\par \textbf{SOCRATES}
\par   Assuredly you have already arrived at the answer to the question which, as you say, you have been so long asking?

\par \textbf{PHILEBUS}
\par   How so?

\par \textbf{SOCRATES}
\par   Did we not begin by enquiring into the comparative eligibility of pleasure and wisdom?

\par \textbf{PHILEBUS}
\par   Certainly.

\par \textbf{SOCRATES}
\par   And we maintain that they are each of them one?

\par \textbf{PHILEBUS}
\par   True.

\par \textbf{SOCRATES}
\par   And the precise question to which the previous discussion desires an answer is, how they are one and also many (i.e., how they have one genus and many species), and are not at once infinite, and what number of species is to be assigned to either of them before they pass into infinity (i.e. into the infinite number of individuals).

\par \textbf{PROTARCHUS}
\par   That is a very serious question, Philebus, to which Socrates has ingeniously brought us round, and please to consider which of us shall answer him; there may be something ridiculous in my being unable to answer, and therefore imposing the task upon you, when I have undertaken the whole charge of the argument, but if neither of us were able to answer, the result methinks would be still more ridiculous. Let us consider, then, what we are to do: —Socrates, if I understood him rightly, is asking whether there are not kinds of pleasure, and what is the number and nature of them, and the same of wisdom.

\par \textbf{SOCRATES}
\par   Most true, O son of Callias; and the previous argument showed that if we are not able to tell the kinds of everything that has unity, likeness, sameness, or their opposites, none of us will be of the smallest use in any enquiry.

\par \textbf{PROTARCHUS}
\par   That seems to be very near the truth, Socrates. Happy would the wise man be if he knew all things, and the next best thing for him is that he should know himself. Why do I say so at this moment? I will tell you. You, Socrates, have granted us this opportunity of conversing with you, and are ready to assist us in determining what is the best of human goods. For when Philebus said that pleasure and delight and enjoyment and the like were the chief good, you answered—No, not those, but another class of goods; and we are constantly reminding ourselves of what you said, and very properly, in order that we may not forget to examine and compare the two. And these goods, which in your opinion are to be designated as superior to pleasure, and are the true objects of pursuit, are mind and knowledge and understanding and art, and the like. There was a dispute about which were the best, and we playfully threatened that you should not be allowed to go home until the question was settled; and you agreed, and placed yourself at our disposal. And now, as children say, what has been fairly given cannot be taken back; cease then to fight against us in this way.

\par \textbf{SOCRATES}
\par   In what way?

\par \textbf{PHILEBUS}
\par   Do not perplex us, and keep asking questions of us to which we have not as yet any sufficient answer to give; let us not imagine that a general puzzling of us all is to be the end of our discussion, but if we are unable to answer, do you answer, as you have promised. Consider, then, whether you will divide pleasure and knowledge according to their kinds; or you may let the matter drop, if you are able and willing to find some other mode of clearing up our controversy.

\par \textbf{SOCRATES}
\par   If you say that, I have nothing to apprehend, for the words 'if you are willing' dispel all my fear; and, moreover, a god seems to have recalled something to my mind.

\par \textbf{PHILEBUS}
\par   What is that?

\par \textbf{SOCRATES}
\par   I remember to have heard long ago certain discussions about pleasure and wisdom, whether awake or in a dream I cannot tell; they were to the effect that neither the one nor the other of them was the good, but some third thing, which was different from them, and better than either. If this be clearly established, then pleasure will lose the victory, for the good will cease to be identified with her: —Am I not right?

\par \textbf{PROTARCHUS}
\par   Yes.

\par \textbf{SOCRATES}
\par   And there will cease to be any need of distinguishing the kinds of pleasures, as I am inclined to think, but this will appear more clearly as we proceed.

\par \textbf{PROTARCHUS}
\par   Capital, Socrates; pray go on as you propose.

\par \textbf{SOCRATES}
\par   But, let us first agree on some little points.

\par \textbf{PROTARCHUS}
\par   What are they?

\par \textbf{SOCRATES}
\par   Is the good perfect or imperfect?

\par \textbf{PROTARCHUS}
\par   The most perfect, Socrates, of all things.

\par \textbf{SOCRATES}
\par   And is the good sufficient?

\par \textbf{PROTARCHUS}
\par   Yes, certainly, and in a degree surpassing all other things.

\par \textbf{SOCRATES}
\par   And no one can deny that all percipient beings desire and hunt after good, and are eager to catch and have the good about them, and care not for the attainment of anything which is not accompanied by good.

\par \textbf{PROTARCHUS}
\par   That is undeniable.

\par \textbf{SOCRATES}
\par   Now let us part off the life of pleasure from the life of wisdom, and pass them in review.

\par \textbf{PROTARCHUS}
\par   How do you mean?

\par \textbf{SOCRATES}
\par   Let there be no wisdom in the life of pleasure, nor any pleasure in the life of wisdom, for if either of them is the chief good, it cannot be supposed to want anything, but if either is shown to want anything, then it cannot really be the chief good.

\par \textbf{PROTARCHUS}
\par   Impossible.

\par \textbf{SOCRATES}
\par   And will you help us to test these two lives?

\par \textbf{PROTARCHUS}
\par   Certainly.

\par \textbf{SOCRATES}
\par   Then answer.

\par \textbf{PROTARCHUS}
\par   Ask.

\par \textbf{SOCRATES}
\par   Would you choose, Protarchus, to live all your life long in the enjoyment of the greatest pleasures?

\par \textbf{PROTARCHUS}
\par   Certainly I should.

\par \textbf{SOCRATES}
\par   Would you consider that there was still anything wanting to you if you had perfect pleasure?

\par \textbf{PROTARCHUS}
\par   Certainly not.

\par \textbf{SOCRATES}
\par   Reflect; would you not want wisdom and intelligence and forethought, and similar qualities? would you not at any rate want sight?

\par \textbf{PROTARCHUS}
\par   Why should I? Having pleasure I should have all things.

\par \textbf{SOCRATES}
\par   Living thus, you would always throughout your life enjoy the greatest pleasures?

\par \textbf{PROTARCHUS}
\par   I should.

\par \textbf{SOCRATES}
\par   But if you had neither mind, nor memory, nor knowledge, nor true opinion, you would in the first place be utterly ignorant of whether you were pleased or not, because you would be entirely devoid of intelligence.

\par \textbf{PROTARCHUS}
\par   Certainly.

\par \textbf{SOCRATES}
\par   And similarly, if you had no memory you would not recollect that you had ever been pleased, nor would the slightest recollection of the pleasure which you feel at any moment remain with you; and if you had no true opinion you would not think that you were pleased when you were; and if you had no power of calculation you would not be able to calculate on future pleasure, and your life would be the life, not of a man, but of an oyster or 'pulmo marinus.' Could this be otherwise?

\par \textbf{PROTARCHUS}
\par   No.

\par \textbf{SOCRATES}
\par   But is such a life eligible?

\par \textbf{PROTARCHUS}
\par   I cannot answer you, Socrates; the argument has taken away from me the power of speech.

\par \textbf{SOCRATES}
\par   We must keep up our spirits;—let us now take the life of mind and examine it in turn.

\par \textbf{PROTARCHUS}
\par   And what is this life of mind?

\par \textbf{SOCRATES}
\par   I want to know whether any one of us would consent to live, having wisdom and mind and knowledge and memory of all things, but having no sense of pleasure or pain, and wholly unaffected by these and the like feelings?

\par \textbf{PROTARCHUS}
\par   Neither life, Socrates, appears eligible to me, nor is likely, as I should imagine, to be chosen by any one else.

\par \textbf{SOCRATES}
\par   What would you say, Protarchus, to both of these in one, or to one that was made out of the union of the two?

\par \textbf{PROTARCHUS}
\par   Out of the union, that is, of pleasure with mind and wisdom?

\par \textbf{SOCRATES}
\par   Yes, that is the life which I mean.

\par \textbf{PROTARCHUS}
\par   There can be no difference of opinion; not some but all would surely choose this third rather than either of the other two, and in addition to them.

\par \textbf{SOCRATES}
\par   But do you see the consequence?

\par \textbf{PROTARCHUS}
\par   To be sure I do. The consequence is, that two out of the three lives which have been proposed are neither sufficient nor eligible for man or for animal.

\par \textbf{SOCRATES}
\par   Then now there can be no doubt that neither of them has the good, for the one which had would certainly have been sufficient and perfect and eligible for every living creature or thing that was able to live such a life; and if any of us had chosen any other, he would have chosen contrary to the nature of the truly eligible, and not of his own free will, but either through ignorance or from some unhappy necessity.

\par \textbf{PROTARCHUS}
\par   Certainly that seems to be true.

\par \textbf{SOCRATES}
\par   And now have I not sufficiently shown that Philebus' goddess is not to be regarded as identical with the good?

\par \textbf{PHILEBUS}
\par   Neither is your 'mind' the good, Socrates, for that will be open to the same objections.

\par \textbf{SOCRATES}
\par   Perhaps, Philebus, you may be right in saying so of my 'mind'; but of the true, which is also the divine mind, far otherwise. However, I will not at present claim the first place for mind as against the mixed life; but we must come to some understanding about the second place. For you might affirm pleasure and I mind to be the cause of the mixed life; and in that case although neither of them would be the good, one of them might be imagined to be the cause of the good. And I might proceed further to argue in opposition to Philebus, that the element which makes this mixed life eligible and good, is more akin and more similar to mind than to pleasure. And if this is true, pleasure cannot be truly said to share either in the first or second place, and does not, if I may trust my own mind, attain even to the third.

\par \textbf{PROTARCHUS}
\par   Truly, Socrates, pleasure appears to me to have had a fall; in fighting for the palm, she has been smitten by the argument, and is laid low. I must say that mind would have fallen too, and may therefore be thought to show discretion in not putting forward a similar claim. And if pleasure were deprived not only of the first but of the second place, she would be terribly damaged in the eyes of her admirers, for not even to them would she still appear as fair as before.

\par \textbf{SOCRATES}
\par   Well, but had we not better leave her now, and not pain her by applying the crucial test, and finally detecting her?

\par \textbf{PROTARCHUS}
\par   Nonsense, Socrates.

\par \textbf{SOCRATES}
\par   Why? because I said that we had better not pain pleasure, which is an impossibility?

\par \textbf{PROTARCHUS}
\par   Yes, and more than that, because you do not seem to be aware that none of us will let you go home until you have finished the argument.

\par \textbf{SOCRATES}
\par   Heavens! Protarchus, that will be a tedious business, and just at present not at all an easy one. For in going to war in the cause of mind, who is aspiring to the second prize, I ought to have weapons of another make from those which I used before; some, however, of the old ones may do again. And must I then finish the argument?

\par \textbf{PROTARCHUS}
\par   Of course you must.

\par \textbf{SOCRATES}
\par   Let us be very careful in laying the foundation.

\par \textbf{PROTARCHUS}
\par   What do you mean?

\par \textbf{SOCRATES}
\par   Let us divide all existing things into two, or rather, if you do not object, into three classes.

\par \textbf{PROTARCHUS}
\par   Upon what principle would you make the division?

\par \textbf{SOCRATES}
\par   Let us take some of our newly-found notions.

\par \textbf{PROTARCHUS}
\par   Which of them?

\par \textbf{SOCRATES}
\par   Were we not saying that God revealed a finite element of existence, and also an infinite?

\par \textbf{PROTARCHUS}
\par   Certainly.

\par \textbf{SOCRATES}
\par   Let us assume these two principles, and also a third, which is compounded out of them; but I fear that I am ridiculously clumsy at these processes of division and enumeration.

\par \textbf{PROTARCHUS}
\par   What do you mean, my good friend?

\par \textbf{SOCRATES}
\par   I say that a fourth class is still wanted.

\par \textbf{PROTARCHUS}
\par   What will that be?

\par \textbf{SOCRATES}
\par   Find the cause of the third or compound, and add this as a fourth class to the three others.

\par \textbf{PROTARCHUS}
\par   And would you like to have a fifth class or cause of resolution as well as a cause of composition?

\par \textbf{SOCRATES}
\par   Not, I think, at present; but if I want a fifth at some future time you shall allow me to have it.

\par \textbf{PROTARCHUS}
\par   Certainly.

\par \textbf{SOCRATES}
\par   Let us begin with the first three; and as we find two out of the three greatly divided and dispersed, let us endeavour to reunite them, and see how in each of them there is a one and many.

\par \textbf{PROTARCHUS}
\par   If you would explain to me a little more about them, perhaps I might be able to follow you.

\par \textbf{SOCRATES}
\par   Well, the two classes are the same which I mentioned before, one the finite, and the other the infinite; I will first show that the infinite is in a certain sense many, and the finite may be hereafter discussed.

\par \textbf{PROTARCHUS}
\par   I agree.

\par \textbf{SOCRATES}
\par   And now consider well; for the question to which I invite your attention is difficult and controverted. When you speak of hotter and colder, can you conceive any limit in those qualities? Does not the more and less, which dwells in their very nature, prevent their having any end? for if they had an end, the more and less would themselves have an end.

\par \textbf{PROTARCHUS}
\par   That is most true.

\par \textbf{SOCRATES}
\par   Ever, as we say, into the hotter and the colder there enters a more and a less.

\par \textbf{PROTARCHUS}
\par   Yes.

\par \textbf{SOCRATES}
\par   Then, says the argument, there is never any end of them, and being endless they must also be infinite.

\par \textbf{PROTARCHUS}
\par   Yes, Socrates, that is exceedingly true.

\par \textbf{SOCRATES}
\par   Yes, my dear Protarchus, and your answer reminds me that such an expression as 'exceedingly,' which you have just uttered, and also the term 'gently,' have the same significance as more or less; for whenever they occur they do not allow of the existence of quantity—they are always introducing degrees into actions, instituting a comparison of a more or a less excessive or a more or a less gentle, and at each creation of more or less, quantity disappears. For, as I was just now saying, if quantity and measure did not disappear, but were allowed to intrude in the sphere of more and less and the other comparatives, these last would be driven out of their own domain. When definite quantity is once admitted, there can be no longer a 'hotter' or a 'colder' (for these are always progressing, and are never in one stay); but definite quantity is at rest, and has ceased to progress. Which proves that comparatives, such as the hotter and the colder, are to be ranked in the class of the infinite.

\par \textbf{PROTARCHUS}
\par   Your remark certainly has the look of truth, Socrates; but these subjects, as you were saying, are difficult to follow at first. I think however, that if I could hear the argument repeated by you once or twice, there would be a substantial agreement between us.

\par \textbf{SOCRATES}
\par   Yes, and I will try to meet your wish; but, as I would rather not waste time in the enumeration of endless particulars, let me know whether I may not assume as a note of the infinite—

\par \textbf{PROTARCHUS}
\par   What?

\par \textbf{SOCRATES}
\par   I want to know whether such things as appear to us to admit of more or less, or are denoted by the words 'exceedingly,' 'gently,' 'extremely,' and the like, may not be referred to the class of the infinite, which is their unity, for, as was asserted in the previous argument, all things that were divided and dispersed should be brought together, and have the mark or seal of some one nature, if possible, set upon them—do you remember?

\par \textbf{PROTARCHUS}
\par   Yes.

\par \textbf{SOCRATES}
\par   And all things which do not admit of more or less, but admit their opposites, that is to say, first of all, equality, and the equal, or again, the double, or any other ratio of number and measure—all these may, I think, be rightly reckoned by us in the class of the limited or finite; what do you say?

\par \textbf{PROTARCHUS}
\par   Excellent, Socrates.

\par \textbf{SOCRATES}
\par   And now what nature shall we ascribe to the third or compound kind?

\par \textbf{PROTARCHUS}
\par   You, I think, will have to tell me that.

\par \textbf{SOCRATES}
\par   Rather God will tell you, if there be any God who will listen to my prayers.

\par \textbf{PROTARCHUS}
\par   Offer up a prayer, then, and think.

\par \textbf{SOCRATES}
\par   I am thinking, Protarchus, and I believe that some God has befriended us.

\par \textbf{PROTARCHUS}
\par   What do you mean, and what proof have you to offer of what you are saying?

\par \textbf{SOCRATES}
\par   I will tell you, and do you listen to my words.

\par \textbf{PROTARCHUS}
\par   Proceed.

\par \textbf{SOCRATES}
\par   Were we not speaking just now of hotter and colder?

\par \textbf{PROTARCHUS}
\par   True.

\par \textbf{SOCRATES}
\par   Add to them drier, wetter, more, less, swifter, slower, greater, smaller, and all that in the preceding argument we placed under the unity of more and less.

\par \textbf{PROTARCHUS}
\par   In the class of the infinite, you mean?

\par \textbf{SOCRATES}
\par   Yes; and now mingle this with the other.

\par \textbf{PROTARCHUS}
\par   What is the other.

\par \textbf{SOCRATES}
\par   The class of the finite which we ought to have brought together as we did the infinite; but, perhaps, it will come to the same thing if we do so now;—when the two are combined, a third will appear.

\par \textbf{PROTARCHUS}
\par   What do you mean by the class of the finite?

\par \textbf{SOCRATES}
\par   The class of the equal and the double, and any class which puts an end to difference and opposition, and by introducing number creates harmony and proportion among the different elements.

\par \textbf{PROTARCHUS}
\par   I understand; you seem to me to mean that the various opposites, when you mingle with them the class of the finite, takes certain forms.

\par \textbf{SOCRATES}
\par   Yes, that is my meaning.

\par \textbf{PROTARCHUS}
\par   Proceed.

\par \textbf{SOCRATES}
\par   Does not the right participation in the finite give health—in disease, for instance?

\par \textbf{PROTARCHUS}
\par   Certainly.

\par \textbf{SOCRATES}
\par   And whereas the high and low, the swift and the slow are infinite or unlimited, does not the addition of the principles aforesaid introduce a limit, and perfect the whole frame of music?

\par \textbf{PROTARCHUS}
\par   Yes, certainly.

\par \textbf{SOCRATES}
\par   Or, again, when cold and heat prevail, does not the introduction of them take away excess and indefiniteness, and infuse moderation and harmony?

\par \textbf{PROTARCHUS}
\par   Certainly.

\par \textbf{SOCRATES}
\par   And from a like admixture of the finite and infinite come the seasons, and all the delights of life?

\par \textbf{PROTARCHUS}
\par   Most true.

\par \textbf{SOCRATES}
\par   I omit ten thousand other things, such as beauty and health and strength, and the many beauties and high perfections of the soul:  O my beautiful Philebus, the goddess, methinks, seeing the universal wantonness and wickedness of all things, and that there was in them no limit to pleasures and self-indulgence, devised the limit of law and order, whereby, as you say, Philebus, she torments, or as I maintain, delivers the soul.—What think you, Protarchus?

\par \textbf{PROTARCHUS}
\par   Her ways are much to my mind, Socrates.

\par \textbf{SOCRATES}
\par   You will observe that I have spoken of three classes?

\par \textbf{PROTARCHUS}
\par   Yes, I think that I understand you:  you mean to say that the infinite is one class, and that the finite is a second class of existences; but what you would make the third I am not so certain.

\par \textbf{SOCRATES}
\par   That is because the amazing variety of the third class is too much for you, my dear friend; but there was not this difficulty with the infinite, which also comprehended many classes, for all of them were sealed with the note of more and less, and therefore appeared one.

\par \textbf{PROTARCHUS}
\par   True.

\par \textbf{SOCRATES}
\par   And the finite or limit had not many divisions, and we readily acknowledged it to be by nature one?

\par \textbf{PROTARCHUS}
\par   Yes.

\par \textbf{SOCRATES}
\par   Yes, indeed; and when I speak of the third class, understand me to mean any offspring of these, being a birth into true being, effected by the measure which the limit introduces.

\par \textbf{PROTARCHUS}
\par   I understand.

\par \textbf{SOCRATES}
\par   Still there was, as we said, a fourth class to be investigated, and you must assist in the investigation; for does not everything which comes into being, of necessity come into being through a cause?

\par \textbf{PROTARCHUS}
\par   Yes, certainly; for how can there be anything which has no cause?

\par \textbf{SOCRATES}
\par   And is not the agent the same as the cause in all except name; the agent and the cause may be rightly called one?

\par \textbf{PROTARCHUS}
\par   Very true.

\par \textbf{SOCRATES}
\par   And the same may be said of the patient, or effect; we shall find that they too differ, as I was saying, only in name—shall we not?

\par \textbf{PROTARCHUS}
\par   We shall.

\par \textbf{SOCRATES}
\par   The agent or cause always naturally leads, and the patient or effect naturally follows it?

\par \textbf{PROTARCHUS}
\par   Certainly.

\par \textbf{SOCRATES}
\par   Then the cause and what is subordinate to it in generation are not the same, but different?

\par \textbf{PROTARCHUS}
\par   True.

\par \textbf{SOCRATES}
\par   Did not the things which were generated, and the things out of which they were generated, furnish all the three classes?

\par \textbf{PROTARCHUS}
\par   Yes.

\par \textbf{SOCRATES}
\par   And the creator or cause of them has been satisfactorily proven to be distinct from them,—and may therefore be called a fourth principle?

\par \textbf{PROTARCHUS}
\par   So let us call it.

\par \textbf{SOCRATES}
\par   Quite right; but now, having distinguished the four, I think that we had better refresh our memories by recapitulating each of them in order.

\par \textbf{PROTARCHUS}
\par   By all means.

\par \textbf{SOCRATES}
\par   Then the first I will call the infinite or unlimited, and the second the finite or limited; then follows the third, an essence compound and generated; and I do not think that I shall be far wrong in speaking of the cause of mixture and generation as the fourth.

\par \textbf{PROTARCHUS}
\par   Certainly not.

\par \textbf{SOCRATES}
\par   And now what is the next question, and how came we hither? Were we not enquiring whether the second place belonged to pleasure or wisdom?

\par \textbf{PROTARCHUS}
\par   We were.

\par \textbf{SOCRATES}
\par   And now, having determined these points, shall we not be better able to decide about the first and second place, which was the original subject of dispute?

\par \textbf{PROTARCHUS}
\par   I dare say.

\par \textbf{SOCRATES}
\par   We said, if you remember, that the mixed life of pleasure and wisdom was the conqueror—did we not?

\par \textbf{PROTARCHUS}
\par   True.

\par \textbf{SOCRATES}
\par   And we see what is the place and nature of this life and to what class it is to be assigned?

\par \textbf{PROTARCHUS}
\par   Beyond a doubt.

\par \textbf{SOCRATES}
\par   This is evidently comprehended in the third or mixed class; which is not composed of any two particular ingredients, but of all the elements of infinity, bound down by the finite, and may therefore be truly said to comprehend the conqueror life.

\par \textbf{PROTARCHUS}
\par   Most true.

\par \textbf{SOCRATES}
\par   And what shall we say, Philebus, of your life which is all sweetness; and in which of the aforesaid classes is that to be placed? Perhaps you will allow me to ask you a question before you answer?

\par \textbf{PHILEBUS}
\par   Let me hear.

\par \textbf{SOCRATES}
\par   Have pleasure and pain a limit, or do they belong to the class which admits of more and less?

\par \textbf{PHILEBUS}
\par   They belong to the class which admits of more, Socrates; for pleasure would not be perfectly good if she were not infinite in quantity and degree.

\par \textbf{SOCRATES}
\par   Nor would pain, Philebus, be perfectly evil. And therefore the infinite cannot be that element which imparts to pleasure some degree of good. But now—admitting, if you like, that pleasure is of the nature of the infinite—in which of the aforesaid classes, O Protarchus and Philebus, can we without irreverence place wisdom and knowledge and mind? And let us be careful, for I think that the danger will be very serious if we err on this point.

\par \textbf{PHILEBUS}
\par   You magnify, Socrates, the importance of your favourite god.

\par \textbf{SOCRATES}
\par   And you, my friend, are also magnifying your favourite goddess; but still I must beg you to answer the question.

\par \textbf{PROTARCHUS}
\par   Socrates is quite right, Philebus, and we must submit to him.

\par \textbf{PHILEBUS}
\par   And did not you, Protarchus, propose to answer in my place?

\par \textbf{PROTARCHUS}
\par   Certainly I did; but I am now in a great strait, and I must entreat you, Socrates, to be our spokesman, and then we shall not say anything wrong or disrespectful of your favourite.

\par \textbf{SOCRATES}
\par   I must obey you, Protarchus; nor is the task which you impose a difficult one; but did I really, as Philebus implies, disconcert you with my playful solemnity, when I asked the question to what class mind and knowledge belong?

\par \textbf{PROTARCHUS}
\par   You did, indeed, Socrates.

\par \textbf{SOCRATES}
\par   Yet the answer is easy, since all philosophers assert with one voice that mind is the king of heaven and earth—in reality they are magnifying themselves. And perhaps they are right. But still I should like to consider the class of mind, if you do not object, a little more fully.

\par \textbf{PHILEBUS}
\par   Take your own course, Socrates, and never mind length; we shall not tire of you.

\par \textbf{SOCRATES}
\par   Very good; let us begin then, Protarchus, by asking a question.

\par \textbf{PROTARCHUS}
\par   What question?

\par \textbf{SOCRATES}
\par   Whether all this which they call the universe is left to the guidance of unreason and chance medley, or, on the contrary, as our fathers have declared, ordered and governed by a marvellous intelligence and wisdom.

\par \textbf{PROTARCHUS}
\par   Wide asunder are the two assertions, illustrious Socrates, for that which you were just now saying to me appears to be blasphemy; but the other assertion, that mind orders all things, is worthy of the aspect of the world, and of the sun, and of the moon, and of the stars and of the whole circle of the heavens; and never will I say or think otherwise.

\par \textbf{SOCRATES}
\par   Shall we then agree with them of old time in maintaining this doctrine,—not merely reasserting the notions of others, without risk to ourselves,—but shall we share in the danger, and take our part of the reproach which will await us, when an ingenious individual declares that all is disorder?

\par \textbf{PROTARCHUS}
\par   That would certainly be my wish.

\par \textbf{SOCRATES}
\par   Then now please to consider the next stage of the argument.

\par \textbf{PROTARCHUS}
\par   Let me hear.

\par \textbf{SOCRATES}
\par   We see that the elements which enter into the nature of the bodies of all animals, fire, water, air, and, as the storm-tossed sailor cries, 'land' (i.e., earth), reappear in the constitution of the world.

\par \textbf{PROTARCHUS}
\par   The proverb may be applied to us; for truly the storm gathers over us, and we are at our wit's end.

\par \textbf{SOCRATES}
\par   There is something to be remarked about each of these elements.

\par \textbf{PROTARCHUS}
\par   What is it?

\par \textbf{SOCRATES}
\par   Only a small fraction of any one of them exists in us, and that of a mean sort, and not in any way pure, or having any power worthy of its nature. One instance will prove this of all of them; there is fire within us, and in the universe.

\par \textbf{PROTARCHUS}
\par   True.

\par \textbf{SOCRATES}
\par   And is not our fire small and weak and mean? But the fire in the universe is wonderful in quantity and beauty, and in every power that fire has.

\par \textbf{PROTARCHUS}
\par   Most true.

\par \textbf{SOCRATES}
\par   And is the fire in the universe nourished and generated and ruled by the fire in us, or is the fire in you and me, and in other animals, dependent on the universal fire?

\par \textbf{PROTARCHUS}
\par   That is a question which does not deserve an answer.

\par \textbf{SOCRATES}
\par   Right; and you would say the same, if I am not mistaken, of the earth which is in animals and the earth which is in the universe, and you would give a similar reply about all the other elements?

\par \textbf{PROTARCHUS}
\par   Why, how could any man who gave any other be deemed in his senses?

\par \textbf{SOCRATES}
\par   I do not think that he could—but now go on to the next step. When we saw those elements of which we have been speaking gathered up in one, did we not call them a body?

\par \textbf{PROTARCHUS}
\par   We did.

\par \textbf{SOCRATES}
\par   And the same may be said of the cosmos, which for the same reason may be considered to be a body, because made up of the same elements.

\par \textbf{PROTARCHUS}
\par   Very true.

\par \textbf{SOCRATES}
\par   But is our body nourished wholly by this body, or is this body nourished by our body, thence deriving and having the qualities of which we were just now speaking?

\par \textbf{PROTARCHUS}
\par   That again, Socrates, is a question which does not deserve to be asked.

\par \textbf{SOCRATES}
\par   Well, tell me, is this question worth asking?

\par \textbf{PROTARCHUS}
\par   What question?

\par \textbf{SOCRATES}
\par   May our body be said to have a soul?

\par \textbf{PROTARCHUS}
\par   Clearly.

\par \textbf{SOCRATES}
\par   And whence comes that soul, my dear Protarchus, unless the body of the universe, which contains elements like those in our bodies but in every way fairer, had also a soul? Can there be another source?

\par \textbf{PROTARCHUS}
\par   Clearly, Socrates, that is the only source.

\par \textbf{SOCRATES}
\par   Why, yes, Protarchus; for surely we cannot imagine that of the four classes, the finite, the infinite, the composition of the two, and the cause, the fourth, which enters into all things, giving to our bodies souls, and the art of self-management, and of healing disease, and operating in other ways to heal and organize, having too all the attributes of wisdom;—we cannot, I say, imagine that whereas the self-same elements exist, both in the entire heaven and in great provinces of the heaven, only fairer and purer, this last should not also in that higher sphere have designed the noblest and fairest things?

\par \textbf{PROTARCHUS}
\par   Such a supposition is quite unreasonable.

\par \textbf{SOCRATES}
\par   Then if this be denied, should we not be wise in adopting the other view and maintaining that there is in the universe a mighty infinite and an adequate limit, of which we have often spoken, as well as a presiding cause of no mean power, which orders and arranges years and seasons and months, and may be justly called wisdom and mind?

\par \textbf{PROTARCHUS}
\par   Most justly.

\par \textbf{SOCRATES}
\par   And wisdom and mind cannot exist without soul?

\par \textbf{PROTARCHUS}
\par   Certainly not.

\par \textbf{SOCRATES}
\par   And in the divine nature of Zeus would you not say that there is the soul and mind of a king, because there is in him the power of the cause? And other gods have other attributes, by which they are pleased to be called.

\par \textbf{PROTARCHUS}
\par   Very true.

\par \textbf{SOCRATES}
\par   Do not then suppose that these words are rashly spoken by us, O Protarchus, for they are in harmony with the testimony of those who said of old time that mind rules the universe.

\par \textbf{PROTARCHUS}
\par   True.

\par \textbf{SOCRATES}
\par   And they furnish an answer to my enquiry; for they imply that mind is the parent of that class of the four which we called the cause of all; and I think that you now have my answer.

\par \textbf{PROTARCHUS}
\par   I have indeed, and yet I did not observe that you had answered.

\par \textbf{SOCRATES}
\par   A jest is sometimes refreshing, Protarchus, when it interrupts earnest.

\par \textbf{PROTARCHUS}
\par   Very true.

\par \textbf{SOCRATES}
\par   I think, friend, that we have now pretty clearly set forth the class to which mind belongs and what is the power of mind.

\par \textbf{PROTARCHUS}
\par   True.

\par \textbf{SOCRATES}
\par   And the class to which pleasure belongs has also been long ago discovered?

\par \textbf{PROTARCHUS}
\par   Yes.

\par \textbf{SOCRATES}
\par   And let us remember, too, of both of them, (1) that mind was akin to the cause and of this family; and (2) that pleasure is infinite and belongs to the class which neither has, nor ever will have in itself, a beginning, middle, or end of its own.

\par \textbf{PROTARCHUS}
\par   I shall be sure to remember.

\par \textbf{SOCRATES}
\par   We must next examine what is their place and under what conditions they are generated. And we will begin with pleasure, since her class was first examined; and yet pleasure cannot be rightly tested apart from pain.

\par \textbf{PROTARCHUS}
\par   If this is the road, let us take it.

\par \textbf{SOCRATES}
\par   I wonder whether you would agree with me about the origin of pleasure and pain.

\par \textbf{PROTARCHUS}
\par   What do you mean?

\par \textbf{SOCRATES}
\par   I mean to say that their natural seat is in the mixed class.

\par \textbf{PROTARCHUS}
\par   And would you tell me again, sweet Socrates, which of the aforesaid classes is the mixed one?

\par \textbf{SOCRATES}
\par   I will, my fine fellow, to the best of my ability.

\par \textbf{PROTARCHUS}
\par   Very good.

\par \textbf{SOCRATES}
\par   Let us then understand the mixed class to be that which we placed third in the list of four.

\par \textbf{PROTARCHUS}
\par   That which followed the infinite and the finite; and in which you ranked health, and, if I am not mistaken, harmony.

\par \textbf{SOCRATES}
\par   Capital; and now will you please to give me your best attention?

\par \textbf{PROTARCHUS}
\par   Proceed; I am attending.

\par \textbf{SOCRATES}
\par   I say that when the harmony in animals is dissolved, there is also a dissolution of nature and a generation of pain.

\par \textbf{PROTARCHUS}
\par   That is very probable.

\par \textbf{SOCRATES}
\par   And the restoration of harmony and return to nature is the source of pleasure, if I may be allowed to speak in the fewest and shortest words about matters of the greatest moment.

\par \textbf{PROTARCHUS}
\par   I believe that you are right, Socrates; but will you try to be a little plainer?

\par \textbf{SOCRATES}
\par   Do not obvious and every-day phenomena furnish the simplest illustration?

\par \textbf{PROTARCHUS}
\par   What phenomena do you mean?

\par \textbf{SOCRATES}
\par   Hunger, for example, is a dissolution and a pain.

\par \textbf{PROTARCHUS}
\par   True.

\par \textbf{SOCRATES}
\par   Whereas eating is a replenishment and a pleasure?

\par \textbf{PROTARCHUS}
\par   Yes.

\par \textbf{SOCRATES}
\par   Thirst again is a destruction and a pain, but the effect of moisture replenishing the dry place is a pleasure:  once more, the unnatural separation and dissolution caused by heat is painful, and the natural restoration and refrigeration is pleasant.

\par \textbf{PROTARCHUS}
\par   Very true.

\par \textbf{SOCRATES}
\par   And the unnatural freezing of the moisture in an animal is pain, and the natural process of resolution and return of the elements to their original state is pleasure. And would not the general proposition seem to you to hold, that the destroying of the natural union of the finite and infinite, which, as I was observing before, make up the class of living beings, is pain, and that the process of return of all things to their own nature is pleasure?

\par \textbf{PROTARCHUS}
\par   Granted; what you say has a general truth.

\par \textbf{SOCRATES}
\par   Here then is one kind of pleasures and pains originating severally in the two processes which we have described?

\par \textbf{PROTARCHUS}
\par   Good.

\par \textbf{SOCRATES}
\par   Let us next assume that in the soul herself there is an antecedent hope of pleasure which is sweet and refreshing, and an expectation of pain, fearful and anxious.

\par \textbf{PROTARCHUS}
\par   Yes; this is another class of pleasures and pains, which is of the soul only, apart from the body, and is produced by expectation.

\par \textbf{SOCRATES}
\par   Right; for in the analysis of these, pure, as I suppose them to be, the pleasures being unalloyed with pain and the pains with pleasure, methinks that we shall see clearly whether the whole class of pleasure is to be desired, or whether this quality of entire desirableness is not rather to be attributed to another of the classes which have been mentioned; and whether pleasure and pain, like heat and cold, and other things of the same kind, are not sometimes to be desired and sometimes not to be desired, as being not in themselves good, but only sometimes and in some instances admitting of the nature of good.

\par \textbf{PROTARCHUS}
\par   You say most truly that this is the track which the investigation should pursue.

\par \textbf{SOCRATES}
\par   Well, then, assuming that pain ensues on the dissolution, and pleasure on the restoration of the harmony, let us now ask what will be the condition of animated beings who are neither in process of restoration nor of dissolution. And mind what you say:  I ask whether any animal who is in that condition can possibly have any feeling of pleasure or pain, great or small?

\par \textbf{PROTARCHUS}
\par   Certainly not.

\par \textbf{SOCRATES}
\par   Then here we have a third state, over and above that of pleasure and of pain?

\par \textbf{PROTARCHUS}
\par   Very true.

\par \textbf{SOCRATES}
\par   And do not forget that there is such a state; it will make a great difference in our judgment of pleasure, whether we remember this or not. And I should like to say a few words about it.

\par \textbf{PROTARCHUS}
\par   What have you to say?

\par \textbf{SOCRATES}
\par   Why, you know that if a man chooses the life of wisdom, there is no reason why he should not live in this neutral state.

\par \textbf{PROTARCHUS}
\par   You mean that he may live neither rejoicing nor sorrowing?

\par \textbf{SOCRATES}
\par   Yes; and if I remember rightly, when the lives were compared, no degree of pleasure, whether great or small, was thought to be necessary to him who chose the life of thought and wisdom.

\par \textbf{PROTARCHUS}
\par   Yes, certainly, we said so.

\par \textbf{SOCRATES}
\par   Then he will live without pleasure; and who knows whether this may not be the most divine of all lives?

\par \textbf{PROTARCHUS}
\par   If so, the gods, at any rate, cannot be supposed to have either joy or sorrow.

\par \textbf{SOCRATES}
\par   Certainly not—there would be a great impropriety in the assumption of either alternative. But whether the gods are or are not indifferent to pleasure is a point which may be considered hereafter if in any way relevant to the argument, and whatever is the conclusion we will place it to the account of mind in her contest for the second place, should she have to resign the first.

\par \textbf{PROTARCHUS}
\par   Just so.

\par \textbf{SOCRATES}
\par   The other class of pleasures, which as we were saying is purely mental, is entirely derived from memory.

\par \textbf{PROTARCHUS}
\par   What do you mean?

\par \textbf{SOCRATES}
\par   I must first of all analyze memory, or rather perception which is prior to memory, if the subject of our discussion is ever to be properly cleared up.

\par \textbf{PROTARCHUS}
\par   How will you proceed?

\par \textbf{SOCRATES}
\par   Let us imagine affections of the body which are extinguished before they reach the soul, and leave her unaffected; and again, other affections which vibrate through both soul and body, and impart a shock to both and to each of them.

\par \textbf{PROTARCHUS}
\par   Granted.

\par \textbf{SOCRATES}
\par   And the soul may be truly said to be oblivious of the first but not of the second?

\par \textbf{PROTARCHUS}
\par   Quite true.

\par \textbf{SOCRATES}
\par   When I say oblivious, do not suppose that I mean forgetfulness in a literal sense; for forgetfulness is the exit of memory, which in this case has not yet entered; and to speak of the loss of that which is not yet in existence, and never has been, is a contradiction; do you see?

\par \textbf{PROTARCHUS}
\par   Yes.

\par \textbf{SOCRATES}
\par   Then just be so good as to change the terms.

\par \textbf{PROTARCHUS}
\par   How shall I change them?

\par \textbf{SOCRATES}
\par   Instead of the oblivion of the soul, when you are describing the state in which she is unaffected by the shocks of the body, say unconsciousness.

\par \textbf{PROTARCHUS}
\par   I see.

\par \textbf{SOCRATES}
\par   And the union or communion of soul and body in one feeling and motion would be properly called consciousness?

\par \textbf{PROTARCHUS}
\par   Most true.

\par \textbf{SOCRATES}
\par   Then now we know the meaning of the word?

\par \textbf{PROTARCHUS}
\par   Yes.

\par \textbf{SOCRATES}
\par   And memory may, I think, be rightly described as the preservation of consciousness?

\par \textbf{PROTARCHUS}
\par   Right.

\par \textbf{SOCRATES}
\par   But do we not distinguish memory from recollection?

\par \textbf{PROTARCHUS}
\par   I think so.

\par \textbf{SOCRATES}
\par   And do we not mean by recollection the power which the soul has of recovering, when by herself, some feeling which she experienced when in company with the body?

\par \textbf{PROTARCHUS}
\par   Certainly.

\par \textbf{SOCRATES}
\par   And when she recovers of herself the lost recollection of some consciousness or knowledge, the recovery is termed recollection and reminiscence?

\par \textbf{PROTARCHUS}
\par   Very true.

\par \textbf{SOCRATES}
\par   There is a reason why I say all this.

\par \textbf{PROTARCHUS}
\par   What is it?

\par \textbf{SOCRATES}
\par   I want to attain the plainest possible notion of pleasure and desire, as they exist in the mind only, apart from the body; and the previous analysis helps to show the nature of both.

\par \textbf{PROTARCHUS}
\par   Then now, Socrates, let us proceed to the next point.

\par \textbf{SOCRATES}
\par   There are certainly many things to be considered in discussing the generation and whole complexion of pleasure. At the outset we must determine the nature and seat of desire.

\par \textbf{PROTARCHUS}
\par   Ay; let us enquire into that, for we shall lose nothing.

\par \textbf{SOCRATES}
\par   Nay, Protarchus, we shall surely lose the puzzle if we find the answer.

\par \textbf{PROTARCHUS}
\par   A fair retort; but let us proceed.

\par \textbf{SOCRATES}
\par   Did we not place hunger, thirst, and the like, in the class of desires?

\par \textbf{PROTARCHUS}
\par   Certainly.

\par \textbf{SOCRATES}
\par   And yet they are very different; what common nature have we in view when we call them by a single name?

\par \textbf{PROTARCHUS}
\par   By heavens, Socrates, that is a question which is not easily answered; but it must be answered.

\par \textbf{SOCRATES}
\par   Then let us go back to our examples.

\par \textbf{PROTARCHUS}
\par   Where shall we begin?

\par \textbf{SOCRATES}
\par   Do we mean anything when we say 'a man thirsts'?

\par \textbf{PROTARCHUS}
\par   Yes.

\par \textbf{SOCRATES}
\par   We mean to say that he 'is empty'?

\par \textbf{PROTARCHUS}
\par   Of course.

\par \textbf{SOCRATES}
\par   And is not thirst desire?

\par \textbf{PROTARCHUS}
\par   Yes, of drink.

\par \textbf{SOCRATES}
\par   Would you say of drink, or of replenishment with drink?

\par \textbf{PROTARCHUS}
\par   I should say, of replenishment with drink.

\par \textbf{SOCRATES}
\par   Then he who is empty desires, as would appear, the opposite of what he experiences; for he is empty and desires to be full?

\par \textbf{PROTARCHUS}
\par   Clearly so.

\par \textbf{SOCRATES}
\par   But how can a man who is empty for the first time, attain either by perception or memory to any apprehension of replenishment, of which he has no present or past experience?

\par \textbf{PROTARCHUS}
\par   Impossible.

\par \textbf{SOCRATES}
\par   And yet he who desires, surely desires something?

\par \textbf{PROTARCHUS}
\par   Of course.

\par \textbf{SOCRATES}
\par   He does not desire that which he experiences, for he experiences thirst, and thirst is emptiness; but he desires replenishment?

\par \textbf{PROTARCHUS}
\par   True.

\par \textbf{SOCRATES}
\par   Then there must be something in the thirsty man which in some way apprehends replenishment?

\par \textbf{PROTARCHUS}
\par   There must.

\par \textbf{SOCRATES}
\par   And that cannot be the body, for the body is supposed to be emptied?

\par \textbf{PROTARCHUS}
\par   Yes.

\par \textbf{SOCRATES}
\par   The only remaining alternative is that the soul apprehends the replenishment by the help of memory; as is obvious, for what other way can there be?

\par \textbf{PROTARCHUS}
\par   I cannot imagine any other.

\par \textbf{SOCRATES}
\par   But do you see the consequence?

\par \textbf{PROTARCHUS}
\par   What is it?

\par \textbf{SOCRATES}
\par   That there is no such thing as desire of the body.

\par \textbf{PROTARCHUS}
\par   Why so?

\par \textbf{SOCRATES}
\par   Why, because the argument shows that the endeavour of every animal is to the reverse of his bodily state.

\par \textbf{PROTARCHUS}
\par   Yes.

\par \textbf{SOCRATES}
\par   And the impulse which leads him to the opposite of what he is experiencing proves that he has a memory of the opposite state.

\par \textbf{PROTARCHUS}
\par   True.

\par \textbf{SOCRATES}
\par   And the argument, having proved that memory attracts us towards the objects of desire, proves also that the impulses and the desires and the moving principle in every living being have their origin in the soul.

\par \textbf{PROTARCHUS}
\par   Most true.

\par \textbf{SOCRATES}
\par   The argument will not allow that our body either hungers or thirsts or has any similar experience.

\par \textbf{PROTARCHUS}
\par   Quite right.

\par \textbf{SOCRATES}
\par   Let me make a further observation; the argument appears to me to imply that there is a kind of life which consists in these affections.

\par \textbf{PROTARCHUS}
\par   Of what affections, and of what kind of life, are you speaking?

\par \textbf{SOCRATES}
\par   I am speaking of being emptied and replenished, and of all that relates to the preservation and destruction of living beings, as well as of the pain which is felt in one of these states and of the pleasure which succeeds to it.

\par \textbf{PROTARCHUS}
\par   True.

\par \textbf{SOCRATES}
\par   And what would you say of the intermediate state?

\par \textbf{PROTARCHUS}
\par   What do you mean by 'intermediate'?

\par \textbf{SOCRATES}
\par   I mean when a person is in actual suffering and yet remembers past pleasures which, if they would only return, would relieve him; but as yet he has them not. May we not say of him, that he is in an intermediate state?

\par \textbf{PROTARCHUS}
\par   Certainly.

\par \textbf{SOCRATES}
\par   Would you say that he was wholly pained or wholly pleased?

\par \textbf{PROTARCHUS}
\par   Nay, I should say that he has two pains; in his body there is the actual experience of pain, and in his soul longing and expectation.

\par \textbf{SOCRATES}
\par   What do you mean, Protarchus, by the two pains? May not a man who is empty have at one time a sure hope of being filled, and at other times be quite in despair?

\par \textbf{PROTARCHUS}
\par   Very true.

\par \textbf{SOCRATES}
\par   And has he not the pleasure of memory when he is hoping to be filled, and yet in that he is empty is he not at the same time in pain?

\par \textbf{PROTARCHUS}
\par   Certainly.

\par \textbf{SOCRATES}
\par   Then man and the other animals have at the same time both pleasure and pain?

\par \textbf{PROTARCHUS}
\par   I suppose so.

\par \textbf{SOCRATES}
\par   But when a man is empty and has no hope of being filled, there will be the double experience of pain. You observed this and inferred that the double experience was the single case possible.

\par \textbf{PROTARCHUS}
\par   Quite true, Socrates.

\par \textbf{SOCRATES}
\par   Shall the enquiry into these states of feeling be made the occasion of raising a question?

\par \textbf{PROTARCHUS}
\par   What question?

\par \textbf{SOCRATES}
\par   Whether we ought to say that the pleasures and pains of which we are speaking are true or false? or some true and some false?

\par \textbf{PROTARCHUS}
\par   But how, Socrates, can there be false pleasures and pains?

\par \textbf{SOCRATES}
\par   And how, Protarchus, can there be true and false fears, or true and false expectations, or true and false opinions?

\par \textbf{PROTARCHUS}
\par   I grant that opinions may be true or false, but not pleasures.

\par \textbf{SOCRATES}
\par   What do you mean? I am afraid that we are raising a very serious enquiry.

\par \textbf{PROTARCHUS}
\par   There I agree.

\par \textbf{SOCRATES}
\par   And yet, my boy, for you are one of Philebus' boys, the point to be considered, is, whether the enquiry is relevant to the argument.

\par \textbf{PROTARCHUS}
\par   Surely.

\par \textbf{SOCRATES}
\par   No tedious and irrelevant discussion can be allowed; what is said should be pertinent.

\par \textbf{PROTARCHUS}
\par   Right.

\par \textbf{SOCRATES}
\par   I am always wondering at the question which has now been raised.

\par \textbf{PROTARCHUS}
\par   How so?

\par \textbf{SOCRATES}
\par   Do you deny that some pleasures are false, and others true?

\par \textbf{PROTARCHUS}
\par   To be sure I do.

\par \textbf{SOCRATES}
\par   Would you say that no one ever seemed to rejoice and yet did not rejoice, or seemed to feel pain and yet did not feel pain, sleeping or waking, mad or lunatic?

\par \textbf{PROTARCHUS}
\par   So we have always held, Socrates.

\par \textbf{SOCRATES}
\par   But were you right? Shall we enquire into the truth of your opinion?

\par \textbf{PROTARCHUS}
\par   I think that we should.

\par \textbf{SOCRATES}
\par   Let us then put into more precise terms the question which has arisen about pleasure and opinion. Is there such a thing as opinion?

\par \textbf{PROTARCHUS}
\par   Yes.

\par \textbf{SOCRATES}
\par   And such a thing as pleasure?

\par \textbf{PROTARCHUS}
\par   Yes.

\par \textbf{SOCRATES}
\par   And an opinion must be of something?

\par \textbf{PROTARCHUS}
\par   True.

\par \textbf{SOCRATES}
\par   And a man must be pleased by something?

\par \textbf{PROTARCHUS}
\par   Quite correct.

\par \textbf{SOCRATES}
\par   And whether the opinion be right or wrong, makes no difference; it will still be an opinion?

\par \textbf{PROTARCHUS}
\par   Certainly.

\par \textbf{SOCRATES}
\par   And he who is pleased, whether he is rightly pleased or not, will always have a real feeling of pleasure?

\par \textbf{PROTARCHUS}
\par   Yes; that is also quite true.

\par \textbf{SOCRATES}
\par   Then, how can opinion be both true and false, and pleasure true only, although pleasure and opinion are both equally real?

\par \textbf{PROTARCHUS}
\par   Yes; that is the question.

\par \textbf{SOCRATES}
\par   You mean that opinion admits of truth and falsehood, and hence becomes not merely opinion, but opinion of a certain quality; and this is what you think should be examined?

\par \textbf{PROTARCHUS}
\par   Yes.

\par \textbf{SOCRATES}
\par   And further, even if we admit the existence of qualities in other objects, may not pleasure and pain be simple and devoid of quality?

\par \textbf{PROTARCHUS}
\par   Clearly.

\par \textbf{SOCRATES}
\par   But there is no difficulty in seeing that pleasure and pain as well as opinion have qualities, for they are great or small, and have various degrees of intensity; as was indeed said long ago by us.

\par \textbf{PROTARCHUS}
\par   Quite true.

\par \textbf{SOCRATES}
\par   And if badness attaches to any of them, Protarchus, then we should speak of a bad opinion or of a bad pleasure?

\par \textbf{PROTARCHUS}
\par   Quite true, Socrates.

\par \textbf{SOCRATES}
\par   And if rightness attaches to any of them, should we not speak of a right opinion or right pleasure; and in like manner of the reverse of rightness?

\par \textbf{PROTARCHUS}
\par   Certainly.

\par \textbf{SOCRATES}
\par   And if the thing opined be erroneous, might we not say that the opinion, being erroneous, is not right or rightly opined?

\par \textbf{PROTARCHUS}
\par   Certainly.

\par \textbf{SOCRATES}
\par   And if we see a pleasure or pain which errs in respect of its object, shall we call that right or good, or by any honourable name?

\par \textbf{PROTARCHUS}
\par   Not if the pleasure is mistaken; how could we?

\par \textbf{SOCRATES}
\par   And surely pleasure often appears to accompany an opinion which is not true, but false?

\par \textbf{PROTARCHUS}
\par   Certainly it does; and in that case, Socrates, as we were saying, the opinion is false, but no one could call the actual pleasure false.

\par \textbf{SOCRATES}
\par   How eagerly, Protarchus, do you rush to the defence of pleasure!

\par \textbf{PROTARCHUS}
\par   Nay, Socrates, I only repeat what I hear.

\par \textbf{SOCRATES}
\par   And is there no difference, my friend, between that pleasure which is associated with right opinion and knowledge, and that which is often found in all of us associated with falsehood and ignorance?

\par \textbf{PROTARCHUS}
\par   There must be a very great difference, between them.

\par \textbf{SOCRATES}
\par   Then, now let us proceed to contemplate this difference.

\par \textbf{PROTARCHUS}
\par   Lead, and I will follow.

\par \textbf{SOCRATES}
\par   Well, then, my view is—

\par \textbf{PROTARCHUS}
\par   What is it?

\par \textbf{SOCRATES}
\par   We agree—do we not?—that there is such a thing as false, and also such a thing as true opinion?

\par \textbf{PROTARCHUS}
\par   Yes.

\par \textbf{SOCRATES}
\par   And pleasure and pain, as I was just now saying, are often consequent upon these—upon true and false opinion, I mean.

\par \textbf{PROTARCHUS}
\par   Very true.

\par \textbf{SOCRATES}
\par   And do not opinion and the endeavour to form an opinion always spring from memory and perception?

\par \textbf{PROTARCHUS}
\par   Certainly.

\par \textbf{SOCRATES}
\par   Might we imagine the process to be something of this nature?

\par \textbf{PROTARCHUS}
\par   Of what nature?

\par \textbf{SOCRATES}
\par   An object may be often seen at a distance not very clearly, and the seer may want to determine what it is which he sees.

\par \textbf{PROTARCHUS}
\par   Very likely.

\par \textbf{SOCRATES}
\par   Soon he begins to interrogate himself.

\par \textbf{PROTARCHUS}
\par   In what manner?

\par \textbf{SOCRATES}
\par   He asks himself—'What is that which appears to be standing by the rock under the tree?' This is the question which he may be supposed to put to himself when he sees such an appearance.

\par \textbf{PROTARCHUS}
\par   True.

\par \textbf{SOCRATES}
\par   To which he may guess the right answer, saying as if in a whisper to himself—'It is a man.'

\par \textbf{PROTARCHUS}
\par   Very good.

\par \textbf{SOCRATES}
\par   Or again, he may be misled, and then he will say—'No, it is a figure made by the shepherds.'

\par \textbf{PROTARCHUS}
\par   Yes.

\par \textbf{SOCRATES}
\par   And if he has a companion, he repeats his thought to him in articulate sounds, and what was before an opinion, has now become a proposition.

\par \textbf{PROTARCHUS}
\par   Certainly.

\par \textbf{SOCRATES}
\par   But if he be walking alone when these thoughts occur to him, he may not unfrequently keep them in his mind for a considerable time.

\par \textbf{PROTARCHUS}
\par   Very true.

\par \textbf{SOCRATES}
\par   Well, now, I wonder whether you would agree in my explanation of this phenomenon.

\par \textbf{PROTARCHUS}
\par   What is your explanation?

\par \textbf{SOCRATES}
\par   I think that the soul at such times is like a book.

\par \textbf{PROTARCHUS}
\par   How so?

\par \textbf{SOCRATES}
\par   Memory and perception meet, and they and their attendant feelings seem to almost to write down words in the soul, and when the inscribing feeling writes truly, then true opinion and true propositions which are the expressions of opinion come into our souls—but when the scribe within us writes falsely, the result is false.

\par \textbf{PROTARCHUS}
\par   I quite assent and agree to your statement.

\par \textbf{SOCRATES}
\par   I must bespeak your favour also for another artist, who is busy at the same time in the chambers of the soul.

\par \textbf{PROTARCHUS}
\par   Who is he?

\par \textbf{SOCRATES}
\par   The painter, who, after the scribe has done his work, draws images in the soul of the things which he has described.

\par \textbf{PROTARCHUS}
\par   But when and how does he do this?

\par \textbf{SOCRATES}
\par   When a man, besides receiving from sight or some other sense certain opinions or statements, sees in his mind the images of the subjects of them;—is not this a very common mental phenomenon?

\par \textbf{PROTARCHUS}
\par   Certainly.

\par \textbf{SOCRATES}
\par   And the images answering to true opinions and words are true, and to false opinions and words false; are they not?

\par \textbf{PROTARCHUS}
\par   They are.

\par \textbf{SOCRATES}
\par   If we are right so far, there arises a further question.

\par \textbf{PROTARCHUS}
\par   What is it?

\par \textbf{SOCRATES}
\par   Whether we experience the feeling of which I am speaking only in relation to the present and the past, or in relation to the future also?

\par \textbf{PROTARCHUS}
\par   I should say in relation to all times alike.

\par \textbf{SOCRATES}
\par   Have not purely mental pleasures and pains been described already as in some cases anticipations of the bodily ones; from which we may infer that anticipatory pleasures and pains have to do with the future?

\par \textbf{PROTARCHUS}
\par   Most true.

\par \textbf{SOCRATES}
\par   And do all those writings and paintings which, as we were saying a little while ago, are produced in us, relate to the past and present only, and not to the future?

\par \textbf{PROTARCHUS}
\par   To the future, very much.

\par \textbf{SOCRATES}
\par   When you say, 'Very much,' you mean to imply that all these representations are hopes about the future, and that mankind are filled with hopes in every stage of existence?

\par \textbf{PROTARCHUS}
\par   Exactly.

\par \textbf{SOCRATES}
\par   Answer me another question.

\par \textbf{PROTARCHUS}
\par   What question?

\par \textbf{SOCRATES}
\par   A just and pious and good man is the friend of the gods; is he not?

\par \textbf{PROTARCHUS}
\par   Certainly he is.

\par \textbf{SOCRATES}
\par   And the unjust and utterly bad man is the reverse?

\par \textbf{PROTARCHUS}
\par   True.

\par \textbf{SOCRATES}
\par   And all men, as we were saying just now, are always filled with hopes?

\par \textbf{PROTARCHUS}
\par   Certainly.

\par \textbf{SOCRATES}
\par   And these hopes, as they are termed, are propositions which exist in the minds of each of us?

\par \textbf{PROTARCHUS}
\par   Yes.

\par \textbf{SOCRATES}
\par   And the fancies of hope are also pictured in us; a man may often have a vision of a heap of gold, and pleasures ensuing, and in the picture there may be a likeness of himself mightily rejoicing over his good fortune.

\par \textbf{PROTARCHUS}
\par   True.

\par \textbf{SOCRATES}
\par   And may we not say that the good, being friends of the gods, have generally true pictures presented to them, and the bad false pictures?

\par \textbf{PROTARCHUS}
\par   Certainly.

\par \textbf{SOCRATES}
\par   The bad, too, have pleasures painted in their fancy as well as the good; but I presume that they are false pleasures.

\par \textbf{PROTARCHUS}
\par   They are.

\par \textbf{SOCRATES}
\par   The bad then commonly delight in false pleasures, and the good in true pleasures?

\par \textbf{PROTARCHUS}
\par   Doubtless.

\par \textbf{SOCRATES}
\par   Then upon this view there are false pleasures in the souls of men which are a ludicrous imitation of the true, and there are pains of a similar character?

\par \textbf{PROTARCHUS}
\par   There are.

\par \textbf{SOCRATES}
\par   And did we not allow that a man who had an opinion at all had a real opinion, but often about things which had no existence either in the past, present, or future?

\par \textbf{PROTARCHUS}
\par   Quite true.

\par \textbf{SOCRATES}
\par   And this was the source of false opinion and opining; am I not right?

\par \textbf{PROTARCHUS}
\par   Yes.

\par \textbf{SOCRATES}
\par   And must we not attribute to pleasure and pain a similar real but illusory character?

\par \textbf{PROTARCHUS}
\par   How do you mean?

\par \textbf{SOCRATES}
\par   I mean to say that a man must be admitted to have real pleasure who is pleased with anything or anyhow; and he may be pleased about things which neither have nor have ever had any real existence, and, more often than not, are never likely to exist.

\par \textbf{PROTARCHUS}
\par   Yes, Socrates, that again is undeniable.

\par \textbf{SOCRATES}
\par   And may not the same be said about fear and anger and the like; are they not often false?

\par \textbf{PROTARCHUS}
\par   Quite so.

\par \textbf{SOCRATES}
\par   And can opinions be good or bad except in as far as they are true or false?

\par \textbf{PROTARCHUS}
\par   In no other way.

\par \textbf{SOCRATES}
\par   Nor can pleasures be conceived to be bad except in so far as they are false.

\par \textbf{PROTARCHUS}
\par   Nay, Socrates, that is the very opposite of truth; for no one would call pleasures and pains bad because they are false, but by reason of some other great corruption to which they are liable.

\par \textbf{SOCRATES}
\par   Well, of pleasures which are corrupt and caused by corruption we will hereafter speak, if we care to continue the enquiry; for the present I would rather show by another argument that there are many false pleasures existing or coming into existence in us, because this may assist our final decision.

\par \textbf{PROTARCHUS}
\par   Very true; that is to say, if there are such pleasures.

\par \textbf{SOCRATES}
\par   I think that there are, Protarchus; but this is an opinion which should be well assured, and not rest upon a mere assertion.

\par \textbf{PROTARCHUS}
\par   Very good.

\par \textbf{SOCRATES}
\par   Then now, like wrestlers, let us approach and grasp this new argument.

\par \textbf{PROTARCHUS}
\par   Proceed.

\par \textbf{SOCRATES}
\par   We were maintaining a little while since, that when desires, as they are termed, exist in us, then the body has separate feelings apart from the soul—do you remember?

\par \textbf{PROTARCHUS}
\par   Yes, I remember that you said so.

\par \textbf{SOCRATES}
\par   And the soul was supposed to desire the opposite of the bodily state, while the body was the source of any pleasure or pain which was experienced.

\par \textbf{PROTARCHUS}
\par   True.

\par \textbf{SOCRATES}
\par   Then now you may infer what happens in such cases.

\par \textbf{PROTARCHUS}
\par   What am I to infer?

\par \textbf{SOCRATES}
\par   That in such cases pleasures and pains come simultaneously; and there is a juxtaposition of the opposite sensations which correspond to them, as has been already shown.

\par \textbf{PROTARCHUS}
\par   Clearly.

\par \textbf{SOCRATES}
\par   And there is another point to which we have agreed.

\par \textbf{PROTARCHUS}
\par   What is it?

\par \textbf{SOCRATES}
\par   That pleasure and pain both admit of more and less, and that they are of the class of infinites.

\par \textbf{PROTARCHUS}
\par   Certainly, we said so.

\par \textbf{SOCRATES}
\par   But how can we rightly judge of them?

\par \textbf{PROTARCHUS}
\par   How can we?

\par \textbf{SOCRATES}
\par   Is it our intention to judge of their comparative importance and intensity, measuring pleasure against pain, and pain against pain, and pleasure against pleasure?

\par \textbf{PROTARCHUS}
\par   Yes, such is our intention, and we shall judge of them accordingly.

\par \textbf{SOCRATES}
\par   Well, take the case of sight. Does not the nearness or distance of magnitudes obscure their true proportions, and make us opine falsely; and do we not find the same illusion happening in the case of pleasures and pains?

\par \textbf{PROTARCHUS}
\par   Yes, Socrates, and in a degree far greater.

\par \textbf{SOCRATES}
\par   Then what we are now saying is the opposite of what we were saying before.

\par \textbf{PROTARCHUS}
\par   What was that?

\par \textbf{SOCRATES}
\par   Then the opinions were true and false, and infected the pleasures and pains with their own falsity.

\par \textbf{PROTARCHUS}
\par   Very true.

\par \textbf{SOCRATES}
\par   But now it is the pleasures which are said to be true and false because they are seen at various distances, and subjected to comparison; the pleasures appear to be greater and more vehement when placed side by side with the pains, and the pains when placed side by side with the pleasures.

\par \textbf{PROTARCHUS}
\par   Certainly, and for the reason which you mention.

\par \textbf{SOCRATES}
\par   And suppose you part off from pleasures and pains the element which makes them appear to be greater or less than they really are:  you will acknowledge that this element is illusory, and you will never say that the corresponding excess or defect of pleasure or pain is real or true.

\par \textbf{PROTARCHUS}
\par   Certainly not.

\par \textbf{SOCRATES}
\par   Next let us see whether in another direction we may not find pleasures and pains existing and appearing in living beings, which are still more false than these.

\par \textbf{PROTARCHUS}
\par   What are they, and how shall we find them?

\par \textbf{SOCRATES}
\par   If I am not mistaken, I have often repeated that pains and aches and suffering and uneasiness of all sorts arise out of a corruption of nature caused by concretions, and dissolutions, and repletions, and evacuations, and also by growth and decay?

\par \textbf{PROTARCHUS}
\par   Yes, that has been often said.

\par \textbf{SOCRATES}
\par   And we have also agreed that the restoration of the natural state is pleasure?

\par \textbf{PROTARCHUS}
\par   Right.

\par \textbf{SOCRATES}
\par   But now let us suppose an interval of time at which the body experiences none of these changes.

\par \textbf{PROTARCHUS}
\par   When can that be, Socrates?

\par \textbf{SOCRATES}
\par   Your question, Protarchus, does not help the argument.

\par \textbf{PROTARCHUS}
\par   Why not, Socrates?

\par \textbf{SOCRATES}
\par   Because it does not prevent me from repeating mine.

\par \textbf{PROTARCHUS}
\par   And what was that?

\par \textbf{SOCRATES}
\par   Why, Protarchus, admitting that there is no such interval, I may ask what would be the necessary consequence if there were?

\par \textbf{PROTARCHUS}
\par   You mean, what would happen if the body were not changed either for good or bad?

\par \textbf{SOCRATES}
\par   Yes.

\par \textbf{PROTARCHUS}
\par   Why then, Socrates, I should suppose that there would be neither pleasure nor pain.

\par \textbf{SOCRATES}
\par   Very good; but still, if I am not mistaken, you do assert that we must always be experiencing one of them; that is what the wise tell us; for, say they, all things are ever flowing up and down.

\par \textbf{PROTARCHUS}
\par   Yes, and their words are of no mean authority.

\par \textbf{SOCRATES}
\par   Of course, for they are no mean authorities themselves; and I should like to avoid the brunt of their argument. Shall I tell you how I mean to escape from them? And you shall be the partner of my flight.

\par \textbf{PROTARCHUS}
\par   How?

\par \textbf{SOCRATES}
\par   To them we will say:  'Good; but are we, or living things in general, always conscious of what happens to us—for example, of our growth, or the like? Are we not, on the contrary, almost wholly unconscious of this and similar phenomena?' You must answer for them.

\par \textbf{PROTARCHUS}
\par   The latter alternative is the true one.

\par \textbf{SOCRATES}
\par   Then we were not right in saying, just now, that motions going up and down cause pleasures and pains?

\par \textbf{PROTARCHUS}
\par   True.

\par \textbf{SOCRATES}
\par   A better and more unexceptionable way of speaking will be—

\par \textbf{PROTARCHUS}
\par   What?

\par \textbf{SOCRATES}
\par   If we say that the great changes produce pleasures and pains, but that the moderate and lesser ones do neither.

\par \textbf{PROTARCHUS}
\par   That, Socrates, is the more correct mode of speaking.

\par \textbf{SOCRATES}
\par   But if this be true, the life to which I was just now referring again appears.

\par \textbf{PROTARCHUS}
\par   What life?

\par \textbf{SOCRATES}
\par   The life which we affirmed to be devoid either of pain or of joy.

\par \textbf{PROTARCHUS}
\par   Very true.

\par \textbf{SOCRATES}
\par   We may assume then that there are three lives, one pleasant, one painful, and the third which is neither; what say you?

\par \textbf{PROTARCHUS}
\par   I should say as you do that there are three of them.

\par \textbf{SOCRATES}
\par   But if so, the negation of pain will not be the same with pleasure.

\par \textbf{PROTARCHUS}
\par   Certainly not.

\par \textbf{SOCRATES}
\par   Then when you hear a person saying, that always to live without pain is the pleasantest of all things, what would you understand him to mean by that statement?

\par \textbf{PROTARCHUS}
\par   I think that by pleasure he must mean the negative of pain.

\par \textbf{SOCRATES}
\par   Let us take any three things; or suppose that we embellish a little and call the first gold, the second silver, and there shall be a third which is neither.

\par \textbf{PROTARCHUS}
\par   Very good.

\par \textbf{SOCRATES}
\par   Now, can that which is neither be either gold or silver?

\par \textbf{PROTARCHUS}
\par   Impossible.

\par \textbf{SOCRATES}
\par   No more can that neutral or middle life be rightly or reasonably spoken or thought of as pleasant or painful.

\par \textbf{PROTARCHUS}
\par   Certainly not.

\par \textbf{SOCRATES}
\par   And yet, my friend, there are, as we know, persons who say and think so.

\par \textbf{PROTARCHUS}
\par   Certainly.

\par \textbf{SOCRATES}
\par   And do they think that they have pleasure when they are free from pain?

\par \textbf{PROTARCHUS}
\par   They say so.

\par \textbf{SOCRATES}
\par   And they must think or they would not say that they have pleasure.

\par \textbf{PROTARCHUS}
\par   I suppose not.

\par \textbf{SOCRATES}
\par   And yet if pleasure and the negation of pain are of distinct natures, they are wrong.

\par \textbf{PROTARCHUS}
\par   But they are undoubtedly of distinct natures.

\par \textbf{SOCRATES}
\par   Then shall we take the view that they are three, as we were just now saying, or that they are two only—the one being a state of pain, which is an evil, and the other a cessation of pain, which is of itself a good, and is called pleasant?

\par \textbf{PROTARCHUS}
\par   But why, Socrates, do we ask the question at all? I do not see the reason.

\par \textbf{SOCRATES}
\par   You, Protarchus, have clearly never heard of certain enemies of our friend Philebus.

\par \textbf{PROTARCHUS}
\par   And who may they be?

\par \textbf{SOCRATES}
\par   Certain persons who are reputed to be masters in natural philosophy, who deny the very existence of pleasure.

\par \textbf{PROTARCHUS}
\par   Indeed!

\par \textbf{SOCRATES}
\par   They say that what the school of Philebus calls pleasures are all of them only avoidances of pain.

\par \textbf{PROTARCHUS}
\par   And would you, Socrates, have us agree with them?

\par \textbf{SOCRATES}
\par   Why, no, I would rather use them as a sort of diviners, who divine the truth, not by rules of art, but by an instinctive repugnance and extreme detestation which a noble nature has of the power of pleasure, in which they think that there is nothing sound, and her seductive influence is declared by them to be witchcraft, and not pleasure. This is the use which you may make of them. And when you have considered the various grounds of their dislike, you shall hear from me what I deem to be true pleasures. Having thus examined the nature of pleasure from both points of view, we will bring her up for judgment.

\par \textbf{PROTARCHUS}
\par   Well said.

\par \textbf{SOCRATES}
\par   Then let us enter into an alliance with these philosophers and follow in the track of their dislike. I imagine that they would say something of this sort; they would begin at the beginning, and ask whether, if we wanted to know the nature of any quality, such as hardness, we should be more likely to discover it by looking at the hardest things, rather than at the least hard? You, Protarchus, shall answer these severe gentlemen as you answer me.

\par \textbf{PROTARCHUS}
\par   By all means, and I reply to them, that you should look at the greatest instances.

\par \textbf{SOCRATES}
\par   Then if we want to see the true nature of pleasures as a class, we should not look at the most diluted pleasures, but at the most extreme and most vehement?

\par \textbf{PROTARCHUS}
\par   In that every one will agree.

\par \textbf{SOCRATES}
\par   And the obvious instances of the greatest pleasures, as we have often said, are the pleasures of the body?

\par \textbf{PROTARCHUS}
\par   Certainly.

\par \textbf{SOCRATES}
\par   And are they felt by us to be or become greater, when we are sick or when we are in health? And here we must be careful in our answer, or we shall come to grief.

\par \textbf{PROTARCHUS}
\par   How will that be?

\par \textbf{SOCRATES}
\par   Why, because we might be tempted to answer, 'When we are in health.'

\par \textbf{PROTARCHUS}
\par   Yes, that is the natural answer.

\par \textbf{SOCRATES}
\par   Well, but are not those pleasures the greatest of which mankind have the greatest desires?

\par \textbf{PROTARCHUS}
\par   True.

\par \textbf{SOCRATES}
\par   And do not people who are in a fever, or any similar illness, feel cold or thirst or other bodily affections more intensely? Am I not right in saying that they have a deeper want and greater pleasure in the satisfaction of their want?

\par \textbf{PROTARCHUS}
\par   That is obvious as soon as it is said.

\par \textbf{SOCRATES}
\par   Well, then, shall we not be right in saying, that if a person would wish to see the greatest pleasures he ought to go and look, not at health, but at disease? And here you must distinguish: —do not imagine that I mean to ask whether those who are very ill have more pleasures than those who are well, but understand that I am speaking of the magnitude of pleasure; I want to know where pleasures are found to be most intense. For, as I say, we have to discover what is pleasure, and what they mean by pleasure who deny her very existence.

\par \textbf{PROTARCHUS}
\par   I think I follow you.

\par \textbf{SOCRATES}
\par   You will soon have a better opportunity of showing whether you do or not, Protarchus. Answer now, and tell me whether you see, I will not say more, but more intense and excessive pleasures in wantonness than in temperance? Reflect before you speak.

\par \textbf{PROTARCHUS}
\par   I understand you, and see that there is a great difference between them; the temperate are restrained by the wise man's aphorism of 'Never too much,' which is their rule, but excess of pleasure possessing the minds of fools and wantons becomes madness and makes them shout with delight.

\par \textbf{SOCRATES}
\par   Very good, and if this be true, then the greatest pleasures and pains will clearly be found in some vicious state of soul and body, and not in a virtuous state.

\par \textbf{PROTARCHUS}
\par   Certainly.

\par \textbf{SOCRATES}
\par   And ought we not to select some of these for examination, and see what makes them the greatest?

\par \textbf{PROTARCHUS}
\par   To be sure we ought.

\par \textbf{SOCRATES}
\par   Take the case of the pleasures which arise out of certain disorders.

\par \textbf{PROTARCHUS}
\par   What disorders?

\par \textbf{SOCRATES}
\par   The pleasures of unseemly disorders, which our severe friends utterly detest.

\par \textbf{PROTARCHUS}
\par   What pleasures?

\par \textbf{SOCRATES}
\par   Such, for example, as the relief of itching and other ailments by scratching, which is the only remedy required. For what in Heaven's name is the feeling to be called which is thus produced in us?—Pleasure or pain?

\par \textbf{PROTARCHUS}
\par   A villainous mixture of some kind, Socrates, I should say.

\par \textbf{SOCRATES}
\par   I did not introduce the argument, O Protarchus, with any personal reference to Philebus, but because, without the consideration of these and similar pleasures, we shall not be able to determine the point at issue.

\par \textbf{PROTARCHUS}
\par   Then we had better proceed to analyze this family of pleasures.

\par \textbf{SOCRATES}
\par   You mean the pleasures which are mingled with pain?

\par \textbf{PROTARCHUS}
\par   Exactly.

\par \textbf{SOCRATES}
\par   There are some mixtures which are of the body, and only in the body, and others which are of the soul, and only in the soul; while there are other mixtures of pleasures with pains, common both to soul and body, which in their composite state are called sometimes pleasures and sometimes pains.

\par \textbf{PROTARCHUS}
\par   How is that?

\par \textbf{SOCRATES}
\par   Whenever, in the restoration or in the derangement of nature, a man experiences two opposite feelings; for example, when he is cold and is growing warm, or again, when he is hot and is becoming cool, and he wants to have the one and be rid of the other;—the sweet has a bitter, as the common saying is, and both together fasten upon him and create irritation and in time drive him to distraction.

\par \textbf{PROTARCHUS}
\par   That description is very true to nature.

\par \textbf{SOCRATES}
\par   And in these sorts of mixtures the pleasures and pains are sometimes equal, and sometimes one or other of them predominates?

\par \textbf{PROTARCHUS}
\par   True.

\par \textbf{SOCRATES}
\par   Of cases in which the pain exceeds the pleasure, an example is afforded by itching, of which we were just now speaking, and by the tingling which we feel when the boiling and fiery element is within, and the rubbing and motion only relieves the surface, and does not reach the parts affected; then if you put them to the fire, and as a last resort apply cold to them, you may often produce the most intense pleasure or pain in the inner parts, which contrasts and mingles with the pain or pleasure, as the case may be, of the outer parts; and this is due to the forcible separation of what is united, or to the union of what is separated, and to the juxtaposition of pleasure and pain.

\par \textbf{PROTARCHUS}
\par   Quite so.

\par \textbf{SOCRATES}
\par   Sometimes the element of pleasure prevails in a man, and the slight undercurrent of pain makes him tingle, and causes a gentle irritation; or again, the excessive infusion of pleasure creates an excitement in him,—he even leaps for joy, he assumes all sorts of attitudes, he changes all manner of colours, he gasps for breath, and is quite amazed, and utters the most irrational exclamations.

\par \textbf{PROTARCHUS}
\par   Yes, indeed.

\par \textbf{SOCRATES}
\par   He will say of himself, and others will say of him, that he is dying with these delights; and the more dissipated and good-for-nothing he is, the more vehemently he pursues them in every way; of all pleasures he declares them to be the greatest; and he reckons him who lives in the most constant enjoyment of them to be the happiest of mankind.

\par \textbf{PROTARCHUS}
\par   That, Socrates, is a very true description of the opinions of the majority about pleasures.

\par \textbf{SOCRATES}
\par   Yes, Protarchus, quite true of the mixed pleasures, which arise out of the communion of external and internal sensations in the body; there are also cases in which the mind contributes an opposite element to the body, whether of pleasure or pain, and the two unite and form one mixture. Concerning these I have already remarked, that when a man is empty he desires to be full, and has pleasure in hope and pain in vacuity. But now I must further add what I omitted before, that in all these and similar emotions in which body and mind are opposed (and they are innumerable), pleasure and pain coalesce in one.

\par \textbf{PROTARCHUS}
\par   I believe that to be quite true.

\par \textbf{SOCRATES}
\par   There still remains one other sort of admixture of pleasures and pains.

\par \textbf{PROTARCHUS}
\par   What is that?

\par \textbf{SOCRATES}
\par   The union which, as we were saying, the mind often experiences of purely mental feelings.

\par \textbf{PROTARCHUS}
\par   What do you mean?

\par \textbf{SOCRATES}
\par   Why, do we not speak of anger, fear, desire, sorrow, love, emulation, envy, and the like, as pains which belong to the soul only?

\par \textbf{PROTARCHUS}
\par   Yes.

\par \textbf{SOCRATES}
\par   And shall we not find them also full of the most wonderful pleasures? need I remind you of the anger

\par  'Which stirs even a wise man to violence, And is sweeter than honey and the honeycomb?'

\par  And you remember how pleasures mingle with pains in lamentation and bereavement?

\par \textbf{PROTARCHUS}
\par   Yes, there is a natural connexion between them.

\par \textbf{SOCRATES}
\par   And you remember also how at the sight of tragedies the spectators smile through their tears?

\par \textbf{PROTARCHUS}
\par   Certainly I do.

\par \textbf{SOCRATES}
\par   And are you aware that even at a comedy the soul experiences a mixed feeling of pain and pleasure?

\par \textbf{PROTARCHUS}
\par   I do not quite understand you.

\par \textbf{SOCRATES}
\par   I admit, Protarchus, that there is some difficulty in recognizing this mixture of feelings at a comedy.

\par \textbf{PROTARCHUS}
\par   There is, I think.

\par \textbf{SOCRATES}
\par   And the greater the obscurity of the case the more desirable is the examination of it, because the difficulty in detecting other cases of mixed pleasures and pains will be less.

\par \textbf{PROTARCHUS}
\par   Proceed.

\par \textbf{SOCRATES}
\par   I have just mentioned envy; would you not call that a pain of the soul?

\par \textbf{PROTARCHUS}
\par   Yes.

\par \textbf{SOCRATES}
\par   And yet the envious man finds something in the misfortunes of his neighbours at which he is pleased?

\par \textbf{PROTARCHUS}
\par   Certainly.

\par \textbf{SOCRATES}
\par   And ignorance, and what is termed clownishness, are surely an evil?

\par \textbf{PROTARCHUS}
\par   To be sure.

\par \textbf{SOCRATES}
\par   From these considerations learn to know the nature of the ridiculous.

\par \textbf{PROTARCHUS}
\par   Explain.

\par \textbf{SOCRATES}
\par   The ridiculous is in short the specific name which is used to describe the vicious form of a certain habit; and of vice in general it is that kind which is most at variance with the inscription at Delphi.

\par \textbf{PROTARCHUS}
\par   You mean, Socrates, 'Know thyself.'

\par \textbf{SOCRATES}
\par   I do; and the opposite would be, 'Know not thyself.'

\par \textbf{PROTARCHUS}
\par   Certainly.

\par \textbf{SOCRATES}
\par   And now, O Protarchus, try to divide this into three.

\par \textbf{PROTARCHUS}
\par   Indeed I am afraid that I cannot.

\par \textbf{SOCRATES}
\par   Do you mean to say that I must make the division for you?

\par \textbf{PROTARCHUS}
\par   Yes, and what is more, I beg that you will.

\par \textbf{SOCRATES}
\par   Are there not three ways in which ignorance of self may be shown?

\par \textbf{PROTARCHUS}
\par   What are they?

\par \textbf{SOCRATES}
\par   In the first place, about money; the ignorant may fancy himself richer than he is.

\par \textbf{PROTARCHUS}
\par   Yes, that is a very common error.

\par \textbf{SOCRATES}
\par   And still more often he will fancy that he is taller or fairer than he is, or that he has some other advantage of person which he really has not.

\par \textbf{PROTARCHUS}
\par   Of course.

\par \textbf{SOCRATES}
\par   And yet surely by far the greatest number err about the goods of the mind; they imagine themselves to be much better men than they are.

\par \textbf{PROTARCHUS}
\par   Yes, that is by far the commonest delusion.

\par \textbf{SOCRATES}
\par   And of all the virtues, is not wisdom the one which the mass of mankind are always claiming, and which most arouses in them a spirit of contention and lying conceit of wisdom?

\par \textbf{PROTARCHUS}
\par   Certainly.

\par \textbf{SOCRATES}
\par   And may not all this be truly called an evil condition?

\par \textbf{PROTARCHUS}
\par   Very evil.

\par \textbf{SOCRATES}
\par   But we must pursue the division a step further, Protarchus, if we would see in envy of the childish sort a singular mixture of pleasure and pain.

\par \textbf{PROTARCHUS}
\par   How can we make the further division which you suggest?

\par \textbf{SOCRATES}
\par   All who are silly enough to entertain this lying conceit of themselves may of course be divided, like the rest of mankind, into two classes—one having power and might; and the other the reverse.

\par \textbf{PROTARCHUS}
\par   Certainly.

\par \textbf{SOCRATES}
\par   Let this, then, be the principle of division; those of them who are weak and unable to revenge themselves, when they are laughed at, may be truly called ridiculous, but those who can defend themselves may be more truly described as strong and formidable; for ignorance in the powerful is hateful and horrible, because hurtful to others both in reality and in fiction, but powerless ignorance may be reckoned, and in truth is, ridiculous.

\par \textbf{PROTARCHUS}
\par   That is very true, but I do not as yet see where is the admixture of pleasures and pains.

\par \textbf{SOCRATES}
\par   Well, then, let us examine the nature of envy.

\par \textbf{PROTARCHUS}
\par   Proceed.

\par \textbf{SOCRATES}
\par   Is not envy an unrighteous pleasure, and also an unrighteous pain?

\par \textbf{PROTARCHUS}
\par   Most true.

\par \textbf{SOCRATES}
\par   There is nothing envious or wrong in rejoicing at the misfortunes of enemies?

\par \textbf{PROTARCHUS}
\par   Certainly not.

\par \textbf{SOCRATES}
\par   But to feel joy instead of sorrow at the sight of our friends' misfortunes—is not that wrong?

\par \textbf{PROTARCHUS}
\par   Undoubtedly.

\par \textbf{SOCRATES}
\par   Did we not say that ignorance was always an evil?

\par \textbf{PROTARCHUS}
\par   True.

\par \textbf{SOCRATES}
\par   And the three kinds of vain conceit in our friends which we enumerated—the vain conceit of beauty, of wisdom, and of wealth, are ridiculous if they are weak, and detestable when they are powerful:  May we not say, as I was saying before, that our friends who are in this state of mind, when harmless to others, are simply ridiculous?

\par \textbf{PROTARCHUS}
\par   They are ridiculous.

\par \textbf{SOCRATES}
\par   And do we not acknowledge this ignorance of theirs to be a misfortune?

\par \textbf{PROTARCHUS}
\par   Certainly.

\par \textbf{SOCRATES}
\par   And do we feel pain or pleasure in laughing at it?

\par \textbf{PROTARCHUS}
\par   Clearly we feel pleasure.

\par \textbf{SOCRATES}
\par   And was not envy the source of this pleasure which we feel at the misfortunes of friends?

\par \textbf{PROTARCHUS}
\par   Certainly.

\par \textbf{SOCRATES}
\par   Then the argument shows that when we laugh at the folly of our friends, pleasure, in mingling with envy, mingles with pain, for envy has been acknowledged by us to be mental pain, and laughter is pleasant; and so we envy and laugh at the same instant.

\par \textbf{PROTARCHUS}
\par   True.

\par \textbf{SOCRATES}
\par   And the argument implies that there are combinations of pleasure and pain in lamentations, and in tragedy and comedy, not only on the stage, but on the greater stage of human life; and so in endless other cases.

\par \textbf{PROTARCHUS}
\par   I do not see how any one can deny what you say, Socrates, however eager he may be to assert the opposite opinion.

\par \textbf{SOCRATES}
\par   I mentioned anger, desire, sorrow, fear, love, emulation, envy, and similar emotions, as examples in which we should find a mixture of the two elements so often named; did I not?

\par \textbf{PROTARCHUS}
\par   Yes.

\par \textbf{SOCRATES}
\par   We may observe that our conclusions hitherto have had reference only to sorrow and envy and anger.

\par \textbf{PROTARCHUS}
\par   I see.

\par \textbf{SOCRATES}
\par   Then many other cases still remain?

\par \textbf{PROTARCHUS}
\par   Certainly.

\par \textbf{SOCRATES}
\par   And why do you suppose me to have pointed out to you the admixture which takes place in comedy? Why but to convince you that there was no difficulty in showing the mixed nature of fear and love and similar affections; and I thought that when I had given you the illustration, you would have let me off, and have acknowledged as a general truth that the body without the soul, and the soul without the body, as well as the two united, are susceptible of all sorts of admixtures of pleasures and pains; and so further discussion would have been unnecessary. And now I want to know whether I may depart; or will you keep me here until midnight? I fancy that I may obtain my release without many words;—if I promise that to-morrow I will give you an account of all these cases. But at present I would rather sail in another direction, and go to other matters which remain to be settled, before the judgment can be given which Philebus demands.

\par \textbf{PROTARCHUS}
\par   Very good, Socrates; in what remains take your own course.

\par \textbf{SOCRATES}
\par   Then after the mixed pleasures the unmixed should have their turn; this is the natural and necessary order.

\par \textbf{PROTARCHUS}
\par   Excellent.

\par \textbf{SOCRATES}
\par   These, in turn, then, I will now endeavour to indicate; for with the maintainers of the opinion that all pleasures are a cessation of pain, I do not agree, but, as I was saying, I use them as witnesses, that there are pleasures which seem only and are not, and there are others again which have great power and appear in many forms, yet are intermingled with pains, and are partly alleviations of agony and distress, both of body and mind.

\par \textbf{PROTARCHUS}
\par   Then what pleasures, Socrates, should we be right in conceiving to be true?

\par \textbf{SOCRATES}
\par   True pleasures are those which are given by beauty of colour and form, and most of those which arise from smells; those of sound, again, and in general those of which the want is painless and unconscious, and of which the fruition is palpable to sense and pleasant and unalloyed with pain.

\par \textbf{PROTARCHUS}
\par   Once more, Socrates, I must ask what you mean.

\par \textbf{SOCRATES}
\par   My meaning is certainly not obvious, and I will endeavour to be plainer. I do not mean by beauty of form such beauty as that of animals or pictures, which the many would suppose to be my meaning; but, says the argument, understand me to mean straight lines and circles, and the plane or solid figures which are formed out of them by turning-lathes and rulers and measurers of angles; for these I affirm to be not only relatively beautiful, like other things, but they are eternally and absolutely beautiful, and they have peculiar pleasures, quite unlike the pleasures of scratching. And there are colours which are of the same character, and have similar pleasures; now do you understand my meaning?

\par \textbf{PROTARCHUS}
\par   I am trying to understand, Socrates, and I hope that you will try to make your meaning clearer.

\par \textbf{SOCRATES}
\par   When sounds are smooth and clear, and have a single pure tone, then I mean to say that they are not relatively but absolutely beautiful, and have natural pleasures associated with them.

\par \textbf{PROTARCHUS}
\par   Yes, there are such pleasures.

\par \textbf{SOCRATES}
\par   The pleasures of smell are of a less ethereal sort, but they have no necessary admixture of pain; and all pleasures, however and wherever experienced, which are unattended by pains, I assign to an analogous class. Here then are two kinds of pleasures.

\par \textbf{PROTARCHUS}
\par   I understand.

\par \textbf{SOCRATES}
\par   To these may be added the pleasures of knowledge, if no hunger of knowledge and no pain caused by such hunger precede them.

\par \textbf{PROTARCHUS}
\par   And this is the case.

\par \textbf{SOCRATES}
\par   Well, but if a man who is full of knowledge loses his knowledge, are there not pains of forgetting?

\par \textbf{PROTARCHUS}
\par   Not necessarily, but there may be times of reflection, when he feels grief at the loss of his knowledge.

\par \textbf{SOCRATES}
\par   Yes, my friend, but at present we are enumerating only the natural perceptions, and have nothing to do with reflection.

\par \textbf{PROTARCHUS}
\par   In that case you are right in saying that the loss of knowledge is not attended with pain.

\par \textbf{SOCRATES}
\par   These pleasures of knowledge, then, are unmixed with pain; and they are not the pleasures of the many but of a very few.

\par \textbf{PROTARCHUS}
\par   Quite true.

\par \textbf{SOCRATES}
\par   And now, having fairly separated the pure pleasures and those which may be rightly termed impure, let us further add to our description of them, that the pleasures which are in excess have no measure, but that those which are not in excess have measure; the great, the excessive, whether more or less frequent, we shall be right in referring to the class of the infinite, and of the more and less, which pours through body and soul alike; and the others we shall refer to the class which has measure.

\par \textbf{PROTARCHUS}
\par   Quite right, Socrates.

\par \textbf{SOCRATES}
\par   Still there is something more to be considered about pleasures.

\par \textbf{PROTARCHUS}
\par   What is it?

\par \textbf{SOCRATES}
\par   When you speak of purity and clearness, or of excess, abundance, greatness and sufficiency, in what relation do these terms stand to truth?

\par \textbf{PROTARCHUS}
\par   Why do you ask, Socrates?

\par \textbf{SOCRATES}
\par   Because, Protarchus, I should wish to test pleasure and knowledge in every possible way, in order that if there be a pure and impure element in either of them, I may present the pure element for judgment, and then they will be more easily judged of by you and by me and by all of us.

\par \textbf{PROTARCHUS}
\par   Most true.

\par \textbf{SOCRATES}
\par   Let us investigate all the pure kinds; first selecting for consideration a single instance.

\par \textbf{PROTARCHUS}
\par   What instance shall we select?

\par \textbf{SOCRATES}
\par   Suppose that we first of all take whiteness.

\par \textbf{PROTARCHUS}
\par   Very good.

\par \textbf{SOCRATES}
\par   How can there be purity in whiteness, and what purity? Is that purest which is greatest or most in quantity, or that which is most unadulterated and freest from any admixture of other colours?

\par \textbf{PROTARCHUS}
\par   Clearly that which is most unadulterated.

\par \textbf{SOCRATES}
\par   True, Protarchus; and so the purest white, and not the greatest or largest in quantity, is to be deemed truest and most beautiful?

\par \textbf{PROTARCHUS}
\par   Right.

\par \textbf{SOCRATES}
\par   And we shall be quite right in saying that a little pure white is whiter and fairer and truer than a great deal that is mixed.

\par \textbf{PROTARCHUS}
\par   Perfectly right.

\par \textbf{SOCRATES}
\par   There is no need of adducing many similar examples in illustration of the argument about pleasure; one such is sufficient to prove to us that a small pleasure or a small amount of pleasure, if pure or unalloyed with pain, is always pleasanter and truer and fairer than a great pleasure or a great amount of pleasure of another kind.

\par \textbf{PROTARCHUS}
\par   Assuredly; and the instance you have given is quite sufficient.

\par \textbf{SOCRATES}
\par   But what do you say of another question: —have we not heard that pleasure is always a generation, and has no true being? Do not certain ingenious philosophers teach this doctrine, and ought not we to be grateful to them?

\par \textbf{PROTARCHUS}
\par   What do they mean?

\par \textbf{SOCRATES}
\par   I will explain to you, my dear Protarchus, what they mean, by putting a question.

\par \textbf{PROTARCHUS}
\par   Ask, and I will answer.

\par \textbf{SOCRATES}
\par   I assume that there are two natures, one self-existent, and the other ever in want of something.

\par \textbf{PROTARCHUS}
\par   What manner of natures are they?

\par \textbf{SOCRATES}
\par   The one majestic ever, the other inferior.

\par \textbf{PROTARCHUS}
\par   You speak riddles.

\par \textbf{SOCRATES}
\par   You have seen loves good and fair, and also brave lovers of them.

\par \textbf{PROTARCHUS}
\par   I should think so.

\par \textbf{SOCRATES}
\par   Search the universe for two terms which are like these two and are present everywhere.

\par \textbf{PROTARCHUS}
\par   Yet a third time I must say, Be a little plainer, Socrates.

\par \textbf{SOCRATES}
\par   There is no difficulty, Protarchus; the argument is only in play, and insinuates that some things are for the sake of something else (relatives), and that other things are the ends to which the former class subserve (absolutes).

\par \textbf{PROTARCHUS}
\par   Your many repetitions make me slow to understand.

\par \textbf{SOCRATES}
\par   As the argument proceeds, my boy, I dare say that the meaning will become clearer.

\par \textbf{PROTARCHUS}
\par   Very likely.

\par \textbf{SOCRATES}
\par   Here are two new principles.

\par \textbf{PROTARCHUS}
\par   What are they?

\par \textbf{SOCRATES}
\par   One is the generation of all things, and the other is essence.

\par \textbf{PROTARCHUS}
\par   I readily accept from you both generation and essence.

\par \textbf{SOCRATES}
\par   Very right; and would you say that generation is for the sake of essence, or essence for the sake of generation?

\par \textbf{PROTARCHUS}
\par   You want to know whether that which is called essence is, properly speaking, for the sake of generation?

\par \textbf{SOCRATES}
\par   Yes.

\par \textbf{PROTARCHUS}
\par   By the gods, I wish that you would repeat your question.

\par \textbf{SOCRATES}
\par   I mean, O my Protarchus, to ask whether you would tell me that ship-building is for the sake of ships, or ships for the sake of ship-building? and in all similar cases I should ask the same question.

\par \textbf{PROTARCHUS}
\par   Why do you not answer yourself, Socrates?

\par \textbf{SOCRATES}
\par   I have no objection, but you must take your part.

\par \textbf{PROTARCHUS}
\par   Certainly.

\par \textbf{SOCRATES}
\par   My answer is, that all things instrumental, remedial, material, are given to us with a view to generation, and that each generation is relative to, or for the sake of, some being or essence, and that the whole of generation is relative to the whole of essence.

\par \textbf{PROTARCHUS}
\par   Assuredly.

\par \textbf{SOCRATES}
\par   Then pleasure, being a generation, must surely be for the sake of some essence?

\par \textbf{PROTARCHUS}
\par   True.

\par \textbf{SOCRATES}
\par   And that for the sake of which something else is done must be placed in the class of good, and that which is done for the sake of something else, in some other class, my good friend.

\par \textbf{PROTARCHUS}
\par   Most certainly.

\par \textbf{SOCRATES}
\par   Then pleasure, being a generation, will be rightly placed in some other class than that of good?

\par \textbf{PROTARCHUS}
\par   Quite right.

\par \textbf{SOCRATES}
\par   Then, as I said at first, we ought to be very grateful to him who first pointed out that pleasure was a generation only, and had no true being at all; for he is clearly one who laughs at the notion of pleasure being a good.

\par \textbf{PROTARCHUS}
\par   Assuredly.

\par \textbf{SOCRATES}
\par   And he would surely laugh also at those who make generation their highest end.

\par \textbf{PROTARCHUS}
\par   Of whom are you speaking, and what do they mean?

\par \textbf{SOCRATES}
\par   I am speaking of those who when they are cured of hunger or thirst or any other defect by some process of generation are delighted at the process as if it were pleasure; and they say that they would not wish to live without these and other feelings of a like kind which might be mentioned.

\par \textbf{PROTARCHUS}
\par   That is certainly what they appear to think.

\par \textbf{SOCRATES}
\par   And is not destruction universally admitted to be the opposite of generation?

\par \textbf{PROTARCHUS}
\par   Certainly.

\par \textbf{SOCRATES}
\par   Then he who chooses thus, would choose generation and destruction rather than that third sort of life, in which, as we were saying, was neither pleasure nor pain, but only the purest possible thought.

\par \textbf{PROTARCHUS}
\par   He who would make us believe pleasure to be a good is involved in great absurdities, Socrates.

\par \textbf{SOCRATES}
\par   Great, indeed; and there is yet another of them.

\par \textbf{PROTARCHUS}
\par   What is it?

\par \textbf{SOCRATES}
\par   Is there not an absurdity in arguing that there is nothing good or noble in the body, or in anything else, but that good is in the soul only, and that the only good of the soul is pleasure; and that courage or temperance or understanding, or any other good of the soul, is not really a good?—and is there not yet a further absurdity in our being compelled to say that he who has a feeling of pain and not of pleasure is bad at the time when he is suffering pain, even though he be the best of men; and again, that he who has a feeling of pleasure, in so far as he is pleased at the time when he is pleased, in that degree excels in virtue?

\par \textbf{PROTARCHUS}
\par   Nothing, Socrates, can be more irrational than all this.

\par \textbf{SOCRATES}
\par   And now, having subjected pleasure to every sort of test, let us not appear to be too sparing of mind and knowledge:  let us ring their metal bravely, and see if there be unsoundness in any part, until we have found out what in them is of the purest nature; and then the truest elements both of pleasure and knowledge may be brought up for judgment.

\par \textbf{PROTARCHUS}
\par   Right.

\par \textbf{SOCRATES}
\par   Knowledge has two parts,—the one productive, and the other educational?

\par \textbf{PROTARCHUS}
\par   True.

\par \textbf{SOCRATES}
\par   And in the productive or handicraft arts, is not one part more akin to knowledge, and the other less; and may not the one part be regarded as the pure, and the other as the impure?

\par \textbf{PROTARCHUS}
\par   Certainly.

\par \textbf{SOCRATES}
\par   Let us separate the superior or dominant elements in each of them.

\par \textbf{PROTARCHUS}
\par   What are they, and how do you separate them?

\par \textbf{SOCRATES}
\par   I mean to say, that if arithmetic, mensuration, and weighing be taken away from any art, that which remains will not be much.

\par \textbf{PROTARCHUS}
\par   Not much, certainly.

\par \textbf{SOCRATES}
\par   The rest will be only conjecture, and the better use of the senses which is given by experience and practice, in addition to a certain power of guessing, which is commonly called art, and is perfected by attention and pains.

\par \textbf{PROTARCHUS}
\par   Nothing more, assuredly.

\par \textbf{SOCRATES}
\par   Music, for instance, is full of this empiricism; for sounds are harmonized, not by measure, but by skilful conjecture; the music of the flute is always trying to guess the pitch of each vibrating note, and is therefore mixed up with much that is doubtful and has little which is certain.

\par \textbf{PROTARCHUS}
\par   Most true.

\par \textbf{SOCRATES}
\par   And the same will be found to hold good of medicine and husbandry and piloting and generalship.

\par \textbf{PROTARCHUS}
\par   Very true.

\par \textbf{SOCRATES}
\par   The art of the builder, on the other hand, which uses a number of measures and instruments, attains by their help to a greater degree of accuracy than the other arts.

\par \textbf{PROTARCHUS}
\par   How is that?

\par \textbf{SOCRATES}
\par   In ship-building and house-building, and in other branches of the art of carpentering, the builder has his rule, lathe, compass, line, and a most ingenious machine for straightening wood.

\par \textbf{PROTARCHUS}
\par   Very true, Socrates.

\par \textbf{SOCRATES}
\par   Then now let us divide the arts of which we were speaking into two kinds,—the arts which, like music, are less exact in their results, and those which, like carpentering, are more exact.

\par \textbf{PROTARCHUS}
\par   Let us make that division.

\par \textbf{SOCRATES}
\par   Of the latter class, the most exact of all are those which we just now spoke of as primary.

\par \textbf{PROTARCHUS}
\par   I see that you mean arithmetic, and the kindred arts of weighing and measuring.

\par \textbf{SOCRATES}
\par   Certainly, Protarchus; but are not these also distinguishable into two kinds?

\par \textbf{PROTARCHUS}
\par   What are the two kinds?

\par \textbf{SOCRATES}
\par   In the first place, arithmetic is of two kinds, one of which is popular, and the other philosophical.

\par \textbf{PROTARCHUS}
\par   How would you distinguish them?

\par \textbf{SOCRATES}
\par   There is a wide difference between them, Protarchus; some arithmeticians reckon unequal units; as for example, two armies, two oxen, two very large things or two very small things. The party who are opposed to them insist that every unit in ten thousand must be the same as every other unit.

\par \textbf{PROTARCHUS}
\par   Undoubtedly there is, as you say, a great difference among the votaries of the science; and there may be reasonably supposed to be two sorts of arithmetic.

\par \textbf{SOCRATES}
\par   And when we compare the art of mensuration which is used in building with philosophical geometry, or the art of computation which is used in trading with exact calculation, shall we say of either of the pairs that it is one or two?

\par \textbf{PROTARCHUS}
\par   On the analogy of what has preceded, I should be of opinion that they were severally two.

\par \textbf{SOCRATES}
\par   Right; but do you understand why I have discussed the subject?

\par \textbf{PROTARCHUS}
\par   I think so, but I should like to be told by you.

\par \textbf{SOCRATES}
\par   The argument has all along been seeking a parallel to pleasure, and true to that original design, has gone on to ask whether one sort of knowledge is purer than another, as one pleasure is purer than another.

\par \textbf{PROTARCHUS}
\par   Clearly; that was the intention.

\par \textbf{SOCRATES}
\par   And has not the argument in what has preceded, already shown that the arts have different provinces, and vary in their degrees of certainty?

\par \textbf{PROTARCHUS}
\par   Very true.

\par \textbf{SOCRATES}
\par   And just now did not the argument first designate a particular art by a common term, thus making us believe in the unity of that art; and then again, as if speaking of two different things, proceed to enquire whether the art as pursed by philosophers, or as pursued by non-philosophers, has more of certainty and purity?

\par \textbf{PROTARCHUS}
\par   That is the very question which the argument is asking.

\par \textbf{SOCRATES}
\par   And how, Protarchus, shall we answer the enquiry?

\par \textbf{PROTARCHUS}
\par   O Socrates, we have reached a point at which the difference of clearness in different kinds of knowledge is enormous.

\par \textbf{SOCRATES}
\par   Then the answer will be the easier.

\par \textbf{PROTARCHUS}
\par   Certainly; and let us say in reply, that those arts into which arithmetic and mensuration enter, far surpass all others; and that of these the arts or sciences which are animated by the pure philosophic impulse are infinitely superior in accuracy and truth.

\par \textbf{SOCRATES}
\par   Then this is your judgment; and this is the answer which, upon your authority, we will give to all masters of the art of misinterpretation?

\par \textbf{PROTARCHUS}
\par   What answer?

\par \textbf{SOCRATES}
\par   That there are two arts of arithmetic, and two of mensuration; and also several other arts which in like manner have this double nature, and yet only one name.

\par \textbf{PROTARCHUS}
\par   Let us boldly return this answer to the masters of whom you speak, Socrates, and hope for good luck.

\par \textbf{SOCRATES}
\par   We have explained what we term the most exact arts or sciences.

\par \textbf{PROTARCHUS}
\par   Very good.

\par \textbf{SOCRATES}
\par   And yet, Protarchus, dialectic will refuse to acknowledge us, if we do not award to her the first place.

\par \textbf{PROTARCHUS}
\par   And pray, what is dialectic?

\par \textbf{SOCRATES}
\par   Clearly the science which has to do with all that knowledge of which we are now speaking; for I am sure that all men who have a grain of intelligence will admit that the knowledge which has to do with being and reality, and sameness and unchangeableness, is by far the truest of all. But how would you decide this question, Protarchus?

\par \textbf{PROTARCHUS}
\par   I have often heard Gorgias maintain, Socrates, that the art of persuasion far surpassed every other; this, as he says, is by far the best of them all, for to it all things submit, not by compulsion, but of their own free will. Now, I should not like to quarrel either with you or with him.

\par \textbf{SOCRATES}
\par   You mean to say that you would like to desert, if you were not ashamed?

\par \textbf{PROTARCHUS}
\par   As you please.

\par \textbf{SOCRATES}
\par   May I not have led you into a misapprehension?

\par \textbf{PROTARCHUS}
\par   How?

\par \textbf{SOCRATES}
\par   Dear Protarchus, I never asked which was the greatest or best or usefullest of arts or sciences, but which had clearness and accuracy, and the greatest amount of truth, however humble and little useful an art. And as for Gorgias, if you do not deny that his art has the advantage in usefulness to mankind, he will not quarrel with you for saying that the study of which I am speaking is superior in this particular of essential truth; as in the comparison of white colours, a little whiteness, if that little be only pure, was said to be superior in truth to a great mass which is impure. And now let us give our best attention and consider well, not the comparative use or reputation of the sciences, but the power or faculty, if there be such, which the soul has of loving the truth, and of doing all things for the sake of it; let us search into the pure element of mind and intelligence, and then we shall be able to say whether the science of which I have been speaking is most likely to possess the faculty, or whether there be some other which has higher claims.

\par \textbf{PROTARCHUS}
\par   Well, I have been considering, and I can hardly think that any other science or art has a firmer grasp of the truth than this.

\par \textbf{SOCRATES}
\par   Do you say so because you observe that the arts in general and those engaged in them make use of opinion, and are resolutely engaged in the investigation of matters of opinion? Even he who supposes himself to be occupied with nature is really occupied with the things of this world, how created, how acting or acted upon. Is not this the sort of enquiry in which his life is spent?

\par \textbf{PROTARCHUS}
\par   True.

\par \textbf{SOCRATES}
\par   He is labouring, not after eternal being, but about things which are becoming, or which will or have become.

\par \textbf{PROTARCHUS}
\par   Very true.

\par \textbf{SOCRATES}
\par   And can we say that any of these things which neither are nor have been nor will be unchangeable, when judged by the strict rule of truth ever become certain?

\par \textbf{PROTARCHUS}
\par   Impossible.

\par \textbf{SOCRATES}
\par   How can anything fixed be concerned with that which has no fixedness?

\par \textbf{PROTARCHUS}
\par   How indeed?

\par \textbf{SOCRATES}
\par   Then mind and science when employed about such changing things do not attain the highest truth?

\par \textbf{PROTARCHUS}
\par   I should imagine not.

\par \textbf{SOCRATES}
\par   And now let us bid farewell, a long farewell, to you or me or Philebus or Gorgias, and urge on behalf of the argument a single point.

\par \textbf{PROTARCHUS}
\par   What point?

\par \textbf{SOCRATES}
\par   Let us say that the stable and pure and true and unalloyed has to do with the things which are eternal and unchangeable and unmixed, or if not, at any rate what is most akin to them has; and that all other things are to be placed in a second or inferior class.

\par \textbf{PROTARCHUS}
\par   Very true.

\par \textbf{SOCRATES}
\par   And of the names expressing cognition, ought not the fairest to be given to the fairest things?

\par \textbf{PROTARCHUS}
\par   That is natural.

\par \textbf{SOCRATES}
\par   And are not mind and wisdom the names which are to be honoured most?

\par \textbf{PROTARCHUS}
\par   Yes.

\par \textbf{SOCRATES}
\par   And these names may be said to have their truest and most exact application when the mind is engaged in the contemplation of true being?

\par \textbf{PROTARCHUS}
\par   Certainly.

\par \textbf{SOCRATES}
\par   And these were the names which I adduced of the rivals of pleasure?

\par \textbf{PROTARCHUS}
\par   Very true, Socrates.

\par \textbf{SOCRATES}
\par   In the next place, as to the mixture, here are the ingredients, pleasure and wisdom, and we may be compared to artists who have their materials ready to their hands.

\par \textbf{PROTARCHUS}
\par   Yes.

\par \textbf{SOCRATES}
\par   And now we must begin to mix them?

\par \textbf{PROTARCHUS}
\par   By all means.

\par \textbf{SOCRATES}
\par   But had we not better have a preliminary word and refresh our memories?

\par \textbf{PROTARCHUS}
\par   Of what?

\par \textbf{SOCRATES}
\par   Of that which I have already mentioned. Well says the proverb, that we ought to repeat twice and even thrice that which is good.

\par \textbf{PROTARCHUS}
\par   Certainly.

\par \textbf{SOCRATES}
\par   Well then, by Zeus, let us proceed, and I will make what I believe to be a fair summary of the argument.

\par \textbf{PROTARCHUS}
\par   Let me hear.

\par \textbf{SOCRATES}
\par   Philebus says that pleasure is the true end of all living beings, at which all ought to aim, and moreover that it is the chief good of all, and that the two names 'good' and 'pleasant' are correctly given to one thing and one nature; Socrates, on the other hand, begins by denying this, and further says, that in nature as in name they are two, and that wisdom partakes more than pleasure of the good. Is not and was not this what we were saying, Protarchus?

\par \textbf{PROTARCHUS}
\par   Certainly.

\par \textbf{SOCRATES}
\par   And is there not and was there not a further point which was conceded between us?

\par \textbf{PROTARCHUS}
\par   What was it?

\par \textbf{SOCRATES}
\par   That the good differs from all other things.

\par \textbf{PROTARCHUS}
\par   In what respect?

\par \textbf{SOCRATES}
\par   In that the being who possesses good always everywhere and in all things has the most perfect sufficiency, and is never in need of anything else.

\par \textbf{PROTARCHUS}
\par   Exactly.

\par \textbf{SOCRATES}
\par   And did we not endeavour to make an imaginary separation of wisdom and pleasure, assigning to each a distinct life, so that pleasure was wholly excluded from wisdom, and wisdom in like manner had no part whatever in pleasure?

\par \textbf{PROTARCHUS}
\par   We did.

\par \textbf{SOCRATES}
\par   And did we think that either of them alone would be sufficient?

\par \textbf{PROTARCHUS}
\par   Certainly not.

\par \textbf{SOCRATES}
\par   And if we erred in any point, then let any one who will, take up the enquiry again and set us right; and assuming memory and wisdom and knowledge and true opinion to belong to the same class, let him consider whether he would desire to possess or acquire,—I will not say pleasure, however abundant or intense, if he has no real perception that he is pleased, nor any consciousness of what he feels, nor any recollection, however momentary, of the feeling,—but would he desire to have anything at all, if these faculties were wanting to him? And about wisdom I ask the same question; can you conceive that any one would choose to have all wisdom absolutely devoid of pleasure, rather than with a certain degree of pleasure, or all pleasure devoid of wisdom, rather than with a certain degree of wisdom?

\par \textbf{PROTARCHUS}
\par   Certainly not, Socrates; but why repeat such questions any more?

\par \textbf{SOCRATES}
\par   Then the perfect and universally eligible and entirely good cannot possibly be either of them?

\par \textbf{PROTARCHUS}
\par   Impossible.

\par \textbf{SOCRATES}
\par   Then now we must ascertain the nature of the good more or less accurately, in order, as we were saying, that the second place may be duly assigned.

\par \textbf{PROTARCHUS}
\par   Right.

\par \textbf{SOCRATES}
\par   Have we not found a road which leads towards the good?

\par \textbf{PROTARCHUS}
\par   What road?

\par \textbf{SOCRATES}
\par   Supposing that a man had to be found, and you could discover in what house he lived, would not that be a great step towards the discovery of the man himself?

\par \textbf{PROTARCHUS}
\par   Certainly.

\par \textbf{SOCRATES}
\par   And now reason intimates to us, as at our first beginning, that we should seek the good, not in the unmixed life but in the mixed.

\par \textbf{PROTARCHUS}
\par   True.

\par \textbf{SOCRATES}
\par   There is greater hope of finding that which we are seeking in the life which is well mixed than in that which is not?

\par \textbf{PROTARCHUS}
\par   Far greater.

\par \textbf{SOCRATES}
\par   Then now let us mingle, Protarchus, at the same time offering up a prayer to Dionysus or Hephaestus, or whoever is the god who presides over the ceremony of mingling.

\par \textbf{PROTARCHUS}
\par   By all means.

\par \textbf{SOCRATES}
\par   Are not we the cup-bearers? and here are two fountains which are flowing at our side:  one, which is pleasure, may be likened to a fountain of honey; the other, wisdom, a sober draught in which no wine mingles, is of water unpleasant but healthful; out of these we must seek to make the fairest of all possible mixtures.

\par \textbf{PROTARCHUS}
\par   Certainly.

\par \textbf{SOCRATES}
\par   Tell me first;—should we be most likely to succeed if we mingled every sort of pleasure with every sort of wisdom?

\par \textbf{PROTARCHUS}
\par   Perhaps we might.

\par \textbf{SOCRATES}
\par   But I should be afraid of the risk, and I think that I can show a safer plan.

\par \textbf{PROTARCHUS}
\par   What is it?

\par \textbf{SOCRATES}
\par   One pleasure was supposed by us to be truer than another, and one art to be more exact than another.

\par \textbf{PROTARCHUS}
\par   Certainly.

\par \textbf{SOCRATES}
\par   There was also supposed to be a difference in sciences; some of them regarding only the transient and perishing, and others the permanent and imperishable and everlasting and immutable; and when judged by the standard of truth, the latter, as we thought, were truer than the former.

\par \textbf{PROTARCHUS}
\par   Very good and right.

\par \textbf{SOCRATES}
\par   If, then, we were to begin by mingling the sections of each class which have the most of truth, will not the union suffice to give us the loveliest of lives, or shall we still want some elements of another kind?

\par \textbf{PROTARCHUS}
\par   I think that we ought to do what you suggest.

\par \textbf{SOCRATES}
\par   Let us suppose a man who understands justice, and has reason as well as understanding about the true nature of this and of all other things.

\par \textbf{PROTARCHUS}
\par   We will suppose such a man.

\par \textbf{SOCRATES}
\par   Will he have enough of knowledge if he is acquainted only with the divine circle and sphere, and knows nothing of our human spheres and circles, but uses only divine circles and measures in the building of a house?

\par \textbf{PROTARCHUS}
\par   The knowledge which is only superhuman, Socrates, is ridiculous in man.

\par \textbf{SOCRATES}
\par   What do you mean? Do you mean that you are to throw into the cup and mingle the impure and uncertain art which uses the false measure and the false circle?

\par \textbf{PROTARCHUS}
\par   Yes, we must, if any of us is ever to find his way home.

\par \textbf{SOCRATES}
\par   And am I to include music, which, as I was saying just now, is full of guesswork and imitation, and is wanting in purity?

\par \textbf{PROTARCHUS}
\par   Yes, I think that you must, if human life is to be a life at all.

\par \textbf{SOCRATES}
\par   Well, then, suppose that I give way, and, like a doorkeeper who is pushed and overborne by the mob, I open the door wide, and let knowledge of every sort stream in, and the pure mingle with the impure?

\par \textbf{PROTARCHUS}
\par   I do not know, Socrates, that any great harm would come of having them all, if only you have the first sort.

\par \textbf{SOCRATES}
\par   Well, then, shall I let them all flow into what Homer poetically terms 'a meeting of the waters'?

\par \textbf{PROTARCHUS}
\par   By all means.

\par \textbf{SOCRATES}
\par   There—I have let them in, and now I must return to the fountain of pleasure. For we were not permitted to begin by mingling in a single stream the true portions of both according to our original intention; but the love of all knowledge constrained us to let all the sciences flow in together before the pleasures.

\par \textbf{PROTARCHUS}
\par   Quite true.

\par \textbf{SOCRATES}
\par   And now the time has come for us to consider about the pleasures also, whether we shall in like manner let them go all at once, or at first only the true ones.

\par \textbf{PROTARCHUS}
\par   It will be by far the safer course to let flow the true ones first.

\par \textbf{SOCRATES}
\par   Let them flow, then; and now, if there are any necessary pleasures, as there were arts and sciences necessary, must we not mingle them?

\par \textbf{PROTARCHUS}
\par   Yes; the necessary pleasures should certainly be allowed to mingle.

\par \textbf{SOCRATES}
\par   The knowledge of the arts has been admitted to be innocent and useful always; and if we say of pleasures in like manner that all of them are good and innocent for all of us at all times, we must let them all mingle?

\par \textbf{PROTARCHUS}
\par   What shall we say about them, and what course shall we take?

\par \textbf{SOCRATES}
\par   Do not ask me, Protarchus; but ask the daughters of pleasure and wisdom to answer for themselves.

\par \textbf{PROTARCHUS}
\par   How?

\par \textbf{SOCRATES}
\par   Tell us, O beloved—shall we call you pleasures or by some other name?—would you rather live with or without wisdom? I am of opinion that they would certainly answer as follows:

\par \textbf{PROTARCHUS}
\par   How?

\par \textbf{SOCRATES}
\par   They would answer, as we said before, that for any single class to be left by itself pure and isolated is not good, nor altogether possible; and that if we are to make comparisons of one class with another and choose, there is no better companion than knowledge of things in general, and likewise the perfect knowledge, if that may be, of ourselves in every respect.

\par \textbf{PROTARCHUS}
\par   And our answer will be: —In that ye have spoken well.

\par \textbf{SOCRATES}
\par   Very true. And now let us go back and interrogate wisdom and mind:  Would you like to have any pleasures in the mixture? And they will reply: —'What pleasures do you mean?'

\par \textbf{PROTARCHUS}
\par   Likely enough.

\par \textbf{SOCRATES}
\par   And we shall take up our parable and say:  Do you wish to have the greatest and most vehement pleasures for your companions in addition to the true ones? 'Why, Socrates,' they will say, 'how can we? seeing that they are the source of ten thousand hindrances to us; they trouble the souls of men, which are our habitation, with their madness; they prevent us from coming to the birth, and are commonly the ruin of the children which are born to us, causing them to be forgotten and unheeded; but the true and pure pleasures, of which you spoke, know to be of our family, and also those pleasures which accompany health and temperance, and which every Virtue, like a goddess, has in her train to follow her about wherever she goes,—mingle these and not the others; there would be great want of sense in any one who desires to see a fair and perfect mixture, and to find in it what is the highest good in man and in the universe, and to divine what is the true form of good—there would be great want of sense in his allowing the pleasures, which are always in the company of folly and vice, to mingle with mind in the cup. '—Is not this a very rational and suitable reply, which mind has made, both on her own behalf, as well as on the behalf of memory and true opinion?

\par \textbf{PROTARCHUS}
\par   Most certainly.

\par \textbf{SOCRATES}
\par   And still there must be something more added, which is a necessary ingredient in every mixture.

\par \textbf{PROTARCHUS}
\par   What is that?

\par \textbf{SOCRATES}
\par   Unless truth enter into the composition, nothing can truly be created or subsist.

\par \textbf{PROTARCHUS}
\par   Impossible.

\par \textbf{SOCRATES}
\par   Quite impossible; and now you and Philebus must tell me whether anything is still wanting in the mixture, for to my way of thinking the argument is now completed, and may be compared to an incorporeal law, which is going to hold fair rule over a living body.

\par \textbf{PROTARCHUS}
\par   I agree with you, Socrates.

\par \textbf{SOCRATES}
\par   And may we not say with reason that we are now at the vestibule of the habitation of the good?

\par \textbf{PROTARCHUS}
\par   I think that we are.

\par \textbf{SOCRATES}
\par   What, then, is there in the mixture which is most precious, and which is the principal cause why such a state is universally beloved by all? When we have discovered it, we will proceed to ask whether this omnipresent nature is more akin to pleasure or to mind.

\par \textbf{PROTARCHUS}
\par   Quite right; in that way we shall be better able to judge.

\par \textbf{SOCRATES}
\par   And there is no difficulty in seeing the cause which renders any mixture either of the highest value or of none at all.

\par \textbf{PROTARCHUS}
\par   What do you mean?

\par \textbf{SOCRATES}
\par   Every man knows it.

\par \textbf{PROTARCHUS}
\par   What?

\par \textbf{SOCRATES}
\par   He knows that any want of measure and symmetry in any mixture whatever must always of necessity be fatal, both to the elements and to the mixture, which is then not a mixture, but only a confused medley which brings confusion on the possessor of it.

\par \textbf{PROTARCHUS}
\par   Most true.

\par \textbf{SOCRATES}
\par   And now the power of the good has retired into the region of the beautiful; for measure and symmetry are beauty and virtue all the world over.

\par \textbf{PROTARCHUS}
\par   True.

\par \textbf{SOCRATES}
\par   Also we said that truth was to form an element in the mixture.

\par \textbf{PROTARCHUS}
\par   Certainly.

\par \textbf{SOCRATES}
\par   Then, if we are not able to hunt the good with one idea only, with three we may catch our prey; Beauty, Symmetry, Truth are the three, and these taken together we may regard as the single cause of the mixture, and the mixture as being good by reason of the infusion of them.

\par \textbf{PROTARCHUS}
\par   Quite right.

\par \textbf{SOCRATES}
\par   And now, Protarchus, any man could decide well enough whether pleasure or wisdom is more akin to the highest good, and more honourable among gods and men.

\par \textbf{PROTARCHUS}
\par   Clearly, and yet perhaps the argument had better be pursued to the end.

\par \textbf{SOCRATES}
\par   We must take each of them separately in their relation to pleasure and mind, and pronounce upon them; for we ought to see to which of the two they are severally most akin.

\par \textbf{PROTARCHUS}
\par   You are speaking of beauty, truth, and measure?

\par \textbf{SOCRATES}
\par   Yes, Protarchus, take truth first, and, after passing in review mind, truth, pleasure, pause awhile and make answer to yourself—as to whether pleasure or mind is more akin to truth.

\par \textbf{PROTARCHUS}
\par   There is no need to pause, for the difference between them is palpable; pleasure is the veriest impostor in the world; and it is said that in the pleasures of love, which appear to be the greatest, perjury is excused by the gods; for pleasures, like children, have not the least particle of reason in them; whereas mind is either the same as truth, or the most like truth, and the truest.

\par \textbf{SOCRATES}
\par   Shall we next consider measure, in like manner, and ask whether pleasure has more of this than wisdom, or wisdom than pleasure?

\par \textbf{PROTARCHUS}
\par   Here is another question which may be easily answered; for I imagine that nothing can ever be more immoderate than the transports of pleasure, or more in conformity with measure than mind and knowledge.

\par \textbf{SOCRATES}
\par   Very good; but there still remains the third test:  Has mind a greater share of beauty than pleasure, and is mind or pleasure the fairer of the two?

\par \textbf{PROTARCHUS}
\par   No one, Socrates, either awake or dreaming, ever saw or imagined mind or wisdom to be in aught unseemly, at any time, past, present, or future.

\par \textbf{SOCRATES}
\par   Right.

\par \textbf{PROTARCHUS}
\par   But when we see some one indulging in pleasures, perhaps in the greatest of pleasures, the ridiculous or disgraceful nature of the action makes us ashamed; and so we put them out of sight, and consign them to darkness, under the idea that they ought not to meet the eye of day.

\par \textbf{SOCRATES}
\par   Then, Protarchus, you will proclaim everywhere, by word of mouth to this company, and by messengers bearing the tidings far and wide, that pleasure is not the first of possessions, nor yet the second, but that in measure, and the mean, and the suitable, and the like, the eternal nature has been found.

\par \textbf{PROTARCHUS}
\par   Yes, that seems to be the result of what has been now said.

\par \textbf{SOCRATES}
\par   In the second class is contained the symmetrical and beautiful and perfect or sufficient, and all which are of that family.

\par \textbf{PROTARCHUS}
\par   True.

\par \textbf{SOCRATES}
\par   And if you reckon in the third class mind and wisdom, you will not be far wrong, if I divine aright.

\par \textbf{PROTARCHUS}
\par   I dare say.

\par \textbf{SOCRATES}
\par   And would you not put in the fourth class the goods which we were affirming to appertain specially to the soul—sciences and arts and true opinions as we called them? These come after the third class, and form the fourth, as they are certainly more akin to good than pleasure is.

\par \textbf{PROTARCHUS}
\par   Surely.

\par \textbf{SOCRATES}
\par   The fifth class are the pleasures which were defined by us as painless, being the pure pleasures of the soul herself, as we termed them, which accompany, some the sciences, and some the senses.

\par \textbf{PROTARCHUS}
\par   Perhaps.

\par \textbf{SOCRATES}
\par   And now, as Orpheus says,
 
\par  Here, at the sixth award, let us make an end; all that remains is to set the crown on our discourse.

\par \textbf{PROTARCHUS}
\par   True.

\par \textbf{SOCRATES}
\par   Then let us sum up and reassert what has been said, thus offering the third libation to the saviour Zeus.

\par \textbf{PROTARCHUS}
\par   How?

\par \textbf{SOCRATES}
\par   Philebus affirmed that pleasure was always and absolutely the good.

\par \textbf{PROTARCHUS}
\par   I understand; this third libation, Socrates, of which you spoke, meant a recapitulation.

\par \textbf{SOCRATES}
\par   Yes, but listen to the sequel; convinced of what I have just been saying, and feeling indignant at the doctrine, which is maintained, not by Philebus only, but by thousands of others, I affirmed that mind was far better and far more excellent, as an element of human life, than pleasure.

\par \textbf{PROTARCHUS}
\par   True.

\par \textbf{SOCRATES}
\par   But, suspecting that there were other things which were also better, I went on to say that if there was anything better than either, then I would claim the second place for mind over pleasure, and pleasure would lose the second place as well as the first.

\par \textbf{PROTARCHUS}
\par   You did.

\par \textbf{SOCRATES}
\par   Nothing could be more satisfactorily shown than the unsatisfactory nature of both of them.

\par \textbf{PROTARCHUS}
\par   Very true.

\par \textbf{SOCRATES}
\par   The claims both of pleasure and mind to be the absolute good have been entirely disproven in this argument, because they are both wanting in self-sufficiency and also in adequacy and perfection.

\par \textbf{PROTARCHUS}
\par   Most true.

\par \textbf{SOCRATES}
\par   But, though they must both resign in favour of another, mind is ten thousand times nearer and more akin to the nature of the conqueror than pleasure.

\par \textbf{PROTARCHUS}
\par   Certainly.

\par \textbf{SOCRATES}
\par   And, according to the judgment which has now been given, pleasure will rank fifth.

\par \textbf{PROTARCHUS}
\par   True.

\par \textbf{SOCRATES}
\par   But not first; no, not even if all the oxen and horses and animals in the world by their pursuit of enjoyment proclaim her to be so;—although the many trusting in them, as diviners trust in birds, determine that pleasures make up the good of life, and deem the lusts of animals to be better witnesses than the inspirations of divine philosophy.

\par \textbf{PROTARCHUS}
\par   And now, Socrates, we tell you that the truth of what you have been saying is approved by the judgment of all of us.

\par \textbf{SOCRATES}
\par   And will you let me go?

\par \textbf{PROTARCHUS}
\par   There is a little which yet remains, and I will remind you of it, for I am sure that you will not be the first to go away from an argument.

\par 
 
\end{document}