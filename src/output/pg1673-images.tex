
\documentclass[11pt,letter]{article}


\begin{document}

\title{Lesser Hippias\thanks{Source: https://www.gutenberg.org/files/1673/1673-h/1673-h.htm. License: http://gutenberg.org/license ds}}
\date{\today}
\author{Plato (spurious and doubtful works), 427? BCE-347? BCE\\ Translated by Jowett, Benjamin, 1817-1893}
\maketitle

\setcounter{tocdepth}{1}
\tableofcontents
\renewcommand{\baselinestretch}{1.0}
\normalsize
\newpage

\section{
      APPENDIX I.
    }
\par  It seems impossible to separate by any exact line the genuine writings of Plato from the spurious. The only external evidence to them which is of much value is that of Aristotle; for the Alexandrian catalogues of a century later include manifest forgeries. Even the value of the Aristotelian authority is a good deal impaired by the uncertainty concerning the date and authorship of the writings which are ascribed to him. And several of the citations of Aristotle omit the name of Plato, and some of them omit the name of the dialogue from which they are taken. Prior, however, to the enquiry about the writings of a particular author, general considerations which equally affect all evidence to the genuineness of ancient writings are the following: Shorter works are more likely to have been forged, or to have received an erroneous designation, than longer ones; and some kinds of composition, such as epistles or panegyrical orations, are more liable to suspicion than others; those, again, which have a taste of sophistry in them, or the ring of a later age, or the slighter character of a rhetorical exercise, or in which a motive or some affinity to spurious writings can be detected, or which seem to have originated in a name or statement really occurring in some classical author, are also of doubtful credit; while there is no instance of any ancient writing proved to be a forgery, which combines excellence with length. A really great and original writer would have no object in fathering his works on Plato; and to the forger or imitator, the 'literary hack' of Alexandria and Athens, the Gods did not grant originality or genius. Further, in attempting to balance the evidence for and against a Platonic dialogue, we must not forget that the form of the Platonic writing was common to several of his contemporaries. Aeschines, Euclid, Phaedo, Antisthenes, and in the next generation Aristotle, are all said to have composed dialogues; and mistakes of names are very likely to have occurred. Greek literature in the third century before Christ was almost as voluminous as our own, and without the safeguards of regular publication, or printing, or binding, or even of distinct titles. An unknown writing was naturally attributed to a known writer whose works bore the same character; and the name once appended easily obtained authority. A tendency may also be observed to blend the works and opinions of the master with those of his scholars. To a later Platonist, the difference between Plato and his imitators was not so perceptible as to ourselves. The Memorabilia of Xenophon and the Dialogues of Plato are but a part of a considerable Socratic literature which has passed away. And we must consider how we should regard the question of the genuineness of a particular writing, if this lost literature had been preserved to us.

\par  These considerations lead us to adopt the following criteria of genuineness: (1) That is most certainly Plato's which Aristotle attributes to him by name, which (2) is of considerable length, of (3) great excellence, and also (4) in harmony with the general spirit of the Platonic writings. But the testimony of Aristotle cannot always be distinguished from that of a later age (see above); and has various degrees of importance. Those writings which he cites without mentioning Plato, under their own names, e.g. the Hippias, the Funeral Oration, the Phaedo, etc., have an inferior degree of evidence in their favour. They may have been supposed by him to be the writings of another, although in the case of really great works, e.g. the Phaedo, this is not credible; those again which are quoted but not named, are still more defective in their external credentials. There may be also a possibility that Aristotle was mistaken, or may have confused the master and his scholars in the case of a short writing; but this is inconceivable about a more important work, e.g. the Laws, especially when we remember that he was living at Athens, and a frequenter of the groves of the Academy, during the last twenty years of Plato's life. Nor must we forget that in all his numerous citations from the Platonic writings he never attributes any passage found in the extant dialogues to any one but Plato. And lastly, we may remark that one or two great writings, such as the Parmenides and the Politicus, which are wholly devoid of Aristotelian (1) credentials may be fairly attributed to Plato, on the ground of (2) length, (3) excellence, and (4) accordance with the general spirit of his writings. Indeed the greater part of the evidence for the genuineness of ancient Greek authors may be summed up under two heads only: (1) excellence; and (2) uniformity of tradition—a kind of evidence, which though in many cases sufficient, is of inferior value.

\par  Proceeding upon these principles we appear to arrive at the conclusion that nineteen-twentieths of all the writings which have ever been ascribed to Plato, are undoubtedly genuine. There is another portion of them, including the Epistles, the Epinomis, the dialogues rejected by the ancients themselves, namely, the Axiochus, De justo, De virtute, Demodocus, Sisyphus, Eryxias, which on grounds, both of internal and external evidence, we are able with equal certainty to reject. But there still remains a small portion of which we are unable to affirm either that they are genuine or spurious. They may have been written in youth, or possibly like the works of some painters, may be partly or wholly the compositions of pupils; or they may have been the writings of some contemporary transferred by accident to the more celebrated name of Plato, or of some Platonist in the next generation who aspired to imitate his master. Not that on grounds either of language or philosophy we should lightly reject them. Some difference of style, or inferiority of execution, or inconsistency of thought, can hardly be considered decisive of their spurious character. For who always does justice to himself, or who writes with equal care at all times? Certainly not Plato, who exhibits the greatest differences in dramatic power, in the formation of sentences, and in the use of words, if his earlier writings are compared with his later ones, say the Protagoras or Phaedrus with the Laws. Or who can be expected to think in the same manner during a period of authorship extending over above fifty years, in an age of great intellectual activity, as well as of political and literary transition? Certainly not Plato, whose earlier writings are separated from his later ones by as wide an interval of philosophical speculation as that which separates his later writings from Aristotle.

\par  The dialogues which have been translated in the first Appendix, and which appear to have the next claim to genuineness among the Platonic writings, are the Lesser Hippias, the Menexenus or Funeral Oration, the First Alcibiades. Of these, the Lesser Hippias and the Funeral Oration are cited by Aristotle; the first in the Metaphysics, the latter in the Rhetoric. Neither of them are expressly attributed to Plato, but in his citation of both of them he seems to be referring to passages in the extant dialogues. From the mention of 'Hippias' in the singular by Aristotle, we may perhaps infer that he was unacquainted with a second dialogue bearing the same name. Moreover, the mere existence of a Greater and Lesser Hippias, and of a First and Second Alcibiades, does to a certain extent throw a doubt upon both of them. Though a very clever and ingenious work, the Lesser Hippias does not appear to contain anything beyond the power of an imitator, who was also a careful student of the earlier Platonic writings, to invent. The motive or leading thought of the dialogue may be detected in Xen. Mem., and there is no similar instance of a 'motive' which is taken from Xenophon in an undoubted dialogue of Plato. On the other hand, the upholders of the genuineness of the dialogue will find in the Hippias a true Socratic spirit; they will compare the Ion as being akin both in subject and treatment; they will urge the authority of Aristotle; and they will detect in the treatment of the Sophist, in the satirical reasoning upon Homer, in the reductio ad absurdum of the doctrine that vice is ignorance, traces of a Platonic authorship. In reference to the last point we are doubtful, as in some of the other dialogues, whether the author is asserting or overthrowing the paradox of Socrates, or merely following the argument 'whither the wind blows.' That no conclusion is arrived at is also in accordance with the character of the earlier dialogues. The resemblances or imitations of the Gorgias, Protagoras, and Euthydemus, which have been observed in the Hippias, cannot with certainty be adduced on either side of the argument. On the whole, more may be said in favour of the genuineness of the Hippias than against it.

\par  The Menexenus or Funeral Oration is cited by Aristotle, and is interesting as supplying an example of the manner in which the orators praised 'the Athenians among the Athenians,' falsifying persons and dates, and casting a veil over the gloomier events of Athenian history. It exhibits an acquaintance with the funeral oration of Thucydides, and was, perhaps, intended to rival that great work. If genuine, the proper place of the Menexenus would be at the end of the Phaedrus. The satirical opening and the concluding words bear a great resemblance to the earlier dialogues; the oration itself is professedly a mimetic work, like the speeches in the Phaedrus, and cannot therefore be tested by a comparison of the other writings of Plato. The funeral oration of Pericles is expressly mentioned in the Phaedrus, and this may have suggested the subject, in the same manner that the Cleitophon appears to be suggested by the slight mention of Cleitophon and his attachment to Thrasymachus in the Republic; and the Theages by the mention of Theages in the Apology and Republic; or as the Second Alcibiades seems to be founded upon the text of Xenophon, Mem. A similar taste for parody appears not only in the Phaedrus, but in the Protagoras, in the Symposium, and to a certain extent in the Parmenides.

\par  To these two doubtful writings of Plato I have added the First Alcibiades, which, of all the disputed dialogues of Plato, has the greatest merit, and is somewhat longer than any of them, though not verified by the testimony of Aristotle, and in many respects at variance with the Symposium in the description of the relations of Socrates and Alcibiades. Like the Lesser Hippias and the Menexenus, it is to be compared to the earlier writings of Plato. The motive of the piece may, perhaps, be found in that passage of the Symposium in which Alcibiades describes himself as self-convicted by the words of Socrates. For the disparaging manner in which Schleiermacher has spoken of this dialogue there seems to be no sufficient foundation. At the same time, the lesson imparted is simple, and the irony more transparent than in the undoubted dialogues of Plato. We know, too, that Alcibiades was a favourite thesis, and that at least five or six dialogues bearing this name passed current in antiquity, and are attributed to contemporaries of Socrates and Plato. (1) In the entire absence of real external evidence (for the catalogues of the Alexandrian librarians cannot be regarded as trustworthy); and (2) in the absence of the highest marks either of poetical or philosophical excellence; and (3) considering that we have express testimony to the existence of contemporary writings bearing the name of Alcibiades, we are compelled to suspend our judgment on the genuineness of the extant dialogue.

\par  Neither at this point, nor at any other, do we propose to draw an absolute line of demarcation between genuine and spurious writings of Plato. They fade off imperceptibly from one class to another. There may have been degrees of genuineness in the dialogues themselves, as there are certainly degrees of evidence by which they are supported. The traditions of the oral discourses both of Socrates and Plato may have formed the basis of semi-Platonic writings; some of them may be of the same mixed character which is apparent in Aristotle and Hippocrates, although the form of them is different. But the writings of Plato, unlike the writings of Aristotle, seem never to have been confused with the writings of his disciples: this was probably due to their definite form, and to their inimitable excellence. The three dialogues which we have offered in the Appendix to the criticism of the reader may be partly spurious and partly genuine; they may be altogether spurious;—that is an alternative which must be frankly admitted. Nor can we maintain of some other dialogues, such as the Parmenides, and the Sophist, and Politicus, that no considerable objection can be urged against them, though greatly overbalanced by the weight (chiefly) of internal evidence in their favour. Nor, on the other hand, can we exclude a bare possibility that some dialogues which are usually rejected, such as the Greater Hippias and the Cleitophon, may be genuine. The nature and object of these semi-Platonic writings require more careful study and more comparison of them with one another, and with forged writings in general, than they have yet received, before we can finally decide on their character. We do not consider them all as genuine until they can be proved to be spurious, as is often maintained and still more often implied in this and similar discussions; but should say of some of them, that their genuineness is neither proven nor disproven until further evidence about them can be adduced. And we are as confident that the Epistles are spurious, as that the Republic, the Timaeus, and the Laws are genuine.

\par  On the whole, not a twentieth part of the writings which pass under the name of Plato, if we exclude the works rejected by the ancients themselves and two or three other plausible inventions, can be fairly doubted by those who are willing to allow that a considerable change and growth may have taken place in his philosophy (see above). That twentieth debatable portion scarcely in any degree affects our judgment of Plato, either as a thinker or a writer, and though suggesting some interesting questions to the scholar and critic, is of little importance to the general reader.

\par 
\section{
      LESSER HIPPIAS
    }
\par 
\section{
      INTRODUCTION.
    }
\par  The Lesser Hippias may be compared with the earlier dialogues of Plato, in which the contrast of Socrates and the Sophists is most strongly exhibited. Hippias, like Protagoras and Gorgias, though civil, is vain and boastful: he knows all things; he can make anything, including his own clothes; he is a manufacturer of poems and declamations, and also of seal-rings, shoes, strigils; his girdle, which he has woven himself, is of a finer than Persian quality. He is a vainer, lighter nature than the two great Sophists (compare Protag. ), but of the same character with them, and equally impatient of the short cut-and-thrust method of Socrates, whom he endeavours to draw into a long oration. At last, he gets tired of being defeated at every point by Socrates, and is with difficulty induced to proceed (compare Thrasymachus, Protagoras, Callicles, and others, to whom the same reluctance is ascribed).

\par  Hippias like Protagoras has common sense on his side, when he argues, citing passages of the Iliad in support of his view, that Homer intended Achilles to be the bravest, Odysseus the wisest of the Greeks. But he is easily overthrown by the superior dialectics of Socrates, who pretends to show that Achilles is not true to his word, and that no similar inconsistency is to be found in Odysseus. Hippias replies that Achilles unintentionally, but Odysseus intentionally, speaks falsehood. But is it better to do wrong intentionally or unintentionally? Socrates, relying on the analogy of the arts, maintains the former, Hippias the latter of the two alternatives...All this is quite conceived in the spirit of Plato, who is very far from making Socrates always argue on the side of truth. The over-reasoning on Homer, which is of course satirical, is also in the spirit of Plato. Poetry turned logic is even more ridiculous than 'rhetoric turned logic,' and equally fallacious. There were reasoners in ancient as well as in modern times, who could never receive the natural impression of Homer, or of any other book which they read. The argument of Socrates, in which he picks out the apparent inconsistencies and discrepancies in the speech and actions of Achilles, and the final paradox, 'that he who is true is also false,' remind us of the interpretation by Socrates of Simonides in the Protagoras, and of similar reasonings in the first book of the Republic. The discrepancies which Socrates discovers in the words of Achilles are perhaps as great as those discovered by some of the modern separatists of the Homeric poems...

\par  At last, Socrates having caught Hippias in the toils of the voluntary and involuntary, is obliged to confess that he is wandering about in the same labyrinth; he makes the reflection on himself which others would make upon him (compare Protagoras). He does not wonder that he should be in a difficulty, but he wonders at Hippias, and he becomes sensible of the gravity of the situation, when ordinary men like himself can no longer go to the wise and be taught by them.

\par  It may be remarked as bearing on the genuineness of this dialogue: (1) that the manners of the speakers are less subtle and refined than in the other dialogues of Plato; (2) that the sophistry of Socrates is more palpable and unblushing, and also more unmeaning; (3) that many turns of thought and style are found in it which appear also in the other dialogues:—whether resemblances of this kind tell in favour of or against the genuineness of an ancient writing, is an important question which will have to be answered differently in different cases. For that a writer may repeat himself is as true as that a forger may imitate; and Plato elsewhere, either of set purpose or from forgetfulness, is full of repetitions. The parallelisms of the Lesser Hippias, as already remarked, are not of the kind which necessarily imply that the dialogue is the work of a forger. The parallelisms of the Greater Hippias with the other dialogues, and the allusion to the Lesser (where Hippias sketches the programme of his next lecture, and invites Socrates to attend and bring any friends with him who may be competent judges), are more than suspicious:—they are of a very poor sort, such as we cannot suppose to have been due to Plato himself. The Greater Hippias more resembles the Euthydemus than any other dialogue; but is immeasurably inferior to it. The Lesser Hippias seems to have more merit than the Greater, and to be more Platonic in spirit. The character of Hippias is the same in both dialogues, but his vanity and boasting are even more exaggerated in the Greater Hippias. His art of memory is specially mentioned in both. He is an inferior type of the same species as Hippodamus of Miletus (Arist. Pol.). Some passages in which the Lesser Hippias may be advantageously compared with the undoubtedly genuine dialogues of Plato are the following:—Less. Hipp. : compare Republic (Socrates' cunning in argument): compare Laches (Socrates' feeling about arguments): compare Republic (Socrates not unthankful): compare Republic (Socrates dishonest in argument).

\par  The Lesser Hippias, though inferior to the other dialogues, may be reasonably believed to have been written by Plato, on the ground (1) of considerable excellence; (2) of uniform tradition beginning with Aristotle and his school. That the dialogue falls below the standard of Plato's other works, or that he has attributed to Socrates an unmeaning paradox (perhaps with the view of showing that he could beat the Sophists at their own weapons; or that he could 'make the worse appear the better cause'; or merely as a dialectical experiment)—are not sufficient reasons for doubting the genuineness of the work.

\par 
\section{
      PERSONS OF THE DIALOGUE: Eudicus, Socrates, Hippias.
    }
\par 

\par 

\par \textbf{EUDICUS}
\par   Why are you silent, Socrates, after the magnificent display which Hippias has been making? Why do you not either refute his words, if he seems to you to have been wrong in any point, or join with us in commending him? There is the more reason why you should speak, because we are now alone, and the audience is confined to those who may fairly claim to take part in a philosophical discussion. SOCRATES:  I should greatly like, Eudicus, to ask Hippias the meaning of what he was saying just now about Homer. I have heard your father, Apemantus, declare that the Iliad of Homer is a finer poem than the Odyssey in the same degree that Achilles was a better man than Odysseus; Odysseus, he would say, is the central figure of the one poem and Achilles of the other. Now, I should like to know, if Hippias has no objection to tell me, what he thinks about these two heroes, and which of them he maintains to be the better; he has already told us in the course of his exhibition many things of various kinds about Homer and divers other poets. EUDICUS:  I am sure that Hippias will be delighted to answer anything which you would like to ask; tell me, Hippias, if Socrates asks you a question, will you answer him? HIPPIAS:  Indeed, Eudicus, I should be strangely inconsistent if I refused to answer Socrates, when at each Olympic festival, as I went up from my house at Elis to the temple of Olympia, where all the Hellenes were assembled, I continually professed my willingness to perform any of the exhibitions which I had prepared, and to answer any questions which any one had to ask. SOCRATES:  Truly, Hippias, you are to be congratulated, if at every Olympic festival you have such an encouraging opinion of your own wisdom when you go up to the temple. I doubt whether any muscular hero would be so fearless and confident in offering his body to the combat at Olympia, as you are in offering your mind. HIPPIAS:  And with good reason, Socrates; for since the day when I first entered the lists at Olympia I have never found any man who was my superior in anything. (Compare Gorgias.) SOCRATES:  What an ornament, Hippias, will the reputation of your wisdom be to the city of Elis and to your parents! But to return:  what say you of Odysseus and Achilles? Which is the better of the two? and in what particular does either surpass the other? For when you were exhibiting and there was company in the room, though I could not follow you, I did not like to ask what you meant, because a crowd of people were present, and I was afraid that the question might interrupt your exhibition. But now that there are not so many of us, and my friend Eudicus bids me ask, I wish you would tell me what you were saying about these two heroes, so that I may clearly understand; how did you distinguish them? HIPPIAS:  I shall have much pleasure, Socrates, in explaining to you more clearly than I could in public my views about these and also about other heroes. I say that Homer intended Achilles to be the bravest of the men who went to Troy, Nestor the wisest, and Odysseus the wiliest. SOCRATES:  O rare Hippias, will you be so good as not to laugh, if I find a difficulty in following you, and repeat my questions several times over? Please to answer me kindly and gently. HIPPIAS:  I should be greatly ashamed of myself, Socrates, if I, who teach others and take money of them, could not, when I was asked by you, answer in a civil and agreeable manner. SOCRATES:  Thank you:  the fact is, that I seemed to understand what you meant when you said that the poet intended Achilles to be the bravest of men, and also that he intended Nestor to be the wisest; but when you said that he meant Odysseus to be the wiliest, I must confess that I could not understand what you were saying. Will you tell me, and then I shall perhaps understand you better; has not Homer made Achilles wily? HIPPIAS:  Certainly not, Socrates; he is the most straight-forward of mankind, and when Homer introduces them talking with one another in the passage called the Prayers, Achilles is supposed by the poet to say to Odysseus: — 'Son of Laertes, sprung from heaven, crafty Odysseus, I will speak out plainly the word which I intend to carry out in act, and which will, I believe, be accomplished. For I hate him like the gates of death who thinks one thing and says another. But I will speak that which shall be accomplished.' Now, in these verses he clearly indicates the character of the two men; he shows Achilles to be true and simple, and Odysseus to be wily and false; for he supposes Achilles to be addressing Odysseus in these lines. SOCRATES:  Now, Hippias, I think that I understand your meaning; when you say that Odysseus is wily, you clearly mean that he is false? HIPPIAS:  Exactly so, Socrates; it is the character of Odysseus, as he is represented by Homer in many passages both of the Iliad and Odyssey. SOCRATES:  And Homer must be presumed to have meant that the true man is not the same as the false? HIPPIAS:  Of course, Socrates. SOCRATES:  And is that your own opinion, Hippias? HIPPIAS:  Certainly; how can I have any other? SOCRATES:  Well, then, as there is no possibility of asking Homer what he meant in these verses of his, let us leave him; but as you show a willingness to take up his cause, and your opinion agrees with what you declare to be his, will you answer on behalf of yourself and him? HIPPIAS:  I will; ask shortly anything which you like. SOCRATES:  Do you say that the false, like the sick, have no power to do things, or that they have the power to do things? HIPPIAS:  I should say that they have power to do many things, and in particular to deceive mankind. SOCRATES:  Then, according to you, they are both powerful and wily, are they not? HIPPIAS:  Yes. SOCRATES:  And are they wily, and do they deceive by reason of their simplicity and folly, or by reason of their cunning and a certain sort of prudence? HIPPIAS:  By reason of their cunning and prudence, most certainly. SOCRATES:  Then they are prudent, I suppose? HIPPIAS:  So they are—very. SOCRATES:  And if they are prudent, do they know or do they not know what they do? HIPPIAS:  Of course, they know very well; and that is why they do mischief to others. SOCRATES:  And having this knowledge, are they ignorant, or are they wise? HIPPIAS:  Wise, certainly; at least, in so far as they can deceive. SOCRATES:  Stop, and let us recall to mind what you are saying; are you not saying that the false are powerful and prudent and knowing and wise in those things about which they are false? HIPPIAS:  To be sure. SOCRATES:  And the true differ from the false—the true and the false are the very opposite of each other? HIPPIAS:  That is my view. SOCRATES:  Then, according to your view, it would seem that the false are to be ranked in the class of the powerful and wise? HIPPIAS:  Assuredly. SOCRATES:  And when you say that the false are powerful and wise in so far as they are false, do you mean that they have or have not the power of uttering their falsehoods if they like? HIPPIAS:  I mean to say that they have the power. SOCRATES:  In a word, then, the false are they who are wise and have the power to speak falsely? HIPPIAS:  Yes. SOCRATES:  Then a man who has not the power of speaking falsely and is ignorant cannot be false? HIPPIAS:  You are right. SOCRATES:  And every man has power who does that which he wishes at the time when he wishes. I am not speaking of any special case in which he is prevented by disease or something of that sort, but I am speaking generally, as I might say of you, that you are able to write my name when you like. Would you not call a man able who could do that? HIPPIAS:  Yes. SOCRATES:  And tell me, Hippias, are you not a skilful calculator and arithmetician? HIPPIAS:  Yes, Socrates, assuredly I am. SOCRATES:  And if some one were to ask you what is the sum of 3 multiplied by 700, you would tell him the true answer in a moment, if you pleased? HIPPIAS:  certainly I should. SOCRATES:  Is not that because you are the wisest and ablest of men in these matters? HIPPIAS:  Yes. SOCRATES:  And being as you are the wisest and ablest of men in these matters of calculation, are you not also the best? HIPPIAS:  To be sure, Socrates, I am the best. SOCRATES:  And therefore you would be the most able to tell the truth about these matters, would you not? HIPPIAS:  Yes, I should. SOCRATES:  And could you speak falsehoods about them equally well? I must beg, Hippias, that you will answer me with the same frankness and magnanimity which has hitherto characterized you. If a person were to ask you what is the sum of 3 multiplied by 700, would not you be the best and most consistent teller of a falsehood, having always the power of speaking falsely as you have of speaking truly, about these same matters, if you wanted to tell a falsehood, and not to answer truly? Would the ignorant man be better able to tell a falsehood in matters of calculation than you would be, if you chose? Might he not sometimes stumble upon the truth, when he wanted to tell a lie, because he did not know, whereas you who are the wise man, if you wanted to tell a lie would always and consistently lie? HIPPIAS:  Yes, there you are quite right. SOCRATES:  Does the false man tell lies about other things, but not about number, or when he is making a calculation? HIPPIAS:  To be sure; he would tell as many lies about number as about other things. SOCRATES:  Then may we further assume, Hippias, that there are men who are false about calculation and number? HIPPIAS:  Yes. SOCRATES:  Who can they be? For you have already admitted that he who is false must have the ability to be false:  you said, as you will remember, that he who is unable to be false will not be false? HIPPIAS:  Yes, I remember; it was so said. SOCRATES:  And were you not yourself just now shown to be best able to speak falsely about calculation? HIPPIAS:  Yes; that was another thing which was said. SOCRATES:  And are you not likewise said to speak truly about calculation? HIPPIAS:  Certainly. SOCRATES:  Then the same person is able to speak both falsely and truly about calculation? And that person is he who is good at calculation—the arithmetician? HIPPIAS:  Yes. SOCRATES:  Who, then, Hippias, is discovered to be false at calculation? Is he not the good man? For the good man is the able man, and he is the true man. HIPPIAS:  That is evident. SOCRATES:  Do you not see, then, that the same man is false and also true about the same matters? And the true man is not a whit better than the false; for indeed he is the same with him and not the very opposite, as you were just now imagining. HIPPIAS:  Not in that instance, clearly. SOCRATES:  Shall we examine other instances? HIPPIAS:  Certainly, if you are disposed. SOCRATES:  Are you not also skilled in geometry? HIPPIAS:  I am. SOCRATES:  Well, and does not the same hold in that science also? Is not the same person best able to speak falsely or to speak truly about diagrams; and he is—the geometrician? HIPPIAS:  Yes. SOCRATES:  He and no one else is good at it? HIPPIAS:  Yes, he and no one else. SOCRATES:  Then the good and wise geometer has this double power in the highest degree; and if there be a man who is false about diagrams the good man will be he, for he is able to be false; whereas the bad is unable, and for this reason is not false, as has been admitted. HIPPIAS:  True. SOCRATES:  Once more—let us examine a third case; that of the astronomer, in whose art, again, you, Hippias, profess to be a still greater proficient than in the preceding—do you not? HIPPIAS:  Yes, I am. SOCRATES:  And does not the same hold of astronomy? HIPPIAS:  True, Socrates. SOCRATES:  And in astronomy, too, if any man be able to speak falsely he will be the good astronomer, but he who is not able will not speak falsely, for he has no knowledge. HIPPIAS:  Clearly not. SOCRATES:  Then in astronomy also, the same man will be true and false? HIPPIAS:  It would seem so. SOCRATES:  And now, Hippias, consider the question at large about all the sciences, and see whether the same principle does not always hold. I know that in most arts you are the wisest of men, as I have heard you boasting in the agora at the tables of the money-changers, when you were setting forth the great and enviable stores of your wisdom; and you said that upon one occasion, when you went to the Olympic games, all that you had on your person was made by yourself. You began with your ring, which was of your own workmanship, and you said that you could engrave rings; and you had another seal which was also of your own workmanship, and a strigil and an oil flask, which you had made yourself; you said also that you had made the shoes which you had on your feet, and the cloak and the short tunic; but what appeared to us all most extraordinary and a proof of singular art, was the girdle of your tunic, which, you said, was as fine as the most costly Persian fabric, and of your own weaving; moreover, you told us that you had brought with you poems, epic, tragic, and dithyrambic, as well as prose writings of the most various kinds; and you said that your skill was also pre-eminent in the arts which I was just now mentioning, and in the true principles of rhythm and harmony and of orthography; and if I remember rightly, there were a great many other accomplishments in which you excelled. I have forgotten to mention your art of memory, which you regard as your special glory, and I dare say that I have forgotten many other things; but, as I was saying, only look to your own arts—and there are plenty of them—and to those of others; and tell me, having regard to the admissions which you and I have made, whether you discover any department of art or any description of wisdom or cunning, whichever name you use, in which the true and false are different and not the same:  tell me, if you can, of any. But you cannot. HIPPIAS:  Not without consideration, Socrates. SOCRATES:  Nor will consideration help you, Hippias, as I believe; but then if I am right, remember what the consequence will be. HIPPIAS:  I do not know what you mean, Socrates. SOCRATES:  I suppose that you are not using your art of memory, doubtless because you think that such an accomplishment is not needed on the present occasion. I will therefore remind you of what you were saying:  were you not saying that Achilles was a true man, and Odysseus false and wily? HIPPIAS:  I was. SOCRATES:  And now do you perceive that the same person has turned out to be false as well as true? If Odysseus is false he is also true, and if Achilles is true he is also false, and so the two men are not opposed to one another, but they are alike. HIPPIAS:  O Socrates, you are always weaving the meshes of an argument, selecting the most difficult point, and fastening upon details instead of grappling with the matter in hand as a whole. Come now, and I will demonstrate to you, if you will allow me, by many satisfactory proofs, that Homer has made Achilles a better man than Odysseus, and a truthful man too; and that he has made the other crafty, and a teller of many untruths, and inferior to Achilles. And then, if you please, you shall make a speech on the other side, in order to prove that Odysseus is the better man; and this may be compared to mine, and then the company will know which of us is the better speaker. SOCRATES:  O Hippias, I do not doubt that you are wiser than I am. But I have a way, when anybody else says anything, of giving close attention to him, especially if the speaker appears to me to be a wise man. Having a desire to understand, I question him, and I examine and analyse and put together what he says, in order that I may understand; but if the speaker appears to me to be a poor hand, I do not interrogate him, or trouble myself about him, and you may know by this who they are whom I deem to be wise men, for you will see that when I am talking with a wise man, I am very attentive to what he says; and I ask questions of him, in order that I may learn, and be improved by him. And I could not help remarking while you were speaking, that when you recited the verses in which Achilles, as you argued, attacks Odysseus as a deceiver, that you must be strangely mistaken, because Odysseus, the man of wiles, is never found to tell a lie; but Achilles is found to be wily on your own showing. At any rate he speaks falsely; for first he utters these words, which you just now repeated,— 'He is hateful to me even as the gates of death who thinks one thing and says another: '— And then he says, a little while afterwards, he will not be persuaded by Odysseus and Agamemnon, neither will he remain at Troy; but, says he,— 'To-morrow, when I have offered sacrifices to Zeus and all the Gods, having loaded my ships well, I will drag them down into the deep; and then you shall see, if you have a mind, and if such things are a care to you, early in the morning my ships sailing over the fishy Hellespont, and my men eagerly plying the oar; and, if the illustrious shaker of the earth gives me a good voyage, on the third day I shall reach the fertile Phthia.' And before that, when he was reviling Agamemnon, he said,— 'And now to Phthia I will go, since to return home in the beaked ships is far better, nor am I inclined to stay here in dishonour and amass wealth and riches for you.' But although on that occasion, in the presence of the whole army, he spoke after this fashion, and on the other occasion to his companions, he appears never to have made any preparation or attempt to draw down the ships, as if he had the least intention of sailing home; so nobly regardless was he of the truth. Now I, Hippias, originally asked you the question, because I was in doubt as to which of the two heroes was intended by the poet to be the best, and because I thought that both of them were the best, and that it would be difficult to decide which was the better of them, not only in respect of truth and falsehood, but of virtue generally, for even in this matter of speaking the truth they are much upon a par. HIPPIAS:  There you are wrong, Socrates; for in so far as Achilles speaks falsely, the falsehood is obviously unintentional. He is compelled against his will to remain and rescue the army in their misfortune. But when Odysseus speaks falsely he is voluntarily and intentionally false. SOCRATES:  You, sweet Hippias, like Odysseus, are a deceiver yourself. HIPPIAS:  Certainly not, Socrates; what makes you say so? SOCRATES:  Because you say that Achilles does not speak falsely from design, when he is not only a deceiver, but besides being a braggart, in Homer's description of him is so cunning, and so far superior to Odysseus in lying and pretending, that he dares to contradict himself, and Odysseus does not find him out; at any rate he does not appear to say anything to him which would imply that he perceived his falsehood. HIPPIAS:  What do you mean, Socrates? SOCRATES:  Did you not observe that afterwards, when he is speaking to Odysseus, he says that he will sail away with the early dawn; but to Ajax he tells quite a different story? HIPPIAS:  Where is that? SOCRATES:  Where he says,— 'I will not think about bloody war until the son of warlike Priam, illustrious Hector, comes to the tents and ships of the Myrmidons, slaughtering the Argives, and burning the ships with fire; and about my tent and dark ship, I suspect that Hector, although eager for the battle, will nevertheless stay his hand.' Now, do you really think, Hippias, that the son of Thetis, who had been the pupil of the sage Cheiron, had such a bad memory, or would have carried the art of lying to such an extent (when he had been assailing liars in the most violent terms only the instant before) as to say to Odysseus that he would sail away, and to Ajax that he would remain, and that he was not rather practising upon the simplicity of Odysseus, whom he regarded as an ancient, and thinking that he would get the better of him by his own cunning and falsehood? HIPPIAS:  No, I do not agree with you, Socrates; but I believe that Achilles is induced to say one thing to Ajax, and another to Odysseus in the innocence of his heart, whereas Odysseus, whether he speaks falsely or truly, speaks always with a purpose. SOCRATES:  Then Odysseus would appear after all to be better than Achilles? HIPPIAS:  Certainly not, Socrates. SOCRATES:  Why, were not the voluntary liars only just now shown to be better than the involuntary? HIPPIAS:  And how, Socrates, can those who intentionally err, and voluntarily and designedly commit iniquities, be better than those who err and do wrong involuntarily? Surely there is a great excuse to be made for a man telling a falsehood, or doing an injury or any sort of harm to another in ignorance. And the laws are obviously far more severe on those who lie or do evil, voluntarily, than on those who do evil involuntarily. SOCRATES:  You see, Hippias, as I have already told you, how pertinacious I am in asking questions of wise men. And I think that this is the only good point about me, for I am full of defects, and always getting wrong in some way or other. My deficiency is proved to me by the fact that when I meet one of you who are famous for wisdom, and to whose wisdom all the Hellenes are witnesses, I am found out to know nothing. For speaking generally, I hardly ever have the same opinion about anything which you have, and what proof of ignorance can be greater than to differ from wise men? But I have one singular good quality, which is my salvation; I am not ashamed to learn, and I ask and enquire, and am very grateful to those who answer me, and never fail to give them my grateful thanks; and when I learn a thing I never deny my teacher, or pretend that the lesson is a discovery of my own; but I praise his wisdom, and proclaim what I have learned from him. And now I cannot agree in what you are saying, but I strongly disagree. Well, I know that this is my own fault, and is a defect in my character, but I will not pretend to be more than I am; and my opinion, Hippias, is the very contrary of what you are saying. For I maintain that those who hurt or injure mankind, and speak falsely and deceive, and err voluntarily, are better far than those who do wrong involuntarily. Sometimes, however, I am of the opposite opinion; for I am all abroad in my ideas about this matter, a condition obviously occasioned by ignorance. And just now I happen to be in a crisis of my disorder at which those who err voluntarily appear to me better than those who err involuntarily. My present state of mind is due to our previous argument, which inclines me to believe that in general those who do wrong involuntarily are worse than those who do wrong voluntarily, and therefore I hope that you will be good to me, and not refuse to heal me; for you will do me a much greater benefit if you cure my soul of ignorance, than you would if you were to cure my body of disease. I must, however, tell you beforehand, that if you make a long oration to me you will not cure me, for I shall not be able to follow you; but if you will answer me, as you did just now, you will do me a great deal of good, and I do not think that you will be any the worse yourself. And I have some claim upon you also, O son of Apemantus, for you incited me to converse with Hippias; and now, if Hippias will not answer me, you must entreat him on my behalf. EUDICUS:  But I do not think, Socrates, that Hippias will require any entreaty of mine; for he has already said that he will refuse to answer no man.—Did you not say so, Hippias? HIPPIAS:  Yes, I did; but then, Eudicus, Socrates is always troublesome in an argument, and appears to be dishonest. (Compare Gorgias; Republic.) SOCRATES:  Excellent Hippias, I do not do so intentionally (if I did, it would show me to be a wise man and a master of wiles, as you would argue), but unintentionally, and therefore you must pardon me; for, as you say, he who is unintentionally dishonest should be pardoned. EUDICUS:  Yes, Hippias, do as he says; and for our sake, and also that you may not belie your profession, answer whatever Socrates asks you. HIPPIAS:  I will answer, as you request me; and do you ask whatever you like. SOCRATES:  I am very desirous, Hippias, of examining this question, as to which are the better—those who err voluntarily or involuntarily? And if you will answer me, I think that I can put you in the way of approaching the subject:  You would admit, would you not, that there are good runners? HIPPIAS:  Yes. SOCRATES:  And there are bad runners? HIPPIAS:  Yes. SOCRATES:  And he who runs well is a good runner, and he who runs ill is a bad runner? HIPPIAS:  Very true. SOCRATES:  And he who runs slowly runs ill, and he who runs quickly runs well? HIPPIAS:  Yes. SOCRATES:  Then in a race, and in running, swiftness is a good, and slowness is an evil quality? HIPPIAS:  To be sure. SOCRATES:  Which of the two then is a better runner? He who runs slowly voluntarily, or he who runs slowly involuntarily? HIPPIAS:  He who runs slowly voluntarily. SOCRATES:  And is not running a species of doing? HIPPIAS:  Certainly. SOCRATES:  And if a species of doing, a species of action? HIPPIAS:  Yes. SOCRATES:  Then he who runs badly does a bad and dishonourable action in a race? HIPPIAS:  Yes; a bad action, certainly. SOCRATES:  And he who runs slowly runs badly? HIPPIAS:  Yes. SOCRATES:  Then the good runner does this bad and disgraceful action voluntarily, and the bad involuntarily? HIPPIAS:  That is to be inferred. SOCRATES:  Then he who involuntarily does evil actions, is worse in a race than he who does them voluntarily? HIPPIAS:  Yes, in a race. SOCRATES:  Well, but at a wrestling match—which is the better wrestler, he who falls voluntarily or involuntarily? HIPPIAS:  He who falls voluntarily, doubtless. SOCRATES:  And is it worse or more dishonourable at a wrestling match, to fall, or to throw another? HIPPIAS:  To fall. SOCRATES:  Then, at a wrestling match, he who voluntarily does base and dishonourable actions is a better wrestler than he who does them involuntarily? HIPPIAS:  That appears to be the truth. SOCRATES:  And what would you say of any other bodily exercise—is not he who is better made able to do both that which is strong and that which is weak—that which is fair and that which is foul?—so that when he does bad actions with the body, he who is better made does them voluntarily, and he who is worse made does them involuntarily. HIPPIAS:  Yes, that appears to be true about strength. SOCRATES:  And what do you say about grace, Hippias? Is not he who is better made able to assume evil and disgraceful figures and postures voluntarily, as he who is worse made assumes them involuntarily? HIPPIAS:  True. SOCRATES:  Then voluntary ungracefulness comes from excellence of the bodily frame, and involuntary from the defect of the bodily frame? HIPPIAS:  True. SOCRATES:  And what would you say of an unmusical voice; would you prefer the voice which is voluntarily or involuntarily out of tune? HIPPIAS:  That which is voluntarily out of tune. SOCRATES:  The involuntary is the worse of the two? HIPPIAS:  Yes. SOCRATES:  And would you choose to possess goods or evils? HIPPIAS:  Goods. SOCRATES:  And would you rather have feet which are voluntarily or involuntarily lame? HIPPIAS:  Feet which are voluntarily lame. SOCRATES:  But is not lameness a defect or deformity? HIPPIAS:  Yes. SOCRATES:  And is not blinking a defect in the eyes? HIPPIAS:  Yes. SOCRATES:  And would you rather always have eyes with which you might voluntarily blink and not see, or with which you might involuntarily blink? HIPPIAS:  I would rather have eyes which voluntarily blink. SOCRATES:  Then in your own case you deem that which voluntarily acts ill, better than that which involuntarily acts ill? HIPPIAS:  Yes, certainly, in cases such as you mention. SOCRATES:  And does not the same hold of ears, nostrils, mouth, and of all the senses—those which involuntarily act ill are not to be desired, as being defective; and those which voluntarily act ill are to be desired as being good? HIPPIAS:  I agree. SOCRATES:  And what would you say of instruments;—which are the better sort of instruments to have to do with?—those with which a man acts ill voluntarily or involuntarily? For example, had a man better have a rudder with which he will steer ill, voluntarily or involuntarily? HIPPIAS:  He had better have a rudder with which he will steer ill voluntarily. SOCRATES:  And does not the same hold of the bow and the lyre, the flute and all other things? HIPPIAS:  Very true. SOCRATES:  And would you rather have a horse of such a temper that you may ride him ill voluntarily or involuntarily? HIPPIAS:  I would rather have a horse which I could ride ill voluntarily. SOCRATES:  That would be the better horse? HIPPIAS:  Yes. SOCRATES:  Then with a horse of better temper, vicious actions would be produced voluntarily; and with a horse of bad temper involuntarily? HIPPIAS:  Certainly. SOCRATES:  And that would be true of a dog, or of any other animal? HIPPIAS:  Yes. SOCRATES:  And is it better to possess the mind of an archer who voluntarily or involuntarily misses the mark? HIPPIAS:  Of him who voluntarily misses. SOCRATES:  This would be the better mind for the purposes of archery? HIPPIAS:  Yes. SOCRATES:  Then the mind which involuntarily errs is worse than the mind which errs voluntarily? HIPPIAS:  Yes, certainly, in the use of the bow. SOCRATES:  And what would you say of the art of medicine;—has not the mind which voluntarily works harm to the body, more of the healing art? HIPPIAS:  Yes. SOCRATES:  Then in the art of medicine the voluntary is better than the involuntary? HIPPIAS:  Yes. SOCRATES:  Well, and in lute-playing and in flute-playing, and in all arts and sciences, is not that mind the better which voluntarily does what is evil and dishonourable, and goes wrong, and is not the worse that which does so involuntarily? HIPPIAS:  That is evident. SOCRATES:  And what would you say of the characters of slaves? Should we not prefer to have those who voluntarily do wrong and make mistakes, and are they not better in their mistakes than those who commit them involuntarily? HIPPIAS:  Yes. SOCRATES:  And should we not desire to have our own minds in the best state possible? HIPPIAS:  Yes. SOCRATES:  And will our minds be better if they do wrong and make mistakes voluntarily or involuntarily? HIPPIAS:  O, Socrates, it would be a monstrous thing to say that those who do wrong voluntarily are better than those who do wrong involuntarily! SOCRATES:  And yet that appears to be the only inference. HIPPIAS:  I do not think so. SOCRATES:  But I imagined, Hippias, that you did. Please to answer once more:  Is not justice a power, or knowledge, or both? Must not justice, at all events, be one of these? HIPPIAS:  Yes. SOCRATES:  But if justice is a power of the soul, then the soul which has the greater power is also the more just; for that which has the greater power, my good friend, has been proved by us to be the better. HIPPIAS:  Yes, that has been proved. SOCRATES:  And if justice is knowledge, then the wiser will be the juster soul, and the more ignorant the more unjust? HIPPIAS:  Yes. SOCRATES:  But if justice be power as well as knowledge—then will not the soul which has both knowledge and power be the more just, and that which is the more ignorant be the more unjust? Must it not be so? HIPPIAS:  Clearly. SOCRATES:  And is not the soul which has the greater power and wisdom also better, and better able to do both good and evil in every action? HIPPIAS:  Certainly. SOCRATES:  The soul, then, which acts ill, acts voluntarily by power and art—and these either one or both of them are elements of justice? HIPPIAS:  That seems to be true. SOCRATES:  And to do injustice is to do ill, and not to do injustice is to do well? HIPPIAS:  Yes. SOCRATES:  And will not the better and abler soul when it does wrong, do wrong voluntarily, and the bad soul involuntarily? HIPPIAS:  Clearly. SOCRATES:  And the good man is he who has the good soul, and the bad man is he who has the bad? HIPPIAS:  Yes. SOCRATES:  Then the good man will voluntarily do wrong, and the bad man involuntarily, if the good man is he who has the good soul? HIPPIAS:  Which he certainly has. SOCRATES:  Then, Hippias, he who voluntarily does wrong and disgraceful things, if there be such a man, will be the good man? HIPPIAS:  There I cannot agree with you. SOCRATES:  Nor can I agree with myself, Hippias; and yet that seems to be the conclusion which, as far as we can see at present, must follow from our argument. As I was saying before, I am all abroad, and being in perplexity am always changing my opinion. Now, that I or any ordinary man should wander in perplexity is not surprising; but if you wise men also wander, and we cannot come to you and rest from our wandering, the matter begins to be serious both to us and to you.

\par \textbf{EUDICUS}
\par   Why are you silent, Socrates, after the magnificent display which Hippias has been making? Why do you not either refute his words, if he seems to you to have been wrong in any point, or join with us in commending him? There is the more reason why you should speak, because we are now alone, and the audience is confined to those who may fairly claim to take part in a philosophical discussion.

\par \textbf{SOCRATES}
\par   I should greatly like, Eudicus, to ask Hippias the meaning of what he was saying just now about Homer. I have heard your father, Apemantus, declare that the Iliad of Homer is a finer poem than the Odyssey in the same degree that Achilles was a better man than Odysseus; Odysseus, he would say, is the central figure of the one poem and Achilles of the other. Now, I should like to know, if Hippias has no objection to tell me, what he thinks about these two heroes, and which of them he maintains to be the better; he has already told us in the course of his exhibition many things of various kinds about Homer and divers other poets.

\par \textbf{EUDICUS}
\par   I am sure that Hippias will be delighted to answer anything which you would like to ask; tell me, Hippias, if Socrates asks you a question, will you answer him?

\par \textbf{HIPPIAS}
\par   Indeed, Eudicus, I should be strangely inconsistent if I refused to answer Socrates, when at each Olympic festival, as I went up from my house at Elis to the temple of Olympia, where all the Hellenes were assembled, I continually professed my willingness to perform any of the exhibitions which I had prepared, and to answer any questions which any one had to ask.

\par \textbf{SOCRATES}
\par   Truly, Hippias, you are to be congratulated, if at every Olympic festival you have such an encouraging opinion of your own wisdom when you go up to the temple. I doubt whether any muscular hero would be so fearless and confident in offering his body to the combat at Olympia, as you are in offering your mind.

\par \textbf{HIPPIAS}
\par   And with good reason, Socrates; for since the day when I first entered the lists at Olympia I have never found any man who was my superior in anything. (Compare Gorgias.)

\par \textbf{SOCRATES}
\par   What an ornament, Hippias, will the reputation of your wisdom be to the city of Elis and to your parents! But to return:  what say you of Odysseus and Achilles? Which is the better of the two? and in what particular does either surpass the other? For when you were exhibiting and there was company in the room, though I could not follow you, I did not like to ask what you meant, because a crowd of people were present, and I was afraid that the question might interrupt your exhibition. But now that there are not so many of us, and my friend Eudicus bids me ask, I wish you would tell me what you were saying about these two heroes, so that I may clearly understand; how did you distinguish them?

\par \textbf{HIPPIAS}
\par   I shall have much pleasure, Socrates, in explaining to you more clearly than I could in public my views about these and also about other heroes. I say that Homer intended Achilles to be the bravest of the men who went to Troy, Nestor the wisest, and Odysseus the wiliest.

\par \textbf{SOCRATES}
\par   O rare Hippias, will you be so good as not to laugh, if I find a difficulty in following you, and repeat my questions several times over? Please to answer me kindly and gently.

\par \textbf{HIPPIAS}
\par   I should be greatly ashamed of myself, Socrates, if I, who teach others and take money of them, could not, when I was asked by you, answer in a civil and agreeable manner.

\par \textbf{SOCRATES}
\par   Thank you:  the fact is, that I seemed to understand what you meant when you said that the poet intended Achilles to be the bravest of men, and also that he intended Nestor to be the wisest; but when you said that he meant Odysseus to be the wiliest, I must confess that I could not understand what you were saying. Will you tell me, and then I shall perhaps understand you better; has not Homer made Achilles wily?

\par \textbf{HIPPIAS}
\par   Certainly not, Socrates; he is the most straight-forward of mankind, and when Homer introduces them talking with one another in the passage called the Prayers, Achilles is supposed by the poet to say to Odysseus: —

\par  'Son of Laertes, sprung from heaven, crafty Odysseus, I will speak out plainly the word which I intend to carry out in act, and which will, I believe, be accomplished. For I hate him like the gates of death who thinks one thing and says another. But I will speak that which shall be accomplished.'

\par  Now, in these verses he clearly indicates the character of the two men; he shows Achilles to be true and simple, and Odysseus to be wily and false; for he supposes Achilles to be addressing Odysseus in these lines.

\par \textbf{SOCRATES}
\par   Now, Hippias, I think that I understand your meaning; when you say that Odysseus is wily, you clearly mean that he is false?

\par \textbf{HIPPIAS}
\par   Exactly so, Socrates; it is the character of Odysseus, as he is represented by Homer in many passages both of the Iliad and Odyssey.

\par \textbf{SOCRATES}
\par   And Homer must be presumed to have meant that the true man is not the same as the false?

\par \textbf{HIPPIAS}
\par   Of course, Socrates.

\par \textbf{SOCRATES}
\par   And is that your own opinion, Hippias?

\par \textbf{HIPPIAS}
\par   Certainly; how can I have any other?

\par \textbf{SOCRATES}
\par   Well, then, as there is no possibility of asking Homer what he meant in these verses of his, let us leave him; but as you show a willingness to take up his cause, and your opinion agrees with what you declare to be his, will you answer on behalf of yourself and him?

\par \textbf{HIPPIAS}
\par   I will; ask shortly anything which you like.

\par \textbf{SOCRATES}
\par   Do you say that the false, like the sick, have no power to do things, or that they have the power to do things?

\par \textbf{HIPPIAS}
\par   I should say that they have power to do many things, and in particular to deceive mankind.

\par \textbf{SOCRATES}
\par   Then, according to you, they are both powerful and wily, are they not?

\par \textbf{HIPPIAS}
\par   Yes.

\par \textbf{SOCRATES}
\par   And are they wily, and do they deceive by reason of their simplicity and folly, or by reason of their cunning and a certain sort of prudence?

\par \textbf{HIPPIAS}
\par   By reason of their cunning and prudence, most certainly.

\par \textbf{SOCRATES}
\par   Then they are prudent, I suppose?

\par \textbf{HIPPIAS}
\par   So they are—very.

\par \textbf{SOCRATES}
\par   And if they are prudent, do they know or do they not know what they do?

\par \textbf{HIPPIAS}
\par   Of course, they know very well; and that is why they do mischief to others.

\par \textbf{SOCRATES}
\par   And having this knowledge, are they ignorant, or are they wise?

\par \textbf{HIPPIAS}
\par   Wise, certainly; at least, in so far as they can deceive.

\par \textbf{SOCRATES}
\par   Stop, and let us recall to mind what you are saying; are you not saying that the false are powerful and prudent and knowing and wise in those things about which they are false?

\par \textbf{HIPPIAS}
\par   To be sure.

\par \textbf{SOCRATES}
\par   And the true differ from the false—the true and the false are the very opposite of each other?

\par \textbf{HIPPIAS}
\par   That is my view.

\par \textbf{SOCRATES}
\par   Then, according to your view, it would seem that the false are to be ranked in the class of the powerful and wise?

\par \textbf{HIPPIAS}
\par   Assuredly.

\par \textbf{SOCRATES}
\par   And when you say that the false are powerful and wise in so far as they are false, do you mean that they have or have not the power of uttering their falsehoods if they like?

\par \textbf{HIPPIAS}
\par   I mean to say that they have the power.

\par \textbf{SOCRATES}
\par   In a word, then, the false are they who are wise and have the power to speak falsely?

\par \textbf{HIPPIAS}
\par   Yes.

\par \textbf{SOCRATES}
\par   Then a man who has not the power of speaking falsely and is ignorant cannot be false?

\par \textbf{HIPPIAS}
\par   You are right.

\par \textbf{SOCRATES}
\par   And every man has power who does that which he wishes at the time when he wishes. I am not speaking of any special case in which he is prevented by disease or something of that sort, but I am speaking generally, as I might say of you, that you are able to write my name when you like. Would you not call a man able who could do that?

\par \textbf{HIPPIAS}
\par   Yes.

\par \textbf{SOCRATES}
\par   And tell me, Hippias, are you not a skilful calculator and arithmetician?

\par \textbf{HIPPIAS}
\par   Yes, Socrates, assuredly I am.

\par \textbf{SOCRATES}
\par   And if some one were to ask you what is the sum of 3 multiplied by 700, you would tell him the true answer in a moment, if you pleased?

\par \textbf{HIPPIAS}
\par   certainly I should.

\par \textbf{SOCRATES}
\par   Is not that because you are the wisest and ablest of men in these matters?

\par \textbf{HIPPIAS}
\par   Yes.

\par \textbf{SOCRATES}
\par   And being as you are the wisest and ablest of men in these matters of calculation, are you not also the best?

\par \textbf{HIPPIAS}
\par   To be sure, Socrates, I am the best.

\par \textbf{SOCRATES}
\par   And therefore you would be the most able to tell the truth about these matters, would you not?

\par \textbf{HIPPIAS}
\par   Yes, I should.

\par \textbf{SOCRATES}
\par   And could you speak falsehoods about them equally well? I must beg, Hippias, that you will answer me with the same frankness and magnanimity which has hitherto characterized you. If a person were to ask you what is the sum of 3 multiplied by 700, would not you be the best and most consistent teller of a falsehood, having always the power of speaking falsely as you have of speaking truly, about these same matters, if you wanted to tell a falsehood, and not to answer truly? Would the ignorant man be better able to tell a falsehood in matters of calculation than you would be, if you chose? Might he not sometimes stumble upon the truth, when he wanted to tell a lie, because he did not know, whereas you who are the wise man, if you wanted to tell a lie would always and consistently lie?

\par \textbf{HIPPIAS}
\par   Yes, there you are quite right.

\par \textbf{SOCRATES}
\par   Does the false man tell lies about other things, but not about number, or when he is making a calculation?

\par \textbf{HIPPIAS}
\par   To be sure; he would tell as many lies about number as about other things.

\par \textbf{SOCRATES}
\par   Then may we further assume, Hippias, that there are men who are false about calculation and number?

\par \textbf{HIPPIAS}
\par   Yes.

\par \textbf{SOCRATES}
\par   Who can they be? For you have already admitted that he who is false must have the ability to be false:  you said, as you will remember, that he who is unable to be false will not be false?

\par \textbf{HIPPIAS}
\par   Yes, I remember; it was so said.

\par \textbf{SOCRATES}
\par   And were you not yourself just now shown to be best able to speak falsely about calculation?

\par \textbf{HIPPIAS}
\par   Yes; that was another thing which was said.

\par \textbf{SOCRATES}
\par   And are you not likewise said to speak truly about calculation?

\par \textbf{HIPPIAS}
\par   Certainly.

\par \textbf{SOCRATES}
\par   Then the same person is able to speak both falsely and truly about calculation? And that person is he who is good at calculation—the arithmetician?

\par \textbf{HIPPIAS}
\par   Yes.

\par \textbf{SOCRATES}
\par   Who, then, Hippias, is discovered to be false at calculation? Is he not the good man? For the good man is the able man, and he is the true man.

\par \textbf{HIPPIAS}
\par   That is evident.

\par \textbf{SOCRATES}
\par   Do you not see, then, that the same man is false and also true about the same matters? And the true man is not a whit better than the false; for indeed he is the same with him and not the very opposite, as you were just now imagining.

\par \textbf{HIPPIAS}
\par   Not in that instance, clearly.

\par \textbf{SOCRATES}
\par   Shall we examine other instances?

\par \textbf{HIPPIAS}
\par   Certainly, if you are disposed.

\par \textbf{SOCRATES}
\par   Are you not also skilled in geometry?

\par \textbf{HIPPIAS}
\par   I am.

\par \textbf{SOCRATES}
\par   Well, and does not the same hold in that science also? Is not the same person best able to speak falsely or to speak truly about diagrams; and he is—the geometrician?

\par \textbf{HIPPIAS}
\par   Yes.

\par \textbf{SOCRATES}
\par   He and no one else is good at it?

\par \textbf{HIPPIAS}
\par   Yes, he and no one else.

\par \textbf{SOCRATES}
\par   Then the good and wise geometer has this double power in the highest degree; and if there be a man who is false about diagrams the good man will be he, for he is able to be false; whereas the bad is unable, and for this reason is not false, as has been admitted.

\par \textbf{HIPPIAS}
\par   True.

\par \textbf{SOCRATES}
\par   Once more—let us examine a third case; that of the astronomer, in whose art, again, you, Hippias, profess to be a still greater proficient than in the preceding—do you not?

\par \textbf{HIPPIAS}
\par   Yes, I am.

\par \textbf{SOCRATES}
\par   And does not the same hold of astronomy?

\par \textbf{HIPPIAS}
\par   True, Socrates.

\par \textbf{SOCRATES}
\par   And in astronomy, too, if any man be able to speak falsely he will be the good astronomer, but he who is not able will not speak falsely, for he has no knowledge.

\par \textbf{HIPPIAS}
\par   Clearly not.

\par \textbf{SOCRATES}
\par   Then in astronomy also, the same man will be true and false?

\par \textbf{HIPPIAS}
\par   It would seem so.

\par \textbf{SOCRATES}
\par   And now, Hippias, consider the question at large about all the sciences, and see whether the same principle does not always hold. I know that in most arts you are the wisest of men, as I have heard you boasting in the agora at the tables of the money-changers, when you were setting forth the great and enviable stores of your wisdom; and you said that upon one occasion, when you went to the Olympic games, all that you had on your person was made by yourself. You began with your ring, which was of your own workmanship, and you said that you could engrave rings; and you had another seal which was also of your own workmanship, and a strigil and an oil flask, which you had made yourself; you said also that you had made the shoes which you had on your feet, and the cloak and the short tunic; but what appeared to us all most extraordinary and a proof of singular art, was the girdle of your tunic, which, you said, was as fine as the most costly Persian fabric, and of your own weaving; moreover, you told us that you had brought with you poems, epic, tragic, and dithyrambic, as well as prose writings of the most various kinds; and you said that your skill was also pre-eminent in the arts which I was just now mentioning, and in the true principles of rhythm and harmony and of orthography; and if I remember rightly, there were a great many other accomplishments in which you excelled. I have forgotten to mention your art of memory, which you regard as your special glory, and I dare say that I have forgotten many other things; but, as I was saying, only look to your own arts—and there are plenty of them—and to those of others; and tell me, having regard to the admissions which you and I have made, whether you discover any department of art or any description of wisdom or cunning, whichever name you use, in which the true and false are different and not the same:  tell me, if you can, of any. But you cannot.

\par \textbf{HIPPIAS}
\par   Not without consideration, Socrates.

\par \textbf{SOCRATES}
\par   Nor will consideration help you, Hippias, as I believe; but then if I am right, remember what the consequence will be.

\par \textbf{HIPPIAS}
\par   I do not know what you mean, Socrates.

\par \textbf{SOCRATES}
\par   I suppose that you are not using your art of memory, doubtless because you think that such an accomplishment is not needed on the present occasion. I will therefore remind you of what you were saying:  were you not saying that Achilles was a true man, and Odysseus false and wily?

\par \textbf{HIPPIAS}
\par   I was.

\par \textbf{SOCRATES}
\par   And now do you perceive that the same person has turned out to be false as well as true? If Odysseus is false he is also true, and if Achilles is true he is also false, and so the two men are not opposed to one another, but they are alike.

\par \textbf{HIPPIAS}
\par   O Socrates, you are always weaving the meshes of an argument, selecting the most difficult point, and fastening upon details instead of grappling with the matter in hand as a whole. Come now, and I will demonstrate to you, if you will allow me, by many satisfactory proofs, that Homer has made Achilles a better man than Odysseus, and a truthful man too; and that he has made the other crafty, and a teller of many untruths, and inferior to Achilles. And then, if you please, you shall make a speech on the other side, in order to prove that Odysseus is the better man; and this may be compared to mine, and then the company will know which of us is the better speaker.

\par \textbf{SOCRATES}
\par   O Hippias, I do not doubt that you are wiser than I am. But I have a way, when anybody else says anything, of giving close attention to him, especially if the speaker appears to me to be a wise man. Having a desire to understand, I question him, and I examine and analyse and put together what he says, in order that I may understand; but if the speaker appears to me to be a poor hand, I do not interrogate him, or trouble myself about him, and you may know by this who they are whom I deem to be wise men, for you will see that when I am talking with a wise man, I am very attentive to what he says; and I ask questions of him, in order that I may learn, and be improved by him. And I could not help remarking while you were speaking, that when you recited the verses in which Achilles, as you argued, attacks Odysseus as a deceiver, that you must be strangely mistaken, because Odysseus, the man of wiles, is never found to tell a lie; but Achilles is found to be wily on your own showing. At any rate he speaks falsely; for first he utters these words, which you just now repeated,—

\par  'He is hateful to me even as the gates of death who thinks one thing and says another:'—

\par  And then he says, a little while afterwards, he will not be persuaded by Odysseus and Agamemnon, neither will he remain at Troy; but, says he,—

\par  'To-morrow, when I have offered sacrifices to Zeus and all the Gods, having loaded my ships well, I will drag them down into the deep; and then you shall see, if you have a mind, and if such things are a care to you, early in the morning my ships sailing over the fishy Hellespont, and my men eagerly plying the oar; and, if the illustrious shaker of the earth gives me a good voyage, on the third day I shall reach the fertile Phthia.'

\par  And before that, when he was reviling Agamemnon, he said,—

\par  'And now to Phthia I will go, since to return home in the beaked ships is far better, nor am I inclined to stay here in dishonour and amass wealth and riches for you.'

\par  But although on that occasion, in the presence of the whole army, he spoke after this fashion, and on the other occasion to his companions, he appears never to have made any preparation or attempt to draw down the ships, as if he had the least intention of sailing home; so nobly regardless was he of the truth. Now I, Hippias, originally asked you the question, because I was in doubt as to which of the two heroes was intended by the poet to be the best, and because I thought that both of them were the best, and that it would be difficult to decide which was the better of them, not only in respect of truth and falsehood, but of virtue generally, for even in this matter of speaking the truth they are much upon a par.

\par \textbf{HIPPIAS}
\par   There you are wrong, Socrates; for in so far as Achilles speaks falsely, the falsehood is obviously unintentional. He is compelled against his will to remain and rescue the army in their misfortune. But when Odysseus speaks falsely he is voluntarily and intentionally false.

\par \textbf{SOCRATES}
\par   You, sweet Hippias, like Odysseus, are a deceiver yourself.

\par \textbf{HIPPIAS}
\par   Certainly not, Socrates; what makes you say so?

\par \textbf{SOCRATES}
\par   Because you say that Achilles does not speak falsely from design, when he is not only a deceiver, but besides being a braggart, in Homer's description of him is so cunning, and so far superior to Odysseus in lying and pretending, that he dares to contradict himself, and Odysseus does not find him out; at any rate he does not appear to say anything to him which would imply that he perceived his falsehood.

\par \textbf{HIPPIAS}
\par   What do you mean, Socrates?

\par \textbf{SOCRATES}
\par   Did you not observe that afterwards, when he is speaking to Odysseus, he says that he will sail away with the early dawn; but to Ajax he tells quite a different story?

\par \textbf{HIPPIAS}
\par   Where is that?

\par \textbf{SOCRATES}
\par   Where he says,—

\par  'I will not think about bloody war until the son of warlike Priam, illustrious Hector, comes to the tents and ships of the Myrmidons, slaughtering the Argives, and burning the ships with fire; and about my tent and dark ship, I suspect that Hector, although eager for the battle, will nevertheless stay his hand.'

\par  Now, do you really think, Hippias, that the son of Thetis, who had been the pupil of the sage Cheiron, had such a bad memory, or would have carried the art of lying to such an extent (when he had been assailing liars in the most violent terms only the instant before) as to say to Odysseus that he would sail away, and to Ajax that he would remain, and that he was not rather practising upon the simplicity of Odysseus, whom he regarded as an ancient, and thinking that he would get the better of him by his own cunning and falsehood?

\par \textbf{HIPPIAS}
\par   No, I do not agree with you, Socrates; but I believe that Achilles is induced to say one thing to Ajax, and another to Odysseus in the innocence of his heart, whereas Odysseus, whether he speaks falsely or truly, speaks always with a purpose.

\par \textbf{SOCRATES}
\par   Then Odysseus would appear after all to be better than Achilles?

\par \textbf{HIPPIAS}
\par   Certainly not, Socrates.

\par \textbf{SOCRATES}
\par   Why, were not the voluntary liars only just now shown to be better than the involuntary?

\par \textbf{HIPPIAS}
\par   And how, Socrates, can those who intentionally err, and voluntarily and designedly commit iniquities, be better than those who err and do wrong involuntarily? Surely there is a great excuse to be made for a man telling a falsehood, or doing an injury or any sort of harm to another in ignorance. And the laws are obviously far more severe on those who lie or do evil, voluntarily, than on those who do evil involuntarily.

\par \textbf{SOCRATES}
\par   You see, Hippias, as I have already told you, how pertinacious I am in asking questions of wise men. And I think that this is the only good point about me, for I am full of defects, and always getting wrong in some way or other. My deficiency is proved to me by the fact that when I meet one of you who are famous for wisdom, and to whose wisdom all the Hellenes are witnesses, I am found out to know nothing. For speaking generally, I hardly ever have the same opinion about anything which you have, and what proof of ignorance can be greater than to differ from wise men? But I have one singular good quality, which is my salvation; I am not ashamed to learn, and I ask and enquire, and am very grateful to those who answer me, and never fail to give them my grateful thanks; and when I learn a thing I never deny my teacher, or pretend that the lesson is a discovery of my own; but I praise his wisdom, and proclaim what I have learned from him. And now I cannot agree in what you are saying, but I strongly disagree. Well, I know that this is my own fault, and is a defect in my character, but I will not pretend to be more than I am; and my opinion, Hippias, is the very contrary of what you are saying. For I maintain that those who hurt or injure mankind, and speak falsely and deceive, and err voluntarily, are better far than those who do wrong involuntarily. Sometimes, however, I am of the opposite opinion; for I am all abroad in my ideas about this matter, a condition obviously occasioned by ignorance. And just now I happen to be in a crisis of my disorder at which those who err voluntarily appear to me better than those who err involuntarily. My present state of mind is due to our previous argument, which inclines me to believe that in general those who do wrong involuntarily are worse than those who do wrong voluntarily, and therefore I hope that you will be good to me, and not refuse to heal me; for you will do me a much greater benefit if you cure my soul of ignorance, than you would if you were to cure my body of disease. I must, however, tell you beforehand, that if you make a long oration to me you will not cure me, for I shall not be able to follow you; but if you will answer me, as you did just now, you will do me a great deal of good, and I do not think that you will be any the worse yourself. And I have some claim upon you also, O son of Apemantus, for you incited me to converse with Hippias; and now, if Hippias will not answer me, you must entreat him on my behalf.

\par \textbf{EUDICUS}
\par   But I do not think, Socrates, that Hippias will require any entreaty of mine; for he has already said that he will refuse to answer no man.—Did you not say so, Hippias?

\par \textbf{HIPPIAS}
\par   Yes, I did; but then, Eudicus, Socrates is always troublesome in an argument, and appears to be dishonest. (Compare Gorgias; Republic.)

\par \textbf{SOCRATES}
\par   Excellent Hippias, I do not do so intentionally (if I did, it would show me to be a wise man and a master of wiles, as you would argue), but unintentionally, and therefore you must pardon me; for, as you say, he who is unintentionally dishonest should be pardoned.

\par \textbf{EUDICUS}
\par   Yes, Hippias, do as he says; and for our sake, and also that you may not belie your profession, answer whatever Socrates asks you.

\par \textbf{HIPPIAS}
\par   I will answer, as you request me; and do you ask whatever you like.

\par \textbf{SOCRATES}
\par   I am very desirous, Hippias, of examining this question, as to which are the better—those who err voluntarily or involuntarily? And if you will answer me, I think that I can put you in the way of approaching the subject:  You would admit, would you not, that there are good runners?

\par \textbf{HIPPIAS}
\par   Yes.

\par \textbf{SOCRATES}
\par   And there are bad runners?

\par \textbf{HIPPIAS}
\par   Yes.

\par \textbf{SOCRATES}
\par   And he who runs well is a good runner, and he who runs ill is a bad runner?

\par \textbf{HIPPIAS}
\par   Very true.

\par \textbf{SOCRATES}
\par   And he who runs slowly runs ill, and he who runs quickly runs well?

\par \textbf{HIPPIAS}
\par   Yes.

\par \textbf{SOCRATES}
\par   Then in a race, and in running, swiftness is a good, and slowness is an evil quality?

\par \textbf{HIPPIAS}
\par   To be sure.

\par \textbf{SOCRATES}
\par   Which of the two then is a better runner? He who runs slowly voluntarily, or he who runs slowly involuntarily?

\par \textbf{HIPPIAS}
\par   He who runs slowly voluntarily.

\par \textbf{SOCRATES}
\par   And is not running a species of doing?

\par \textbf{HIPPIAS}
\par   Certainly.

\par \textbf{SOCRATES}
\par   And if a species of doing, a species of action?

\par \textbf{HIPPIAS}
\par   Yes.

\par \textbf{SOCRATES}
\par   Then he who runs badly does a bad and dishonourable action in a race?

\par \textbf{HIPPIAS}
\par   Yes; a bad action, certainly.

\par \textbf{SOCRATES}
\par   And he who runs slowly runs badly?

\par \textbf{HIPPIAS}
\par   Yes.

\par \textbf{SOCRATES}
\par   Then the good runner does this bad and disgraceful action voluntarily, and the bad involuntarily?

\par \textbf{HIPPIAS}
\par   That is to be inferred.

\par \textbf{SOCRATES}
\par   Then he who involuntarily does evil actions, is worse in a race than he who does them voluntarily?

\par \textbf{HIPPIAS}
\par   Yes, in a race.

\par \textbf{SOCRATES}
\par   Well, but at a wrestling match—which is the better wrestler, he who falls voluntarily or involuntarily?

\par \textbf{HIPPIAS}
\par   He who falls voluntarily, doubtless.

\par \textbf{SOCRATES}
\par   And is it worse or more dishonourable at a wrestling match, to fall, or to throw another?

\par \textbf{HIPPIAS}
\par   To fall.

\par \textbf{SOCRATES}
\par   Then, at a wrestling match, he who voluntarily does base and dishonourable actions is a better wrestler than he who does them involuntarily?

\par \textbf{HIPPIAS}
\par   That appears to be the truth.

\par \textbf{SOCRATES}
\par   And what would you say of any other bodily exercise—is not he who is better made able to do both that which is strong and that which is weak—that which is fair and that which is foul?—so that when he does bad actions with the body, he who is better made does them voluntarily, and he who is worse made does them involuntarily.

\par \textbf{HIPPIAS}
\par   Yes, that appears to be true about strength.

\par \textbf{SOCRATES}
\par   And what do you say about grace, Hippias? Is not he who is better made able to assume evil and disgraceful figures and postures voluntarily, as he who is worse made assumes them involuntarily?

\par \textbf{HIPPIAS}
\par   True.

\par \textbf{SOCRATES}
\par   Then voluntary ungracefulness comes from excellence of the bodily frame, and involuntary from the defect of the bodily frame?

\par \textbf{HIPPIAS}
\par   True.

\par \textbf{SOCRATES}
\par   And what would you say of an unmusical voice; would you prefer the voice which is voluntarily or involuntarily out of tune?

\par \textbf{HIPPIAS}
\par   That which is voluntarily out of tune.

\par \textbf{SOCRATES}
\par   The involuntary is the worse of the two?

\par \textbf{HIPPIAS}
\par   Yes.

\par \textbf{SOCRATES}
\par   And would you choose to possess goods or evils?

\par \textbf{HIPPIAS}
\par   Goods.

\par \textbf{SOCRATES}
\par   And would you rather have feet which are voluntarily or involuntarily lame?

\par \textbf{HIPPIAS}
\par   Feet which are voluntarily lame.

\par \textbf{SOCRATES}
\par   But is not lameness a defect or deformity?

\par \textbf{HIPPIAS}
\par   Yes.

\par \textbf{SOCRATES}
\par   And is not blinking a defect in the eyes?

\par \textbf{HIPPIAS}
\par   Yes.

\par \textbf{SOCRATES}
\par   And would you rather always have eyes with which you might voluntarily blink and not see, or with which you might involuntarily blink?

\par \textbf{HIPPIAS}
\par   I would rather have eyes which voluntarily blink.

\par \textbf{SOCRATES}
\par   Then in your own case you deem that which voluntarily acts ill, better than that which involuntarily acts ill?

\par \textbf{HIPPIAS}
\par   Yes, certainly, in cases such as you mention.

\par \textbf{SOCRATES}
\par   And does not the same hold of ears, nostrils, mouth, and of all the senses—those which involuntarily act ill are not to be desired, as being defective; and those which voluntarily act ill are to be desired as being good?

\par \textbf{HIPPIAS}
\par   I agree.

\par \textbf{SOCRATES}
\par   And what would you say of instruments;—which are the better sort of instruments to have to do with?—those with which a man acts ill voluntarily or involuntarily? For example, had a man better have a rudder with which he will steer ill, voluntarily or involuntarily?

\par \textbf{HIPPIAS}
\par   He had better have a rudder with which he will steer ill voluntarily.

\par \textbf{SOCRATES}
\par   And does not the same hold of the bow and the lyre, the flute and all other things?

\par \textbf{HIPPIAS}
\par   Very true.

\par \textbf{SOCRATES}
\par   And would you rather have a horse of such a temper that you may ride him ill voluntarily or involuntarily?

\par \textbf{HIPPIAS}
\par   I would rather have a horse which I could ride ill voluntarily.

\par \textbf{SOCRATES}
\par   That would be the better horse?

\par \textbf{HIPPIAS}
\par   Yes.

\par \textbf{SOCRATES}
\par   Then with a horse of better temper, vicious actions would be produced voluntarily; and with a horse of bad temper involuntarily?

\par \textbf{HIPPIAS}
\par   Certainly.

\par \textbf{SOCRATES}
\par   And that would be true of a dog, or of any other animal?

\par \textbf{HIPPIAS}
\par   Yes.

\par \textbf{SOCRATES}
\par   And is it better to possess the mind of an archer who voluntarily or involuntarily misses the mark?

\par \textbf{HIPPIAS}
\par   Of him who voluntarily misses.

\par \textbf{SOCRATES}
\par   This would be the better mind for the purposes of archery?

\par \textbf{HIPPIAS}
\par   Yes.

\par \textbf{SOCRATES}
\par   Then the mind which involuntarily errs is worse than the mind which errs voluntarily?

\par \textbf{HIPPIAS}
\par   Yes, certainly, in the use of the bow.

\par \textbf{SOCRATES}
\par   And what would you say of the art of medicine;—has not the mind which voluntarily works harm to the body, more of the healing art?

\par \textbf{HIPPIAS}
\par   Yes.

\par \textbf{SOCRATES}
\par   Then in the art of medicine the voluntary is better than the involuntary?

\par \textbf{HIPPIAS}
\par   Yes.

\par \textbf{SOCRATES}
\par   Well, and in lute-playing and in flute-playing, and in all arts and sciences, is not that mind the better which voluntarily does what is evil and dishonourable, and goes wrong, and is not the worse that which does so involuntarily?

\par \textbf{HIPPIAS}
\par   That is evident.

\par \textbf{SOCRATES}
\par   And what would you say of the characters of slaves? Should we not prefer to have those who voluntarily do wrong and make mistakes, and are they not better in their mistakes than those who commit them involuntarily?

\par \textbf{HIPPIAS}
\par   Yes.

\par \textbf{SOCRATES}
\par   And should we not desire to have our own minds in the best state possible?

\par \textbf{HIPPIAS}
\par   Yes.

\par \textbf{SOCRATES}
\par   And will our minds be better if they do wrong and make mistakes voluntarily or involuntarily?

\par \textbf{HIPPIAS}
\par   O, Socrates, it would be a monstrous thing to say that those who do wrong voluntarily are better than those who do wrong involuntarily!

\par \textbf{SOCRATES}
\par   And yet that appears to be the only inference.

\par \textbf{HIPPIAS}
\par   I do not think so.

\par \textbf{SOCRATES}
\par   But I imagined, Hippias, that you did. Please to answer once more:  Is not justice a power, or knowledge, or both? Must not justice, at all events, be one of these?

\par \textbf{HIPPIAS}
\par   Yes.

\par \textbf{SOCRATES}
\par   But if justice is a power of the soul, then the soul which has the greater power is also the more just; for that which has the greater power, my good friend, has been proved by us to be the better.

\par \textbf{HIPPIAS}
\par   Yes, that has been proved.

\par \textbf{SOCRATES}
\par   And if justice is knowledge, then the wiser will be the juster soul, and the more ignorant the more unjust?

\par \textbf{HIPPIAS}
\par   Yes.

\par \textbf{SOCRATES}
\par   But if justice be power as well as knowledge—then will not the soul which has both knowledge and power be the more just, and that which is the more ignorant be the more unjust? Must it not be so?

\par \textbf{HIPPIAS}
\par   Clearly.

\par \textbf{SOCRATES}
\par   And is not the soul which has the greater power and wisdom also better, and better able to do both good and evil in every action?

\par \textbf{HIPPIAS}
\par   Certainly.

\par \textbf{SOCRATES}
\par   The soul, then, which acts ill, acts voluntarily by power and art—and these either one or both of them are elements of justice?

\par \textbf{HIPPIAS}
\par   That seems to be true.

\par \textbf{SOCRATES}
\par   And to do injustice is to do ill, and not to do injustice is to do well?

\par \textbf{HIPPIAS}
\par   Yes.

\par \textbf{SOCRATES}
\par   And will not the better and abler soul when it does wrong, do wrong voluntarily, and the bad soul involuntarily?

\par \textbf{HIPPIAS}
\par   Clearly.

\par \textbf{SOCRATES}
\par   And the good man is he who has the good soul, and the bad man is he who has the bad?

\par \textbf{HIPPIAS}
\par   Yes.

\par \textbf{SOCRATES}
\par   Then the good man will voluntarily do wrong, and the bad man involuntarily, if the good man is he who has the good soul?

\par \textbf{HIPPIAS}
\par   Which he certainly has.

\par \textbf{SOCRATES}
\par   Then, Hippias, he who voluntarily does wrong and disgraceful things, if there be such a man, will be the good man?

\par \textbf{HIPPIAS}
\par   There I cannot agree with you.

\par \textbf{SOCRATES}
\par   Nor can I agree with myself, Hippias; and yet that seems to be the conclusion which, as far as we can see at present, must follow from our argument. As I was saying before, I am all abroad, and being in perplexity am always changing my opinion. Now, that I or any ordinary man should wander in perplexity is not surprising; but if you wise men also wander, and we cannot come to you and rest from our wandering, the matter begins to be serious both to us and to you.

\par \textbf{EUDICUS}
\par   Why are you silent, Socrates, after the magnificent display which Hippias has been making? Why do you not either refute his words, if he seems to you to have been wrong in any point, or join with us in commending him? There is the more reason why you should speak, because we are now alone, and the audience is confined to those who may fairly claim to take part in a philosophical discussion.

\par \textbf{SOCRATES}
\par   I should greatly like, Eudicus, to ask Hippias the meaning of what he was saying just now about Homer. I have heard your father, Apemantus, declare that the Iliad of Homer is a finer poem than the Odyssey in the same degree that Achilles was a better man than Odysseus; Odysseus, he would say, is the central figure of the one poem and Achilles of the other. Now, I should like to know, if Hippias has no objection to tell me, what he thinks about these two heroes, and which of them he maintains to be the better; he has already told us in the course of his exhibition many things of various kinds about Homer and divers other poets.

\par \textbf{EUDICUS}
\par   I am sure that Hippias will be delighted to answer anything which you would like to ask; tell me, Hippias, if Socrates asks you a question, will you answer him?

\par \textbf{HIPPIAS}
\par   Indeed, Eudicus, I should be strangely inconsistent if I refused to answer Socrates, when at each Olympic festival, as I went up from my house at Elis to the temple of Olympia, where all the Hellenes were assembled, I continually professed my willingness to perform any of the exhibitions which I had prepared, and to answer any questions which any one had to ask.

\par \textbf{SOCRATES}
\par   Truly, Hippias, you are to be congratulated, if at every Olympic festival you have such an encouraging opinion of your own wisdom when you go up to the temple. I doubt whether any muscular hero would be so fearless and confident in offering his body to the combat at Olympia, as you are in offering your mind.

\par \textbf{HIPPIAS}
\par   And with good reason, Socrates; for since the day when I first entered the lists at Olympia I have never found any man who was my superior in anything. (Compare Gorgias.)

\par \textbf{SOCRATES}
\par   What an ornament, Hippias, will the reputation of your wisdom be to the city of Elis and to your parents! But to return:  what say you of Odysseus and Achilles? Which is the better of the two? and in what particular does either surpass the other? For when you were exhibiting and there was company in the room, though I could not follow you, I did not like to ask what you meant, because a crowd of people were present, and I was afraid that the question might interrupt your exhibition. But now that there are not so many of us, and my friend Eudicus bids me ask, I wish you would tell me what you were saying about these two heroes, so that I may clearly understand; how did you distinguish them?

\par \textbf{HIPPIAS}
\par   I shall have much pleasure, Socrates, in explaining to you more clearly than I could in public my views about these and also about other heroes. I say that Homer intended Achilles to be the bravest of the men who went to Troy, Nestor the wisest, and Odysseus the wiliest.

\par \textbf{SOCRATES}
\par   O rare Hippias, will you be so good as not to laugh, if I find a difficulty in following you, and repeat my questions several times over? Please to answer me kindly and gently.

\par \textbf{HIPPIAS}
\par   I should be greatly ashamed of myself, Socrates, if I, who teach others and take money of them, could not, when I was asked by you, answer in a civil and agreeable manner.

\par \textbf{SOCRATES}
\par   Thank you:  the fact is, that I seemed to understand what you meant when you said that the poet intended Achilles to be the bravest of men, and also that he intended Nestor to be the wisest; but when you said that he meant Odysseus to be the wiliest, I must confess that I could not understand what you were saying. Will you tell me, and then I shall perhaps understand you better; has not Homer made Achilles wily?

\par \textbf{HIPPIAS}
\par   Certainly not, Socrates; he is the most straight-forward of mankind, and when Homer introduces them talking with one another in the passage called the Prayers, Achilles is supposed by the poet to say to Odysseus: —

\par  'Son of Laertes, sprung from heaven, crafty Odysseus, I will speak out plainly the word which I intend to carry out in act, and which will, I believe, be accomplished. For I hate him like the gates of death who thinks one thing and says another. But I will speak that which shall be accomplished.'

\par  Now, in these verses he clearly indicates the character of the two men; he shows Achilles to be true and simple, and Odysseus to be wily and false; for he supposes Achilles to be addressing Odysseus in these lines.

\par \textbf{SOCRATES}
\par   Now, Hippias, I think that I understand your meaning; when you say that Odysseus is wily, you clearly mean that he is false?

\par \textbf{HIPPIAS}
\par   Exactly so, Socrates; it is the character of Odysseus, as he is represented by Homer in many passages both of the Iliad and Odyssey.

\par \textbf{SOCRATES}
\par   And Homer must be presumed to have meant that the true man is not the same as the false?

\par \textbf{HIPPIAS}
\par   Of course, Socrates.

\par \textbf{SOCRATES}
\par   And is that your own opinion, Hippias?

\par \textbf{HIPPIAS}
\par   Certainly; how can I have any other?

\par \textbf{SOCRATES}
\par   Well, then, as there is no possibility of asking Homer what he meant in these verses of his, let us leave him; but as you show a willingness to take up his cause, and your opinion agrees with what you declare to be his, will you answer on behalf of yourself and him?

\par \textbf{HIPPIAS}
\par   I will; ask shortly anything which you like.

\par \textbf{SOCRATES}
\par   Do you say that the false, like the sick, have no power to do things, or that they have the power to do things?

\par \textbf{HIPPIAS}
\par   I should say that they have power to do many things, and in particular to deceive mankind.

\par \textbf{SOCRATES}
\par   Then, according to you, they are both powerful and wily, are they not?

\par \textbf{HIPPIAS}
\par   Yes.

\par \textbf{SOCRATES}
\par   And are they wily, and do they deceive by reason of their simplicity and folly, or by reason of their cunning and a certain sort of prudence?

\par \textbf{HIPPIAS}
\par   By reason of their cunning and prudence, most certainly.

\par \textbf{SOCRATES}
\par   Then they are prudent, I suppose?

\par \textbf{HIPPIAS}
\par   So they are—very.

\par \textbf{SOCRATES}
\par   And if they are prudent, do they know or do they not know what they do?

\par \textbf{HIPPIAS}
\par   Of course, they know very well; and that is why they do mischief to others.

\par \textbf{SOCRATES}
\par   And having this knowledge, are they ignorant, or are they wise?

\par \textbf{HIPPIAS}
\par   Wise, certainly; at least, in so far as they can deceive.

\par \textbf{SOCRATES}
\par   Stop, and let us recall to mind what you are saying; are you not saying that the false are powerful and prudent and knowing and wise in those things about which they are false?

\par \textbf{HIPPIAS}
\par   To be sure.

\par \textbf{SOCRATES}
\par   And the true differ from the false—the true and the false are the very opposite of each other?

\par \textbf{HIPPIAS}
\par   That is my view.

\par \textbf{SOCRATES}
\par   Then, according to your view, it would seem that the false are to be ranked in the class of the powerful and wise?

\par \textbf{HIPPIAS}
\par   Assuredly.

\par \textbf{SOCRATES}
\par   And when you say that the false are powerful and wise in so far as they are false, do you mean that they have or have not the power of uttering their falsehoods if they like?

\par \textbf{HIPPIAS}
\par   I mean to say that they have the power.

\par \textbf{SOCRATES}
\par   In a word, then, the false are they who are wise and have the power to speak falsely?

\par \textbf{HIPPIAS}
\par   Yes.

\par \textbf{SOCRATES}
\par   Then a man who has not the power of speaking falsely and is ignorant cannot be false?

\par \textbf{HIPPIAS}
\par   You are right.

\par \textbf{SOCRATES}
\par   And every man has power who does that which he wishes at the time when he wishes. I am not speaking of any special case in which he is prevented by disease or something of that sort, but I am speaking generally, as I might say of you, that you are able to write my name when you like. Would you not call a man able who could do that?

\par \textbf{HIPPIAS}
\par   Yes.

\par \textbf{SOCRATES}
\par   And tell me, Hippias, are you not a skilful calculator and arithmetician?

\par \textbf{HIPPIAS}
\par   Yes, Socrates, assuredly I am.

\par \textbf{SOCRATES}
\par   And if some one were to ask you what is the sum of 3 multiplied by 700, you would tell him the true answer in a moment, if you pleased?

\par \textbf{HIPPIAS}
\par   certainly I should.

\par \textbf{SOCRATES}
\par   Is not that because you are the wisest and ablest of men in these matters?

\par \textbf{HIPPIAS}
\par   Yes.

\par \textbf{SOCRATES}
\par   And being as you are the wisest and ablest of men in these matters of calculation, are you not also the best?

\par \textbf{HIPPIAS}
\par   To be sure, Socrates, I am the best.

\par \textbf{SOCRATES}
\par   And therefore you would be the most able to tell the truth about these matters, would you not?

\par \textbf{HIPPIAS}
\par   Yes, I should.

\par \textbf{SOCRATES}
\par   And could you speak falsehoods about them equally well? I must beg, Hippias, that you will answer me with the same frankness and magnanimity which has hitherto characterized you. If a person were to ask you what is the sum of 3 multiplied by 700, would not you be the best and most consistent teller of a falsehood, having always the power of speaking falsely as you have of speaking truly, about these same matters, if you wanted to tell a falsehood, and not to answer truly? Would the ignorant man be better able to tell a falsehood in matters of calculation than you would be, if you chose? Might he not sometimes stumble upon the truth, when he wanted to tell a lie, because he did not know, whereas you who are the wise man, if you wanted to tell a lie would always and consistently lie?

\par \textbf{HIPPIAS}
\par   Yes, there you are quite right.

\par \textbf{SOCRATES}
\par   Does the false man tell lies about other things, but not about number, or when he is making a calculation?

\par \textbf{HIPPIAS}
\par   To be sure; he would tell as many lies about number as about other things.

\par \textbf{SOCRATES}
\par   Then may we further assume, Hippias, that there are men who are false about calculation and number?

\par \textbf{HIPPIAS}
\par   Yes.

\par \textbf{SOCRATES}
\par   Who can they be? For you have already admitted that he who is false must have the ability to be false:  you said, as you will remember, that he who is unable to be false will not be false?

\par \textbf{HIPPIAS}
\par   Yes, I remember; it was so said.

\par \textbf{SOCRATES}
\par   And were you not yourself just now shown to be best able to speak falsely about calculation?

\par \textbf{HIPPIAS}
\par   Yes; that was another thing which was said.

\par \textbf{SOCRATES}
\par   And are you not likewise said to speak truly about calculation?

\par \textbf{HIPPIAS}
\par   Certainly.

\par \textbf{SOCRATES}
\par   Then the same person is able to speak both falsely and truly about calculation? And that person is he who is good at calculation—the arithmetician?

\par \textbf{HIPPIAS}
\par   Yes.

\par \textbf{SOCRATES}
\par   Who, then, Hippias, is discovered to be false at calculation? Is he not the good man? For the good man is the able man, and he is the true man.

\par \textbf{HIPPIAS}
\par   That is evident.

\par \textbf{SOCRATES}
\par   Do you not see, then, that the same man is false and also true about the same matters? And the true man is not a whit better than the false; for indeed he is the same with him and not the very opposite, as you were just now imagining.

\par \textbf{HIPPIAS}
\par   Not in that instance, clearly.

\par \textbf{SOCRATES}
\par   Shall we examine other instances?

\par \textbf{HIPPIAS}
\par   Certainly, if you are disposed.

\par \textbf{SOCRATES}
\par   Are you not also skilled in geometry?

\par \textbf{HIPPIAS}
\par   I am.

\par \textbf{SOCRATES}
\par   Well, and does not the same hold in that science also? Is not the same person best able to speak falsely or to speak truly about diagrams; and he is—the geometrician?

\par \textbf{HIPPIAS}
\par   Yes.

\par \textbf{SOCRATES}
\par   He and no one else is good at it?

\par \textbf{HIPPIAS}
\par   Yes, he and no one else.

\par \textbf{SOCRATES}
\par   Then the good and wise geometer has this double power in the highest degree; and if there be a man who is false about diagrams the good man will be he, for he is able to be false; whereas the bad is unable, and for this reason is not false, as has been admitted.

\par \textbf{HIPPIAS}
\par   True.

\par \textbf{SOCRATES}
\par   Once more—let us examine a third case; that of the astronomer, in whose art, again, you, Hippias, profess to be a still greater proficient than in the preceding—do you not?

\par \textbf{HIPPIAS}
\par   Yes, I am.

\par \textbf{SOCRATES}
\par   And does not the same hold of astronomy?

\par \textbf{HIPPIAS}
\par   True, Socrates.

\par \textbf{SOCRATES}
\par   And in astronomy, too, if any man be able to speak falsely he will be the good astronomer, but he who is not able will not speak falsely, for he has no knowledge.

\par \textbf{HIPPIAS}
\par   Clearly not.

\par \textbf{SOCRATES}
\par   Then in astronomy also, the same man will be true and false?

\par \textbf{HIPPIAS}
\par   It would seem so.

\par \textbf{SOCRATES}
\par   And now, Hippias, consider the question at large about all the sciences, and see whether the same principle does not always hold. I know that in most arts you are the wisest of men, as I have heard you boasting in the agora at the tables of the money-changers, when you were setting forth the great and enviable stores of your wisdom; and you said that upon one occasion, when you went to the Olympic games, all that you had on your person was made by yourself. You began with your ring, which was of your own workmanship, and you said that you could engrave rings; and you had another seal which was also of your own workmanship, and a strigil and an oil flask, which you had made yourself; you said also that you had made the shoes which you had on your feet, and the cloak and the short tunic; but what appeared to us all most extraordinary and a proof of singular art, was the girdle of your tunic, which, you said, was as fine as the most costly Persian fabric, and of your own weaving; moreover, you told us that you had brought with you poems, epic, tragic, and dithyrambic, as well as prose writings of the most various kinds; and you said that your skill was also pre-eminent in the arts which I was just now mentioning, and in the true principles of rhythm and harmony and of orthography; and if I remember rightly, there were a great many other accomplishments in which you excelled. I have forgotten to mention your art of memory, which you regard as your special glory, and I dare say that I have forgotten many other things; but, as I was saying, only look to your own arts—and there are plenty of them—and to those of others; and tell me, having regard to the admissions which you and I have made, whether you discover any department of art or any description of wisdom or cunning, whichever name you use, in which the true and false are different and not the same:  tell me, if you can, of any. But you cannot.

\par \textbf{HIPPIAS}
\par   Not without consideration, Socrates.

\par \textbf{SOCRATES}
\par   Nor will consideration help you, Hippias, as I believe; but then if I am right, remember what the consequence will be.

\par \textbf{HIPPIAS}
\par   I do not know what you mean, Socrates.

\par \textbf{SOCRATES}
\par   I suppose that you are not using your art of memory, doubtless because you think that such an accomplishment is not needed on the present occasion. I will therefore remind you of what you were saying:  were you not saying that Achilles was a true man, and Odysseus false and wily?

\par \textbf{HIPPIAS}
\par   I was.

\par \textbf{SOCRATES}
\par   And now do you perceive that the same person has turned out to be false as well as true? If Odysseus is false he is also true, and if Achilles is true he is also false, and so the two men are not opposed to one another, but they are alike.

\par \textbf{HIPPIAS}
\par   O Socrates, you are always weaving the meshes of an argument, selecting the most difficult point, and fastening upon details instead of grappling with the matter in hand as a whole. Come now, and I will demonstrate to you, if you will allow me, by many satisfactory proofs, that Homer has made Achilles a better man than Odysseus, and a truthful man too; and that he has made the other crafty, and a teller of many untruths, and inferior to Achilles. And then, if you please, you shall make a speech on the other side, in order to prove that Odysseus is the better man; and this may be compared to mine, and then the company will know which of us is the better speaker.

\par \textbf{SOCRATES}
\par   O Hippias, I do not doubt that you are wiser than I am. But I have a way, when anybody else says anything, of giving close attention to him, especially if the speaker appears to me to be a wise man. Having a desire to understand, I question him, and I examine and analyse and put together what he says, in order that I may understand; but if the speaker appears to me to be a poor hand, I do not interrogate him, or trouble myself about him, and you may know by this who they are whom I deem to be wise men, for you will see that when I am talking with a wise man, I am very attentive to what he says; and I ask questions of him, in order that I may learn, and be improved by him. And I could not help remarking while you were speaking, that when you recited the verses in which Achilles, as you argued, attacks Odysseus as a deceiver, that you must be strangely mistaken, because Odysseus, the man of wiles, is never found to tell a lie; but Achilles is found to be wily on your own showing. At any rate he speaks falsely; for first he utters these words, which you just now repeated,—

\par  'He is hateful to me even as the gates of death who thinks one thing and says another:'—

\par  And then he says, a little while afterwards, he will not be persuaded by Odysseus and Agamemnon, neither will he remain at Troy; but, says he,—

\par  'To-morrow, when I have offered sacrifices to Zeus and all the Gods, having loaded my ships well, I will drag them down into the deep; and then you shall see, if you have a mind, and if such things are a care to you, early in the morning my ships sailing over the fishy Hellespont, and my men eagerly plying the oar; and, if the illustrious shaker of the earth gives me a good voyage, on the third day I shall reach the fertile Phthia.'

\par  And before that, when he was reviling Agamemnon, he said,—

\par  'And now to Phthia I will go, since to return home in the beaked ships is far better, nor am I inclined to stay here in dishonour and amass wealth and riches for you.'

\par  But although on that occasion, in the presence of the whole army, he spoke after this fashion, and on the other occasion to his companions, he appears never to have made any preparation or attempt to draw down the ships, as if he had the least intention of sailing home; so nobly regardless was he of the truth. Now I, Hippias, originally asked you the question, because I was in doubt as to which of the two heroes was intended by the poet to be the best, and because I thought that both of them were the best, and that it would be difficult to decide which was the better of them, not only in respect of truth and falsehood, but of virtue generally, for even in this matter of speaking the truth they are much upon a par.

\par \textbf{HIPPIAS}
\par   There you are wrong, Socrates; for in so far as Achilles speaks falsely, the falsehood is obviously unintentional. He is compelled against his will to remain and rescue the army in their misfortune. But when Odysseus speaks falsely he is voluntarily and intentionally false.

\par \textbf{SOCRATES}
\par   You, sweet Hippias, like Odysseus, are a deceiver yourself.

\par \textbf{HIPPIAS}
\par   Certainly not, Socrates; what makes you say so?

\par \textbf{SOCRATES}
\par   Because you say that Achilles does not speak falsely from design, when he is not only a deceiver, but besides being a braggart, in Homer's description of him is so cunning, and so far superior to Odysseus in lying and pretending, that he dares to contradict himself, and Odysseus does not find him out; at any rate he does not appear to say anything to him which would imply that he perceived his falsehood.

\par \textbf{HIPPIAS}
\par   What do you mean, Socrates?

\par \textbf{SOCRATES}
\par   Did you not observe that afterwards, when he is speaking to Odysseus, he says that he will sail away with the early dawn; but to Ajax he tells quite a different story?

\par \textbf{HIPPIAS}
\par   Where is that?

\par \textbf{SOCRATES}
\par   Where he says,—

\par  'I will not think about bloody war until the son of warlike Priam, illustrious Hector, comes to the tents and ships of the Myrmidons, slaughtering the Argives, and burning the ships with fire; and about my tent and dark ship, I suspect that Hector, although eager for the battle, will nevertheless stay his hand.'

\par  Now, do you really think, Hippias, that the son of Thetis, who had been the pupil of the sage Cheiron, had such a bad memory, or would have carried the art of lying to such an extent (when he had been assailing liars in the most violent terms only the instant before) as to say to Odysseus that he would sail away, and to Ajax that he would remain, and that he was not rather practising upon the simplicity of Odysseus, whom he regarded as an ancient, and thinking that he would get the better of him by his own cunning and falsehood?

\par \textbf{HIPPIAS}
\par   No, I do not agree with you, Socrates; but I believe that Achilles is induced to say one thing to Ajax, and another to Odysseus in the innocence of his heart, whereas Odysseus, whether he speaks falsely or truly, speaks always with a purpose.

\par \textbf{SOCRATES}
\par   Then Odysseus would appear after all to be better than Achilles?

\par \textbf{HIPPIAS}
\par   Certainly not, Socrates.

\par \textbf{SOCRATES}
\par   Why, were not the voluntary liars only just now shown to be better than the involuntary?

\par \textbf{HIPPIAS}
\par   And how, Socrates, can those who intentionally err, and voluntarily and designedly commit iniquities, be better than those who err and do wrong involuntarily? Surely there is a great excuse to be made for a man telling a falsehood, or doing an injury or any sort of harm to another in ignorance. And the laws are obviously far more severe on those who lie or do evil, voluntarily, than on those who do evil involuntarily.

\par \textbf{SOCRATES}
\par   You see, Hippias, as I have already told you, how pertinacious I am in asking questions of wise men. And I think that this is the only good point about me, for I am full of defects, and always getting wrong in some way or other. My deficiency is proved to me by the fact that when I meet one of you who are famous for wisdom, and to whose wisdom all the Hellenes are witnesses, I am found out to know nothing. For speaking generally, I hardly ever have the same opinion about anything which you have, and what proof of ignorance can be greater than to differ from wise men? But I have one singular good quality, which is my salvation; I am not ashamed to learn, and I ask and enquire, and am very grateful to those who answer me, and never fail to give them my grateful thanks; and when I learn a thing I never deny my teacher, or pretend that the lesson is a discovery of my own; but I praise his wisdom, and proclaim what I have learned from him. And now I cannot agree in what you are saying, but I strongly disagree. Well, I know that this is my own fault, and is a defect in my character, but I will not pretend to be more than I am; and my opinion, Hippias, is the very contrary of what you are saying. For I maintain that those who hurt or injure mankind, and speak falsely and deceive, and err voluntarily, are better far than those who do wrong involuntarily. Sometimes, however, I am of the opposite opinion; for I am all abroad in my ideas about this matter, a condition obviously occasioned by ignorance. And just now I happen to be in a crisis of my disorder at which those who err voluntarily appear to me better than those who err involuntarily. My present state of mind is due to our previous argument, which inclines me to believe that in general those who do wrong involuntarily are worse than those who do wrong voluntarily, and therefore I hope that you will be good to me, and not refuse to heal me; for you will do me a much greater benefit if you cure my soul of ignorance, than you would if you were to cure my body of disease. I must, however, tell you beforehand, that if you make a long oration to me you will not cure me, for I shall not be able to follow you; but if you will answer me, as you did just now, you will do me a great deal of good, and I do not think that you will be any the worse yourself. And I have some claim upon you also, O son of Apemantus, for you incited me to converse with Hippias; and now, if Hippias will not answer me, you must entreat him on my behalf.

\par \textbf{EUDICUS}
\par   But I do not think, Socrates, that Hippias will require any entreaty of mine; for he has already said that he will refuse to answer no man.—Did you not say so, Hippias?

\par \textbf{HIPPIAS}
\par   Yes, I did; but then, Eudicus, Socrates is always troublesome in an argument, and appears to be dishonest. (Compare Gorgias; Republic.)

\par \textbf{SOCRATES}
\par   Excellent Hippias, I do not do so intentionally (if I did, it would show me to be a wise man and a master of wiles, as you would argue), but unintentionally, and therefore you must pardon me; for, as you say, he who is unintentionally dishonest should be pardoned.

\par \textbf{EUDICUS}
\par   Yes, Hippias, do as he says; and for our sake, and also that you may not belie your profession, answer whatever Socrates asks you.

\par \textbf{HIPPIAS}
\par   I will answer, as you request me; and do you ask whatever you like.

\par \textbf{SOCRATES}
\par   I am very desirous, Hippias, of examining this question, as to which are the better—those who err voluntarily or involuntarily? And if you will answer me, I think that I can put you in the way of approaching the subject:  You would admit, would you not, that there are good runners?

\par \textbf{HIPPIAS}
\par   Yes.

\par \textbf{SOCRATES}
\par   And there are bad runners?

\par \textbf{HIPPIAS}
\par   Yes.

\par \textbf{SOCRATES}
\par   And he who runs well is a good runner, and he who runs ill is a bad runner?

\par \textbf{HIPPIAS}
\par   Very true.

\par \textbf{SOCRATES}
\par   And he who runs slowly runs ill, and he who runs quickly runs well?

\par \textbf{HIPPIAS}
\par   Yes.

\par \textbf{SOCRATES}
\par   Then in a race, and in running, swiftness is a good, and slowness is an evil quality?

\par \textbf{HIPPIAS}
\par   To be sure.

\par \textbf{SOCRATES}
\par   Which of the two then is a better runner? He who runs slowly voluntarily, or he who runs slowly involuntarily?

\par \textbf{HIPPIAS}
\par   He who runs slowly voluntarily.

\par \textbf{SOCRATES}
\par   And is not running a species of doing?

\par \textbf{HIPPIAS}
\par   Certainly.

\par \textbf{SOCRATES}
\par   And if a species of doing, a species of action?

\par \textbf{HIPPIAS}
\par   Yes.

\par \textbf{SOCRATES}
\par   Then he who runs badly does a bad and dishonourable action in a race?

\par \textbf{HIPPIAS}
\par   Yes; a bad action, certainly.

\par \textbf{SOCRATES}
\par   And he who runs slowly runs badly?

\par \textbf{HIPPIAS}
\par   Yes.

\par \textbf{SOCRATES}
\par   Then the good runner does this bad and disgraceful action voluntarily, and the bad involuntarily?

\par \textbf{HIPPIAS}
\par   That is to be inferred.

\par \textbf{SOCRATES}
\par   Then he who involuntarily does evil actions, is worse in a race than he who does them voluntarily?

\par \textbf{HIPPIAS}
\par   Yes, in a race.

\par \textbf{SOCRATES}
\par   Well, but at a wrestling match—which is the better wrestler, he who falls voluntarily or involuntarily?

\par \textbf{HIPPIAS}
\par   He who falls voluntarily, doubtless.

\par \textbf{SOCRATES}
\par   And is it worse or more dishonourable at a wrestling match, to fall, or to throw another?

\par \textbf{HIPPIAS}
\par   To fall.

\par \textbf{SOCRATES}
\par   Then, at a wrestling match, he who voluntarily does base and dishonourable actions is a better wrestler than he who does them involuntarily?

\par \textbf{HIPPIAS}
\par   That appears to be the truth.

\par \textbf{SOCRATES}
\par   And what would you say of any other bodily exercise—is not he who is better made able to do both that which is strong and that which is weak—that which is fair and that which is foul?—so that when he does bad actions with the body, he who is better made does them voluntarily, and he who is worse made does them involuntarily.

\par \textbf{HIPPIAS}
\par   Yes, that appears to be true about strength.

\par \textbf{SOCRATES}
\par   And what do you say about grace, Hippias? Is not he who is better made able to assume evil and disgraceful figures and postures voluntarily, as he who is worse made assumes them involuntarily?

\par \textbf{HIPPIAS}
\par   True.

\par \textbf{SOCRATES}
\par   Then voluntary ungracefulness comes from excellence of the bodily frame, and involuntary from the defect of the bodily frame?

\par \textbf{HIPPIAS}
\par   True.

\par \textbf{SOCRATES}
\par   And what would you say of an unmusical voice; would you prefer the voice which is voluntarily or involuntarily out of tune?

\par \textbf{HIPPIAS}
\par   That which is voluntarily out of tune.

\par \textbf{SOCRATES}
\par   The involuntary is the worse of the two?

\par \textbf{HIPPIAS}
\par   Yes.

\par \textbf{SOCRATES}
\par   And would you choose to possess goods or evils?

\par \textbf{HIPPIAS}
\par   Goods.

\par \textbf{SOCRATES}
\par   And would you rather have feet which are voluntarily or involuntarily lame?

\par \textbf{HIPPIAS}
\par   Feet which are voluntarily lame.

\par \textbf{SOCRATES}
\par   But is not lameness a defect or deformity?

\par \textbf{HIPPIAS}
\par   Yes.

\par \textbf{SOCRATES}
\par   And is not blinking a defect in the eyes?

\par \textbf{HIPPIAS}
\par   Yes.

\par \textbf{SOCRATES}
\par   And would you rather always have eyes with which you might voluntarily blink and not see, or with which you might involuntarily blink?

\par \textbf{HIPPIAS}
\par   I would rather have eyes which voluntarily blink.

\par \textbf{SOCRATES}
\par   Then in your own case you deem that which voluntarily acts ill, better than that which involuntarily acts ill?

\par \textbf{HIPPIAS}
\par   Yes, certainly, in cases such as you mention.

\par \textbf{SOCRATES}
\par   And does not the same hold of ears, nostrils, mouth, and of all the senses—those which involuntarily act ill are not to be desired, as being defective; and those which voluntarily act ill are to be desired as being good?

\par \textbf{HIPPIAS}
\par   I agree.

\par \textbf{SOCRATES}
\par   And what would you say of instruments;—which are the better sort of instruments to have to do with?—those with which a man acts ill voluntarily or involuntarily? For example, had a man better have a rudder with which he will steer ill, voluntarily or involuntarily?

\par \textbf{HIPPIAS}
\par   He had better have a rudder with which he will steer ill voluntarily.

\par \textbf{SOCRATES}
\par   And does not the same hold of the bow and the lyre, the flute and all other things?

\par \textbf{HIPPIAS}
\par   Very true.

\par \textbf{SOCRATES}
\par   And would you rather have a horse of such a temper that you may ride him ill voluntarily or involuntarily?

\par \textbf{HIPPIAS}
\par   I would rather have a horse which I could ride ill voluntarily.

\par \textbf{SOCRATES}
\par   That would be the better horse?

\par \textbf{HIPPIAS}
\par   Yes.

\par \textbf{SOCRATES}
\par   Then with a horse of better temper, vicious actions would be produced voluntarily; and with a horse of bad temper involuntarily?

\par \textbf{HIPPIAS}
\par   Certainly.

\par \textbf{SOCRATES}
\par   And that would be true of a dog, or of any other animal?

\par \textbf{HIPPIAS}
\par   Yes.

\par \textbf{SOCRATES}
\par   And is it better to possess the mind of an archer who voluntarily or involuntarily misses the mark?

\par \textbf{HIPPIAS}
\par   Of him who voluntarily misses.

\par \textbf{SOCRATES}
\par   This would be the better mind for the purposes of archery?

\par \textbf{HIPPIAS}
\par   Yes.

\par \textbf{SOCRATES}
\par   Then the mind which involuntarily errs is worse than the mind which errs voluntarily?

\par \textbf{HIPPIAS}
\par   Yes, certainly, in the use of the bow.

\par \textbf{SOCRATES}
\par   And what would you say of the art of medicine;—has not the mind which voluntarily works harm to the body, more of the healing art?

\par \textbf{HIPPIAS}
\par   Yes.

\par \textbf{SOCRATES}
\par   Then in the art of medicine the voluntary is better than the involuntary?

\par \textbf{HIPPIAS}
\par   Yes.

\par \textbf{SOCRATES}
\par   Well, and in lute-playing and in flute-playing, and in all arts and sciences, is not that mind the better which voluntarily does what is evil and dishonourable, and goes wrong, and is not the worse that which does so involuntarily?

\par \textbf{HIPPIAS}
\par   That is evident.

\par \textbf{SOCRATES}
\par   And what would you say of the characters of slaves? Should we not prefer to have those who voluntarily do wrong and make mistakes, and are they not better in their mistakes than those who commit them involuntarily?

\par \textbf{HIPPIAS}
\par   Yes.

\par \textbf{SOCRATES}
\par   And should we not desire to have our own minds in the best state possible?

\par \textbf{HIPPIAS}
\par   Yes.

\par \textbf{SOCRATES}
\par   And will our minds be better if they do wrong and make mistakes voluntarily or involuntarily?

\par \textbf{HIPPIAS}
\par   O, Socrates, it would be a monstrous thing to say that those who do wrong voluntarily are better than those who do wrong involuntarily!

\par \textbf{SOCRATES}
\par   And yet that appears to be the only inference.

\par \textbf{HIPPIAS}
\par   I do not think so.

\par \textbf{SOCRATES}
\par   But I imagined, Hippias, that you did. Please to answer once more:  Is not justice a power, or knowledge, or both? Must not justice, at all events, be one of these?

\par \textbf{HIPPIAS}
\par   Yes.

\par \textbf{SOCRATES}
\par   But if justice is a power of the soul, then the soul which has the greater power is also the more just; for that which has the greater power, my good friend, has been proved by us to be the better.

\par \textbf{HIPPIAS}
\par   Yes, that has been proved.

\par \textbf{SOCRATES}
\par   And if justice is knowledge, then the wiser will be the juster soul, and the more ignorant the more unjust?

\par \textbf{HIPPIAS}
\par   Yes.

\par \textbf{SOCRATES}
\par   But if justice be power as well as knowledge—then will not the soul which has both knowledge and power be the more just, and that which is the more ignorant be the more unjust? Must it not be so?

\par \textbf{HIPPIAS}
\par   Clearly.

\par \textbf{SOCRATES}
\par   And is not the soul which has the greater power and wisdom also better, and better able to do both good and evil in every action?

\par \textbf{HIPPIAS}
\par   Certainly.

\par \textbf{SOCRATES}
\par   The soul, then, which acts ill, acts voluntarily by power and art—and these either one or both of them are elements of justice?

\par \textbf{HIPPIAS}
\par   That seems to be true.

\par \textbf{SOCRATES}
\par   And to do injustice is to do ill, and not to do injustice is to do well?

\par \textbf{HIPPIAS}
\par   Yes.

\par \textbf{SOCRATES}
\par   And will not the better and abler soul when it does wrong, do wrong voluntarily, and the bad soul involuntarily?

\par \textbf{HIPPIAS}
\par   Clearly.

\par \textbf{SOCRATES}
\par   And the good man is he who has the good soul, and the bad man is he who has the bad?

\par \textbf{HIPPIAS}
\par   Yes.

\par \textbf{SOCRATES}
\par   Then the good man will voluntarily do wrong, and the bad man involuntarily, if the good man is he who has the good soul?

\par \textbf{HIPPIAS}
\par   Which he certainly has.

\par \textbf{SOCRATES}
\par   Then, Hippias, he who voluntarily does wrong and disgraceful things, if there be such a man, will be the good man?

\par \textbf{HIPPIAS}
\par   There I cannot agree with you.

\par \textbf{SOCRATES}
\par   Nor can I agree with myself, Hippias; and yet that seems to be the conclusion which, as far as we can see at present, must follow from our argument. As I was saying before, I am all abroad, and being in perplexity am always changing my opinion. Now, that I or any ordinary man should wander in perplexity is not surprising; but if you wise men also wander, and we cannot come to you and rest from our wandering, the matter begins to be serious both to us and to you.

\par 
 
\end{document}