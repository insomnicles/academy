
\documentclass[11pt,letter]{article}


\begin{document}

\title{Phaedo\thanks{Source: https://www.gutenberg.org/files/1658/1658-h/1658-h.htm. License: http://gutenberg.org/license ds}}
\date{\today}
\author{Plato, 427? BCE-347? BCE\\ Translated by Jowett, Benjamin, 1817-1893}
\maketitle

\setcounter{tocdepth}{1}
\tableofcontents
\renewcommand{\baselinestretch}{1.0}
\normalsize
\newpage

\section{
      INTRODUCTION.
    }
\par  After an interval of some months or years, and at Phlius, a town of Peloponnesus, the tale of the last hours of Socrates is narrated to Echecrates and other Phliasians by Phaedo the 'beloved disciple.' The Dialogue necessarily takes the form of a narrative, because Socrates has to be described acting as well as speaking. The minutest particulars of the event are interesting to distant friends, and the narrator has an equal interest in them.

\par  During the voyage of the sacred ship to and from Delos, which has occupied thirty days, the execution of Socrates has been deferred. (Compare Xen. Mem.) The time has been passed by him in conversation with a select company of disciples. But now the holy season is over, and the disciples meet earlier than usual in order that they may converse with Socrates for the last time. Those who were present, and those who might have been expected to be present, are mentioned by name. There are Simmias and Cebes (Crito), two disciples of Philolaus whom Socrates 'by his enchantments has attracted from Thebes' (Mem. ), Crito the aged friend, the attendant of the prison, who is as good as a friend—these take part in the conversation. There are present also, Hermogenes, from whom Xenophon derived his information about the trial of Socrates (Mem. ), the 'madman' Apollodorus (Symp. ), Euclid and Terpsion from Megara (compare Theaet. ), Ctesippus, Antisthenes, Menexenus, and some other less-known members of the Socratic circle, all of whom are silent auditors. Aristippus, Cleombrotus, and Plato are noted as absent. Almost as soon as the friends of Socrates enter the prison Xanthippe and her children are sent home in the care of one of Crito's servants. Socrates himself has just been released from chains, and is led by this circumstance to make the natural remark that 'pleasure follows pain.' (Observe that Plato is preparing the way for his doctrine of the alternation of opposites.) 'Aesop would have represented them in a fable as a two-headed creature of the gods.' The mention of Aesop reminds Cebes of a question which had been asked by Evenus the poet (compare Apol. ): 'Why Socrates, who was not a poet, while in prison had been putting Aesop into verse? '—'Because several times in his life he had been warned in dreams that he should practise music; and as he was about to die and was not certain of what was meant, he wished to fulfil the admonition in the letter as well as in the spirit, by writing verses as well as by cultivating philosophy. Tell this to Evenus; and say that I would have him follow me in death.' 'He is not at all the sort of man to comply with your request, Socrates.' 'Why, is he not a philosopher?' 'Yes.' 'Then he will be willing to die, although he will not take his own life, for that is held to be unlawful.'

\par  Cebes asks why suicide is thought not to be right, if death is to be accounted a good? Well, (1) according to one explanation, because man is a prisoner, who must not open the door of his prison and run away—this is the truth in a 'mystery.' Or (2) rather, because he is not his own property, but a possession of the gods, and has no right to make away with that which does not belong to him. But why, asks Cebes, if he is a possession of the gods, should he wish to die and leave them? For he is under their protection; and surely he cannot take better care of himself than they take of him. Simmias explains that Cebes is really referring to Socrates, whom they think too unmoved at the prospect of leaving the gods and his friends. Socrates answers that he is going to other gods who are wise and good, and perhaps to better friends; and he professes that he is ready to defend himself against the charge of Cebes. The company shall be his judges, and he hopes that he will be more successful in convincing them than he had been in convincing the court.

\par  The philosopher desires death—which the wicked world will insinuate that he also deserves: and perhaps he does, but not in any sense which they are capable of understanding. Enough of them: the real question is, What is the nature of that death which he desires? Death is the separation of soul and body—and the philosopher desires such a separation. He would like to be freed from the dominion of bodily pleasures and of the senses, which are always perturbing his mental vision. He wants to get rid of eyes and ears, and with the light of the mind only to behold the light of truth. All the evils and impurities and necessities of men come from the body. And death separates him from these corruptions, which in life he cannot wholly lay aside. Why then should he repine when the hour of separation arrives? Why, if he is dead while he lives, should he fear that other death, through which alone he can behold wisdom in her purity?

\par  Besides, the philosopher has notions of good and evil unlike those of other men. For they are courageous because they are afraid of greater dangers, and temperate because they desire greater pleasures. But he disdains this balancing of pleasures and pains, which is the exchange of commerce and not of virtue. All the virtues, including wisdom, are regarded by him only as purifications of the soul. And this was the meaning of the founders of the mysteries when they said, 'Many are the wand-bearers but few are the mystics.' (Compare Matt. xxii. : 'Many are called but few are chosen.') And in the hope that he is one of these mystics, Socrates is now departing. This is his answer to any one who charges him with indifference at the prospect of leaving the gods and his friends.

\par  Still, a fear is expressed that the soul upon leaving the body may vanish away like smoke or air. Socrates in answer appeals first of all to the old Orphic tradition that the souls of the dead are in the world below, and that the living come from them. This he attempts to found on a philosophical assumption that all opposites—e.g. less, greater; weaker, stronger; sleeping, waking; life, death—are generated out of each other. Nor can the process of generation be only a passage from living to dying, for then all would end in death. The perpetual sleeper (Endymion) would be no longer distinguished from the rest of mankind. The circle of nature is not complete unless the living come from the dead as well as pass to them.

\par  The Platonic doctrine of reminiscence is then adduced as a confirmation of the pre-existence of the soul. Some proofs of this doctrine are demanded. One proof given is the same as that of the Meno, and is derived from the latent knowledge of mathematics, which may be elicited from an unlearned person when a diagram is presented to him. Again, there is a power of association, which from seeing Simmias may remember Cebes, or from seeing a picture of Simmias may remember Simmias. The lyre may recall the player of the lyre, and equal pieces of wood or stone may be associated with the higher notion of absolute equality. But here observe that material equalities fall short of the conception of absolute equality with which they are compared, and which is the measure of them. And the measure or standard must be prior to that which is measured, the idea of equality prior to the visible equals. And if prior to them, then prior also to the perceptions of the senses which recall them, and therefore either given before birth or at birth. But all men have not this knowledge, nor have any without a process of reminiscence; which is a proof that it is not innate or given at birth, unless indeed it was given and taken away at the same instant. But if not given to men in birth, it must have been given before birth—this is the only alternative which remains. And if we had ideas in a former state, then our souls must have existed and must have had intelligence in a former state. The pre-existence of the soul stands or falls with the doctrine of ideas.

\par  It is objected by Simmias and Cebes that these arguments only prove a former and not a future existence. Socrates answers this objection by recalling the previous argument, in which he had shown that the living come from the dead. But the fear that the soul at departing may vanish into air (especially if there is a wind blowing at the time) has not yet been charmed away. He proceeds: When we fear that the soul will vanish away, let us ask ourselves what is that which we suppose to be liable to dissolution? Is it the simple or the compound, the unchanging or the changing, the invisible idea or the visible object of sense? Clearly the latter and not the former; and therefore not the soul, which in her own pure thought is unchangeable, and only when using the senses descends into the region of change. Again, the soul commands, the body serves: in this respect too the soul is akin to the divine, and the body to the mortal. And in every point of view the soul is the image of divinity and immortality, and the body of the human and mortal. And whereas the body is liable to speedy dissolution, the soul is almost if not quite indissoluble. (Compare Tim.) Yet even the body may be preserved for ages by the embalmer's art: how unlikely, then, that the soul will perish and be dissipated into air while on her way to the good and wise God! She has been gathered into herself, holding aloof from the body, and practising death all her life long, and she is now finally released from the errors and follies and passions of men, and for ever dwells in the company of the gods.

\par  But the soul which is polluted and engrossed by the corporeal, and has no eye except that of the senses, and is weighed down by the bodily appetites, cannot attain to this abstraction. In her fear of the world below she lingers about the sepulchre, loath to leave the body which she loved, a ghostly apparition, saturated with sense, and therefore visible. At length entering into some animal of a nature congenial to her former life of sensuality or violence, she takes the form of an ass, a wolf or a kite. And of these earthly souls the happiest are those who have practised virtue without philosophy; they are allowed to pass into gentle and social natures, such as bees and ants. (Compare Republic, Meno.) But only the philosopher who departs pure is permitted to enter the company of the gods. (Compare Phaedrus.) This is the reason why he abstains from fleshly lusts, and not because he fears loss or disgrace, which is the motive of other men. He too has been a captive, and the willing agent of his own captivity. But philosophy has spoken to him, and he has heard her voice; she has gently entreated him, and brought him out of the 'miry clay,' and purged away the mists of passion and the illusions of sense which envelope him; his soul has escaped from the influence of pleasures and pains, which are like nails fastening her to the body. To that prison-house she will not return; and therefore she abstains from bodily pleasures—not from a desire of having more or greater ones, but because she knows that only when calm and free from the dominion of the body can she behold the light of truth.

\par  Simmias and Cebes remain in doubt; but they are unwilling to raise objections at such a time. Socrates wonders at their reluctance. Let them regard him rather as the swan, who, having sung the praises of Apollo all his life long, sings at his death more lustily than ever. Simmias acknowledges that there is cowardice in not probing truth to the bottom. 'And if truth divine and inspired is not to be had, then let a man take the best of human notions, and upon this frail bark let him sail through life.' He proceeds to state his difficulty: It has been argued that the soul is invisible and incorporeal, and therefore immortal, and prior to the body. But is not the soul acknowledged to be a harmony, and has she not the same relation to the body, as the harmony—which like her is invisible—has to the lyre? And yet the harmony does not survive the lyre. Cebes has also an objection, which like Simmias he expresses in a figure. He is willing to admit that the soul is more lasting than the body. But the more lasting nature of the soul does not prove her immortality; for after having worn out many bodies in a single life, and many more in successive births and deaths, she may at last perish, or, as Socrates afterwards restates the objection, the very act of birth may be the beginning of her death, and her last body may survive her, just as the coat of an old weaver is left behind him after he is dead, although a man is more lasting than his coat. And he who would prove the immortality of the soul, must prove not only that the soul outlives one or many bodies, but that she outlives them all.

\par  The audience, like the chorus in a play, for a moment interpret the feelings of the actors; there is a temporary depression, and then the enquiry is resumed. It is a melancholy reflection that arguments, like men, are apt to be deceivers; and those who have been often deceived become distrustful both of arguments and of friends. But this unfortunate experience should not make us either haters of men or haters of arguments. The want of health and truth is not in the argument, but in ourselves. Socrates, who is about to die, is sensible of his own weakness; he desires to be impartial, but he cannot help feeling that he has too great an interest in the truth of the argument. And therefore he would have his friends examine and refute him, if they think that he is in error.

\par  At his request Simmias and Cebes repeat their objections. They do not go to the length of denying the pre-existence of ideas. Simmias is of opinion that the soul is a harmony of the body. But the admission of the pre-existence of ideas, and therefore of the soul, is at variance with this. (Compare a parallel difficulty in Theaet.) For a harmony is an effect, whereas the soul is not an effect, but a cause; a harmony follows, but the soul leads; a harmony admits of degrees, and the soul has no degrees. Again, upon the supposition that the soul is a harmony, why is one soul better than another? Are they more or less harmonized, or is there one harmony within another? But the soul does not admit of degrees, and cannot therefore be more or less harmonized. Further, the soul is often engaged in resisting the affections of the body, as Homer describes Odysseus 'rebuking his heart.' Could he have written this under the idea that the soul is a harmony of the body? Nay rather, are we not contradicting Homer and ourselves in affirming anything of the sort?

\par  The goddess Harmonia, as Socrates playfully terms the argument of Simmias, has been happily disposed of; and now an answer has to be given to the Theban Cadmus. Socrates recapitulates the argument of Cebes, which, as he remarks, involves the whole question of natural growth or causation; about this he proposes to narrate his own mental experience. When he was young he had puzzled himself with physics: he had enquired into the growth and decay of animals, and the origin of thought, until at last he began to doubt the self-evident fact that growth is the result of eating and drinking; and so he arrived at the conclusion that he was not meant for such enquiries. Nor was he less perplexed with notions of comparison and number. At first he had imagined himself to understand differences of greater and less, and to know that ten is two more than eight, and the like. But now those very notions appeared to him to contain a contradiction. For how can one be divided into two? Or two be compounded into one? These are difficulties which Socrates cannot answer. Of generation and destruction he knows nothing. But he has a confused notion of another method in which matters of this sort are to be investigated. (Compare Republic; Charm.)

\par  Then he heard some one reading out of a book of Anaxagoras, that mind is the cause of all things. And he said to himself: If mind is the cause of all things, surely mind must dispose them all for the best. The new teacher will show me this 'order of the best' in man and nature. How great had been his hopes and how great his disappointment! For he found that his new friend was anything but consistent in his use of mind as a cause, and that he soon introduced winds, waters, and other eccentric notions. (Compare Arist. Metaph.) It was as if a person had said that Socrates is sitting here because he is made up of bones and muscles, instead of telling the true reason—that he is here because the Athenians have thought good to sentence him to death, and he has thought good to await his sentence. Had his bones and muscles been left by him to their own ideas of right, they would long ago have taken themselves off. But surely there is a great confusion of the cause and condition in all this. And this confusion also leads people into all sorts of erroneous theories about the position and motions of the earth. None of them know how much stronger than any Atlas is the power of the best. But this 'best' is still undiscovered; and in enquiring after the cause, we can only hope to attain the second best.

\par  Now there is a danger in the contemplation of the nature of things, as there is a danger in looking at the sun during an eclipse, unless the precaution is taken of looking only at the image reflected in the water, or in a glass. (Compare Laws; Republic.) 'I was afraid,' says Socrates, 'that I might injure the eye of the soul. I thought that I had better return to the old and safe method of ideas. Though I do not mean to say that he who contemplates existence through the medium of ideas sees only through a glass darkly, any more than he who contemplates actual effects.'

\par  If the existence of ideas is granted to him, Socrates is of opinion that he will then have no difficulty in proving the immortality of the soul. He will only ask for a further admission:—that beauty is the cause of the beautiful, greatness the cause of the great, smallness of the small, and so on of other things. This is a safe and simple answer, which escapes the contradictions of greater and less (greater by reason of that which is smaller! ), of addition and subtraction, and the other difficulties of relation. These subtleties he is for leaving to wiser heads than his own; he prefers to test ideas by the consistency of their consequences, and, if asked to give an account of them, goes back to some higher idea or hypothesis which appears to him to be the best, until at last he arrives at a resting-place. (Republic; Phil.)

\par  The doctrine of ideas, which has long ago received the assent of the Socratic circle, is now affirmed by the Phliasian auditor to command the assent of any man of sense. The narrative is continued; Socrates is desirous of explaining how opposite ideas may appear to co-exist but do not really co-exist in the same thing or person. For example, Simmias may be said to have greatness and also smallness, because he is greater than Socrates and less than Phaedo. And yet Simmias is not really great and also small, but only when compared to Phaedo and Socrates. I use the illustration, says Socrates, because I want to show you not only that ideal opposites exclude one another, but also the opposites in us. I, for example, having the attribute of smallness remain small, and cannot become great: the smallness which is in me drives out greatness.

\par  One of the company here remarked that this was inconsistent with the old assertion that opposites generated opposites. But that, replies Socrates, was affirmed, not of opposite ideas either in us or in nature, but of opposition in the concrete—not of life and death, but of individuals living and dying. When this objection has been removed, Socrates proceeds: This doctrine of the mutual exclusion of opposites is not only true of the opposites themselves, but of things which are inseparable from them. For example, cold and heat are opposed; and fire, which is inseparable from heat, cannot co-exist with cold, or snow, which is inseparable from cold, with heat. Again, the number three excludes the number four, because three is an odd number and four is an even number, and the odd is opposed to the even. Thus we are able to proceed a step beyond 'the safe and simple answer.' We may say, not only that the odd excludes the even, but that the number three, which participates in oddness, excludes the even. And in like manner, not only does life exclude death, but the soul, of which life is the inseparable attribute, also excludes death. And that of which life is the inseparable attribute is by the force of the terms imperishable. If the odd principle were imperishable, then the number three would not perish but remove, on the approach of the even principle. But the immortal is imperishable; and therefore the soul on the approach of death does not perish but removes.

\par  Thus all objections appear to be finally silenced. And now the application has to be made: If the soul is immortal, 'what manner of persons ought we to be?' having regard not only to time but to eternity. For death is not the end of all, and the wicked is not released from his evil by death; but every one carries with him into the world below that which he is or has become, and that only.

\par  For after death the soul is carried away to judgment, and when she has received her punishment returns to earth in the course of ages. The wise soul is conscious of her situation, and follows the attendant angel who guides her through the windings of the world below; but the impure soul wanders hither and thither without companion or guide, and is carried at last to her own place, as the pure soul is also carried away to hers. 'In order that you may understand this, I must first describe to you the nature and conformation of the earth.'

\par  Now the whole earth is a globe placed in the centre of the heavens, and is maintained there by the perfection of balance. That which we call the earth is only one of many small hollows, wherein collect the mists and waters and the thick lower air; but the true earth is above, and is in a finer and subtler element. And if, like birds, we could fly to the surface of the air, in the same manner that fishes come to the top of the sea, then we should behold the true earth and the true heaven and the true stars. Our earth is everywhere corrupted and corroded; and even the land which is fairer than the sea, for that is a mere chaos or waste of water and mud and sand, has nothing to show in comparison of the other world. But the heavenly earth is of divers colours, sparkling with jewels brighter than gold and whiter than any snow, having flowers and fruits innumerable. And the inhabitants dwell some on the shore of the sea of air, others in 'islets of the blest,' and they hold converse with the gods, and behold the sun, moon and stars as they truly are, and their other blessedness is of a piece with this.

\par  The hollows on the surface of the globe vary in size and shape from that which we inhabit: but all are connected by passages and perforations in the interior of the earth. And there is one huge chasm or opening called Tartarus, into which streams of fire and water and liquid mud are ever flowing; of these small portions find their way to the surface and form seas and rivers and volcanoes. There is a perpetual inhalation and exhalation of the air rising and falling as the waters pass into the depths of the earth and return again, in their course forming lakes and rivers, but never descending below the centre of the earth; for on either side the rivers flowing either way are stopped by a precipice. These rivers are many and mighty, and there are four principal ones, Oceanus, Acheron, Pyriphlegethon, and Cocytus. Oceanus is the river which encircles the earth; Acheron takes an opposite direction, and after flowing under the earth through desert places, at last reaches the Acherusian lake,—this is the river at which the souls of the dead await their return to earth. Pyriphlegethon is a stream of fire, which coils round the earth and flows into the depths of Tartarus. The fourth river, Cocytus, is that which is called by the poets the Stygian river, and passes into and forms the lake Styx, from the waters of which it gains new and strange powers. This river, too, falls into Tartarus.

\par  The dead are first of all judged according to their deeds, and those who are incurable are thrust into Tartarus, from which they never come out. Those who have only committed venial sins are first purified of them, and then rewarded for the good which they have done. Those who have committed crimes, great indeed, but not unpardonable, are thrust into Tartarus, but are cast forth at the end of a year by way of Pyriphlegethon or Cocytus, and these carry them as far as the Acherusian lake, where they call upon their victims to let them come out of the rivers into the lake. And if they prevail, then they are let out and their sufferings cease: if not, they are borne unceasingly into Tartarus and back again, until they at last obtain mercy. The pure souls also receive their reward, and have their abode in the upper earth, and a select few in still fairer 'mansions.'

\par  Socrates is not prepared to insist on the literal accuracy of this description, but he is confident that something of the kind is true. He who has sought after the pleasures of knowledge and rejected the pleasures of the body, has reason to be of good hope at the approach of death; whose voice is already speaking to him, and who will one day be heard calling all men.

\par  The hour has come at which he must drink the poison, and not much remains to be done. How shall they bury him? That is a question which he refuses to entertain, for they are burying, not him, but his dead body. His friends had once been sureties that he would remain, and they shall now be sureties that he has run away. Yet he would not die without the customary ceremonies of washing and burial. Shall he make a libation of the poison? In the spirit he will, but not in the letter. One request he utters in the very act of death, which has been a puzzle to after ages. With a sort of irony he remembers that a trifling religious duty is still unfulfilled, just as above he desires before he departs to compose a few verses in order to satisfy a scruple about a dream—unless, indeed, we suppose him to mean, that he was now restored to health, and made the customary offering to Asclepius in token of his recovery.

\par  1. The doctrine of the immortality of the soul has sunk deep into the heart of the human race; and men are apt to rebel against any examination of the nature or grounds of their belief. They do not like to acknowledge that this, as well as the other 'eternal ideas; of man, has a history in time, which may be traced in Greek poetry or philosophy, and also in the Hebrew Scriptures. They convert feeling into reasoning, and throw a network of dialectics over that which is really a deeply-rooted instinct. In the same temper which Socrates reproves in himself they are disposed to think that even fallacies will do no harm, for they will die with them, and while they live they will gain by the delusion. And when they consider the numberless bad arguments which have been pressed into the service of theology, they say, like the companions of Socrates, 'What argument can we ever trust again?' But there is a better and higher spirit to be gathered from the Phaedo, as well as from the other writings of Plato, which says that first principles should be most constantly reviewed (Phaedo and Crat. ), and that the highest subjects demand of us the greatest accuracy (Republic); also that we must not become misologists because arguments are apt to be deceivers.

\par  2. In former ages there was a customary rather than a reasoned belief in the immortality of the soul. It was based on the authority of the Church, on the necessity of such a belief to morality and the order of society, on the evidence of an historical fact, and also on analogies and figures of speech which filled up the void or gave an expression in words to a cherished instinct. The mass of mankind went on their way busy with the affairs of this life, hardly stopping to think about another. But in our own day the question has been reopened, and it is doubtful whether the belief which in the first ages of Christianity was the strongest motive of action can survive the conflict with a scientific age in which the rules of evidence are stricter and the mind has become more sensitive to criticism. It has faded into the distance by a natural process as it was removed further and further from the historical fact on which it has been supposed to rest. Arguments derived from material things such as the seed and the ear of corn or transitions in the life of animals from one state of being to another (the chrysalis and the butterfly) are not 'in pari materia' with arguments from the visible to the invisible, and are therefore felt to be no longer applicable. The evidence to the historical fact seems to be weaker than was once supposed: it is not consistent with itself, and is based upon documents which are of unknown origin. The immortality of man must be proved by other arguments than these if it is again to become a living belief. We must ask ourselves afresh why we still maintain it, and seek to discover a foundation for it in the nature of God and in the first principles of morality.

\par  3. At the outset of the discussion we may clear away a confusion. We certainly do not mean by the immortality of the soul the immortality of fame, which whether worth having or not can only be ascribed to a very select class of the whole race of mankind, and even the interest in these few is comparatively short-lived. To have been a benefactor to the world, whether in a higher or a lower sphere of life and thought, is a great thing: to have the reputation of being one, when men have passed out of the sphere of earthly praise or blame, is hardly worthy of consideration. The memory of a great man, so far from being immortal, is really limited to his own generation:—so long as his friends or his disciples are alive, so long as his books continue to be read, so long as his political or military successes fill a page in the history of his country. The praises which are bestowed upon him at his death hardly last longer than the flowers which are strewed upon his coffin or the 'immortelles' which are laid upon his tomb. Literature makes the most of its heroes, but the true man is well aware that far from enjoying an immortality of fame, in a generation or two, or even in a much shorter time, he will be forgotten and the world will get on without him.

\par  4. Modern philosophy is perplexed at this whole question, which is sometimes fairly given up and handed over to the realm of faith. The perplexity should not be forgotten by us when we attempt to submit the Phaedo of Plato to the requirements of logic. For what idea can we form of the soul when separated from the body? Or how can the soul be united with the body and still be independent? Is the soul related to the body as the ideal to the real, or as the whole to the parts, or as the subject to the object, or as the cause to the effect, or as the end to the means? Shall we say with Aristotle, that the soul is the entelechy or form of an organized living body? or with Plato, that she has a life of her own? Is the Pythagorean image of the harmony, or that of the monad, the truer expression? Is the soul related to the body as sight to the eye, or as the boatman to his boat? (Arist. de Anim.) And in another state of being is the soul to be conceived of as vanishing into infinity, hardly possessing an existence which she can call her own, as in the pantheistic system of Spinoza: or as an individual informing another body and entering into new relations, but retaining her own character? (Compare Gorgias.) Or is the opposition of soul and body a mere illusion, and the true self neither soul nor body, but the union of the two in the 'I' which is above them? And is death the assertion of this individuality in the higher nature, and the falling away into nothingness of the lower? Or are we vainly attempting to pass the boundaries of human thought? The body and the soul seem to be inseparable, not only in fact, but in our conceptions of them; and any philosophy which too closely unites them, or too widely separates them, either in this life or in another, disturbs the balance of human nature. No thinker has perfectly adjusted them, or been entirely consistent with himself in describing their relation to one another. Nor can we wonder that Plato in the infancy of human thought should have confused mythology and philosophy, or have mistaken verbal arguments for real ones.

\par  5. Again, believing in the immortality of the soul, we must still ask the question of Socrates, 'What is that which we suppose to be immortal?' Is it the personal and individual element in us, or the spiritual and universal? Is it the principle of knowledge or of goodness, or the union of the two? Is it the mere force of life which is determined to be, or the consciousness of self which cannot be got rid of, or the fire of genius which refuses to be extinguished? Or is there a hidden being which is allied to the Author of all existence, who is because he is perfect, and to whom our ideas of perfection give us a title to belong? Whatever answer is given by us to these questions, there still remains the necessity of allowing the permanence of evil, if not for ever, at any rate for a time, in order that the wicked 'may not have too good a bargain.' For the annihilation of evil at death, or the eternal duration of it, seem to involve equal difficulties in the moral government of the universe. Sometimes we are led by our feelings, rather than by our reason, to think of the good and wise only as existing in another life. Why should the mean, the weak, the idiot, the infant, the herd of men who have never in any proper sense the use of reason, reappear with blinking eyes in the light of another world? But our second thought is that the hope of humanity is a common one, and that all or none will be partakers of immortality. Reason does not allow us to suppose that we have any greater claims than others, and experience may often reveal to us unexpected flashes of the higher nature in those whom we had despised. Why should the wicked suffer any more than ourselves? had we been placed in their circumstances should we have been any better than they? The worst of men are objects of pity rather than of anger to the philanthropist; must they not be equally such to divine benevolence? Even more than the good they have need of another life; not that they may be punished, but that they may be educated. These are a few of the reflections which arise in our minds when we attempt to assign any form to our conceptions of a future state.

\par  There are some other questions which are disturbing to us because we have no answer to them. What is to become of the animals in a future state? Have we not seen dogs more faithful and intelligent than men, and men who are more stupid and brutal than any animals? Does their life cease at death, or is there some 'better thing reserved' also for them? They may be said to have a shadow or imitation of morality, and imperfect moral claims upon the benevolence of man and upon the justice of God. We cannot think of the least or lowest of them, the insect, the bird, the inhabitants of the sea or the desert, as having any place in a future world, and if not all, why should those who are specially attached to man be deemed worthy of any exceptional privilege? When we reason about such a subject, almost at once we degenerate into nonsense. It is a passing thought which has no real hold on the mind. We may argue for the existence of animals in a future state from the attributes of God, or from texts of Scripture ('Are not two sparrows sold for one farthing?' etc. ), but the truth is that we are only filling up the void of another world with our own fancies. Again, we often talk about the origin of evil, that great bugbear of theologians, by which they frighten us into believing any superstition. What answer can be made to the old commonplace, 'Is not God the author of evil, if he knowingly permitted, but could have prevented it?' Even if we assume that the inequalities of this life are rectified by some transposition of human beings in another, still the existence of the very least evil if it could have been avoided, seems to be at variance with the love and justice of God. And so we arrive at the conclusion that we are carrying logic too far, and that the attempt to frame the world according to a rule of divine perfection is opposed to experience and had better be given up. The case of the animals is our own. We must admit that the Divine Being, although perfect himself, has placed us in a state of life in which we may work together with him for good, but we are very far from having attained to it.

\par  6. Again, ideas must be given through something; and we are always prone to argue about the soul from analogies of outward things which may serve to embody our thoughts, but are also partly delusive. For we cannot reason from the natural to the spiritual, or from the outward to the inward. The progress of physiological science, without bringing us nearer to the great secret, has tended to remove some erroneous notions respecting the relations of body and mind, and in this we have the advantage of the ancients. But no one imagines that any seed of immortality is to be discerned in our mortal frames. Most people have been content to rest their belief in another life on the agreement of the more enlightened part of mankind, and on the inseparable connection of such a doctrine with the existence of a God—also in a less degree on the impossibility of doubting about the continued existence of those whom we love and reverence in this world. And after all has been said, the figure, the analogy, the argument, are felt to be only approximations in different forms to an expression of the common sentiment of the human heart. That we shall live again is far more certain than that we shall take any particular form of life.

\par  7. When we speak of the immortality of the soul, we must ask further what we mean by the word immortality. For of the duration of a living being in countless ages we can form no conception; far less than a three years' old child of the whole of life. The naked eye might as well try to see the furthest star in the infinity of heaven. Whether time and space really exist when we take away the limits of them may be doubted; at any rate the thought of them when unlimited us so overwhelming to us as to lose all distinctness. Philosophers have spoken of them as forms of the human mind, but what is the mind without them? As then infinite time, or an existence out of time, which are the only possible explanations of eternal duration, are equally inconceivable to us, let us substitute for them a hundred or a thousand years after death, and ask not what will be our employment in eternity, but what will happen to us in that definite portion of time; or what is now happening to those who passed out of life a hundred or a thousand years ago. Do we imagine that the wicked are suffering torments, or that the good are singing the praises of God, during a period longer than that of a whole life, or of ten lives of men? Is the suffering physical or mental? And does the worship of God consist only of praise, or of many forms of service? Who are the wicked, and who are the good, whom we venture to divide by a hard and fast line; and in which of the two classes should we place ourselves and our friends? May we not suspect that we are making differences of kind, because we are unable to imagine differences of degree?—putting the whole human race into heaven or hell for the greater convenience of logical division? Are we not at the same time describing them both in superlatives, only that we may satisfy the demands of rhetoric? What is that pain which does not become deadened after a thousand years? or what is the nature of that pleasure or happiness which never wearies by monotony? Earthly pleasures and pains are short in proportion as they are keen; of any others which are both intense and lasting we have no experience, and can form no idea. The words or figures of speech which we use are not consistent with themselves. For are we not imagining Heaven under the similitude of a church, and Hell as a prison, or perhaps a madhouse or chamber of horrors? And yet to beings constituted as we are, the monotony of singing psalms would be as great an infliction as the pains of hell, and might be even pleasantly interrupted by them. Where are the actions worthy of rewards greater than those which are conferred on the greatest benefactors of mankind? And where are the crimes which according to Plato's merciful reckoning,—more merciful, at any rate, than the eternal damnation of so-called Christian teachers,—for every ten years in this life deserve a hundred of punishment in the life to come? We should be ready to die of pity if we could see the least of the sufferings which the writers of Infernos and Purgatorios have attributed to the damned. Yet these joys and terrors seem hardly to exercise an appreciable influence over the lives of men. The wicked man when old, is not, as Plato supposes (Republic), more agitated by the terrors of another world when he is nearer to them, nor the good in an ecstasy at the joys of which he is soon to be the partaker. Age numbs the sense of both worlds; and the habit of life is strongest in death. Even the dying mother is dreaming of her lost children as they were forty or fifty years before, 'pattering over the boards,' not of reunion with them in another state of being. Most persons when the last hour comes are resigned to the order of nature and the will of God. They are not thinking of Dante's Inferno or Paradiso, or of the Pilgrim's Progress. Heaven and hell are not realities to them, but words or ideas; the outward symbols of some great mystery, they hardly know what. Many noble poems and pictures have been suggested by the traditional representations of them, which have been fixed in forms of art and can no longer be altered. Many sermons have been filled with descriptions of celestial or infernal mansions. But hardly even in childhood did the thought of heaven and hell supply the motives of our actions, or at any time seriously affect the substance of our belief.

\par  8. Another life must be described, if at all, in forms of thought and not of sense. To draw pictures of heaven and hell, whether in the language of Scripture or any other, adds nothing to our real knowledge, but may perhaps disguise our ignorance. The truest conception which we can form of a future life is a state of progress or education—a progress from evil to good, from ignorance to knowledge. To this we are led by the analogy of the present life, in which we see different races and nations of men, and different men and women of the same nation, in various states or stages of cultivation; some more and some less developed, and all of them capable of improvement under favourable circumstances. There are punishments too of children when they are growing up inflicted by their parents, of elder offenders which are imposed by the law of the land, of all men at all times of life, which are attached by the laws of nature to the performance of certain actions. All these punishments are really educational; that is to say, they are not intended to retaliate on the offender, but to teach him a lesson. Also there is an element of chance in them, which is another name for our ignorance of the laws of nature. There is evil too inseparable from good (compare Lysis); not always punished here, as good is not always rewarded. It is capable of being indefinitely diminished; and as knowledge increases, the element of chance may more and more disappear.

\par  For we do not argue merely from the analogy of the present state of this world to another, but from the analogy of a probable future to which we are tending. The greatest changes of which we have had experience as yet are due to our increasing knowledge of history and of nature. They have been produced by a few minds appearing in three or four favoured nations, in a comparatively short period of time. May we be allowed to imagine the minds of men everywhere working together during many ages for the completion of our knowledge? May not the science of physiology transform the world? Again, the majority of mankind have really experienced some moral improvement; almost every one feels that he has tendencies to good, and is capable of becoming better. And these germs of good are often found to be developed by new circumstances, like stunted trees when transplanted to a better soil. The differences between the savage and the civilized man, or between the civilized man in old and new countries, may be indefinitely increased. The first difference is the effect of a few thousand, the second of a few hundred years. We congratulate ourselves that slavery has become industry; that law and constitutional government have superseded despotism and violence; that an ethical religion has taken the place of Fetichism. There may yet come a time when the many may be as well off as the few; when no one will be weighed down by excessive toil; when the necessity of providing for the body will not interfere with mental improvement; when the physical frame may be strengthened and developed; and the religion of all men may become a reasonable service.

\par  Nothing therefore, either in the present state of man or in the tendencies of the future, as far as we can entertain conjecture of them, would lead us to suppose that God governs us vindictively in this world, and therefore we have no reason to infer that he will govern us vindictively in another. The true argument from analogy is not, 'This life is a mixed state of justice and injustice, of great waste, of sudden casualties, of disproportionate punishments, and therefore the like inconsistencies, irregularities, injustices are to be expected in another;' but 'This life is subject to law, and is in a state of progress, and therefore law and progress may be believed to be the governing principles of another.' All the analogies of this world would be against unmeaning punishments inflicted a hundred or a thousand years after an offence had been committed. Suffering there might be as a part of education, but not hopeless or protracted; as there might be a retrogression of individuals or of bodies of men, yet not such as to interfere with a plan for the improvement of the whole (compare Laws.)

\par  9. But some one will say: That we cannot reason from the seen to the unseen, and that we are creating another world after the image of this, just as men in former ages have created gods in their own likeness. And we, like the companions of Socrates, may feel discouraged at hearing our favourite 'argument from analogy' thus summarily disposed of. Like himself, too, we may adduce other arguments in which he seems to have anticipated us, though he expresses them in different language. For we feel that the soul partakes of the ideal and invisible; and can never fall into the error of confusing the external circumstances of man with his higher self; or his origin with his nature. It is as repugnant to us as it was to him to imagine that our moral ideas are to be attributed only to cerebral forces. The value of a human soul, like the value of a man's life to himself, is inestimable, and cannot be reckoned in earthly or material things. The human being alone has the consciousness of truth and justice and love, which is the consciousness of God. And the soul becoming more conscious of these, becomes more conscious of her own immortality.

\par  10. The last ground of our belief in immortality, and the strongest, is the perfection of the divine nature. The mere fact of the existence of God does not tend to show the continued existence of man. An evil God or an indifferent God might have had the power, but not the will, to preserve us. He might have regarded us as fitted to minister to his service by a succession of existences,—like the animals, without attributing to each soul an incomparable value. But if he is perfect, he must will that all rational beings should partake of that perfection which he himself is. In the words of the Timaeus, he is good, and therefore he desires that all other things should be as like himself as possible. And the manner in which he accomplishes this is by permitting evil, or rather degrees of good, which are otherwise called evil. For all progress is good relatively to the past, and yet may be comparatively evil when regarded in the light of the future. Good and evil are relative terms, and degrees of evil are merely the negative aspect of degrees of good. Of the absolute goodness of any finite nature we can form no conception; we are all of us in process of transition from one degree of good or evil to another. The difficulties which are urged about the origin or existence of evil are mere dialectical puzzles, standing in the same relation to Christian philosophy as the puzzles of the Cynics and Megarians to the philosophy of Plato. They arise out of the tendency of the human mind to regard good and evil both as relative and absolute; just as the riddles about motion are to be explained by the double conception of space or matter, which the human mind has the power of regarding either as continuous or discrete.

\par  In speaking of divine perfection, we mean to say that God is just and true and loving, the author of order and not of disorder, of good and not of evil. Or rather, that he is justice, that he is truth, that he is love, that he is order, that he is the very progress of which we were speaking; and that wherever these qualities are present, whether in the human soul or in the order of nature, there is God. We might still see him everywhere, if we had not been mistakenly seeking for him apart from us, instead of in us; away from the laws of nature, instead of in them. And we become united to him not by mystical absorption, but by partaking, whether consciously or unconsciously, of that truth and justice and love which he himself is.

\par  Thus the belief in the immortality of the soul rests at last on the belief in God. If there is a good and wise God, then there is a progress of mankind towards perfection; and if there is no progress of men towards perfection, then there is no good and wise God. We cannot suppose that the moral government of God of which we see the beginnings in the world and in ourselves will cease when we pass out of life.

\par  11. Considering the 'feebleness of the human faculties and the uncertainty of the subject,' we are inclined to believe that the fewer our words the better. At the approach of death there is not much said; good men are too honest to go out of the world professing more than they know. There is perhaps no important subject about which, at any time, even religious people speak so little to one another. In the fulness of life the thought of death is mostly awakened by the sight or recollection of the death of others rather than by the prospect of our own. We must also acknowledge that there are degrees of the belief in immortality, and many forms in which it presents itself to the mind. Some persons will say no more than that they trust in God, and that they leave all to Him. It is a great part of true religion not to pretend to know more than we do. Others when they quit this world are comforted with the hope 'That they will see and know their friends in heaven.' But it is better to leave them in the hands of God and to be assured that 'no evil shall touch them.' There are others again to whom the belief in a divine personality has ceased to have any longer a meaning; yet they are satisfied that the end of all is not here, but that something still remains to us, 'and some better thing for the good than for the evil.' They are persuaded, in spite of their theological nihilism, that the ideas of justice and truth and holiness and love are realities. They cherish an enthusiastic devotion to the first principles of morality. Through these they see, or seem to see, darkly, and in a figure, that the soul is immortal.

\par  But besides differences of theological opinion which must ever prevail about things unseen, the hope of immortality is weaker or stronger in men at one time of life than at another; it even varies from day to day. It comes and goes; the mind, like the sky, is apt to be overclouded. Other generations of men may have sometimes lived under an 'eclipse of faith,' to us the total disappearance of it might be compared to the 'sun falling from heaven.' And we may sometimes have to begin again and acquire the belief for ourselves; or to win it back again when it is lost. It is really weakest in the hour of death. For Nature, like a kind mother or nurse, lays us to sleep without frightening us; physicians, who are the witnesses of such scenes, say that under ordinary circumstances there is no fear of the future. Often, as Plato tells us, death is accompanied 'with pleasure.' (Tim.) When the end is still uncertain, the cry of many a one has been, 'Pray, that I may be taken.' The last thoughts even of the best men depend chiefly on the accidents of their bodily state. Pain soon overpowers the desire of life; old age, like the child, is laid to sleep almost in a moment. The long experience of life will often destroy the interest which mankind have in it. So various are the feelings with which different persons draw near to death; and still more various the forms in which imagination clothes it. For this alternation of feeling compare the Old Testament,—Psalm vi. ; Isaiah; Eccles.

\par  12. When we think of God and of man in his relation to God; of the imperfection of our present state and yet of the progress which is observable in the history of the world and of the human mind; of the depth and power of our moral ideas which seem to partake of the very nature of God Himself; when we consider the contrast between the physical laws to which we are subject and the higher law which raises us above them and is yet a part of them; when we reflect on our capacity of becoming the 'spectators of all time and all existence,' and of framing in our own minds the ideal of a perfect Being; when we see how the human mind in all the higher religions of the world, including Buddhism, notwithstanding some aberrations, has tended towards such a belief—we have reason to think that our destiny is different from that of animals; and though we cannot altogether shut out the childish fear that the soul upon leaving the body may 'vanish into thin air,' we have still, so far as the nature of the subject admits, a hope of immortality with which we comfort ourselves on sufficient grounds. The denial of the belief takes the heart out of human life; it lowers men to the level of the material. As Goethe also says, 'He is dead even in this world who has no belief in another.'

\par  13. It is well also that we should sometimes think of the forms of thought under which the idea of immortality is most naturally presented to us. It is clear that to our minds the risen soul can no longer be described, as in a picture, by the symbol of a creature half-bird, half-human, nor in any other form of sense. The multitude of angels, as in Milton, singing the Almighty's praises, are a noble image, and may furnish a theme for the poet or the painter, but they are no longer an adequate expression of the kingdom of God which is within us. Neither is there any mansion, in this world or another, in which the departed can be imagined to dwell and carry on their occupations. When this earthly tabernacle is dissolved, no other habitation or building can take them in: it is in the language of ideas only that we speak of them.

\par  First of all there is the thought of rest and freedom from pain; they have gone home, as the common saying is, and the cares of this world touch them no more. Secondly, we may imagine them as they were at their best and brightest, humbly fulfilling their daily round of duties—selfless, childlike, unaffected by the world; when the eye was single and the whole body seemed to be full of light; when the mind was clear and saw into the purposes of God. Thirdly, we may think of them as possessed by a great love of God and man, working out His will at a further stage in the heavenly pilgrimage. And yet we acknowledge that these are the things which eye hath not seen nor ear heard and therefore it hath not entered into the heart of man in any sensible manner to conceive them. Fourthly, there may have been some moments in our own lives when we have risen above ourselves, or been conscious of our truer selves, in which the will of God has superseded our wills, and we have entered into communion with Him, and been partakers for a brief season of the Divine truth and love, in which like Christ we have been inspired to utter the prayer, 'I in them, and thou in me, that we may be all made perfect in one.' These precious moments, if we have ever known them, are the nearest approach which we can make to the idea of immortality.

\par  14. Returning now to the earlier stage of human thought which is represented by the writings of Plato, we find that many of the same questions have already arisen: there is the same tendency to materialism; the same inconsistency in the application of the idea of mind; the same doubt whether the soul is to be regarded as a cause or as an effect; the same falling back on moral convictions. In the Phaedo the soul is conscious of her divine nature, and the separation from the body which has been commenced in this life is perfected in another. Beginning in mystery, Socrates, in the intermediate part of the Dialogue, attempts to bring the doctrine of a future life into connection with his theory of knowledge. In proportion as he succeeds in this, the individual seems to disappear in a more general notion of the soul; the contemplation of ideas 'under the form of eternity' takes the place of past and future states of existence. His language may be compared to that of some modern philosophers, who speak of eternity, not in the sense of perpetual duration of time, but as an ever-present quality of the soul. Yet at the conclusion of the Dialogue, having 'arrived at the end of the intellectual world' (Republic), he replaces the veil of mythology, and describes the soul and her attendant genius in the language of the mysteries or of a disciple of Zoroaster. Nor can we fairly demand of Plato a consistency which is wanting among ourselves, who acknowledge that another world is beyond the range of human thought, and yet are always seeking to represent the mansions of heaven or hell in the colours of the painter, or in the descriptions of the poet or rhetorician.

\par  15. The doctrine of the immortality of the soul was not new to the Greeks in the age of Socrates, but, like the unity of God, had a foundation in the popular belief. The old Homeric notion of a gibbering ghost flitting away to Hades; or of a few illustrious heroes enjoying the isles of the blest; or of an existence divided between the two; or the Hesiodic, of righteous spirits, who become guardian angels,—had given place in the mysteries and the Orphic poets to representations, partly fanciful, of a future state of rewards and punishments. (Laws.) The reticence of the Greeks on public occasions and in some part of their literature respecting this 'underground' religion, is not to be taken as a measure of the diffusion of such beliefs. If Pericles in the funeral oration is silent on the consolations of immortality, the poet Pindar and the tragedians on the other hand constantly assume the continued existence of the dead in an upper or under world. Darius and Laius are still alive; Antigone will be dear to her brethren after death; the way to the palace of Cronos is found by those who 'have thrice departed from evil.' The tragedy of the Greeks is not 'rounded' by this life, but is deeply set in decrees of fate and mysterious workings of powers beneath the earth. In the caricature of Aristophanes there is also a witness to the common sentiment. The Ionian and Pythagorean philosophies arose, and some new elements were added to the popular belief. The individual must find an expression as well as the world. Either the soul was supposed to exist in the form of a magnet, or of a particle of fire, or of light, or air, or water; or of a number or of a harmony of number; or to be or have, like the stars, a principle of motion (Arist. de Anim.). At length Anaxagoras, hardly distinguishing between life and mind, or between mind human and divine, attained the pure abstraction; and this, like the other abstractions of Greek philosophy, sank deep into the human intelligence. The opposition of the intelligible and the sensible, and of God to the world, supplied an analogy which assisted in the separation of soul and body. If ideas were separable from phenomena, mind was also separable from matter; if the ideas were eternal, the mind that conceived them was eternal too. As the unity of God was more distinctly acknowledged, the conception of the human soul became more developed. The succession, or alternation of life and death, had occurred to Heracleitus. The Eleatic Parmenides had stumbled upon the modern thesis, that 'thought and being are the same.' The Eastern belief in transmigration defined the sense of individuality; and some, like Empedocles, fancied that the blood which they had shed in another state of being was crying against them, and that for thirty thousand years they were to be 'fugitives and vagabonds upon the earth.' The desire of recognizing a lost mother or love or friend in the world below (Phaedo) was a natural feeling which, in that age as well as in every other, has given distinctness to the hope of immortality. Nor were ethical considerations wanting, partly derived from the necessity of punishing the greater sort of criminals, whom no avenging power of this world could reach. The voice of conscience, too, was heard reminding the good man that he was not altogether innocent. (Republic.) To these indistinct longings and fears an expression was given in the mysteries and Orphic poets: a 'heap of books' (Republic), passing under the names of Musaeus and Orpheus in Plato's time, were filled with notions of an under-world.

\par  16. Yet after all the belief in the individuality of the soul after death had but a feeble hold on the Greek mind. Like the personality of God, the personality of man in a future state was not inseparably bound up with the reality of his existence. For the distinction between the personal and impersonal, and also between the divine and human, was far less marked to the Greek than to ourselves. And as Plato readily passes from the notion of the good to that of God, he also passes almost imperceptibly to himself and his reader from the future life of the individual soul to the eternal being of the absolute soul. There has been a clearer statement and a clearer denial of the belief in modern times than is found in early Greek philosophy, and hence the comparative silence on the whole subject which is often remarked in ancient writers, and particularly in Aristotle. For Plato and Aristotle are not further removed in their teaching about the immortality of the soul than they are in their theory of knowledge.

\par  17. Living in an age when logic was beginning to mould human thought, Plato naturally cast his belief in immortality into a logical form. And when we consider how much the doctrine of ideas was also one of words, it is not surprising that he should have fallen into verbal fallacies: early logic is always mistaking the truth of the form for the truth of the matter. It is easy to see that the alternation of opposites is not the same as the generation of them out of each other; and that the generation of them out of each other, which is the first argument in the Phaedo, is at variance with their mutual exclusion of each other, whether in themselves or in us, which is the last. For even if we admit the distinction which he draws between the opposites and the things which have the opposites, still individuals fall under the latter class; and we have to pass out of the region of human hopes and fears to a conception of an abstract soul which is the impersonation of the ideas. Such a conception, which in Plato himself is but half expressed, is unmeaning to us, and relative only to a particular stage in the history of thought. The doctrine of reminiscence is also a fragment of a former world, which has no place in the philosophy of modern times. But Plato had the wonders of psychology just opening to him, and he had not the explanation of them which is supplied by the analysis of language and the history of the human mind. The question, 'Whence come our abstract ideas?' he could only answer by an imaginary hypothesis. Nor is it difficult to see that his crowning argument is purely verbal, and is but the expression of an instinctive confidence put into a logical form:—'The soul is immortal because it contains a principle of imperishableness.' Nor does he himself seem at all to be aware that nothing is added to human knowledge by his 'safe and simple answer,' that beauty is the cause of the beautiful; and that he is merely reasserting the Eleatic being 'divided by the Pythagorean numbers,' against the Heracleitean doctrine of perpetual generation. The answer to the 'very serious question' of generation and destruction is really the denial of them. For this he would substitute, as in the Republic, a system of ideas, tested, not by experience, but by their consequences, and not explained by actual causes, but by a higher, that is, a more general notion. Consistency with themselves is the only test which is to be applied to them. (Republic, and Phaedo.)

\par  18. To deal fairly with such arguments, they should be translated as far as possible into their modern equivalents. 'If the ideas of men are eternal, their souls are eternal, and if not the ideas, then not the souls.' Such an argument stands nearly in the same relation to Plato and his age, as the argument from the existence of God to immortality among ourselves. 'If God exists, then the soul exists after death; and if there is no God, there is no existence of the soul after death.' For the ideas are to his mind the reality, the truth, the principle of permanence, as well as of intelligence and order in the world. When Simmias and Cebes say that they are more strongly persuaded of the existence of ideas than they are of the immortality of the soul, they represent fairly enough the order of thought in Greek philosophy. And we might say in the same way that we are more certain of the existence of God than we are of the immortality of the soul, and are led by the belief in the one to a belief in the other. The parallel, as Socrates would say, is not perfect, but agrees in as far as the mind in either case is regarded as dependent on something above and beyond herself. The analogy may even be pressed a step further: 'We are more certain of our ideas of truth and right than we are of the existence of God, and are led on in the order of thought from one to the other.' Or more correctly: 'The existence of right and truth is the existence of God, and can never for a moment be separated from Him.'

\par  19. The main argument of the Phaedo is derived from the existence of eternal ideas of which the soul is a partaker; the other argument of the alternation of opposites is replaced by this. And there have not been wanting philosophers of the idealist school who have imagined that the doctrine of the immortality of the soul is a theory of knowledge, and that in what has preceded Plato is accommodating himself to the popular belief. Such a view can only be elicited from the Phaedo by what may be termed the transcendental method of interpretation, and is obviously inconsistent with the Gorgias and the Republic. Those who maintain it are immediately compelled to renounce the shadow which they have grasped, as a play of words only. But the truth is, that Plato in his argument for the immortality of the soul has collected many elements of proof or persuasion, ethical and mythological as well as dialectical, which are not easily to be reconciled with one another; and he is as much in earnest about his doctrine of retribution, which is repeated in all his more ethical writings, as about his theory of knowledge. And while we may fairly translate the dialectical into the language of Hegel, and the religious and mythological into the language of Dante or Bunyan, the ethical speaks to us still in the same voice, and appeals to a common feeling.

\par  20. Two arguments of this ethical character occur in the Phaedo. The first may be described as the aspiration of the soul after another state of being. Like the Oriental or Christian mystic, the philosopher is seeking to withdraw from impurities of sense, to leave the world and the things of the world, and to find his higher self. Plato recognizes in these aspirations the foretaste of immortality; as Butler and Addison in modern times have argued, the one from the moral tendencies of mankind, the other from the progress of the soul towards perfection. In using this argument Plato has certainly confused the soul which has left the body, with the soul of the good and wise. (Compare Republic.) Such a confusion was natural, and arose partly out of the antithesis of soul and body. The soul in her own essence, and the soul 'clothed upon' with virtues and graces, were easily interchanged with one another, because on a subject which passes expression the distinctions of language can hardly be maintained.

\par  21. The ethical proof of the immortality of the soul is derived from the necessity of retribution. The wicked would be too well off if their evil deeds came to an end. It is not to be supposed that an Ardiaeus, an Archelaus, an Ismenias could ever have suffered the penalty of their crimes in this world. The manner in which this retribution is accomplished Plato represents under the figures of mythology. Doubtless he felt that it was easier to improve than to invent, and that in religion especially the traditional form was required in order to give verisimilitude to the myth. The myth too is far more probable to that age than to ours, and may fairly be regarded as 'one guess among many' about the nature of the earth, which he cleverly supports by the indications of geology. Not that he insists on the absolute truth of his own particular notions: 'no man of sense will be confident in such matters; but he will be confident that something of the kind is true.' As in other passages (Gorg., Tim., compare Crito), he wins belief for his fictions by the moderation of his statements; he does not, like Dante or Swedenborg, allow himself to be deceived by his own creations.

\par  The Dialogue must be read in the light of the situation. And first of all we are struck by the calmness of the scene. Like the spectators at the time, we cannot pity Socrates; his mien and his language are so noble and fearless. He is the same that he ever was, but milder and gentler, and he has in no degree lost his interest in dialectics; he will not forego the delight of an argument in compliance with the jailer's intimation that he should not heat himself with talking. At such a time he naturally expresses the hope of his life, that he has been a true mystic and not a mere retainer or wand-bearer: and he refers to passages of his personal history. To his old enemies the Comic poets, and to the proceedings on the trial, he alludes playfully; but he vividly remembers the disappointment which he felt in reading the books of Anaxagoras. The return of Xanthippe and his children indicates that the philosopher is not 'made of oak or rock.' Some other traits of his character may be noted; for example, the courteous manner in which he inclines his head to the last objector, or the ironical touch, 'Me already, as the tragic poet would say, the voice of fate calls;' or the depreciation of the arguments with which 'he comforted himself and them;' or his fear of 'misology;' or his references to Homer; or the playful smile with which he 'talks like a book' about greater and less; or the allusion to the possibility of finding another teacher among barbarous races (compare Polit. ); or the mysterious reference to another science (mathematics?) of generation and destruction for which he is vainly feeling. There is no change in him; only now he is invested with a sort of sacred character, as the prophet or priest of Apollo the God of the festival, in whose honour he first of all composes a hymn, and then like the swan pours forth his dying lay. Perhaps the extreme elevation of Socrates above his own situation, and the ordinary interests of life (compare his jeu d'esprit about his burial, in which for a moment he puts on the 'Silenus mask'), create in the mind of the reader an impression stronger than could be derived from arguments that such a one has in him 'a principle which does not admit of death.'

\par  The other persons of the Dialogue may be considered under two heads: (1) private friends; (2) the respondents in the argument.

\par  First there is Crito, who has been already introduced to us in the Euthydemus and the Crito; he is the equal in years of Socrates, and stands in quite a different relation to him from his younger disciples. He is a man of the world who is rich and prosperous (compare the jest in the Euthydemus), the best friend of Socrates, who wants to know his commands, in whose presence he talks to his family, and who performs the last duty of closing his eyes. It is observable too that, as in the Euthydemus, Crito shows no aptitude for philosophical discussions. Nor among the friends of Socrates must the jailer be forgotten, who seems to have been introduced by Plato in order to show the impression made by the extraordinary man on the common. The gentle nature of the man is indicated by his weeping at the announcement of his errand and then turning away, and also by the words of Socrates to his disciples: 'How charming the man is! since I have been in prison he has been always coming to me, and is as good as could be to me.' We are reminded too that he has retained this gentle nature amid scenes of death and violence by the contrasts which he draws between the behaviour of Socrates and of others when about to die.

\par  Another person who takes no part in the philosophical discussion is the excitable Apollodorus, the same who, in the Symposium, of which he is the narrator, is called 'the madman,' and who testifies his grief by the most violent emotions. Phaedo is also present, the 'beloved disciple' as he may be termed, who is described, if not 'leaning on his bosom,' as seated next to Socrates, who is playing with his hair. He too, like Apollodorus, takes no part in the discussion, but he loves above all things to hear and speak of Socrates after his death. The calmness of his behaviour, veiling his face when he can no longer restrain his tears, contrasts with the passionate outcries of the other. At a particular point the argument is described as falling before the attack of Simmias. A sort of despair is introduced in the minds of the company. The effect of this is heightened by the description of Phaedo, who has been the eye-witness of the scene, and by the sympathy of his Phliasian auditors who are beginning to think 'that they too can never trust an argument again.' And the intense interest of the company is communicated not only to the first auditors, but to us who in a distant country read the narrative of their emotions after more than two thousand years have passed away.

\par  The two principal interlocutors are Simmias and Cebes, the disciples of Philolaus the Pythagorean philosopher of Thebes. Simmias is described in the Phaedrus as fonder of an argument than any man living; and Cebes, although finally persuaded by Socrates, is said to be the most incredulous of human beings. It is Cebes who at the commencement of the Dialogue asks why 'suicide is held to be unlawful,' and who first supplies the doctrine of recollection in confirmation of the pre-existence of the soul. It is Cebes who urges that the pre-existence does not necessarily involve the future existence of the soul, as is shown by the illustration of the weaver and his coat. Simmias, on the other hand, raises the question about harmony and the lyre, which is naturally put into the mouth of a Pythagorean disciple. It is Simmias, too, who first remarks on the uncertainty of human knowledge, and only at last concedes to the argument such a qualified approval as is consistent with the feebleness of the human faculties. Cebes is the deeper and more consecutive thinker, Simmias more superficial and rhetorical; they are distinguished in much the same manner as Adeimantus and Glaucon in the Republic.

\par  Other persons, Menexenus, Ctesippus, Lysis, are old friends; Evenus has been already satirized in the Apology; Aeschines and Epigenes were present at the trial; Euclid and Terpsion will reappear in the Introduction to the Theaetetus, Hermogenes has already appeared in the Cratylus. No inference can fairly be drawn from the absence of Aristippus, nor from the omission of Xenophon, who at the time of Socrates' death was in Asia. The mention of Plato's own absence seems like an expression of sorrow, and may, perhaps, be an indication that the report of the conversation is not to be taken literally.

\par  The place of the Dialogue in the series is doubtful. The doctrine of ideas is certainly carried beyond the Socratic point of view; in no other of the writings of Plato is the theory of them so completely developed. Whether the belief in immortality can be attributed to Socrates or not is uncertain; the silence of the Memorabilia, and of the earlier Dialogues of Plato, is an argument to the contrary. Yet in the Cyropaedia Xenophon has put language into the mouth of the dying Cyrus which recalls the Phaedo, and may have been derived from the teaching of Socrates. It may be fairly urged that the greatest religious interest of mankind could not have been wholly ignored by one who passed his life in fulfilling the commands of an oracle, and who recognized a Divine plan in man and nature. (Xen. Mem.) And the language of the Apology and of the Crito confirms this view.

\par  The Phaedo is not one of the Socratic Dialogues of Plato; nor, on the other hand, can it be assigned to that later stage of the Platonic writings at which the doctrine of ideas appears to be forgotten. It belongs rather to the intermediate period of the Platonic philosophy, which roughly corresponds to the Phaedrus, Gorgias, Republic, Theaetetus. Without pretending to determine the real time of their composition, the Symposium, Meno, Euthyphro, Apology, Phaedo may be conveniently read by us in this order as illustrative of the life of Socrates. Another chain may be formed of the Meno, Phaedrus, Phaedo, in which the immortality of the soul is connected with the doctrine of ideas. In the Meno the theory of ideas is based on the ancient belief in transmigration, which reappears again in the Phaedrus as well as in the Republic and Timaeus, and in all of them is connected with a doctrine of retribution. In the Phaedrus the immortality of the soul is supposed to rest on the conception of the soul as a principle of motion, whereas in the Republic the argument turns on the natural continuance of the soul, which, if not destroyed by her own proper evil, can hardly be destroyed by any other. The soul of man in the Timaeus is derived from the Supreme Creator, and either returns after death to her kindred star, or descends into the lower life of an animal. The Apology expresses the same view as the Phaedo, but with less confidence; there the probability of death being a long sleep is not excluded. The Theaetetus also describes, in a digression, the desire of the soul to fly away and be with God—'and to fly to him is to be like him.' The Symposium may be observed to resemble as well as to differ from the Phaedo. While the first notion of immortality is only in the way of natural procreation or of posthumous fame and glory, the higher revelation of beauty, like the good in the Republic, is the vision of the eternal idea. So deeply rooted in Plato's mind is the belief in immortality; so various are the forms of expression which he employs.

\par  As in several other Dialogues, there is more of system in the Phaedo than appears at first sight. The succession of arguments is based on previous philosophies; beginning with the mysteries and the Heracleitean alternation of opposites, and proceeding to the Pythagorean harmony and transmigration; making a step by the aid of Platonic reminiscence, and a further step by the help of the nous of Anaxagoras; until at last we rest in the conviction that the soul is inseparable from the ideas, and belongs to the world of the invisible and unknown. Then, as in the Gorgias or Republic, the curtain falls, and the veil of mythology descends upon the argument. After the confession of Socrates that he is an interested party, and the acknowledgment that no man of sense will think the details of his narrative true, but that something of the kind is true, we return from speculation to practice. He is himself more confident of immortality than he is of his own arguments; and the confidence which he expresses is less strong than that which his cheerfulness and composure in death inspire in us.

\par  Difficulties of two kinds occur in the Phaedo—one kind to be explained out of contemporary philosophy, the other not admitting of an entire solution. (1) The difficulty which Socrates says that he experienced in explaining generation and corruption; the assumption of hypotheses which proceed from the less general to the more general, and are tested by their consequences; the puzzle about greater and less; the resort to the method of ideas, which to us appear only abstract terms,—these are to be explained out of the position of Socrates and Plato in the history of philosophy. They were living in a twilight between the sensible and the intellectual world, and saw no way of connecting them. They could neither explain the relation of ideas to phenomena, nor their correlation to one another. The very idea of relation or comparison was embarrassing to them. Yet in this intellectual uncertainty they had a conception of a proof from results, and of a moral truth, which remained unshaken amid the questionings of philosophy. (2) The other is a difficulty which is touched upon in the Republic as well as in the Phaedo, and is common to modern and ancient philosophy. Plato is not altogether satisfied with his safe and simple method of ideas. He wants to have proved to him by facts that all things are for the best, and that there is one mind or design which pervades them all. But this 'power of the best' he is unable to explain; and therefore takes refuge in universal ideas. And are not we at this day seeking to discover that which Socrates in a glass darkly foresaw?

\par  Some resemblances to the Greek drama may be noted in all the Dialogues of Plato. The Phaedo is the tragedy of which Socrates is the protagonist and Simmias and Cebes the secondary performers, standing to them in the same relation as to Glaucon and Adeimantus in the Republic. No Dialogue has a greater unity of subject and feeling. Plato has certainly fulfilled the condition of Greek, or rather of all art, which requires that scenes of death and suffering should be clothed in beauty. The gathering of the friends at the commencement of the Dialogue, the dismissal of Xanthippe, whose presence would have been out of place at a philosophical discussion, but who returns again with her children to take a final farewell, the dejection of the audience at the temporary overthrow of the argument, the picture of Socrates playing with the hair of Phaedo, the final scene in which Socrates alone retains his composure—are masterpieces of art. And the chorus at the end might have interpreted the feeling of the play: 'There can no evil happen to a good man in life or death.'

\par  'The art of concealing art' is nowhere more perfect than in those writings of Plato which describe the trial and death of Socrates. Their charm is their simplicity, which gives them verisimilitude; and yet they touch, as if incidentally, and because they were suitable to the occasion, on some of the deepest truths of philosophy. There is nothing in any tragedy, ancient or modern, nothing in poetry or history (with one exception), like the last hours of Socrates in Plato. The master could not be more fitly occupied at such a time than in discoursing of immortality; nor the disciples more divinely consoled. The arguments, taken in the spirit and not in the letter, are our arguments; and Socrates by anticipation may be even thought to refute some 'eccentric notions; current in our own age. For there are philosophers among ourselves who do not seem to understand how much stronger is the power of intelligence, or of the best, than of Atlas, or mechanical force. How far the words attributed to Socrates were actually uttered by him we forbear to ask; for no answer can be given to this question. And it is better to resign ourselves to the feeling of a great work, than to linger among critical uncertainties.

\par 
\section{
      PHAEDO
    } 
\par  Phaedo, who is the narrator of the dialogue to Echecrates of Phlius. Socrates, Apollodorus, Simmias, Cebes, Crito and an Attendant of the Prison.
  
\par \textbf{ECHECRATES}
\par   Were you yourself, Phaedo, in the prison with Socrates on the day when he drank the poison?

\par \textbf{PHAEDO}
\par   Yes, Echecrates, I was.

\par \textbf{ECHECRATES}
\par   I should so like to hear about his death. What did he say in his last hours? We were informed that he died by taking poison, but no one knew anything more; for no Phliasian ever goes to Athens now, and it is a long time since any stranger from Athens has found his way hither; so that we had no clear account.

\par \textbf{PHAEDO}
\par   Did you not hear of the proceedings at the trial?

\par \textbf{ECHECRATES}
\par   Yes; some one told us about the trial, and we could not understand why, having been condemned, he should have been put to death, not at the time, but long afterwards. What was the reason of this?

\par \textbf{PHAEDO}
\par   An accident, Echecrates:  the stern of the ship which the Athenians send to Delos happened to have been crowned on the day before he was tried.

\par \textbf{ECHECRATES}
\par   What is this ship?

\par \textbf{PHAEDO}
\par   It is the ship in which, according to Athenian tradition, Theseus went to Crete when he took with him the fourteen youths, and was the saviour of them and of himself. And they were said to have vowed to Apollo at the time, that if they were saved they would send a yearly mission to Delos. Now this custom still continues, and the whole period of the voyage to and from Delos, beginning when the priest of Apollo crowns the stern of the ship, is a holy season, during which the city is not allowed to be polluted by public executions; and when the vessel is detained by contrary winds, the time spent in going and returning is very considerable. As I was saying, the ship was crowned on the day before the trial, and this was the reason why Socrates lay in prison and was not put to death until long after he was condemned.

\par \textbf{ECHECRATES}
\par   What was the manner of his death, Phaedo? What was said or done? And which of his friends were with him? Or did the authorities forbid them to be present—so that he had no friends near him when he died?

\par \textbf{PHAEDO}
\par   No; there were several of them with him.

\par \textbf{ECHECRATES}
\par   If you have nothing to do, I wish that you would tell me what passed, as exactly as you can.

\par \textbf{PHAEDO}
\par   I have nothing at all to do, and will try to gratify your wish. To be reminded of Socrates is always the greatest delight to me, whether I speak myself or hear another speak of him.

\par \textbf{ECHECRATES}
\par   You will have listeners who are of the same mind with you, and I hope that you will be as exact as you can.

\par \textbf{PHAEDO}
\par   I had a singular feeling at being in his company. For I could hardly believe that I was present at the death of a friend, and therefore I did not pity him, Echecrates; he died so fearlessly, and his words and bearing were so noble and gracious, that to me he appeared blessed. I thought that in going to the other world he could not be without a divine call, and that he would be happy, if any man ever was, when he arrived there, and therefore I did not pity him as might have seemed natural at such an hour. But I had not the pleasure which I usually feel in philosophical discourse (for philosophy was the theme of which we spoke). I was pleased, but in the pleasure there was also a strange admixture of pain; for I reflected that he was soon to die, and this double feeling was shared by us all; we were laughing and weeping by turns, especially the excitable Apollodorus—you know the sort of man?

\par \textbf{ECHECRATES}
\par   Yes.

\par \textbf{PHAEDO}
\par   He was quite beside himself; and I and all of us were greatly moved.

\par \textbf{ECHECRATES}
\par   Who were present?

\par \textbf{PHAEDO}
\par   Of native Athenians there were, besides Apollodorus, Critobulus and his father Crito, Hermogenes, Epigenes, Aeschines, Antisthenes; likewise Ctesippus of the deme of Paeania, Menexenus, and some others; Plato, if I am not mistaken, was ill.

\par \textbf{ECHECRATES}
\par   Were there any strangers?

\par \textbf{PHAEDO}
\par   Yes, there were; Simmias the Theban, and Cebes, and Phaedondes; Euclid and Terpison, who came from Megara.

\par \textbf{ECHECRATES}
\par   And was Aristippus there, and Cleombrotus?

\par \textbf{PHAEDO}
\par   No, they were said to be in Aegina.

\par \textbf{ECHECRATES}
\par   Any one else?

\par \textbf{PHAEDO}
\par   I think that these were nearly all.

\par \textbf{ECHECRATES}
\par   Well, and what did you talk about?

\par \textbf{PHAEDO}
\par   I will begin at the beginning, and endeavour to repeat the entire conversation. On the previous days we had been in the habit of assembling early in the morning at the court in which the trial took place, and which is not far from the prison. There we used to wait talking with one another until the opening of the doors (for they were not opened very early); then we went in and generally passed the day with Socrates. On the last morning we assembled sooner than usual, having heard on the day before when we quitted the prison in the evening that the sacred ship had come from Delos, and so we arranged to meet very early at the accustomed place. On our arrival the jailer who answered the door, instead of admitting us, came out and told us to stay until he called us. 'For the Eleven,' he said, 'are now with Socrates; they are taking off his chains, and giving orders that he is to die to-day.' He soon returned and said that we might come in. On entering we found Socrates just released from chains, and Xanthippe, whom you know, sitting by him, and holding his child in her arms. When she saw us she uttered a cry and said, as women will:  'O Socrates, this is the last time that either you will converse with your friends, or they with you.' Socrates turned to Crito and said:  'Crito, let some one take her home.' Some of Crito's people accordingly led her away, crying out and beating herself. And when she was gone, Socrates, sitting up on the couch, bent and rubbed his leg, saying, as he was rubbing:  How singular is the thing called pleasure, and how curiously related to pain, which might be thought to be the opposite of it; for they are never present to a man at the same instant, and yet he who pursues either is generally compelled to take the other; their bodies are two, but they are joined by a single head. And I cannot help thinking that if Aesop had remembered them, he would have made a fable about God trying to reconcile their strife, and how, when he could not, he fastened their heads together; and this is the reason why when one comes the other follows, as I know by my own experience now, when after the pain in my leg which was caused by the chain pleasure appears to succeed.

\par  Upon this Cebes said: I am glad, Socrates, that you have mentioned the name of Aesop. For it reminds me of a question which has been asked by many, and was asked of me only the day before yesterday by Evenus the poet—he will be sure to ask it again, and therefore if you would like me to have an answer ready for him, you may as well tell me what I should say to him:—he wanted to know why you, who never before wrote a line of poetry, now that you are in prison are turning Aesop's fables into verse, and also composing that hymn in honour of Apollo.

\par  Tell him, Cebes, he replied, what is the truth—that I had no idea of rivalling him or his poems; to do so, as I knew, would be no easy task. But I wanted to see whether I could purge away a scruple which I felt about the meaning of certain dreams. In the course of my life I have often had intimations in dreams 'that I should compose music.' The same dream came to me sometimes in one form, and sometimes in another, but always saying the same or nearly the same words: 'Cultivate and make music,' said the dream. And hitherto I had imagined that this was only intended to exhort and encourage me in the study of philosophy, which has been the pursuit of my life, and is the noblest and best of music. The dream was bidding me do what I was already doing, in the same way that the competitor in a race is bidden by the spectators to run when he is already running. But I was not certain of this, for the dream might have meant music in the popular sense of the word, and being under sentence of death, and the festival giving me a respite, I thought that it would be safer for me to satisfy the scruple, and, in obedience to the dream, to compose a few verses before I departed. And first I made a hymn in honour of the god of the festival, and then considering that a poet, if he is really to be a poet, should not only put together words, but should invent stories, and that I have no invention, I took some fables of Aesop, which I had ready at hand and which I knew—they were the first I came upon—and turned them into verse. Tell this to Evenus, Cebes, and bid him be of good cheer; say that I would have him come after me if he be a wise man, and not tarry; and that to-day I am likely to be going, for the Athenians say that I must.

\par  Simmias said: What a message for such a man! having been a frequent companion of his I should say that, as far as I know him, he will never take your advice unless he is obliged.

\par  Why, said Socrates,—is not Evenus a philosopher?

\par  I think that he is, said Simmias.

\par  Then he, or any man who has the spirit of philosophy, will be willing to die, but he will not take his own life, for that is held to be unlawful.

\par  Here he changed his position, and put his legs off the couch on to the ground, and during the rest of the conversation he remained sitting.

\par  Why do you say, enquired Cebes, that a man ought not to take his own life, but that the philosopher will be ready to follow the dying?

\par  Socrates replied: And have you, Cebes and Simmias, who are the disciples of Philolaus, never heard him speak of this?

\par  Yes, but his language was obscure, Socrates.

\par  My words, too, are only an echo; but there is no reason why I should not repeat what I have heard: and indeed, as I am going to another place, it is very meet for me to be thinking and talking of the nature of the pilgrimage which I am about to make. What can I do better in the interval between this and the setting of the sun?

\par  Then tell me, Socrates, why is suicide held to be unlawful? as I have certainly heard Philolaus, about whom you were just now asking, affirm when he was staying with us at Thebes: and there are others who say the same, although I have never understood what was meant by any of them.

\par  Do not lose heart, replied Socrates, and the day may come when you will understand. I suppose that you wonder why, when other things which are evil may be good at certain times and to certain persons, death is to be the only exception, and why, when a man is better dead, he is not permitted to be his own benefactor, but must wait for the hand of another.

\par  Very true, said Cebes, laughing gently and speaking in his native Boeotian.

\par  I admit the appearance of inconsistency in what I am saying; but there may not be any real inconsistency after all. There is a doctrine whispered in secret that man is a prisoner who has no right to open the door and run away; this is a great mystery which I do not quite understand. Yet I too believe that the gods are our guardians, and that we are a possession of theirs. Do you not agree?

\par  Yes, I quite agree, said Cebes.

\par  And if one of your own possessions, an ox or an ass, for example, took the liberty of putting himself out of the way when you had given no intimation of your wish that he should die, would you not be angry with him, and would you not punish him if you could?

\par  Certainly, replied Cebes.

\par  Then, if we look at the matter thus, there may be reason in saying that a man should wait, and not take his own life until God summons him, as he is now summoning me.

\par  Yes, Socrates, said Cebes, there seems to be truth in what you say. And yet how can you reconcile this seemingly true belief that God is our guardian and we his possessions, with the willingness to die which we were just now attributing to the philosopher? That the wisest of men should be willing to leave a service in which they are ruled by the gods who are the best of rulers, is not reasonable; for surely no wise man thinks that when set at liberty he can take better care of himself than the gods take of him. A fool may perhaps think so—he may argue that he had better run away from his master, not considering that his duty is to remain to the end, and not to run away from the good, and that there would be no sense in his running away. The wise man will want to be ever with him who is better than himself. Now this, Socrates, is the reverse of what was just now said; for upon this view the wise man should sorrow and the fool rejoice at passing out of life.

\par  The earnestness of Cebes seemed to please Socrates. Here, said he, turning to us, is a man who is always inquiring, and is not so easily convinced by the first thing which he hears.

\par  And certainly, added Simmias, the objection which he is now making does appear to me to have some force. For what can be the meaning of a truly wise man wanting to fly away and lightly leave a master who is better than himself? And I rather imagine that Cebes is referring to you; he thinks that you are too ready to leave us, and too ready to leave the gods whom you acknowledge to be our good masters.

\par  Yes, replied Socrates; there is reason in what you say. And so you think that I ought to answer your indictment as if I were in a court?

\par  We should like you to do so, said Simmias.

\par  Then I must try to make a more successful defence before you than I did when before the judges. For I am quite ready to admit, Simmias and Cebes, that I ought to be grieved at death, if I were not persuaded in the first place that I am going to other gods who are wise and good (of which I am as certain as I can be of any such matters), and secondly (though I am not so sure of this last) to men departed, better than those whom I leave behind; and therefore I do not grieve as I might have done, for I have good hope that there is yet something remaining for the dead, and as has been said of old, some far better thing for the good than for the evil.

\par  But do you mean to take away your thoughts with you, Socrates? said Simmias. Will you not impart them to us?—for they are a benefit in which we too are entitled to share. Moreover, if you succeed in convincing us, that will be an answer to the charge against yourself.

\par  I will do my best, replied Socrates. But you must first let me hear what Crito wants; he has long been wishing to say something to me.

\par  Only this, Socrates, replied Crito:—the attendant who is to give you the poison has been telling me, and he wants me to tell you, that you are not to talk much, talking, he says, increases heat, and this is apt to interfere with the action of the poison; persons who excite themselves are sometimes obliged to take a second or even a third dose.

\par  Then, said Socrates, let him mind his business and be prepared to give the poison twice or even thrice if necessary; that is all.

\par  I knew quite well what you would say, replied Crito; but I was obliged to satisfy him.

\par  Never mind him, he said.

\par  And now, O my judges, I desire to prove to you that the real philosopher has reason to be of good cheer when he is about to die, and that after death he may hope to obtain the greatest good in the other world. And how this may be, Simmias and Cebes, I will endeavour to explain. For I deem that the true votary of philosophy is likely to be misunderstood by other men; they do not perceive that he is always pursuing death and dying; and if this be so, and he has had the desire of death all his life long, why when his time comes should he repine at that which he has been always pursuing and desiring?

\par  Simmias said laughingly: Though not in a laughing humour, you have made me laugh, Socrates; for I cannot help thinking that the many when they hear your words will say how truly you have described philosophers, and our people at home will likewise say that the life which philosophers desire is in reality death, and that they have found them out to be deserving of the death which they desire.

\par  And they are right, Simmias, in thinking so, with the exception of the words 'they have found them out'; for they have not found out either what is the nature of that death which the true philosopher deserves, or how he deserves or desires death. But enough of them:—let us discuss the matter among ourselves: Do we believe that there is such a thing as death?

\par  To be sure, replied Simmias.

\par  Is it not the separation of soul and body? And to be dead is the completion of this; when the soul exists in herself, and is released from the body and the body is released from the soul, what is this but death?

\par  Just so, he replied.

\par  There is another question, which will probably throw light on our present inquiry if you and I can agree about it:—Ought the philosopher to care about the pleasures—if they are to be called pleasures—of eating and drinking?

\par  Certainly not, answered Simmias.

\par  And what about the pleasures of love—should he care for them?

\par  By no means.

\par  And will he think much of the other ways of indulging the body, for example, the acquisition of costly raiment, or sandals, or other adornments of the body? Instead of caring about them, does he not rather despise anything more than nature needs? What do you say?

\par  I should say that the true philosopher would despise them.

\par  Would you not say that he is entirely concerned with the soul and not with the body? He would like, as far as he can, to get away from the body and to turn to the soul.

\par  Quite true.

\par  In matters of this sort philosophers, above all other men, may be observed in every sort of way to dissever the soul from the communion of the body.

\par  Very true.

\par  Whereas, Simmias, the rest of the world are of opinion that to him who has no sense of pleasure and no part in bodily pleasure, life is not worth having; and that he who is indifferent about them is as good as dead.

\par  That is also true.

\par  What again shall we say of the actual acquirement of knowledge?—is the body, if invited to share in the enquiry, a hinderer or a helper? I mean to say, have sight and hearing any truth in them? Are they not, as the poets are always telling us, inaccurate witnesses? and yet, if even they are inaccurate and indistinct, what is to be said of the other senses?—for you will allow that they are the best of them?

\par  Certainly, he replied.

\par  Then when does the soul attain truth?—for in attempting to consider anything in company with the body she is obviously deceived.

\par  True.

\par  Then must not true existence be revealed to her in thought, if at all?

\par  Yes.

\par  And thought is best when the mind is gathered into herself and none of these things trouble her—neither sounds nor sights nor pain nor any pleasure,—when she takes leave of the body, and has as little as possible to do with it, when she has no bodily sense or desire, but is aspiring after true being?

\par  Certainly.

\par  And in this the philosopher dishonours the body; his soul runs away from his body and desires to be alone and by herself?

\par  That is true.

\par  Well, but there is another thing, Simmias: Is there or is there not an absolute justice?

\par  Assuredly there is.

\par  And an absolute beauty and absolute good?

\par  Of course.

\par  But did you ever behold any of them with your eyes?

\par  Certainly not.

\par  Or did you ever reach them with any other bodily sense?—and I speak not of these alone, but of absolute greatness, and health, and strength, and of the essence or true nature of everything. Has the reality of them ever been perceived by you through the bodily organs? or rather, is not the nearest approach to the knowledge of their several natures made by him who so orders his intellectual vision as to have the most exact conception of the essence of each thing which he considers?

\par  Certainly.

\par  And he attains to the purest knowledge of them who goes to each with the mind alone, not introducing or intruding in the act of thought sight or any other sense together with reason, but with the very light of the mind in her own clearness searches into the very truth of each; he who has got rid, as far as he can, of eyes and ears and, so to speak, of the whole body, these being in his opinion distracting elements which when they infect the soul hinder her from acquiring truth and knowledge—who, if not he, is likely to attain the knowledge of true being?

\par  What you say has a wonderful truth in it, Socrates, replied Simmias.

\par  And when real philosophers consider all these things, will they not be led to make a reflection which they will express in words something like the following? 'Have we not found,' they will say, 'a path of thought which seems to bring us and our argument to the conclusion, that while we are in the body, and while the soul is infected with the evils of the body, our desire will not be satisfied? and our desire is of the truth. For the body is a source of endless trouble to us by reason of the mere requirement of food; and is liable also to diseases which overtake and impede us in the search after true being: it fills us full of loves, and lusts, and fears, and fancies of all kinds, and endless foolery, and in fact, as men say, takes away from us the power of thinking at all. Whence come wars, and fightings, and factions? whence but from the body and the lusts of the body? wars are occasioned by the love of money, and money has to be acquired for the sake and in the service of the body; and by reason of all these impediments we have no time to give to philosophy; and, last and worst of all, even if we are at leisure and betake ourselves to some speculation, the body is always breaking in upon us, causing turmoil and confusion in our enquiries, and so amazing us that we are prevented from seeing the truth. It has been proved to us by experience that if we would have pure knowledge of anything we must be quit of the body—the soul in herself must behold things in themselves: and then we shall attain the wisdom which we desire, and of which we say that we are lovers, not while we live, but after death; for if while in company with the body, the soul cannot have pure knowledge, one of two things follows—either knowledge is not to be attained at all, or, if at all, after death. For then, and not till then, the soul will be parted from the body and exist in herself alone. In this present life, I reckon that we make the nearest approach to knowledge when we have the least possible intercourse or communion with the body, and are not surfeited with the bodily nature, but keep ourselves pure until the hour when God himself is pleased to release us. And thus having got rid of the foolishness of the body we shall be pure and hold converse with the pure, and know of ourselves the clear light everywhere, which is no other than the light of truth.' For the impure are not permitted to approach the pure. These are the sort of words, Simmias, which the true lovers of knowledge cannot help saying to one another, and thinking. You would agree; would you not?

\par  Undoubtedly, Socrates.

\par  But, O my friend, if this is true, there is great reason to hope that, going whither I go, when I have come to the end of my journey, I shall attain that which has been the pursuit of my life. And therefore I go on my way rejoicing, and not I only, but every other man who believes that his mind has been made ready and that he is in a manner purified.

\par  Certainly, replied Simmias.

\par  And what is purification but the separation of the soul from the body, as I was saying before; the habit of the soul gathering and collecting herself into herself from all sides out of the body; the dwelling in her own place alone, as in another life, so also in this, as far as she can;—the release of the soul from the chains of the body?

\par  Very true, he said.

\par  And this separation and release of the soul from the body is termed death?

\par  To be sure, he said.

\par  And the true philosophers, and they only, are ever seeking to release the soul. Is not the separation and release of the soul from the body their especial study?

\par  That is true.

\par  And, as I was saying at first, there would be a ridiculous contradiction in men studying to live as nearly as they can in a state of death, and yet repining when it comes upon them.

\par  Clearly.

\par  And the true philosophers, Simmias, are always occupied in the practice of dying, wherefore also to them least of all men is death terrible. Look at the matter thus:—if they have been in every way the enemies of the body, and are wanting to be alone with the soul, when this desire of theirs is granted, how inconsistent would they be if they trembled and repined, instead of rejoicing at their departure to that place where, when they arrive, they hope to gain that which in life they desired—and this was wisdom—and at the same time to be rid of the company of their enemy. Many a man has been willing to go to the world below animated by the hope of seeing there an earthly love, or wife, or son, and conversing with them. And will he who is a true lover of wisdom, and is strongly persuaded in like manner that only in the world below he can worthily enjoy her, still repine at death? Will he not depart with joy? Surely he will, O my friend, if he be a true philosopher. For he will have a firm conviction that there and there only, he can find wisdom in her purity. And if this be true, he would be very absurd, as I was saying, if he were afraid of death.

\par  He would, indeed, replied Simmias.

\par  And when you see a man who is repining at the approach of death, is not his reluctance a sufficient proof that he is not a lover of wisdom, but a lover of the body, and probably at the same time a lover of either money or power, or both?

\par  Quite so, he replied.

\par  And is not courage, Simmias, a quality which is specially characteristic of the philosopher?

\par  Certainly.

\par  There is temperance again, which even by the vulgar is supposed to consist in the control and regulation of the passions, and in the sense of superiority to them—is not temperance a virtue belonging to those only who despise the body, and who pass their lives in philosophy?

\par  Most assuredly.

\par  For the courage and temperance of other men, if you will consider them, are really a contradiction.

\par  How so?

\par  Well, he said, you are aware that death is regarded by men in general as a great evil.

\par  Very true, he said.

\par  And do not courageous men face death because they are afraid of yet greater evils?

\par  That is quite true.

\par  Then all but the philosophers are courageous only from fear, and because they are afraid; and yet that a man should be courageous from fear, and because he is a coward, is surely a strange thing.

\par  Very true.

\par  And are not the temperate exactly in the same case? They are temperate because they are intemperate—which might seem to be a contradiction, but is nevertheless the sort of thing which happens with this foolish temperance. For there are pleasures which they are afraid of losing; and in their desire to keep them, they abstain from some pleasures, because they are overcome by others; and although to be conquered by pleasure is called by men intemperance, to them the conquest of pleasure consists in being conquered by pleasure. And that is what I mean by saying that, in a sense, they are made temperate through intemperance.

\par  Such appears to be the case.

\par  Yet the exchange of one fear or pleasure or pain for another fear or pleasure or pain, and of the greater for the less, as if they were coins, is not the exchange of virtue. O my blessed Simmias, is there not one true coin for which all things ought to be exchanged?—and that is wisdom; and only in exchange for this, and in company with this, is anything truly bought or sold, whether courage or temperance or justice. And is not all true virtue the companion of wisdom, no matter what fears or pleasures or other similar goods or evils may or may not attend her? But the virtue which is made up of these goods, when they are severed from wisdom and exchanged with one another, is a shadow of virtue only, nor is there any freedom or health or truth in her; but in the true exchange there is a purging away of all these things, and temperance, and justice, and courage, and wisdom herself are the purgation of them. The founders of the mysteries would appear to have had a real meaning, and were not talking nonsense when they intimated in a figure long ago that he who passes unsanctified and uninitiated into the world below will lie in a slough, but that he who arrives there after initiation and purification will dwell with the gods. For 'many,' as they say in the mysteries, 'are the thyrsus-bearers, but few are the mystics,'—meaning, as I interpret the words, 'the true philosophers.' In the number of whom, during my whole life, I have been seeking, according to my ability, to find a place;—whether I have sought in a right way or not, and whether I have succeeded or not, I shall truly know in a little while, if God will, when I myself arrive in the other world—such is my belief. And therefore I maintain that I am right, Simmias and Cebes, in not grieving or repining at parting from you and my masters in this world, for I believe that I shall equally find good masters and friends in another world. But most men do not believe this saying; if then I succeed in convincing you by my defence better than I did the Athenian judges, it will be well.

\par  Cebes answered: I agree, Socrates, in the greater part of what you say. But in what concerns the soul, men are apt to be incredulous; they fear that when she has left the body her place may be nowhere, and that on the very day of death she may perish and come to an end—immediately on her release from the body, issuing forth dispersed like smoke or air and in her flight vanishing away into nothingness. If she could only be collected into herself after she has obtained release from the evils of which you are speaking, there would be good reason to hope, Socrates, that what you say is true. But surely it requires a great deal of argument and many proofs to show that when the man is dead his soul yet exists, and has any force or intelligence.

\par  True, Cebes, said Socrates; and shall I suggest that we converse a little of the probabilities of these things?

\par  I am sure, said Cebes, that I should greatly like to know your opinion about them.

\par  I reckon, said Socrates, that no one who heard me now, not even if he were one of my old enemies, the Comic poets, could accuse me of idle talking about matters in which I have no concern:—If you please, then, we will proceed with the inquiry.

\par  Suppose we consider the question whether the souls of men after death are or are not in the world below. There comes into my mind an ancient doctrine which affirms that they go from hence into the other world, and returning hither, are born again from the dead. Now if it be true that the living come from the dead, then our souls must exist in the other world, for if not, how could they have been born again? And this would be conclusive, if there were any real evidence that the living are only born from the dead; but if this is not so, then other arguments will have to be adduced.

\par  Very true, replied Cebes.

\par  Then let us consider the whole question, not in relation to man only, but in relation to animals generally, and to plants, and to everything of which there is generation, and the proof will be easier. Are not all things which have opposites generated out of their opposites? I mean such things as good and evil, just and unjust—and there are innumerable other opposites which are generated out of opposites. And I want to show that in all opposites there is of necessity a similar alternation; I mean to say, for example, that anything which becomes greater must become greater after being less.

\par  True.

\par  And that which becomes less must have been once greater and then have become less.

\par  Yes.

\par  And the weaker is generated from the stronger, and the swifter from the slower.

\par  Very true.

\par  And the worse is from the better, and the more just is from the more unjust.

\par  Of course.

\par  And is this true of all opposites? and are we convinced that all of them are generated out of opposites?

\par  Yes.

\par  And in this universal opposition of all things, are there not also two intermediate processes which are ever going on, from one to the other opposite, and back again; where there is a greater and a less there is also an intermediate process of increase and diminution, and that which grows is said to wax, and that which decays to wane?

\par  Yes, he said.

\par  And there are many other processes, such as division and composition, cooling and heating, which equally involve a passage into and out of one another. And this necessarily holds of all opposites, even though not always expressed in words—they are really generated out of one another, and there is a passing or process from one to the other of them?

\par  Very true, he replied.

\par  Well, and is there not an opposite of life, as sleep is the opposite of waking?

\par  True, he said.

\par  And what is it?

\par  Death, he answered.

\par  And these, if they are opposites, are generated the one from the other, and have there their two intermediate processes also?

\par  Of course.

\par  Now, said Socrates, I will analyze one of the two pairs of opposites which I have mentioned to you, and also its intermediate processes, and you shall analyze the other to me. One of them I term sleep, the other waking. The state of sleep is opposed to the state of waking, and out of sleeping waking is generated, and out of waking, sleeping; and the process of generation is in the one case falling asleep, and in the other waking up. Do you agree?

\par  I entirely agree.

\par  Then, suppose that you analyze life and death to me in the same manner. Is not death opposed to life?

\par  Yes.

\par  And they are generated one from the other?

\par  Yes.

\par  What is generated from the living?

\par  The dead.

\par  And what from the dead?

\par  I can only say in answer—the living.

\par  Then the living, whether things or persons, Cebes, are generated from the dead?

\par  That is clear, he replied.

\par  Then the inference is that our souls exist in the world below?

\par  That is true.

\par  And one of the two processes or generations is visible—for surely the act of dying is visible?

\par  Surely, he said.

\par  What then is to be the result? Shall we exclude the opposite process? And shall we suppose nature to walk on one leg only? Must we not rather assign to death some corresponding process of generation?

\par  Certainly, he replied.

\par  And what is that process?

\par  Return to life.

\par  And return to life, if there be such a thing, is the birth of the dead into the world of the living?

\par  Quite true.

\par  Then here is a new way by which we arrive at the conclusion that the living come from the dead, just as the dead come from the living; and this, if true, affords a most certain proof that the souls of the dead exist in some place out of which they come again.

\par  Yes, Socrates, he said; the conclusion seems to flow necessarily out of our previous admissions.

\par  And that these admissions were not unfair, Cebes, he said, may be shown, I think, as follows: If generation were in a straight line only, and there were no compensation or circle in nature, no turn or return of elements into their opposites, then you know that all things would at last have the same form and pass into the same state, and there would be no more generation of them.

\par  What do you mean? he said.

\par  A simple thing enough, which I will illustrate by the case of sleep, he replied. You know that if there were no alternation of sleeping and waking, the tale of the sleeping Endymion would in the end have no meaning, because all other things would be asleep, too, and he would not be distinguishable from the rest. Or if there were composition only, and no division of substances, then the chaos of Anaxagoras would come again. And in like manner, my dear Cebes, if all things which partook of life were to die, and after they were dead remained in the form of death, and did not come to life again, all would at last die, and nothing would be alive—what other result could there be? For if the living spring from any other things, and they too die, must not all things at last be swallowed up in death? (But compare Republic.)

\par  There is no escape, Socrates, said Cebes; and to me your argument seems to be absolutely true.

\par  Yes, he said, Cebes, it is and must be so, in my opinion; and we have not been deluded in making these admissions; but I am confident that there truly is such a thing as living again, and that the living spring from the dead, and that the souls of the dead are in existence, and that the good souls have a better portion than the evil.

\par  Cebes added: Your favorite doctrine, Socrates, that knowledge is simply recollection, if true, also necessarily implies a previous time in which we have learned that which we now recollect. But this would be impossible unless our soul had been in some place before existing in the form of man; here then is another proof of the soul's immortality.

\par  But tell me, Cebes, said Simmias, interposing, what arguments are urged in favour of this doctrine of recollection. I am not very sure at the moment that I remember them.

\par  One excellent proof, said Cebes, is afforded by questions. If you put a question to a person in a right way, he will give a true answer of himself, but how could he do this unless there were knowledge and right reason already in him? And this is most clearly shown when he is taken to a diagram or to anything of that sort. (Compare Meno.)

\par  But if, said Socrates, you are still incredulous, Simmias, I would ask you whether you may not agree with me when you look at the matter in another way;—I mean, if you are still incredulous as to whether knowledge is recollection.

\par  Incredulous, I am not, said Simmias; but I want to have this doctrine of recollection brought to my own recollection, and, from what Cebes has said, I am beginning to recollect and be convinced; but I should still like to hear what you were going to say.

\par  This is what I would say, he replied:—We should agree, if I am not mistaken, that what a man recollects he must have known at some previous time.

\par  Very true.

\par  And what is the nature of this knowledge or recollection? I mean to ask, Whether a person who, having seen or heard or in any way perceived anything, knows not only that, but has a conception of something else which is the subject, not of the same but of some other kind of knowledge, may not be fairly said to recollect that of which he has the conception?

\par  What do you mean?

\par  I mean what I may illustrate by the following instance:—The knowledge of a lyre is not the same as the knowledge of a man?

\par  True.

\par  And yet what is the feeling of lovers when they recognize a lyre, or a garment, or anything else which the beloved has been in the habit of using? Do not they, from knowing the lyre, form in the mind's eye an image of the youth to whom the lyre belongs? And this is recollection. In like manner any one who sees Simmias may remember Cebes; and there are endless examples of the same thing.

\par  Endless, indeed, replied Simmias.

\par  And recollection is most commonly a process of recovering that which has been already forgotten through time and inattention.

\par  Very true, he said.

\par  Well; and may you not also from seeing the picture of a horse or a lyre remember a man? and from the picture of Simmias, you may be led to remember Cebes?

\par  True.

\par  Or you may also be led to the recollection of Simmias himself?

\par  Quite so.

\par  And in all these cases, the recollection may be derived from things either like or unlike?

\par  It may be.

\par  And when the recollection is derived from like things, then another consideration is sure to arise, which is—whether the likeness in any degree falls short or not of that which is recollected?

\par  Very true, he said.

\par  And shall we proceed a step further, and affirm that there is such a thing as equality, not of one piece of wood or stone with another, but that, over and above this, there is absolute equality? Shall we say so?

\par  Say so, yes, replied Simmias, and swear to it, with all the confidence in life.

\par  And do we know the nature of this absolute essence?

\par  To be sure, he said.

\par  And whence did we obtain our knowledge? Did we not see equalities of material things, such as pieces of wood and stones, and gather from them the idea of an equality which is different from them? For you will acknowledge that there is a difference. Or look at the matter in another way:—Do not the same pieces of wood or stone appear at one time equal, and at another time unequal?

\par  That is certain.

\par  But are real equals ever unequal? or is the idea of equality the same as of inequality?

\par  Impossible, Socrates.

\par  Then these (so-called) equals are not the same with the idea of equality?

\par  I should say, clearly not, Socrates.

\par  And yet from these equals, although differing from the idea of equality, you conceived and attained that idea?

\par  Very true, he said.

\par  Which might be like, or might be unlike them?

\par  Yes.

\par  But that makes no difference; whenever from seeing one thing you conceived another, whether like or unlike, there must surely have been an act of recollection?

\par  Very true.

\par  But what would you say of equal portions of wood and stone, or other material equals? and what is the impression produced by them? Are they equals in the same sense in which absolute equality is equal? or do they fall short of this perfect equality in a measure?

\par  Yes, he said, in a very great measure too.

\par  And must we not allow, that when I or any one, looking at any object, observes that the thing which he sees aims at being some other thing, but falls short of, and cannot be, that other thing, but is inferior, he who makes this observation must have had a previous knowledge of that to which the other, although similar, was inferior?

\par  Certainly.

\par  And has not this been our own case in the matter of equals and of absolute equality?

\par  Precisely.

\par  Then we must have known equality previously to the time when we first saw the material equals, and reflected that all these apparent equals strive to attain absolute equality, but fall short of it?

\par  Very true.

\par  And we recognize also that this absolute equality has only been known, and can only be known, through the medium of sight or touch, or of some other of the senses, which are all alike in this respect?

\par  Yes, Socrates, as far as the argument is concerned, one of them is the same as the other.

\par  From the senses then is derived the knowledge that all sensible things aim at an absolute equality of which they fall short?

\par  Yes.

\par  Then before we began to see or hear or perceive in any way, we must have had a knowledge of absolute equality, or we could not have referred to that standard the equals which are derived from the senses?—for to that they all aspire, and of that they fall short.

\par  No other inference can be drawn from the previous statements.

\par  And did we not see and hear and have the use of our other senses as soon as we were born?

\par  Certainly.

\par  Then we must have acquired the knowledge of equality at some previous time?

\par  Yes.

\par  That is to say, before we were born, I suppose?

\par  True.

\par  And if we acquired this knowledge before we were born, and were born having the use of it, then we also knew before we were born and at the instant of birth not only the equal or the greater or the less, but all other ideas; for we are not speaking only of equality, but of beauty, goodness, justice, holiness, and of all which we stamp with the name of essence in the dialectical process, both when we ask and when we answer questions. Of all this we may certainly affirm that we acquired the knowledge before birth?

\par  We may.

\par  But if, after having acquired, we have not forgotten what in each case we acquired, then we must always have come into life having knowledge, and shall always continue to know as long as life lasts—for knowing is the acquiring and retaining knowledge and not forgetting. Is not forgetting, Simmias, just the losing of knowledge?

\par  Quite true, Socrates.

\par  But if the knowledge which we acquired before birth was lost by us at birth, and if afterwards by the use of the senses we recovered what we previously knew, will not the process which we call learning be a recovering of the knowledge which is natural to us, and may not this be rightly termed recollection?

\par  Very true.

\par  So much is clear—that when we perceive something, either by the help of sight, or hearing, or some other sense, from that perception we are able to obtain a notion of some other thing like or unlike which is associated with it but has been forgotten. Whence, as I was saying, one of two alternatives follows:—either we had this knowledge at birth, and continued to know through life; or, after birth, those who are said to learn only remember, and learning is simply recollection.

\par  Yes, that is quite true, Socrates.

\par  And which alternative, Simmias, do you prefer? Had we the knowledge at our birth, or did we recollect the things which we knew previously to our birth?

\par  I cannot decide at the moment.

\par  At any rate you can decide whether he who has knowledge will or will not be able to render an account of his knowledge? What do you say?

\par  Certainly, he will.

\par  But do you think that every man is able to give an account of these very matters about which we are speaking?

\par  Would that they could, Socrates, but I rather fear that to-morrow, at this time, there will no longer be any one alive who is able to give an account of them such as ought to be given.

\par  Then you are not of opinion, Simmias, that all men know these things?

\par  Certainly not.

\par  They are in process of recollecting that which they learned before?

\par  Certainly.

\par  But when did our souls acquire this knowledge?—not since we were born as men?

\par  Certainly not.

\par  And therefore, previously?

\par  Yes.

\par  Then, Simmias, our souls must also have existed without bodies before they were in the form of man, and must have had intelligence.

\par  Unless indeed you suppose, Socrates, that these notions are given us at the very moment of birth; for this is the only time which remains.

\par  Yes, my friend, but if so, when do we lose them? for they are not in us when we are born—that is admitted. Do we lose them at the moment of receiving them, or if not at what other time?

\par  No, Socrates, I perceive that I was unconsciously talking nonsense.

\par  Then may we not say, Simmias, that if, as we are always repeating, there is an absolute beauty, and goodness, and an absolute essence of all things; and if to this, which is now discovered to have existed in our former state, we refer all our sensations, and with this compare them, finding these ideas to be pre-existent and our inborn possession—then our souls must have had a prior existence, but if not, there would be no force in the argument? There is the same proof that these ideas must have existed before we were born, as that our souls existed before we were born; and if not the ideas, then not the souls.

\par  Yes, Socrates; I am convinced that there is precisely the same necessity for the one as for the other; and the argument retreats successfully to the position that the existence of the soul before birth cannot be separated from the existence of the essence of which you speak. For there is nothing which to my mind is so patent as that beauty, goodness, and the other notions of which you were just now speaking, have a most real and absolute existence; and I am satisfied with the proof.

\par  Well, but is Cebes equally satisfied? for I must convince him too.

\par  I think, said Simmias, that Cebes is satisfied: although he is the most incredulous of mortals, yet I believe that he is sufficiently convinced of the existence of the soul before birth. But that after death the soul will continue to exist is not yet proven even to my own satisfaction. I cannot get rid of the feeling of the many to which Cebes was referring—the feeling that when the man dies the soul will be dispersed, and that this may be the extinction of her. For admitting that she may have been born elsewhere, and framed out of other elements, and was in existence before entering the human body, why after having entered in and gone out again may she not herself be destroyed and come to an end?

\par  Very true, Simmias, said Cebes; about half of what was required has been proven; to wit, that our souls existed before we were born:—that the soul will exist after death as well as before birth is the other half of which the proof is still wanting, and has to be supplied; when that is given the demonstration will be complete.

\par  But that proof, Simmias and Cebes, has been already given, said Socrates, if you put the two arguments together—I mean this and the former one, in which we admitted that everything living is born of the dead. For if the soul exists before birth, and in coming to life and being born can be born only from death and dying, must she not after death continue to exist, since she has to be born again?—Surely the proof which you desire has been already furnished. Still I suspect that you and Simmias would be glad to probe the argument further. Like children, you are haunted with a fear that when the soul leaves the body, the wind may really blow her away and scatter her; especially if a man should happen to die in a great storm and not when the sky is calm.

\par  Cebes answered with a smile: Then, Socrates, you must argue us out of our fears—and yet, strictly speaking, they are not our fears, but there is a child within us to whom death is a sort of hobgoblin; him too we must persuade not to be afraid when he is alone in the dark.

\par  Socrates said: Let the voice of the charmer be applied daily until you have charmed away the fear.

\par  And where shall we find a good charmer of our fears, Socrates, when you are gone?

\par  Hellas, he replied, is a large place, Cebes, and has many good men, and there are barbarous races not a few: seek for him among them all, far and wide, sparing neither pains nor money; for there is no better way of spending your money. And you must seek among yourselves too; for you will not find others better able to make the search.

\par  The search, replied Cebes, shall certainly be made. And now, if you please, let us return to the point of the argument at which we digressed.

\par  By all means, replied Socrates; what else should I please?

\par  Very good.

\par  Must we not, said Socrates, ask ourselves what that is which, as we imagine, is liable to be scattered, and about which we fear? and what again is that about which we have no fear? And then we may proceed further to enquire whether that which suffers dispersion is or is not of the nature of soul—our hopes and fears as to our own souls will turn upon the answers to these questions.

\par  Very true, he said.

\par  Now the compound or composite may be supposed to be naturally capable, as of being compounded, so also of being dissolved; but that which is uncompounded, and that only, must be, if anything is, indissoluble.

\par  Yes; I should imagine so, said Cebes.

\par  And the uncompounded may be assumed to be the same and unchanging, whereas the compound is always changing and never the same.

\par  I agree, he said.

\par  Then now let us return to the previous discussion. Is that idea or essence, which in the dialectical process we define as essence or true existence—whether essence of equality, beauty, or anything else—are these essences, I say, liable at times to some degree of change? or are they each of them always what they are, having the same simple self-existent and unchanging forms, not admitting of variation at all, or in any way, or at any time?

\par  They must be always the same, Socrates, replied Cebes.

\par  And what would you say of the many beautiful—whether men or horses or garments or any other things which are named by the same names and may be called equal or beautiful,—are they all unchanging and the same always, or quite the reverse? May they not rather be described as almost always changing and hardly ever the same, either with themselves or with one another?

\par  The latter, replied Cebes; they are always in a state of change.

\par  And these you can touch and see and perceive with the senses, but the unchanging things you can only perceive with the mind—they are invisible and are not seen?

\par  That is very true, he said.

\par  Well, then, added Socrates, let us suppose that there are two sorts of existences—one seen, the other unseen.

\par  Let us suppose them.

\par  The seen is the changing, and the unseen is the unchanging?

\par  That may be also supposed.

\par  And, further, is not one part of us body, another part soul?

\par  To be sure.

\par  And to which class is the body more alike and akin?

\par  Clearly to the seen—no one can doubt that.

\par  And is the soul seen or not seen?

\par  Not by man, Socrates.

\par  And what we mean by 'seen' and 'not seen' is that which is or is not visible to the eye of man?

\par  Yes, to the eye of man.

\par  And is the soul seen or not seen?

\par  Not seen.

\par  Unseen then?

\par  Yes.

\par  Then the soul is more like to the unseen, and the body to the seen?

\par  That follows necessarily, Socrates.

\par  And were we not saying long ago that the soul when using the body as an instrument of perception, that is to say, when using the sense of sight or hearing or some other sense (for the meaning of perceiving through the body is perceiving through the senses)—were we not saying that the soul too is then dragged by the body into the region of the changeable, and wanders and is confused; the world spins round her, and she is like a drunkard, when she touches change?

\par  Very true.

\par  But when returning into herself she reflects, then she passes into the other world, the region of purity, and eternity, and immortality, and unchangeableness, which are her kindred, and with them she ever lives, when she is by herself and is not let or hindered; then she ceases from her erring ways, and being in communion with the unchanging is unchanging. And this state of the soul is called wisdom?

\par  That is well and truly said, Socrates, he replied.

\par  And to which class is the soul more nearly alike and akin, as far as may be inferred from this argument, as well as from the preceding one?

\par  I think, Socrates, that, in the opinion of every one who follows the argument, the soul will be infinitely more like the unchangeable—even the most stupid person will not deny that.

\par  And the body is more like the changing?

\par  Yes.

\par  Yet once more consider the matter in another light: When the soul and the body are united, then nature orders the soul to rule and govern, and the body to obey and serve. Now which of these two functions is akin to the divine? and which to the mortal? Does not the divine appear to you to be that which naturally orders and rules, and the mortal to be that which is subject and servant?

\par  True.

\par  And which does the soul resemble?

\par  The soul resembles the divine, and the body the mortal—there can be no doubt of that, Socrates.

\par  Then reflect, Cebes: of all which has been said is not this the conclusion?—that the soul is in the very likeness of the divine, and immortal, and intellectual, and uniform, and indissoluble, and unchangeable; and that the body is in the very likeness of the human, and mortal, and unintellectual, and multiform, and dissoluble, and changeable. Can this, my dear Cebes, be denied?

\par  It cannot.

\par  But if it be true, then is not the body liable to speedy dissolution? and is not the soul almost or altogether indissoluble?

\par  Certainly.

\par  And do you further observe, that after a man is dead, the body, or visible part of him, which is lying in the visible world, and is called a corpse, and would naturally be dissolved and decomposed and dissipated, is not dissolved or decomposed at once, but may remain for a for some time, nay even for a long time, if the constitution be sound at the time of death, and the season of the year favourable? For the body when shrunk and embalmed, as the manner is in Egypt, may remain almost entire through infinite ages; and even in decay, there are still some portions, such as the bones and ligaments, which are practically indestructible:—Do you agree?

\par  Yes.

\par  And is it likely that the soul, which is invisible, in passing to the place of the true Hades, which like her is invisible, and pure, and noble, and on her way to the good and wise God, whither, if God will, my soul is also soon to go,—that the soul, I repeat, if this be her nature and origin, will be blown away and destroyed immediately on quitting the body, as the many say? That can never be, my dear Simmias and Cebes. The truth rather is, that the soul which is pure at departing and draws after her no bodily taint, having never voluntarily during life had connection with the body, which she is ever avoiding, herself gathered into herself;—and making such abstraction her perpetual study—which means that she has been a true disciple of philosophy; and therefore has in fact been always engaged in the practice of dying? For is not philosophy the practice of death?—

\par  Certainly—

\par  That soul, I say, herself invisible, departs to the invisible world—to the divine and immortal and rational: thither arriving, she is secure of bliss and is released from the error and folly of men, their fears and wild passions and all other human ills, and for ever dwells, as they say of the initiated, in company with the gods (compare Apol.). Is not this true, Cebes?

\par  Yes, said Cebes, beyond a doubt.

\par  But the soul which has been polluted, and is impure at the time of her departure, and is the companion and servant of the body always, and is in love with and fascinated by the body and by the desires and pleasures of the body, until she is led to believe that the truth only exists in a bodily form, which a man may touch and see and taste, and use for the purposes of his lusts,—the soul, I mean, accustomed to hate and fear and avoid the intellectual principle, which to the bodily eye is dark and invisible, and can be attained only by philosophy;—do you suppose that such a soul will depart pure and unalloyed?

\par  Impossible, he replied.

\par  She is held fast by the corporeal, which the continual association and constant care of the body have wrought into her nature.

\par  Very true.

\par  And this corporeal element, my friend, is heavy and weighty and earthy, and is that element of sight by which a soul is depressed and dragged down again into the visible world, because she is afraid of the invisible and of the world below—prowling about tombs and sepulchres, near which, as they tell us, are seen certain ghostly apparitions of souls which have not departed pure, but are cloyed with sight and therefore visible.

\par  (Compare Milton, Comus:—
 
\par  That is very likely, Socrates.

\par  Yes, that is very likely, Cebes; and these must be the souls, not of the good, but of the evil, which are compelled to wander about such places in payment of the penalty of their former evil way of life; and they continue to wander until through the craving after the corporeal which never leaves them, they are imprisoned finally in another body. And they may be supposed to find their prisons in the same natures which they have had in their former lives.

\par  What natures do you mean, Socrates?

\par  What I mean is that men who have followed after gluttony, and wantonness, and drunkenness, and have had no thought of avoiding them, would pass into asses and animals of that sort. What do you think?

\par  I think such an opinion to be exceedingly probable.

\par  And those who have chosen the portion of injustice, and tyranny, and violence, will pass into wolves, or into hawks and kites;—whither else can we suppose them to go?

\par  Yes, said Cebes; with such natures, beyond question.

\par  And there is no difficulty, he said, in assigning to all of them places answering to their several natures and propensities?

\par  There is not, he said.

\par  Some are happier than others; and the happiest both in themselves and in the place to which they go are those who have practised the civil and social virtues which are called temperance and justice, and are acquired by habit and attention without philosophy and mind. (Compare Republic.)

\par  Why are they the happiest?

\par  Because they may be expected to pass into some gentle and social kind which is like their own, such as bees or wasps or ants, or back again into the form of man, and just and moderate men may be supposed to spring from them.

\par  Very likely.

\par  No one who has not studied philosophy and who is not entirely pure at the time of his departure is allowed to enter the company of the Gods, but the lover of knowledge only. And this is the reason, Simmias and Cebes, why the true votaries of philosophy abstain from all fleshly lusts, and hold out against them and refuse to give themselves up to them,—not because they fear poverty or the ruin of their families, like the lovers of money, and the world in general; nor like the lovers of power and honour, because they dread the dishonour or disgrace of evil deeds.

\par  No, Socrates, that would not become them, said Cebes.

\par  No indeed, he replied; and therefore they who have any care of their own souls, and do not merely live moulding and fashioning the body, say farewell to all this; they will not walk in the ways of the blind: and when philosophy offers them purification and release from evil, they feel that they ought not to resist her influence, and whither she leads they turn and follow.

\par  What do you mean, Socrates?

\par  I will tell you, he said. The lovers of knowledge are conscious that the soul was simply fastened and glued to the body—until philosophy received her, she could only view real existence through the bars of a prison, not in and through herself; she was wallowing in the mire of every sort of ignorance; and by reason of lust had become the principal accomplice in her own captivity. This was her original state; and then, as I was saying, and as the lovers of knowledge are well aware, philosophy, seeing how terrible was her confinement, of which she was to herself the cause, received and gently comforted her and sought to release her, pointing out that the eye and the ear and the other senses are full of deception, and persuading her to retire from them, and abstain from all but the necessary use of them, and be gathered up and collected into herself, bidding her trust in herself and her own pure apprehension of pure existence, and to mistrust whatever comes to her through other channels and is subject to variation; for such things are visible and tangible, but what she sees in her own nature is intelligible and invisible. And the soul of the true philosopher thinks that she ought not to resist this deliverance, and therefore abstains from pleasures and desires and pains and fears, as far as she is able; reflecting that when a man has great joys or sorrows or fears or desires, he suffers from them, not merely the sort of evil which might be anticipated—as for example, the loss of his health or property which he has sacrificed to his lusts—but an evil greater far, which is the greatest and worst of all evils, and one of which he never thinks.

\par  What is it, Socrates? said Cebes.

\par  The evil is that when the feeling of pleasure or pain is most intense, every soul of man imagines the objects of this intense feeling to be then plainest and truest: but this is not so, they are really the things of sight.

\par  Very true.

\par  And is not this the state in which the soul is most enthralled by the body?

\par  How so?

\par  Why, because each pleasure and pain is a sort of nail which nails and rivets the soul to the body, until she becomes like the body, and believes that to be true which the body affirms to be true; and from agreeing with the body and having the same delights she is obliged to have the same habits and haunts, and is not likely ever to be pure at her departure to the world below, but is always infected by the body; and so she sinks into another body and there germinates and grows, and has therefore no part in the communion of the divine and pure and simple.

\par  Most true, Socrates, answered Cebes.

\par  And this, Cebes, is the reason why the true lovers of knowledge are temperate and brave; and not for the reason which the world gives.

\par  Certainly not.

\par  Certainly not! The soul of a philosopher will reason in quite another way; she will not ask philosophy to release her in order that when released she may deliver herself up again to the thraldom of pleasures and pains, doing a work only to be undone again, weaving instead of unweaving her Penelope's web. But she will calm passion, and follow reason, and dwell in the contemplation of her, beholding the true and divine (which is not matter of opinion), and thence deriving nourishment. Thus she seeks to live while she lives, and after death she hopes to go to her own kindred and to that which is like her, and to be freed from human ills. Never fear, Simmias and Cebes, that a soul which has been thus nurtured and has had these pursuits, will at her departure from the body be scattered and blown away by the winds and be nowhere and nothing.

\par  When Socrates had done speaking, for a considerable time there was silence; he himself appeared to be meditating, as most of us were, on what had been said; only Cebes and Simmias spoke a few words to one another. And Socrates observing them asked what they thought of the argument, and whether there was anything wanting? For, said he, there are many points still open to suspicion and attack, if any one were disposed to sift the matter thoroughly. Should you be considering some other matter I say no more, but if you are still in doubt do not hesitate to say exactly what you think, and let us have anything better which you can suggest; and if you think that I can be of any use, allow me to help you.

\par  Simmias said: I must confess, Socrates, that doubts did arise in our minds, and each of us was urging and inciting the other to put the question which we wanted to have answered and which neither of us liked to ask, fearing that our importunity might be troublesome under present at such a time.

\par  Socrates replied with a smile: O Simmias, what are you saying? I am not very likely to persuade other men that I do not regard my present situation as a misfortune, if I cannot even persuade you that I am no worse off now than at any other time in my life. Will you not allow that I have as much of the spirit of prophecy in me as the swans? For they, when they perceive that they must die, having sung all their life long, do then sing more lustily than ever, rejoicing in the thought that they are about to go away to the god whose ministers they are. But men, because they are themselves afraid of death, slanderously affirm of the swans that they sing a lament at the last, not considering that no bird sings when cold, or hungry, or in pain, not even the nightingale, nor the swallow, nor yet the hoopoe; which are said indeed to tune a lay of sorrow, although I do not believe this to be true of them any more than of the swans. But because they are sacred to Apollo, they have the gift of prophecy, and anticipate the good things of another world, wherefore they sing and rejoice in that day more than they ever did before. And I too, believing myself to be the consecrated servant of the same God, and the fellow-servant of the swans, and thinking that I have received from my master gifts of prophecy which are not inferior to theirs, would not go out of life less merrily than the swans. Never mind then, if this be your only objection, but speak and ask anything which you like, while the eleven magistrates of Athens allow.

\par  Very good, Socrates, said Simmias; then I will tell you my difficulty, and Cebes will tell you his. I feel myself, (and I daresay that you have the same feeling), how hard or rather impossible is the attainment of any certainty about questions such as these in the present life. And yet I should deem him a coward who did not prove what is said about them to the uttermost, or whose heart failed him before he had examined them on every side. For he should persevere until he has achieved one of two things: either he should discover, or be taught the truth about them; or, if this be impossible, I would have him take the best and most irrefragable of human theories, and let this be the raft upon which he sails through life—not without risk, as I admit, if he cannot find some word of God which will more surely and safely carry him. And now, as you bid me, I will venture to question you, and then I shall not have to reproach myself hereafter with not having said at the time what I think. For when I consider the matter, either alone or with Cebes, the argument does certainly appear to me, Socrates, to be not sufficient.

\par  Socrates answered: I dare say, my friend, that you may be right, but I should like to know in what respect the argument is insufficient.

\par  In this respect, replied Simmias:—Suppose a person to use the same argument about harmony and the lyre—might he not say that harmony is a thing invisible, incorporeal, perfect, divine, existing in the lyre which is harmonized, but that the lyre and the strings are matter and material, composite, earthy, and akin to mortality? And when some one breaks the lyre, or cuts and rends the strings, then he who takes this view would argue as you do, and on the same analogy, that the harmony survives and has not perished—you cannot imagine, he would say, that the lyre without the strings, and the broken strings themselves which are mortal remain, and yet that the harmony, which is of heavenly and immortal nature and kindred, has perished—perished before the mortal. The harmony must still be somewhere, and the wood and strings will decay before anything can happen to that. The thought, Socrates, must have occurred to your own mind that such is our conception of the soul; and that when the body is in a manner strung and held together by the elements of hot and cold, wet and dry, then the soul is the harmony or due proportionate admixture of them. But if so, whenever the strings of the body are unduly loosened or overstrained through disease or other injury, then the soul, though most divine, like other harmonies of music or of works of art, of course perishes at once, although the material remains of the body may last for a considerable time, until they are either decayed or burnt. And if any one maintains that the soul, being the harmony of the elements of the body, is first to perish in that which is called death, how shall we answer him?

\par  Socrates looked fixedly at us as his manner was, and said with a smile: Simmias has reason on his side; and why does not some one of you who is better able than myself answer him? for there is force in his attack upon me. But perhaps, before we answer him, we had better also hear what Cebes has to say that we may gain time for reflection, and when they have both spoken, we may either assent to them, if there is truth in what they say, or if not, we will maintain our position. Please to tell me then, Cebes, he said, what was the difficulty which troubled you?

\par  Cebes said: I will tell you. My feeling is that the argument is where it was, and open to the same objections which were urged before; for I am ready to admit that the existence of the soul before entering into the bodily form has been very ingeniously, and, if I may say so, quite sufficiently proven; but the existence of the soul after death is still, in my judgment, unproven. Now my objection is not the same as that of Simmias; for I am not disposed to deny that the soul is stronger and more lasting than the body, being of opinion that in all such respects the soul very far excels the body. Well, then, says the argument to me, why do you remain unconvinced?—When you see that the weaker continues in existence after the man is dead, will you not admit that the more lasting must also survive during the same period of time? Now I will ask you to consider whether the objection, which, like Simmias, I will express in a figure, is of any weight. The analogy which I will adduce is that of an old weaver, who dies, and after his death somebody says:—He is not dead, he must be alive;—see, there is the coat which he himself wove and wore, and which remains whole and undecayed. And then he proceeds to ask of some one who is incredulous, whether a man lasts longer, or the coat which is in use and wear; and when he is answered that a man lasts far longer, thinks that he has thus certainly demonstrated the survival of the man, who is the more lasting, because the less lasting remains. But that, Simmias, as I would beg you to remark, is a mistake; any one can see that he who talks thus is talking nonsense. For the truth is, that the weaver aforesaid, having woven and worn many such coats, outlived several of them, and was outlived by the last; but a man is not therefore proved to be slighter and weaker than a coat. Now the relation of the body to the soul may be expressed in a similar figure; and any one may very fairly say in like manner that the soul is lasting, and the body weak and shortlived in comparison. He may argue in like manner that every soul wears out many bodies, especially if a man live many years. While he is alive the body deliquesces and decays, and the soul always weaves another garment and repairs the waste. But of course, whenever the soul perishes, she must have on her last garment, and this will survive her; and then at length, when the soul is dead, the body will show its native weakness, and quickly decompose and pass away. I would therefore rather not rely on the argument from superior strength to prove the continued existence of the soul after death. For granting even more than you affirm to be possible, and acknowledging not only that the soul existed before birth, but also that the souls of some exist, and will continue to exist after death, and will be born and die again and again, and that there is a natural strength in the soul which will hold out and be born many times—nevertheless, we may be still inclined to think that she will weary in the labours of successive births, and may at last succumb in one of her deaths and utterly perish; and this death and dissolution of the body which brings destruction to the soul may be unknown to any of us, for no one of us can have had any experience of it: and if so, then I maintain that he who is confident about death has but a foolish confidence, unless he is able to prove that the soul is altogether immortal and imperishable. But if he cannot prove the soul's immortality, he who is about to die will always have reason to fear that when the body is disunited, the soul also may utterly perish.

\par  All of us, as we afterwards remarked to one another, had an unpleasant feeling at hearing what they said. When we had been so firmly convinced before, now to have our faith shaken seemed to introduce a confusion and uncertainty, not only into the previous argument, but into any future one; either we were incapable of forming a judgment, or there were no grounds of belief.

\par \textbf{ECHECRATES}
\par   There I feel with you—by heaven I do, Phaedo, and when you were speaking, I was beginning to ask myself the same question:  What argument can I ever trust again? For what could be more convincing than the argument of Socrates, which has now fallen into discredit? That the soul is a harmony is a doctrine which has always had a wonderful attraction for me, and, when mentioned, came back to me at once, as my own original conviction. And now I must begin again and find another argument which will assure me that when the man is dead the soul survives. Tell me, I implore you, how did Socrates proceed? Did he appear to share the unpleasant feeling which you mention? or did he calmly meet the attack? And did he answer forcibly or feebly? Narrate what passed as exactly as you can.

\par \textbf{PHAEDO}
\par   Often, Echecrates, I have wondered at Socrates, but never more than on that occasion. That he should be able to answer was nothing, but what astonished me was, first, the gentle and pleasant and approving manner in which he received the words of the young men, and then his quick sense of the wound which had been inflicted by the argument, and the readiness with which he healed it. He might be compared to a general rallying his defeated and broken army, urging them to accompany him and return to the field of argument.

\par \textbf{ECHECRATES}
\par   What followed?

\par \textbf{PHAEDO}
\par   You shall hear, for I was close to him on his right hand, seated on a sort of stool, and he on a couch which was a good deal higher. He stroked my head, and pressed the hair upon my neck—he had a way of playing with my hair; and then he said:  To-morrow, Phaedo, I suppose that these fair locks of yours will be severed.

\par  Yes, Socrates, I suppose that they will, I replied.

\par  Not so, if you will take my advice.

\par  What shall I do with them? I said.

\par  To-day, he replied, and not to-morrow, if this argument dies and we cannot bring it to life again, you and I will both shave our locks; and if I were you, and the argument got away from me, and I could not hold my ground against Simmias and Cebes, I would myself take an oath, like the Argives, not to wear hair any more until I had renewed the conflict and defeated them.

\par  Yes, I said, but Heracles himself is said not to be a match for two.

\par  Summon me then, he said, and I will be your Iolaus until the sun goes down.

\par  I summon you rather, I rejoined, not as Heracles summoning Iolaus, but as Iolaus might summon Heracles.

\par  That will do as well, he said. But first let us take care that we avoid a danger.

\par  Of what nature? I said.

\par  Lest we become misologists, he replied, no worse thing can happen to a man than this. For as there are misanthropists or haters of men, there are also misologists or haters of ideas, and both spring from the same cause, which is ignorance of the world. Misanthropy arises out of the too great confidence of inexperience;—you trust a man and think him altogether true and sound and faithful, and then in a little while he turns out to be false and knavish; and then another and another, and when this has happened several times to a man, especially when it happens among those whom he deems to be his own most trusted and familiar friends, and he has often quarreled with them, he at last hates all men, and believes that no one has any good in him at all. You must have observed this trait of character?

\par  I have.

\par  And is not the feeling discreditable? Is it not obvious that such an one having to deal with other men, was clearly without any experience of human nature; for experience would have taught him the true state of the case, that few are the good and few the evil, and that the great majority are in the interval between them.

\par  What do you mean? I said.

\par  I mean, he replied, as you might say of the very large and very small, that nothing is more uncommon than a very large or very small man; and this applies generally to all extremes, whether of great and small, or swift and slow, or fair and foul, or black and white: and whether the instances you select be men or dogs or anything else, few are the extremes, but many are in the mean between them. Did you never observe this?

\par  Yes, I said, I have.

\par  And do you not imagine, he said, that if there were a competition in evil, the worst would be found to be very few?

\par  Yes, that is very likely, I said.

\par  Yes, that is very likely, he replied; although in this respect arguments are unlike men—there I was led on by you to say more than I had intended; but the point of comparison was, that when a simple man who has no skill in dialectics believes an argument to be true which he afterwards imagines to be false, whether really false or not, and then another and another, he has no longer any faith left, and great disputers, as you know, come to think at last that they have grown to be the wisest of mankind; for they alone perceive the utter unsoundness and instability of all arguments, or indeed, of all things, which, like the currents in the Euripus, are going up and down in never-ceasing ebb and flow.

\par  That is quite true, I said.

\par  Yes, Phaedo, he replied, and how melancholy, if there be such a thing as truth or certainty or possibility of knowledge—that a man should have lighted upon some argument or other which at first seemed true and then turned out to be false, and instead of blaming himself and his own want of wit, because he is annoyed, should at last be too glad to transfer the blame from himself to arguments in general: and for ever afterwards should hate and revile them, and lose truth and the knowledge of realities.

\par  Yes, indeed, I said; that is very melancholy.

\par  Let us then, in the first place, he said, be careful of allowing or of admitting into our souls the notion that there is no health or soundness in any arguments at all. Rather say that we have not yet attained to soundness in ourselves, and that we must struggle manfully and do our best to gain health of mind—you and all other men having regard to the whole of your future life, and I myself in the prospect of death. For at this moment I am sensible that I have not the temper of a philosopher; like the vulgar, I am only a partisan. Now the partisan, when he is engaged in a dispute, cares nothing about the rights of the question, but is anxious only to convince his hearers of his own assertions. And the difference between him and me at the present moment is merely this—that whereas he seeks to convince his hearers that what he says is true, I am rather seeking to convince myself; to convince my hearers is a secondary matter with me. And do but see how much I gain by the argument. For if what I say is true, then I do well to be persuaded of the truth, but if there be nothing after death, still, during the short time that remains, I shall not distress my friends with lamentations, and my ignorance will not last, but will die with me, and therefore no harm will be done. This is the state of mind, Simmias and Cebes, in which I approach the argument. And I would ask you to be thinking of the truth and not of Socrates: agree with me, if I seem to you to be speaking the truth; or if not, withstand me might and main, that I may not deceive you as well as myself in my enthusiasm, and like the bee, leave my sting in you before I die.

\par  And now let us proceed, he said. And first of all let me be sure that I have in my mind what you were saying. Simmias, if I remember rightly, has fears and misgivings whether the soul, although a fairer and diviner thing than the body, being as she is in the form of harmony, may not perish first. On the other hand, Cebes appeared to grant that the soul was more lasting than the body, but he said that no one could know whether the soul, after having worn out many bodies, might not perish herself and leave her last body behind her; and that this is death, which is the destruction not of the body but of the soul, for in the body the work of destruction is ever going on. Are not these, Simmias and Cebes, the points which we have to consider?

\par  They both agreed to this statement of them.

\par  He proceeded: And did you deny the force of the whole preceding argument, or of a part only?

\par  Of a part only, they replied.

\par  And what did you think, he said, of that part of the argument in which we said that knowledge was recollection, and hence inferred that the soul must have previously existed somewhere else before she was enclosed in the body?

\par  Cebes said that he had been wonderfully impressed by that part of the argument, and that his conviction remained absolutely unshaken. Simmias agreed, and added that he himself could hardly imagine the possibility of his ever thinking differently.

\par  But, rejoined Socrates, you will have to think differently, my Theban friend, if you still maintain that harmony is a compound, and that the soul is a harmony which is made out of strings set in the frame of the body; for you will surely never allow yourself to say that a harmony is prior to the elements which compose it.

\par  Never, Socrates.

\par  But do you not see that this is what you imply when you say that the soul existed before she took the form and body of man, and was made up of elements which as yet had no existence? For harmony is not like the soul, as you suppose; but first the lyre, and the strings, and the sounds exist in a state of discord, and then harmony is made last of all, and perishes first. And how can such a notion of the soul as this agree with the other?

\par  Not at all, replied Simmias.

\par  And yet, he said, there surely ought to be harmony in a discourse of which harmony is the theme.

\par  There ought, replied Simmias.

\par  But there is no harmony, he said, in the two propositions that knowledge is recollection, and that the soul is a harmony. Which of them will you retain?

\par  I think, he replied, that I have a much stronger faith, Socrates, in the first of the two, which has been fully demonstrated to me, than in the latter, which has not been demonstrated at all, but rests only on probable and plausible grounds; and is therefore believed by the many. I know too well that these arguments from probabilities are impostors, and unless great caution is observed in the use of them, they are apt to be deceptive—in geometry, and in other things too. But the doctrine of knowledge and recollection has been proven to me on trustworthy grounds; and the proof was that the soul must have existed before she came into the body, because to her belongs the essence of which the very name implies existence. Having, as I am convinced, rightly accepted this conclusion, and on sufficient grounds, I must, as I suppose, cease to argue or allow others to argue that the soul is a harmony.

\par  Let me put the matter, Simmias, he said, in another point of view: Do you imagine that a harmony or any other composition can be in a state other than that of the elements, out of which it is compounded?

\par  Certainly not.

\par  Or do or suffer anything other than they do or suffer?

\par  He agreed.

\par  Then a harmony does not, properly speaking, lead the parts or elements which make up the harmony, but only follows them.

\par  He assented.

\par  For harmony cannot possibly have any motion, or sound, or other quality which is opposed to its parts.

\par  That would be impossible, he replied.

\par  And does not the nature of every harmony depend upon the manner in which the elements are harmonized?

\par  I do not understand you, he said.

\par  I mean to say that a harmony admits of degrees, and is more of a harmony, and more completely a harmony, when more truly and fully harmonized, to any extent which is possible; and less of a harmony, and less completely a harmony, when less truly and fully harmonized.

\par  True.

\par  But does the soul admit of degrees? or is one soul in the very least degree more or less, or more or less completely, a soul than another?

\par  Not in the least.

\par  Yet surely of two souls, one is said to have intelligence and virtue, and to be good, and the other to have folly and vice, and to be an evil soul: and this is said truly?

\par  Yes, truly.

\par  But what will those who maintain the soul to be a harmony say of this presence of virtue and vice in the soul?—will they say that here is another harmony, and another discord, and that the virtuous soul is harmonized, and herself being a harmony has another harmony within her, and that the vicious soul is inharmonical and has no harmony within her?

\par  I cannot tell, replied Simmias; but I suppose that something of the sort would be asserted by those who say that the soul is a harmony.

\par  And we have already admitted that no soul is more a soul than another; which is equivalent to admitting that harmony is not more or less harmony, or more or less completely a harmony?

\par  Quite true.

\par  And that which is not more or less a harmony is not more or less harmonized?

\par  True.

\par  And that which is not more or less harmonized cannot have more or less of harmony, but only an equal harmony?

\par  Yes, an equal harmony.

\par  Then one soul not being more or less absolutely a soul than another, is not more or less harmonized?

\par  Exactly.

\par  And therefore has neither more nor less of discord, nor yet of harmony?

\par  She has not.

\par  And having neither more nor less of harmony or of discord, one soul has no more vice or virtue than another, if vice be discord and virtue harmony?

\par  Not at all more.

\par  Or speaking more correctly, Simmias, the soul, if she is a harmony, will never have any vice; because a harmony, being absolutely a harmony, has no part in the inharmonical.

\par  No.

\par  And therefore a soul which is absolutely a soul has no vice?

\par  How can she have, if the previous argument holds?

\par  Then, if all souls are equally by their nature souls, all souls of all living creatures will be equally good?

\par  I agree with you, Socrates, he said.

\par  And can all this be true, think you? he said; for these are the consequences which seem to follow from the assumption that the soul is a harmony?

\par  It cannot be true.

\par  Once more, he said, what ruler is there of the elements of human nature other than the soul, and especially the wise soul? Do you know of any?

\par  Indeed, I do not.

\par  And is the soul in agreement with the affections of the body? or is she at variance with them? For example, when the body is hot and thirsty, does not the soul incline us against drinking? and when the body is hungry, against eating? And this is only one instance out of ten thousand of the opposition of the soul to the things of the body.

\par  Very true.

\par  But we have already acknowledged that the soul, being a harmony, can never utter a note at variance with the tensions and relaxations and vibrations and other affections of the strings out of which she is composed; she can only follow, she cannot lead them?

\par  It must be so, he replied.

\par  And yet do we not now discover the soul to be doing the exact opposite—leading the elements of which she is believed to be composed; almost always opposing and coercing them in all sorts of ways throughout life, sometimes more violently with the pains of medicine and gymnastic; then again more gently; now threatening, now admonishing the desires, passions, fears, as if talking to a thing which is not herself, as Homer in the Odyssee represents Odysseus doing in the words—

\par  'He beat his breast, and thus reproached his heart: Endure, my heart; far worse hast thou endured!'

\par  Do you think that Homer wrote this under the idea that the soul is a harmony capable of being led by the affections of the body, and not rather of a nature which should lead and master them—herself a far diviner thing than any harmony?

\par  Yes, Socrates, I quite think so.

\par  Then, my friend, we can never be right in saying that the soul is a harmony, for we should contradict the divine Homer, and contradict ourselves.

\par  True, he said.

\par  Thus much, said Socrates, of Harmonia, your Theban goddess, who has graciously yielded to us; but what shall I say, Cebes, to her husband Cadmus, and how shall I make peace with him?

\par  I think that you will discover a way of propitiating him, said Cebes; I am sure that you have put the argument with Harmonia in a manner that I could never have expected. For when Simmias was mentioning his difficulty, I quite imagined that no answer could be given to him, and therefore I was surprised at finding that his argument could not sustain the first onset of yours, and not impossibly the other, whom you call Cadmus, may share a similar fate.

\par  Nay, my good friend, said Socrates, let us not boast, lest some evil eye should put to flight the word which I am about to speak. That, however, may be left in the hands of those above, while I draw near in Homeric fashion, and try the mettle of your words. Here lies the point:—You want to have it proven to you that the soul is imperishable and immortal, and the philosopher who is confident in death appears to you to have but a vain and foolish confidence, if he believes that he will fare better in the world below than one who has led another sort of life, unless he can prove this; and you say that the demonstration of the strength and divinity of the soul, and of her existence prior to our becoming men, does not necessarily imply her immortality. Admitting the soul to be longlived, and to have known and done much in a former state, still she is not on that account immortal; and her entrance into the human form may be a sort of disease which is the beginning of dissolution, and may at last, after the toils of life are over, end in that which is called death. And whether the soul enters into the body once only or many times, does not, as you say, make any difference in the fears of individuals. For any man, who is not devoid of sense, must fear, if he has no knowledge and can give no account of the soul's immortality. This, or something like this, I suspect to be your notion, Cebes; and I designedly recur to it in order that nothing may escape us, and that you may, if you wish, add or subtract anything.

\par  But, said Cebes, as far as I see at present, I have nothing to add or subtract: I mean what you say that I mean.

\par  Socrates paused awhile, and seemed to be absorbed in reflection. At length he said: You are raising a tremendous question, Cebes, involving the whole nature of generation and corruption, about which, if you like, I will give you my own experience; and if anything which I say is likely to avail towards the solution of your difficulty you may make use of it.

\par  I should very much like, said Cebes, to hear what you have to say.

\par  Then I will tell you, said Socrates. When I was young, Cebes, I had a prodigious desire to know that department of philosophy which is called the investigation of nature; to know the causes of things, and why a thing is and is created or destroyed appeared to me to be a lofty profession; and I was always agitating myself with the consideration of questions such as these:—Is the growth of animals the result of some decay which the hot and cold principle contracts, as some have said? Is the blood the element with which we think, or the air, or the fire? or perhaps nothing of the kind—but the brain may be the originating power of the perceptions of hearing and sight and smell, and memory and opinion may come from them, and science may be based on memory and opinion when they have attained fixity. And then I went on to examine the corruption of them, and then to the things of heaven and earth, and at last I concluded myself to be utterly and absolutely incapable of these enquiries, as I will satisfactorily prove to you. For I was fascinated by them to such a degree that my eyes grew blind to things which I had seemed to myself, and also to others, to know quite well; I forgot what I had before thought self-evident truths; e.g. such a fact as that the growth of man is the result of eating and drinking; for when by the digestion of food flesh is added to flesh and bone to bone, and whenever there is an aggregation of congenial elements, the lesser bulk becomes larger and the small man great. Was not that a reasonable notion?

\par  Yes, said Cebes, I think so.

\par  Well; but let me tell you something more. There was a time when I thought that I understood the meaning of greater and less pretty well; and when I saw a great man standing by a little one, I fancied that one was taller than the other by a head; or one horse would appear to be greater than another horse: and still more clearly did I seem to perceive that ten is two more than eight, and that two cubits are more than one, because two is the double of one.

\par  And what is now your notion of such matters? said Cebes.

\par  I should be far enough from imagining, he replied, that I knew the cause of any of them, by heaven I should; for I cannot satisfy myself that, when one is added to one, the one to which the addition is made becomes two, or that the two units added together make two by reason of the addition. I cannot understand how, when separated from the other, each of them was one and not two, and now, when they are brought together, the mere juxtaposition or meeting of them should be the cause of their becoming two: neither can I understand how the division of one is the way to make two; for then a different cause would produce the same effect,—as in the former instance the addition and juxtaposition of one to one was the cause of two, in this the separation and subtraction of one from the other would be the cause. Nor am I any longer satisfied that I understand the reason why one or anything else is either generated or destroyed or is at all, but I have in my mind some confused notion of a new method, and can never admit the other.

\par  Then I heard some one reading, as he said, from a book of Anaxagoras, that mind was the disposer and cause of all, and I was delighted at this notion, which appeared quite admirable, and I said to myself: If mind is the disposer, mind will dispose all for the best, and put each particular in the best place; and I argued that if any one desired to find out the cause of the generation or destruction or existence of anything, he must find out what state of being or doing or suffering was best for that thing, and therefore a man had only to consider the best for himself and others, and then he would also know the worse, since the same science comprehended both. And I rejoiced to think that I had found in Anaxagoras a teacher of the causes of existence such as I desired, and I imagined that he would tell me first whether the earth is flat or round; and whichever was true, he would proceed to explain the cause and the necessity of this being so, and then he would teach me the nature of the best and show that this was best; and if he said that the earth was in the centre, he would further explain that this position was the best, and I should be satisfied with the explanation given, and not want any other sort of cause. And I thought that I would then go on and ask him about the sun and moon and stars, and that he would explain to me their comparative swiftness, and their returnings and various states, active and passive, and how all of them were for the best. For I could not imagine that when he spoke of mind as the disposer of them, he would give any other account of their being as they are, except that this was best; and I thought that when he had explained to me in detail the cause of each and the cause of all, he would go on to explain to me what was best for each and what was good for all. These hopes I would not have sold for a large sum of money, and I seized the books and read them as fast as I could in my eagerness to know the better and the worse.

\par  What expectations I had formed, and how grievously was I disappointed! As I proceeded, I found my philosopher altogether forsaking mind or any other principle of order, but having recourse to air, and ether, and water, and other eccentricities. I might compare him to a person who began by maintaining generally that mind is the cause of the actions of Socrates, but who, when he endeavoured to explain the causes of my several actions in detail, went on to show that I sit here because my body is made up of bones and muscles; and the bones, as he would say, are hard and have joints which divide them, and the muscles are elastic, and they cover the bones, which have also a covering or environment of flesh and skin which contains them; and as the bones are lifted at their joints by the contraction or relaxation of the muscles, I am able to bend my limbs, and this is why I am sitting here in a curved posture—that is what he would say, and he would have a similar explanation of my talking to you, which he would attribute to sound, and air, and hearing, and he would assign ten thousand other causes of the same sort, forgetting to mention the true cause, which is, that the Athenians have thought fit to condemn me, and accordingly I have thought it better and more right to remain here and undergo my sentence; for I am inclined to think that these muscles and bones of mine would have gone off long ago to Megara or Boeotia—by the dog they would, if they had been moved only by their own idea of what was best, and if I had not chosen the better and nobler part, instead of playing truant and running away, of enduring any punishment which the state inflicts. There is surely a strange confusion of causes and conditions in all this. It may be said, indeed, that without bones and muscles and the other parts of the body I cannot execute my purposes. But to say that I do as I do because of them, and that this is the way in which mind acts, and not from the choice of the best, is a very careless and idle mode of speaking. I wonder that they cannot distinguish the cause from the condition, which the many, feeling about in the dark, are always mistaking and misnaming. And thus one man makes a vortex all round and steadies the earth by the heaven; another gives the air as a support to the earth, which is a sort of broad trough. Any power which in arranging them as they are arranges them for the best never enters into their minds; and instead of finding any superior strength in it, they rather expect to discover another Atlas of the world who is stronger and more everlasting and more containing than the good;—of the obligatory and containing power of the good they think nothing; and yet this is the principle which I would fain learn if any one would teach me. But as I have failed either to discover myself, or to learn of any one else, the nature of the best, I will exhibit to you, if you like, what I have found to be the second best mode of enquiring into the cause.

\par  I should very much like to hear, he replied.

\par  Socrates proceeded:—I thought that as I had failed in the contemplation of true existence, I ought to be careful that I did not lose the eye of my soul; as people may injure their bodily eye by observing and gazing on the sun during an eclipse, unless they take the precaution of only looking at the image reflected in the water, or in some similar medium. So in my own case, I was afraid that my soul might be blinded altogether if I looked at things with my eyes or tried to apprehend them by the help of the senses. And I thought that I had better have recourse to the world of mind and seek there the truth of existence. I dare say that the simile is not perfect—for I am very far from admitting that he who contemplates existences through the medium of thought, sees them only 'through a glass darkly,' any more than he who considers them in action and operation. However, this was the method which I adopted: I first assumed some principle which I judged to be the strongest, and then I affirmed as true whatever seemed to agree with this, whether relating to the cause or to anything else; and that which disagreed I regarded as untrue. But I should like to explain my meaning more clearly, as I do not think that you as yet understand me.

\par  No indeed, replied Cebes, not very well.

\par  There is nothing new, he said, in what I am about to tell you; but only what I have been always and everywhere repeating in the previous discussion and on other occasions: I want to show you the nature of that cause which has occupied my thoughts. I shall have to go back to those familiar words which are in the mouth of every one, and first of all assume that there is an absolute beauty and goodness and greatness, and the like; grant me this, and I hope to be able to show you the nature of the cause, and to prove the immortality of the soul.

\par  Cebes said: You may proceed at once with the proof, for I grant you this.

\par  Well, he said, then I should like to know whether you agree with me in the next step; for I cannot help thinking, if there be anything beautiful other than absolute beauty should there be such, that it can be beautiful only in as far as it partakes of absolute beauty—and I should say the same of everything. Do you agree in this notion of the cause?

\par  Yes, he said, I agree.

\par  He proceeded: I know nothing and can understand nothing of any other of those wise causes which are alleged; and if a person says to me that the bloom of colour, or form, or any such thing is a source of beauty, I leave all that, which is only confusing to me, and simply and singly, and perhaps foolishly, hold and am assured in my own mind that nothing makes a thing beautiful but the presence and participation of beauty in whatever way or manner obtained; for as to the manner I am uncertain, but I stoutly contend that by beauty all beautiful things become beautiful. This appears to me to be the safest answer which I can give, either to myself or to another, and to this I cling, in the persuasion that this principle will never be overthrown, and that to myself or to any one who asks the question, I may safely reply, That by beauty beautiful things become beautiful. Do you not agree with me?

\par  I do.

\par  And that by greatness only great things become great and greater greater, and by smallness the less become less?

\par  True.

\par  Then if a person were to remark that A is taller by a head than B, and B less by a head than A, you would refuse to admit his statement, and would stoutly contend that what you mean is only that the greater is greater by, and by reason of, greatness, and the less is less only by, and by reason of, smallness; and thus you would avoid the danger of saying that the greater is greater and the less less by the measure of the head, which is the same in both, and would also avoid the monstrous absurdity of supposing that the greater man is greater by reason of the head, which is small. You would be afraid to draw such an inference, would you not?

\par  Indeed, I should, said Cebes, laughing.

\par  In like manner you would be afraid to say that ten exceeded eight by, and by reason of, two; but would say by, and by reason of, number; or you would say that two cubits exceed one cubit not by a half, but by magnitude?-for there is the same liability to error in all these cases.

\par  Very true, he said.

\par  Again, would you not be cautious of affirming that the addition of one to one, or the division of one, is the cause of two? And you would loudly asseverate that you know of no way in which anything comes into existence except by participation in its own proper essence, and consequently, as far as you know, the only cause of two is the participation in duality—this is the way to make two, and the participation in one is the way to make one. You would say: I will let alone puzzles of division and addition—wiser heads than mine may answer them; inexperienced as I am, and ready to start, as the proverb says, at my own shadow, I cannot afford to give up the sure ground of a principle. And if any one assails you there, you would not mind him, or answer him, until you had seen whether the consequences which follow agree with one another or not, and when you are further required to give an explanation of this principle, you would go on to assume a higher principle, and a higher, until you found a resting-place in the best of the higher; but you would not confuse the principle and the consequences in your reasoning, like the Eristics—at least if you wanted to discover real existence. Not that this confusion signifies to them, who never care or think about the matter at all, for they have the wit to be well pleased with themselves however great may be the turmoil of their ideas. But you, if you are a philosopher, will certainly do as I say.

\par  What you say is most true, said Simmias and Cebes, both speaking at once.

\par \textbf{ECHECRATES}
\par   Yes, Phaedo; and I do not wonder at their assenting. Any one who has the least sense will acknowledge the wonderful clearness of Socrates' reasoning.

\par \textbf{PHAEDO}
\par   Certainly, Echecrates; and such was the feeling of the whole company at the time.

\par \textbf{ECHECRATES}
\par   Yes, and equally of ourselves, who were not of the company, and are now listening to your recital. But what followed?

\par \textbf{PHAEDO}
\par   After all this had been admitted, and they had that ideas exist, and that other things participate in them and derive their names from them, Socrates, if I remember rightly, said: —

\par  This is your way of speaking; and yet when you say that Simmias is greater than Socrates and less than Phaedo, do you not predicate of Simmias both greatness and smallness?

\par  Yes, I do.

\par  But still you allow that Simmias does not really exceed Socrates, as the words may seem to imply, because he is Simmias, but by reason of the size which he has; just as Simmias does not exceed Socrates because he is Simmias, any more than because Socrates is Socrates, but because he has smallness when compared with the greatness of Simmias?

\par  True.

\par  And if Phaedo exceeds him in size, this is not because Phaedo is Phaedo, but because Phaedo has greatness relatively to Simmias, who is comparatively smaller?

\par  That is true.

\par  And therefore Simmias is said to be great, and is also said to be small, because he is in a mean between them, exceeding the smallness of the one by his greatness, and allowing the greatness of the other to exceed his smallness. He added, laughing, I am speaking like a book, but I believe that what I am saying is true.

\par  Simmias assented.

\par  I speak as I do because I want you to agree with me in thinking, not only that absolute greatness will never be great and also small, but that greatness in us or in the concrete will never admit the small or admit of being exceeded: instead of this, one of two things will happen, either the greater will fly or retire before the opposite, which is the less, or at the approach of the less has already ceased to exist; but will not, if allowing or admitting of smallness, be changed by that; even as I, having received and admitted smallness when compared with Simmias, remain just as I was, and am the same small person. And as the idea of greatness cannot condescend ever to be or become small, in like manner the smallness in us cannot be or become great; nor can any other opposite which remains the same ever be or become its own opposite, but either passes away or perishes in the change.

\par  That, replied Cebes, is quite my notion.

\par  Hereupon one of the company, though I do not exactly remember which of them, said: In heaven's name, is not this the direct contrary of what was admitted before—that out of the greater came the less and out of the less the greater, and that opposites were simply generated from opposites; but now this principle seems to be utterly denied.

\par  Socrates inclined his head to the speaker and listened. I like your courage, he said, in reminding us of this. But you do not observe that there is a difference in the two cases. For then we were speaking of opposites in the concrete, and now of the essential opposite which, as is affirmed, neither in us nor in nature can ever be at variance with itself: then, my friend, we were speaking of things in which opposites are inherent and which are called after them, but now about the opposites which are inherent in them and which give their name to them; and these essential opposites will never, as we maintain, admit of generation into or out of one another. At the same time, turning to Cebes, he said: Are you at all disconcerted, Cebes, at our friend's objection?

\par  No, I do not feel so, said Cebes; and yet I cannot deny that I am often disturbed by objections.

\par  Then we are agreed after all, said Socrates, that the opposite will never in any case be opposed to itself?

\par  To that we are quite agreed, he replied.

\par  Yet once more let me ask you to consider the question from another point of view, and see whether you agree with me:—There is a thing which you term heat, and another thing which you term cold?

\par  Certainly.

\par  But are they the same as fire and snow?

\par  Most assuredly not.

\par  Heat is a thing different from fire, and cold is not the same with snow?

\par  Yes.

\par  And yet you will surely admit, that when snow, as was before said, is under the influence of heat, they will not remain snow and heat; but at the advance of the heat, the snow will either retire or perish?

\par  Very true, he replied.

\par  And the fire too at the advance of the cold will either retire or perish; and when the fire is under the influence of the cold, they will not remain as before, fire and cold.

\par  That is true, he said.

\par  And in some cases the name of the idea is not only attached to the idea in an eternal connection, but anything else which, not being the idea, exists only in the form of the idea, may also lay claim to it. I will try to make this clearer by an example:—The odd number is always called by the name of odd?

\par  Very true.

\par  But is this the only thing which is called odd? Are there not other things which have their own name, and yet are called odd, because, although not the same as oddness, they are never without oddness?—that is what I mean to ask—whether numbers such as the number three are not of the class of odd. And there are many other examples: would you not say, for example, that three may be called by its proper name, and also be called odd, which is not the same with three? and this may be said not only of three but also of five, and of every alternate number—each of them without being oddness is odd, and in the same way two and four, and the other series of alternate numbers, has every number even, without being evenness. Do you agree?

\par  Of course.

\par  Then now mark the point at which I am aiming:—not only do essential opposites exclude one another, but also concrete things, which, although not in themselves opposed, contain opposites; these, I say, likewise reject the idea which is opposed to that which is contained in them, and when it approaches them they either perish or withdraw. For example; Will not the number three endure annihilation or anything sooner than be converted into an even number, while remaining three?

\par  Very true, said Cebes.

\par  And yet, he said, the number two is certainly not opposed to the number three?

\par  It is not.

\par  Then not only do opposite ideas repel the advance of one another, but also there are other natures which repel the approach of opposites.

\par  Very true, he said.

\par  Suppose, he said, that we endeavour, if possible, to determine what these are.

\par  By all means.

\par  Are they not, Cebes, such as compel the things of which they have possession, not only to take their own form, but also the form of some opposite?

\par  What do you mean?

\par  I mean, as I was just now saying, and as I am sure that you know, that those things which are possessed by the number three must not only be three in number, but must also be odd.

\par  Quite true.

\par  And on this oddness, of which the number three has the impress, the opposite idea will never intrude?

\par  No.

\par  And this impress was given by the odd principle?

\par  Yes.

\par  And to the odd is opposed the even?

\par  True.

\par  Then the idea of the even number will never arrive at three?

\par  No.

\par  Then three has no part in the even?

\par  None.

\par  Then the triad or number three is uneven?

\par  Very true.

\par  To return then to my distinction of natures which are not opposed, and yet do not admit opposites—as, in the instance given, three, although not opposed to the even, does not any the more admit of the even, but always brings the opposite into play on the other side; or as two does not receive the odd, or fire the cold—from these examples (and there are many more of them) perhaps you may be able to arrive at the general conclusion, that not only opposites will not receive opposites, but also that nothing which brings the opposite will admit the opposite of that which it brings, in that to which it is brought. And here let me recapitulate—for there is no harm in repetition. The number five will not admit the nature of the even, any more than ten, which is the double of five, will admit the nature of the odd. The double has another opposite, and is not strictly opposed to the odd, but nevertheless rejects the odd altogether. Nor again will parts in the ratio 3:2, nor any fraction in which there is a half, nor again in which there is a third, admit the notion of the whole, although they are not opposed to the whole: You will agree?

\par  Yes, he said, I entirely agree and go along with you in that.

\par  And now, he said, let us begin again; and do not you answer my question in the words in which I ask it: let me have not the old safe answer of which I spoke at first, but another equally safe, of which the truth will be inferred by you from what has been just said. I mean that if any one asks you 'what that is, of which the inherence makes the body hot,' you will reply not heat (this is what I call the safe and stupid answer), but fire, a far superior answer, which we are now in a condition to give. Or if any one asks you 'why a body is diseased,' you will not say from disease, but from fever; and instead of saying that oddness is the cause of odd numbers, you will say that the monad is the cause of them: and so of things in general, as I dare say that you will understand sufficiently without my adducing any further examples.

\par  Yes, he said, I quite understand you.

\par  Tell me, then, what is that of which the inherence will render the body alive?

\par  The soul, he replied.

\par  And is this always the case?

\par  Yes, he said, of course.

\par  Then whatever the soul possesses, to that she comes bearing life?

\par  Yes, certainly.

\par  And is there any opposite to life?

\par  There is, he said.

\par  And what is that?

\par  Death.

\par  Then the soul, as has been acknowledged, will never receive the opposite of what she brings.

\par  Impossible, replied Cebes.

\par  And now, he said, what did we just now call that principle which repels the even?

\par  The odd.

\par  And that principle which repels the musical, or the just?

\par  The unmusical, he said, and the unjust.

\par  And what do we call the principle which does not admit of death?

\par  The immortal, he said.

\par  And does the soul admit of death?

\par  No.

\par  Then the soul is immortal?

\par  Yes, he said.

\par  And may we say that this has been proven?

\par  Yes, abundantly proven, Socrates, he replied.

\par  Supposing that the odd were imperishable, must not three be imperishable?

\par  Of course.

\par  And if that which is cold were imperishable, when the warm principle came attacking the snow, must not the snow have retired whole and unmelted—for it could never have perished, nor could it have remained and admitted the heat?

\par  True, he said.

\par  Again, if the uncooling or warm principle were imperishable, the fire when assailed by cold would not have perished or have been extinguished, but would have gone away unaffected?

\par  Certainly, he said.

\par  And the same may be said of the immortal: if the immortal is also imperishable, the soul when attacked by death cannot perish; for the preceding argument shows that the soul will not admit of death, or ever be dead, any more than three or the odd number will admit of the even, or fire or the heat in the fire, of the cold. Yet a person may say: 'But although the odd will not become even at the approach of the even, why may not the odd perish and the even take the place of the odd?' Now to him who makes this objection, we cannot answer that the odd principle is imperishable; for this has not been acknowledged, but if this had been acknowledged, there would have been no difficulty in contending that at the approach of the even the odd principle and the number three took their departure; and the same argument would have held good of fire and heat and any other thing.

\par  Very true.

\par  And the same may be said of the immortal: if the immortal is also imperishable, then the soul will be imperishable as well as immortal; but if not, some other proof of her imperishableness will have to be given.

\par  No other proof is needed, he said; for if the immortal, being eternal, is liable to perish, then nothing is imperishable.

\par  Yes, replied Socrates, and yet all men will agree that God, and the essential form of life, and the immortal in general, will never perish.

\par  Yes, all men, he said—that is true; and what is more, gods, if I am not mistaken, as well as men.

\par  Seeing then that the immortal is indestructible, must not the soul, if she is immortal, be also imperishable?

\par  Most certainly.

\par  Then when death attacks a man, the mortal portion of him may be supposed to die, but the immortal retires at the approach of death and is preserved safe and sound?

\par  True.

\par  Then, Cebes, beyond question, the soul is immortal and imperishable, and our souls will truly exist in another world!

\par  I am convinced, Socrates, said Cebes, and have nothing more to object; but if my friend Simmias, or any one else, has any further objection to make, he had better speak out, and not keep silence, since I do not know to what other season he can defer the discussion, if there is anything which he wants to say or to have said.

\par  But I have nothing more to say, replied Simmias; nor can I see any reason for doubt after what has been said. But I still feel and cannot help feeling uncertain in my own mind, when I think of the greatness of the subject and the feebleness of man.

\par  Yes, Simmias, replied Socrates, that is well said: and I may add that first principles, even if they appear certain, should be carefully considered; and when they are satisfactorily ascertained, then, with a sort of hesitating confidence in human reason, you may, I think, follow the course of the argument; and if that be plain and clear, there will be no need for any further enquiry.

\par  Very true.

\par  But then, O my friends, he said, if the soul is really immortal, what care should be taken of her, not only in respect of the portion of time which is called life, but of eternity! And the danger of neglecting her from this point of view does indeed appear to be awful. If death had only been the end of all, the wicked would have had a good bargain in dying, for they would have been happily quit not only of their body, but of their own evil together with their souls. But now, inasmuch as the soul is manifestly immortal, there is no release or salvation from evil except the attainment of the highest virtue and wisdom. For the soul when on her progress to the world below takes nothing with her but nurture and education; and these are said greatly to benefit or greatly to injure the departed, at the very beginning of his journey thither.

\par  For after death, as they say, the genius of each individual, to whom he belonged in life, leads him to a certain place in which the dead are gathered together, whence after judgment has been given they pass into the world below, following the guide, who is appointed to conduct them from this world to the other: and when they have there received their due and remained their time, another guide brings them back again after many revolutions of ages. Now this way to the other world is not, as Aeschylus says in the Telephus, a single and straight path—if that were so no guide would be needed, for no one could miss it; but there are many partings of the road, and windings, as I infer from the rites and sacrifices which are offered to the gods below in places where three ways meet on earth. The wise and orderly soul follows in the straight path and is conscious of her surroundings; but the soul which desires the body, and which, as I was relating before, has long been fluttering about the lifeless frame and the world of sight, is after many struggles and many sufferings hardly and with violence carried away by her attendant genius, and when she arrives at the place where the other souls are gathered, if she be impure and have done impure deeds, whether foul murders or other crimes which are the brothers of these, and the works of brothers in crime—from that soul every one flees and turns away; no one will be her companion, no one her guide, but alone she wanders in extremity of evil until certain times are fulfilled, and when they are fulfilled, she is borne irresistibly to her own fitting habitation; as every pure and just soul which has passed through life in the company and under the guidance of the gods has also her own proper home.

\par  Now the earth has divers wonderful regions, and is indeed in nature and extent very unlike the notions of geographers, as I believe on the authority of one who shall be nameless.

\par  What do you mean, Socrates? said Simmias. I have myself heard many descriptions of the earth, but I do not know, and I should very much like to know, in which of these you put faith.

\par  And I, Simmias, replied Socrates, if I had the art of Glaucus would tell you; although I know not that the art of Glaucus could prove the truth of my tale, which I myself should never be able to prove, and even if I could, I fear, Simmias, that my life would come to an end before the argument was completed. I may describe to you, however, the form and regions of the earth according to my conception of them.

\par  That, said Simmias, will be enough.

\par  Well, then, he said, my conviction is, that the earth is a round body in the centre of the heavens, and therefore has no need of air or any similar force to be a support, but is kept there and hindered from falling or inclining any way by the equability of the surrounding heaven and by her own equipoise. For that which, being in equipoise, is in the centre of that which is equably diffused, will not incline any way in any degree, but will always remain in the same state and not deviate. And this is my first notion.

\par  Which is surely a correct one, said Simmias.

\par  Also I believe that the earth is very vast, and that we who dwell in the region extending from the river Phasis to the Pillars of Heracles inhabit a small portion only about the sea, like ants or frogs about a marsh, and that there are other inhabitants of many other like places; for everywhere on the face of the earth there are hollows of various forms and sizes, into which the water and the mist and the lower air collect. But the true earth is pure and situated in the pure heaven—there are the stars also; and it is the heaven which is commonly spoken of by us as the ether, and of which our own earth is the sediment gathering in the hollows beneath. But we who live in these hollows are deceived into the notion that we are dwelling above on the surface of the earth; which is just as if a creature who was at the bottom of the sea were to fancy that he was on the surface of the water, and that the sea was the heaven through which he saw the sun and the other stars, he having never come to the surface by reason of his feebleness and sluggishness, and having never lifted up his head and seen, nor ever heard from one who had seen, how much purer and fairer the world above is than his own. And such is exactly our case: for we are dwelling in a hollow of the earth, and fancy that we are on the surface; and the air we call the heaven, in which we imagine that the stars move. But the fact is, that owing to our feebleness and sluggishness we are prevented from reaching the surface of the air: for if any man could arrive at the exterior limit, or take the wings of a bird and come to the top, then like a fish who puts his head out of the water and sees this world, he would see a world beyond; and, if the nature of man could sustain the sight, he would acknowledge that this other world was the place of the true heaven and the true light and the true earth. For our earth, and the stones, and the entire region which surrounds us, are spoilt and corroded, as in the sea all things are corroded by the brine, neither is there any noble or perfect growth, but caverns only, and sand, and an endless slough of mud: and even the shore is not to be compared to the fairer sights of this world. And still less is this our world to be compared with the other. Of that upper earth which is under the heaven, I can tell you a charming tale, Simmias, which is well worth hearing.

\par  And we, Socrates, replied Simmias, shall be charmed to listen to you.

\par  The tale, my friend, he said, is as follows:—In the first place, the earth, when looked at from above, is in appearance streaked like one of those balls which have leather coverings in twelve pieces, and is decked with various colours, of which the colours used by painters on earth are in a manner samples. But there the whole earth is made up of them, and they are brighter far and clearer than ours; there is a purple of wonderful lustre, also the radiance of gold, and the white which is in the earth is whiter than any chalk or snow. Of these and other colours the earth is made up, and they are more in number and fairer than the eye of man has ever seen; the very hollows (of which I was speaking) filled with air and water have a colour of their own, and are seen like light gleaming amid the diversity of the other colours, so that the whole presents a single and continuous appearance of variety in unity. And in this fair region everything that grows—trees, and flowers, and fruits—are in a like degree fairer than any here; and there are hills, having stones in them in a like degree smoother, and more transparent, and fairer in colour than our highly-valued emeralds and sardonyxes and jaspers, and other gems, which are but minute fragments of them: for there all the stones are like our precious stones, and fairer still (compare Republic). The reason is, that they are pure, and not, like our precious stones, infected or corroded by the corrupt briny elements which coagulate among us, and which breed foulness and disease both in earth and stones, as well as in animals and plants. They are the jewels of the upper earth, which also shines with gold and silver and the like, and they are set in the light of day and are large and abundant and in all places, making the earth a sight to gladden the beholder's eye. And there are animals and men, some in a middle region, others dwelling about the air as we dwell about the sea; others in islands which the air flows round, near the continent: and in a word, the air is used by them as the water and the sea are by us, and the ether is to them what the air is to us. Moreover, the temperament of their seasons is such that they have no disease, and live much longer than we do, and have sight and hearing and smell, and all the other senses, in far greater perfection, in the same proportion that air is purer than water or the ether than air. Also they have temples and sacred places in which the gods really dwell, and they hear their voices and receive their answers, and are conscious of them and hold converse with them, and they see the sun, moon, and stars as they truly are, and their other blessedness is of a piece with this.

\par  Such is the nature of the whole earth, and of the things which are around the earth; and there are divers regions in the hollows on the face of the globe everywhere, some of them deeper and more extended than that which we inhabit, others deeper but with a narrower opening than ours, and some are shallower and also wider. All have numerous perforations, and there are passages broad and narrow in the interior of the earth, connecting them with one another; and there flows out of and into them, as into basins, a vast tide of water, and huge subterranean streams of perennial rivers, and springs hot and cold, and a great fire, and great rivers of fire, and streams of liquid mud, thin or thick (like the rivers of mud in Sicily, and the lava streams which follow them), and the regions about which they happen to flow are filled up with them. And there is a swinging or see-saw in the interior of the earth which moves all this up and down, and is due to the following cause:—There is a chasm which is the vastest of them all, and pierces right through the whole earth; this is that chasm which Homer describes in the words,—
 
\par  and which he in other places, and many other poets, have called Tartarus. And the see-saw is caused by the streams flowing into and out of this chasm, and they each have the nature of the soil through which they flow. And the reason why the streams are always flowing in and out, is that the watery element has no bed or bottom, but is swinging and surging up and down, and the surrounding wind and air do the same; they follow the water up and down, hither and thither, over the earth—just as in the act of respiration the air is always in process of inhalation and exhalation;—and the wind swinging with the water in and out produces fearful and irresistible blasts: when the waters retire with a rush into the lower parts of the earth, as they are called, they flow through the earth in those regions, and fill them up like water raised by a pump, and then when they leave those regions and rush back hither, they again fill the hollows here, and when these are filled, flow through subterranean channels and find their way to their several places, forming seas, and lakes, and rivers, and springs. Thence they again enter the earth, some of them making a long circuit into many lands, others going to a few places and not so distant; and again fall into Tartarus, some at a point a good deal lower than that at which they rose, and others not much lower, but all in some degree lower than the point from which they came. And some burst forth again on the opposite side, and some on the same side, and some wind round the earth with one or many folds like the coils of a serpent, and descend as far as they can, but always return and fall into the chasm. The rivers flowing in either direction can descend only to the centre and no further, for opposite to the rivers is a precipice.

\par  Now these rivers are many, and mighty, and diverse, and there are four principal ones, of which the greatest and outermost is that called Oceanus, which flows round the earth in a circle; and in the opposite direction flows Acheron, which passes under the earth through desert places into the Acherusian lake: this is the lake to the shores of which the souls of the many go when they are dead, and after waiting an appointed time, which is to some a longer and to some a shorter time, they are sent back to be born again as animals. The third river passes out between the two, and near the place of outlet pours into a vast region of fire, and forms a lake larger than the Mediterranean Sea, boiling with water and mud; and proceeding muddy and turbid, and winding about the earth, comes, among other places, to the extremities of the Acherusian Lake, but mingles not with the waters of the lake, and after making many coils about the earth plunges into Tartarus at a deeper level. This is that Pyriphlegethon, as the stream is called, which throws up jets of fire in different parts of the earth. The fourth river goes out on the opposite side, and falls first of all into a wild and savage region, which is all of a dark-blue colour, like lapis lazuli; and this is that river which is called the Stygian river, and falls into and forms the Lake Styx, and after falling into the lake and receiving strange powers in the waters, passes under the earth, winding round in the opposite direction, and comes near the Acherusian lake from the opposite side to Pyriphlegethon. And the water of this river too mingles with no other, but flows round in a circle and falls into Tartarus over against Pyriphlegethon; and the name of the river, as the poets say, is Cocytus.

\par  Such is the nature of the other world; and when the dead arrive at the place to which the genius of each severally guides them, first of all, they have sentence passed upon them, as they have lived well and piously or not. And those who appear to have lived neither well nor ill, go to the river Acheron, and embarking in any vessels which they may find, are carried in them to the lake, and there they dwell and are purified of their evil deeds, and having suffered the penalty of the wrongs which they have done to others, they are absolved, and receive the rewards of their good deeds, each of them according to his deserts. But those who appear to be incurable by reason of the greatness of their crimes—who have committed many and terrible deeds of sacrilege, murders foul and violent, or the like—such are hurled into Tartarus which is their suitable destiny, and they never come out. Those again who have committed crimes, which, although great, are not irremediable—who in a moment of anger, for example, have done violence to a father or a mother, and have repented for the remainder of their lives, or, who have taken the life of another under the like extenuating circumstances—these are plunged into Tartarus, the pains of which they are compelled to undergo for a year, but at the end of the year the wave casts them forth—mere homicides by way of Cocytus, parricides and matricides by Pyriphlegethon—and they are borne to the Acherusian lake, and there they lift up their voices and call upon the victims whom they have slain or wronged, to have pity on them, and to be kind to them, and let them come out into the lake. And if they prevail, then they come forth and cease from their troubles; but if not, they are carried back again into Tartarus and from thence into the rivers unceasingly, until they obtain mercy from those whom they have wronged: for that is the sentence inflicted upon them by their judges. Those too who have been pre-eminent for holiness of life are released from this earthly prison, and go to their pure home which is above, and dwell in the purer earth; and of these, such as have duly purified themselves with philosophy live henceforth altogether without the body, in mansions fairer still which may not be described, and of which the time would fail me to tell.

\par  Wherefore, Simmias, seeing all these things, what ought not we to do that we may obtain virtue and wisdom in this life? Fair is the prize, and the hope great!

\par  A man of sense ought not to say, nor will I be very confident, that the description which I have given of the soul and her mansions is exactly true. But I do say that, inasmuch as the soul is shown to be immortal, he may venture to think, not improperly or unworthily, that something of the kind is true. The venture is a glorious one, and he ought to comfort himself with words like these, which is the reason why I lengthen out the tale. Wherefore, I say, let a man be of good cheer about his soul, who having cast away the pleasures and ornaments of the body as alien to him and working harm rather than good, has sought after the pleasures of knowledge; and has arrayed the soul, not in some foreign attire, but in her own proper jewels, temperance, and justice, and courage, and nobility, and truth—in these adorned she is ready to go on her journey to the world below, when her hour comes. You, Simmias and Cebes, and all other men, will depart at some time or other. Me already, as the tragic poet would say, the voice of fate calls. Soon I must drink the poison; and I think that I had better repair to the bath first, in order that the women may not have the trouble of washing my body after I am dead.

\par  When he had done speaking, Crito said: And have you any commands for us, Socrates—anything to say about your children, or any other matter in which we can serve you?

\par  Nothing particular, Crito, he replied: only, as I have always told you, take care of yourselves; that is a service which you may be ever rendering to me and mine and to all of us, whether you promise to do so or not. But if you have no thought for yourselves, and care not to walk according to the rule which I have prescribed for you, not now for the first time, however much you may profess or promise at the moment, it will be of no avail.

\par  We will do our best, said Crito: And in what way shall we bury you?

\par  In any way that you like; but you must get hold of me, and take care that I do not run away from you. Then he turned to us, and added with a smile:—I cannot make Crito believe that I am the same Socrates who have been talking and conducting the argument; he fancies that I am the other Socrates whom he will soon see, a dead body—and he asks, How shall he bury me? And though I have spoken many words in the endeavour to show that when I have drunk the poison I shall leave you and go to the joys of the blessed,—these words of mine, with which I was comforting you and myself, have had, as I perceive, no effect upon Crito. And therefore I want you to be surety for me to him now, as at the trial he was surety to the judges for me: but let the promise be of another sort; for he was surety for me to the judges that I would remain, and you must be my surety to him that I shall not remain, but go away and depart; and then he will suffer less at my death, and not be grieved when he sees my body being burned or buried. I would not have him sorrow at my hard lot, or say at the burial, Thus we lay out Socrates, or, Thus we follow him to the grave or bury him; for false words are not only evil in themselves, but they infect the soul with evil. Be of good cheer, then, my dear Crito, and say that you are burying my body only, and do with that whatever is usual, and what you think best.

\par  When he had spoken these words, he arose and went into a chamber to bathe; Crito followed him and told us to wait. So we remained behind, talking and thinking of the subject of discourse, and also of the greatness of our sorrow; he was like a father of whom we were being bereaved, and we were about to pass the rest of our lives as orphans. When he had taken the bath his children were brought to him—(he had two young sons and an elder one); and the women of his family also came, and he talked to them and gave them a few directions in the presence of Crito; then he dismissed them and returned to us.

\par  Now the hour of sunset was near, for a good deal of time had passed while he was within. When he came out, he sat down with us again after his bath, but not much was said. Soon the jailer, who was the servant of the Eleven, entered and stood by him, saying:—To you, Socrates, whom I know to be the noblest and gentlest and best of all who ever came to this place, I will not impute the angry feelings of other men, who rage and swear at me, when, in obedience to the authorities, I bid them drink the poison—indeed, I am sure that you will not be angry with me; for others, as you are aware, and not I, are to blame. And so fare you well, and try to bear lightly what must needs be—you know my errand. Then bursting into tears he turned away and went out.

\par  Socrates looked at him and said: I return your good wishes, and will do as you bid. Then turning to us, he said, How charming the man is: since I have been in prison he has always been coming to see me, and at times he would talk to me, and was as good to me as could be, and now see how generously he sorrows on my account. We must do as he says, Crito; and therefore let the cup be brought, if the poison is prepared: if not, let the attendant prepare some.

\par  Yet, said Crito, the sun is still upon the hill-tops, and I know that many a one has taken the draught late, and after the announcement has been made to him, he has eaten and drunk, and enjoyed the society of his beloved; do not hurry—there is time enough.

\par  Socrates said: Yes, Crito, and they of whom you speak are right in so acting, for they think that they will be gainers by the delay; but I am right in not following their example, for I do not think that I should gain anything by drinking the poison a little later; I should only be ridiculous in my own eyes for sparing and saving a life which is already forfeit. Please then to do as I say, and not to refuse me.

\par  Crito made a sign to the servant, who was standing by; and he went out, and having been absent for some time, returned with the jailer carrying the cup of poison. Socrates said: You, my good friend, who are experienced in these matters, shall give me directions how I am to proceed. The man answered: You have only to walk about until your legs are heavy, and then to lie down, and the poison will act. At the same time he handed the cup to Socrates, who in the easiest and gentlest manner, without the least fear or change of colour or feature, looking at the man with all his eyes, Echecrates, as his manner was, took the cup and said: What do you say about making a libation out of this cup to any god? May I, or not? The man answered: We only prepare, Socrates, just so much as we deem enough. I understand, he said: but I may and must ask the gods to prosper my journey from this to the other world—even so—and so be it according to my prayer. Then raising the cup to his lips, quite readily and cheerfully he drank off the poison. And hitherto most of us had been able to control our sorrow; but now when we saw him drinking, and saw too that he had finished the draught, we could no longer forbear, and in spite of myself my own tears were flowing fast; so that I covered my face and wept, not for him, but at the thought of my own calamity in having to part from such a friend. Nor was I the first; for Crito, when he found himself unable to restrain his tears, had got up, and I followed; and at that moment, Apollodorus, who had been weeping all the time, broke out in a loud and passionate cry which made cowards of us all. Socrates alone retained his calmness: What is this strange outcry? he said. I sent away the women mainly in order that they might not misbehave in this way, for I have been told that a man should die in peace. Be quiet, then, and have patience. When we heard his words we were ashamed, and refrained our tears; and he walked about until, as he said, his legs began to fail, and then he lay on his back, according to the directions, and the man who gave him the poison now and then looked at his feet and legs; and after a while he pressed his foot hard, and asked him if he could feel; and he said, No; and then his leg, and so upwards and upwards, and showed us that he was cold and stiff. And he felt them himself, and said: When the poison reaches the heart, that will be the end. He was beginning to grow cold about the groin, when he uncovered his face, for he had covered himself up, and said—they were his last words—he said: Crito, I owe a cock to Asclepius; will you remember to pay the debt? The debt shall be paid, said Crito; is there anything else? There was no answer to this question; but in a minute or two a movement was heard, and the attendants uncovered him; his eyes were set, and Crito closed his eyes and mouth.

\par  Such was the end, Echecrates, of our friend; concerning whom I may truly say, that of all the men of his time whom I have known, he was the wisest and justest and best.

\par 
 
\end{document}