
\documentclass[11pt,letter]{article}


\begin{document}

\title{Protagoras\thanks{Source: https://www.gutenberg.org/files/1591/1591-h/1591-h.htm. License: http://gutenberg.org/license ds}}
\date{\today}
\author{Plato, 427? BCE-347? BCE\\ Translated by Jowett, Benjamin, 1817-1893}
\maketitle

\setcounter{tocdepth}{1}
\tableofcontents
\renewcommand{\baselinestretch}{1.0}
\normalsize
\newpage

\section{
      INTRODUCTION.
    }
\par  The Protagoras, like several of the Dialogues of Plato, is put into the mouth of Socrates, who describes a conversation which had taken place between himself and the great Sophist at the house of Callias—'the man who had spent more upon the Sophists than all the rest of the world'—and in which the learned Hippias and the grammarian Prodicus had also shared, as well as Alcibiades and Critias, both of whom said a few words—in the presence of a distinguished company consisting of disciples of Protagoras and of leading Athenians belonging to the Socratic circle. The dialogue commences with a request on the part of Hippocrates that Socrates would introduce him to the celebrated teacher. He has come before the dawn had risen—so fervid is his zeal. Socrates moderates his excitement and advises him to find out 'what Protagoras will make of him,' before he becomes his pupil.

\par  They go together to the house of Callias; and Socrates, after explaining the purpose of their visit to Protagoras, asks the question, 'What he will make of Hippocrates.' Protagoras answers, 'That he will make him a better and a wiser man.' 'But in what will he be better? '—Socrates desires to have a more precise answer. Protagoras replies, 'That he will teach him prudence in affairs private and public; in short, the science or knowledge of human life.'

\par  This, as Socrates admits, is a noble profession; but he is or rather would have been doubtful, whether such knowledge can be taught, if Protagoras had not assured him of the fact, for two reasons: (1) Because the Athenian people, who recognize in their assemblies the distinction between the skilled and the unskilled in the arts, do not distinguish between the trained politician and the untrained; (2) Because the wisest and best Athenian citizens do not teach their sons political virtue. Will Protagoras answer these objections?

\par  Protagoras explains his views in the form of an apologue, in which, after Prometheus had given men the arts, Zeus is represented as sending Hermes to them, bearing with him Justice and Reverence. These are not, like the arts, to be imparted to a few only, but all men are to be partakers of them. Therefore the Athenian people are right in distinguishing between the skilled and unskilled in the arts, and not between skilled and unskilled politicians. (1) For all men have the political virtues to a certain degree, and are obliged to say that they have them, whether they have them or not. A man would be thought a madman who professed an art which he did not know; but he would be equally thought a madman if he did not profess a virtue which he had not. (2) And that the political virtues can be taught and acquired, in the opinion of the Athenians, is proved by the fact that they punish evil-doers, with a view to prevention, of course—mere retribution is for beasts, and not for men. (3) Again, would parents who teach her sons lesser matters leave them ignorant of the common duty of citizens? To the doubt of Socrates the best answer is the fact, that the education of youth in virtue begins almost as soon as they can speak, and is continued by the state when they pass out of the parental control. (4) Nor need we wonder that wise and good fathers sometimes have foolish and worthless sons. Virtue, as we were saying, is not the private possession of any man, but is shared by all, only however to the extent of which each individual is by nature capable. And, as a matter of fact, even the worst of civilized mankind will appear virtuous and just, if we compare them with savages. (5) The error of Socrates lies in supposing that there are no teachers of virtue, whereas all men are teachers in a degree. Some, like Protagoras, are better than others, and with this result we ought to be satisfied.

\par  Socrates is highly delighted with the explanation of Protagoras. But he has still a doubt lingering in his mind. Protagoras has spoken of the virtues: are they many, or one? are they parts of a whole, or different names of the same thing? Protagoras replies that they are parts, like the parts of a face, which have their several functions, and no one part is like any other part. This admission, which has been somewhat hastily made, is now taken up and cross-examined by Socrates:—

\par  'Is justice just, and is holiness holy? And are justice and holiness opposed to one another? '—'Then justice is unholy.' Protagoras would rather say that justice is different from holiness, and yet in a certain point of view nearly the same. He does not, however, escape in this way from the cunning of Socrates, who inveigles him into an admission that everything has but one opposite. Folly, for example, is opposed to wisdom; and folly is also opposed to temperance; and therefore temperance and wisdom are the same. And holiness has been already admitted to be nearly the same as justice. Temperance, therefore, has now to be compared with justice.

\par  Protagoras, whose temper begins to get a little ruffled at the process to which he has been subjected, is aware that he will soon be compelled by the dialectics of Socrates to admit that the temperate is the just. He therefore defends himself with his favourite weapon; that is to say, he makes a long speech not much to the point, which elicits the applause of the audience.

\par  Here occurs a sort of interlude, which commences with a declaration on the part of Socrates that he cannot follow a long speech, and therefore he must beg Protagoras to speak shorter. As Protagoras declines to accommodate him, he rises to depart, but is detained by Callias, who thinks him unreasonable in not allowing Protagoras the liberty which he takes himself of speaking as he likes. But Alcibiades answers that the two cases are not parallel. For Socrates admits his inability to speak long; will Protagoras in like manner acknowledge his inability to speak short?

\par  Counsels of moderation are urged first in a few words by Critias, and then by Prodicus in balanced and sententious language: and Hippias proposes an umpire. But who is to be the umpire? rejoins Socrates; he would rather suggest as a compromise that Protagoras shall ask and he will answer, and that when Protagoras is tired of asking he himself will ask and Protagoras shall answer. To this the latter yields a reluctant assent.

\par  Protagoras selects as his thesis a poem of Simonides of Ceos, in which he professes to find a contradiction. First the poet says,
 
\par  and then reproaches Pittacus for having said, 'Hard is it to be good.' How is this to be reconciled? Socrates, who is familiar with the poem, is embarrassed at first, and invokes the aid of Prodicus, the countryman of Simonides, but apparently only with the intention of flattering him into absurdities. First a distinction is drawn between (Greek) to be, and (Greek) to become: to become good is difficult; to be good is easy. Then the word difficult or hard is explained to mean 'evil' in the Cean dialect. To all this Prodicus assents; but when Protagoras reclaims, Socrates slily withdraws Prodicus from the fray, under the pretence that his assent was only intended to test the wits of his adversary. He then proceeds to give another and more elaborate explanation of the whole passage. The explanation is as follows:—

\par  The Lacedaemonians are great philosophers (although this is a fact which is not generally known); and the soul of their philosophy is brevity, which was also the style of primitive antiquity and of the seven sages. Now Pittacus had a saying, 'Hard is it to be good:' and Simonides, who was jealous of the fame of this saying, wrote a poem which was designed to controvert it. No, says he, Pittacus; not 'hard to be good,' but 'hard to become good.' Socrates proceeds to argue in a highly impressive manner that the whole composition is intended as an attack upon Pittacus. This, though manifestly absurd, is accepted by the company, and meets with the special approval of Hippias, who has however a favourite interpretation of his own, which he is requested by Alcibiades to defer.

\par  The argument is now resumed, not without some disdainful remarks of Socrates on the practice of introducing the poets, who ought not to be allowed, any more than flute-girls, to come into good society. Men's own thoughts should supply them with the materials for discussion. A few soothing flatteries are addressed to Protagoras by Callias and Socrates, and then the old question is repeated, 'Whether the virtues are one or many?' To which Protagoras is now disposed to reply, that four out of the five virtues are in some degree similar; but he still contends that the fifth, courage, is unlike the rest. Socrates proceeds to undermine the last stronghold of the adversary, first obtaining from him the admission that all virtue is in the highest degree good:—

\par  The courageous are the confident; and the confident are those who know their business or profession: those who have no such knowledge and are still confident are madmen. This is admitted. Then, says Socrates, courage is knowledge—an inference which Protagoras evades by drawing a futile distinction between the courageous and the confident in a fluent speech.

\par  Socrates renews the attack from another side: he would like to know whether pleasure is not the only good, and pain the only evil? Protagoras seems to doubt the morality or propriety of assenting to this; he would rather say that 'some pleasures are good, some pains are evil,' which is also the opinion of the generality of mankind. What does he think of knowledge? Does he agree with the common opinion that knowledge is overcome by passion? or does he hold that knowledge is power? Protagoras agrees that knowledge is certainly a governing power.

\par  This, however, is not the doctrine of men in general, who maintain that many who know what is best, act contrary to their knowledge under the influence of pleasure. But this opposition of good and evil is really the opposition of a greater or lesser amount of pleasure. Pleasures are evils because they end in pain, and pains are goods because they end in pleasures. Thus pleasure is seen to be the only good; and the only evil is the preference of the lesser pleasure to the greater. But then comes in the illusion of distance. Some art of mensuration is required in order to show us pleasures and pains in their true proportion. This art of mensuration is a kind of knowledge, and knowledge is thus proved once more to be the governing principle of human life, and ignorance the origin of all evil: for no one prefers the less pleasure to the greater, or the greater pain to the less, except from ignorance. The argument is drawn out in an imaginary 'dialogue within a dialogue,' conducted by Socrates and Protagoras on the one part, and the rest of the world on the other. Hippias and Prodicus, as well as Protagoras, admit the soundness of the conclusion.

\par  Socrates then applies this new conclusion to the case of courage—the only virtue which still holds out against the assaults of the Socratic dialectic. No one chooses the evil or refuses the good except through ignorance. This explains why cowards refuse to go to war:—because they form a wrong estimate of good, and honour, and pleasure. And why are the courageous willing to go to war?—because they form a right estimate of pleasures and pains, of things terrible and not terrible. Courage then is knowledge, and cowardice is ignorance. And the five virtues, which were originally maintained to have five different natures, after having been easily reduced to two only, at last coalesce in one. The assent of Protagoras to this last position is extracted with great difficulty.

\par  Socrates concludes by professing his disinterested love of the truth, and remarks on the singular manner in which he and his adversary had changed sides. Protagoras began by asserting, and Socrates by denying, the teachableness of virtue, and now the latter ends by affirming that virtue is knowledge, which is the most teachable of all things, while Protagoras has been striving to show that virtue is not knowledge, and this is almost equivalent to saying that virtue cannot be taught. He is not satisfied with the result, and would like to renew the enquiry with the help of Protagoras in a different order, asking (1) What virtue is, and (2) Whether virtue can be taught. Protagoras declines this offer, but commends Socrates' earnestness and his style of discussion.

\par  The Protagoras is often supposed to be full of difficulties. These are partly imaginary and partly real. The imaginary ones are (1) Chronological,—which were pointed out in ancient times by Athenaeus, and are noticed by Schleiermacher and others, and relate to the impossibility of all the persons in the Dialogue meeting at any one time, whether in the year 425 B.C., or in any other. But Plato, like all writers of fiction, aims only at the probable, and shows in many Dialogues (e.g. the Symposium and Republic, and already in the Laches) an extreme disregard of the historical accuracy which is sometimes demanded of him. (2) The exact place of the Protagoras among the Dialogues, and the date of composition, have also been much disputed. But there are no criteria which afford any real grounds for determining the date of composition; and the affinities of the Dialogues, when they are not indicated by Plato himself, must always to a great extent remain uncertain. (3) There is another class of difficulties, which may be ascribed to preconceived notions of commentators, who imagine that Protagoras the Sophist ought always to be in the wrong, and his adversary Socrates in the right; or that in this or that passage—e.g. in the explanation of good as pleasure—Plato is inconsistent with himself; or that the Dialogue fails in unity, and has not a proper beginning, middle, and ending. They seem to forget that Plato is a dramatic writer who throws his thoughts into both sides of the argument, and certainly does not aim at any unity which is inconsistent with freedom, and with a natural or even wild manner of treating his subject; also that his mode of revealing the truth is by lights and shadows, and far-off and opposing points of view, and not by dogmatic statements or definite results.

\par  The real difficulties arise out of the extreme subtlety of the work, which, as Socrates says of the poem of Simonides, is a most perfect piece of art. There are dramatic contrasts and interests, threads of philosophy broken and resumed, satirical reflections on mankind, veils thrown over truths which are lightly suggested, and all woven together in a single design, and moving towards one end.

\par  In the introductory scene Plato raises the expectation that a 'great personage' is about to appear on the stage; perhaps with a further view of showing that he is destined to be overthrown by a greater still, who makes no pretensions. Before introducing Hippocrates to him, Socrates thinks proper to warn the youth against the dangers of 'influence,' of which the invidious nature is recognized by Protagoras himself. Hippocrates readily adopts the suggestion of Socrates that he shall learn of Protagoras only the accomplishments which befit an Athenian gentleman, and let alone his 'sophistry.' There is nothing however in the introduction which leads to the inference that Plato intended to blacken the character of the Sophists; he only makes a little merry at their expense.

\par  The 'great personage' is somewhat ostentatious, but frank and honest. He is introduced on a stage which is worthy of him—at the house of the rich Callias, in which are congregated the noblest and wisest of the Athenians. He considers openness to be the best policy, and particularly mentions his own liberal mode of dealing with his pupils, as if in answer to the favourite accusation of the Sophists that they received pay. He is remarkable for the good temper which he exhibits throughout the discussion under the trying and often sophistical cross-examination of Socrates. Although once or twice ruffled, and reluctant to continue the discussion, he parts company on perfectly good terms, and appears to be, as he says of himself, the 'least jealous of mankind.'

\par  Nor is there anything in the sentiments of Protagoras which impairs this pleasing impression of the grave and weighty old man. His real defect is that he is inferior to Socrates in dialectics. The opposition between him and Socrates is not the opposition of good and bad, true and false, but of the old art of rhetoric and the new science of interrogation and argument; also of the irony of Socrates and the self-assertion of the Sophists. There is quite as much truth on the side of Protagoras as of Socrates; but the truth of Protagoras is based on common sense and common maxims of morality, while that of Socrates is paradoxical or transcendental, and though full of meaning and insight, hardly intelligible to the rest of mankind. Here as elsewhere is the usual contrast between the Sophists representing average public opinion and Socrates seeking for increased clearness and unity of ideas. But to a great extent Protagoras has the best of the argument and represents the better mind of man.

\par  For example: (1) one of the noblest statements to be found in antiquity about the preventive nature of punishment is put into his mouth; (2) he is clearly right also in maintaining that virtue can be taught (which Socrates himself, at the end of the Dialogue, is disposed to concede); and also (3) in his explanation of the phenomenon that good fathers have bad sons; (4) he is right also in observing that the virtues are not like the arts, gifts or attainments of special individuals, but the common property of all: this, which in all ages has been the strength and weakness of ethics and politics, is deeply seated in human nature; (5) there is a sort of half-truth in the notion that all civilized men are teachers of virtue; and more than a half-truth (6) in ascribing to man, who in his outward conditions is more helpless than the other animals, the power of self-improvement; (7) the religious allegory should be noticed, in which the arts are said to be given by Prometheus (who stole them), whereas justice and reverence and the political virtues could only be imparted by Zeus; (8) in the latter part of the Dialogue, when Socrates is arguing that 'pleasure is the only good,' Protagoras deems it more in accordance with his character to maintain that 'some pleasures only are good;' and admits that 'he, above all other men, is bound to say "that wisdom and knowledge are the highest of human things."'

\par  There is no reason to suppose that in all this Plato is depicting an imaginary Protagoras; he seems to be showing us the teaching of the Sophists under the milder aspect under which he once regarded them. Nor is there any reason to doubt that Socrates is equally an historical character, paradoxical, ironical, tiresome, but seeking for the unity of virtue and knowledge as for a precious treasure; willing to rest this even on a calculation of pleasure, and irresistible here, as everywhere in Plato, in his intellectual superiority.

\par  The aim of Socrates, and of the Dialogue, is to show the unity of virtue. In the determination of this question the identity of virtue and knowledge is found to be involved. But if virtue and knowledge are one, then virtue can be taught; the end of the Dialogue returns to the beginning. Had Protagoras been allowed by Plato to make the Aristotelian distinction, and say that virtue is not knowledge, but is accompanied with knowledge; or to point out with Aristotle that the same quality may have more than one opposite; or with Plato himself in the Phaedo to deny that good is a mere exchange of a greater pleasure for a less—the unity of virtue and the identity of virtue and knowledge would have required to be proved by other arguments.

\par  The victory of Socrates over Protagoras is in every way complete when their minds are fairly brought together. Protagoras falls before him after two or three blows. Socrates partially gains his object in the first part of the Dialogue, and completely in the second. Nor does he appear at any disadvantage when subjected to 'the question' by Protagoras. He succeeds in making his two 'friends,' Prodicus and Hippias, ludicrous by the way; he also makes a long speech in defence of the poem of Simonides, after the manner of the Sophists, showing, as Alcibiades says, that he is only pretending to have a bad memory, and that he and not Protagoras is really a master in the two styles of speaking; and that he can undertake, not one side of the argument only, but both, when Protagoras begins to break down. Against the authority of the poets with whom Protagoras has ingeniously identified himself at the commencement of the Dialogue, Socrates sets up the proverbial philosophers and those masters of brevity the Lacedaemonians. The poets, the Laconizers, and Protagoras are satirized at the same time.

\par  Not having the whole of this poem before us, it is impossible for us to answer certainly the question of Protagoras, how the two passages of Simonides are to be reconciled. We can only follow the indications given by Plato himself. But it seems likely that the reconcilement offered by Socrates is a caricature of the methods of interpretation which were practised by the Sophists—for the following reasons: (1) The transparent irony of the previous interpretations given by Socrates. (2) The ludicrous opening of the speech in which the Lacedaemonians are described as the true philosophers, and Laconic brevity as the true form of philosophy, evidently with an allusion to Protagoras' long speeches. (3) The manifest futility and absurdity of the explanation of (Greek), which is hardly consistent with the rational interpretation of the rest of the poem. The opposition of (Greek) and (Greek) seems also intended to express the rival doctrines of Socrates and Protagoras, and is a facetious commentary on their differences. (4) The general treatment in Plato both of the Poets and the Sophists, who are their interpreters, and whom he delights to identify with them. (5) The depreciating spirit in which Socrates speaks of the introduction of the poets as a substitute for original conversation, which is intended to contrast with Protagoras' exaltation of the study of them—this again is hardly consistent with the serious defence of Simonides. (6) the marked approval of Hippias, who is supposed at once to catch the familiar sound, just as in the previous conversation Prodicus is represented as ready to accept any distinctions of language however absurd. At the same time Hippias is desirous of substituting a new interpretation of his own; as if the words might really be made to mean anything, and were only to be regarded as affording a field for the ingenuity of the interpreter.

\par  This curious passage is, therefore, to be regarded as Plato's satire on the tedious and hypercritical arts of interpretation which prevailed in his own day, and may be compared with his condemnation of the same arts when applied to mythology in the Phaedrus, and with his other parodies, e.g. with the two first speeches in the Phaedrus and with the Menexenus. Several lesser touches of satire may be observed, such as the claim of philosophy advanced for the Lacedaemonians, which is a parody of the claims advanced for the Poets by Protagoras; the mistake of the Laconizing set in supposing that the Lacedaemonians are a great nation because they bruise their ears; the far-fetched notion, which is 'really too bad,' that Simonides uses the Lesbian (?) word, (Greek), because he is addressing a Lesbian. The whole may also be considered as a satire on those who spin pompous theories out of nothing. As in the arguments of the Euthydemus and of the Cratylus, the veil of irony is never withdrawn; and we are left in doubt at last how far in this interpretation of Simonides Socrates is 'fooling,' how far he is in earnest.

\par  All the interests and contrasts of character in a great dramatic work like the Protagoras are not easily exhausted. The impressiveness of the scene should not be lost upon us, or the gradual substitution of Socrates in the second part for Protagoras in the first. The characters to whom we are introduced at the beginning of the Dialogue all play a part more or less conspicuous towards the end. There is Alcibiades, who is compelled by the necessity of his nature to be a partisan, lending effectual aid to Socrates; there is Critias assuming the tone of impartiality; Callias, here as always inclining to the Sophists, but eager for any intellectual repast; Prodicus, who finds an opportunity for displaying his distinctions of language, which are valueless and pedantic, because they are not based on dialectic; Hippias, who has previously exhibited his superficial knowledge of natural philosophy, to which, as in both the Dialogues called by his name, he now adds the profession of an interpreter of the Poets. The two latter personages have been already damaged by the mock heroic description of them in the introduction. It may be remarked that Protagoras is consistently presented to us throughout as the teacher of moral and political virtue; there is no allusion to the theories of sensation which are attributed to him in the Theaetetus and elsewhere, or to his denial of the existence of the gods in a well-known fragment ascribed to him; he is the religious rather than the irreligious teacher in this Dialogue. Also it may be observed that Socrates shows him as much respect as is consistent with his own ironical character; he admits that the dialectic which has overthrown Protagoras has carried himself round to a conclusion opposed to his first thesis. The force of argument, therefore, and not Socrates or Protagoras, has won the day.

\par  But is Socrates serious in maintaining (1) that virtue cannot be taught; (2) that the virtues are one; (3) that virtue is the knowledge of pleasures and pains present and future? These propositions to us have an appearance of paradox—they are really moments or aspects of the truth by the help of which we pass from the old conventional morality to a higher conception of virtue and knowledge. That virtue cannot be taught is a paradox of the same sort as the profession of Socrates that he knew nothing. Plato means to say that virtue is not brought to a man, but must be drawn out of him; and cannot be taught by rhetorical discourses or citations from the poets. The second question, whether the virtues are one or many, though at first sight distinct, is really a part of the same subject; for if the virtues are to be taught, they must be reducible to a common principle; and this common principle is found to be knowledge. Here, as Aristotle remarks, Socrates and Plato outstep the truth—they make a part of virtue into the whole. Further, the nature of this knowledge, which is assumed to be a knowledge of pleasures and pains, appears to us too superficial and at variance with the spirit of Plato himself. Yet, in this, Plato is only following the historical Socrates as he is depicted to us in Xenophon's Memorabilia. Like Socrates, he finds on the surface of human life one common bond by which the virtues are united,—their tendency to produce happiness,—though such a principle is afterwards repudiated by him.

\par  It remains to be considered in what relation the Protagoras stands to the other Dialogues of Plato. That it is one of the earlier or purely Socratic works—perhaps the last, as it is certainly the greatest of them—is indicated by the absence of any allusion to the doctrine of reminiscence; and also by the different attitude assumed towards the teaching and persons of the Sophists in some of the later Dialogues. The Charmides, Laches, Lysis, all touch on the question of the relation of knowledge to virtue, and may be regarded, if not as preliminary studies or sketches of the more important work, at any rate as closely connected with it. The Io and the lesser Hippias contain discussions of the Poets, which offer a parallel to the ironical criticism of Simonides, and are conceived in a similar spirit. The affinity of the Protagoras to the Meno is more doubtful. For there, although the same question is discussed, 'whether virtue can be taught,' and the relation of Meno to the Sophists is much the same as that of Hippocrates, the answer to the question is supplied out of the doctrine of ideas; the real Socrates is already passing into the Platonic one. At a later stage of the Platonic philosophy we shall find that both the paradox and the solution of it appear to have been retracted. The Phaedo, the Gorgias, and the Philebus offer further corrections of the teaching of the Protagoras; in all of them the doctrine that virtue is pleasure, or that pleasure is the chief or only good, is distinctly renounced.

\par  Thus after many preparations and oppositions, both of the characters of men and aspects of the truth, especially of the popular and philosophical aspect; and after many interruptions and detentions by the way, which, as Theodorus says in the Theaetetus, are quite as agreeable as the argument, we arrive at the great Socratic thesis that virtue is knowledge. This is an aspect of the truth which was lost almost as soon as it was found; and yet has to be recovered by every one for himself who would pass the limits of proverbial and popular philosophy. The moral and intellectual are always dividing, yet they must be reunited, and in the highest conception of them are inseparable. The thesis of Socrates is not merely a hasty assumption, but may be also deemed an anticipation of some 'metaphysic of the future,' in which the divided elements of human nature are reconciled.

\par 
\section{
      PROTAGORAS
    }  
\par \textbf{COMPANION}
\par   Where do you come from, Socrates? And yet I need hardly ask the question, for I know that you have been in chase of the fair Alcibiades. I saw him the day before yesterday; and he had got a beard like a man,—and he is a man, as I may tell you in your ear. But I thought that he was still very charming.

\par \textbf{SOCRATES}
\par   What of his beard? Are you not of Homer's opinion, who says
 
\par  And that is now the charm of Alcibiades.

\par \textbf{COMPANION}
\par   Well, and how do matters proceed? Have you been visiting him, and was he gracious to you?

\par \textbf{SOCRATES}
\par   Yes, I thought that he was very gracious; and especially to-day, for I have just come from him, and he has been helping me in an argument. But shall I tell you a strange thing? I paid no attention to him, and several times I quite forgot that he was present.

\par \textbf{COMPANION}
\par   What is the meaning of this? Has anything happened between you and him? For surely you cannot have discovered a fairer love than he is; certainly not in this city of Athens.

\par \textbf{SOCRATES}
\par   Yes, much fairer.

\par \textbf{COMPANION}
\par   What do you mean—a citizen or a foreigner?

\par \textbf{SOCRATES}
\par   A foreigner.

\par \textbf{COMPANION}
\par   Of what country?

\par \textbf{SOCRATES}
\par   Of Abdera.

\par \textbf{COMPANION}
\par   And is this stranger really in your opinion a fairer love than the son of Cleinias?

\par \textbf{SOCRATES}
\par   And is not the wiser always the fairer, sweet friend?

\par \textbf{COMPANION}
\par   But have you really met, Socrates, with some wise one?

\par \textbf{SOCRATES}
\par   Say rather, with the wisest of all living men, if you are willing to accord that title to Protagoras.

\par \textbf{COMPANION}
\par   What! Is Protagoras in Athens?

\par \textbf{SOCRATES}
\par   Yes; he has been here two days.

\par \textbf{COMPANION}
\par   And do you just come from an interview with him?

\par \textbf{SOCRATES}
\par   Yes; and I have heard and said many things.

\par \textbf{COMPANION}
\par   Then, if you have no engagement, suppose that you sit down and tell me what passed, and my attendant here shall give up his place to you.

\par \textbf{SOCRATES}
\par   To be sure; and I shall be grateful to you for listening.

\par \textbf{COMPANION}
\par   Thank you, too, for telling us.

\par \textbf{SOCRATES}
\par   That is thank you twice over. Listen then: —

\par  Last night, or rather very early this morning, Hippocrates, the son of Apollodorus and the brother of Phason, gave a tremendous thump with his staff at my door; some one opened to him, and he came rushing in and bawled out: Socrates, are you awake or asleep?

\par  I knew his voice, and said: Hippocrates, is that you? and do you bring any news?

\par  Good news, he said; nothing but good.

\par  Delightful, I said; but what is the news? and why have you come hither at this unearthly hour?

\par  He drew nearer to me and said: Protagoras is come.

\par  Yes, I replied; he came two days ago: have you only just heard of his arrival?

\par  Yes, by the gods, he said; but not until yesterday evening.

\par  At the same time he felt for the truckle-bed, and sat down at my feet, and then he said: Yesterday quite late in the evening, on my return from Oenoe whither I had gone in pursuit of my runaway slave Satyrus, as I meant to have told you, if some other matter had not come in the way;—on my return, when we had done supper and were about to retire to rest, my brother said to me: Protagoras is come. I was going to you at once, and then I thought that the night was far spent. But the moment sleep left me after my fatigue, I got up and came hither direct.

\par  I, who knew the very courageous madness of the man, said: What is the matter? Has Protagoras robbed you of anything?

\par  He replied, laughing: Yes, indeed he has, Socrates, of the wisdom which he keeps from me.

\par  But, surely, I said, if you give him money, and make friends with him, he will make you as wise as he is himself.

\par  Would to heaven, he replied, that this were the case! He might take all that I have, and all that my friends have, if he pleased. But that is why I have come to you now, in order that you may speak to him on my behalf; for I am young, and also I have never seen nor heard him; (when he visited Athens before I was but a child;) and all men praise him, Socrates; he is reputed to be the most accomplished of speakers. There is no reason why we should not go to him at once, and then we shall find him at home. He lodges, as I hear, with Callias the son of Hipponicus: let us start.

\par  I replied: Not yet, my good friend; the hour is too early. But let us rise and take a turn in the court and wait about there until day-break; when the day breaks, then we will go. For Protagoras is generally at home, and we shall be sure to find him; never fear.

\par  Upon this we got up and walked about in the court, and I thought that I would make trial of the strength of his resolution. So I examined him and put questions to him. Tell me, Hippocrates, I said, as you are going to Protagoras, and will be paying your money to him, what is he to whom you are going? and what will he make of you? If, for example, you had thought of going to Hippocrates of Cos, the Asclepiad, and were about to give him your money, and some one had said to you: You are paying money to your namesake Hippocrates, O Hippocrates; tell me, what is he that you give him money? how would you have answered?

\par  I should say, he replied, that I gave money to him as a physician.

\par  And what will he make of you?

\par  A physician, he said.

\par  And if you were resolved to go to Polycleitus the Argive, or Pheidias the Athenian, and were intending to give them money, and some one had asked you: What are Polycleitus and Pheidias? and why do you give them this money?—how would you have answered?

\par  I should have answered, that they were statuaries.

\par  And what will they make of you?

\par  A statuary, of course.

\par  Well now, I said, you and I are going to Protagoras, and we are ready to pay him money on your behalf. If our own means are sufficient, and we can gain him with these, we shall be only too glad; but if not, then we are to spend the money of your friends as well. Now suppose, that while we are thus enthusiastically pursuing our object some one were to say to us: Tell me, Socrates, and you Hippocrates, what is Protagoras, and why are you going to pay him money,—how should we answer? I know that Pheidias is a sculptor, and that Homer is a poet; but what appellation is given to Protagoras? how is he designated?

\par  They call him a Sophist, Socrates, he replied.

\par  Then we are going to pay our money to him in the character of a Sophist?

\par  Certainly.

\par  But suppose a person were to ask this further question: And how about yourself? What will Protagoras make of you, if you go to see him?

\par  He answered, with a blush upon his face (for the day was just beginning to dawn, so that I could see him): Unless this differs in some way from the former instances, I suppose that he will make a Sophist of me.

\par  By the gods, I said, and are you not ashamed at having to appear before the Hellenes in the character of a Sophist?

\par  Indeed, Socrates, to confess the truth, I am.

\par  But you should not assume, Hippocrates, that the instruction of Protagoras is of this nature: may you not learn of him in the same way that you learned the arts of the grammarian, or musician, or trainer, not with the view of making any of them a profession, but only as a part of education, and because a private gentleman and freeman ought to know them?

\par  Just so, he said; and that, in my opinion, is a far truer account of the teaching of Protagoras.

\par  I said: I wonder whether you know what you are doing?

\par  And what am I doing?

\par  You are going to commit your soul to the care of a man whom you call a Sophist. And yet I hardly think that you know what a Sophist is; and if not, then you do not even know to whom you are committing your soul and whether the thing to which you commit yourself be good or evil.

\par  I certainly think that I do know, he replied.

\par  Then tell me, what do you imagine that he is?

\par  I take him to be one who knows wise things, he replied, as his name implies.

\par  And might you not, I said, affirm this of the painter and of the carpenter also: Do not they, too, know wise things? But suppose a person were to ask us: In what are the painters wise? We should answer: In what relates to the making of likenesses, and similarly of other things. And if he were further to ask: What is the wisdom of the Sophist, and what is the manufacture over which he presides?—how should we answer him?

\par  How should we answer him, Socrates? What other answer could there be but that he presides over the art which makes men eloquent?

\par  Yes, I replied, that is very likely true, but not enough; for in the answer a further question is involved: Of what does the Sophist make a man talk eloquently? The player on the lyre may be supposed to make a man talk eloquently about that which he makes him understand, that is about playing the lyre. Is not that true?

\par  Yes.

\par  Then about what does the Sophist make him eloquent? Must not he make him eloquent in that which he understands?

\par  Yes, that may be assumed.

\par  And what is that which the Sophist knows and makes his disciple know?

\par  Indeed, he said, I cannot tell.

\par  Then I proceeded to say: Well, but are you aware of the danger which you are incurring? If you were going to commit your body to some one, who might do good or harm to it, would you not carefully consider and ask the opinion of your friends and kindred, and deliberate many days as to whether you should give him the care of your body? But when the soul is in question, which you hold to be of far more value than the body, and upon the good or evil of which depends the well-being of your all,—about this you never consulted either with your father or with your brother or with any one of us who are your companions. But no sooner does this foreigner appear, than you instantly commit your soul to his keeping. In the evening, as you say, you hear of him, and in the morning you go to him, never deliberating or taking the opinion of any one as to whether you ought to intrust yourself to him or not;—you have quite made up your mind that you will at all hazards be a pupil of Protagoras, and are prepared to expend all the property of yourself and of your friends in carrying out at any price this determination, although, as you admit, you do not know him, and have never spoken with him: and you call him a Sophist, but are manifestly ignorant of what a Sophist is; and yet you are going to commit yourself to his keeping.

\par  When he heard me say this, he replied: No other inference, Socrates, can be drawn from your words.

\par  I proceeded: Is not a Sophist, Hippocrates, one who deals wholesale or retail in the food of the soul? To me that appears to be his nature.

\par  And what, Socrates, is the food of the soul?

\par  Surely, I said, knowledge is the food of the soul; and we must take care, my friend, that the Sophist does not deceive us when he praises what he sells, like the dealers wholesale or retail who sell the food of the body; for they praise indiscriminately all their goods, without knowing what are really beneficial or hurtful: neither do their customers know, with the exception of any trainer or physician who may happen to buy of them. In like manner those who carry about the wares of knowledge, and make the round of the cities, and sell or retail them to any customer who is in want of them, praise them all alike; though I should not wonder, O my friend, if many of them were really ignorant of their effect upon the soul; and their customers equally ignorant, unless he who buys of them happens to be a physician of the soul. If, therefore, you have understanding of what is good and evil, you may safely buy knowledge of Protagoras or of any one; but if not, then, O my friend, pause, and do not hazard your dearest interests at a game of chance. For there is far greater peril in buying knowledge than in buying meat and drink: the one you purchase of the wholesale or retail dealer, and carry them away in other vessels, and before you receive them into the body as food, you may deposit them at home and call in any experienced friend who knows what is good to be eaten or drunken, and what not, and how much, and when; and then the danger of purchasing them is not so great. But you cannot buy the wares of knowledge and carry them away in another vessel; when you have paid for them you must receive them into the soul and go your way, either greatly harmed or greatly benefited; and therefore we should deliberate and take counsel with our elders; for we are still young—too young to determine such a matter. And now let us go, as we were intending, and hear Protagoras; and when we have heard what he has to say, we may take counsel of others; for not only is Protagoras at the house of Callias, but there is Hippias of Elis, and, if I am not mistaken, Prodicus of Ceos, and several other wise men.

\par  To this we agreed, and proceeded on our way until we reached the vestibule of the house; and there we stopped in order to conclude a discussion which had arisen between us as we were going along; and we stood talking in the vestibule until we had finished and come to an understanding. And I think that the door-keeper, who was a eunuch, and who was probably annoyed at the great inroad of the Sophists, must have heard us talking. At any rate, when we knocked at the door, and he opened and saw us, he grumbled: They are Sophists—he is not at home; and instantly gave the door a hearty bang with both his hands. Again we knocked, and he answered without opening: Did you not hear me say that he is not at home, fellows? But, my friend, I said, you need not be alarmed; for we are not Sophists, and we are not come to see Callias, but we want to see Protagoras; and I must request you to announce us. At last, after a good deal of difficulty, the man was persuaded to open the door.

\par  When we entered, we found Protagoras taking a walk in the cloister; and next to him, on one side, were walking Callias, the son of Hipponicus, and Paralus, the son of Pericles, who, by the mother's side, is his half-brother, and Charmides, the son of Glaucon. On the other side of him were Xanthippus, the other son of Pericles, Philippides, the son of Philomelus; also Antimoerus of Mende, who of all the disciples of Protagoras is the most famous, and intends to make sophistry his profession. A train of listeners followed him; the greater part of them appeared to be foreigners, whom Protagoras had brought with him out of the various cities visited by him in his journeys, he, like Orpheus, attracting them his voice, and they following (Compare Rep.). I should mention also that there were some Athenians in the company. Nothing delighted me more than the precision of their movements: they never got into his way at all; but when he and those who were with him turned back, then the band of listeners parted regularly on either side; he was always in front, and they wheeled round and took their places behind him in perfect order.

\par  After him, as Homer says (Od. ), 'I lifted up my eyes and saw' Hippias the Elean sitting in the opposite cloister on a chair of state, and around him were seated on benches Eryximachus, the son of Acumenus, and Phaedrus the Myrrhinusian, and Andron the son of Androtion, and there were strangers whom he had brought with him from his native city of Elis, and some others: they were putting to Hippias certain physical and astronomical questions, and he, ex cathedra, was determining their several questions to them, and discoursing of them.

\par  Also, 'my eyes beheld Tantalus (Od. );' for Prodicus the Cean was at Athens: he had been lodged in a room which, in the days of Hipponicus, was a storehouse; but, as the house was full, Callias had cleared this out and made the room into a guest-chamber. Now Prodicus was still in bed, wrapped up in sheepskins and bedclothes, of which there seemed to be a great heap; and there was sitting by him on the couches near, Pausanias of the deme of Cerameis, and with Pausanias was a youth quite young, who is certainly remarkable for his good looks, and, if I am not mistaken, is also of a fair and gentle nature. I thought that I heard him called Agathon, and my suspicion is that he is the beloved of Pausanias. There was this youth, and also there were the two Adeimantuses, one the son of Cepis, and the other of Leucolophides, and some others. I was very anxious to hear what Prodicus was saying, for he seems to me to be an all-wise and inspired man; but I was not able to get into the inner circle, and his fine deep voice made an echo in the room which rendered his words inaudible.

\par  No sooner had we entered than there followed us Alcibiades the beautiful, as you say, and I believe you; and also Critias the son of Callaeschrus.

\par  On entering we stopped a little, in order to look about us, and then walked up to Protagoras, and I said: Protagoras, my friend Hippocrates and I have come to see you.

\par  Do you wish, he said, to speak with me alone, or in the presence of the company?

\par  Whichever you please, I said; you shall determine when you have heard the purpose of our visit.

\par  And what is your purpose? he said.

\par  I must explain, I said, that my friend Hippocrates is a native Athenian; he is the son of Apollodorus, and of a great and prosperous house, and he is himself in natural ability quite a match for anybody of his own age. I believe that he aspires to political eminence; and this he thinks that conversation with you is most likely to procure for him. And now you can determine whether you would wish to speak to him of your teaching alone or in the presence of the company.

\par  Thank you, Socrates, for your consideration of me. For certainly a stranger finding his way into great cities, and persuading the flower of the youth in them to leave company of their kinsmen or any other acquaintances, old or young, and live with him, under the idea that they will be improved by his conversation, ought to be very cautious; great jealousies are aroused by his proceedings, and he is the subject of many enmities and conspiracies. Now the art of the Sophist is, as I believe, of great antiquity; but in ancient times those who practised it, fearing this odium, veiled and disguised themselves under various names, some under that of poets, as Homer, Hesiod, and Simonides, some, of hierophants and prophets, as Orpheus and Musaeus, and some, as I observe, even under the name of gymnastic-masters, like Iccus of Tarentum, or the more recently celebrated Herodicus, now of Selymbria and formerly of Megara, who is a first-rate Sophist. Your own Agathocles pretended to be a musician, but was really an eminent Sophist; also Pythocleides the Cean; and there were many others; and all of them, as I was saying, adopted these arts as veils or disguises because they were afraid of the odium which they would incur. But that is not my way, for I do not believe that they effected their purpose, which was to deceive the government, who were not blinded by them; and as to the people, they have no understanding, and only repeat what their rulers are pleased to tell them. Now to run away, and to be caught in running away, is the very height of folly, and also greatly increases the exasperation of mankind; for they regard him who runs away as a rogue, in addition to any other objections which they have to him; and therefore I take an entirely opposite course, and acknowledge myself to be a Sophist and instructor of mankind; such an open acknowledgement appears to me to be a better sort of caution than concealment. Nor do I neglect other precautions, and therefore I hope, as I may say, by the favour of heaven that no harm will come of the acknowledgment that I am a Sophist. And I have been now many years in the profession—for all my years when added up are many: there is no one here present of whom I might not be the father. Wherefore I should much prefer conversing with you, if you want to speak with me, in the presence of the company.

\par  As I suspected that he would like to have a little display and glorification in the presence of Prodicus and Hippias, and would gladly show us to them in the light of his admirers, I said: But why should we not summon Prodicus and Hippias and their friends to hear us?

\par  Very good, he said.

\par  Suppose, said Callias, that we hold a council in which you may sit and discuss.—This was agreed upon, and great delight was felt at the prospect of hearing wise men talk; we ourselves took the chairs and benches, and arranged them by Hippias, where the other benches had been already placed. Meanwhile Callias and Alcibiades got Prodicus out of bed and brought in him and his companions.

\par  When we were all seated, Protagoras said: Now that the company are assembled, Socrates, tell me about the young man of whom you were just now speaking.

\par  I replied: I will begin again at the same point, Protagoras, and tell you once more the purport of my visit: this is my friend Hippocrates, who is desirous of making your acquaintance; he would like to know what will happen to him if he associates with you. I have no more to say.

\par  Protagoras answered: Young man, if you associate with me, on the very first day you will return home a better man than you came, and better on the second day than on the first, and better every day than you were on the day before.

\par  When I heard this, I said: Protagoras, I do not at all wonder at hearing you say this; even at your age, and with all your wisdom, if any one were to teach you what you did not know before, you would become better no doubt: but please to answer in a different way—I will explain how by an example. Let me suppose that Hippocrates, instead of desiring your acquaintance, wished to become acquainted with the young man Zeuxippus of Heraclea, who has lately been in Athens, and he had come to him as he has come to you, and had heard him say, as he has heard you say, that every day he would grow and become better if he associated with him: and then suppose that he were to ask him, 'In what shall I become better, and in what shall I grow? '—Zeuxippus would answer, 'In painting.' And suppose that he went to Orthagoras the Theban, and heard him say the same thing, and asked him, 'In what shall I become better day by day?' he would reply, 'In flute-playing.' Now I want you to make the same sort of answer to this young man and to me, who am asking questions on his account. When you say that on the first day on which he associates with you he will return home a better man, and on every day will grow in like manner,—in what, Protagoras, will he be better? and about what?

\par  When Protagoras heard me say this, he replied: You ask questions fairly, and I like to answer a question which is fairly put. If Hippocrates comes to me he will not experience the sort of drudgery with which other Sophists are in the habit of insulting their pupils; who, when they have just escaped from the arts, are taken and driven back into them by these teachers, and made to learn calculation, and astronomy, and geometry, and music (he gave a look at Hippias as he said this); but if he comes to me, he will learn that which he comes to learn. And this is prudence in affairs private as well as public; he will learn to order his own house in the best manner, and he will be able to speak and act for the best in the affairs of the state.

\par  Do I understand you, I said; and is your meaning that you teach the art of politics, and that you promise to make men good citizens?

\par  That, Socrates, is exactly the profession which I make.

\par  Then, I said, you do indeed possess a noble art, if there is no mistake about this; for I will freely confess to you, Protagoras, that I have a doubt whether this art is capable of being taught, and yet I know not how to disbelieve your assertion. And I ought to tell you why I am of opinion that this art cannot be taught or communicated by man to man. I say that the Athenians are an understanding people, and indeed they are esteemed to be such by the other Hellenes. Now I observe that when we are met together in the assembly, and the matter in hand relates to building, the builders are summoned as advisers; when the question is one of ship-building, then the ship-wrights; and the like of other arts which they think capable of being taught and learned. And if some person offers to give them advice who is not supposed by them to have any skill in the art, even though he be good-looking, and rich, and noble, they will not listen to him, but laugh and hoot at him, until either he is clamoured down and retires of himself; or if he persist, he is dragged away or put out by the constables at the command of the prytanes. This is their way of behaving about professors of the arts. But when the question is an affair of state, then everybody is free to have a say—carpenter, tinker, cobbler, sailor, passenger; rich and poor, high and low—any one who likes gets up, and no one reproaches him, as in the former case, with not having learned, and having no teacher, and yet giving advice; evidently because they are under the impression that this sort of knowledge cannot be taught. And not only is this true of the state, but of individuals; the best and wisest of our citizens are unable to impart their political wisdom to others: as for example, Pericles, the father of these young men, who gave them excellent instruction in all that could be learned from masters, in his own department of politics neither taught them, nor gave them teachers; but they were allowed to wander at their own free will in a sort of hope that they would light upon virtue of their own accord. Or take another example: there was Cleinias the younger brother of our friend Alcibiades, of whom this very same Pericles was the guardian; and he being in fact under the apprehension that Cleinias would be corrupted by Alcibiades, took him away, and placed him in the house of Ariphron to be educated; but before six months had elapsed, Ariphron sent him back, not knowing what to do with him. And I could mention numberless other instances of persons who were good themselves, and never yet made any one else good, whether friend or stranger. Now I, Protagoras, having these examples before me, am inclined to think that virtue cannot be taught. But then again, when I listen to your words, I waver; and am disposed to think that there must be something in what you say, because I know that you have great experience, and learning, and invention. And I wish that you would, if possible, show me a little more clearly that virtue can be taught. Will you be so good?

\par  That I will, Socrates, and gladly. But what would you like? Shall I, as an elder, speak to you as younger men in an apologue or myth, or shall I argue out the question?

\par  To this several of the company answered that he should choose for himself.

\par  Well, then, he said, I think that the myth will be more interesting.

\par  Once upon a time there were gods only, and no mortal creatures. But when the time came that these also should be created, the gods fashioned them out of earth and fire and various mixtures of both elements in the interior of the earth; and when they were about to bring them into the light of day, they ordered Prometheus and Epimetheus to equip them, and to distribute to them severally their proper qualities. Epimetheus said to Prometheus: 'Let me distribute, and do you inspect.' This was agreed, and Epimetheus made the distribution. There were some to whom he gave strength without swiftness, while he equipped the weaker with swiftness; some he armed, and others he left unarmed; and devised for the latter some other means of preservation, making some large, and having their size as a protection, and others small, whose nature was to fly in the air or burrow in the ground; this was to be their way of escape. Thus did he compensate them with the view of preventing any race from becoming extinct. And when he had provided against their destruction by one another, he contrived also a means of protecting them against the seasons of heaven; clothing them with close hair and thick skins sufficient to defend them against the winter cold and able to resist the summer heat, so that they might have a natural bed of their own when they wanted to rest; also he furnished them with hoofs and hair and hard and callous skins under their feet. Then he gave them varieties of food,—herb of the soil to some, to others fruits of trees, and to others roots, and to some again he gave other animals as food. And some he made to have few young ones, while those who were their prey were very prolific; and in this manner the race was preserved. Thus did Epimetheus, who, not being very wise, forgot that he had distributed among the brute animals all the qualities which he had to give,—and when he came to man, who was still unprovided, he was terribly perplexed. Now while he was in this perplexity, Prometheus came to inspect the distribution, and he found that the other animals were suitably furnished, but that man alone was naked and shoeless, and had neither bed nor arms of defence. The appointed hour was approaching when man in his turn was to go forth into the light of day; and Prometheus, not knowing how he could devise his salvation, stole the mechanical arts of Hephaestus and Athene, and fire with them (they could neither have been acquired nor used without fire), and gave them to man. Thus man had the wisdom necessary to the support of life, but political wisdom he had not; for that was in the keeping of Zeus, and the power of Prometheus did not extend to entering into the citadel of heaven, where Zeus dwelt, who moreover had terrible sentinels; but he did enter by stealth into the common workshop of Athene and Hephaestus, in which they used to practise their favourite arts, and carried off Hephaestus' art of working by fire, and also the art of Athene, and gave them to man. And in this way man was supplied with the means of life. But Prometheus is said to have been afterwards prosecuted for theft, owing to the blunder of Epimetheus.

\par  Now man, having a share of the divine attributes, was at first the only one of the animals who had any gods, because he alone was of their kindred; and he would raise altars and images of them. He was not long in inventing articulate speech and names; and he also constructed houses and clothes and shoes and beds, and drew sustenance from the earth. Thus provided, mankind at first lived dispersed, and there were no cities. But the consequence was that they were destroyed by the wild beasts, for they were utterly weak in comparison of them, and their art was only sufficient to provide them with the means of life, and did not enable them to carry on war against the animals: food they had, but not as yet the art of government, of which the art of war is a part. After a while the desire of self-preservation gathered them into cities; but when they were gathered together, having no art of government, they evil intreated one another, and were again in process of dispersion and destruction. Zeus feared that the entire race would be exterminated, and so he sent Hermes to them, bearing reverence and justice to be the ordering principles of cities and the bonds of friendship and conciliation. Hermes asked Zeus how he should impart justice and reverence among men:—Should he distribute them as the arts are distributed; that is to say, to a favoured few only, one skilled individual having enough of medicine or of any other art for many unskilled ones? 'Shall this be the manner in which I am to distribute justice and reverence among men, or shall I give them to all?' 'To all,' said Zeus; 'I should like them all to have a share; for cities cannot exist, if a few only share in the virtues, as in the arts. And further, make a law by my order, that he who has no part in reverence and justice shall be put to death, for he is a plague of the state.'

\par  And this is the reason, Socrates, why the Athenians and mankind in general, when the question relates to carpentering or any other mechanical art, allow but a few to share in their deliberations; and when any one else interferes, then, as you say, they object, if he be not of the favoured few; which, as I reply, is very natural. But when they meet to deliberate about political virtue, which proceeds only by way of justice and wisdom, they are patient enough of any man who speaks of them, as is also natural, because they think that every man ought to share in this sort of virtue, and that states could not exist if this were otherwise. I have explained to you, Socrates, the reason of this phenomenon.

\par  And that you may not suppose yourself to be deceived in thinking that all men regard every man as having a share of justice or honesty and of every other political virtue, let me give you a further proof, which is this. In other cases, as you are aware, if a man says that he is a good flute-player, or skilful in any other art in which he has no skill, people either laugh at him or are angry with him, and his relations think that he is mad and go and admonish him; but when honesty is in question, or some other political virtue, even if they know that he is dishonest, yet, if the man comes publicly forward and tells the truth about his dishonesty, then, what in the other case was held by them to be good sense, they now deem to be madness. They say that all men ought to profess honesty whether they are honest or not, and that a man is out of his mind who says anything else. Their notion is, that a man must have some degree of honesty; and that if he has none at all he ought not to be in the world.

\par  I have been showing that they are right in admitting every man as a counsellor about this sort of virtue, as they are of opinion that every man is a partaker of it. And I will now endeavour to show further that they do not conceive this virtue to be given by nature, or to grow spontaneously, but to be a thing which may be taught; and which comes to a man by taking pains. No one would instruct, no one would rebuke, or be angry with those whose calamities they suppose to be due to nature or chance; they do not try to punish or to prevent them from being what they are; they do but pity them. Who is so foolish as to chastise or instruct the ugly, or the diminutive, or the feeble? And for this reason. Because he knows that good and evil of this kind is the work of nature and of chance; whereas if a man is wanting in those good qualities which are attained by study and exercise and teaching, and has only the contrary evil qualities, other men are angry with him, and punish and reprove him—of these evil qualities one is impiety, another injustice, and they may be described generally as the very opposite of political virtue. In such cases any man will be angry with another, and reprimand him,—clearly because he thinks that by study and learning, the virtue in which the other is deficient may be acquired. If you will think, Socrates, of the nature of punishment, you will see at once that in the opinion of mankind virtue may be acquired; no one punishes the evil-doer under the notion, or for the reason, that he has done wrong,—only the unreasonable fury of a beast acts in that manner. But he who desires to inflict rational punishment does not retaliate for a past wrong which cannot be undone; he has regard to the future, and is desirous that the man who is punished, and he who sees him punished, may be deterred from doing wrong again. He punishes for the sake of prevention, thereby clearly implying that virtue is capable of being taught. This is the notion of all who retaliate upon others either privately or publicly. And the Athenians, too, your own citizens, like other men, punish and take vengeance on all whom they regard as evil doers; and hence, we may infer them to be of the number of those who think that virtue may be acquired and taught. Thus far, Socrates, I have shown you clearly enough, if I am not mistaken, that your countrymen are right in admitting the tinker and the cobbler to advise about politics, and also that they deem virtue to be capable of being taught and acquired.

\par  There yet remains one difficulty which has been raised by you about the sons of good men. What is the reason why good men teach their sons the knowledge which is gained from teachers, and make them wise in that, but do nothing towards improving them in the virtues which distinguish themselves? And here, Socrates, I will leave the apologue and resume the argument. Please to consider: Is there or is there not some one quality of which all the citizens must be partakers, if there is to be a city at all? In the answer to this question is contained the only solution of your difficulty; there is no other. For if there be any such quality, and this quality or unity is not the art of the carpenter, or the smith, or the potter, but justice and temperance and holiness and, in a word, manly virtue—if this is the quality of which all men must be partakers, and which is the very condition of their learning or doing anything else, and if he who is wanting in this, whether he be a child only or a grown-up man or woman, must be taught and punished, until by punishment he becomes better, and he who rebels against instruction and punishment is either exiled or condemned to death under the idea that he is incurable—if what I am saying be true, good men have their sons taught other things and not this, do consider how extraordinary their conduct would appear to be. For we have shown that they think virtue capable of being taught and cultivated both in private and public; and, notwithstanding, they have their sons taught lesser matters, ignorance of which does not involve the punishment of death: but greater things, of which the ignorance may cause death and exile to those who have no training or knowledge of them—aye, and confiscation as well as death, and, in a word, may be the ruin of families—those things, I say, they are supposed not to teach them,—not to take the utmost care that they should learn. How improbable is this, Socrates!

\par  Education and admonition commence in the first years of childhood, and last to the very end of life. Mother and nurse and father and tutor are vying with one another about the improvement of the child as soon as ever he is able to understand what is being said to him: he cannot say or do anything without their setting forth to him that this is just and that is unjust; this is honourable, that is dishonourable; this is holy, that is unholy; do this and abstain from that. And if he obeys, well and good; if not, he is straightened by threats and blows, like a piece of bent or warped wood. At a later stage they send him to teachers, and enjoin them to see to his manners even more than to his reading and music; and the teachers do as they are desired. And when the boy has learned his letters and is beginning to understand what is written, as before he understood only what was spoken, they put into his hands the works of great poets, which he reads sitting on a bench at school; in these are contained many admonitions, and many tales, and praises, and encomia of ancient famous men, which he is required to learn by heart, in order that he may imitate or emulate them and desire to become like them. Then, again, the teachers of the lyre take similar care that their young disciple is temperate and gets into no mischief; and when they have taught him the use of the lyre, they introduce him to the poems of other excellent poets, who are the lyric poets; and these they set to music, and make their harmonies and rhythms quite familiar to the children's souls, in order that they may learn to be more gentle, and harmonious, and rhythmical, and so more fitted for speech and action; for the life of man in every part has need of harmony and rhythm. Then they send them to the master of gymnastic, in order that their bodies may better minister to the virtuous mind, and that they may not be compelled through bodily weakness to play the coward in war or on any other occasion. This is what is done by those who have the means, and those who have the means are the rich; their children begin to go to school soonest and leave off latest. When they have done with masters, the state again compels them to learn the laws, and live after the pattern which they furnish, and not after their own fancies; and just as in learning to write, the writing-master first draws lines with a style for the use of the young beginner, and gives him the tablet and makes him follow the lines, so the city draws the laws, which were the invention of good lawgivers living in the olden time; these are given to the young man, in order to guide him in his conduct whether he is commanding or obeying; and he who transgresses them is to be corrected, or, in other words, called to account, which is a term used not only in your country, but also in many others, seeing that justice calls men to account. Now when there is all this care about virtue private and public, why, Socrates, do you still wonder and doubt whether virtue can be taught? Cease to wonder, for the opposite would be far more surprising.

\par  But why then do the sons of good fathers often turn out ill? There is nothing very wonderful in this; for, as I have been saying, the existence of a state implies that virtue is not any man's private possession. If so—and nothing can be truer—then I will further ask you to imagine, as an illustration, some other pursuit or branch of knowledge which may be assumed equally to be the condition of the existence of a state. Suppose that there could be no state unless we were all flute-players, as far as each had the capacity, and everybody was freely teaching everybody the art, both in private and public, and reproving the bad player as freely and openly as every man now teaches justice and the laws, not concealing them as he would conceal the other arts, but imparting them—for all of us have a mutual interest in the justice and virtue of one another, and this is the reason why every one is so ready to teach justice and the laws;—suppose, I say, that there were the same readiness and liberality among us in teaching one another flute-playing, do you imagine, Socrates, that the sons of good flute-players would be more likely to be good than the sons of bad ones? I think not. Would not their sons grow up to be distinguished or undistinguished according to their own natural capacities as flute-players, and the son of a good player would often turn out to be a bad one, and the son of a bad player to be a good one, all flute-players would be good enough in comparison of those who were ignorant and unacquainted with the art of flute-playing? In like manner I would have you consider that he who appears to you to be the worst of those who have been brought up in laws and humanities, would appear to be a just man and a master of justice if he were to be compared with men who had no education, or courts of justice, or laws, or any restraints upon them which compelled them to practise virtue—with the savages, for example, whom the poet Pherecrates exhibited on the stage at the last year's Lenaean festival. If you were living among men such as the man-haters in his Chorus, you would be only too glad to meet with Eurybates and Phrynondas, and you would sorrowfully long to revisit the rascality of this part of the world. You, Socrates, are discontented, and why? Because all men are teachers of virtue, each one according to his ability; and you say Where are the teachers? You might as well ask, Who teaches Greek? For of that too there will not be any teachers found. Or you might ask, Who is to teach the sons of our artisans this same art which they have learned of their fathers? He and his fellow-workmen have taught them to the best of their ability,—but who will carry them further in their arts? And you would certainly have a difficulty, Socrates, in finding a teacher of them; but there would be no difficulty in finding a teacher of those who are wholly ignorant. And this is true of virtue or of anything else; if a man is better able than we are to promote virtue ever so little, we must be content with the result. A teacher of this sort I believe myself to be, and above all other men to have the knowledge which makes a man noble and good; and I give my pupils their money's-worth, and even more, as they themselves confess. And therefore I have introduced the following mode of payment:—When a man has been my pupil, if he likes he pays my price, but there is no compulsion; and if he does not like, he has only to go into a temple and take an oath of the value of the instructions, and he pays no more than he declares to be their value.

\par  Such is my Apologue, Socrates, and such is the argument by which I endeavour to show that virtue may be taught, and that this is the opinion of the Athenians. And I have also attempted to show that you are not to wonder at good fathers having bad sons, or at good sons having bad fathers, of which the sons of Polycleitus afford an example, who are the companions of our friends here, Paralus and Xanthippus, but are nothing in comparison with their father; and this is true of the sons of many other artists. As yet I ought not to say the same of Paralus and Xanthippus themselves, for they are young and there is still hope of them.

\par  Protagoras ended, and in my ear

\par  'So charming left his voice, that I the while Thought him still speaking; still stood fixed to hear (Borrowed by Milton, "Paradise Lost".).'

\par  At length, when the truth dawned upon me, that he had really finished, not without difficulty I began to collect myself, and looking at Hippocrates, I said to him: O son of Apollodorus, how deeply grateful I am to you for having brought me hither; I would not have missed the speech of Protagoras for a great deal. For I used to imagine that no human care could make men good; but I know better now. Yet I have still one very small difficulty which I am sure that Protagoras will easily explain, as he has already explained so much. If a man were to go and consult Pericles or any of our great speakers about these matters, he might perhaps hear as fine a discourse; but then when one has a question to ask of any of them, like books, they can neither answer nor ask; and if any one challenges the least particular of their speech, they go ringing on in a long harangue, like brazen pots, which when they are struck continue to sound unless some one puts his hand upon them; whereas our friend Protagoras can not only make a good speech, as he has already shown, but when he is asked a question he can answer briefly; and when he asks he will wait and hear the answer; and this is a very rare gift. Now I, Protagoras, want to ask of you a little question, which if you will only answer, I shall be quite satisfied. You were saying that virtue can be taught;—that I will take upon your authority, and there is no one to whom I am more ready to trust. But I marvel at one thing about which I should like to have my mind set at rest. You were speaking of Zeus sending justice and reverence to men; and several times while you were speaking, justice, and temperance, and holiness, and all these qualities, were described by you as if together they made up virtue. Now I want you to tell me truly whether virtue is one whole, of which justice and temperance and holiness are parts; or whether all these are only the names of one and the same thing: that is the doubt which still lingers in my mind.

\par  There is no difficulty, Socrates, in answering that the qualities of which you are speaking are the parts of virtue which is one.

\par  And are they parts, I said, in the same sense in which mouth, nose, and eyes, and ears, are the parts of a face; or are they like the parts of gold, which differ from the whole and from one another only in being larger or smaller?

\par  I should say that they differed, Socrates, in the first way; they are related to one another as the parts of a face are related to the whole face.

\par  And do men have some one part and some another part of virtue? Or if a man has one part, must he also have all the others?

\par  By no means, he said; for many a man is brave and not just, or just and not wise.

\par  You would not deny, then, that courage and wisdom are also parts of virtue?

\par  Most undoubtedly they are, he answered; and wisdom is the noblest of the parts.

\par  And they are all different from one another? I said.

\par  Yes.

\par  And has each of them a distinct function like the parts of the face;—the eye, for example, is not like the ear, and has not the same functions; and the other parts are none of them like one another, either in their functions, or in any other way? I want to know whether the comparison holds concerning the parts of virtue. Do they also differ from one another in themselves and in their functions? For that is clearly what the simile would imply.

\par  Yes, Socrates, you are right in supposing that they differ.

\par  Then, I said, no other part of virtue is like knowledge, or like justice, or like courage, or like temperance, or like holiness?

\par  No, he answered.

\par  Well then, I said, suppose that you and I enquire into their natures. And first, you would agree with me that justice is of the nature of a thing, would you not? That is my opinion: would it not be yours also?

\par  Mine also, he said.

\par  And suppose that some one were to ask us, saying, 'O Protagoras, and you, Socrates, what about this thing which you were calling justice, is it just or unjust? '—and I were to answer, just: would you vote with me or against me?

\par  With you, he said.

\par  Thereupon I should answer to him who asked me, that justice is of the nature of the just: would not you?

\par  Yes, he said.

\par  And suppose that he went on to say: 'Well now, is there also such a thing as holiness? '—we should answer, 'Yes,' if I am not mistaken?

\par  Yes, he said.

\par  Which you would also acknowledge to be a thing—should we not say so?

\par  He assented.

\par  'And is this a sort of thing which is of the nature of the holy, or of the nature of the unholy?' I should be angry at his putting such a question, and should say, 'Peace, man; nothing can be holy if holiness is not holy.' What would you say? Would you not answer in the same way?

\par  Certainly, he said.

\par  And then after this suppose that he came and asked us, 'What were you saying just now? Perhaps I may not have heard you rightly, but you seemed to me to be saying that the parts of virtue were not the same as one another.' I should reply, 'You certainly heard that said, but not, as you imagine, by me; for I only asked the question; Protagoras gave the answer.' And suppose that he turned to you and said, 'Is this true, Protagoras? and do you maintain that one part of virtue is unlike another, and is this your position? '—how would you answer him?

\par  I could not help acknowledging the truth of what he said, Socrates.

\par  Well then, Protagoras, we will assume this; and now supposing that he proceeded to say further, 'Then holiness is not of the nature of justice, nor justice of the nature of holiness, but of the nature of unholiness; and holiness is of the nature of the not just, and therefore of the unjust, and the unjust is the unholy': how shall we answer him? I should certainly answer him on my own behalf that justice is holy, and that holiness is just; and I would say in like manner on your behalf also, if you would allow me, that justice is either the same with holiness, or very nearly the same; and above all I would assert that justice is like holiness and holiness is like justice; and I wish that you would tell me whether I may be permitted to give this answer on your behalf, and whether you would agree with me.

\par  He replied, I cannot simply agree, Socrates, to the proposition that justice is holy and that holiness is just, for there appears to me to be a difference between them. But what matter? if you please I please; and let us assume, if you will I, that justice is holy, and that holiness is just.

\par  Pardon me, I replied; I do not want this 'if you wish' or 'if you will' sort of conclusion to be proven, but I want you and me to be proven: I mean to say that the conclusion will be best proven if there be no 'if.'

\par  Well, he said, I admit that justice bears a resemblance to holiness, for there is always some point of view in which everything is like every other thing; white is in a certain way like black, and hard is like soft, and the most extreme opposites have some qualities in common; even the parts of the face which, as we were saying before, are distinct and have different functions, are still in a certain point of view similar, and one of them is like another of them. And you may prove that they are like one another on the same principle that all things are like one another; and yet things which are like in some particular ought not to be called alike, nor things which are unlike in some particular, however slight, unlike.

\par  And do you think, I said in a tone of surprise, that justice and holiness have but a small degree of likeness?

\par  Certainly not; any more than I agree with what I understand to be your view.

\par  Well, I said, as you appear to have a difficulty about this, let us take another of the examples which you mentioned instead. Do you admit the existence of folly?

\par  I do.

\par  And is not wisdom the very opposite of folly?

\par  That is true, he said.

\par  And when men act rightly and advantageously they seem to you to be temperate?

\par  Yes, he said.

\par  And temperance makes them temperate?

\par  Certainly.

\par  And they who do not act rightly act foolishly, and in acting thus are not temperate?

\par  I agree, he said.

\par  Then to act foolishly is the opposite of acting temperately?

\par  He assented.

\par  And foolish actions are done by folly, and temperate actions by temperance?

\par  He agreed.

\par  And that is done strongly which is done by strength, and that which is weakly done, by weakness?

\par  He assented.

\par  And that which is done with swiftness is done swiftly, and that which is done with slowness, slowly?

\par  He assented again.

\par  And that which is done in the same manner, is done by the same; and that which is done in an opposite manner by the opposite?

\par  He agreed.

\par  Once more, I said, is there anything beautiful?

\par  Yes.

\par  To which the only opposite is the ugly?

\par  There is no other.

\par  And is there anything good?

\par  There is.

\par  To which the only opposite is the evil?

\par  There is no other.

\par  And there is the acute in sound?

\par  True.

\par  To which the only opposite is the grave?

\par  There is no other, he said, but that.

\par  Then every opposite has one opposite only and no more?

\par  He assented.

\par  Then now, I said, let us recapitulate our admissions. First of all we admitted that everything has one opposite and not more than one?

\par  We did so.

\par  And we admitted also that what was done in opposite ways was done by opposites?

\par  Yes.

\par  And that which was done foolishly, as we further admitted, was done in the opposite way to that which was done temperately?

\par  Yes.

\par  And that which was done temperately was done by temperance, and that which was done foolishly by folly?

\par  He agreed.

\par  And that which is done in opposite ways is done by opposites?

\par  Yes.

\par  And one thing is done by temperance, and quite another thing by folly?

\par  Yes.

\par  And in opposite ways?

\par  Certainly.

\par  And therefore by opposites:—then folly is the opposite of temperance?

\par  Clearly.

\par  And do you remember that folly has already been acknowledged by us to be the opposite of wisdom?

\par  He assented.

\par  And we said that everything has only one opposite?

\par  Yes.

\par  Then, Protagoras, which of the two assertions shall we renounce? One says that everything has but one opposite; the other that wisdom is distinct from temperance, and that both of them are parts of virtue; and that they are not only distinct, but dissimilar, both in themselves and in their functions, like the parts of a face. Which of these two assertions shall we renounce? For both of them together are certainly not in harmony; they do not accord or agree: for how can they be said to agree if everything is assumed to have only one opposite and not more than one, and yet folly, which is one, has clearly the two opposites—wisdom and temperance? Is not that true, Protagoras? What else would you say?

\par  He assented, but with great reluctance.

\par  Then temperance and wisdom are the same, as before justice and holiness appeared to us to be nearly the same. And now, Protagoras, I said, we must finish the enquiry, and not faint. Do you think that an unjust man can be temperate in his injustice?

\par  I should be ashamed, Socrates, he said, to acknowledge this, which nevertheless many may be found to assert.

\par  And shall I argue with them or with you? I replied.

\par  I would rather, he said, that you should argue with the many first, if you will.

\par  Whichever you please, if you will only answer me and say whether you are of their opinion or not. My object is to test the validity of the argument; and yet the result may be that I who ask and you who answer may both be put on our trial.

\par  Protagoras at first made a show of refusing, as he said that the argument was not encouraging; at length, he consented to answer.

\par  Now then, I said, begin at the beginning and answer me. You think that some men are temperate, and yet unjust?

\par  Yes, he said; let that be admitted.

\par  And temperance is good sense?

\par  Yes.

\par  And good sense is good counsel in doing injustice?

\par  Granted.

\par  If they succeed, I said, or if they do not succeed?

\par  If they succeed.

\par  And you would admit the existence of goods?

\par  Yes.

\par  And is the good that which is expedient for man?

\par  Yes, indeed, he said: and there are some things which may be inexpedient, and yet I call them good.

\par  I thought that Protagoras was getting ruffled and excited; he seemed to be setting himself in an attitude of war. Seeing this, I minded my business, and gently said:—

\par  When you say, Protagoras, that things inexpedient are good, do you mean inexpedient for man only, or inexpedient altogether? and do you call the latter good?

\par  Certainly not the last, he replied; for I know of many things—meats, drinks, medicines, and ten thousand other things, which are inexpedient for man, and some which are expedient; and some which are neither expedient nor inexpedient for man, but only for horses; and some for oxen only, and some for dogs; and some for no animals, but only for trees; and some for the roots of trees and not for their branches, as for example, manure, which is a good thing when laid about the roots of a tree, but utterly destructive if thrown upon the shoots and young branches; or I may instance olive oil, which is mischievous to all plants, and generally most injurious to the hair of every animal with the exception of man, but beneficial to human hair and to the human body generally; and even in this application (so various and changeable is the nature of the benefit), that which is the greatest good to the outward parts of a man, is a very great evil to his inward parts: and for this reason physicians always forbid their patients the use of oil in their food, except in very small quantities, just enough to extinguish the disagreeable sensation of smell in meats and sauces.

\par  When he had given this answer, the company cheered him. And I said: Protagoras, I have a wretched memory, and when any one makes a long speech to me I never remember what he is talking about. As then, if I had been deaf, and you were going to converse with me, you would have had to raise your voice; so now, having such a bad memory, I will ask you to cut your answers shorter, if you would take me with you.

\par  What do you mean? he said: how am I to shorten my answers? shall I make them too short?

\par  Certainly not, I said.

\par  But short enough?

\par  Yes, I said.

\par  Shall I answer what appears to me to be short enough, or what appears to you to be short enough?

\par  I have heard, I said, that you can speak and teach others to speak about the same things at such length that words never seemed to fail, or with such brevity that no one could use fewer of them. Please therefore, if you talk with me, to adopt the latter or more compendious method.

\par  Socrates, he replied, many a battle of words have I fought, and if I had followed the method of disputation which my adversaries desired, as you want me to do, I should have been no better than another, and the name of Protagoras would have been nowhere.

\par  I saw that he was not satisfied with his previous answers, and that he would not play the part of answerer any more if he could help; and I considered that there was no call upon me to continue the conversation; so I said: Protagoras, I do not wish to force the conversation upon you if you had rather not, but when you are willing to argue with me in such a way that I can follow you, then I will argue with you. Now you, as is said of you by others and as you say of yourself, are able to have discussions in shorter forms of speech as well as in longer, for you are a master of wisdom; but I cannot manage these long speeches: I only wish that I could. You, on the other hand, who are capable of either, ought to speak shorter as I beg you, and then we might converse. But I see that you are disinclined, and as I have an engagement which will prevent my staying to hear you at greater length (for I have to be in another place), I will depart; although I should have liked to have heard you.

\par  Thus I spoke, and was rising from my seat, when Callias seized me by the right hand, and in his left hand caught hold of this old cloak of mine. He said: We cannot let you go, Socrates, for if you leave us there will be an end of our discussions: I must therefore beg you to remain, as there is nothing in the world that I should like better than to hear you and Protagoras discourse. Do not deny the company this pleasure.

\par  Now I had got up, and was in the act of departure. Son of Hipponicus, I replied, I have always admired, and do now heartily applaud and love your philosophical spirit, and I would gladly comply with your request, if I could. But the truth is that I cannot. And what you ask is as great an impossibility to me, as if you bade me run a race with Crison of Himera, when in his prime, or with some one of the long or day course runners. To such a request I should reply that I would fain ask the same of my own legs; but they refuse to comply. And therefore if you want to see Crison and me in the same stadium, you must bid him slacken his speed to mine, for I cannot run quickly, and he can run slowly. And in like manner if you want to hear me and Protagoras discoursing, you must ask him to shorten his answers, and keep to the point, as he did at first; if not, how can there be any discussion? For discussion is one thing, and making an oration is quite another, in my humble opinion.

\par  But you see, Socrates, said Callias, that Protagoras may fairly claim to speak in his own way, just as you claim to speak in yours.

\par  Here Alcibiades interposed, and said: That, Callias, is not a true statement of the case. For our friend Socrates admits that he cannot make a speech—in this he yields the palm to Protagoras: but I should be greatly surprised if he yielded to any living man in the power of holding and apprehending an argument. Now if Protagoras will make a similar admission, and confess that he is inferior to Socrates in argumentative skill, that is enough for Socrates; but if he claims a superiority in argument as well, let him ask and answer—not, when a question is asked, slipping away from the point, and instead of answering, making a speech at such length that most of his hearers forget the question at issue (not that Socrates is likely to forget—I will be bound for that, although he may pretend in fun that he has a bad memory). And Socrates appears to me to be more in the right than Protagoras; that is my view, and every man ought to say what he thinks.

\par  When Alcibiades had done speaking, some one—Critias, I believe—went on to say: O Prodicus and Hippias, Callias appears to me to be a partisan of Protagoras: and this led Alcibiades, who loves opposition, to take the other side. But we should not be partisans either of Socrates or of Protagoras; let us rather unite in entreating both of them not to break up the discussion.

\par  Prodicus added: That, Critias, seems to me to be well said, for those who are present at such discussions ought to be impartial hearers of both the speakers; remembering, however, that impartiality is not the same as equality, for both sides should be impartially heard, and yet an equal meed should not be assigned to both of them; but to the wiser a higher meed should be given, and a lower to the less wise. And I as well as Critias would beg you, Protagoras and Socrates, to grant our request, which is, that you will argue with one another and not wrangle; for friends argue with friends out of good-will, but only adversaries and enemies wrangle. And then our meeting will be delightful; for in this way you, who are the speakers, will be most likely to win esteem, and not praise only, among us who are your audience; for esteem is a sincere conviction of the hearers' souls, but praise is often an insincere expression of men uttering falsehoods contrary to their conviction. And thus we who are the hearers will be gratified and not pleased; for gratification is of the mind when receiving wisdom and knowledge, but pleasure is of the body when eating or experiencing some other bodily delight. Thus spoke Prodicus, and many of the company applauded his words.

\par  Hippias the sage spoke next. He said: All of you who are here present I reckon to be kinsmen and friends and fellow-citizens, by nature and not by law; for by nature like is akin to like, whereas law is the tyrant of mankind, and often compels us to do many things which are against nature. How great would be the disgrace then, if we, who know the nature of things, and are the wisest of the Hellenes, and as such are met together in this city, which is the metropolis of wisdom, and in the greatest and most glorious house of this city, should have nothing to show worthy of this height of dignity, but should only quarrel with one another like the meanest of mankind! I do pray and advise you, Protagoras, and you, Socrates, to agree upon a compromise. Let us be your peacemakers. And do not you, Socrates, aim at this precise and extreme brevity in discourse, if Protagoras objects, but loosen and let go the reins of speech, that your words may be grander and more becoming to you. Neither do you, Protagoras, go forth on the gale with every sail set out of sight of land into an ocean of words, but let there be a mean observed by both of you. Do as I say. And let me also persuade you to choose an arbiter or overseer or president; he will keep watch over your words and will prescribe their proper length.

\par  This proposal was received by the company with universal approval; Callias said that he would not let me off, and they begged me to choose an arbiter. But I said that to choose an umpire of discourse would be unseemly; for if the person chosen was inferior, then the inferior or worse ought not to preside over the better; or if he was equal, neither would that be well; for he who is our equal will do as we do, and what will be the use of choosing him? And if you say, 'Let us have a better then,'—to that I answer that you cannot have any one who is wiser than Protagoras. And if you choose another who is not really better, and whom you only say is better, to put another over him as though he were an inferior person would be an unworthy reflection on him; not that, as far as I am concerned, any reflection is of much consequence to me. Let me tell you then what I will do in order that the conversation and discussion may go on as you desire. If Protagoras is not disposed to answer, let him ask and I will answer; and I will endeavour to show at the same time how, as I maintain, he ought to answer: and when I have answered as many questions as he likes to ask, let him in like manner answer me; and if he seems to be not very ready at answering the precise question asked of him, you and I will unite in entreating him, as you entreated me, not to spoil the discussion. And this will require no special arbiter—all of you shall be arbiters.

\par  This was generally approved, and Protagoras, though very much against his will, was obliged to agree that he would ask questions; and when he had put a sufficient number of them, that he would answer in his turn those which he was asked in short replies. He began to put his questions as follows:—

\par  I am of opinion, Socrates, he said, that skill in poetry is the principal part of education; and this I conceive to be the power of knowing what compositions of the poets are correct, and what are not, and how they are to be distinguished, and of explaining when asked the reason of the difference. And I propose to transfer the question which you and I have been discussing to the domain of poetry; we will speak as before of virtue, but in reference to a passage of a poet. Now Simonides says to Scopas the son of Creon the Thessalian:

\par  'Hardly on the one hand can a man become truly good, built four-square in hands and feet and mind, a work without a flaw.'

\par  Do you know the poem? or shall I repeat the whole?

\par  There is no need, I said; for I am perfectly well acquainted with the ode,—I have made a careful study of it.

\par  Very well, he said. And do you think that the ode is a good composition, and true?

\par  Yes, I said, both good and true.

\par  But if there is a contradiction, can the composition be good or true?

\par  No, not in that case, I replied.

\par  And is there not a contradiction? he asked. Reflect.

\par  Well, my friend, I have reflected.

\par  And does not the poet proceed to say, 'I do not agree with the word of Pittacus, albeit the utterance of a wise man: Hardly can a man be good'? Now you will observe that this is said by the same poet.

\par  I know it.

\par  And do you think, he said, that the two sayings are consistent?

\par  Yes, I said, I think so (at the same time I could not help fearing that there might be something in what he said). And you think otherwise?

\par  Why, he said, how can he be consistent in both? First of all, premising as his own thought, 'Hardly can a man become truly good'; and then a little further on in the poem, forgetting, and blaming Pittacus and refusing to agree with him, when he says, 'Hardly can a man be good,' which is the very same thing. And yet when he blames him who says the same with himself, he blames himself; so that he must be wrong either in his first or his second assertion.

\par  Many of the audience cheered and applauded this. And I felt at first giddy and faint, as if I had received a blow from the hand of an expert boxer, when I heard his words and the sound of the cheering; and to confess the truth, I wanted to get time to think what the meaning of the poet really was. So I turned to Prodicus and called him. Prodicus, I said, Simonides is a countryman of yours, and you ought to come to his aid. I must appeal to you, like the river Scamander in Homer, who, when beleaguered by Achilles, summons the Simois to aid him, saying:

\par  'Brother dear, let us both together stay the force of the hero (Il.).'

\par  And I summon you, for I am afraid that Protagoras will make an end of Simonides. Now is the time to rehabilitate Simonides, by the application of your philosophy of synonyms, which enables you to distinguish 'will' and 'wish,' and make other charming distinctions like those which you drew just now. And I should like to know whether you would agree with me; for I am of opinion that there is no contradiction in the words of Simonides. And first of all I wish that you would say whether, in your opinion, Prodicus, 'being' is the same as 'becoming.'

\par  Not the same, certainly, replied Prodicus.

\par  Did not Simonides first set forth, as his own view, that 'Hardly can a man become truly good'?

\par  Quite right, said Prodicus.

\par  And then he blames Pittacus, not, as Protagoras imagines, for repeating that which he says himself, but for saying something different from himself. Pittacus does not say as Simonides says, that hardly can a man become good, but hardly can a man be good: and our friend Prodicus would maintain that being, Protagoras, is not the same as becoming; and if they are not the same, then Simonides is not inconsistent with himself. I dare say that Prodicus and many others would say, as Hesiod says,
 
\par  Prodicus heard and approved; but Protagoras said: Your correction, Socrates, involves a greater error than is contained in the sentence which you are correcting.

\par  Alas! I said, Protagoras; then I am a sorry physician, and do but aggravate a disorder which I am seeking to cure.

\par  Such is the fact, he said.

\par  How so? I asked.

\par  The poet, he replied, could never have made such a mistake as to say that virtue, which in the opinion of all men is the hardest of all things, can be easily retained.

\par  Well, I said, and how fortunate are we in having Prodicus among us, at the right moment; for he has a wisdom, Protagoras, which, as I imagine, is more than human and of very ancient date, and may be as old as Simonides or even older. Learned as you are in many things, you appear to know nothing of this; but I know, for I am a disciple of his. And now, if I am not mistaken, you do not understand the word 'hard' (chalepon) in the sense which Simonides intended; and I must correct you, as Prodicus corrects me when I use the word 'awful' (deinon) as a term of praise. If I say that Protagoras or any one else is an 'awfully' wise man, he asks me if I am not ashamed of calling that which is good 'awful'; and then he explains to me that the term 'awful' is always taken in a bad sense, and that no one speaks of being 'awfully' healthy or wealthy, or of 'awful' peace, but of 'awful' disease, 'awful' war, 'awful' poverty, meaning by the term 'awful,' evil. And I think that Simonides and his countrymen the Ceans, when they spoke of 'hard' meant 'evil,' or something which you do not understand. Let us ask Prodicus, for he ought to be able to answer questions about the dialect of Simonides. What did he mean, Prodicus, by the term 'hard'?

\par  Evil, said Prodicus.

\par  And therefore, I said, Prodicus, he blames Pittacus for saying, 'Hard is the good,' just as if that were equivalent to saying, Evil is the good.

\par  Yes, he said, that was certainly his meaning; and he is twitting Pittacus with ignorance of the use of terms, which in a Lesbian, who has been accustomed to speak a barbarous language, is natural.

\par  Do you hear, Protagoras, I asked, what our friend Prodicus is saying? And have you an answer for him?

\par  You are entirely mistaken, Prodicus, said Protagoras; and I know very well that Simonides in using the word 'hard' meant what all of us mean, not evil, but that which is not easy—that which takes a great deal of trouble: of this I am positive.

\par  I said: I also incline to believe, Protagoras, that this was the meaning of Simonides, of which our friend Prodicus was very well aware, but he thought that he would make fun, and try if you could maintain your thesis; for that Simonides could never have meant the other is clearly proved by the context, in which he says that God only has this gift. Now he cannot surely mean to say that to be good is evil, when he afterwards proceeds to say that God only has this gift, and that this is the attribute of him and of no other. For if this be his meaning, Prodicus would impute to Simonides a character of recklessness which is very unlike his countrymen. And I should like to tell you, I said, what I imagine to be the real meaning of Simonides in this poem, if you will test what, in your way of speaking, would be called my skill in poetry; or if you would rather, I will be the listener.

\par  To this proposal Protagoras replied: As you please;—and Hippias, Prodicus, and the others told me by all means to do as I proposed.

\par  Then now, I said, I will endeavour to explain to you my opinion about this poem of Simonides. There is a very ancient philosophy which is more cultivated in Crete and Lacedaemon than in any other part of Hellas, and there are more philosophers in those countries than anywhere else in the world. This, however, is a secret which the Lacedaemonians deny; and they pretend to be ignorant, just because they do not wish to have it thought that they rule the world by wisdom, like the Sophists of whom Protagoras was speaking, and not by valour of arms; considering that if the reason of their superiority were disclosed, all men would be practising their wisdom. And this secret of theirs has never been discovered by the imitators of Lacedaemonian fashions in other cities, who go about with their ears bruised in imitation of them, and have the caestus bound on their arms, and are always in training, and wear short cloaks; for they imagine that these are the practices which have enabled the Lacedaemonians to conquer the other Hellenes. Now when the Lacedaemonians want to unbend and hold free conversation with their wise men, and are no longer satisfied with mere secret intercourse, they drive out all these laconizers, and any other foreigners who may happen to be in their country, and they hold a philosophical seance unknown to strangers; and they themselves forbid their young men to go out into other cities—in this they are like the Cretans—in order that they may not unlearn the lessons which they have taught them. And in Lacedaemon and Crete not only men but also women have a pride in their high cultivation. And hereby you may know that I am right in attributing to the Lacedaemonians this excellence in philosophy and speculation: If a man converses with the most ordinary Lacedaemonian, he will find him seldom good for much in general conversation, but at any point in the discourse he will be darting out some notable saying, terse and full of meaning, with unerring aim; and the person with whom he is talking seems to be like a child in his hands. And many of our own age and of former ages have noted that the true Lacedaemonian type of character has the love of philosophy even stronger than the love of gymnastics; they are conscious that only a perfectly educated man is capable of uttering such expressions. Such were Thales of Miletus, and Pittacus of Mitylene, and Bias of Priene, and our own Solon, and Cleobulus the Lindian, and Myson the Chenian; and seventh in the catalogue of wise men was the Lacedaemonian Chilo. All these were lovers and emulators and disciples of the culture of the Lacedaemonians, and any one may perceive that their wisdom was of this character; consisting of short memorable sentences, which they severally uttered. And they met together and dedicated in the temple of Apollo at Delphi, as the first-fruits of their wisdom, the far-famed inscriptions, which are in all men's mouths—'Know thyself,' and 'Nothing too much.'

\par  Why do I say all this? I am explaining that this Lacedaemonian brevity was the style of primitive philosophy. Now there was a saying of Pittacus which was privately circulated and received the approbation of the wise, 'Hard is it to be good.' And Simonides, who was ambitious of the fame of wisdom, was aware that if he could overthrow this saying, then, as if he had won a victory over some famous athlete, he would carry off the palm among his contemporaries. And if I am not mistaken, he composed the entire poem with the secret intention of damaging Pittacus and his saying.

\par  Let us all unite in examining his words, and see whether I am speaking the truth. Simonides must have been a lunatic, if, in the very first words of the poem, wanting to say only that to become good is hard, he inserted (Greek) 'on the one hand' ('on the one hand to become good is hard'); there would be no reason for the introduction of (Greek), unless you suppose him to speak with a hostile reference to the words of Pittacus. Pittacus is saying 'Hard is it to be good,' and he, in refutation of this thesis, rejoins that the truly hard thing, Pittacus, is to become good, not joining 'truly' with 'good,' but with 'hard.' Not, that the hard thing is to be truly good, as though there were some truly good men, and there were others who were good but not truly good (this would be a very simple observation, and quite unworthy of Simonides); but you must suppose him to make a trajection of the word 'truly' (Greek), construing the saying of Pittacus thus (and let us imagine Pittacus to be speaking and Simonides answering him): 'O my friends,' says Pittacus, 'hard is it to be good,' and Simonides answers, 'In that, Pittacus, you are mistaken; the difficulty is not to be good, but on the one hand, to become good, four-square in hands and feet and mind, without a flaw—that is hard truly.' This way of reading the passage accounts for the insertion of (Greek) 'on the one hand,' and for the position at the end of the clause of the word 'truly,' and all that follows shows this to be the meaning. A great deal might be said in praise of the details of the poem, which is a charming piece of workmanship, and very finished, but such minutiae would be tedious. I should like, however, to point out the general intention of the poem, which is certainly designed in every part to be a refutation of the saying of Pittacus. For he speaks in what follows a little further on as if he meant to argue that although there is a difficulty in becoming good, yet this is possible for a time, and only for a time. But having become good, to remain in a good state and be good, as you, Pittacus, affirm, is not possible, and is not granted to man; God only has this blessing; 'but man cannot help being bad when the force of circumstances overpowers him.' Now whom does the force of circumstance overpower in the command of a vessel?—not the private individual, for he is always overpowered; and as one who is already prostrate cannot be overthrown, and only he who is standing upright but not he who is prostrate can be laid prostrate, so the force of circumstances can only overpower him who, at some time or other, has resources, and not him who is at all times helpless. The descent of a great storm may make the pilot helpless, or the severity of the season the husbandman or the physician; for the good may become bad, as another poet witnesses:—

\par  'The good are sometimes good and sometimes bad.'

\par  But the bad does not become bad; he is always bad. So that when the force of circumstances overpowers the man of resources and skill and virtue, then he cannot help being bad. And you, Pittacus, are saying, 'Hard is it to be good.' Now there is a difficulty in becoming good; and yet this is possible: but to be good is an impossibility—

\par  'For he who does well is the good man, and he who does ill is the bad.'

\par  But what sort of doing is good in letters? and what sort of doing makes a man good in letters? Clearly the knowing of them. And what sort of well-doing makes a man a good physician? Clearly the knowledge of the art of healing the sick. 'But he who does ill is the bad.' Now who becomes a bad physician? Clearly he who is in the first place a physician, and in the second place a good physician; for he may become a bad one also: but none of us unskilled individuals can by any amount of doing ill become physicians, any more than we can become carpenters or anything of that sort; and he who by doing ill cannot become a physician at all, clearly cannot become a bad physician. In like manner the good may become deteriorated by time, or toil, or disease, or other accident (the only real doing ill is to be deprived of knowledge), but the bad man will never become bad, for he is always bad; and if he were to become bad, he must previously have been good. Thus the words of the poem tend to show that on the one hand a man cannot be continuously good, but that he may become good and may also become bad; and again that

\par  'They are the best for the longest time whom the gods love.'

\par  All this relates to Pittacus, as is further proved by the sequel. For he adds:—

\par  'Therefore I will not throw away my span of life to no purpose in searching after the impossible, hoping in vain to find a perfectly faultless man among those who partake of the fruit of the broad-bosomed earth: if I find him, I will send you word.'

\par  (this is the vehement way in which he pursues his attack upon Pittacus throughout the whole poem):

\par  'But him who does no evil, voluntarily I praise and love;—not even the gods war against necessity.'

\par  All this has a similar drift, for Simonides was not so ignorant as to say that he praised those who did no evil voluntarily, as though there were some who did evil voluntarily. For no wise man, as I believe, will allow that any human being errs voluntarily, or voluntarily does evil and dishonourable actions; but they are very well aware that all who do evil and dishonourable things do them against their will. And Simonides never says that he praises him who does no evil voluntarily; the word 'voluntarily' applies to himself. For he was under the impression that a good man might often compel himself to love and praise another, and to be the friend and approver of another; and that there might be an involuntary love, such as a man might feel to an unnatural father or mother, or country, or the like. Now bad men, when their parents or country have any defects, look on them with malignant joy, and find fault with them and expose and denounce them to others, under the idea that the rest of mankind will be less likely to take themselves to task and accuse them of neglect; and they blame their defects far more than they deserve, in order that the odium which is necessarily incurred by them may be increased: but the good man dissembles his feelings, and constrains himself to praise them; and if they have wronged him and he is angry, he pacifies his anger and is reconciled, and compels himself to love and praise his own flesh and blood. And Simonides, as is probable, considered that he himself had often had to praise and magnify a tyrant or the like, much against his will, and he also wishes to imply to Pittacus that he does not censure him because he is censorious.

\par  'For I am satisfied' he says, 'when a man is neither bad nor very stupid; and when he knows justice (which is the health of states), and is of sound mind, I will find no fault with him, for I am not given to finding fault, and there are innumerable fools'

\par  (implying that if he delighted in censure he might have abundant opportunity of finding fault).

\par  'All things are good with which evil is unmingled.'

\par  In these latter words he does not mean to say that all things are good which have no evil in them, as you might say 'All things are white which have no black in them,' for that would be ridiculous; but he means to say that he accepts and finds no fault with the moderate or intermediate state.

\par  ('I do not hope' he says, 'to find a perfectly blameless man among those who partake of the fruits of the broad-bosomed earth (if I find him, I will send you word); in this sense I praise no man. But he who is moderately good, and does no evil, is good enough for me, who love and approve every one')

\par  (and here observe that he uses a Lesbian word, epainemi (approve), because he is addressing Pittacus,
 
\par  and that the stop should be put after 'voluntarily'); 'but there are some whom I involuntarily praise and love. And you, Pittacus, I would never have blamed, if you had spoken what was moderately good and true; but I do blame you because, putting on the appearance of truth, you are speaking falsely about the highest matters. '—And this, I said, Prodicus and Protagoras, I take to be the meaning of Simonides in this poem.

\par  Hippias said: I think, Socrates, that you have given a very good explanation of the poem; but I have also an excellent interpretation of my own which I will propound to you, if you will allow me.

\par  Nay, Hippias, said Alcibiades; not now, but at some other time. At present we must abide by the compact which was made between Socrates and Protagoras, to the effect that as long as Protagoras is willing to ask, Socrates should answer; or that if he would rather answer, then that Socrates should ask.

\par  I said: I wish Protagoras either to ask or answer as he is inclined; but I would rather have done with poems and odes, if he does not object, and come back to the question about which I was asking you at first, Protagoras, and by your help make an end of that. The talk about the poets seems to me like a commonplace entertainment to which a vulgar company have recourse; who, because they are not able to converse or amuse one another, while they are drinking, with the sound of their own voices and conversation, by reason of their stupidity, raise the price of flute-girls in the market, hiring for a great sum the voice of a flute instead of their own breath, to be the medium of intercourse among them: but where the company are real gentlemen and men of education, you will see no flute-girls, nor dancing-girls, nor harp-girls; and they have no nonsense or games, but are contented with one another's conversation, of which their own voices are the medium, and which they carry on by turns and in an orderly manner, even though they are very liberal in their potations. And a company like this of ours, and men such as we profess to be, do not require the help of another's voice, or of the poets whom you cannot interrogate about the meaning of what they are saying; people who cite them declaring, some that the poet has one meaning, and others that he has another, and the point which is in dispute can never be decided. This sort of entertainment they decline, and prefer to talk with one another, and put one another to the proof in conversation. And these are the models which I desire that you and I should imitate. Leaving the poets, and keeping to ourselves, let us try the mettle of one another and make proof of the truth in conversation. If you have a mind to ask, I am ready to answer; or if you would rather, do you answer, and give me the opportunity of resuming and completing our unfinished argument.

\par  I made these and some similar observations; but Protagoras would not distinctly say which he would do. Thereupon Alcibiades turned to Callias, and said:—Do you think, Callias, that Protagoras is fair in refusing to say whether he will or will not answer? for I certainly think that he is unfair; he ought either to proceed with the argument, or distinctly refuse to proceed, that we may know his intention; and then Socrates will be able to discourse with some one else, and the rest of the company will be free to talk with one another.

\par  I think that Protagoras was really made ashamed by these words of Alcibiades, and when the prayers of Callias and the company were superadded, he was at last induced to argue, and said that I might ask and he would answer.

\par  So I said: Do not imagine, Protagoras, that I have any other interest in asking questions of you but that of clearing up my own difficulties. For I think that Homer was very right in saying that
 
\par  for all men who have a companion are readier in deed, word, or thought; but if a man
 
\par  he goes about straightway seeking until he finds some one to whom he may show his discoveries, and who may confirm him in them. And I would rather hold discourse with you than with any one, because I think that no man has a better understanding of most things which a good man may be expected to understand, and in particular of virtue. For who is there, but you?—who not only claim to be a good man and a gentleman, for many are this, and yet have not the power of making others good—whereas you are not only good yourself, but also the cause of goodness in others. Moreover such confidence have you in yourself, that although other Sophists conceal their profession, you proclaim in the face of Hellas that you are a Sophist or teacher of virtue and education, and are the first who demanded pay in return. How then can I do otherwise than invite you to the examination of these subjects, and ask questions and consult with you? I must, indeed. And I should like once more to have my memory refreshed by you about the questions which I was asking you at first, and also to have your help in considering them. If I am not mistaken the question was this: Are wisdom and temperance and courage and justice and holiness five names of the same thing? or has each of the names a separate underlying essence and corresponding thing having a peculiar function, no one of them being like any other of them? And you replied that the five names were not the names of the same thing, but that each of them had a separate object, and that all these objects were parts of virtue, not in the same way that the parts of gold are like each other and the whole of which they are parts, but as the parts of the face are unlike the whole of which they are parts and one another, and have each of them a distinct function. I should like to know whether this is still your opinion; or if not, I will ask you to define your meaning, and I shall not take you to task if you now make a different statement. For I dare say that you may have said what you did only in order to make trial of me.

\par  I answer, Socrates, he said, that all these qualities are parts of virtue, and that four out of the five are to some extent similar, and that the fifth of them, which is courage, is very different from the other four, as I prove in this way: You may observe that many men are utterly unrighteous, unholy, intemperate, ignorant, who are nevertheless remarkable for their courage.

\par  Stop, I said; I should like to think about that. When you speak of brave men, do you mean the confident, or another sort of nature?

\par  Yes, he said; I mean the impetuous, ready to go at that which others are afraid to approach.

\par  In the next place, you would affirm virtue to be a good thing, of which good thing you assert yourself to be a teacher.

\par  Yes, he said; I should say the best of all things, if I am in my right mind.

\par  And is it partly good and partly bad, I said, or wholly good?

\par  Wholly good, and in the highest degree.

\par  Tell me then; who are they who have confidence when diving into a well?

\par  I should say, the divers.

\par  And the reason of this is that they have knowledge?

\par  Yes, that is the reason.

\par  And who have confidence when fighting on horseback—the skilled horseman or the unskilled?

\par  The skilled.

\par  And who when fighting with light shields—the peltasts or the nonpeltasts?

\par  The peltasts. And that is true of all other things, he said, if that is your point: those who have knowledge are more confident than those who have no knowledge, and they are more confident after they have learned than before.

\par  And have you not seen persons utterly ignorant, I said, of these things, and yet confident about them?

\par  Yes, he said, I have seen such persons far too confident.

\par  And are not these confident persons also courageous?

\par  In that case, he replied, courage would be a base thing, for the men of whom we are speaking are surely madmen.

\par  Then who are the courageous? Are they not the confident?

\par  Yes, he said; to that statement I adhere.

\par  And those, I said, who are thus confident without knowledge are really not courageous, but mad; and in that case the wisest are also the most confident, and being the most confident are also the bravest, and upon that view again wisdom will be courage.

\par  Nay, Socrates, he replied, you are mistaken in your remembrance of what was said by me. When you asked me, I certainly did say that the courageous are the confident; but I was never asked whether the confident are the courageous; if you had asked me, I should have answered 'Not all of them': and what I did answer you have not proved to be false, although you proceeded to show that those who have knowledge are more courageous than they were before they had knowledge, and more courageous than others who have no knowledge, and were then led on to think that courage is the same as wisdom. But in this way of arguing you might come to imagine that strength is wisdom. You might begin by asking whether the strong are able, and I should say 'Yes'; and then whether those who know how to wrestle are not more able to wrestle than those who do not know how to wrestle, and more able after than before they had learned, and I should assent. And when I had admitted this, you might use my admissions in such a way as to prove that upon my view wisdom is strength; whereas in that case I should not have admitted, any more than in the other, that the able are strong, although I have admitted that the strong are able. For there is a difference between ability and strength; the former is given by knowledge as well as by madness or rage, but strength comes from nature and a healthy state of the body. And in like manner I say of confidence and courage, that they are not the same; and I argue that the courageous are confident, but not all the confident courageous. For confidence may be given to men by art, and also, like ability, by madness and rage; but courage comes to them from nature and the healthy state of the soul.

\par  I said: You would admit, Protagoras, that some men live well and others ill?

\par  He assented.

\par  And do you think that a man lives well who lives in pain and grief?

\par  He does not.

\par  But if he lives pleasantly to the end of his life, will he not in that case have lived well?

\par  He will.

\par  Then to live pleasantly is a good, and to live unpleasantly an evil?

\par  Yes, he said, if the pleasure be good and honourable.

\par  And do you, Protagoras, like the rest of the world, call some pleasant things evil and some painful things good?—for I am rather disposed to say that things are good in as far as they are pleasant, if they have no consequences of another sort, and in as far as they are painful they are bad.

\par  I do not know, Socrates, he said, whether I can venture to assert in that unqualified manner that the pleasant is the good and the painful the evil. Having regard not only to my present answer, but also to the whole of my life, I shall be safer, if I am not mistaken, in saying that there are some pleasant things which are not good, and that there are some painful things which are good, and some which are not good, and that there are some which are neither good nor evil.

\par  And you would call pleasant, I said, the things which participate in pleasure or create pleasure?

\par  Certainly, he said.

\par  Then my meaning is, that in as far as they are pleasant they are good; and my question would imply that pleasure is a good in itself.

\par  According to your favourite mode of speech, Socrates, 'Let us reflect about this,' he said; and if the reflection is to the point, and the result proves that pleasure and good are really the same, then we will agree; but if not, then we will argue.

\par  And would you wish to begin the enquiry? I said; or shall I begin?

\par  You ought to take the lead, he said; for you are the author of the discussion.

\par  May I employ an illustration? I said. Suppose some one who is enquiring into the health or some other bodily quality of another:—he looks at his face and at the tips of his fingers, and then he says, Uncover your chest and back to me that I may have a better view:—that is the sort of thing which I desire in this speculation. Having seen what your opinion is about good and pleasure, I am minded to say to you: Uncover your mind to me, Protagoras, and reveal your opinion about knowledge, that I may know whether you agree with the rest of the world. Now the rest of the world are of opinion that knowledge is a principle not of strength, or of rule, or of command: their notion is that a man may have knowledge, and yet that the knowledge which is in him may be overmastered by anger, or pleasure, or pain, or love, or perhaps by fear,—just as if knowledge were a slave, and might be dragged about anyhow. Now is that your view? or do you think that knowledge is a noble and commanding thing, which cannot be overcome, and will not allow a man, if he only knows the difference of good and evil, to do anything which is contrary to knowledge, but that wisdom will have strength to help him?

\par  I agree with you, Socrates, said Protagoras; and not only so, but I, above all other men, am bound to say that wisdom and knowledge are the highest of human things.

\par  Good, I said, and true. But are you aware that the majority of the world are of another mind; and that men are commonly supposed to know the things which are best, and not to do them when they might? And most persons whom I have asked the reason of this have said that when men act contrary to knowledge they are overcome by pain, or pleasure, or some of those affections which I was just now mentioning.

\par  Yes, Socrates, he replied; and that is not the only point about which mankind are in error.

\par  Suppose, then, that you and I endeavour to instruct and inform them what is the nature of this affection which they call 'being overcome by pleasure,' and which they affirm to be the reason why they do not always do what is best. When we say to them: Friends, you are mistaken, and are saying what is not true, they would probably reply: Socrates and Protagoras, if this affection of the soul is not to be called 'being overcome by pleasure,' pray, what is it, and by what name would you describe it?

\par  But why, Socrates, should we trouble ourselves about the opinion of the many, who just say anything that happens to occur to them?

\par  I believe, I said, that they may be of use in helping us to discover how courage is related to the other parts of virtue. If you are disposed to abide by our agreement, that I should show the way in which, as I think, our recent difficulty is most likely to be cleared up, do you follow; but if not, never mind.

\par  You are quite right, he said; and I would have you proceed as you have begun.

\par  Well then, I said, let me suppose that they repeat their question, What account do you give of that which, in our way of speaking, is termed being overcome by pleasure? I should answer thus: Listen, and Protagoras and I will endeavour to show you. When men are overcome by eating and drinking and other sensual desires which are pleasant, and they, knowing them to be evil, nevertheless indulge in them, would you not say that they were overcome by pleasure? They will not deny this. And suppose that you and I were to go on and ask them again: 'In what way do you say that they are evil,—in that they are pleasant and give pleasure at the moment, or because they cause disease and poverty and other like evils in the future? Would they still be evil, if they had no attendant evil consequences, simply because they give the consciousness of pleasure of whatever nature? '—Would they not answer that they are not evil on account of the pleasure which is immediately given by them, but on account of the after consequences—diseases and the like?

\par  I believe, said Protagoras, that the world in general would answer as you do.

\par  And in causing diseases do they not cause pain? and in causing poverty do they not cause pain;—they would agree to that also, if I am not mistaken?

\par  Protagoras assented.

\par  Then I should say to them, in my name and yours: Do you think them evil for any other reason, except because they end in pain and rob us of other pleasures:—there again they would agree?

\par  We both of us thought that they would.

\par  And then I should take the question from the opposite point of view, and say: 'Friends, when you speak of goods being painful, do you not mean remedial goods, such as gymnastic exercises, and military service, and the physician's use of burning, cutting, drugging, and starving? Are these the things which are good but painful? '—they would assent to me?

\par  He agreed.

\par  'And do you call them good because they occasion the greatest immediate suffering and pain; or because, afterwards, they bring health and improvement of the bodily condition and the salvation of states and power over others and wealth? '—they would agree to the latter alternative, if I am not mistaken?

\par  He assented.

\par  'Are these things good for any other reason except that they end in pleasure, and get rid of and avert pain? Are you looking to any other standard but pleasure and pain when you call them good? '—they would acknowledge that they were not?

\par  I think so, said Protagoras.

\par  'And do you not pursue after pleasure as a good, and avoid pain as an evil?'

\par  He assented.

\par  'Then you think that pain is an evil and pleasure is a good: and even pleasure you deem an evil, when it robs you of greater pleasures than it gives, or causes pains greater than the pleasure. If, however, you call pleasure an evil in relation to some other end or standard, you will be able to show us that standard. But you have none to show.'

\par  I do not think that they have, said Protagoras.

\par  'And have you not a similar way of speaking about pain? You call pain a good when it takes away greater pains than those which it has, or gives pleasures greater than the pains: then if you have some standard other than pleasure and pain to which you refer when you call actual pain a good, you can show what that is. But you cannot.'

\par  True, said Protagoras.

\par  Suppose again, I said, that the world says to me: 'Why do you spend many words and speak in many ways on this subject?' Excuse me, friends, I should reply; but in the first place there is a difficulty in explaining the meaning of the expression 'overcome by pleasure'; and the whole argument turns upon this. And even now, if you see any possible way in which evil can be explained as other than pain, or good as other than pleasure, you may still retract. Are you satisfied, then, at having a life of pleasure which is without pain? If you are, and if you are unable to show any good or evil which does not end in pleasure and pain, hear the consequences:—If what you say is true, then the argument is absurd which affirms that a man often does evil knowingly, when he might abstain, because he is seduced and overpowered by pleasure; or again, when you say that a man knowingly refuses to do what is good because he is overcome at the moment by pleasure. And that this is ridiculous will be evident if only we give up the use of various names, such as pleasant and painful, and good and evil. As there are two things, let us call them by two names—first, good and evil, and then pleasant and painful. Assuming this, let us go on to say that a man does evil knowing that he does evil. But some one will ask, Why? Because he is overcome, is the first answer. And by what is he overcome? the enquirer will proceed to ask. And we shall not be able to reply 'By pleasure,' for the name of pleasure has been exchanged for that of good. In our answer, then, we shall only say that he is overcome. 'By what?' he will reiterate. By the good, we shall have to reply; indeed we shall. Nay, but our questioner will rejoin with a laugh, if he be one of the swaggering sort, 'That is too ridiculous, that a man should do what he knows to be evil when he ought not, because he is overcome by good. Is that, he will ask, because the good was worthy or not worthy of conquering the evil'? And in answer to that we shall clearly reply, Because it was not worthy; for if it had been worthy, then he who, as we say, was overcome by pleasure, would not have been wrong. 'But how,' he will reply, 'can the good be unworthy of the evil, or the evil of the good'? Is not the real explanation that they are out of proportion to one another, either as greater and smaller, or more and fewer? This we cannot deny. And when you speak of being overcome—'what do you mean,' he will say, 'but that you choose the greater evil in exchange for the lesser good?' Admitted. And now substitute the names of pleasure and pain for good and evil, and say, not as before, that a man does what is evil knowingly, but that he does what is painful knowingly, and because he is overcome by pleasure, which is unworthy to overcome. What measure is there of the relations of pleasure to pain other than excess and defect, which means that they become greater and smaller, and more and fewer, and differ in degree? For if any one says: 'Yes, Socrates, but immediate pleasure differs widely from future pleasure and pain'—To that I should reply: And do they differ in anything but in pleasure and pain? There can be no other measure of them. And do you, like a skilful weigher, put into the balance the pleasures and the pains, and their nearness and distance, and weigh them, and then say which outweighs the other. If you weigh pleasures against pleasures, you of course take the more and greater; or if you weigh pains against pains, you take the fewer and the less; or if pleasures against pains, then you choose that course of action in which the painful is exceeded by the pleasant, whether the distant by the near or the near by the distant; and you avoid that course of action in which the pleasant is exceeded by the painful. Would you not admit, my friends, that this is true? I am confident that they cannot deny this.

\par  He agreed with me.

\par  Well then, I shall say, if you agree so far, be so good as to answer me a question: Do not the same magnitudes appear larger to your sight when near, and smaller when at a distance? They will acknowledge that. And the same holds of thickness and number; also sounds, which are in themselves equal, are greater when near, and lesser when at a distance. They will grant that also. Now suppose happiness to consist in doing or choosing the greater, and in not doing or in avoiding the less, what would be the saving principle of human life? Would not the art of measuring be the saving principle; or would the power of appearance? Is not the latter that deceiving art which makes us wander up and down and take the things at one time of which we repent at another, both in our actions and in our choice of things great and small? But the art of measurement would do away with the effect of appearances, and, showing the truth, would fain teach the soul at last to find rest in the truth, and would thus save our life. Would not mankind generally acknowledge that the art which accomplishes this result is the art of measurement?

\par  Yes, he said, the art of measurement.

\par  Suppose, again, the salvation of human life to depend on the choice of odd and even, and on the knowledge of when a man ought to choose the greater or less, either in reference to themselves or to each other, and whether near or at a distance; what would be the saving principle of our lives? Would not knowledge?—a knowledge of measuring, when the question is one of excess and defect, and a knowledge of number, when the question is of odd and even? The world will assent, will they not?

\par  Protagoras himself thought that they would.

\par  Well then, my friends, I say to them; seeing that the salvation of human life has been found to consist in the right choice of pleasures and pains,—in the choice of the more and the fewer, and the greater and the less, and the nearer and remoter, must not this measuring be a consideration of their excess and defect and equality in relation to each other?

\par  This is undeniably true.

\par  And this, as possessing measure, must undeniably also be an art and science?

\par  They will agree, he said.

\par  The nature of that art or science will be a matter of future consideration; but the existence of such a science furnishes a demonstrative answer to the question which you asked of me and Protagoras. At the time when you asked the question, if you remember, both of us were agreeing that there was nothing mightier than knowledge, and that knowledge, in whatever existing, must have the advantage over pleasure and all other things; and then you said that pleasure often got the advantage even over a man who has knowledge; and we refused to allow this, and you rejoined: O Protagoras and Socrates, what is the meaning of being overcome by pleasure if not this?—tell us what you call such a state:—if we had immediately and at the time answered 'Ignorance,' you would have laughed at us. But now, in laughing at us, you will be laughing at yourselves: for you also admitted that men err in their choice of pleasures and pains; that is, in their choice of good and evil, from defect of knowledge; and you admitted further, that they err, not only from defect of knowledge in general, but of that particular knowledge which is called measuring. And you are also aware that the erring act which is done without knowledge is done in ignorance. This, therefore, is the meaning of being overcome by pleasure;—ignorance, and that the greatest. And our friends Protagoras and Prodicus and Hippias declare that they are the physicians of ignorance; but you, who are under the mistaken impression that ignorance is not the cause, and that the art of which I am speaking cannot be taught, neither go yourselves, nor send your children, to the Sophists, who are the teachers of these things—you take care of your money and give them none; and the result is, that you are the worse off both in public and private life:—Let us suppose this to be our answer to the world in general: And now I should like to ask you, Hippias, and you, Prodicus, as well as Protagoras (for the argument is to be yours as well as ours), whether you think that I am speaking the truth or not?

\par  They all thought that what I said was entirely true.

\par  Then you agree, I said, that the pleasant is the good, and the painful evil. And here I would beg my friend Prodicus not to introduce his distinction of names, whether he is disposed to say pleasurable, delightful, joyful. However, by whatever name he prefers to call them, I will ask you, most excellent Prodicus, to answer in my sense of the words.

\par  Prodicus laughed and assented, as did the others.

\par  Then, my friends, what do you say to this? Are not all actions honourable and useful, of which the tendency is to make life painless and pleasant? The honourable work is also useful and good?

\par  This was admitted.

\par  Then, I said, if the pleasant is the good, nobody does anything under the idea or conviction that some other thing would be better and is also attainable, when he might do the better. And this inferiority of a man to himself is merely ignorance, as the superiority of a man to himself is wisdom.

\par  They all assented.

\par  And is not ignorance the having a false opinion and being deceived about important matters?

\par  To this also they unanimously assented.

\par  Then, I said, no man voluntarily pursues evil, or that which he thinks to be evil. To prefer evil to good is not in human nature; and when a man is compelled to choose one of two evils, no one will choose the greater when he may have the less.

\par  All of us agreed to every word of this.

\par  Well, I said, there is a certain thing called fear or terror; and here, Prodicus, I should particularly like to know whether you would agree with me in defining this fear or terror as expectation of evil.

\par  Protagoras and Hippias agreed, but Prodicus said that this was fear and not terror.

\par  Never mind, Prodicus, I said; but let me ask whether, if our former assertions are true, a man will pursue that which he fears when he is not compelled? Would not this be in flat contradiction to the admission which has been already made, that he thinks the things which he fears to be evil; and no one will pursue or voluntarily accept that which he thinks to be evil?

\par  That also was universally admitted.

\par  Then, I said, these, Hippias and Prodicus, are our premisses; and I would beg Protagoras to explain to us how he can be right in what he said at first. I do not mean in what he said quite at first, for his first statement, as you may remember, was that whereas there were five parts of virtue none of them was like any other of them; each of them had a separate function. To this, however, I am not referring, but to the assertion which he afterwards made that of the five virtues four were nearly akin to each other, but that the fifth, which was courage, differed greatly from the others. And of this he gave me the following proof. He said: You will find, Socrates, that some of the most impious, and unrighteous, and intemperate, and ignorant of men are among the most courageous; which proves that courage is very different from the other parts of virtue. I was surprised at his saying this at the time, and I am still more surprised now that I have discussed the matter with you. So I asked him whether by the brave he meant the confident. Yes, he replied, and the impetuous or goers. (You may remember, Protagoras, that this was your answer.)

\par  He assented.

\par  Well then, I said, tell us against what are the courageous ready to go—against the same dangers as the cowards?

\par  No, he answered.

\par  Then against something different?

\par  Yes, he said.

\par  Then do cowards go where there is safety, and the courageous where there is danger?

\par  Yes, Socrates, so men say.

\par  Very true, I said. But I want to know against what do you say that the courageous are ready to go—against dangers, believing them to be dangers, or not against dangers?

\par  No, said he; the former case has been proved by you in the previous argument to be impossible.

\par  That, again, I replied, is quite true. And if this has been rightly proven, then no one goes to meet what he thinks to be dangers, since the want of self-control, which makes men rush into dangers, has been shown to be ignorance.

\par  He assented.

\par  And yet the courageous man and the coward alike go to meet that about which they are confident; so that, in this point of view, the cowardly and the courageous go to meet the same things.

\par  And yet, Socrates, said Protagoras, that to which the coward goes is the opposite of that to which the courageous goes; the one, for example, is ready to go to battle, and the other is not ready.

\par  And is going to battle honourable or disgraceful? I said.

\par  Honourable, he replied.

\par  And if honourable, then already admitted by us to be good; for all honourable actions we have admitted to be good.

\par  That is true; and to that opinion I shall always adhere.

\par  True, I said. But which of the two are they who, as you say, are unwilling to go to war, which is a good and honourable thing?

\par  The cowards, he replied.

\par  And what is good and honourable, I said, is also pleasant?

\par  It has certainly been acknowledged to be so, he replied.

\par  And do the cowards knowingly refuse to go to the nobler, and pleasanter, and better?

\par  The admission of that, he replied, would belie our former admissions.

\par  But does not the courageous man also go to meet the better, and pleasanter, and nobler?

\par  That must be admitted.

\par  And the courageous man has no base fear or base confidence?

\par  True, he replied.

\par  And if not base, then honourable?

\par  He admitted this.

\par  And if honourable, then good?

\par  Yes.

\par  But the fear and confidence of the coward or foolhardy or madman, on the contrary, are base?

\par  He assented.

\par  And these base fears and confidences originate in ignorance and uninstructedness?

\par  True, he said.

\par  Then as to the motive from which the cowards act, do you call it cowardice or courage?

\par  I should say cowardice, he replied.

\par  And have they not been shown to be cowards through their ignorance of dangers?

\par  Assuredly, he said.

\par  And because of that ignorance they are cowards?

\par  He assented.

\par  And the reason why they are cowards is admitted by you to be cowardice?

\par  He again assented.

\par  Then the ignorance of what is and is not dangerous is cowardice?

\par  He nodded assent.

\par  But surely courage, I said, is opposed to cowardice?

\par  Yes.

\par  Then the wisdom which knows what are and are not dangers is opposed to the ignorance of them?

\par  To that again he nodded assent.

\par  And the ignorance of them is cowardice?

\par  To that he very reluctantly nodded assent.

\par  And the knowledge of that which is and is not dangerous is courage, and is opposed to the ignorance of these things?

\par  At this point he would no longer nod assent, but was silent.

\par  And why, I said, do you neither assent nor dissent, Protagoras?

\par  Finish the argument by yourself, he said.

\par  I only want to ask one more question, I said. I want to know whether you still think that there are men who are most ignorant and yet most courageous?

\par  You seem to have a great ambition to make me answer, Socrates, and therefore I will gratify you, and say, that this appears to me to be impossible consistently with the argument.

\par  My only object, I said, in continuing the discussion, has been the desire to ascertain the nature and relations of virtue; for if this were clear, I am very sure that the other controversy which has been carried on at great length by both of us—you affirming and I denying that virtue can be taught—would also become clear. The result of our discussion appears to me to be singular. For if the argument had a human voice, that voice would be heard laughing at us and saying: 'Protagoras and Socrates, you are strange beings; there are you, Socrates, who were saying that virtue cannot be taught, contradicting yourself now by your attempt to prove that all things are knowledge, including justice, and temperance, and courage,—which tends to show that virtue can certainly be taught; for if virtue were other than knowledge, as Protagoras attempted to prove, then clearly virtue cannot be taught; but if virtue is entirely knowledge, as you are seeking to show, then I cannot but suppose that virtue is capable of being taught. Protagoras, on the other hand, who started by saying that it might be taught, is now eager to prove it to be anything rather than knowledge; and if this is true, it must be quite incapable of being taught.' Now I, Protagoras, perceiving this terrible confusion of our ideas, have a great desire that they should be cleared up. And I should like to carry on the discussion until we ascertain what virtue is, whether capable of being taught or not, lest haply Epimetheus should trip us up and deceive us in the argument, as he forgot us in the story; I prefer your Prometheus to your Epimetheus, for of him I make use, whenever I am busy about these questions, in Promethean care of my own life. And if you have no objection, as I said at first, I should like to have your help in the enquiry.

\par  Protagoras replied: Socrates, I am not of a base nature, and I am the last man in the world to be envious. I cannot but applaud your energy and your conduct of an argument. As I have often said, I admire you above all men whom I know, and far above all men of your age; and I believe that you will become very eminent in philosophy. Let us come back to the subject at some future time; at present we had better turn to something else.

\par  By all means, I said, if that is your wish; for I too ought long since to have kept the engagement of which I spoke before, and only tarried because I could not refuse the request of the noble Callias. So the conversation ended, and we went our way.

\par 
 
\end{document}