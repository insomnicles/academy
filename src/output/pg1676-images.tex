
\documentclass[11pt,letter]{article}


\begin{document}

\title{Alcibiades I\thanks{Source: https://www.gutenberg.org/files/1676/1676-h/1676-h.htm. License: http://gutenberg.org/license ds}}
\date{\today}
\author{Plato (spurious and doubtful works), 427? BCE-347? BCE\\ Translated by }
\maketitle

\setcounter{tocdepth}{1}
\tableofcontents
\renewcommand{\baselinestretch}{1.0}
\normalsize
\newpage


\par 
\section{
      ALCIBIADES I
    }
\par 
\section{
      by Plato (see Appendix I)
    }
\par 
\section{
      Translated by Benjamin Jowett
    }
\par 

\par 
\section{
      APPENDIX I.
    }
\par  It seems impossible to separate by any exact line the genuine writings of Plato from the spurious. The only external evidence to them which is of much value is that of Aristotle; for the Alexandrian catalogues of a century later include manifest forgeries. Even the value of the Aristotelian authority is a good deal impaired by the uncertainty concerning the date and authorship of the writings which are ascribed to him. And several of the citations of Aristotle omit the name of Plato, and some of them omit the name of the dialogue from which they are taken. Prior, however, to the enquiry about the writings of a particular author, general considerations which equally affect all evidence to the genuineness of ancient writings are the following: Shorter works are more likely to have been forged, or to have received an erroneous designation, than longer ones; and some kinds of composition, such as epistles or panegyrical orations, are more liable to suspicion than others; those, again, which have a taste of sophistry in them, or the ring of a later age, or the slighter character of a rhetorical exercise, or in which a motive or some affinity to spurious writings can be detected, or which seem to have originated in a name or statement really occurring in some classical author, are also of doubtful credit; while there is no instance of any ancient writing proved to be a forgery, which combines excellence with length. A really great and original writer would have no object in fathering his works on Plato; and to the forger or imitator, the 'literary hack' of Alexandria and Athens, the Gods did not grant originality or genius. Further, in attempting to balance the evidence for and against a Platonic dialogue, we must not forget that the form of the Platonic writing was common to several of his contemporaries. Aeschines, Euclid, Phaedo, Antisthenes, and in the next generation Aristotle, are all said to have composed dialogues; and mistakes of names are very likely to have occurred. Greek literature in the third century before Christ was almost as voluminous as our own, and without the safeguards of regular publication, or printing, or binding, or even of distinct titles. An unknown writing was naturally attributed to a known writer whose works bore the same character; and the name once appended easily obtained authority. A tendency may also be observed to blend the works and opinions of the master with those of his scholars. To a later Platonist, the difference between Plato and his imitators was not so perceptible as to ourselves. The Memorabilia of Xenophon and the Dialogues of Plato are but a part of a considerable Socratic literature which has passed away. And we must consider how we should regard the question of the genuineness of a particular writing, if this lost literature had been preserved to us.

\par  These considerations lead us to adopt the following criteria of genuineness: (1) That is most certainly Plato's which Aristotle attributes to him by name, which (2) is of considerable length, of (3) great excellence, and also (4) in harmony with the general spirit of the Platonic writings. But the testimony of Aristotle cannot always be distinguished from that of a later age (see above); and has various degrees of importance. Those writings which he cites without mentioning Plato, under their own names, e.g. the Hippias, the Funeral Oration, the Phaedo, etc., have an inferior degree of evidence in their favour. They may have been supposed by him to be the writings of another, although in the case of really great works, e.g. the Phaedo, this is not credible; those again which are quoted but not named, are still more defective in their external credentials. There may be also a possibility that Aristotle was mistaken, or may have confused the master and his scholars in the case of a short writing; but this is inconceivable about a more important work, e.g. the Laws, especially when we remember that he was living at Athens, and a frequenter of the groves of the Academy, during the last twenty years of Plato's life. Nor must we forget that in all his numerous citations from the Platonic writings he never attributes any passage found in the extant dialogues to any one but Plato. And lastly, we may remark that one or two great writings, such as the Parmenides and the Politicus, which are wholly devoid of Aristotelian (1) credentials may be fairly attributed to Plato, on the ground of (2) length, (3) excellence, and (4) accordance with the general spirit of his writings. Indeed the greater part of the evidence for the genuineness of ancient Greek authors may be summed up under two heads only: (1) excellence; and (2) uniformity of tradition—a kind of evidence, which though in many cases sufficient, is of inferior value.

\par  Proceeding upon these principles we appear to arrive at the conclusion that nineteen-twentieths of all the writings which have ever been ascribed to Plato, are undoubtedly genuine. There is another portion of them, including the Epistles, the Epinomis, the dialogues rejected by the ancients themselves, namely, the Axiochus, De justo, De virtute, Demodocus, Sisyphus, Eryxias, which on grounds, both of internal and external evidence, we are able with equal certainty to reject. But there still remains a small portion of which we are unable to affirm either that they are genuine or spurious. They may have been written in youth, or possibly like the works of some painters, may be partly or wholly the compositions of pupils; or they may have been the writings of some contemporary transferred by accident to the more celebrated name of Plato, or of some Platonist in the next generation who aspired to imitate his master. Not that on grounds either of language or philosophy we should lightly reject them. Some difference of style, or inferiority of execution, or inconsistency of thought, can hardly be considered decisive of their spurious character. For who always does justice to himself, or who writes with equal care at all times? Certainly not Plato, who exhibits the greatest differences in dramatic power, in the formation of sentences, and in the use of words, if his earlier writings are compared with his later ones, say the Protagoras or Phaedrus with the Laws. Or who can be expected to think in the same manner during a period of authorship extending over above fifty years, in an age of great intellectual activity, as well as of political and literary transition? Certainly not Plato, whose earlier writings are separated from his later ones by as wide an interval of philosophical speculation as that which separates his later writings from Aristotle.

\par  The dialogues which have been translated in the first Appendix, and which appear to have the next claim to genuineness among the Platonic writings, are the Lesser Hippias, the Menexenus or Funeral Oration, the First Alcibiades. Of these, the Lesser Hippias and the Funeral Oration are cited by Aristotle; the first in the Metaphysics, the latter in the Rhetoric. Neither of them are expressly attributed to Plato, but in his citation of both of them he seems to be referring to passages in the extant dialogues. From the mention of 'Hippias' in the singular by Aristotle, we may perhaps infer that he was unacquainted with a second dialogue bearing the same name. Moreover, the mere existence of a Greater and Lesser Hippias, and of a First and Second Alcibiades, does to a certain extent throw a doubt upon both of them. Though a very clever and ingenious work, the Lesser Hippias does not appear to contain anything beyond the power of an imitator, who was also a careful student of the earlier Platonic writings, to invent. The motive or leading thought of the dialogue may be detected in Xen. Mem., and there is no similar instance of a 'motive' which is taken from Xenophon in an undoubted dialogue of Plato. On the other hand, the upholders of the genuineness of the dialogue will find in the Hippias a true Socratic spirit; they will compare the Ion as being akin both in subject and treatment; they will urge the authority of Aristotle; and they will detect in the treatment of the Sophist, in the satirical reasoning upon Homer, in the reductio ad absurdum of the doctrine that vice is ignorance, traces of a Platonic authorship. In reference to the last point we are doubtful, as in some of the other dialogues, whether the author is asserting or overthrowing the paradox of Socrates, or merely following the argument 'whither the wind blows.' That no conclusion is arrived at is also in accordance with the character of the earlier dialogues. The resemblances or imitations of the Gorgias, Protagoras, and Euthydemus, which have been observed in the Hippias, cannot with certainty be adduced on either side of the argument. On the whole, more may be said in favour of the genuineness of the Hippias than against it.

\par  The Menexenus or Funeral Oration is cited by Aristotle, and is interesting as supplying an example of the manner in which the orators praised 'the Athenians among the Athenians,' falsifying persons and dates, and casting a veil over the gloomier events of Athenian history. It exhibits an acquaintance with the funeral oration of Thucydides, and was, perhaps, intended to rival that great work. If genuine, the proper place of the Menexenus would be at the end of the Phaedrus. The satirical opening and the concluding words bear a great resemblance to the earlier dialogues; the oration itself is professedly a mimetic work, like the speeches in the Phaedrus, and cannot therefore be tested by a comparison of the other writings of Plato. The funeral oration of Pericles is expressly mentioned in the Phaedrus, and this may have suggested the subject, in the same manner that the Cleitophon appears to be suggested by the slight mention of Cleitophon and his attachment to Thrasymachus in the Republic; and the Theages by the mention of Theages in the Apology and Republic; or as the Second Alcibiades seems to be founded upon the text of Xenophon, Mem. A similar taste for parody appears not only in the Phaedrus, but in the Protagoras, in the Symposium, and to a certain extent in the Parmenides.

\par  To these two doubtful writings of Plato I have added the First Alcibiades, which, of all the disputed dialogues of Plato, has the greatest merit, and is somewhat longer than any of them, though not verified by the testimony of Aristotle, and in many respects at variance with the Symposium in the description of the relations of Socrates and Alcibiades. Like the Lesser Hippias and the Menexenus, it is to be compared to the earlier writings of Plato. The motive of the piece may, perhaps, be found in that passage of the Symposium in which Alcibiades describes himself as self-convicted by the words of Socrates. For the disparaging manner in which Schleiermacher has spoken of this dialogue there seems to be no sufficient foundation. At the same time, the lesson imparted is simple, and the irony more transparent than in the undoubted dialogues of Plato. We know, too, that Alcibiades was a favourite thesis, and that at least five or six dialogues bearing this name passed current in antiquity, and are attributed to contemporaries of Socrates and Plato. (1) In the entire absence of real external evidence (for the catalogues of the Alexandrian librarians cannot be regarded as trustworthy); and (2) in the absence of the highest marks either of poetical or philosophical excellence; and (3) considering that we have express testimony to the existence of contemporary writings bearing the name of Alcibiades, we are compelled to suspend our judgment on the genuineness of the extant dialogue.

\par  Neither at this point, nor at any other, do we propose to draw an absolute line of demarcation between genuine and spurious writings of Plato. They fade off imperceptibly from one class to another. There may have been degrees of genuineness in the dialogues themselves, as there are certainly degrees of evidence by which they are supported. The traditions of the oral discourses both of Socrates and Plato may have formed the basis of semi-Platonic writings; some of them may be of the same mixed character which is apparent in Aristotle and Hippocrates, although the form of them is different. But the writings of Plato, unlike the writings of Aristotle, seem never to have been confused with the writings of his disciples: this was probably due to their definite form, and to their inimitable excellence. The three dialogues which we have offered in the Appendix to the criticism of the reader may be partly spurious and partly genuine; they may be altogether spurious;—that is an alternative which must be frankly admitted. Nor can we maintain of some other dialogues, such as the Parmenides, and the Sophist, and Politicus, that no considerable objection can be urged against them, though greatly overbalanced by the weight (chiefly) of internal evidence in their favour. Nor, on the other hand, can we exclude a bare possibility that some dialogues which are usually rejected, such as the Greater Hippias and the Cleitophon, may be genuine. The nature and object of these semi-Platonic writings require more careful study and more comparison of them with one another, and with forged writings in general, than they have yet received, before we can finally decide on their character. We do not consider them all as genuine until they can be proved to be spurious, as is often maintained and still more often implied in this and similar discussions; but should say of some of them, that their genuineness is neither proven nor disproven until further evidence about them can be adduced. And we are as confident that the Epistles are spurious, as that the Republic, the Timaeus, and the Laws are genuine.

\par  On the whole, not a twentieth part of the writings which pass under the name of Plato, if we exclude the works rejected by the ancients themselves and two or three other plausible inventions, can be fairly doubted by those who are willing to allow that a considerable change and growth may have taken place in his philosophy (see above). That twentieth debatable portion scarcely in any degree affects our judgment of Plato, either as a thinker or a writer, and though suggesting some interesting questions to the scholar and critic, is of little importance to the general reader.

\par 

\par 
\section{
      ALCIBIADES I
    }
\par 
\section{
      INTRODUCTION.
    }
\par  The First Alcibiades is a conversation between Socrates and Alcibiades. Socrates is represented in the character which he attributes to himself in the Apology of a know-nothing who detects the conceit of knowledge in others. The two have met already in the Protagoras and in the Symposium; in the latter dialogue, as in this, the relation between them is that of a lover and his beloved. But the narrative of their loves is told differently in different places; for in the Symposium Alcibiades is depicted as the impassioned but rejected lover; here, as coldly receiving the advances of Socrates, who, for the best of purposes, lies in wait for the aspiring and ambitious youth.

\par  Alcibiades, who is described as a very young man, is about to enter on public life, having an inordinate opinion of himself, and an extravagant ambition. Socrates, 'who knows what is in man,' astonishes him by a revelation of his designs. But has he the knowledge which is necessary for carrying them out? He is going to persuade the Athenians—about what? Not about any particular art, but about politics—when to fight and when to make peace. Now, men should fight and make peace on just grounds, and therefore the question of justice and injustice must enter into peace and war; and he who advises the Athenians must know the difference between them. Does Alcibiades know? If he does, he must either have been taught by some master, or he must have discovered the nature of them himself. If he has had a master, Socrates would like to be informed who he is, that he may go and learn of him also. Alcibiades admits that he has never learned. Then has he enquired for himself? He may have, if he was ever aware of a time when he was ignorant. But he never was ignorant; for when he played with other boys at dice, he charged them with cheating, and this implied a knowledge of just and unjust. According to his own explanation, he had learned of the multitude. Why, he asks, should he not learn of them the nature of justice, as he has learned the Greek language of them? To this Socrates answers, that they can teach Greek, but they cannot teach justice; for they are agreed about the one, but they are not agreed about the other: and therefore Alcibiades, who has admitted that if he knows he must either have learned from a master or have discovered for himself the nature of justice, is convicted out of his own mouth.

\par  Alcibiades rejoins, that the Athenians debate not about what is just, but about what is expedient; and he asserts that the two principles of justice and expediency are opposed. Socrates, by a series of questions, compels him to admit that the just and the expedient coincide. Alcibiades is thus reduced to the humiliating conclusion that he knows nothing of politics, even if, as he says, they are concerned with the expedient.

\par  However, he is no worse than other Athenian statesmen; and he will not need training, for others are as ignorant as he is. He is reminded that he has to contend, not only with his own countrymen, but with their enemies—with the Spartan kings and with the great king of Persia; and he can only attain this higher aim of ambition by the assistance of Socrates. Not that Socrates himself professes to have attained the truth, but the questions which he asks bring others to a knowledge of themselves, and this is the first step in the practice of virtue.

\par  The dialogue continues:—We wish to become as good as possible. But to be good in what? Alcibiades replies—'Good in transacting business.' But what business? 'The business of the most intelligent men at Athens.' The cobbler is intelligent in shoemaking, and is therefore good in that; he is not intelligent, and therefore not good, in weaving. Is he good in the sense which Alcibiades means, who is also bad? 'I mean,' replies Alcibiades, 'the man who is able to command in the city.' But to command what—horses or men? and if men, under what circumstances? 'I mean to say, that he is able to command men living in social and political relations.' And what is their aim? 'The better preservation of the city.' But when is a city better? 'When there is unanimity, such as exists between husband and wife.' Then, when husbands and wives perform their own special duties, there can be no unanimity between them; nor can a city be well ordered when each citizen does his own work only. Alcibiades, having stated first that goodness consists in the unanimity of the citizens, and then in each of them doing his own separate work, is brought to the required point of self-contradiction, leading him to confess his own ignorance.

\par  But he is not too old to learn, and may still arrive at the truth, if he is willing to be cross-examined by Socrates. He must know himself; that is to say, not his body, or the things of the body, but his mind, or truer self. The physician knows the body, and the tradesman knows his own business, but they do not necessarily know themselves. Self-knowledge can be obtained only by looking into the mind and virtue of the soul, which is the diviner part of a man, as we see our own image in another's eye. And if we do not know ourselves, we cannot know what belongs to ourselves or belongs to others, and are unfit to take a part in political affairs. Both for the sake of the individual and of the state, we ought to aim at justice and temperance, not at wealth or power. The evil and unjust should have no power,—they should be the slaves of better men than themselves. None but the virtuous are deserving of freedom.

\par  And are you, Alcibiades, a freeman? 'I feel that I am not; but I hope, Socrates, that by your aid I may become free, and from this day forward I will never leave you.'

\par  The Alcibiades has several points of resemblance to the undoubted dialogues of Plato. The process of interrogation is of the same kind with that which Socrates practises upon the youthful Cleinias in the Euthydemus; and he characteristically attributes to Alcibiades the answers which he has elicited from him. The definition of good is narrowed by successive questions, and virtue is shown to be identical with knowledge. Here, as elsewhere, Socrates awakens the consciousness not of sin but of ignorance. Self-humiliation is the first step to knowledge, even of the commonest things. No man knows how ignorant he is, and no man can arrive at virtue and wisdom who has not once in his life, at least, been convicted of error. The process by which the soul is elevated is not unlike that which religious writers describe under the name of 'conversion,' if we substitute the sense of ignorance for the consciousness of sin.

\par  In some respects the dialogue differs from any other Platonic composition. The aim is more directly ethical and hortatory; the process by which the antagonist is undermined is simpler than in other Platonic writings, and the conclusion more decided. There is a good deal of humour in the manner in which the pride of Alcibiades, and of the Greeks generally, is supposed to be taken down by the Spartan and Persian queens; and the dialogue has considerable dialectical merit. But we have a difficulty in supposing that the same writer, who has given so profound and complex a notion of the characters both of Alcibiades and Socrates in the Symposium, should have treated them in so thin and superficial a manner in the Alcibiades, or that he would have ascribed to the ironical Socrates the rather unmeaning boast that Alcibiades could not attain the objects of his ambition without his help; or that he should have imagined that a mighty nature like his could have been reformed by a few not very conclusive words of Socrates. For the arguments by which Alcibiades is reformed are not convincing; the writer of the dialogue, whoever he was, arrives at his idealism by crooked and tortuous paths, in which many pitfalls are concealed. The anachronism of making Alcibiades about twenty years old during the life of his uncle, Pericles, may be noted; and the repetition of the favourite observation, which occurs also in the Laches and Protagoras, that great Athenian statesmen, like Pericles, failed in the education of their sons. There is none of the undoubted dialogues of Plato in which there is so little dramatic verisimilitude.

\par  ALCIBIADES I

\par  by

\par  Plato (see Appendix I above)

\par  Translated by Benjamin Jowett
 
\par \textbf{SOCRATES}
\par   I dare say that you may be surprised to find, O son of Cleinias, that I, who am your first lover, not having spoken to you for many years, when the rest of the world were wearying you with their attentions, am the last of your lovers who still speaks to you. The cause of my silence has been that I was hindered by a power more than human, of which I will some day explain to you the nature; this impediment has now been removed; I therefore here present myself before you, and I greatly hope that no similar hindrance will again occur. Meanwhile, I have observed that your pride has been too much for the pride of your admirers; they were numerous and high-spirited, but they have all run away, overpowered by your superior force of character; not one of them remains. And I want you to understand the reason why you have been too much for them. You think that you have no need of them or of any other man, for you have great possessions and lack nothing, beginning with the body, and ending with the soul. In the first place, you say to yourself that you are the fairest and tallest of the citizens, and this every one who has eyes may see to be true; in the second place, that you are among the noblest of them, highly connected both on the father's and the mother's side, and sprung from one of the most distinguished families in your own state, which is the greatest in Hellas, and having many friends and kinsmen of the best sort, who can assist you when in need; and there is one potent relative, who is more to you than all the rest, Pericles the son of Xanthippus, whom your father left guardian of you, and of your brother, and who can do as he pleases not only in this city, but in all Hellas, and among many and mighty barbarous nations. Moreover, you are rich; but I must say that you value yourself least of all upon your possessions. And all these things have lifted you up; you have overcome your lovers, and they have acknowledged that you were too much for them. Have you not remarked their absence? And now I know that you wonder why I, unlike the rest of them, have not gone away, and what can be my motive in remaining.

\par \textbf{ALCIBIADES}
\par   Perhaps, Socrates, you are not aware that I was just going to ask you the very same question—What do you want? And what is your motive in annoying me, and always, wherever I am, making a point of coming? (Compare Symp.) I do really wonder what you mean, and should greatly like to know.

\par \textbf{SOCRATES}
\par   Then if, as you say, you desire to know, I suppose that you will be willing to hear, and I may consider myself to be speaking to an auditor who will remain, and will not run away?

\par \textbf{ALCIBIADES}
\par   Certainly, let me hear.

\par \textbf{SOCRATES}
\par   You had better be careful, for I may very likely be as unwilling to end as I have hitherto been to begin.

\par \textbf{ALCIBIADES}
\par   Proceed, my good man, and I will listen.

\par \textbf{SOCRATES}
\par   I will proceed; and, although no lover likes to speak with one who has no feeling of love in him (compare Symp. ), I will make an effort, and tell you what I meant:  My love, Alcibiades, which I hardly like to confess, would long ago have passed away, as I flatter myself, if I saw you loving your good things, or thinking that you ought to pass life in the enjoyment of them. But I shall reveal other thoughts of yours, which you keep to yourself; whereby you will know that I have always had my eye on you. Suppose that at this moment some God came to you and said:  Alcibiades, will you live as you are, or die in an instant if you are forbidden to make any further acquisition?—I verily believe that you would choose death. And I will tell you the hope in which you are at present living:  Before many days have elapsed, you think that you will come before the Athenian assembly, and will prove to them that you are more worthy of honour than Pericles, or any other man that ever lived, and having proved this, you will have the greatest power in the state. When you have gained the greatest power among us, you will go on to other Hellenic states, and not only to Hellenes, but to all the barbarians who inhabit the same continent with us. And if the God were then to say to you again:  Here in Europe is to be your seat of empire, and you must not cross over into Asia or meddle with Asiatic affairs, I do not believe that you would choose to live upon these terms; but the world, as I may say, must be filled with your power and name—no man less than Cyrus and Xerxes is of any account with you. Such I know to be your hopes—I am not guessing only—and very likely you, who know that I am speaking the truth, will reply, Well, Socrates, but what have my hopes to do with the explanation which you promised of your unwillingness to leave me? And that is what I am now going to tell you, sweet son of Cleinias and Dinomache. The explanation is, that all these designs of yours cannot be accomplished by you without my help; so great is the power which I believe myself to have over you and your concerns; and this I conceive to be the reason why the God has hitherto forbidden me to converse with you, and I have been long expecting his permission. For, as you hope to prove your own great value to the state, and having proved it, to attain at once to absolute power, so do I indulge a hope that I shall be the supreme power over you, if I am able to prove my own great value to you, and to show you that neither guardian, nor kinsman, nor any one is able to deliver into your hands the power which you desire, but I only, God being my helper. When you were young (compare Symp.) and your hopes were not yet matured, I should have wasted my time, and therefore, as I conceive, the God forbade me to converse with you; but now, having his permission, I will speak, for now you will listen to me.

\par \textbf{ALCIBIADES}
\par   Your silence, Socrates, was always a surprise to me. I never could understand why you followed me about, and now that you have begun to speak again, I am still more amazed. Whether I think all this or not, is a matter about which you seem to have already made up your mind, and therefore my denial will have no effect upon you. But granting, if I must, that you have perfectly divined my purposes, why is your assistance necessary to the attainment of them? Can you tell me why?

\par \textbf{SOCRATES}
\par   You want to know whether I can make a long speech, such as you are in the habit of hearing; but that is not my way. I think, however, that I can prove to you the truth of what I am saying, if you will grant me one little favour.

\par \textbf{ALCIBIADES}
\par   Yes, if the favour which you mean be not a troublesome one.

\par \textbf{SOCRATES}
\par   Will you be troubled at having questions to answer?

\par \textbf{ALCIBIADES}
\par   Not at all.

\par \textbf{SOCRATES}
\par   Then please to answer.

\par \textbf{ALCIBIADES}
\par   Ask me.

\par \textbf{SOCRATES}
\par   Have you not the intention which I attribute to you?

\par \textbf{ALCIBIADES}
\par   I will grant anything you like, in the hope of hearing what more you have to say.

\par \textbf{SOCRATES}
\par   You do, then, mean, as I was saying, to come forward in a little while in the character of an adviser of the Athenians? And suppose that when you are ascending the bema, I pull you by the sleeve and say, Alcibiades, you are getting up to advise the Athenians—do you know the matter about which they are going to deliberate, better than they?—How would you answer?

\par \textbf{ALCIBIADES}
\par   I should reply, that I was going to advise them about a matter which I do know better than they.

\par \textbf{SOCRATES}
\par   Then you are a good adviser about the things which you know?

\par \textbf{ALCIBIADES}
\par   Certainly.

\par \textbf{SOCRATES}
\par   And do you know anything but what you have learned of others, or found out yourself?

\par \textbf{ALCIBIADES}
\par   That is all.

\par \textbf{SOCRATES}
\par   And would you have ever learned or discovered anything, if you had not been willing either to learn of others or to examine yourself?

\par \textbf{ALCIBIADES}
\par   I should not.

\par \textbf{SOCRATES}
\par   And would you have been willing to learn or to examine what you supposed that you knew?

\par \textbf{ALCIBIADES}
\par   Certainly not.

\par \textbf{SOCRATES}
\par   Then there was a time when you thought that you did not know what you are now supposed to know?

\par \textbf{ALCIBIADES}
\par   Certainly.

\par \textbf{SOCRATES}
\par   I think that I know tolerably well the extent of your acquirements; and you must tell me if I forget any of them:  according to my recollection, you learned the arts of writing, of playing on the lyre, and of wrestling; the flute you never would learn; this is the sum of your accomplishments, unless there were some which you acquired in secret; and I think that secrecy was hardly possible, as you could not have come out of your door, either by day or night, without my seeing you.

\par \textbf{ALCIBIADES}
\par   Yes, that was the whole of my schooling.

\par \textbf{SOCRATES}
\par   And are you going to get up in the Athenian assembly, and give them advice about writing?

\par \textbf{ALCIBIADES}
\par   No, indeed.

\par \textbf{SOCRATES}
\par   Or about the touch of the lyre?

\par \textbf{ALCIBIADES}
\par   Certainly not.

\par \textbf{SOCRATES}
\par   And they are not in the habit of deliberating about wrestling, in the assembly?

\par \textbf{ALCIBIADES}
\par   Hardly.

\par \textbf{SOCRATES}
\par   Then what are the deliberations in which you propose to advise them? Surely not about building?

\par \textbf{ALCIBIADES}
\par   No.

\par \textbf{SOCRATES}
\par   For the builder will advise better than you will about that?

\par \textbf{ALCIBIADES}
\par   He will.

\par \textbf{SOCRATES}
\par   Nor about divination?

\par \textbf{ALCIBIADES}
\par   No.

\par \textbf{SOCRATES}
\par   About that again the diviner will advise better than you will?

\par \textbf{ALCIBIADES}
\par   True.

\par \textbf{SOCRATES}
\par   Whether he be little or great, good or ill-looking, noble or ignoble—makes no difference.

\par \textbf{ALCIBIADES}
\par   Certainly not.

\par \textbf{SOCRATES}
\par   A man is a good adviser about anything, not because he has riches, but because he has knowledge?

\par \textbf{ALCIBIADES}
\par   Assuredly.

\par \textbf{SOCRATES}
\par   Whether their counsellor is rich or poor, is not a matter which will make any difference to the Athenians when they are deliberating about the health of the citizens; they only require that he should be a physician.

\par \textbf{ALCIBIADES}
\par   Of course.

\par \textbf{SOCRATES}
\par   Then what will be the subject of deliberation about which you will be justified in getting up and advising them?

\par \textbf{ALCIBIADES}
\par   About their own concerns, Socrates.

\par \textbf{SOCRATES}
\par   You mean about shipbuilding, for example, when the question is what sort of ships they ought to build?

\par \textbf{ALCIBIADES}
\par   No, I should not advise them about that.

\par \textbf{SOCRATES}
\par   I suppose, because you do not understand shipbuilding: —is that the reason?

\par \textbf{ALCIBIADES}
\par   It is.

\par \textbf{SOCRATES}
\par   Then about what concerns of theirs will you advise them?

\par \textbf{ALCIBIADES}
\par   About war, Socrates, or about peace, or about any other concerns of the state.

\par \textbf{SOCRATES}
\par   You mean, when they deliberate with whom they ought to make peace, and with whom they ought to go to war, and in what manner?

\par \textbf{ALCIBIADES}
\par   Yes.

\par \textbf{SOCRATES}
\par   And they ought to go to war with those against whom it is better to go to war?

\par \textbf{ALCIBIADES}
\par   Yes.

\par \textbf{SOCRATES}
\par   And when it is better?

\par \textbf{ALCIBIADES}
\par   Certainly.

\par \textbf{SOCRATES}
\par   And for as long a time as is better?

\par \textbf{ALCIBIADES}
\par   Yes.

\par \textbf{SOCRATES}
\par   But suppose the Athenians to deliberate with whom they ought to close in wrestling, and whom they should grasp by the hand, would you, or the master of gymnastics, be a better adviser of them?

\par \textbf{ALCIBIADES}
\par   Clearly, the master of gymnastics.

\par \textbf{SOCRATES}
\par   And can you tell me on what grounds the master of gymnastics would decide, with whom they ought or ought not to close, and when and how? To take an instance:  Would he not say that they should wrestle with those against whom it is best to wrestle?

\par \textbf{ALCIBIADES}
\par   Yes.

\par \textbf{SOCRATES}
\par   And as much as is best?

\par \textbf{ALCIBIADES}
\par   Certainly.

\par \textbf{SOCRATES}
\par   And at such times as are best?

\par \textbf{ALCIBIADES}
\par   Yes.

\par \textbf{SOCRATES}
\par   Again; you sometimes accompany the lyre with the song and dance?

\par \textbf{ALCIBIADES}
\par   Yes.

\par \textbf{SOCRATES}
\par   When it is well to do so?

\par \textbf{ALCIBIADES}
\par   Yes.

\par \textbf{SOCRATES}
\par   And as much as is well?

\par \textbf{ALCIBIADES}
\par   Just so.

\par \textbf{SOCRATES}
\par   And as you speak of an excellence or art of the best in wrestling, and of an excellence in playing the lyre, I wish you would tell me what this latter is;—the excellence of wrestling I call gymnastic, and I want to know what you call the other.

\par \textbf{ALCIBIADES}
\par   I do not understand you.

\par \textbf{SOCRATES}
\par   Then try to do as I do; for the answer which I gave is universally right, and when I say right, I mean according to rule.

\par \textbf{ALCIBIADES}
\par   Yes.

\par \textbf{SOCRATES}
\par   And was not the art of which I spoke gymnastic?

\par \textbf{ALCIBIADES}
\par   Certainly.

\par \textbf{SOCRATES}
\par   And I called the excellence in wrestling gymnastic?

\par \textbf{ALCIBIADES}
\par   You did.

\par \textbf{SOCRATES}
\par   And I was right?

\par \textbf{ALCIBIADES}
\par   I think that you were.

\par \textbf{SOCRATES}
\par   Well, now,—for you should learn to argue prettily—let me ask you in return to tell me, first, what is that art of which playing and singing, and stepping properly in the dance, are parts,—what is the name of the whole? I think that by this time you must be able to tell.

\par \textbf{ALCIBIADES}
\par   Indeed I cannot.

\par \textbf{SOCRATES}
\par   Then let me put the matter in another way:  what do you call the Goddesses who are the patronesses of art?

\par \textbf{ALCIBIADES}
\par   The Muses do you mean, Socrates?

\par \textbf{SOCRATES}
\par   Yes, I do; and what is the name of the art which is called after them?

\par \textbf{ALCIBIADES}
\par   I suppose that you mean music.

\par \textbf{SOCRATES}
\par   Yes, that is my meaning; and what is the excellence of the art of music, as I told you truly that the excellence of wrestling was gymnastic—what is the excellence of music—to be what?

\par \textbf{ALCIBIADES}
\par   To be musical, I suppose.

\par \textbf{SOCRATES}
\par   Very good; and now please to tell me what is the excellence of war and peace; as the more musical was the more excellent, or the more gymnastical was the more excellent, tell me, what name do you give to the more excellent in war and peace?

\par \textbf{ALCIBIADES}
\par   But I really cannot tell you.

\par \textbf{SOCRATES}
\par   But if you were offering advice to another and said to him—This food is better than that, at this time and in this quantity, and he said to you—What do you mean, Alcibiades, by the word 'better'? you would have no difficulty in replying that you meant 'more wholesome,' although you do not profess to be a physician:  and when the subject is one of which you profess to have knowledge, and about which you are ready to get up and advise as if you knew, are you not ashamed, when you are asked, not to be able to answer the question? Is it not disgraceful?

\par \textbf{ALCIBIADES}
\par   Very.

\par \textbf{SOCRATES}
\par   Well, then, consider and try to explain what is the meaning of 'better,' in the matter of making peace and going to war with those against whom you ought to go to war? To what does the word refer?

\par \textbf{ALCIBIADES}
\par   I am thinking, and I cannot tell.

\par \textbf{SOCRATES}
\par   But you surely know what are the charges which we bring against one another, when we arrive at the point of making war, and what name we give them?

\par \textbf{ALCIBIADES}
\par   Yes, certainly; we say that deceit or violence has been employed, or that we have been defrauded.

\par \textbf{SOCRATES}
\par   And how does this happen? Will you tell me how? For there may be a difference in the manner.

\par \textbf{ALCIBIADES}
\par   Do you mean by 'how,' Socrates, whether we suffered these things justly or unjustly?

\par \textbf{SOCRATES}
\par   Exactly.

\par \textbf{ALCIBIADES}
\par   There can be no greater difference than between just and unjust.

\par \textbf{SOCRATES}
\par   And would you advise the Athenians to go to war with the just or with the unjust?

\par \textbf{ALCIBIADES}
\par   That is an awkward question; for certainly, even if a person did intend to go to war with the just, he would not admit that they were just.

\par \textbf{SOCRATES}
\par   He would not go to war, because it would be unlawful?

\par \textbf{ALCIBIADES}
\par   Neither lawful nor honourable.

\par \textbf{SOCRATES}
\par   Then you, too, would address them on principles of justice?

\par \textbf{ALCIBIADES}
\par   Certainly.

\par \textbf{SOCRATES}
\par   What, then, is justice but that better, of which I spoke, in going to war or not going to war with those against whom we ought or ought not, and when we ought or ought not to go to war?

\par \textbf{ALCIBIADES}
\par   Clearly.

\par \textbf{SOCRATES}
\par   But how is this, friend Alcibiades? Have you forgotten that you do not know this, or have you been to the schoolmaster without my knowledge, and has he taught you to discern the just from the unjust? Who is he? I wish you would tell me, that I may go and learn of him—you shall introduce me.

\par \textbf{ALCIBIADES}
\par   You are mocking, Socrates.

\par \textbf{SOCRATES}
\par   No, indeed; I most solemnly declare to you by Zeus, who is the God of our common friendship, and whom I never will forswear, that I am not; tell me, then, who this instructor is, if he exists.

\par \textbf{ALCIBIADES}
\par   But, perhaps, he does not exist; may I not have acquired the knowledge of just and unjust in some other way?

\par \textbf{SOCRATES}
\par   Yes; if you have discovered them.

\par \textbf{ALCIBIADES}
\par   But do you not think that I could discover them?

\par \textbf{SOCRATES}
\par   I am sure that you might, if you enquired about them.

\par \textbf{ALCIBIADES}
\par   And do you not think that I would enquire?

\par \textbf{SOCRATES}
\par   Yes; if you thought that you did not know them.

\par \textbf{ALCIBIADES}
\par   And was there not a time when I did so think?

\par \textbf{SOCRATES}
\par   Very good; and can you tell me how long it is since you thought that you did not know the nature of the just and the unjust? What do you say to a year ago? Were you then in a state of conscious ignorance and enquiry? Or did you think that you knew? And please to answer truly, that our discussion may not be in vain.

\par \textbf{ALCIBIADES}
\par   Well, I thought that I knew.

\par \textbf{SOCRATES}
\par   And two years ago, and three years ago, and four years ago, you knew all the same?

\par \textbf{ALCIBIADES}
\par   I did.

\par \textbf{SOCRATES}
\par   And more than four years ago you were a child—were you not?

\par \textbf{ALCIBIADES}
\par   Yes.

\par \textbf{SOCRATES}
\par   And then I am quite sure that you thought you knew.

\par \textbf{ALCIBIADES}
\par   Why are you so sure?

\par \textbf{SOCRATES}
\par   Because I often heard you when a child, in your teacher's house, or elsewhere, playing at dice or some other game with the boys, not hesitating at all about the nature of the just and unjust; but very confident—crying and shouting that one of the boys was a rogue and a cheat, and had been cheating. Is it not true?

\par \textbf{ALCIBIADES}
\par   But what was I to do, Socrates, when anybody cheated me?

\par \textbf{SOCRATES}
\par   And how can you say, 'What was I to do'? if at the time you did not know whether you were wronged or not?

\par \textbf{ALCIBIADES}
\par   To be sure I knew; I was quite aware that I was being cheated.

\par \textbf{SOCRATES}
\par   Then you suppose yourself even when a child to have known the nature of just and unjust?

\par \textbf{ALCIBIADES}
\par   Certainly; and I did know then.

\par \textbf{SOCRATES}
\par   And when did you discover them—not, surely, at the time when you thought that you knew them?

\par \textbf{ALCIBIADES}
\par   Certainly not.

\par \textbf{SOCRATES}
\par   And when did you think that you were ignorant—if you consider, you will find that there never was such a time?

\par \textbf{ALCIBIADES}
\par   Really, Socrates, I cannot say.

\par \textbf{SOCRATES}
\par   Then you did not learn them by discovering them?

\par \textbf{ALCIBIADES}
\par   Clearly not.

\par \textbf{SOCRATES}
\par   But just before you said that you did not know them by learning; now, if you have neither discovered nor learned them, how and whence do you come to know them?

\par \textbf{ALCIBIADES}
\par   I suppose that I was mistaken in saying that I knew them through my own discovery of them; whereas, in truth, I learned them in the same way that other people learn.

\par \textbf{SOCRATES}
\par   So you said before, and I must again ask, of whom? Do tell me.

\par \textbf{ALCIBIADES}
\par   Of the many.

\par \textbf{SOCRATES}
\par   Do you take refuge in them? I cannot say much for your teachers.

\par \textbf{ALCIBIADES}
\par   Why, are they not able to teach?

\par \textbf{SOCRATES}
\par   They could not teach you how to play at draughts, which you would acknowledge (would you not) to be a much smaller matter than justice?

\par \textbf{ALCIBIADES}
\par   Yes.

\par \textbf{SOCRATES}
\par   And can they teach the better who are unable to teach the worse?

\par \textbf{ALCIBIADES}
\par   I think that they can; at any rate, they can teach many far better things than to play at draughts.

\par \textbf{SOCRATES}
\par   What things?

\par \textbf{ALCIBIADES}
\par   Why, for example, I learned to speak Greek of them, and I cannot say who was my teacher, or to whom I am to attribute my knowledge of Greek, if not to those good-for-nothing teachers, as you call them.

\par \textbf{SOCRATES}
\par   Why, yes, my friend; and the many are good enough teachers of Greek, and some of their instructions in that line may be justly praised.

\par \textbf{ALCIBIADES}
\par   Why is that?

\par \textbf{SOCRATES}
\par   Why, because they have the qualities which good teachers ought to have.

\par \textbf{ALCIBIADES}
\par   What qualities?

\par \textbf{SOCRATES}
\par   Why, you know that knowledge is the first qualification of any teacher?

\par \textbf{ALCIBIADES}
\par   Certainly.

\par \textbf{SOCRATES}
\par   And if they know, they must agree together and not differ?

\par \textbf{ALCIBIADES}
\par   Yes.

\par \textbf{SOCRATES}
\par   And would you say that they knew the things about which they differ?

\par \textbf{ALCIBIADES}
\par   No.

\par \textbf{SOCRATES}
\par   Then how can they teach them?

\par \textbf{ALCIBIADES}
\par   They cannot.

\par \textbf{SOCRATES}
\par   Well, but do you imagine that the many would differ about the nature of wood and stone? are they not agreed if you ask them what they are? and do they not run to fetch the same thing, when they want a piece of wood or a stone? And so in similar cases, which I suspect to be pretty nearly all that you mean by speaking Greek.

\par \textbf{ALCIBIADES}
\par   True.

\par \textbf{SOCRATES}
\par   These, as we were saying, are matters about which they are agreed with one another and with themselves; both individuals and states use the same words about them; they do not use some one word and some another.

\par \textbf{ALCIBIADES}
\par   They do not.

\par \textbf{SOCRATES}
\par   Then they may be expected to be good teachers of these things?

\par \textbf{ALCIBIADES}
\par   Yes.

\par \textbf{SOCRATES}
\par   And if we want to instruct any one in them, we shall be right in sending him to be taught by our friends the many?

\par \textbf{ALCIBIADES}
\par   Very true.

\par \textbf{SOCRATES}
\par   But if we wanted further to know not only which are men and which are horses, but which men or horses have powers of running, would the many still be able to inform us?

\par \textbf{ALCIBIADES}
\par   Certainly not.

\par \textbf{SOCRATES}
\par   And you have a sufficient proof that they do not know these things and are not the best teachers of them, inasmuch as they are never agreed about them?

\par \textbf{ALCIBIADES}
\par   Yes.

\par \textbf{SOCRATES}
\par   And suppose that we wanted to know not only what men are like, but what healthy or diseased men are like—would the many be able to teach us?

\par \textbf{ALCIBIADES}
\par   They would not.

\par \textbf{SOCRATES}
\par   And you would have a proof that they were bad teachers of these matters, if you saw them at variance?

\par \textbf{ALCIBIADES}
\par   I should.

\par \textbf{SOCRATES}
\par   Well, but are the many agreed with themselves, or with one another, about the justice or injustice of men and things?

\par \textbf{ALCIBIADES}
\par   Assuredly not, Socrates.

\par \textbf{SOCRATES}
\par   There is no subject about which they are more at variance?

\par \textbf{ALCIBIADES}
\par   None.

\par \textbf{SOCRATES}
\par   I do not suppose that you ever saw or heard of men quarrelling over the principles of health and disease to such an extent as to go to war and kill one another for the sake of them?

\par \textbf{ALCIBIADES}
\par   No indeed.

\par \textbf{SOCRATES}
\par   But of the quarrels about justice and injustice, even if you have never seen them, you have certainly heard from many people, including Homer; for you have heard of the Iliad and Odyssey?

\par \textbf{ALCIBIADES}
\par   To be sure, Socrates.

\par \textbf{SOCRATES}
\par   A difference of just and unjust is the argument of those poems?

\par \textbf{ALCIBIADES}
\par   True.

\par \textbf{SOCRATES}
\par   Which difference caused all the wars and deaths of Trojans and Achaeans, and the deaths of the suitors of Penelope in their quarrel with Odysseus.

\par \textbf{ALCIBIADES}
\par   Very true.

\par \textbf{SOCRATES}
\par   And when the Athenians and Lacedaemonians and Boeotians fell at Tanagra, and afterwards in the battle of Coronea, at which your father Cleinias met his end, the question was one of justice—this was the sole cause of the battles, and of their deaths.

\par \textbf{ALCIBIADES}
\par   Very true.

\par \textbf{SOCRATES}
\par   But can they be said to understand that about which they are quarrelling to the death?

\par \textbf{ALCIBIADES}
\par   Clearly not.

\par \textbf{SOCRATES}
\par   And yet those whom you thus allow to be ignorant are the teachers to whom you are appealing.

\par \textbf{ALCIBIADES}
\par   Very true.

\par \textbf{SOCRATES}
\par   But how are you ever likely to know the nature of justice and injustice, about which you are so perplexed, if you have neither learned them of others nor discovered them yourself?

\par \textbf{ALCIBIADES}
\par   From what you say, I suppose not.

\par \textbf{SOCRATES}
\par   See, again, how inaccurately you speak, Alcibiades!

\par \textbf{ALCIBIADES}
\par   In what respect?

\par \textbf{SOCRATES}
\par   In saying that I say so.

\par \textbf{ALCIBIADES}
\par   Why, did you not say that I know nothing of the just and unjust?

\par \textbf{SOCRATES}
\par   No; I did not.

\par \textbf{ALCIBIADES}
\par   Did I, then?

\par \textbf{SOCRATES}
\par   Yes.

\par \textbf{ALCIBIADES}
\par   How was that?

\par \textbf{SOCRATES}
\par   Let me explain. Suppose I were to ask you which is the greater number, two or one; you would reply 'two'?

\par \textbf{ALCIBIADES}
\par   I should.

\par \textbf{SOCRATES}
\par   And by how much greater?

\par \textbf{ALCIBIADES}
\par   By one.

\par \textbf{SOCRATES}
\par   Which of us now says that two is more than one?

\par \textbf{ALCIBIADES}
\par   I do.

\par \textbf{SOCRATES}
\par   Did not I ask, and you answer the question?

\par \textbf{ALCIBIADES}
\par   Yes.

\par \textbf{SOCRATES}
\par   Then who is speaking? I who put the question, or you who answer me?

\par \textbf{ALCIBIADES}
\par   I am.

\par \textbf{SOCRATES}
\par   Or suppose that I ask and you tell me the letters which make up the name Socrates, which of us is the speaker?

\par \textbf{ALCIBIADES}
\par   I am.

\par \textbf{SOCRATES}
\par   Now let us put the case generally:  whenever there is a question and answer, who is the speaker,—the questioner or the answerer?

\par \textbf{ALCIBIADES}
\par   I should say, Socrates, that the answerer was the speaker.

\par \textbf{SOCRATES}
\par   And have I not been the questioner all through?

\par \textbf{ALCIBIADES}
\par   Yes.

\par \textbf{SOCRATES}
\par   And you the answerer?

\par \textbf{ALCIBIADES}
\par   Just so.

\par \textbf{SOCRATES}
\par   Which of us, then, was the speaker?

\par \textbf{ALCIBIADES}
\par   The inference is, Socrates, that I was the speaker.

\par \textbf{SOCRATES}
\par   Did not some one say that Alcibiades, the fair son of Cleinias, not understanding about just and unjust, but thinking that he did understand, was going to the assembly to advise the Athenians about what he did not know? Was not that said?

\par \textbf{ALCIBIADES}
\par   Very true.

\par \textbf{SOCRATES}
\par   Then, Alcibiades, the result may be expressed in the language of Euripides. I think that you have heard all this 'from yourself, and not from me'; nor did I say this, which you erroneously attribute to me, but you yourself, and what you said was very true. For indeed, my dear fellow, the design which you meditate of teaching what you do not know, and have not taken any pains to learn, is downright insanity.

\par \textbf{ALCIBIADES}
\par   But, Socrates, I think that the Athenians and the rest of the Hellenes do not often advise as to the more just or unjust; for they see no difficulty in them, and therefore they leave them, and consider which course of action will be most expedient; for there is a difference between justice and expediency. Many persons have done great wrong and profited by their injustice; others have done rightly and come to no good.

\par \textbf{SOCRATES}
\par   Well, but granting that the just and the expedient are ever so much opposed, you surely do not imagine that you know what is expedient for mankind, or why a thing is expedient?

\par \textbf{ALCIBIADES}
\par   Why not, Socrates?—But I am not going to be asked again from whom I learned, or when I made the discovery.

\par \textbf{SOCRATES}
\par   What a way you have! When you make a mistake which might be refuted by a previous argument, you insist on having a new and different refutation; the old argument is a worn-our garment which you will no longer put on, but some one must produce another which is clean and new. Now I shall disregard this move of yours, and shall ask over again,—Where did you learn and how do you know the nature of the expedient, and who is your teacher? All this I comprehend in a single question, and now you will manifestly be in the old difficulty, and will not be able to show that you know the expedient, either because you learned or because you discovered it yourself. But, as I perceive that you are dainty, and dislike the taste of a stale argument, I will enquire no further into your knowledge of what is expedient or what is not expedient for the Athenian people, and simply request you to say why you do not explain whether justice and expediency are the same or different? And if you like you may examine me as I have examined you, or, if you would rather, you may carry on the discussion by yourself.

\par \textbf{ALCIBIADES}
\par   But I am not certain, Socrates, whether I shall be able to discuss the matter with you.

\par \textbf{SOCRATES}
\par   Then imagine, my dear fellow, that I am the demus and the ecclesia; for in the ecclesia, too, you will have to persuade men individually.

\par \textbf{ALCIBIADES}
\par   Yes.

\par \textbf{SOCRATES}
\par   And is not the same person able to persuade one individual singly and many individuals of the things which he knows? The grammarian, for example, can persuade one and he can persuade many about letters.

\par \textbf{ALCIBIADES}
\par   True.

\par \textbf{SOCRATES}
\par   And about number, will not the same person persuade one and persuade many?

\par \textbf{ALCIBIADES}
\par   Yes.

\par \textbf{SOCRATES}
\par   And this will be he who knows number, or the arithmetician?

\par \textbf{ALCIBIADES}
\par   Quite true.

\par \textbf{SOCRATES}
\par   And cannot you persuade one man about that of which you can persuade many?

\par \textbf{ALCIBIADES}
\par   I suppose so.

\par \textbf{SOCRATES}
\par   And that of which you can persuade either is clearly what you know?

\par \textbf{ALCIBIADES}
\par   Yes.

\par \textbf{SOCRATES}
\par   And the only difference between one who argues as we are doing, and the orator who is addressing an assembly, is that the one seeks to persuade a number, and the other an individual, of the same things.

\par \textbf{ALCIBIADES}
\par   I suppose so.

\par \textbf{SOCRATES}
\par   Well, then, since the same person who can persuade a multitude can persuade individuals, try conclusions upon me, and prove to me that the just is not always expedient.

\par \textbf{ALCIBIADES}
\par   You take liberties, Socrates.

\par \textbf{SOCRATES}
\par   I shall take the liberty of proving to you the opposite of that which you will not prove to me.

\par \textbf{ALCIBIADES}
\par   Proceed.

\par \textbf{SOCRATES}
\par   Answer my questions—that is all.

\par \textbf{ALCIBIADES}
\par   Nay, I should like you to be the speaker.

\par \textbf{SOCRATES}
\par   What, do you not wish to be persuaded?

\par \textbf{ALCIBIADES}
\par   Certainly I do.

\par \textbf{SOCRATES}
\par   And can you be persuaded better than out of your own mouth?

\par \textbf{ALCIBIADES}
\par   I think not.

\par \textbf{SOCRATES}
\par   Then you shall answer; and if you do not hear the words, that the just is the expedient, coming from your own lips, never believe another man again.

\par \textbf{ALCIBIADES}
\par   I won't; but answer I will, for I do not see how I can come to any harm.

\par \textbf{SOCRATES}
\par   A true prophecy! Let me begin then by enquiring of you whether you allow that the just is sometimes expedient and sometimes not?

\par \textbf{ALCIBIADES}
\par   Yes.

\par \textbf{SOCRATES}
\par   And sometimes honourable and sometimes not?

\par \textbf{ALCIBIADES}
\par   What do you mean?

\par \textbf{SOCRATES}
\par   I am asking if you ever knew any one who did what was dishonourable and yet just?

\par \textbf{ALCIBIADES}
\par   Never.

\par \textbf{SOCRATES}
\par   All just things are honourable?

\par \textbf{ALCIBIADES}
\par   Yes.

\par \textbf{SOCRATES}
\par   And are honourable things sometimes good and sometimes not good, or are they always good?

\par \textbf{ALCIBIADES}
\par   I rather think, Socrates, that some honourable things are evil.

\par \textbf{SOCRATES}
\par   And are some dishonourable things good?

\par \textbf{ALCIBIADES}
\par   Yes.

\par \textbf{SOCRATES}
\par   You mean in such a case as the following: —In time of war, men have been wounded or have died in rescuing a companion or kinsman, when others who have neglected the duty of rescuing them have escaped in safety?

\par \textbf{ALCIBIADES}
\par   True.

\par \textbf{SOCRATES}
\par   And to rescue another under such circumstances is honourable, in respect of the attempt to save those whom we ought to save; and this is courage?

\par \textbf{ALCIBIADES}
\par   True.

\par \textbf{SOCRATES}
\par   But evil in respect of death and wounds?

\par \textbf{ALCIBIADES}
\par   Yes.

\par \textbf{SOCRATES}
\par   And the courage which is shown in the rescue is one thing, and the death another?

\par \textbf{ALCIBIADES}
\par   Certainly.

\par \textbf{SOCRATES}
\par   Then the rescue of one's friends is honourable in one point of view, but evil in another?

\par \textbf{ALCIBIADES}
\par   True.

\par \textbf{SOCRATES}
\par   And if honourable, then also good:  Will you consider now whether I may not be right, for you were acknowledging that the courage which is shown in the rescue is honourable? Now is this courage good or evil? Look at the matter thus:  which would you rather choose, good or evil?

\par \textbf{ALCIBIADES}
\par   Good.

\par \textbf{SOCRATES}
\par   And the greatest goods you would be most ready to choose, and would least like to be deprived of them?

\par \textbf{ALCIBIADES}
\par   Certainly.

\par \textbf{SOCRATES}
\par   What would you say of courage? At what price would you be willing to be deprived of courage?

\par \textbf{ALCIBIADES}
\par   I would rather die than be a coward.

\par \textbf{SOCRATES}
\par   Then you think that cowardice is the worst of evils?

\par \textbf{ALCIBIADES}
\par   I do.

\par \textbf{SOCRATES}
\par   As bad as death, I suppose?

\par \textbf{ALCIBIADES}
\par   Yes.

\par \textbf{SOCRATES}
\par   And life and courage are the extreme opposites of death and cowardice?

\par \textbf{ALCIBIADES}
\par   Yes.

\par \textbf{SOCRATES}
\par   And they are what you would most desire to have, and their opposites you would least desire?

\par \textbf{ALCIBIADES}
\par   Yes.

\par \textbf{SOCRATES}
\par   Is this because you think life and courage the best, and death and cowardice the worst?

\par \textbf{ALCIBIADES}
\par   Yes.

\par \textbf{SOCRATES}
\par   And you would term the rescue of a friend in battle honourable, in as much as courage does a good work?

\par \textbf{ALCIBIADES}
\par   I should.

\par \textbf{SOCRATES}
\par   But evil because of the death which ensues?

\par \textbf{ALCIBIADES}
\par   Yes.

\par \textbf{SOCRATES}
\par   Might we not describe their different effects as follows: —You may call either of them evil in respect of the evil which is the result, and good in respect of the good which is the result of either of them?

\par \textbf{ALCIBIADES}
\par   Yes.

\par \textbf{SOCRATES}
\par   And they are honourable in so far as they are good, and dishonourable in so far as they are evil?

\par \textbf{ALCIBIADES}
\par   True.

\par \textbf{SOCRATES}
\par   Then when you say that the rescue of a friend in battle is honourable and yet evil, that is equivalent to saying that the rescue is good and yet evil?

\par \textbf{ALCIBIADES}
\par   I believe that you are right, Socrates.

\par \textbf{SOCRATES}
\par   Nothing honourable, regarded as honourable, is evil; nor anything base, regarded as base, good.

\par \textbf{ALCIBIADES}
\par   Clearly not.

\par \textbf{SOCRATES}
\par   Look at the matter yet once more in a further light:  he who acts honourably acts well?

\par \textbf{ALCIBIADES}
\par   Yes.

\par \textbf{SOCRATES}
\par   And he who acts well is happy?

\par \textbf{ALCIBIADES}
\par   Of course.

\par \textbf{SOCRATES}
\par   And the happy are those who obtain good?

\par \textbf{ALCIBIADES}
\par   True.

\par \textbf{SOCRATES}
\par   And they obtain good by acting well and honourably?

\par \textbf{ALCIBIADES}
\par   Yes.

\par \textbf{SOCRATES}
\par   Then acting well is a good?

\par \textbf{ALCIBIADES}
\par   Certainly.

\par \textbf{SOCRATES}
\par   And happiness is a good?

\par \textbf{ALCIBIADES}
\par   Yes.

\par \textbf{SOCRATES}
\par   Then the good and the honourable are again identified.

\par \textbf{ALCIBIADES}
\par   Manifestly.

\par \textbf{SOCRATES}
\par   Then, if the argument holds, what we find to be honourable we shall also find to be good?

\par \textbf{ALCIBIADES}
\par   Certainly.

\par \textbf{SOCRATES}
\par   And is the good expedient or not?

\par \textbf{ALCIBIADES}
\par   Expedient.

\par \textbf{SOCRATES}
\par   Do you remember our admissions about the just?

\par \textbf{ALCIBIADES}
\par   Yes; if I am not mistaken, we said that those who acted justly must also act honourably.

\par \textbf{SOCRATES}
\par   And the honourable is the good?

\par \textbf{ALCIBIADES}
\par   Yes.

\par \textbf{SOCRATES}
\par   And the good is expedient?

\par \textbf{ALCIBIADES}
\par   Yes.

\par \textbf{SOCRATES}
\par   Then, Alcibiades, the just is expedient?

\par \textbf{ALCIBIADES}
\par   I should infer so.

\par \textbf{SOCRATES}
\par   And all this I prove out of your own mouth, for I ask and you answer?

\par \textbf{ALCIBIADES}
\par   I must acknowledge it to be true.

\par \textbf{SOCRATES}
\par   And having acknowledged that the just is the same as the expedient, are you not (let me ask) prepared to ridicule any one who, pretending to understand the principles of justice and injustice, gets up to advise the noble Athenians or the ignoble Peparethians, that the just may be the evil?

\par \textbf{ALCIBIADES}
\par   I solemnly declare, Socrates, that I do not know what I am saying. Verily, I am in a strange state, for when you put questions to me I am of different minds in successive instants.

\par \textbf{SOCRATES}
\par   And are you not aware of the nature of this perplexity, my friend?

\par \textbf{ALCIBIADES}
\par   Indeed I am not.

\par \textbf{SOCRATES}
\par   Do you suppose that if some one were to ask you whether you have two eyes or three, or two hands or four, or anything of that sort, you would then be of different minds in successive instants?

\par \textbf{ALCIBIADES}
\par   I begin to distrust myself, but still I do not suppose that I should.

\par \textbf{SOCRATES}
\par   You would feel no doubt; and for this reason—because you would know?

\par \textbf{ALCIBIADES}
\par   I suppose so.

\par \textbf{SOCRATES}
\par   And the reason why you involuntarily contradict yourself is clearly that you are ignorant?

\par \textbf{ALCIBIADES}
\par   Very likely.

\par \textbf{SOCRATES}
\par   And if you are perplexed in answering about just and unjust, honourable and dishonourable, good and evil, expedient and inexpedient, the reason is that you are ignorant of them, and therefore in perplexity. Is not that clear?

\par \textbf{ALCIBIADES}
\par   I agree.

\par \textbf{SOCRATES}
\par   But is this always the case, and is a man necessarily perplexed about that of which he has no knowledge?

\par \textbf{ALCIBIADES}
\par   Certainly he is.

\par \textbf{SOCRATES}
\par   And do you know how to ascend into heaven?

\par \textbf{ALCIBIADES}
\par   Certainly not.

\par \textbf{SOCRATES}
\par   And in this case, too, is your judgment perplexed?

\par \textbf{ALCIBIADES}
\par   No.

\par \textbf{SOCRATES}
\par   Do you see the reason why, or shall I tell you?

\par \textbf{ALCIBIADES}
\par   Tell me.

\par \textbf{SOCRATES}
\par   The reason is, that you not only do not know, my friend, but you do not think that you know.

\par \textbf{ALCIBIADES}
\par   There again; what do you mean?

\par \textbf{SOCRATES}
\par   Ask yourself; are you in any perplexity about things of which you are ignorant? You know, for example, that you know nothing about the preparation of food.

\par \textbf{ALCIBIADES}
\par   Very true.

\par \textbf{SOCRATES}
\par   And do you think and perplex yourself about the preparation of food:  or do you leave that to some one who understands the art?

\par \textbf{ALCIBIADES}
\par   The latter.

\par \textbf{SOCRATES}
\par   Or if you were on a voyage, would you bewilder yourself by considering whether the rudder is to be drawn inwards or outwards, or do you leave that to the pilot, and do nothing?

\par \textbf{ALCIBIADES}
\par   It would be the concern of the pilot.

\par \textbf{SOCRATES}
\par   Then you are not perplexed about what you do not know, if you know that you do not know it?

\par \textbf{ALCIBIADES}
\par   I imagine not.

\par \textbf{SOCRATES}
\par   Do you not see, then, that mistakes in life and practice are likewise to be attributed to the ignorance which has conceit of knowledge?

\par \textbf{ALCIBIADES}
\par   Once more, what do you mean?

\par \textbf{SOCRATES}
\par   I suppose that we begin to act when we think that we know what we are doing?

\par \textbf{ALCIBIADES}
\par   Yes.

\par \textbf{SOCRATES}
\par   But when people think that they do not know, they entrust their business to others?

\par \textbf{ALCIBIADES}
\par   Yes.

\par \textbf{SOCRATES}
\par   And so there is a class of ignorant persons who do not make mistakes in life, because they trust others about things of which they are ignorant?

\par \textbf{ALCIBIADES}
\par   True.

\par \textbf{SOCRATES}
\par   Who, then, are the persons who make mistakes? They cannot, of course, be those who know?

\par \textbf{ALCIBIADES}
\par   Certainly not.

\par \textbf{SOCRATES}
\par   But if neither those who know, nor those who know that they do not know, make mistakes, there remain those only who do not know and think that they know.

\par \textbf{ALCIBIADES}
\par   Yes, only those.

\par \textbf{SOCRATES}
\par   Then this is ignorance of the disgraceful sort which is mischievous?

\par \textbf{ALCIBIADES}
\par   Yes.

\par \textbf{SOCRATES}
\par   And most mischievous and most disgraceful when having to do with the greatest matters?

\par \textbf{ALCIBIADES}
\par   By far.

\par \textbf{SOCRATES}
\par   And can there be any matters greater than the just, the honourable, the good, and the expedient?

\par \textbf{ALCIBIADES}
\par   There cannot be.

\par \textbf{SOCRATES}
\par   And these, as you were saying, are what perplex you?

\par \textbf{ALCIBIADES}
\par   Yes.

\par \textbf{SOCRATES}
\par   But if you are perplexed, then, as the previous argument has shown, you are not only ignorant of the greatest matters, but being ignorant you fancy that you know them?

\par \textbf{ALCIBIADES}
\par   I fear that you are right.

\par \textbf{SOCRATES}
\par   And now see what has happened to you, Alcibiades! I hardly like to speak of your evil case, but as we are alone I will:  My good friend, you are wedded to ignorance of the most disgraceful kind, and of this you are convicted, not by me, but out of your own mouth and by your own argument; wherefore also you rush into politics before you are educated. Neither is your case to be deemed singular. For I might say the same of almost all our statesmen, with the exception, perhaps of your guardian, Pericles.

\par \textbf{ALCIBIADES}
\par   Yes, Socrates; and Pericles is said not to have got his wisdom by the light of nature, but to have associated with several of the philosophers; with Pythocleides, for example, and with Anaxagoras, and now in advanced life with Damon, in the hope of gaining wisdom.

\par \textbf{SOCRATES}
\par   Very good; but did you ever know a man wise in anything who was unable to impart his particular wisdom? For example, he who taught you letters was not only wise, but he made you and any others whom he liked wise.

\par \textbf{ALCIBIADES}
\par   Yes.

\par \textbf{SOCRATES}
\par   And you, whom he taught, can do the same?

\par \textbf{ALCIBIADES}
\par   True.

\par \textbf{SOCRATES}
\par   And in like manner the harper and gymnastic-master?

\par \textbf{ALCIBIADES}
\par   Certainly.

\par \textbf{SOCRATES}
\par   When a person is enabled to impart knowledge to another, he thereby gives an excellent proof of his own understanding of any matter.

\par \textbf{ALCIBIADES}
\par   I agree.

\par \textbf{SOCRATES}
\par   Well, and did Pericles make any one wise; did he begin by making his sons wise?

\par \textbf{ALCIBIADES}
\par   But, Socrates, if the two sons of Pericles were simpletons, what has that to do with the matter?

\par \textbf{SOCRATES}
\par   Well, but did he make your brother, Cleinias, wise?

\par \textbf{ALCIBIADES}
\par   Cleinias is a madman; there is no use in talking of him.

\par \textbf{SOCRATES}
\par   But if Cleinias is a madman and the two sons of Pericles were simpletons, what reason can be given why he neglects you, and lets you be as you are?

\par \textbf{ALCIBIADES}
\par   I believe that I am to blame for not listening to him.

\par \textbf{SOCRATES}
\par   But did you ever hear of any other Athenian or foreigner, bond or free, who was deemed to have grown wiser in the society of Pericles,—as I might cite Pythodorus, the son of Isolochus, and Callias, the son of Calliades, who have grown wiser in the society of Zeno, for which privilege they have each of them paid him the sum of a hundred minae (about 406 pounds sterling) to the increase of their wisdom and fame.

\par \textbf{ALCIBIADES}
\par   I certainly never did hear of any one.

\par \textbf{SOCRATES}
\par   Well, and in reference to your own case, do you mean to remain as you are, or will you take some pains about yourself?

\par \textbf{ALCIBIADES}
\par   With your aid, Socrates, I will. And indeed, when I hear you speak, the truth of what you are saying strikes home to me, and I agree with you, for our statesmen, all but a few, do appear to be quite uneducated.

\par \textbf{SOCRATES}
\par   What is the inference?

\par \textbf{ALCIBIADES}
\par   Why, that if they were educated they would be trained athletes, and he who means to rival them ought to have knowledge and experience when he attacks them; but now, as they have become politicians without any special training, why should I have the trouble of learning and practising? For I know well that by the light of nature I shall get the better of them.

\par \textbf{SOCRATES}
\par   My dear friend, what a sentiment! And how unworthy of your noble form and your high estate!

\par \textbf{ALCIBIADES}
\par   What do you mean, Socrates; why do you say so?

\par \textbf{SOCRATES}
\par   I am grieved when I think of our mutual love.

\par \textbf{ALCIBIADES}
\par   At what?

\par \textbf{SOCRATES}
\par   At your fancying that the contest on which you are entering is with people here.

\par \textbf{ALCIBIADES}
\par   Why, what others are there?

\par \textbf{SOCRATES}
\par   Is that a question which a magnanimous soul should ask?

\par \textbf{ALCIBIADES}
\par   Do you mean to say that the contest is not with these?

\par \textbf{SOCRATES}
\par   And suppose that you were going to steer a ship into action, would you only aim at being the best pilot on board? Would you not, while acknowledging that you must possess this degree of excellence, rather look to your antagonists, and not, as you are now doing, to your fellow combatants? You ought to be so far above these latter, that they will not even dare to be your rivals; and, being regarded by you as inferiors, will do battle for you against the enemy; this is the kind of superiority which you must establish over them, if you mean to accomplish any noble action really worthy of yourself and of the state.

\par \textbf{ALCIBIADES}
\par   That would certainly be my aim.

\par \textbf{SOCRATES}
\par   Verily, then, you have good reason to be satisfied, if you are better than the soldiers; and you need not, when you are their superior and have your thoughts and actions fixed upon them, look away to the generals of the enemy.

\par \textbf{ALCIBIADES}
\par   Of whom are you speaking, Socrates?

\par \textbf{SOCRATES}
\par   Why, you surely know that our city goes to war now and then with the Lacedaemonians and with the great king?

\par \textbf{ALCIBIADES}
\par   True enough.

\par \textbf{SOCRATES}
\par   And if you meant to be the ruler of this city, would you not be right in considering that the Lacedaemonian and Persian king were your true rivals?

\par \textbf{ALCIBIADES}
\par   I believe that you are right.

\par \textbf{SOCRATES}
\par   Oh no, my friend, I am quite wrong, and I think that you ought rather to turn your attention to Midias the quail-breeder and others like him, who manage our politics; in whom, as the women would remark, you may still see the slaves' cut of hair, cropping out in their minds as well as on their pates; and they come with their barbarous lingo to flatter us and not to rule us. To these, I say, you should look, and then you need not trouble yourself about your own fitness to contend in such a noble arena:  there is no reason why you should either learn what has to be learned, or practise what has to be practised, and only when thoroughly prepared enter on a political career.

\par \textbf{ALCIBIADES}
\par   There, I think, Socrates, that you are right; I do not suppose, however, that the Spartan generals or the great king are really different from anybody else.

\par \textbf{SOCRATES}
\par   But, my dear friend, do consider what you are saying.

\par \textbf{ALCIBIADES}
\par   What am I to consider?

\par \textbf{SOCRATES}
\par   In the first place, will you be more likely to take care of yourself, if you are in a wholesome fear and dread of them, or if you are not?

\par \textbf{ALCIBIADES}
\par   Clearly, if I have such a fear of them.

\par \textbf{SOCRATES}
\par   And do you think that you will sustain any injury if you take care of yourself?

\par \textbf{ALCIBIADES}
\par   No, I shall be greatly benefited.

\par \textbf{SOCRATES}
\par   And this is one very important respect in which that notion of yours is bad.

\par \textbf{ALCIBIADES}
\par   True.

\par \textbf{SOCRATES}
\par   In the next place, consider that what you say is probably false.

\par \textbf{ALCIBIADES}
\par   How so?

\par \textbf{SOCRATES}
\par   Let me ask you whether better natures are likely to be found in noble races or not in noble races?

\par \textbf{ALCIBIADES}
\par   Clearly in noble races.

\par \textbf{SOCRATES}
\par   Are not those who are well born and well bred most likely to be perfect in virtue?

\par \textbf{ALCIBIADES}
\par   Certainly.

\par \textbf{SOCRATES}
\par   Then let us compare our antecedents with those of the Lacedaemonian and Persian kings; are they inferior to us in descent? Have we not heard that the former are sprung from Heracles, and the latter from Achaemenes, and that the race of Heracles and the race of Achaemenes go back to Perseus, son of Zeus?

\par \textbf{ALCIBIADES}
\par   Why, so does mine go back to Eurysaces, and he to Zeus!

\par \textbf{SOCRATES}
\par   And mine, noble Alcibiades, to Daedalus, and he to Hephaestus, son of Zeus. But, for all that, we are far inferior to them. For they are descended 'from Zeus,' through a line of kings—either kings of Argos and Lacedaemon, or kings of Persia, a country which the descendants of Achaemenes have always possessed, besides being at various times sovereigns of Asia, as they now are; whereas, we and our fathers were but private persons. How ridiculous would you be thought if you were to make a display of your ancestors and of Salamis the island of Eurysaces, or of Aegina, the habitation of the still more ancient Aeacus, before Artaxerxes, son of Xerxes. You should consider how inferior we are to them both in the derivation of our birth and in other particulars. Did you never observe how great is the property of the Spartan kings? And their wives are under the guardianship of the Ephori, who are public officers and watch over them, in order to preserve as far as possible the purity of the Heracleid blood. Still greater is the difference among the Persians; for no one entertains a suspicion that the father of a prince of Persia can be any one but the king. Such is the awe which invests the person of the queen, that any other guard is needless. And when the heir of the kingdom is born, all the subjects of the king feast; and the day of his birth is for ever afterwards kept as a holiday and time of sacrifice by all Asia; whereas, when you and I were born, Alcibiades, as the comic poet says, the neighbours hardly knew of the important event. After the birth of the royal child, he is tended, not by a good-for-nothing woman-nurse, but by the best of the royal eunuchs, who are charged with the care of him, and especially with the fashioning and right formation of his limbs, in order that he may be as shapely as possible; which being their calling, they are held in great honour. And when the young prince is seven years old he is put upon a horse and taken to the riding-masters, and begins to go out hunting. And at fourteen years of age he is handed over to the royal schoolmasters, as they are termed:  these are four chosen men, reputed to be the best among the Persians of a certain age; and one of them is the wisest, another the justest, a third the most temperate, and a fourth the most valiant. The first instructs him in the magianism of Zoroaster, the son of Oromasus, which is the worship of the Gods, and teaches him also the duties of his royal office; the second, who is the justest, teaches him always to speak the truth; the third, or most temperate, forbids him to allow any pleasure to be lord over him, that he may be accustomed to be a freeman and king indeed,—lord of himself first, and not a slave; the most valiant trains him to be bold and fearless, telling him that if he fears he is to deem himself a slave; whereas Pericles gave you, Alcibiades, for a tutor Zopyrus the Thracian, a slave of his who was past all other work. I might enlarge on the nurture and education of your rivals, but that would be tedious; and what I have said is a sufficient sample of what remains to be said. I have only to remark, by way of contrast, that no one cares about your birth or nurture or education, or, I may say, about that of any other Athenian, unless he has a lover who looks after him. And if you cast an eye on the wealth, the luxury, the garments with their flowing trains, the anointings with myrrh, the multitudes of attendants, and all the other bravery of the Persians, you will be ashamed when you discern your own inferiority; or if you look at the temperance and orderliness and ease and grace and magnanimity and courage and endurance and love of toil and desire of glory and ambition of the Lacedaemonians—in all these respects you will see that you are but a child in comparison of them. Even in the matter of wealth, if you value yourself upon that, I must reveal to you how you stand; for if you form an estimate of the wealth of the Lacedaemonians, you will see that our possessions fall far short of theirs. For no one here can compete with them either in the extent and fertility of their own and the Messenian territory, or in the number of their slaves, and especially of the Helots, or of their horses, or of the animals which feed on the Messenian pastures. But I have said enough of this:  and as to gold and silver, there is more of them in Lacedaemon than in all the rest of Hellas, for during many generations gold has been always flowing in to them from the whole Hellenic world, and often from the barbarian also, and never going out, as in the fable of Aesop the fox said to the lion, 'The prints of the feet of those going in are distinct enough;' but who ever saw the trace of money going out of Lacedaemon? And therefore you may safely infer that the inhabitants are the richest of the Hellenes in gold and silver, and that their kings are the richest of them, for they have a larger share of these things, and they have also a tribute paid to them which is very considerable. Yet the Spartan wealth, though great in comparison of the wealth of the other Hellenes, is as nothing in comparison of that of the Persians and their kings. Why, I have been informed by a credible person who went up to the king (at Susa), that he passed through a large tract of excellent land, extending for nearly a day's journey, which the people of the country called the queen's girdle, and another, which they called her veil; and several other fair and fertile districts, which were reserved for the adornment of the queen, and are named after her several habiliments. Now, I cannot help thinking to myself, What if some one were to go to Amestris, the wife of Xerxes and mother of Artaxerxes, and say to her, There is a certain Dinomache, whose whole wardrobe is not worth fifty minae—and that will be more than the value—and she has a son who is possessed of a three-hundred acre patch at Erchiae, and he has a mind to go to war with your son—would she not wonder to what this Alcibiades trusts for success in the conflict? 'He must rely,' she would say to herself, 'upon his training and wisdom—these are the things which Hellenes value.' And if she heard that this Alcibiades who is making the attempt is not as yet twenty years old, and is wholly uneducated, and when his lover tells him that he ought to get education and training first, and then go and fight the king, he refuses, and says that he is well enough as he is, would she not be amazed, and ask 'On what, then, does the youth rely?' And if we replied:  He relies on his beauty, and stature, and birth, and mental endowments, she would think that we were mad, Alcibiades, when she compared the advantages which you possess with those of her own people. And I believe that even Lampido, the daughter of Leotychides, the wife of Archidamus and mother of Agis, all of whom were kings, would have the same feeling; if, in your present uneducated state, you were to turn your thoughts against her son, she too would be equally astonished. But how disgraceful, that we should not have as high a notion of what is required in us as our enemies' wives and mothers have of the qualities which are required in their assailants! O my friend, be persuaded by me, and hear the Delphian inscription, 'Know thyself'—not the men whom you think, but these kings are our rivals, and we can only overcome them by pains and skill. And if you fail in the required qualities, you will fail also in becoming renowned among Hellenes and Barbarians, which you seem to desire more than any other man ever desired anything.

\par \textbf{ALCIBIADES}
\par   I entirely believe you; but what are the sort of pains which are required, Socrates,—can you tell me?

\par \textbf{SOCRATES}
\par   Yes, I can; but we must take counsel together concerning the manner in which both of us may be most improved. For what I am telling you of the necessity of education applies to myself as well as to you; and there is only one point in which I have an advantage over you.

\par \textbf{ALCIBIADES}
\par   What is that?

\par \textbf{SOCRATES}
\par   I have a guardian who is better and wiser than your guardian, Pericles.

\par \textbf{ALCIBIADES}
\par   Who is he, Socrates?

\par \textbf{SOCRATES}
\par   God, Alcibiades, who up to this day has not allowed me to converse with you; and he inspires in me the faith that I am especially designed to bring you to honour.

\par \textbf{ALCIBIADES}
\par   You are jesting, Socrates.

\par \textbf{SOCRATES}
\par   Perhaps, at any rate, I am right in saying that all men greatly need pains and care, and you and I above all men.

\par \textbf{ALCIBIADES}
\par   You are not far wrong about me.

\par \textbf{SOCRATES}
\par   And certainly not about myself.

\par \textbf{ALCIBIADES}
\par   But what can we do?

\par \textbf{SOCRATES}
\par   There must be no hesitation or cowardice, my friend.

\par \textbf{ALCIBIADES}
\par   That would not become us, Socrates.

\par \textbf{SOCRATES}
\par   No, indeed, and we ought to take counsel together:  for do we not wish to be as good as possible?

\par \textbf{ALCIBIADES}
\par   We do.

\par \textbf{SOCRATES}
\par   In what sort of virtue?

\par \textbf{ALCIBIADES}
\par   Plainly, in the virtue of good men.

\par \textbf{SOCRATES}
\par   Who are good in what?

\par \textbf{ALCIBIADES}
\par   Those, clearly, who are good in the management of affairs.

\par \textbf{SOCRATES}
\par   What sort of affairs? Equestrian affairs?

\par \textbf{ALCIBIADES}
\par   Certainly not.

\par \textbf{SOCRATES}
\par   You mean that about them we should have recourse to horsemen?

\par \textbf{ALCIBIADES}
\par   Yes.

\par \textbf{SOCRATES}
\par   Well, naval affairs?

\par \textbf{ALCIBIADES}
\par   No.

\par \textbf{SOCRATES}
\par   You mean that we should have recourse to sailors about them?

\par \textbf{ALCIBIADES}
\par   Yes.

\par \textbf{SOCRATES}
\par   Then what affairs? And who do them?

\par \textbf{ALCIBIADES}
\par   The affairs which occupy Athenian gentlemen.

\par \textbf{SOCRATES}
\par   And when you speak of gentlemen, do you mean the wise or the unwise?

\par \textbf{ALCIBIADES}
\par   The wise.

\par \textbf{SOCRATES}
\par   And a man is good in respect of that in which he is wise?

\par \textbf{ALCIBIADES}
\par   Yes.

\par \textbf{SOCRATES}
\par   And evil in respect of that in which he is unwise?

\par \textbf{ALCIBIADES}
\par   Certainly.

\par \textbf{SOCRATES}
\par   The shoemaker, for example, is wise in respect of the making of shoes?

\par \textbf{ALCIBIADES}
\par   Yes.

\par \textbf{SOCRATES}
\par   Then he is good in that?

\par \textbf{ALCIBIADES}
\par   He is.

\par \textbf{SOCRATES}
\par   But in respect of the making of garments he is unwise?

\par \textbf{ALCIBIADES}
\par   Yes.

\par \textbf{SOCRATES}
\par   Then in that he is bad?

\par \textbf{ALCIBIADES}
\par   Yes.

\par \textbf{SOCRATES}
\par   Then upon this view of the matter the same man is good and also bad?

\par \textbf{ALCIBIADES}
\par   True.

\par \textbf{SOCRATES}
\par   But would you say that the good are the same as the bad?

\par \textbf{ALCIBIADES}
\par   Certainly not.

\par \textbf{SOCRATES}
\par   Then whom do you call the good?

\par \textbf{ALCIBIADES}
\par   I mean by the good those who are able to rule in the city.

\par \textbf{SOCRATES}
\par   Not, surely, over horses?

\par \textbf{ALCIBIADES}
\par   Certainly not.

\par \textbf{SOCRATES}
\par   But over men?

\par \textbf{ALCIBIADES}
\par   Yes.

\par \textbf{SOCRATES}
\par   When they are sick?

\par \textbf{ALCIBIADES}
\par   No.

\par \textbf{SOCRATES}
\par   Or on a voyage?

\par \textbf{ALCIBIADES}
\par   No.

\par \textbf{SOCRATES}
\par   Or reaping the harvest?

\par \textbf{ALCIBIADES}
\par   No.

\par \textbf{SOCRATES}
\par   When they are doing something or nothing?

\par \textbf{ALCIBIADES}
\par   When they are doing something, I should say.

\par \textbf{SOCRATES}
\par   I wish that you would explain to me what this something is.

\par \textbf{ALCIBIADES}
\par   When they are having dealings with one another, and using one another's services, as we citizens do in our daily life.

\par \textbf{SOCRATES}
\par   Those of whom you speak are ruling over men who are using the services of other men?

\par \textbf{ALCIBIADES}
\par   Yes.

\par \textbf{SOCRATES}
\par   Are they ruling over the signal-men who give the time to the rowers?

\par \textbf{ALCIBIADES}
\par   No; they are not.

\par \textbf{SOCRATES}
\par   That would be the office of the pilot?

\par \textbf{ALCIBIADES}
\par   Yes.

\par \textbf{SOCRATES}
\par   But, perhaps you mean that they rule over flute-players, who lead the singers and use the services of the dancers?

\par \textbf{ALCIBIADES}
\par   Certainly not.

\par \textbf{SOCRATES}
\par   That would be the business of the teacher of the chorus?

\par \textbf{ALCIBIADES}
\par   Yes.

\par \textbf{SOCRATES}
\par   Then what is the meaning of being able to rule over men who use other men?

\par \textbf{ALCIBIADES}
\par   I mean that they rule over men who have common rights of citizenship, and dealings with one another.

\par \textbf{SOCRATES}
\par   And what sort of an art is this? Suppose that I ask you again, as I did just now, What art makes men know how to rule over their fellow-sailors,—how would you answer?

\par \textbf{ALCIBIADES}
\par   The art of the pilot.

\par \textbf{SOCRATES}
\par   And, if I may recur to another old instance, what art enables them to rule over their fellow-singers?

\par \textbf{ALCIBIADES}
\par   The art of the teacher of the chorus, which you were just now mentioning.

\par \textbf{SOCRATES}
\par   And what do you call the art of fellow-citizens?

\par \textbf{ALCIBIADES}
\par   I should say, good counsel, Socrates.

\par \textbf{SOCRATES}
\par   And is the art of the pilot evil counsel?

\par \textbf{ALCIBIADES}
\par   No.

\par \textbf{SOCRATES}
\par   But good counsel?

\par \textbf{ALCIBIADES}
\par   Yes, that is what I should say,—good counsel, of which the aim is the preservation of the voyagers.

\par \textbf{SOCRATES}
\par   True. And what is the aim of that other good counsel of which you speak?

\par \textbf{ALCIBIADES}
\par   The aim is the better order and preservation of the city.

\par \textbf{SOCRATES}
\par   And what is that of which the absence or presence improves and preserves the order of the city? Suppose you were to ask me, what is that of which the presence or absence improves or preserves the order of the body? I should reply, the presence of health and the absence of disease. You would say the same?

\par \textbf{ALCIBIADES}
\par   Yes.

\par \textbf{SOCRATES}
\par   And if you were to ask me the same question about the eyes, I should reply in the same way, 'the presence of sight and the absence of blindness;' or about the ears, I should reply, that they were improved and were in better case, when deafness was absent, and hearing was present in them.

\par \textbf{ALCIBIADES}
\par   True.

\par \textbf{SOCRATES}
\par   And what would you say of a state? What is that by the presence or absence of which the state is improved and better managed and ordered?

\par \textbf{ALCIBIADES}
\par   I should say, Socrates: —the presence of friendship and the absence of hatred and division.

\par \textbf{SOCRATES}
\par   And do you mean by friendship agreement or disagreement?

\par \textbf{ALCIBIADES}
\par   Agreement.

\par \textbf{SOCRATES}
\par   What art makes cities agree about numbers?

\par \textbf{ALCIBIADES}
\par   Arithmetic.

\par \textbf{SOCRATES}
\par   And private individuals?

\par \textbf{ALCIBIADES}
\par   The same.

\par \textbf{SOCRATES}
\par   And what art makes each individual agree with himself?

\par \textbf{ALCIBIADES}
\par   The same.

\par \textbf{SOCRATES}
\par   And what art makes each of us agree with himself about the comparative length of the span and of the cubit? Does not the art of measure?

\par \textbf{ALCIBIADES}
\par   Yes.

\par \textbf{SOCRATES}
\par   Individuals are agreed with one another about this; and states, equally?

\par \textbf{ALCIBIADES}
\par   Yes.

\par \textbf{SOCRATES}
\par   And the same holds of the balance?

\par \textbf{ALCIBIADES}
\par   True.

\par \textbf{SOCRATES}
\par   But what is the other agreement of which you speak, and about what? what art can give that agreement? And does that which gives it to the state give it also to the individual, so as to make him consistent with himself and with another?

\par \textbf{ALCIBIADES}
\par   I should suppose so.

\par \textbf{SOCRATES}
\par   But what is the nature of the agreement?—answer, and faint not.

\par \textbf{ALCIBIADES}
\par   I mean to say that there should be such friendship and agreement as exists between an affectionate father and mother and their son, or between brothers, or between husband and wife.

\par \textbf{SOCRATES}
\par   But can a man, Alcibiades, agree with a woman about the spinning of wool, which she understands and he does not?

\par \textbf{ALCIBIADES}
\par   No, truly.

\par \textbf{SOCRATES}
\par   Nor has he any need, for spinning is a female accomplishment.

\par \textbf{ALCIBIADES}
\par   Yes.

\par \textbf{SOCRATES}
\par   And would a woman agree with a man about the science of arms, which she has never learned?

\par \textbf{ALCIBIADES}
\par   Certainly not.

\par \textbf{SOCRATES}
\par   I suppose that the use of arms would be regarded by you as a male accomplishment?

\par \textbf{ALCIBIADES}
\par   It would.

\par \textbf{SOCRATES}
\par   Then, upon your view, women and men have two sorts of knowledge?

\par \textbf{ALCIBIADES}
\par   Certainly.

\par \textbf{SOCRATES}
\par   Then in their knowledge there is no agreement of women and men?

\par \textbf{ALCIBIADES}
\par   There is not.

\par \textbf{SOCRATES}
\par   Nor can there be friendship, if friendship is agreement?

\par \textbf{ALCIBIADES}
\par   Plainly not.

\par \textbf{SOCRATES}
\par   Then women are not loved by men when they do their own work?

\par \textbf{ALCIBIADES}
\par   I suppose not.

\par \textbf{SOCRATES}
\par   Nor men by women when they do their own work?

\par \textbf{ALCIBIADES}
\par   No.

\par \textbf{SOCRATES}
\par   Nor are states well administered, when individuals do their own work?

\par \textbf{ALCIBIADES}
\par   I should rather think, Socrates, that the reverse is the truth. (Compare Republic.)

\par \textbf{SOCRATES}
\par   What! do you mean to say that states are well administered when friendship is absent, the presence of which, as we were saying, alone secures their good order?

\par \textbf{ALCIBIADES}
\par   But I should say that there is friendship among them, for this very reason, that the two parties respectively do their own work.

\par \textbf{SOCRATES}
\par   That was not what you were saying before; and what do you mean now by affirming that friendship exists when there is no agreement? How can there be agreement about matters which the one party knows, and of which the other is in ignorance?

\par \textbf{ALCIBIADES}
\par   Impossible.

\par \textbf{SOCRATES}
\par   And when individuals are doing their own work, are they doing what is just or unjust?

\par \textbf{ALCIBIADES}
\par   What is just, certainly.

\par \textbf{SOCRATES}
\par   And when individuals do what is just in the state, is there no friendship among them?

\par \textbf{ALCIBIADES}
\par   I suppose that there must be, Socrates.

\par \textbf{SOCRATES}
\par   Then what do you mean by this friendship or agreement about which we must be wise and discreet in order that we may be good men? I cannot make out where it exists or among whom; according to you, the same persons may sometimes have it, and sometimes not.

\par \textbf{ALCIBIADES}
\par   But, indeed, Socrates, I do not know what I am saying; and I have long been, unconsciously to myself, in a most disgraceful state.

\par \textbf{SOCRATES}
\par   Nevertheless, cheer up; at fifty, if you had discovered your deficiency, you would have been too old, and the time for taking care of yourself would have passed away, but yours is just the age at which the discovery should be made.

\par \textbf{ALCIBIADES}
\par   And what should he do, Socrates, who would make the discovery?

\par \textbf{SOCRATES}
\par   Answer questions, Alcibiades; and that is a process which, by the grace of God, if I may put any faith in my oracle, will be very improving to both of us.

\par \textbf{ALCIBIADES}
\par   If I can be improved by answering, I will answer.

\par \textbf{SOCRATES}
\par   And first of all, that we may not peradventure be deceived by appearances, fancying, perhaps, that we are taking care of ourselves when we are not, what is the meaning of a man taking care of himself? and when does he take care? Does he take care of himself when he takes care of what belongs to him?

\par \textbf{ALCIBIADES}
\par   I should think so.

\par \textbf{SOCRATES}
\par   When does a man take care of his feet? Does he not take care of them when he takes care of that which belongs to his feet?

\par \textbf{ALCIBIADES}
\par   I do not understand.

\par \textbf{SOCRATES}
\par   Let me take the hand as an illustration; does not a ring belong to the finger, and to the finger only?

\par \textbf{ALCIBIADES}
\par   Yes.

\par \textbf{SOCRATES}
\par   And the shoe in like manner to the foot?

\par \textbf{ALCIBIADES}
\par   Yes.

\par \textbf{SOCRATES}
\par   And when we take care of our shoes, do we not take care of our feet?

\par \textbf{ALCIBIADES}
\par   I do not comprehend, Socrates.

\par \textbf{SOCRATES}
\par   But you would admit, Alcibiades, that to take proper care of a thing is a correct expression?

\par \textbf{ALCIBIADES}
\par   Yes.

\par \textbf{SOCRATES}
\par   And taking proper care means improving?

\par \textbf{ALCIBIADES}
\par   Yes.

\par \textbf{SOCRATES}
\par   And what is the art which improves our shoes?

\par \textbf{ALCIBIADES}
\par   Shoemaking.

\par \textbf{SOCRATES}
\par   Then by shoemaking we take care of our shoes?

\par \textbf{ALCIBIADES}
\par   Yes.

\par \textbf{SOCRATES}
\par   And do we by shoemaking take care of our feet, or by some other art which improves the feet?

\par \textbf{ALCIBIADES}
\par   By some other art.

\par \textbf{SOCRATES}
\par   And the same art improves the feet which improves the rest of the body?

\par \textbf{ALCIBIADES}
\par   Very true.

\par \textbf{SOCRATES}
\par   Which is gymnastic?

\par \textbf{ALCIBIADES}
\par   Certainly.

\par \textbf{SOCRATES}
\par   Then by gymnastic we take care of our feet, and by shoemaking of that which belongs to our feet?

\par \textbf{ALCIBIADES}
\par   Very true.

\par \textbf{SOCRATES}
\par   And by gymnastic we take care of our hands, and by the art of graving rings of that which belongs to our hands?

\par \textbf{ALCIBIADES}
\par   Yes.

\par \textbf{SOCRATES}
\par   And by gymnastic we take care of the body, and by the art of weaving and the other arts we take care of the things of the body?

\par \textbf{ALCIBIADES}
\par   Clearly.

\par \textbf{SOCRATES}
\par   Then the art which takes care of each thing is different from that which takes care of the belongings of each thing?

\par \textbf{ALCIBIADES}
\par   True.

\par \textbf{SOCRATES}
\par   Then in taking care of what belongs to you, you do not take care of yourself?

\par \textbf{ALCIBIADES}
\par   Certainly not.

\par \textbf{SOCRATES}
\par   For the art which takes care of our belongings appears not to be the same as that which takes care of ourselves?

\par \textbf{ALCIBIADES}
\par   Clearly not.

\par \textbf{SOCRATES}
\par   And now let me ask you what is the art with which we take care of ourselves?

\par \textbf{ALCIBIADES}
\par   I cannot say.

\par \textbf{SOCRATES}
\par   At any rate, thus much has been admitted, that the art is not one which makes any of our possessions, but which makes ourselves better?

\par \textbf{ALCIBIADES}
\par   True.

\par \textbf{SOCRATES}
\par   But should we ever have known what art makes a shoe better, if we did not know a shoe?

\par \textbf{ALCIBIADES}
\par   Impossible.

\par \textbf{SOCRATES}
\par   Nor should we know what art makes a ring better, if we did not know a ring?

\par \textbf{ALCIBIADES}
\par   That is true.

\par \textbf{SOCRATES}
\par   And can we ever know what art makes a man better, if we do not know what we are ourselves?

\par \textbf{ALCIBIADES}
\par   Impossible.

\par \textbf{SOCRATES}
\par   And is self-knowledge such an easy thing, and was he to be lightly esteemed who inscribed the text on the temple at Delphi? Or is self-knowledge a difficult thing, which few are able to attain?

\par \textbf{ALCIBIADES}
\par   At times I fancy, Socrates, that anybody can know himself; at other times the task appears to be very difficult.

\par \textbf{SOCRATES}
\par   But whether easy or difficult, Alcibiades, still there is no other way; knowing what we are, we shall know how to take care of ourselves, and if we are ignorant we shall not know.

\par \textbf{ALCIBIADES}
\par   That is true.

\par \textbf{SOCRATES}
\par   Well, then, let us see in what way the self-existent can be discovered by us; that will give us a chance of discovering our own existence, which otherwise we can never know.

\par \textbf{ALCIBIADES}
\par   You say truly.

\par \textbf{SOCRATES}
\par   Come, now, I beseech you, tell me with whom you are conversing?—with whom but with me?

\par \textbf{ALCIBIADES}
\par   Yes.

\par \textbf{SOCRATES}
\par   As I am, with you?

\par \textbf{ALCIBIADES}
\par   Yes.

\par \textbf{SOCRATES}
\par   That is to say, I, Socrates, am talking?

\par \textbf{ALCIBIADES}
\par   Yes.

\par \textbf{SOCRATES}
\par   And Alcibiades is my hearer?

\par \textbf{ALCIBIADES}
\par   Yes.

\par \textbf{SOCRATES}
\par   And I in talking use words?

\par \textbf{ALCIBIADES}
\par   Certainly.

\par \textbf{SOCRATES}
\par   And talking and using words have, I suppose, the same meaning?

\par \textbf{ALCIBIADES}
\par   To be sure.

\par \textbf{SOCRATES}
\par   And the user is not the same as the thing which he uses?

\par \textbf{ALCIBIADES}
\par   What do you mean?

\par \textbf{SOCRATES}
\par   I will explain; the shoemaker, for example, uses a square tool, and a circular tool, and other tools for cutting?

\par \textbf{ALCIBIADES}
\par   Yes.

\par \textbf{SOCRATES}
\par   But the tool is not the same as the cutter and user of the tool?

\par \textbf{ALCIBIADES}
\par   Of course not.

\par \textbf{SOCRATES}
\par   And in the same way the instrument of the harper is to be distinguished from the harper himself?

\par \textbf{ALCIBIADES}
\par   It is.

\par \textbf{SOCRATES}
\par   Now the question which I asked was whether you conceive the user to be always different from that which he uses?

\par \textbf{ALCIBIADES}
\par   I do.

\par \textbf{SOCRATES}
\par   Then what shall we say of the shoemaker? Does he cut with his tools only or with his hands?

\par \textbf{ALCIBIADES}
\par   With his hands as well.

\par \textbf{SOCRATES}
\par   He uses his hands too?

\par \textbf{ALCIBIADES}
\par   Yes.

\par \textbf{SOCRATES}
\par   And does he use his eyes in cutting leather?

\par \textbf{ALCIBIADES}
\par   He does.

\par \textbf{SOCRATES}
\par   And we admit that the user is not the same with the things which he uses?

\par \textbf{ALCIBIADES}
\par   Yes.

\par \textbf{SOCRATES}
\par   Then the shoemaker and the harper are to be distinguished from the hands and feet which they use?

\par \textbf{ALCIBIADES}
\par   Clearly.

\par \textbf{SOCRATES}
\par   And does not a man use the whole body?

\par \textbf{ALCIBIADES}
\par   Certainly.

\par \textbf{SOCRATES}
\par   And that which uses is different from that which is used?

\par \textbf{ALCIBIADES}
\par   True.

\par \textbf{SOCRATES}
\par   Then a man is not the same as his own body?

\par \textbf{ALCIBIADES}
\par   That is the inference.

\par \textbf{SOCRATES}
\par   What is he, then?

\par \textbf{ALCIBIADES}
\par   I cannot say.

\par \textbf{SOCRATES}
\par   Nay, you can say that he is the user of the body.

\par \textbf{ALCIBIADES}
\par   Yes.

\par \textbf{SOCRATES}
\par   And the user of the body is the soul?

\par \textbf{ALCIBIADES}
\par   Yes, the soul.

\par \textbf{SOCRATES}
\par   And the soul rules?

\par \textbf{ALCIBIADES}
\par   Yes.

\par \textbf{SOCRATES}
\par   Let me make an assertion which will, I think, be universally admitted.

\par \textbf{ALCIBIADES}
\par   What is it?

\par \textbf{SOCRATES}
\par   That man is one of three things.

\par \textbf{ALCIBIADES}
\par   What are they?

\par \textbf{SOCRATES}
\par   Soul, body, or both together forming a whole.

\par \textbf{ALCIBIADES}
\par   Certainly.

\par \textbf{SOCRATES}
\par   But did we not say that the actual ruling principle of the body is man?

\par \textbf{ALCIBIADES}
\par   Yes, we did.

\par \textbf{SOCRATES}
\par   And does the body rule over itself?

\par \textbf{ALCIBIADES}
\par   Certainly not.

\par \textbf{SOCRATES}
\par   It is subject, as we were saying?

\par \textbf{ALCIBIADES}
\par   Yes.

\par \textbf{SOCRATES}
\par   Then that is not the principle which we are seeking?

\par \textbf{ALCIBIADES}
\par   It would seem not.

\par \textbf{SOCRATES}
\par   But may we say that the union of the two rules over the body, and consequently that this is man?

\par \textbf{ALCIBIADES}
\par   Very likely.

\par \textbf{SOCRATES}
\par   The most unlikely of all things; for if one of the members is subject, the two united cannot possibly rule.

\par \textbf{ALCIBIADES}
\par   True.

\par \textbf{SOCRATES}
\par   But since neither the body, nor the union of the two, is man, either man has no real existence, or the soul is man?

\par \textbf{ALCIBIADES}
\par   Just so.

\par \textbf{SOCRATES}
\par   Is anything more required to prove that the soul is man?

\par \textbf{ALCIBIADES}
\par   Certainly not; the proof is, I think, quite sufficient.

\par \textbf{SOCRATES}
\par   And if the proof, although not perfect, be sufficient, we shall be satisfied;—more precise proof will be supplied when we have discovered that which we were led to omit, from a fear that the enquiry would be too much protracted.

\par \textbf{ALCIBIADES}
\par   What was that?

\par \textbf{SOCRATES}
\par   What I meant, when I said that absolute existence must be first considered; but now, instead of absolute existence, we have been considering the nature of individual existence, and this may, perhaps, be sufficient; for surely there is nothing which may be called more properly ourselves than the soul?

\par \textbf{ALCIBIADES}
\par   There is nothing.

\par \textbf{SOCRATES}
\par   Then we may truly conceive that you and I are conversing with one another, soul to soul?

\par \textbf{ALCIBIADES}
\par   Very true.

\par \textbf{SOCRATES}
\par   And that is just what I was saying before—that I, Socrates, am not arguing or talking with the face of Alcibiades, but with the real Alcibiades; or in other words, with his soul.

\par \textbf{ALCIBIADES}
\par   True.

\par \textbf{SOCRATES}
\par   Then he who bids a man know himself, would have him know his soul?

\par \textbf{ALCIBIADES}
\par   That appears to be true.

\par \textbf{SOCRATES}
\par   He whose knowledge only extends to the body, knows the things of a man, and not the man himself?

\par \textbf{ALCIBIADES}
\par   That is true.

\par \textbf{SOCRATES}
\par   Then neither the physician regarded as a physician, nor the trainer regarded as a trainer, knows himself?

\par \textbf{ALCIBIADES}
\par   He does not.

\par \textbf{SOCRATES}
\par   The husbandmen and the other craftsmen are very far from knowing themselves, for they would seem not even to know their own belongings? When regarded in relation to the arts which they practise they are even further removed from self-knowledge, for they only know the belongings of the body, which minister to the body.

\par \textbf{ALCIBIADES}
\par   That is true.

\par \textbf{SOCRATES}
\par   Then if temperance is the knowledge of self, in respect of his art none of them is temperate?

\par \textbf{ALCIBIADES}
\par   I agree.

\par \textbf{SOCRATES}
\par   And this is the reason why their arts are accounted vulgar, and are not such as a good man would practise?

\par \textbf{ALCIBIADES}
\par   Quite true.

\par \textbf{SOCRATES}
\par   Again, he who cherishes his body cherishes not himself, but what belongs to him?

\par \textbf{ALCIBIADES}
\par   That is true.

\par \textbf{SOCRATES}
\par   But he who cherishes his money, cherishes neither himself nor his belongings, but is in a stage yet further removed from himself?

\par \textbf{ALCIBIADES}
\par   I agree.

\par \textbf{SOCRATES}
\par   Then the money-maker has really ceased to be occupied with his own concerns?

\par \textbf{ALCIBIADES}
\par   True.

\par \textbf{SOCRATES}
\par   And if any one has fallen in love with the person of Alcibiades, he loves not Alcibiades, but the belongings of Alcibiades?

\par \textbf{ALCIBIADES}
\par   True.

\par \textbf{SOCRATES}
\par   But he who loves your soul is the true lover?

\par \textbf{ALCIBIADES}
\par   That is the necessary inference.

\par \textbf{SOCRATES}
\par   The lover of the body goes away when the flower of youth fades?

\par \textbf{ALCIBIADES}
\par   True.

\par \textbf{SOCRATES}
\par   But he who loves the soul goes not away, as long as the soul follows after virtue?

\par \textbf{ALCIBIADES}
\par   Yes.

\par \textbf{SOCRATES}
\par   And I am the lover who goes not away, but remains with you, when you are no longer young and the rest are gone?

\par \textbf{ALCIBIADES}
\par   Yes, Socrates; and therein you do well, and I hope that you will remain.

\par \textbf{SOCRATES}
\par   Then you must try to look your best.

\par \textbf{ALCIBIADES}
\par   I will.

\par \textbf{SOCRATES}
\par   The fact is, that there is only one lover of Alcibiades the son of Cleinias; there neither is nor ever has been seemingly any other; and he is his darling,—Socrates, the son of Sophroniscus and Phaenarete.

\par \textbf{ALCIBIADES}
\par   True.

\par \textbf{SOCRATES}
\par   And did you not say, that if I had not spoken first, you were on the point of coming to me, and enquiring why I only remained?

\par \textbf{ALCIBIADES}
\par   That is true.

\par \textbf{SOCRATES}
\par   The reason was that I loved you for your own sake, whereas other men love what belongs to you; and your beauty, which is not you, is fading away, just as your true self is beginning to bloom. And I will never desert you, if you are not spoiled and deformed by the Athenian people; for the danger which I most fear is that you will become a lover of the people and will be spoiled by them. Many a noble Athenian has been ruined in this way. For the demus of the great-hearted Erechteus is of a fair countenance, but you should see him naked; wherefore observe the caution which I give you.

\par \textbf{ALCIBIADES}
\par   What caution?

\par \textbf{SOCRATES}
\par   Practise yourself, sweet friend, in learning what you ought to know, before you enter on politics; and then you will have an antidote which will keep you out of harm's way.

\par \textbf{ALCIBIADES}
\par   Good advice, Socrates, but I wish that you would explain to me in what way I am to take care of myself.

\par \textbf{SOCRATES}
\par   Have we not made an advance? for we are at any rate tolerably well agreed as to what we are, and there is no longer any danger, as we once feared, that we might be taking care not of ourselves, but of something which is not ourselves.

\par \textbf{ALCIBIADES}
\par   That is true.

\par \textbf{SOCRATES}
\par   And the next step will be to take care of the soul, and look to that?

\par \textbf{ALCIBIADES}
\par   Certainly.

\par \textbf{SOCRATES}
\par   Leaving the care of our bodies and of our properties to others?

\par \textbf{ALCIBIADES}
\par   Very good.

\par \textbf{SOCRATES}
\par   But how can we have a perfect knowledge of the things of the soul?—For if we know them, then I suppose we shall know ourselves. Can we really be ignorant of the excellent meaning of the Delphian inscription, of which we were just now speaking?

\par \textbf{ALCIBIADES}
\par   What have you in your thoughts, Socrates?

\par \textbf{SOCRATES}
\par   I will tell you what I suspect to be the meaning and lesson of that inscription. Let me take an illustration from sight, which I imagine to be the only one suitable to my purpose.

\par \textbf{ALCIBIADES}
\par   What do you mean?

\par \textbf{SOCRATES}
\par   Consider; if some one were to say to the eye, 'See thyself,' as you might say to a man, 'Know thyself,' what is the nature and meaning of this precept? Would not his meaning be: —That the eye should look at that in which it would see itself?

\par \textbf{ALCIBIADES}
\par   Clearly.

\par \textbf{SOCRATES}
\par   And what are the objects in looking at which we see ourselves?

\par \textbf{ALCIBIADES}
\par   Clearly, Socrates, in looking at mirrors and the like.

\par \textbf{SOCRATES}
\par   Very true; and is there not something of the nature of a mirror in our own eyes?

\par \textbf{ALCIBIADES}
\par   Certainly.

\par \textbf{SOCRATES}
\par   Did you ever observe that the face of the person looking into the eye of another is reflected as in a mirror; and in the visual organ which is over against him, and which is called the pupil, there is a sort of image of the person looking?

\par \textbf{ALCIBIADES}
\par   That is quite true.

\par \textbf{SOCRATES}
\par   Then the eye, looking at another eye, and at that in the eye which is most perfect, and which is the instrument of vision, will there see itself?

\par \textbf{ALCIBIADES}
\par   That is evident.

\par \textbf{SOCRATES}
\par   But looking at anything else either in man or in the world, and not to what resembles this, it will not see itself?

\par \textbf{ALCIBIADES}
\par   Very true.

\par \textbf{SOCRATES}
\par   Then if the eye is to see itself, it must look at the eye, and at that part of the eye where sight which is the virtue of the eye resides?

\par \textbf{ALCIBIADES}
\par   True.

\par \textbf{SOCRATES}
\par   And if the soul, my dear Alcibiades, is ever to know herself, must she not look at the soul; and especially at that part of the soul in which her virtue resides, and to any other which is like this?

\par \textbf{ALCIBIADES}
\par   I agree, Socrates.

\par \textbf{SOCRATES}
\par   And do we know of any part of our souls more divine than that which has to do with wisdom and knowledge?

\par \textbf{ALCIBIADES}
\par   There is none.

\par \textbf{SOCRATES}
\par   Then this is that part of the soul which resembles the divine; and he who looks at this and at the whole class of things divine, will be most likely to know himself?

\par \textbf{ALCIBIADES}
\par   Clearly.

\par \textbf{SOCRATES}
\par   And self-knowledge we agree to be wisdom?

\par \textbf{ALCIBIADES}
\par   True.

\par \textbf{SOCRATES}
\par   But if we have no self-knowledge and no wisdom, can we ever know our own good and evil?

\par \textbf{ALCIBIADES}
\par   How can we, Socrates?

\par \textbf{SOCRATES}
\par   You mean, that if you did not know Alcibiades, there would be no possibility of your knowing that what belonged to Alcibiades was really his?

\par \textbf{ALCIBIADES}
\par   It would be quite impossible.

\par \textbf{SOCRATES}
\par   Nor should we know that we were the persons to whom anything belonged, if we did not know ourselves?

\par \textbf{ALCIBIADES}
\par   How could we?

\par \textbf{SOCRATES}
\par   And if we did not know our own belongings, neither should we know the belongings of our belongings?

\par \textbf{ALCIBIADES}
\par   Clearly not.

\par \textbf{SOCRATES}
\par   Then we were not altogether right in acknowledging just now that a man may know what belongs to him and yet not know himself; nay, rather he cannot even know the belongings of his belongings; for the discernment of the things of self, and of the things which belong to the things of self, appear all to be the business of the same man, and of the same art.

\par \textbf{ALCIBIADES}
\par   So much may be supposed.

\par \textbf{SOCRATES}
\par   And he who knows not the things which belong to himself, will in like manner be ignorant of the things which belong to others?

\par \textbf{ALCIBIADES}
\par   Very true.

\par \textbf{SOCRATES}
\par   And if he knows not the affairs of others, he will not know the affairs of states?

\par \textbf{ALCIBIADES}
\par   Certainly not.

\par \textbf{SOCRATES}
\par   Then such a man can never be a statesman?

\par \textbf{ALCIBIADES}
\par   He cannot.

\par \textbf{SOCRATES}
\par   Nor an economist?

\par \textbf{ALCIBIADES}
\par   He cannot.

\par \textbf{SOCRATES}
\par   He will not know what he is doing?

\par \textbf{ALCIBIADES}
\par   He will not.

\par \textbf{SOCRATES}
\par   And will not he who is ignorant fall into error?

\par \textbf{ALCIBIADES}
\par   Assuredly.

\par \textbf{SOCRATES}
\par   And if he falls into error will he not fail both in his public and private capacity?

\par \textbf{ALCIBIADES}
\par   Yes, indeed.

\par \textbf{SOCRATES}
\par   And failing, will he not be miserable?

\par \textbf{ALCIBIADES}
\par   Very.

\par \textbf{SOCRATES}
\par   And what will become of those for whom he is acting?

\par \textbf{ALCIBIADES}
\par   They will be miserable also.

\par \textbf{SOCRATES}
\par   Then he who is not wise and good cannot be happy?

\par \textbf{ALCIBIADES}
\par   He cannot.

\par \textbf{SOCRATES}
\par   The bad, then, are miserable?

\par \textbf{ALCIBIADES}
\par   Yes, very.

\par \textbf{SOCRATES}
\par   And if so, not he who has riches, but he who has wisdom, is delivered from his misery?

\par \textbf{ALCIBIADES}
\par   Clearly.

\par \textbf{SOCRATES}
\par   Cities, then, if they are to be happy, do not want walls, or triremes, or docks, or numbers, or size, Alcibiades, without virtue? (Compare Arist. Pol.)

\par \textbf{ALCIBIADES}
\par   Indeed they do not.

\par \textbf{SOCRATES}
\par   And you must give the citizens virtue, if you mean to administer their affairs rightly or nobly?

\par \textbf{ALCIBIADES}
\par   Certainly.

\par \textbf{SOCRATES}
\par   But can a man give that which he has not?

\par \textbf{ALCIBIADES}
\par   Impossible.

\par \textbf{SOCRATES}
\par   Then you or any one who means to govern and superintend, not only himself and the things of himself, but the state and the things of the state, must in the first place acquire virtue.

\par \textbf{ALCIBIADES}
\par   That is true.

\par \textbf{SOCRATES}
\par   You have not therefore to obtain power or authority, in order to enable you to do what you wish for yourself and the state, but justice and wisdom.

\par \textbf{ALCIBIADES}
\par   Clearly.

\par \textbf{SOCRATES}
\par   You and the state, if you act wisely and justly, will act according to the will of God?

\par \textbf{ALCIBIADES}
\par   Certainly.

\par \textbf{SOCRATES}
\par   As I was saying before, you will look only at what is bright and divine, and act with a view to them?

\par \textbf{ALCIBIADES}
\par   Yes.

\par \textbf{SOCRATES}
\par   In that mirror you will see and know yourselves and your own good?

\par \textbf{ALCIBIADES}
\par   Yes.

\par \textbf{SOCRATES}
\par   And so you will act rightly and well?

\par \textbf{ALCIBIADES}
\par   Yes.

\par \textbf{SOCRATES}
\par   In which case, I will be security for your happiness.

\par \textbf{ALCIBIADES}
\par   I accept the security.

\par \textbf{SOCRATES}
\par   But if you act unrighteously, your eye will turn to the dark and godless, and being in darkness and ignorance of yourselves, you will probably do deeds of darkness.

\par \textbf{ALCIBIADES}
\par   Very possibly.

\par \textbf{SOCRATES}
\par   For if a man, my dear Alcibiades, has the power to do what he likes, but has no understanding, what is likely to be the result, either to him as an individual or to the state—for example, if he be sick and is able to do what he likes, not having the mind of a physician—having moreover tyrannical power, and no one daring to reprove him, what will happen to him? Will he not be likely to have his constitution ruined?

\par \textbf{ALCIBIADES}
\par   That is true.

\par \textbf{SOCRATES}
\par   Or again, in a ship, if a man having the power to do what he likes, has no intelligence or skill in navigation, do you see what will happen to him and to his fellow-sailors?

\par \textbf{ALCIBIADES}
\par   Yes; I see that they will all perish.

\par \textbf{SOCRATES}
\par   And in like manner, in a state, and where there is any power and authority which is wanting in virtue, will not misfortune, in like manner, ensue?

\par \textbf{ALCIBIADES}
\par   Certainly.

\par \textbf{SOCRATES}
\par   Not tyrannical power, then, my good Alcibiades, should be the aim either of individuals or states, if they would be happy, but virtue.

\par \textbf{ALCIBIADES}
\par   That is true.

\par \textbf{SOCRATES}
\par   And before they have virtue, to be commanded by a superior is better for men as well as for children? (Compare Arist. Pol.)

\par \textbf{ALCIBIADES}
\par   That is evident.

\par \textbf{SOCRATES}
\par   And that which is better is also nobler?

\par \textbf{ALCIBIADES}
\par   True.

\par \textbf{SOCRATES}
\par   And what is nobler is more becoming?

\par \textbf{ALCIBIADES}
\par   Certainly.

\par \textbf{SOCRATES}
\par   Then to the bad man slavery is more becoming, because better?

\par \textbf{ALCIBIADES}
\par   True.

\par \textbf{SOCRATES}
\par   Then vice is only suited to a slave?

\par \textbf{ALCIBIADES}
\par   Yes.

\par \textbf{SOCRATES}
\par   And virtue to a freeman?

\par \textbf{ALCIBIADES}
\par   Yes.

\par \textbf{SOCRATES}
\par   And, O my friend, is not the condition of a slave to be avoided?

\par \textbf{ALCIBIADES}
\par   Certainly, Socrates.

\par \textbf{SOCRATES}
\par   And are you now conscious of your own state? And do you know whether you are a freeman or not?

\par \textbf{ALCIBIADES}
\par   I think that I am very conscious indeed of my own state.

\par \textbf{SOCRATES}
\par   And do you know how to escape out of a state which I do not even like to name to my beauty?

\par \textbf{ALCIBIADES}
\par   Yes, I do.

\par \textbf{SOCRATES}
\par   How?

\par \textbf{ALCIBIADES}
\par   By your help, Socrates.

\par \textbf{SOCRATES}
\par   That is not well said, Alcibiades.

\par \textbf{ALCIBIADES}
\par   What ought I to have said?

\par \textbf{SOCRATES}
\par   By the help of God.

\par \textbf{ALCIBIADES}
\par   I agree; and I further say, that our relations are likely to be reversed. From this day forward, I must and will follow you as you have followed me; I will be the disciple, and you shall be my master.

\par \textbf{SOCRATES}
\par   O that is rare! My love breeds another love:  and so like the stork I shall be cherished by the bird whom I have hatched.

\par \textbf{ALCIBIADES}
\par   Strange, but true; and henceforward I shall begin to think about justice.

\par \textbf{SOCRATES}
\par   And I hope that you will persist; although I have fears, not because I doubt you; but I see the power of the state, which may be too much for both of us.

\par 

\end{document}