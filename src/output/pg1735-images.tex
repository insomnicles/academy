
\documentclass[11pt,letter]{article}


\begin{document}

\title{Sophist\thanks{Source: https://www.gutenberg.org/files/1735/1735-h/1735-h.htm. License: http://gutenberg.org/license ds}}
\date{\today}
\author{Plato, 427? BCE-347? BCE\\ Translated by Jowett, Benjamin, 1817-1893}
\maketitle

\setcounter{tocdepth}{1}
\tableofcontents
\renewcommand{\baselinestretch}{1.0}
\normalsize
\newpage

\section{
      INTRODUCTION AND ANALYSIS.
    }
\par  The dramatic power of the dialogues of Plato appears to diminish as the metaphysical interest of them increases (compare Introd. to the Philebus). There are no descriptions of time, place or persons, in the Sophist and Statesman, but we are plunged at once into philosophical discussions; the poetical charm has disappeared, and those who have no taste for abstruse metaphysics will greatly prefer the earlier dialogues to the later ones. Plato is conscious of the change, and in the Statesman expressly accuses himself of a tediousness in the two dialogues, which he ascribes to his desire of developing the dialectical method. On the other hand, the kindred spirit of Hegel seemed to find in the Sophist the crown and summit of the Platonic philosophy—here is the place at which Plato most nearly approaches to the Hegelian identity of Being and Not-being. Nor will the great importance of the two dialogues be doubted by any one who forms a conception of the state of mind and opinion which they are intended to meet. The sophisms of the day were undermining philosophy; the denial of the existence of Not-being, and of the connexion of ideas, was making truth and falsehood equally impossible. It has been said that Plato would have written differently, if he had been acquainted with the Organon of Aristotle. But could the Organon of Aristotle ever have been written unless the Sophist and Statesman had preceded? The swarm of fallacies which arose in the infancy of mental science, and which was born and bred in the decay of the pre-Socratic philosophies, was not dispelled by Aristotle, but by Socrates and Plato. The summa genera of thought, the nature of the proposition, of definition, of generalization, of synthesis and analysis, of division and cross-division, are clearly described, and the processes of induction and deduction are constantly employed in the dialogues of Plato. The 'slippery' nature of comparison, the danger of putting words in the place of things, the fallacy of arguing 'a dicto secundum,' and in a circle, are frequently indicated by him. To all these processes of truth and error, Aristotle, in the next generation, gave distinctness; he brought them together in a separate science. But he is not to be regarded as the original inventor of any of the great logical forms, with the exception of the syllogism.

\par  There is little worthy of remark in the characters of the Sophist. The most noticeable point is the final retirement of Socrates from the field of argument, and the substitution for him of an Eleatic stranger, who is described as a pupil of Parmenides and Zeno, and is supposed to have descended from a higher world in order to convict the Socratic circle of error. As in the Timaeus, Plato seems to intimate by the withdrawal of Socrates that he is passing beyond the limits of his teaching; and in the Sophist and Statesman, as well as in the Parmenides, he probably means to imply that he is making a closer approach to the schools of Elea and Megara. He had much in common with them, but he must first submit their ideas to criticism and revision. He had once thought as he says, speaking by the mouth of the Eleatic, that he understood their doctrine of Not-being; but now he does not even comprehend the nature of Being. The friends of ideas (Soph.) are alluded to by him as distant acquaintances, whom he criticizes ab extra; we do not recognize at first sight that he is criticizing himself. The character of the Eleatic stranger is colourless; he is to a certain extent the reflection of his father and master, Parmenides, who is the protagonist in the dialogue which is called by his name. Theaetetus himself is not distinguished by the remarkable traits which are attributed to him in the preceding dialogue. He is no longer under the spell of Socrates, or subject to the operation of his midwifery, though the fiction of question and answer is still maintained, and the necessity of taking Theaetetus along with him is several times insisted upon by his partner in the discussion. There is a reminiscence of the old Theaetetus in his remark that he will not tire of the argument, and in his conviction, which the Eleatic thinks likely to be permanent, that the course of events is governed by the will of God. Throughout the two dialogues Socrates continues a silent auditor, in the Statesman just reminding us of his presence, at the commencement, by a characteristic jest about the statesman and the philosopher, and by an allusion to his namesake, with whom on that ground he claims relationship, as he had already claimed an affinity with Theaetetus, grounded on the likeness of his ugly face. But in neither dialogue, any more than in the Timaeus, does he offer any criticism on the views which are propounded by another.

\par  The style, though wanting in dramatic power,—in this respect resembling the Philebus and the Laws,—is very clear and accurate, and has several touches of humour and satire. The language is less fanciful and imaginative than that of the earlier dialogues; and there is more of bitterness, as in the Laws, though traces of a similar temper may also be observed in the description of the 'great brute' in the Republic, and in the contrast of the lawyer and philosopher in the Theaetetus. The following are characteristic passages: 'The ancient philosophers, of whom we may say, without offence, that they went on their way rather regardless of whether we understood them or not;' the picture of the materialists, or earth-born giants, 'who grasped oaks and rocks in their hands,' and who must be improved before they can be reasoned with; and the equally humourous delineation of the friends of ideas, who defend themselves from a fastness in the invisible world; or the comparison of the Sophist to a painter or maker (compare Republic), and the hunt after him in the rich meadow-lands of youth and wealth; or, again, the light and graceful touch with which the older philosophies are painted ('Ionian and Sicilian muses'), the comparison of them to mythological tales, and the fear of the Eleatic that he will be counted a parricide if he ventures to lay hands on his father Parmenides; or, once more, the likening of the Eleatic stranger to a god from heaven.—All these passages, notwithstanding the decline of the style, retain the impress of the great master of language. But the equably diffused grace is gone; instead of the endless variety of the early dialogues, traces of the rhythmical monotonous cadence of the Laws begin to appear; and already an approach is made to the technical language of Aristotle, in the frequent use of the words 'essence,' 'power,' 'generation,' 'motion,' 'rest,' 'action,' 'passion,' and the like.

\par  The Sophist, like the Phaedrus, has a double character, and unites two enquirers, which are only in a somewhat forced manner connected with each other. The first is the search after the Sophist, the second is the enquiry into the nature of Not-being, which occupies the middle part of the work. For 'Not-being' is the hole or division of the dialectical net in which the Sophist has hidden himself. He is the imaginary impersonation of false opinion. Yet he denies the possibility of false opinion; for falsehood is that which is not, and therefore has no existence. At length the difficulty is solved; the answer, in the language of the Republic, appears 'tumbling out at our feet.' Acknowledging that there is a communion of kinds with kinds, and not merely one Being or Good having different names, or several isolated ideas or classes incapable of communion, we discover 'Not-being' to be the other of 'Being.' Transferring this to language and thought, we have no difficulty in apprehending that a proposition may be false as well as true. The Sophist, drawn out of the shelter which Cynic and Megarian paradoxes have temporarily afforded him, is proved to be a dissembler and juggler with words.

\par  The chief points of interest in the dialogue are: (I) the character attributed to the Sophist: (II) the dialectical method: (III) the nature of the puzzle about 'Not-being:' (IV) the battle of the philosophers: (V) the relation of the Sophist to other dialogues.

\par  I. The Sophist in Plato is the master of the art of illusion; the charlatan, the foreigner, the prince of esprits-faux, the hireling who is not a teacher, and who, from whatever point of view he is regarded, is the opposite of the true teacher. He is the 'evil one,' the ideal representative of all that Plato most disliked in the moral and intellectual tendencies of his own age; the adversary of the almost equally ideal Socrates. He seems to be always growing in the fancy of Plato, now boastful, now eristic, now clothing himself in rags of philosophy, now more akin to the rhetorician or lawyer, now haranguing, now questioning, until the final appearance in the Politicus of his departing shadow in the disguise of a statesman. We are not to suppose that Plato intended by such a description to depict Protagoras or Gorgias, or even Thrasymachus, who all turn out to be 'very good sort of people when we know them,' and all of them part on good terms with Socrates. But he is speaking of a being as imaginary as the wise man of the Stoics, and whose character varies in different dialogues. Like mythology, Greek philosophy has a tendency to personify ideas. And the Sophist is not merely a teacher of rhetoric for a fee of one or fifty drachmae (Crat. ), but an ideal of Plato's in which the falsehood of all mankind is reflected.

\par  A milder tone is adopted towards the Sophists in a well-known passage of the Republic, where they are described as the followers rather than the leaders of the rest of mankind. Plato ridicules the notion that any individuals can corrupt youth to a degree worth speaking of in comparison with the greater influence of public opinion. But there is no real inconsistency between this and other descriptions of the Sophist which occur in the Platonic writings. For Plato is not justifying the Sophists in the passage just quoted, but only representing their power to be contemptible; they are to be despised rather than feared, and are no worse than the rest of mankind. But a teacher or statesman may be justly condemned, who is on a level with mankind when he ought to be above them. There is another point of view in which this passage should also be considered. The great enemy of Plato is the world, not exactly in the theological sense, yet in one not wholly different—the world as the hater of truth and lover of appearance, occupied in the pursuit of gain and pleasure rather than of knowledge, banded together against the few good and wise men, and devoid of true education. This creature has many heads: rhetoricians, lawyers, statesmen, poets, sophists. But the Sophist is the Proteus who takes the likeness of all of them; all other deceivers have a piece of him in them. And sometimes he is represented as the corrupter of the world; and sometimes the world as the corrupter of him and of itself.

\par  Of late years the Sophists have found an enthusiastic defender in the distinguished historian of Greece. He appears to maintain (1) that the term 'Sophist' is not the name of a particular class, and would have been applied indifferently to Socrates and Plato, as well as to Gorgias and Protagoras; (2) that the bad sense was imprinted on the word by the genius of Plato; (3) that the principal Sophists were not the corrupters of youth (for the Athenian youth were no more corrupted in the age of Demosthenes than in the age of Pericles), but honourable and estimable persons, who supplied a training in literature which was generally wanted at the time. We will briefly consider how far these statements appear to be justified by facts: and, 1, about the meaning of the word there arises an interesting question:—

\par  Many words are used both in a general and a specific sense, and the two senses are not always clearly distinguished. Sometimes the generic meaning has been narrowed to the specific, while in other cases the specific meaning has been enlarged or altered. Examples of the former class are furnished by some ecclesiastical terms: apostles, prophets, bishops, elders, catholics. Examples of the latter class may also be found in a similar field: jesuits, puritans, methodists, and the like. Sometimes the meaning is both narrowed and enlarged; and a good or bad sense will subsist side by side with a neutral one. A curious effect is produced on the meaning of a word when the very term which is stigmatized by the world (e.g. Methodists) is adopted by the obnoxious or derided class; this tends to define the meaning. Or, again, the opposite result is produced, when the world refuses to allow some sect or body of men the possession of an honourable name which they have assumed, or applies it to them only in mockery or irony.

\par  The term 'Sophist' is one of those words of which the meaning has been both contracted and enlarged. Passages may be quoted from Herodotus and the tragedians, in which the word is used in a neutral sense for a contriver or deviser or inventor, without including any ethical idea of goodness or badness. Poets as well as philosophers were called Sophists in the fifth century before Christ. In Plato himself the term is applied in the sense of a 'master in art,' without any bad meaning attaching to it (Symp. ; Meno). In the later Greek, again, 'sophist' and 'philosopher' became almost indistinguishable. There was no reproach conveyed by the word; the additional association, if any, was only that of rhetorician or teacher. Philosophy had become eclecticism and imitation: in the decline of Greek thought there was no original voice lifted up 'which reached to a thousand years because of the god.' Hence the two words, like the characters represented by them, tended to pass into one another. Yet even here some differences appeared; for the term 'Sophist' would hardly have been applied to the greater names, such as Plotinus, and would have been more often used of a professor of philosophy in general than of a maintainer of particular tenets.

\par  But the real question is, not whether the word 'Sophist' has all these senses, but whether there is not also a specific bad sense in which the term is applied to certain contemporaries of Socrates. Would an Athenian, as Mr. Grote supposes, in the fifth century before Christ, have included Socrates and Plato, as well as Gorgias and Protagoras, under the specific class of Sophists? To this question we must answer, No: if ever the term is applied to Socrates and Plato, either the application is made by an enemy out of mere spite, or the sense in which it is used is neutral. Plato, Xenophon, Isocrates, Aristotle, all give a bad import to the word; and the Sophists are regarded as a separate class in all of them. And in later Greek literature, the distinction is quite marked between the succession of philosophers from Thales to Aristotle, and the Sophists of the age of Socrates, who appeared like meteors for a short time in different parts of Greece. For the purposes of comedy, Socrates may have been identified with the Sophists, and he seems to complain of this in the Apology. But there is no reason to suppose that Socrates, differing by so many outward marks, would really have been confounded in the mind of Anytus, or Callicles, or of any intelligent Athenian, with the splendid foreigners who from time to time visited Athens, or appeared at the Olympic games. The man of genius, the great original thinker, the disinterested seeker after truth, the master of repartee whom no one ever defeated in an argument, was separated, even in the mind of the vulgar Athenian, by an 'interval which no geometry can express,' from the balancer of sentences, the interpreter and reciter of the poets, the divider of the meanings of words, the teacher of rhetoric, the professor of morals and manners.

\par  2. The use of the term 'Sophist' in the dialogues of Plato also shows that the bad sense was not affixed by his genius, but already current. When Protagoras says, 'I confess that I am a Sophist,' he implies that the art which he professes has already a bad name; and the words of the young Hippocrates, when with a blush upon his face which is just seen by the light of dawn he admits that he is going to be made 'a Sophist,' would lose their point, unless the term had been discredited. There is nothing surprising in the Sophists having an evil name; that, whether deserved or not, was a natural consequence of their vocation. That they were foreigners, that they made fortunes, that they taught novelties, that they excited the minds of youth, are quite sufficient reasons to account for the opprobrium which attached to them. The genius of Plato could not have stamped the word anew, or have imparted the associations which occur in contemporary writers, such as Xenophon and Isocrates. Changes in the meaning of words can only be made with great difficulty, and not unless they are supported by a strong current of popular feeling. There is nothing improbable in supposing that Plato may have extended and envenomed the meaning, or that he may have done the Sophists the same kind of disservice with posterity which Pascal did to the Jesuits. But the bad sense of the word was not and could not have been invented by him, and is found in his earlier dialogues, e.g. the Protagoras, as well as in the later.

\par  3. There is no ground for disbelieving that the principal Sophists, Gorgias, Protagoras, Prodicus, Hippias, were good and honourable men. The notion that they were corrupters of the Athenian youth has no real foundation, and partly arises out of the use of the term 'Sophist' in modern times. The truth is, that we know little about them; and the witness of Plato in their favour is probably not much more historical than his witness against them. Of that national decline of genius, unity, political force, which has been sometimes described as the corruption of youth, the Sophists were one among many signs;—in these respects Athens may have degenerated; but, as Mr. Grote remarks, there is no reason to suspect any greater moral corruption in the age of Demosthenes than in the age of Pericles. The Athenian youth were not corrupted in this sense, and therefore the Sophists could not have corrupted them. It is remarkable, and may be fairly set down to their credit, that Plato nowhere attributes to them that peculiar Greek sympathy with youth, which he ascribes to Parmenides, and which was evidently common in the Socratic circle. Plato delights to exhibit them in a ludicrous point of view, and to show them always rather at a disadvantage in the company of Socrates. But he has no quarrel with their characters, and does not deny that they are respectable men.

\par  The Sophist, in the dialogue which is called after him, is exhibited in many different lights, and appears and reappears in a variety of forms. There is some want of the higher Platonic art in the Eleatic Stranger eliciting his true character by a labourious process of enquiry, when he had already admitted that he knew quite well the difference between the Sophist and the Philosopher, and had often heard the question discussed;—such an anticipation would hardly have occurred in the earlier dialogues. But Plato could not altogether give up his Socratic method, of which another trace may be thought to be discerned in his adoption of a common instance before he proceeds to the greater matter in hand. Yet the example is also chosen in order to damage the 'hooker of men' as much as possible; each step in the pedigree of the angler suggests some injurious reflection about the Sophist. They are both hunters after a living prey, nearly related to tyrants and thieves, and the Sophist is the cousin of the parasite and flatterer. The effect of this is heightened by the accidental manner in which the discovery is made, as the result of a scientific division. His descent in another branch affords the opportunity of more 'unsavoury comparisons.' For he is a retail trader, and his wares are either imported or home-made, like those of other retail traders; his art is thus deprived of the character of a liberal profession. But the most distinguishing characteristic of him is, that he is a disputant, and higgles over an argument. A feature of the Eristic here seems to blend with Plato's usual description of the Sophists, who in the early dialogues, and in the Republic, are frequently depicted as endeavouring to save themselves from disputing with Socrates by making long orations. In this character he parts company from the vain and impertinent talker in private life, who is a loser of money, while he is a maker of it.

\par  But there is another general division under which his art may be also supposed to fall, and that is purification; and from purification is descended education, and the new principle of education is to interrogate men after the manner of Socrates, and make them teach themselves. Here again we catch a glimpse rather of a Socratic or Eristic than of a Sophist in the ordinary sense of the term. And Plato does not on this ground reject the claim of the Sophist to be the true philosopher. One more feature of the Eristic rather than of the Sophist is the tendency of the troublesome animal to run away into the darkness of Not-being. Upon the whole, we detect in him a sort of hybrid or double nature, of which, except perhaps in the Euthydemus of Plato, we find no other trace in Greek philosophy; he combines the teacher of virtue with the Eristic; while in his omniscience, in his ignorance of himself, in his arts of deception, and in his lawyer-like habit of writing and speaking about all things, he is still the antithesis of Socrates and of the true teacher.

\par  II. The question has been asked, whether the method of 'abscissio infinti,' by which the Sophist is taken, is a real and valuable logical process. Modern science feels that this, like other processes of formal logic, presents a very inadequate conception of the actual complex procedure of the mind by which scientific truth is detected and verified. Plato himself seems to be aware that mere division is an unsafe and uncertain weapon, first, in the Statesman, when he says that we should divide in the middle, for in that way we are more likely to attain species; secondly, in the parallel precept of the Philebus, that we should not pass from the most general notions to infinity, but include all the intervening middle principles, until, as he also says in the Statesman, we arrive at the infima species; thirdly, in the Phaedrus, when he says that the dialectician will carve the limbs of truth without mangling them; and once more in the Statesman, if we cannot bisect species, we must carve them as well as we can. No better image of nature or truth, as an organic whole, can be conceived than this. So far is Plato from supposing that mere division and subdivision of general notions will guide men into all truth.

\par  Plato does not really mean to say that the Sophist or the Statesman can be caught in this way. But these divisions and subdivisions were favourite logical exercises of the age in which he lived; and while indulging his dialectical fancy, and making a contribution to logical method, he delights also to transfix the Eristic Sophist with weapons borrowed from his own armoury. As we have already seen, the division gives him the opportunity of making the most damaging reflections on the Sophist and all his kith and kin, and to exhibit him in the most discreditable light.

\par  Nor need we seriously consider whether Plato was right in assuming that an animal so various could not be confined within the limits of a single definition. In the infancy of logic, men sought only to obtain a definition of an unknown or uncertain term; the after reflection scarcely occurred to them that the word might have several senses, which shaded off into one another, and were not capable of being comprehended in a single notion. There is no trace of this reflection in Plato. But neither is there any reason to think, even if the reflection had occurred to him, that he would have been deterred from carrying on the war with weapons fair or unfair against the outlaw Sophist.

\par  III. The puzzle about 'Not-being' appears to us to be one of the most unreal difficulties of ancient philosophy. We cannot understand the attitude of mind which could imagine that falsehood had no existence, if reality was denied to Not-being: How could such a question arise at all, much less become of serious importance? The answer to this, and to nearly all other difficulties of early Greek philosophy, is to be sought for in the history of ideas, and the answer is only unsatisfactory because our knowledge is defective. In the passage from the world of sense and imagination and common language to that of opinion and reflection the human mind was exposed to many dangers, and often
 
\par  On the other hand, the discovery of abstractions was the great source of all mental improvement in after ages. It was the pushing aside of the old, the revelation of the new. But each one of the company of abstractions, if we may speak in the metaphorical language of Plato, became in turn the tyrant of the mind, the dominant idea, which would allow no other to have a share in the throne. This is especially true of the Eleatic philosophy: while the absoluteness of Being was asserted in every form of language, the sensible world and all the phenomena of experience were comprehended under Not-being. Nor was any difficulty or perplexity thus created, so long as the mind, lost in the contemplation of Being, asked no more questions, and never thought of applying the categories of Being or Not-being to mind or opinion or practical life.

\par  But the negative as well as the positive idea had sunk deep into the intellect of man. The effect of the paradoxes of Zeno extended far beyond the Eleatic circle. And now an unforeseen consequence began to arise. If the Many were not, if all things were names of the One, and nothing could be predicated of any other thing, how could truth be distinguished from falsehood? The Eleatic philosopher would have replied that Being is alone true. But mankind had got beyond his barren abstractions: they were beginning to analyze, to classify, to define, to ask what is the nature of knowledge, opinion, sensation. Still less could they be content with the description which Achilles gives in Homer of the man whom his soul hates—

\par  os chi eteron men keuthe eni phresin, allo de eipe.

\par  For their difficulty was not a practical but a metaphysical one; and their conception of falsehood was really impaired and weakened by a metaphysical illusion.

\par  The strength of the illusion seems to lie in the alternative: If we once admit the existence of Being and Not-being, as two spheres which exclude each other, no Being or reality can be ascribed to Not-being, and therefore not to falsehood, which is the image or expression of Not-being. Falsehood is wholly false; and to speak of true falsehood, as Theaetetus does (Theaet. ), is a contradiction in terms. The fallacy to us is ridiculous and transparent,—no better than those which Plato satirizes in the Euthydemus. It is a confusion of falsehood and negation, from which Plato himself is not entirely free. Instead of saying, 'This is not in accordance with facts,' 'This is proved by experience to be false,' and from such examples forming a general notion of falsehood, the mind of the Greek thinker was lost in the mazes of the Eleatic philosophy. And the greater importance which Plato attributes to this fallacy, compared with others, is due to the influence which the Eleatic philosophy exerted over him. He sees clearly to a certain extent; but he has not yet attained a complete mastery over the ideas of his predecessors—they are still ends to him, and not mere instruments of thought. They are too rough-hewn to be harmonized in a single structure, and may be compared to rocks which project or overhang in some ancient city's walls. There are many such imperfect syncretisms or eclecticisms in the history of philosophy. A modern philosopher, though emancipated from scholastic notions of essence or substance, might still be seriously affected by the abstract idea of necessity; or though accustomed, like Bacon, to criticize abstract notions, might not extend his criticism to the syllogism.

\par  The saying or thinking the thing that is not, would be the popular definition of falsehood or error. If we were met by the Sophist's objection, the reply would probably be an appeal to experience. Ten thousands, as Homer would say (mala murioi), tell falsehoods and fall into errors. And this is Plato's reply, both in the Cratylus and Sophist. 'Theaetetus is flying,' is a sentence in form quite as grammatical as 'Theaetetus is sitting'; the difference between the two sentences is, that the one is true and the other false. But, before making this appeal to common sense, Plato propounds for our consideration a theory of the nature of the negative.

\par  The theory is, that Not-being is relation. Not-being is the other of Being, and has as many kinds as there are differences in Being. This doctrine is the simple converse of the famous proposition of Spinoza,—not 'Omnis determinatio est negatio,' but 'Omnis negatio est determinatio';—not, All distinction is negation, but, All negation is distinction. Not-being is the unfolding or determining of Being, and is a necessary element in all other things that are. We should be careful to observe, first, that Plato does not identify Being with Not-being; he has no idea of progression by antagonism, or of the Hegelian vibration of moments: he would not have said with Heracleitus, 'All things are and are not, and become and become not.' Secondly, he has lost sight altogether of the other sense of Not-being, as the negative of Being; although he again and again recognizes the validity of the law of contradiction. Thirdly, he seems to confuse falsehood with negation. Nor is he quite consistent in regarding Not-being as one class of Being, and yet as coextensive with Being in general. Before analyzing further the topics thus suggested, we will endeavour to trace the manner in which Plato arrived at his conception of Not-being.

\par  In all the later dialogues of Plato, the idea of mind or intelligence becomes more and more prominent. That idea which Anaxagoras employed inconsistently in the construction of the world, Plato, in the Philebus, the Sophist, and the Laws, extends to all things, attributing to Providence a care, infinitesimal as well as infinite, of all creation. The divine mind is the leading religious thought of the later works of Plato. The human mind is a sort of reflection of this, having ideas of Being, Sameness, and the like. At times they seem to be parted by a great gulf (Parmenides); at other times they have a common nature, and the light of a common intelligence.

\par  But this ever-growing idea of mind is really irreconcilable with the abstract Pantheism of the Eleatics. To the passionate language of Parmenides, Plato replies in a strain equally passionate:—What! has not Being mind? and is not Being capable of being known? and, if this is admitted, then capable of being affected or acted upon?—in motion, then, and yet not wholly incapable of rest. Already we have been compelled to attribute opposite determinations to Being. And the answer to the difficulty about Being may be equally the answer to the difficulty about Not-being.

\par  The answer is, that in these and all other determinations of any notion we are attributing to it 'Not-being.' We went in search of Not-being and seemed to lose Being, and now in the hunt after Being we recover both. Not-being is a kind of Being, and in a sense co-extensive with Being. And there are as many divisions of Not-being as of Being. To every positive idea—'just,' 'beautiful,' and the like, there is a corresponding negative idea—'not-just,' 'not-beautiful,' and the like.

\par  A doubt may be raised whether this account of the negative is really the true one. The common logicians would say that the 'not-just,' 'not-beautiful,' are not really classes at all, but are merged in one great class of the infinite or negative. The conception of Plato, in the days before logic, seems to be more correct than this. For the word 'not' does not altogether annihilate the positive meaning of the word 'just': at least, it does not prevent our looking for the 'not-just' in or about the same class in which we might expect to find the 'just.' 'Not-just is not-honourable' is neither a false nor an unmeaning proposition. The reason is that the negative proposition has really passed into an undefined positive. To say that 'not-just' has no more meaning than 'not-honourable'—that is to say, that the two cannot in any degree be distinguished, is clearly repugnant to the common use of language.

\par  The ordinary logic is also jealous of the explanation of negation as relation, because seeming to take away the principle of contradiction. Plato, as far as we know, is the first philosopher who distinctly enunciated this principle; and though we need not suppose him to have been always consistent with himself, there is no real inconsistency between his explanation of the negative and the principle of contradiction. Neither the Platonic notion of the negative as the principle of difference, nor the Hegelian identity of Being and Not-being, at all touch the principle of contradiction. For what is asserted about Being and Not-Being only relates to our most abstract notions, and in no way interferes with the principle of contradiction employed in the concrete. Because Not-being is identified with Other, or Being with Not-being, this does not make the proposition 'Some have not eaten' any the less a contradiction of 'All have eaten.'

\par  The explanation of the negative given by Plato in the Sophist is a true but partial one; for the word 'not,' besides the meaning of 'other,' may also imply 'opposition.' And difference or opposition may be either total or partial: the not-beautiful may be other than the beautiful, or in no relation to the beautiful, or a specific class in various degrees opposed to the beautiful. And the negative may be a negation of fact or of thought (ou and me). Lastly, there are certain ideas, such as 'beginning,' 'becoming,' 'the finite,' 'the abstract,' in which the negative cannot be separated from the positive, and 'Being' and 'Not-being' are inextricably blended.

\par  Plato restricts the conception of Not-being to difference. Man is a rational animal, and is not—as many other things as are not included under this definition. He is and is not, and is because he is not. Besides the positive class to which he belongs, there are endless negative classes to which he may be referred. This is certainly intelligible, but useless. To refer a subject to a negative class is unmeaning, unless the 'not' is a mere modification of the positive, as in the example of 'not honourable' and 'dishonourable'; or unless the class is characterized by the absence rather than the presence of a particular quality.

\par  Nor is it easy to see how Not-being any more than Sameness or Otherness is one of the classes of Being. They are aspects rather than classes of Being. Not-being can only be included in Being, as the denial of some particular class of Being. If we attempt to pursue such airy phantoms at all, the Hegelian identity of Being and Not-being is a more apt and intelligible expression of the same mental phenomenon. For Plato has not distinguished between the Being which is prior to Not-being, and the Being which is the negation of Not-being (compare Parm. ).

\par  But he is not thinking of this when he says that Being comprehends Not-being. Again, we should probably go back for the true explanation to the influence which the Eleatic philosophy exercised over him. Under 'Not-being' the Eleatic had included all the realities of the sensible world. Led by this association and by the common use of language, which has been already noticed, we cannot be much surprised that Plato should have made classes of Not-being. It is observable that he does not absolutely deny that there is an opposite of Being. He is inclined to leave the question, merely remarking that the opposition, if admissible at all, is not expressed by the term 'Not-being.'

\par  On the whole, we must allow that the great service rendered by Plato to metaphysics in the Sophist, is not his explanation of 'Not-being' as difference. With this he certainly laid the ghost of 'Not-being'; and we may attribute to him in a measure the credit of anticipating Spinoza and Hegel. But his conception is not clear or consistent; he does not recognize the different senses of the negative, and he confuses the different classes of Not-being with the abstract notion. As the Pre-Socratic philosopher failed to distinguish between the universal and the true, while he placed the particulars of sense under the false and apparent, so Plato appears to identify negation with falsehood, or is unable to distinguish them. The greatest service rendered by him to mental science is the recognition of the communion of classes, which, although based by him on his account of 'Not-being,' is independent of it. He clearly saw that the isolation of ideas or classes is the annihilation of reasoning. Thus, after wandering in many diverging paths, we return to common sense. And for this reason we may be inclined to do less than justice to Plato,—because the truth which he attains by a real effort of thought is to us a familiar and unconscious truism, which no one would any longer think either of doubting or examining.

\par  IV. The later dialogues of Plato contain many references to contemporary philosophy. Both in the Theaetetus and in the Sophist he recognizes that he is in the midst of a fray; a huge irregular battle everywhere surrounds him (Theaet.). First, there are the two great philosophies going back into cosmogony and poetry: the philosophy of Heracleitus, supposed to have a poetical origin in Homer, and that of the Eleatics, which in a similar spirit he conceives to be even older than Xenophanes (compare Protag.). Still older were theories of two and three principles, hot and cold, moist and dry, which were ever marrying and being given in marriage: in speaking of these, he is probably referring to Pherecydes and the early Ionians. In the philosophy of motion there were different accounts of the relation of plurality and unity, which were supposed to be joined and severed by love and hate, some maintaining that this process was perpetually going on (e.g. Heracleitus); others (e.g. Empedocles) that there was an alternation of them. Of the Pythagoreans or of Anaxagoras he makes no distinct mention. His chief opponents are, first, Eristics or Megarians; secondly, the Materialists.

\par  The picture which he gives of both these latter schools is indistinct; and he appears reluctant to mention the names of their teachers. Nor can we easily determine how much is to be assigned to the Cynics, how much to the Megarians, or whether the 'repellent Materialists' (Theaet.) are Cynics or Atomists, or represent some unknown phase of opinion at Athens. To the Cynics and Antisthenes is commonly attributed, on the authority of Aristotle, the denial of predication, while the Megarians are said to have been Nominalists, asserting the One Good under many names to be the true Being of Zeno and the Eleatics, and, like Zeno, employing their negative dialectic in the refutation of opponents. But the later Megarians also denied predication; and this tenet, which is attributed to all of them by Simplicius, is certainly in accordance with their over-refining philosophy. The 'tyros young and old,' of whom Plato speaks, probably include both. At any rate, we shall be safer in accepting the general description of them which he has given, and in not attempting to draw a precise line between them.

\par  Of these Eristics, whether Cynics or Megarians, several characteristics are found in Plato:—

\par  1. They pursue verbal oppositions; 2. they make reasoning impossible by their over-accuracy in the use of language; 3. they deny predication; 4. they go from unity to plurality, without passing through the intermediate stages; 5. they refuse to attribute motion or power to Being; 6. they are the enemies of sense;—whether they are the 'friends of ideas,' who carry on the polemic against sense, is uncertain; probably under this remarkable expression Plato designates those who more nearly approached himself, and may be criticizing an earlier form of his own doctrines. We may observe (1) that he professes only to give us a few opinions out of many which were at that time current in Greece; (2) that he nowhere alludes to the ethical teaching of the Cynics—unless the argument in the Protagoras, that the virtues are one and not many, may be supposed to contain a reference to their views, as well as to those of Socrates; and unless they are the school alluded to in the Philebus, which is described as 'being very skilful in physics, and as maintaining pleasure to be the absence of pain.' That Antisthenes wrote a book called 'Physicus,' is hardly a sufficient reason for describing them as skilful in physics, which appear to have been very alien to the tendency of the Cynics.

\par  The Idealism of the fourth century before Christ in Greece, as in other ages and countries, seems to have provoked a reaction towards Materialism. The maintainers of this doctrine are described in the Theaetetus as obstinate persons who will believe in nothing which they cannot hold in their hands, and in the Sophist as incapable of argument. They are probably the same who are said in the Tenth Book of the Laws to attribute the course of events to nature, art, and chance. Who they were, we have no means of determining except from Plato's description of them. His silence respecting the Atomists might lead us to suppose that here we have a trace of them. But the Atomists were not Materialists in the grosser sense of the term, nor were they incapable of reasoning; and Plato would hardly have described a great genius like Democritus in the disdainful terms which he uses of the Materialists. Upon the whole, we must infer that the persons here spoken of are unknown to us, like the many other writers and talkers at Athens and elsewhere, of whose endless activity of mind Aristotle in his Metaphysics has preserved an anonymous memorial.

\par  V. The Sophist is the sequel of the Theaetetus, and is connected with the Parmenides by a direct allusion (compare Introductions to Theaetetus and Parmenides). In the Theaetetus we sought to discover the nature of knowledge and false opinion. But the nature of false opinion seemed impenetrable; for we were unable to understand how there could be any reality in Not-being. In the Sophist the question is taken up again; the nature of Not-being is detected, and there is no longer any metaphysical impediment in the way of admitting the possibility of falsehood. To the Parmenides, the Sophist stands in a less defined and more remote relation. There human thought is in process of disorganization; no absurdity or inconsistency is too great to be elicited from the analysis of the simple ideas of Unity or Being. In the Sophist the same contradictions are pursued to a certain extent, but only with a view to their resolution. The aim of the dialogue is to show how the few elemental conceptions of the human mind admit of a natural connexion in thought and speech, which Megarian or other sophistry vainly attempts to deny.

\par  ...

\par  True to the appointment of the previous day, Theodorus and Theaetetus meet Socrates at the same spot, bringing with them an Eleatic Stranger, whom Theodorus introduces as a true philosopher. Socrates, half in jest, half in earnest, declares that he must be a god in disguise, who, as Homer would say, has come to earth that he may visit the good and evil among men, and detect the foolishness of Athenian wisdom. At any rate he is a divine person, one of a class who are hardly recognized on earth; who appear in divers forms—now as statesmen, now as sophists, and are often deemed madmen. 'Philosopher, statesman, sophist,' says Socrates, repeating the words—'I should like to ask our Eleatic friend what his countrymen think of them; do they regard them as one, or three?'

\par  The Stranger has been already asked the same question by Theodorus and Theaetetus; and he at once replies that they are thought to be three; but to explain the difference fully would take time. He is pressed to give this fuller explanation, either in the form of a speech or of question and answer. He prefers the latter, and chooses as his respondent Theaetetus, whom he already knows, and who is recommended to him by Socrates.

\par  We are agreed, he says, about the name Sophist, but we may not be equally agreed about his nature. Great subjects should be approached through familiar examples, and, considering that he is a creature not easily caught, I think that, before approaching him, we should try our hand upon some more obvious animal, who may be made the subject of logical experiment; shall we say an angler? 'Very good.'

\par  In the first place, the angler is an artist; and there are two kinds of art,—productive art, which includes husbandry, manufactures, imitations; and acquisitive art, which includes learning, trading, fighting, hunting. The angler's is an acquisitive art, and acquisition may be effected either by exchange or by conquest; in the latter case, either by force or craft. Conquest by craft is called hunting, and of hunting there is one kind which pursues inanimate, and another which pursues animate objects; and animate objects may be either land animals or water animals, and water animals either fly over the water or live in the water. The hunting of the last is called fishing; and of fishing, one kind uses enclosures, catching the fish in nets and baskets, and another kind strikes them either with spears by night or with barbed spears or barbed hooks by day; the barbed spears are impelled from above, the barbed hooks are jerked into the head and lips of the fish, which are then drawn from below upwards. Thus, by a series of divisions, we have arrived at the definition of the angler's art.

\par  And now by the help of this example we may proceed to bring to light the nature of the Sophist. Like the angler, he is an artist, and the resemblance does not end here. For they are both hunters, and hunters of animals; the one of water, and the other of land animals. But at this point they diverge, the one going to the sea and the rivers, and the other to the rivers of wealth and rich meadow-lands, in which generous youth abide. On land you may hunt tame animals, or you may hunt wild animals. And man is a tame animal, and he may be hunted either by force or persuasion;—either by the pirate, man-stealer, soldier, or by the lawyer, orator, talker. The latter use persuasion, and persuasion is either private or public. Of the private practitioners of the art, some bring gifts to those whom they hunt: these are lovers. And others take hire; and some of these flatter, and in return are fed; others profess to teach virtue and receive a round sum. And who are these last? Tell me who? Have we not unearthed the Sophist?

\par  But he is a many-sided creature, and may still be traced in another line of descent. The acquisitive art had a branch of exchange as well as of hunting, and exchange is either giving or selling; and the seller is either a manufacturer or a merchant; and the merchant either retails or exports; and the exporter may export either food for the body or food for the mind. And of this trading in food for the mind, one kind may be termed the art of display, and another the art of selling learning; and learning may be a learning of the arts or of virtue. The seller of the arts may be called an art-seller; the seller of virtue, a Sophist.

\par  Again, there is a third line, in which a Sophist may be traced. For is he less a Sophist when, instead of exporting his wares to another country, he stays at home, and retails goods, which he not only buys of others, but manufactures himself?

\par  Or he may be descended from the acquisitive art in the combative line, through the pugnacious, the controversial, the disputatious arts; and he will be found at last in the eristic section of the latter, and in that division of it which disputes in private for gain about the general principles of right and wrong.

\par  And still there is a track of him which has not yet been followed out by us. Do not our household servants talk of sifting, straining, winnowing? And they also speak of carding, spinning, and the like. All these are processes of division; and of division there are two kinds,—one in which like is divided from like, and another in which the good is separated from the bad. The latter of the two is termed purification; and again, of purification, there are two sorts,—of animate bodies (which may be internal or external), and of inanimate. Medicine and gymnastic are the internal purifications of the animate, and bathing the external; and of the inanimate, fulling and cleaning and other humble processes, some of which have ludicrous names. Not that dialectic is a respecter of names or persons, or a despiser of humble occupations; nor does she think much of the greater or less benefits conferred by them. For her aim is knowledge; she wants to know how the arts are related to one another, and would quite as soon learn the nature of hunting from the vermin-destroyer as from the general. And she only desires to have a general name, which shall distinguish purifications of the soul from purifications of the body.

\par  Now purification is the taking away of evil; and there are two kinds of evil in the soul,—the one answering to disease in the body, and the other to deformity. Disease is the discord or war of opposite principles in the soul; and deformity is the want of symmetry, or failure in the attainment of a mark or measure. The latter arises from ignorance, and no one is voluntarily ignorant; ignorance is only the aberration of the soul moving towards knowledge. And as medicine cures the diseases and gymnastic the deformity of the body, so correction cures the injustice, and education (which differs among the Hellenes from mere instruction in the arts) cures the ignorance of the soul. Again, ignorance is twofold, simple ignorance, and ignorance having the conceit of knowledge. And education is also twofold: there is the old-fashioned moral training of our forefathers, which was very troublesome and not very successful; and another, of a more subtle nature, which proceeds upon a notion that all ignorance is involuntary. The latter convicts a man out of his own mouth, by pointing out to him his inconsistencies and contradictions; and the consequence is that he quarrels with himself, instead of quarrelling with his neighbours, and is cured of prejudices and obstructions by a mode of treatment which is equally entertaining and effectual. The physician of the soul is aware that his patient will receive no nourishment unless he has been cleaned out; and the soul of the Great King himself, if he has not undergone this purification, is unclean and impure.

\par  And who are the ministers of the purification? Sophists I may not call them. Yet they bear about the same likeness to Sophists as the dog, who is the gentlest of animals, does to the wolf, who is the fiercest. Comparisons are slippery things; but for the present let us assume the resemblance of the two, which may probably be disallowed hereafter. And so, from division comes purification; and from this, mental purification; and from mental purification, instruction; and from instruction, education; and from education, the nobly-descended art of Sophistry, which is engaged in the detection of conceit. I do not however think that we have yet found the Sophist, or that his will ultimately prove to be the desired art of education; but neither do I think that he can long escape me, for every way is blocked. Before we make the final assault, let us take breath, and reckon up the many forms which he has assumed: (1) he was the paid hunter of wealth and birth; (2) he was the trader in the goods of the soul; (3) he was the retailer of them; (4) he was the manufacturer of his own learned wares; (5) he was the disputant; and (6) he was the purger away of prejudices—although this latter point is admitted to be doubtful.

\par  Now, there must surely be something wrong in the professor of any art having so many names and kinds of knowledge. Does not the very number of them imply that the nature of his art is not understood? And that we may not be involved in the misunderstanding, let us observe which of his characteristics is the most prominent. Above all things he is a disputant. He will dispute and teach others to dispute about things visible and invisible—about man, about the gods, about politics, about law, about wrestling, about all things. But can he know all things? 'He cannot.' How then can he dispute satisfactorily with any one who knows? 'Impossible.' Then what is the trick of his art, and why does he receive money from his admirers? 'Because he is believed by them to know all things.' You mean to say that he seems to have a knowledge of them? 'Yes.'

\par  Suppose a person were to say, not that he would dispute about all things, but that he would make all things, you and me, and all other creatures, the earth and the heavens and the gods, and would sell them all for a few pence—this would be a great jest; but not greater than if he said that he knew all things, and could teach them in a short time, and at a small cost. For all imitation is a jest, and the most graceful form of jest. Now the painter is a man who professes to make all things, and children, who see his pictures at a distance, sometimes take them for realities: and the Sophist pretends to know all things, and he, too, can deceive young men, who are still at a distance from the truth, not through their eyes, but through their ears, by the mummery of words, and induce them to believe him. But as they grow older, and come into contact with realities, they learn by experience the futility of his pretensions. The Sophist, then, has not real knowledge; he is only an imitator, or image-maker.

\par  And now, having got him in a corner of the dialectical net, let us divide and subdivide until we catch him. Of image-making there are two kinds,—the art of making likenesses, and the art of making appearances. The latter may be illustrated by sculpture and painting, which often use illusions, and alter the proportions of figures, in order to adapt their works to the eye. And the Sophist also uses illusions, and his imitations are apparent and not real. But how can anything be an appearance only? Here arises a difficulty which has always beset the subject of appearances. For the argument is asserting the existence of not-being. And this is what the great Parmenides was all his life denying in prose and also in verse. 'You will never find,' he says, 'that not-being is.' And the words prove themselves! Not-being cannot be attributed to any being; for how can any being be wholly abstracted from being? Again, in every predication there is an attribution of singular or plural. But number is the most real of all things, and cannot be attributed to not-being. Therefore not-being cannot be predicated or expressed; for how can we say 'is,' 'are not,' without number?

\par  And now arises the greatest difficulty of all. If not-being is inconceivable, how can not-being be refuted? And am I not contradicting myself at this moment, in speaking either in the singular or the plural of that to which I deny both plurality and unity? You, Theaetetus, have the might of youth, and I conjure you to exert yourself, and, if you can, to find an expression for not-being which does not imply being and number. 'But I cannot.' Then the Sophist must be left in his hole. We may call him an image-maker if we please, but he will only say, 'And pray, what is an image?' And we shall reply, 'A reflection in the water, or in a mirror'; and he will say, 'Let us shut our eyes and open our minds; what is the common notion of all images?' 'I should answer, Such another, made in the likeness of the true.' Real or not real? 'Not real; at least, not in a true sense.' And the real 'is,' and the not-real 'is not'? 'Yes.' Then a likeness is really unreal, and essentially not. Here is a pretty complication of being and not-being, in which the many-headed Sophist has entangled us. He will at once point out that he is compelling us to contradict ourselves, by affirming being of not-being. I think that we must cease to look for him in the class of imitators.

\par  But ought we to give him up? 'I should say, certainly not.' Then I fear that I must lay hands on my father Parmenides; but do not call me a parricide; for there is no way out of the difficulty except to show that in some sense not-being is; and if this is not admitted, no one can speak of falsehood, or false opinion, or imitation, without falling into a contradiction. You observe how unwilling I am to undertake the task; for I know that I am exposing myself to the charge of inconsistency in asserting the being of not-being. But if I am to make the attempt, I think that I had better begin at the beginning.

\par  Lightly in the days of our youth, Parmenides and others told us tales about the origin of the universe: one spoke of three principles warring and at peace again, marrying and begetting children; another of two principles, hot and cold, dry and moist, which also formed relationships. There were the Eleatics in our part of the world, saying that all things are one; whose doctrine begins with Xenophanes, and is even older. Ionian, and, more recently, Sicilian muses speak of a one and many which are held together by enmity and friendship, ever parting, ever meeting. Some of them do not insist on the perpetual strife, but adopt a gentler strain, and speak of alternation only. Whether they are right or not, who can say? But one thing we can say—that they went on their way without much caring whether we understood them or not. For tell me, Theaetetus, do you understand what they mean by their assertion of unity, or by their combinations and separations of two or more principles? I used to think, when I was young, that I knew all about not-being, and now I am in great difficulties even about being.

\par  Let us proceed first to the examination of being. Turning to the dualist philosophers, we say to them: Is being a third element besides hot and cold? or do you identify one or both of the two elements with being? At any rate, you can hardly avoid resolving them into one. Let us next interrogate the patrons of the one. To them we say: Are being and one two different names for the same thing? But how can there be two names when there is nothing but one? Or you may identify them; but then the name will be either the name of nothing or of itself, i.e. of a name. Again, the notion of being is conceived of as a whole—in the words of Parmenides, 'like every way unto a rounded sphere.' And a whole has parts; but that which has parts is not one, for unity has no parts. Is being, then, one, because the parts of being are one, or shall we say that being is not a whole? In the former case, one is made up of parts; and in the latter there is still plurality, viz. being, and a whole which is apart from being. And being, if not all things, lacks something of the nature of being, and becomes not-being. Nor can being ever have come into existence, for nothing comes into existence except as a whole; nor can being have number, for that which has number is a whole or sum of number. These are a few of the difficulties which are accumulating one upon another in the consideration of being.

\par  We may proceed now to the less exact sort of philosophers. Some of them drag down everything to earth, and carry on a war like that of the giants, grasping rocks and oaks in their hands. Their adversaries defend themselves warily from an invisible world, and reduce the substances of their opponents to the minutest fractions, until they are lost in generation and flux. The latter sort are civil people enough; but the materialists are rude and ignorant of dialectics; they must be taught how to argue before they can answer. Yet, for the sake of the argument, we may assume them to be better than they are, and able to give an account of themselves. They admit the existence of a mortal living creature, which is a body containing a soul, and to this they would not refuse to attribute qualities—wisdom, folly, justice and injustice. The soul, as they say, has a kind of body, but they do not like to assert of these qualities of the soul, either that they are corporeal, or that they have no existence; at this point they begin to make distinctions. 'Sons of earth,' we say to them, 'if both visible and invisible qualities exist, what is the common nature which is attributed to them by the term "being" or "existence"?' And, as they are incapable of answering this question, we may as well reply for them, that being is the power of doing or suffering. Then we turn to the friends of ideas: to them we say, 'You distinguish becoming from being?' 'Yes,' they will reply. 'And in becoming you participate through the bodily senses, and in being, by thought and the mind?' 'Yes.' And you mean by the word 'participation' a power of doing or suffering? To this they answer—I am acquainted with them, Theaetetus, and know their ways better than you do—that being can neither do nor suffer, though becoming may. And we rejoin: Does not the soul know? And is not 'being' known? And are not 'knowing' and 'being known' active and passive? That which is known is affected by knowledge, and therefore is in motion. And, indeed, how can we imagine that perfect being is a mere everlasting form, devoid of motion and soul? for there can be no thought without soul, nor can soul be devoid of motion. But neither can thought or mind be devoid of some principle of rest or stability. And as children say entreatingly, 'Give us both,' so the philosopher must include both the moveable and immoveable in his idea of being. And yet, alas! he and we are in the same difficulty with which we reproached the dualists; for motion and rest are contradictions—how then can they both exist? Does he who affirms this mean to say that motion is rest, or rest motion? 'No; he means to assert the existence of some third thing, different from them both, which neither rests nor moves.' But how can there be anything which neither rests nor moves? Here is a second difficulty about being, quite as great as that about not-being. And we may hope that any light which is thrown upon the one may extend to the other.

\par  Leaving them for the present, let us enquire what we mean by giving many names to the same thing, e.g. white, good, tall, to man; out of which tyros old and young derive such a feast of amusement. Their meagre minds refuse to predicate anything of anything; they say that good is good, and man is man; and that to affirm one of the other would be making the many one and the one many. Let us place them in a class with our previous opponents, and interrogate both of them at once. Shall we assume (1) that being and rest and motion, and all other things, are incommunicable with one another? or (2) that they all have indiscriminate communion? or (3) that there is communion of some and not of others? And we will consider the first hypothesis first of all.

\par  (1) If we suppose the universal separation of kinds, all theories alike are swept away; the patrons of a single principle of rest or of motion, or of a plurality of immutable ideas—all alike have the ground cut from under them; and all creators of the universe by theories of composition and division, whether out of or into a finite or infinite number of elemental forms, in alternation or continuance, share the same fate. Most ridiculous is the discomfiture which attends the opponents of predication, who, like the ventriloquist Eurycles, have the voice that answers them in their own breast. For they cannot help using the words 'is,' 'apart,' 'from others,' and the like; and their adversaries are thus saved the trouble of refuting them. But (2) if all things have communion with all things, motion will rest, and rest will move; here is a reductio ad absurdum. Two out of the three hypotheses are thus seen to be false. The third (3) remains, which affirms that only certain things communicate with certain other things. In the alphabet and the scale there are some letters and notes which combine with others, and some which do not; and the laws according to which they combine or are separated are known to the grammarian and musician. And there is a science which teaches not only what notes and letters, but what classes admit of combination with one another, and what not. This is a noble science, on which we have stumbled unawares; in seeking after the Sophist we have found the philosopher. He is the master who discerns one whole or form pervading a scattered multitude, and many such wholes combined under a higher one, and many entirely apart—he is the true dialectician. Like the Sophist, he is hard to recognize, though for the opposite reasons; the Sophist runs away into the obscurity of not-being, the philosopher is dark from excess of light. And now, leaving him, we will return to our pursuit of the Sophist.

\par  Agreeing in the truth of the third hypothesis, that some things have communion and others not, and that some may have communion with all, let us examine the most important kinds which are capable of admixture; and in this way we may perhaps find out a sense in which not-being may be affirmed to have being. Now the highest kinds are being, rest, motion; and of these, rest and motion exclude each other, but both of them are included in being; and again, they are the same with themselves and the other of each other. What is the meaning of these words, 'same' and 'other'? Are there two more kinds to be added to the three others? For sameness cannot be either rest or motion, because predicated both of rest and motion; nor yet being; because if being were attributed to both of them we should attribute sameness to both of them. Nor can other be identified with being; for then other, which is relative, would have the absoluteness of being. Therefore we must assume a fifth principle, which is universal, and runs through all things, for each thing is other than all other things. Thus there are five principles: (1) being, (2) motion, which is not (3) rest, and because participating both in the same and other, is and is not (4) the same with itself, and is and is not (5) other than the other. And motion is not being, but partakes of being, and therefore is and is not in the most absolute sense. Thus we have discovered that not-being is the principle of the other which runs through all things, being not excepted. And 'being' is one thing, and 'not-being' includes and is all other things. And not-being is not the opposite of being, but only the other. Knowledge has many branches, and the other or difference has as many, each of which is described by prefixing the word 'not' to some kind of knowledge. The not-beautiful is as real as the beautiful, the not-just as the just. And the essence of the not-beautiful is to be separated from and opposed to a certain kind of existence which is termed beautiful. And this opposition and negation is the not-being of which we are in search, and is one kind of being. Thus, in spite of Parmenides, we have not only discovered the existence, but also the nature of not-being—that nature we have found to be relation. In the communion of different kinds, being and other mutually interpenetrate; other is, but is other than being, and other than each and all of the remaining kinds, and therefore in an infinity of ways 'is not.' And the argument has shown that the pursuit of contradictions is childish and useless, and the very opposite of that higher spirit which criticizes the words of another according to the natural meaning of them. Nothing can be more unphilosophical than the denial of all communion of kinds. And we are fortunate in having established such a communion for another reason, because in continuing the hunt after the Sophist we have to examine the nature of discourse, and there could be no discourse if there were no communion. For the Sophist, although he can no longer deny the existence of not-being, may still affirm that not-being cannot enter into discourse, and as he was arguing before that there could be no such thing as falsehood, because there was no such thing as not-being, he may continue to argue that there is no such thing as the art of image-making and phantastic, because not-being has no place in language. Hence arises the necessity of examining speech, opinion, and imagination.

\par  And first concerning speech; let us ask the same question about words which we have already answered about the kinds of being and the letters of the alphabet: To what extent do they admit of combination? Some words have a meaning when combined, and others have no meaning. One class of words describes action, another class agents: 'walks,' 'runs,' 'sleeps' are examples of the first; 'stag,' 'horse,' 'lion' of the second. But no combination of words can be formed without a verb and a noun, e.g. 'A man learns'; the simplest sentence is composed of two words, and one of these must be a subject. For example, in the sentence, 'Theaetetus sits,' which is not very long, 'Theaetetus' is the subject, and in the sentence 'Theaetetus flies,' 'Theaetetus' is again the subject. But the two sentences differ in quality, for the first says of you that which is true, and the second says of you that which is not true, or, in other words, attributes to you things which are not as though they were. Here is false discourse in the shortest form. And thus not only speech, but thought and opinion and imagination are proved to be both true and false. For thought is only the process of silent speech, and opinion is only the silent assent or denial which follows this, and imagination is only the expression of this in some form of sense. All of them are akin to speech, and therefore, like speech, admit of true and false. And we have discovered false opinion, which is an encouraging sign of our probable success in the rest of the enquiry.

\par  Then now let us return to our old division of likeness-making and phantastic. When we were going to place the Sophist in one of them, a doubt arose whether there could be such a thing as an appearance, because there was no such thing as falsehood. At length falsehood has been discovered by us to exist, and we have acknowledged that the Sophist is to be found in the class of imitators. All art was divided originally by us into two branches—productive and acquisitive. And now we may divide both on a different principle into the creations or imitations which are of human, and those which are of divine, origin. For we must admit that the world and ourselves and the animals did not come into existence by chance, or the spontaneous working of nature, but by divine reason and knowledge. And there are not only divine creations but divine imitations, such as apparitions and shadows and reflections, which are equally the work of a divine mind. And there are human creations and human imitations too,—there is the actual house and the drawing of it. Nor must we forget that image-making may be an imitation of realities or an imitation of appearances, which last has been called by us phantastic. And this phantastic may be again divided into imitation by the help of instruments and impersonations. And the latter may be either dissembling or unconscious, either with or without knowledge. A man cannot imitate you, Theaetetus, without knowing you, but he can imitate the form of justice or virtue if he have a sentiment or opinion about them. Not being well provided with names, the former I will venture to call the imitation of science, and the latter the imitation of opinion.
 
\par  ...

\par  In commenting on the dialogue in which Plato most nearly approaches the great modern master of metaphysics there are several points which it will be useful to consider, such as the unity of opposites, the conception of the ideas as causes, and the relation of the Platonic and Hegelian dialectic.

\par  The unity of opposites was the crux of ancient thinkers in the age of Plato: How could one thing be or become another? That substances have attributes was implied in common language; that heat and cold, day and night, pass into one another was a matter of experience 'on a level with the cobbler's understanding' (Theat.). But how could philosophy explain the connexion of ideas, how justify the passing of them into one another? The abstractions of one, other, being, not-being, rest, motion, individual, universal, which successive generations of philosophers had recently discovered, seemed to be beyond the reach of human thought, like stars shining in a distant heaven. They were the symbols of different schools of philosophy: but in what relation did they stand to one another and to the world of sense? It was hardly conceivable that one could be other, or the same different. Yet without some reconciliation of these elementary ideas thought was impossible. There was no distinction between truth and falsehood, between the Sophist and the philosopher. Everything could be predicated of everything, or nothing of anything. To these difficulties Plato finds what to us appears to be the answer of common sense—that Not-being is the relative or other of Being, the defining and distinguishing principle, and that some ideas combine with others, but not all with all. It is remarkable however that he offers this obvious reply only as the result of a long and tedious enquiry; by a great effort he is able to look down as 'from a height' on the 'friends of the ideas' as well as on the pre-Socratic philosophies. Yet he is merely asserting principles which no one who could be made to understand them would deny.

\par  The Platonic unity of differences or opposites is the beginning of the modern view that all knowledge is of relations; it also anticipates the doctrine of Spinoza that all determination is negation. Plato takes or gives so much of either of these theories as was necessary or possible in the age in which he lived. In the Sophist, as in the Cratylus, he is opposed to the Heracleitean flux and equally to the Megarian and Cynic denial of predication, because he regards both of them as making knowledge impossible. He does not assert that everything is and is not, or that the same thing can be affected in the same and in opposite ways at the same time and in respect of the same part of itself. The law of contradiction is as clearly laid down by him in the Republic, as by Aristotle in his Organon. Yet he is aware that in the negative there is also a positive element, and that oppositions may be only differences. And in the Parmenides he deduces the many from the one and Not-being from Being, and yet shows that the many are included in the one, and that Not-being returns to Being.

\par  In several of the later dialogues Plato is occupied with the connexion of the sciences, which in the Philebus he divides into two classes of pure and applied, adding to them there as elsewhere (Phaedr., Crat., Republic, States.) a superintending science of dialectic. This is the origin of Aristotle's Architectonic, which seems, however, to have passed into an imaginary science of essence, and no longer to retain any relation to other branches of knowledge. Of such a science, whether described as 'philosophia prima,' the science of ousia, logic or metaphysics, philosophers have often dreamed. But even now the time has not arrived when the anticipation of Plato can be realized. Though many a thinker has framed a 'hierarchy of the sciences,' no one has as yet found the higher science which arrays them in harmonious order, giving to the organic and inorganic, to the physical and moral, their respective limits, and showing how they all work together in the world and in man.

\par  Plato arranges in order the stages of knowledge and of existence. They are the steps or grades by which he rises from sense and the shadows of sense to the idea of beauty and good. Mind is in motion as well as at rest (Soph. ); and may be described as a dialectical progress which passes from one limit or determination of thought to another and back again to the first. This is the account of dialectic given by Plato in the Sixth Book of the Republic, which regarded under another aspect is the mysticism of the Symposium. He does not deny the existence of objects of sense, but according to him they only receive their true meaning when they are incorporated in a principle which is above them (Republic). In modern language they might be said to come first in the order of experience, last in the order of nature and reason. They are assumed, as he is fond of repeating, upon the condition that they shall give an account of themselves and that the truth of their existence shall be hereafter proved. For philosophy must begin somewhere and may begin anywhere,—with outward objects, with statements of opinion, with abstract principles. But objects of sense must lead us onward to the ideas or universals which are contained in them; the statements of opinion must be verified; the abstract principles must be filled up and connected with one another. In Plato we find, as we might expect, the germs of many thoughts which have been further developed by the genius of Spinoza and Hegel. But there is a difficulty in separating the germ from the flower, or in drawing the line which divides ancient from modern philosophy. Many coincidences which occur in them are unconscious, seeming to show a natural tendency in the human mind towards certain ideas and forms of thought. And there are many speculations of Plato which would have passed away unheeded, and their meaning, like that of some hieroglyphic, would have remained undeciphered, unless two thousand years and more afterwards an interpreter had arisen of a kindred spirit and of the same intellectual family. For example, in the Sophist Plato begins with the abstract and goes on to the concrete, not in the lower sense of returning to outward objects, but to the Hegelian concrete or unity of abstractions. In the intervening period hardly any importance would have been attached to the question which is so full of meaning to Plato and Hegel.

\par  They differ however in their manner of regarding the question. For Plato is answering a difficulty; he is seeking to justify the use of common language and of ordinary thought into which philosophy had introduced a principle of doubt and dissolution. Whereas Hegel tries to go beyond common thought, and to combine abstractions in a higher unity: the ordinary mechanism of language and logic is carried by him into another region in which all oppositions are absorbed and all contradictions affirmed, only that they may be done away with. But Plato, unlike Hegel, nowhere bases his system on the unity of opposites, although in the Parmenides he shows an Hegelian subtlety in the analysis of one and Being.

\par  It is difficult within the compass of a few pages to give even a faint outline of the Hegelian dialectic. No philosophy which is worth understanding can be understood in a moment; common sense will not teach us metaphysics any more than mathematics. If all sciences demand of us protracted study and attention, the highest of all can hardly be matter of immediate intuition. Neither can we appreciate a great system without yielding a half assent to it—like flies we are caught in the spider's web; and we can only judge of it truly when we place ourselves at a distance from it. Of all philosophies Hegelianism is the most obscure: and the difficulty inherent in the subject is increased by the use of a technical language. The saying of Socrates respecting the writings of Heracleitus—'Noble is that which I understand, and that which I do not understand may be as noble; but the strength of a Delian diver is needed to swim through it'—expresses the feeling with which the reader rises from the perusal of Hegel. We may truly apply to him the words in which Plato describes the Pre-Socratic philosophers: 'He went on his way rather regardless of whether we understood him or not'; or, as he is reported himself to have said of his own pupils: 'There is only one of you who understands me, and he does NOT understand me.'

\par  Nevertheless the consideration of a few general aspects of the Hegelian philosophy may help to dispel some errors and to awaken an interest about it. (i) It is an ideal philosophy which, in popular phraseology, maintains not matter but mind to be the truth of things, and this not by a mere crude substitution of one word for another, but by showing either of them to be the complement of the other. Both are creations of thought, and the difference in kind which seems to divide them may also be regarded as a difference of degree. One is to the other as the real to the ideal, and both may be conceived together under the higher form of the notion. (ii) Under another aspect it views all the forms of sense and knowledge as stages of thought which have always existed implicitly and unconsciously, and to which the mind of the world, gradually disengaged from sense, has become awakened. The present has been the past. The succession in time of human ideas is also the eternal 'now'; it is historical and also a divine ideal. The history of philosophy stripped of personality and of the other accidents of time and place is gathered up into philosophy, and again philosophy clothed in circumstance expands into history. (iii) Whether regarded as present or past, under the form of time or of eternity, the spirit of dialectic is always moving onwards from one determination of thought to another, receiving each successive system of philosophy and subordinating it to that which follows—impelled by an irresistible necessity from one idea to another until the cycle of human thought and existence is complete. It follows from this that all previous philosophies which are worthy of the name are not mere opinions or speculations, but stages or moments of thought which have a necessary place in the world of mind. They are no longer the last word of philosophy, for another and another has succeeded them, but they still live and are mighty; in the language of the Greek poet, 'There is a great God in them, and he grows not old.' (iv) This vast ideal system is supposed to be based upon experience. At each step it professes to carry with it the 'witness of eyes and ears' and of common sense, as well as the internal evidence of its own consistency; it has a place for every science, and affirms that no philosophy of a narrower type is capable of comprehending all true facts.

\par  The Hegelian dialectic may be also described as a movement from the simple to the complex. Beginning with the generalizations of sense, (1) passing through ideas of quality, quantity, measure, number, and the like, (2) ascending from presentations, that is pictorial forms of sense, to representations in which the picture vanishes and the essence is detached in thought from the outward form, (3) combining the I and the not-I, or the subject and object, the natural order of thought is at last found to include the leading ideas of the sciences and to arrange them in relation to one another. Abstractions grow together and again become concrete in a new and higher sense. They also admit of development from within their own spheres. Everywhere there is a movement of attraction and repulsion going on—an attraction or repulsion of ideas of which the physical phenomenon described under a similar name is a figure. Freedom and necessity, mind and matter, the continuous and the discrete, cause and effect, are perpetually being severed from one another in thought, only to be perpetually reunited. The finite and infinite, the absolute and relative are not really opposed; the finite and the negation of the finite are alike lost in a higher or positive infinity, and the absolute is the sum or correlation of all relatives. When this reconciliation of opposites is finally completed in all its stages, the mind may come back again and review the things of sense, the opinions of philosophers, the strife of theology and politics, without being disturbed by them. Whatever is, if not the very best—and what is the best, who can tell?—is, at any rate, historical and rational, suitable to its own age, unsuitable to any other. Nor can any efforts of speculative thinkers or of soldiers and statesmen materially quicken the 'process of the suns.'

\par  Hegel was quite sensible how great would be the difficulty of presenting philosophy to mankind under the form of opposites. Most of us live in the one-sided truth which the understanding offers to us, and if occasionally we come across difficulties like the time-honoured controversy of necessity and free-will, or the Eleatic puzzle of Achilles and the tortoise, we relegate some of them to the sphere of mystery, others to the book of riddles, and go on our way rejoicing. Most men (like Aristotle) have been accustomed to regard a contradiction in terms as the end of strife; to be told that contradiction is the life and mainspring of the intellectual world is indeed a paradox to them. Every abstraction is at first the enemy of every other, yet they are linked together, each with all, in the chain of Being. The struggle for existence is not confined to the animals, but appears in the kingdom of thought. The divisions which arise in thought between the physical and moral and between the moral and intellectual, and the like, are deepened and widened by the formal logic which elevates the defects of the human faculties into Laws of Thought; they become a part of the mind which makes them and is also made up of them. Such distinctions become so familiar to us that we regard the thing signified by them as absolutely fixed and defined. These are some of the illusions from which Hegel delivers us by placing us above ourselves, by teaching us to analyze the growth of 'what we are pleased to call our minds,' by reverting to a time when our present distinctions of thought and language had no existence.

\par  Of the great dislike and childish impatience of his system which would be aroused among his opponents, he was fully aware, and would often anticipate the jests which the rest of the world, 'in the superfluity of their wits,' were likely to make upon him. Men are annoyed at what puzzles them; they think what they cannot easily understand to be full of danger. Many a sceptic has stood, as he supposed, firmly rooted in the categories of the understanding which Hegel resolves into their original nothingness. For, like Plato, he 'leaves no stone unturned' in the intellectual world. Nor can we deny that he is unnecessarily difficult, or that his own mind, like that of all metaphysicians, was too much under the dominion of his system and unable to see beyond: or that the study of philosophy, if made a serious business (compare Republic), involves grave results to the mind and life of the student. For it may encumber him without enlightening his path; and it may weaken his natural faculties of thought and expression without increasing his philosophical power. The mind easily becomes entangled among abstractions, and loses hold of facts. The glass which is adapted to distant objects takes away the vision of what is near and present to us.

\par  To Hegel, as to the ancient Greek thinkers, philosophy was a religion, a principle of life as well as of knowledge, like the idea of good in the Sixth Book of the Republic, a cause as well as an effect, the source of growth as well as of light. In forms of thought which by most of us are regarded as mere categories, he saw or thought that he saw a gradual revelation of the Divine Being. He would have been said by his opponents to have confused God with the history of philosophy, and to have been incapable of distinguishing ideas from facts. And certainly we can scarcely understand how a deep thinker like Hegel could have hoped to revive or supplant the old traditional faith by an unintelligible abstraction: or how he could have imagined that philosophy consisted only or chiefly in the categories of logic. For abstractions, though combined by him in the notion, seem to be never really concrete; they are a metaphysical anatomy, not a living and thinking substance. Though we are reminded by him again and again that we are gathering up the world in ideas, we feel after all that we have not really spanned the gulf which separates phainomena from onta.

\par  Having in view some of these difficulties, he seeks—and we may follow his example—to make the understanding of his system easier (a) by illustrations, and (b) by pointing out the coincidence of the speculative idea and the historical order of thought.

\par  (a) If we ask how opposites can coexist, we are told that many different qualities inhere in a flower or a tree or in any other concrete object, and that any conception of space or matter or time involves the two contradictory attributes of divisibility and continuousness. We may ponder over the thought of number, reminding ourselves that every unit both implies and denies the existence of every other, and that the one is many—a sum of fractions, and the many one—a sum of units. We may be reminded that in nature there is a centripetal as well as a centrifugal force, a regulator as well as a spring, a law of attraction as well as of repulsion. The way to the West is the way also to the East; the north pole of the magnet cannot be divided from the south pole; two minus signs make a plus in Arithmetic and Algebra. Again, we may liken the successive layers of thought to the deposits of geological strata which were once fluid and are now solid, which were at one time uppermost in the series and are now hidden in the earth; or to the successive rinds or barks of trees which year by year pass inward; or to the ripple of water which appears and reappears in an ever-widening circle. Or our attention may be drawn to ideas which the moment we analyze them involve a contradiction, such as 'beginning' or 'becoming,' or to the opposite poles, as they are sometimes termed, of necessity and freedom, of idea and fact. We may be told to observe that every negative is a positive, that differences of kind are resolvable into differences of degree, and that differences of degree may be heightened into differences of kind. We may remember the common remark that there is much to be said on both sides of a question. We may be recommended to look within and to explain how opposite ideas can coexist in our own minds; and we may be told to imagine the minds of all mankind as one mind in which the true ideas of all ages and countries inhere. In our conception of God in his relation to man or of any union of the divine and human nature, a contradiction appears to be unavoidable. Is not the reconciliation of mind and body a necessity, not only of speculation but of practical life? Reflections such as these will furnish the best preparation and give the right attitude of mind for understanding the Hegelian philosophy.

\par  (b) Hegel's treatment of the early Greek thinkers affords the readiest illustration of his meaning in conceiving all philosophy under the form of opposites. The first abstraction is to him the beginning of thought. Hitherto there had only existed a tumultuous chaos of mythological fancy, but when Thales said 'All is water' a new era began to dawn upon the world. Man was seeking to grasp the universe under a single form which was at first simply a material element, the most equable and colourless and universal which could be found. But soon the human mind became dissatisfied with the emblem, and after ringing the changes on one element after another, demanded a more abstract and perfect conception, such as one or Being, which was absolutely at rest. But the positive had its negative, the conception of Being involved Not-being, the conception of one, many, the conception of a whole, parts. Then the pendulum swung to the other side, from rest to motion, from Xenophanes to Heracleitus. The opposition of Being and Not-being projected into space became the atoms and void of Leucippus and Democritus. Until the Atomists, the abstraction of the individual did not exist; in the philosophy of Anaxagoras the idea of mind, whether human or divine, was beginning to be realized. The pendulum gave another swing, from the individual to the universal, from the object to the subject. The Sophist first uttered the word 'Man is the measure of all things,' which Socrates presented in a new form as the study of ethics. Once more we return from mind to the object of mind, which is knowledge, and out of knowledge the various degrees or kinds of knowledge more or less abstract were gradually developed. The threefold division of logic, physic, and ethics, foreshadowed in Plato, was finally established by Aristotle and the Stoics. Thus, according to Hegel, in the course of about two centuries by a process of antagonism and negation the leading thoughts of philosophy were evolved.

\par  There is nothing like this progress of opposites in Plato, who in the Symposium denies the possibility of reconciliation until the opposition has passed away. In his own words, there is an absurdity in supposing that 'harmony is discord; for in reality harmony consists of notes of a higher and lower pitch which disagreed once, but are now reconciled by the art of music' (Symp.). He does indeed describe objects of sense as regarded by us sometimes from one point of view and sometimes from another. As he says at the end of the Fifth Book of the Republic, 'There is nothing light which is not heavy, or great which is not small.' And he extends this relativity to the conceptions of just and good, as well as to great and small. In like manner he acknowledges that the same number may be more or less in relation to other numbers without any increase or diminution (Theat.). But the perplexity only arises out of the confusion of the human faculties; the art of measuring shows us what is truly great and truly small. Though the just and good in particular instances may vary, the IDEA of good is eternal and unchangeable. And the IDEA of good is the source of knowledge and also of Being, in which all the stages of sense and knowledge are gathered up and from being hypotheses become realities.

\par  Leaving the comparison with Plato we may now consider the value of this invention of Hegel. There can be no question of the importance of showing that two contraries or contradictories may in certain cases be both true. The silliness of the so-called laws of thought ('All A = A,' or, in the negative form, 'Nothing can at the same time be both A, and not A') has been well exposed by Hegel himself (Wallace's Hegel), who remarks that 'the form of the maxim is virtually self-contradictory, for a proposition implies a distinction between subject and predicate, whereas the maxim of identity, as it is called, A = A, does not fulfil what its form requires. Nor does any mind ever think or form conceptions in accordance with this law, nor does any existence conform to it.' Wisdom of this sort is well parodied in Shakespeare (Twelfth Night, 'Clown: For as the old hermit of Prague, that never saw pen and ink, very wittily said to a niece of King Gorboduc, "That that is is"...for what is "that" but "that," and "is" but "is"?'). Unless we are willing to admit that two contradictories may be true, many questions which lie at the threshold of mathematics and of morals will be insoluble puzzles to us.

\par  The influence of opposites is felt in practical life. The understanding sees one side of a question only—the common sense of mankind joins one of two parties in politics, in religion, in philosophy. Yet, as everybody knows, truth is not wholly the possession of either. But the characters of men are one-sided and accept this or that aspect of the truth. The understanding is strong in a single abstract principle and with this lever moves mankind. Few attain to a balance of principles or recognize truly how in all human things there is a thesis and antithesis, a law of action and of reaction. In politics we require order as well as liberty, and have to consider the proportions in which under given circumstances they may be safely combined. In religion there is a tendency to lose sight of morality, to separate goodness from the love of truth, to worship God without attempting to know him. In philosophy again there are two opposite principles, of immediate experience and of those general or a priori truths which are supposed to transcend experience. But the common sense or common opinion of mankind is incapable of apprehending these opposite sides or views—men are determined by their natural bent to one or other of them; they go straight on for a time in a single line, and may be many things by turns but not at once.

\par  Hence the importance of familiarizing the mind with forms which will assist us in conceiving or expressing the complex or contrary aspects of life and nature. The danger is that they may be too much for us, and obscure our appreciation of facts. As the complexity of mechanics cannot be understood without mathematics, so neither can the many-sidedness of the mental and moral world be truly apprehended without the assistance of new forms of thought. One of these forms is the unity of opposites. Abstractions have a great power over us, but they are apt to be partial and one-sided, and only when modified by other abstractions do they make an approach to the truth. Many a man has become a fatalist because he has fallen under the dominion of a single idea. He says to himself, for example, that he must be either free or necessary—he cannot be both. Thus in the ancient world whole schools of philosophy passed away in the vain attempt to solve the problem of the continuity or divisibility of matter. And in comparatively modern times, though in the spirit of an ancient philosopher, Bishop Berkeley, feeling a similar perplexity, is inclined to deny the truth of infinitesimals in mathematics. Many difficulties arise in practical religion from the impossibility of conceiving body and mind at once and in adjusting their movements to one another. There is a border ground between them which seems to belong to both; and there is as much difficulty in conceiving the body without the soul as the soul without the body. To the 'either' and 'or' philosophy ('Everything is either A or not A') should at least be added the clause 'or neither,' 'or both.' The double form makes reflection easier and more conformable to experience, and also more comprehensive. But in order to avoid paradox and the danger of giving offence to the unmetaphysical part of mankind, we may speak of it as due to the imperfection of language or the limitation of human faculties. It is nevertheless a discovery which, in Platonic language, may be termed a 'most gracious aid to thought.'

\par  The doctrine of opposite moments of thought or of progression by antagonism, further assists us in framing a scheme or system of the sciences. The negation of one gives birth to another of them. The double notions are the joints which hold them together. The simple is developed into the complex, the complex returns again into the simple. Beginning with the highest notion of mind or thought, we may descend by a series of negations to the first generalizations of sense. Or again we may begin with the simplest elements of sense and proceed upwards to the highest being or thought. Metaphysic is the negation or absorption of physiology—physiology of chemistry—chemistry of mechanical philosophy. Similarly in mechanics, when we can no further go we arrive at chemistry—when chemistry becomes organic we arrive at physiology: when we pass from the outward and animal to the inward nature of man we arrive at moral and metaphysical philosophy. These sciences have each of them their own methods and are pursued independently of one another. But to the mind of the thinker they are all one—latent in one another—developed out of one another.

\par  This method of opposites has supplied new instruments of thought for the solution of metaphysical problems, and has thrown down many of the walls within which the human mind was confined. Formerly when philosophers arrived at the infinite and absolute, they seemed to be lost in a region beyond human comprehension. But Hegel has shown that the absolute and infinite are no more true than the relative and finite, and that they must alike be negatived before we arrive at a true absolute or a true infinite. The conceptions of the infinite and absolute as ordinarily understood are tiresome because they are unmeaning, but there is no peculiar sanctity or mystery in them. We might as well make an infinitesimal series of fractions or a perpetually recurring decimal the object of our worship. They are the widest and also the thinnest of human ideas, or, in the language of logicians, they have the greatest extension and the least comprehension. Of all words they may be truly said to be the most inflated with a false meaning. They have been handed down from one philosopher to another until they have acquired a religious character. They seem also to derive a sacredness from their association with the Divine Being. Yet they are the poorest of the predicates under which we describe him—signifying no more than this, that he is not finite, that he is not relative, and tending to obscure his higher attributes of wisdom, goodness, truth.

\par  The system of Hegel frees the mind from the dominion of abstract ideas. We acknowledge his originality, and some of us delight to wander in the mazes of thought which he has opened to us. For Hegel has found admirers in England and Scotland when his popularity in Germany has departed, and he, like the philosophers whom he criticizes, is of the past. No other thinker has ever dissected the human mind with equal patience and minuteness. He has lightened the burden of thought because he has shown us that the chains which we wear are of our own forging. To be able to place ourselves not only above the opinions of men but above their modes of thinking, is a great height of philosophy. This dearly obtained freedom, however, we are not disposed to part with, or to allow him to build up in a new form the 'beggarly elements' of scholastic logic which he has thrown down. So far as they are aids to reflection and expression, forms of thought are useful, but no further:—we may easily have too many of them.

\par  And when we are asked to believe the Hegelian to be the sole or universal logic, we naturally reply that there are other ways in which our ideas may be connected. The triplets of Hegel, the division into being, essence, and notion, are not the only or necessary modes in which the world of thought can be conceived. There may be an evolution by degrees as well as by opposites. The word 'continuity' suggests the possibility of resolving all differences into differences of quantity. Again, the opposites themselves may vary from the least degree of diversity up to contradictory opposition. They are not like numbers and figures, always and everywhere of the same value. And therefore the edifice which is constructed out of them has merely an imaginary symmetry, and is really irregular and out of proportion. The spirit of Hegelian criticism should be applied to his own system, and the terms Being, Not-being, existence, essence, notion, and the like challenged and defined. For if Hegel introduces a great many distinctions, he obliterates a great many others by the help of the universal solvent 'is not,' which appears to be the simplest of negations, and yet admits of several meanings. Neither are we able to follow him in the play of metaphysical fancy which conducts him from one determination of thought to another. But we begin to suspect that this vast system is not God within us, or God immanent in the world, and may be only the invention of an individual brain. The 'beyond' is always coming back upon us however often we expel it. We do not easily believe that we have within the compass of the mind the form of universal knowledge. We rather incline to think that the method of knowledge is inseparable from actual knowledge, and wait to see what new forms may be developed out of our increasing experience and observation of man and nature. We are conscious of a Being who is without us as well as within us. Even if inclined to Pantheism we are unwilling to imagine that the meagre categories of the understanding, however ingeniously arranged or displayed, are the image of God;—that what all religions were seeking after from the beginning was the Hegelian philosophy which has been revealed in the latter days. The great metaphysician, like a prophet of old, was naturally inclined to believe that his own thoughts were divine realities. We may almost say that whatever came into his head seemed to him to be a necessary truth. He never appears to have criticized himself, or to have subjected his own ideas to the process of analysis which he applies to every other philosopher.

\par  Hegel would have insisted that his philosophy should be accepted as a whole or not at all. He would have urged that the parts derived their meaning from one another and from the whole. He thought that he had supplied an outline large enough to contain all future knowledge, and a method to which all future philosophies must conform. His metaphysical genius is especially shown in the construction of the categories—a work which was only begun by Kant, and elaborated to the utmost by himself. But is it really true that the part has no meaning when separated from the whole, or that knowledge to be knowledge at all must be universal? Do all abstractions shine only by the reflected light of other abstractions? May they not also find a nearer explanation in their relation to phenomena? If many of them are correlatives they are not all so, and the relations which subsist between them vary from a mere association up to a necessary connexion. Nor is it easy to determine how far the unknown element affects the known, whether, for example, new discoveries may not one day supersede our most elementary notions about nature. To a certain extent all our knowledge is conditional upon what may be known in future ages of the world. We must admit this hypothetical element, which we cannot get rid of by an assumption that we have already discovered the method to which all philosophy must conform. Hegel is right in preferring the concrete to the abstract, in setting actuality before possibility, in excluding from the philosopher's vocabulary the word 'inconceivable.' But he is too well satisfied with his own system ever to consider the effect of what is unknown on the element which is known. To the Hegelian all things are plain and clear, while he who is outside the charmed circle is in the mire of ignorance and 'logical impurity': he who is within is omniscient, or at least has all the elements of knowledge under his hand.

\par  Hegelianism may be said to be a transcendental defence of the world as it is. There is no room for aspiration and no need of any: 'What is actual is rational, what is rational is actual.' But a good man will not readily acquiesce in this aphorism. He knows of course that all things proceed according to law whether for good or evil. But when he sees the misery and ignorance of mankind he is convinced that without any interruption of the uniformity of nature the condition of the world may be indefinitely improved by human effort. There is also an adaptation of persons to times and countries, but this is very far from being the fulfilment of their higher natures. The man of the seventeenth century is unfitted for the eighteenth, and the man of the eighteenth for the nineteenth, and most of us would be out of place in the world of a hundred years hence. But all higher minds are much more akin than they are different: genius is of all ages, and there is perhaps more uniformity in excellence than in mediocrity. The sublimer intelligences of mankind—Plato, Dante, Sir Thomas More—meet in a higher sphere above the ordinary ways of men; they understand one another from afar, notwithstanding the interval which separates them. They are 'the spectators of all time and of all existence;' their works live for ever; and there is nothing to prevent the force of their individuality breaking through the uniformity which surrounds them. But such disturbers of the order of thought Hegel is reluctant to acknowledge.

\par  The doctrine of Hegel will to many seem the expression of an indolent conservatism, and will at any rate be made an excuse for it. The mind of the patriot rebels when he is told that the worst tyranny and oppression has a natural fitness: he cannot be persuaded, for example, that the conquest of Prussia by Napoleon I. was either natural or necessary, or that any similar calamity befalling a nation should be a matter of indifference to the poet or philosopher. We may need such a philosophy or religion to console us under evils which are irremediable, but we see that it is fatal to the higher life of man. It seems to say to us, 'The world is a vast system or machine which can be conceived under the forms of logic, but in which no single man can do any great good or any great harm. Even if it were a thousand times worse than it is, it could be arranged in categories and explained by philosophers. And what more do we want?'

\par  The philosophy of Hegel appeals to an historical criterion: the ideas of men have a succession in time as well as an order of thought. But the assumption that there is a correspondence between the succession of ideas in history and the natural order of philosophy is hardly true even of the beginnings of thought. And in later systems forms of thought are too numerous and complex to admit of our tracing in them a regular succession. They seem also to be in part reflections of the past, and it is difficult to separate in them what is original and what is borrowed. Doubtless they have a relation to one another—the transition from Descartes to Spinoza or from Locke to Berkeley is not a matter of chance, but it can hardly be described as an alternation of opposites or figured to the mind by the vibrations of a pendulum. Even in Aristotle and Plato, rightly understood, we cannot trace this law of action and reaction. They are both idealists, although to the one the idea is actual and immanent,—to the other only potential and transcendent, as Hegel himself has pointed out (Wallace's Hegel). The true meaning of Aristotle has been disguised from us by his own appeal to fact and the opinions of mankind in his more popular works, and by the use made of his writings in the Middle Ages. No book, except the Scriptures, has been so much read, and so little understood. The Pre-Socratic philosophies are simpler, and we may observe a progress in them; but is there any regular succession? The ideas of Being, change, number, seem to have sprung up contemporaneously in different parts of Greece and we have no difficulty in constructing them out of one another—we can see that the union of Being and Not-being gave birth to the idea of change or Becoming and that one might be another aspect of Being. Again, the Eleatics may be regarded as developing in one direction into the Megarian school, in the other into the Atomists, but there is no necessary connexion between them. Nor is there any indication that the deficiency which was felt in one school was supplemented or compensated by another. They were all efforts to supply the want which the Greeks began to feel at the beginning of the sixth century before Christ,—the want of abstract ideas. Nor must we forget the uncertainty of chronology;—if, as Aristotle says, there were Atomists before Leucippus, Eleatics before Xenophanes, and perhaps 'patrons of the flux' before Heracleitus, Hegel's order of thought in the history of philosophy would be as much disarranged as his order of religious thought by recent discoveries in the history of religion.

\par  Hegel is fond of repeating that all philosophies still live and that the earlier are preserved in the later; they are refuted, and they are not refuted, by those who succeed them. Once they reigned supreme, now they are subordinated to a power or idea greater or more comprehensive than their own. The thoughts of Socrates and Plato and Aristotle have certainly sunk deep into the mind of the world, and have exercised an influence which will never pass away; but can we say that they have the same meaning in modern and ancient philosophy? Some of them, as for example the words 'Being,' 'essence,' 'matter,' 'form,' either have become obsolete, or are used in new senses, whereas 'individual,' 'cause,' 'motive,' have acquired an exaggerated importance. Is the manner in which the logical determinations of thought, or 'categories' as they may be termed, have been handed down to us, really different from that in which other words have come down to us? Have they not been equally subject to accident, and are they not often used by Hegel himself in senses which would have been quite unintelligible to their original inventors—as for example, when he speaks of the 'ground' of Leibnitz ('Everything has a sufficient ground') as identical with his own doctrine of the 'notion' (Wallace's Hegel), or the 'Being and Not-being' of Heracleitus as the same with his own 'Becoming'?

\par  As the historical order of thought has been adapted to the logical, so we have reason for suspecting that the Hegelian logic has been in some degree adapted to the order of thought in history. There is unfortunately no criterion to which either of them can be subjected, and not much forcing was required to bring either into near relations with the other. We may fairly doubt whether the division of the first and second parts of logic in the Hegelian system has not really arisen from a desire to make them accord with the first and second stages of the early Greek philosophy. Is there any reason why the conception of measure in the first part, which is formed by the union of quality and quantity, should not have been equally placed in the second division of mediate or reflected ideas? The more we analyze them the less exact does the coincidence of philosophy and the history of philosophy appear. Many terms which were used absolutely in the beginning of philosophy, such as 'Being,' 'matter,' 'cause,' and the like, became relative in the subsequent history of thought. But Hegel employs some of them absolutely, some relatively, seemingly without any principle and without any regard to their original significance.

\par  The divisions of the Hegelian logic bear a superficial resemblance to the divisions of the scholastic logic. The first part answers to the term, the second to the proposition, the third to the syllogism. These are the grades of thought under which we conceive the world, first, in the general terms of quality, quantity, measure; secondly, under the relative forms of 'ground' and existence, substance and accidents, and the like; thirdly in syllogistic forms of the individual mediated with the universal by the help of the particular. Of syllogisms there are various kinds,—qualitative, quantitative, inductive, mechanical, teleological,—which are developed out of one another. But is there any meaning in reintroducing the forms of the old logic? Who ever thinks of the world as a syllogism? What connexion is there between the proposition and our ideas of reciprocity, cause and effect, and similar relations? It is difficult enough to conceive all the powers of nature and mind gathered up in one. The difficulty is greatly increased when the new is confused with the old, and the common logic is the Procrustes' bed into which they are forced.

\par  The Hegelian philosophy claims, as we have seen, to be based upon experience: it abrogates the distinction of a priori and a posteriori truth. It also acknowledges that many differences of kind are resolvable into differences of degree. It is familiar with the terms 'evolution,' 'development,' and the like. Yet it can hardly be said to have considered the forms of thought which are best adapted for the expression of facts. It has never applied the categories to experience; it has not defined the differences in our ideas of opposition, or development, or cause and effect, in the different sciences which make use of these terms. It rests on a knowledge which is not the result of exact or serious enquiry, but is floating in the air; the mind has been imperceptibly informed of some of the methods required in the sciences. Hegel boasts that the movement of dialectic is at once necessary and spontaneous: in reality it goes beyond experience and is unverified by it. Further, the Hegelian philosophy, while giving us the power of thinking a great deal more than we are able to fill up, seems to be wanting in some determinations of thought which we require. We cannot say that physical science, which at present occupies so large a share of popular attention, has been made easier or more intelligible by the distinctions of Hegel. Nor can we deny that he has sometimes interpreted physics by metaphysics, and confused his own philosophical fancies with the laws of nature. The very freedom of the movement is not without suspicion, seeming to imply a state of the human mind which has entirely lost sight of facts. Nor can the necessity which is attributed to it be very stringent, seeing that the successive categories or determinations of thought in different parts of his writings are arranged by the philosopher in different ways. What is termed necessary evolution seems to be only the order in which a succession of ideas presented themselves to the mind of Hegel at a particular time.

\par  The nomenclature of Hegel has been made by himself out of the language of common life. He uses a few words only which are borrowed from his predecessors, or from the Greek philosophy, and these generally in a sense peculiar to himself. The first stage of his philosophy answers to the word 'is,' the second to the word 'has been,' the third to the words 'has been' and 'is' combined. In other words, the first sphere is immediate, the second mediated by reflection, the third or highest returns into the first, and is both mediate and immediate. As Luther's Bible was written in the language of the common people, so Hegel seems to have thought that he gave his philosophy a truly German character by the use of idiomatic German words. But it may be doubted whether the attempt has been successful. First because such words as 'in sich seyn,' 'an sich seyn,' 'an und fur sich seyn,' though the simplest combinations of nouns and verbs, require a difficult and elaborate explanation. The simplicity of the words contrasts with the hardness of their meaning. Secondly, the use of technical phraseology necessarily separates philosophy from general literature; the student has to learn a new language of uncertain meaning which he with difficulty remembers. No former philosopher had ever carried the use of technical terms to the same extent as Hegel. The language of Plato or even of Aristotle is but slightly removed from that of common life, and was introduced naturally by a series of thinkers: the language of the scholastic logic has become technical to us, but in the Middle Ages was the vernacular Latin of priests and students. The higher spirit of philosophy, the spirit of Plato and Socrates, rebels against the Hegelian use of language as mechanical and technical.

\par  Hegel is fond of etymologies and often seems to trifle with words. He gives etymologies which are bad, and never considers that the meaning of a word may have nothing to do with its derivation. He lived before the days of Comparative Philology or of Comparative Mythology and Religion, which would have opened a new world to him. He makes no allowance for the element of chance either in language or thought; and perhaps there is no greater defect in his system than the want of a sound theory of language. He speaks as if thought, instead of being identical with language, was wholly independent of it. It is not the actual growth of the mind, but the imaginary growth of the Hegelian system, which is attractive to him.

\par  Neither are we able to say why of the common forms of thought some are rejected by him, while others have an undue prominence given to them. Some of them, such as 'ground' and 'existence,' have hardly any basis either in language or philosophy, while others, such as 'cause' and 'effect,' are but slightly considered. All abstractions are supposed by Hegel to derive their meaning from one another. This is true of some, but not of all, and in different degrees. There is an explanation of abstractions by the phenomena which they represent, as well as by their relation to other abstractions. If the knowledge of all were necessary to the knowledge of any one of them, the mind would sink under the load of thought. Again, in every process of reflection we seem to require a standing ground, and in the attempt to obtain a complete analysis we lose all fixedness. If, for example, the mind is viewed as the complex of ideas, or the difference between things and persons denied, such an analysis may be justified from the point of view of Hegel: but we shall find that in the attempt to criticize thought we have lost the power of thinking, and, like the Heracliteans of old, have no words in which our meaning can be expressed. Such an analysis may be of value as a corrective of popular language or thought, but should still allow us to retain the fundamental distinctions of philosophy.

\par  In the Hegelian system ideas supersede persons. The world of thought, though sometimes described as Spirit or 'Geist,' is really impersonal. The minds of men are to be regarded as one mind, or more correctly as a succession of ideas. Any comprehensive view of the world must necessarily be general, and there may be a use with a view to comprehensiveness in dropping individuals and their lives and actions. In all things, if we leave out details, a certain degree of order begins to appear; at any rate we can make an order which, with a little exaggeration or disproportion in some of the parts, will cover the whole field of philosophy. But are we therefore justified in saying that ideas are the causes of the great movement of the world rather than the personalities which conceived them? The great man is the expression of his time, and there may be peculiar difficulties in his age which he cannot overcome. He may be out of harmony with his circumstances, too early or too late, and then all his thoughts perish; his genius passes away unknown. But not therefore is he to be regarded as a mere waif or stray in human history, any more than he is the mere creature or expression of the age in which he lives. His ideas are inseparable from himself, and would have been nothing without him. Through a thousand personal influences they have been brought home to the minds of others. He starts from antecedents, but he is great in proportion as he disengages himself from them or absorbs himself in them. Moreover the types of greatness differ; while one man is the expression of the influences of his age, another is in antagonism to them. One man is borne on the surface of the water; another is carried forward by the current which flows beneath. The character of an individual, whether he be independent of circumstances or not, inspires others quite as much as his words. What is the teaching of Socrates apart from his personal history, or the doctrines of Christ apart from the Divine life in which they are embodied? Has not Hegel himself delineated the greatness of the life of Christ as consisting in his 'Schicksalslosigkeit' or independence of the destiny of his race? Do not persons become ideas, and is there any distinction between them? Take away the five greatest legislators, the five greatest warriors, the five greatest poets, the five greatest founders or teachers of a religion, the five greatest philosophers, the five greatest inventors,—where would have been all that we most value in knowledge or in life? And can that be a true theory of the history of philosophy which, in Hegel's own language, 'does not allow the individual to have his right'?

\par  Once more, while we readily admit that the world is relative to the mind, and the mind to the world, and that we must suppose a common or correlative growth in them, we shrink from saying that this complex nature can contain, even in outline, all the endless forms of Being and knowledge. Are we not 'seeking the living among the dead' and dignifying a mere logical skeleton with the name of philosophy and almost of God? When we look far away into the primeval sources of thought and belief, do we suppose that the mere accident of our being the heirs of the Greek philosophers can give us a right to set ourselves up as having the true and only standard of reason in the world? Or when we contemplate the infinite worlds in the expanse of heaven can we imagine that a few meagre categories derived from language and invented by the genius of one or two great thinkers contain the secret of the universe? Or, having regard to the ages during which the human race may yet endure, do we suppose that we can anticipate the proportions human knowledge may attain even within the short space of one or two thousand years?

\par  Again, we have a difficulty in understanding how ideas can be causes, which to us seems to be as much a figure of speech as the old notion of a creator artist, 'who makes the world by the help of the demigods' (Plato, Tim. ), or with 'a golden pair of compasses' measures out the circumference of the universe (Milton, P.L.). We can understand how the idea in the mind of an inventor is the cause of the work which is produced by it; and we can dimly imagine how this universal frame may be animated by a divine intelligence. But we cannot conceive how all the thoughts of men that ever were, which are themselves subject to so many external conditions of climate, country, and the like, even if regarded as the single thought of a Divine Being, can be supposed to have made the world. We appear to be only wrapping up ourselves in our own conceits—to be confusing cause and effect—to be losing the distinction between reflection and action, between the human and divine.

\par  These are some of the doubts and suspicions which arise in the mind of a student of Hegel, when, after living for a time within the charmed circle, he removes to a little distance and looks back upon what he has learnt, from the vantage-ground of history and experience. The enthusiasm of his youth has passed away, the authority of the master no longer retains a hold upon him. But he does not regret the time spent in the study of him. He finds that he has received from him a real enlargement of mind, and much of the true spirit of philosophy, even when he has ceased to believe in him. He returns again and again to his writings as to the recollections of a first love, not undeserving of his admiration still. Perhaps if he were asked how he can admire without believing, or what value he can attribute to what he knows to be erroneous, he might answer in some such manner as the following:—

\par  1. That in Hegel he finds glimpses of the genius of the poet and of the common sense of the man of the world. His system is not cast in a poetic form, but neither has all this load of logic extinguished in him the feeling of poetry. He is the true countryman of his contemporaries Goethe and Schiller. Many fine expressions are scattered up and down in his writings, as when he tells us that 'the Crusaders went to the Sepulchre but found it empty.' He delights to find vestiges of his own philosophy in the older German mystics. And though he can be scarcely said to have mixed much in the affairs of men, for, as his biographer tells us, 'he lived for thirty years in a single room,' yet he is far from being ignorant of the world. No one can read his writings without acquiring an insight into life. He loves to touch with the spear of logic the follies and self-deceptions of mankind, and make them appear in their natural form, stripped of the disguises of language and custom. He will not allow men to defend themselves by an appeal to one-sided or abstract principles. In this age of reason any one can too easily find a reason for doing what he likes (Wallace). He is suspicious of a distinction which is often made between a person's character and his conduct. His spirit is the opposite of that of Jesuitism or casuistry (Wallace). He affords an example of a remark which has been often made, that in order to know the world it is not necessary to have had a great experience of it.

\par  2. Hegel, if not the greatest philosopher, is certainly the greatest critic of philosophy who ever lived. No one else has equally mastered the opinions of his predecessors or traced the connexion of them in the same manner. No one has equally raised the human mind above the trivialities of the common logic and the unmeaningness of 'mere' abstractions, and above imaginary possibilities, which, as he truly says, have no place in philosophy. No one has won so much for the kingdom of ideas. Whatever may be thought of his own system it will hardly be denied that he has overthrown Locke, Kant, Hume, and the so-called philosophy of common sense. He shows us that only by the study of metaphysics can we get rid of metaphysics, and that those who are in theory most opposed to them are in fact most entirely and hopelessly enslaved by them: 'Die reinen Physiker sind nur die Thiere.' The disciple of Hegel will hardly become the slave of any other system-maker. What Bacon seems to promise him he will find realized in the great German thinker, an emancipation nearly complete from the influences of the scholastic logic.

\par  3. Many of those who are least disposed to become the votaries of Hegelianism nevertheless recognize in his system a new logic supplying a variety of instruments and methods hitherto unemployed. We may not be able to agree with him in assimilating the natural order of human thought with the history of philosophy, and still less in identifying both with the divine idea or nature. But we may acknowledge that the great thinker has thrown a light on many parts of human knowledge, and has solved many difficulties. We cannot receive his doctrine of opposites as the last word of philosophy, but still we may regard it as a very important contribution to logic. We cannot affirm that words have no meaning when taken out of their connexion in the history of thought. But we recognize that their meaning is to a great extent due to association, and to their correlation with one another. We see the advantage of viewing in the concrete what mankind regard only in the abstract. There is much to be said for his faith or conviction, that God is immanent in the world,—within the sphere of the human mind, and not beyond it. It was natural that he himself, like a prophet of old, should regard the philosophy which he had invented as the voice of God in man. But this by no means implies that he conceived himself as creating God in thought. He was the servant of his own ideas and not the master of them. The philosophy of history and the history of philosophy may be almost said to have been discovered by him. He has done more to explain Greek thought than all other writers put together. Many ideas of development, evolution, reciprocity, which have become the symbols of another school of thinkers may be traced to his speculations. In the theology and philosophy of England as well as of Germany, and also in the lighter literature of both countries, there are always appearing 'fragments of the great banquet' of Hegel.

\par 
\section{
      SOPHIST
    } 
\par \textbf{THEODORUS}
\par   Here we are, Socrates, true to our agreement of yesterday; and we bring with us a stranger from Elea, who is a disciple of Parmenides and Zeno, and a true philosopher.

\par \textbf{SOCRATES}
\par   Is he not rather a god, Theodorus, who comes to us in the disguise of a stranger? For Homer says that all the gods, and especially the god of strangers, are companions of the meek and just, and visit the good and evil among men. And may not your companion be one of those higher powers, a cross-examining deity, who has come to spy out our weakness in argument, and to cross-examine us?

\par \textbf{THEODORUS}
\par   Nay, Socrates, he is not one of the disputatious sort—he is too good for that. And, in my opinion, he is not a god at all; but divine he certainly is, for this is a title which I should give to all philosophers.

\par \textbf{SOCRATES}
\par   Capital, my friend! and I may add that they are almost as hard to be discerned as the gods. For the true philosophers, and such as are not merely made up for the occasion, appear in various forms unrecognized by the ignorance of men, and they 'hover about cities,' as Homer declares, looking from above upon human life; and some think nothing of them, and others can never think enough; and sometimes they appear as statesmen, and sometimes as sophists; and then, again, to many they seem to be no better than madmen. I should like to ask our Eleatic friend, if he would tell us, what is thought about them in Italy, and to whom the terms are applied.

\par \textbf{THEODORUS}
\par   What terms?

\par \textbf{SOCRATES}
\par   Sophist, statesman, philosopher.

\par \textbf{THEODORUS}
\par   What is your difficulty about them, and what made you ask?

\par \textbf{SOCRATES}
\par   I want to know whether by his countrymen they are regarded as one or two; or do they, as the names are three, distinguish also three kinds, and assign one to each name?

\par \textbf{THEODORUS}
\par   I dare say that the Stranger will not object to discuss the question. What do you say, Stranger?

\par \textbf{STRANGER}
\par   I am far from objecting, Theodorus, nor have I any difficulty in replying that by us they are regarded as three. But to define precisely the nature of each of them is by no means a slight or easy task.

\par \textbf{THEODORUS}
\par   You have happened to light, Socrates, almost on the very question which we were asking our friend before we came hither, and he excused himself to us, as he does now to you; although he admitted that the matter had been fully discussed, and that he remembered the answer.

\par \textbf{SOCRATES}
\par   Then do not, Stranger, deny us the first favour which we ask of you:  I am sure that you will not, and therefore I shall only beg of you to say whether you like and are accustomed to make a long oration on a subject which you want to explain to another, or to proceed by the method of question and answer. I remember hearing a very noble discussion in which Parmenides employed the latter of the two methods, when I was a young man, and he was far advanced in years. (Compare Parm.)

\par \textbf{STRANGER}
\par   I prefer to talk with another when he responds pleasantly, and is light in hand; if not, I would rather have my own say.

\par \textbf{SOCRATES}
\par   Any one of the present company will respond kindly to you, and you can choose whom you like of them; I should recommend you to take a young person—Theaetetus, for example—unless you have a preference for some one else.

\par \textbf{STRANGER}
\par   I feel ashamed, Socrates, being a new-comer into your society, instead of talking a little and hearing others talk, to be spinning out a long soliloquy or address, as if I wanted to show off. For the true answer will certainly be a very long one, a great deal longer than might be expected from such a short and simple question. At the same time, I fear that I may seem rude and ungracious if I refuse your courteous request, especially after what you have said. For I certainly cannot object to your proposal, that Theaetetus should respond, having already conversed with him myself, and being recommended by you to take him.

\par \textbf{THEAETETUS}
\par   But are you sure, Stranger, that this will be quite so acceptable to the rest of the company as Socrates imagines?

\par \textbf{STRANGER}
\par   You hear them applauding, Theaetetus; after that, there is nothing more to be said. Well then, I am to argue with you, and if you tire of the argument, you may complain of your friends and not of me.

\par \textbf{THEAETETUS}
\par   I do not think that I shall tire, and if I do, I shall get my friend here, young Socrates, the namesake of the elder Socrates, to help; he is about my own age, and my partner at the gymnasium, and is constantly accustomed to work with me.

\par \textbf{STRANGER}
\par   Very good; you can decide about that for yourself as we proceed. Meanwhile you and I will begin together and enquire into the nature of the Sophist, first of the three:  I should like you to make out what he is and bring him to light in a discussion; for at present we are only agreed about the name, but of the thing to which we both apply the name possibly you have one notion and I another; whereas we ought always to come to an understanding about the thing itself in terms of a definition, and not merely about the name minus the definition. Now the tribe of Sophists which we are investigating is not easily caught or defined; and the world has long ago agreed, that if great subjects are to be adequately treated, they must be studied in the lesser and easier instances of them before we proceed to the greatest of all. And as I know that the tribe of Sophists is troublesome and hard to be caught, I should recommend that we practise beforehand the method which is to be applied to him on some simple and smaller thing, unless you can suggest a better way.

\par \textbf{THEAETETUS}
\par   Indeed I cannot.

\par \textbf{STRANGER}
\par   Then suppose that we work out some lesser example which will be a pattern of the greater?

\par \textbf{THEAETETUS}
\par   Good.

\par \textbf{STRANGER}
\par   What is there which is well known and not great, and is yet as susceptible of definition as any larger thing? Shall I say an angler? He is familiar to all of us, and not a very interesting or important person.

\par \textbf{THEAETETUS}
\par   He is not.

\par \textbf{STRANGER}
\par   Yet I suspect that he will furnish us with the sort of definition and line of enquiry which we want.

\par \textbf{THEAETETUS}
\par   Very good.

\par \textbf{STRANGER}
\par   Let us begin by asking whether he is a man having art or not having art, but some other power.

\par \textbf{THEAETETUS}
\par   He is clearly a man of art.

\par \textbf{STRANGER}
\par   And of arts there are two kinds?

\par \textbf{THEAETETUS}
\par   What are they?

\par \textbf{STRANGER}
\par   There is agriculture, and the tending of mortal creatures, and the art of constructing or moulding vessels, and there is the art of imitation—all these may be appropriately called by a single name.

\par \textbf{THEAETETUS}
\par   What do you mean? And what is the name?

\par \textbf{STRANGER}
\par   He who brings into existence something that did not exist before is said to be a producer, and that which is brought into existence is said to be produced.

\par \textbf{THEAETETUS}
\par   True.

\par \textbf{STRANGER}
\par   And all the arts which were just now mentioned are characterized by this power of producing?

\par \textbf{THEAETETUS}
\par   They are.

\par \textbf{STRANGER}
\par   Then let us sum them up under the name of productive or creative art.

\par \textbf{THEAETETUS}
\par   Very good.

\par \textbf{STRANGER}
\par   Next follows the whole class of learning and cognition; then comes trade, fighting, hunting. And since none of these produces anything, but is only engaged in conquering by word or deed, or in preventing others from conquering, things which exist and have been already produced—in each and all of these branches there appears to be an art which may be called acquisitive.

\par \textbf{THEAETETUS}
\par   Yes, that is the proper name.

\par \textbf{STRANGER}
\par   Seeing, then, that all arts are either acquisitive or creative, in which class shall we place the art of the angler?

\par \textbf{THEAETETUS}
\par   Clearly in the acquisitive class.

\par \textbf{STRANGER}
\par   And the acquisitive may be subdivided into two parts:  there is exchange, which is voluntary and is effected by gifts, hire, purchase; and the other part of acquisitive, which takes by force of word or deed, may be termed conquest?

\par \textbf{THEAETETUS}
\par   That is implied in what has been said.

\par \textbf{STRANGER}
\par   And may not conquest be again subdivided?

\par \textbf{THEAETETUS}
\par   How?

\par \textbf{STRANGER}
\par   Open force may be called fighting, and secret force may have the general name of hunting?

\par \textbf{THEAETETUS}
\par   Yes.

\par \textbf{STRANGER}
\par   And there is no reason why the art of hunting should not be further divided.

\par \textbf{THEAETETUS}
\par   How would you make the division?

\par \textbf{STRANGER}
\par   Into the hunting of living and of lifeless prey.

\par \textbf{THEAETETUS}
\par   Yes, if both kinds exist.

\par \textbf{STRANGER}
\par   Of course they exist; but the hunting after lifeless things having no special name, except some sorts of diving, and other small matters, may be omitted; the hunting after living things may be called animal hunting.

\par \textbf{THEAETETUS}
\par   Yes.

\par \textbf{STRANGER}
\par   And animal hunting may be truly said to have two divisions, land-animal hunting, which has many kinds and names, and water-animal hunting, or the hunting after animals who swim?

\par \textbf{THEAETETUS}
\par   True.

\par \textbf{STRANGER}
\par   And of swimming animals, one class lives on the wing and the other in the water?

\par \textbf{THEAETETUS}
\par   Certainly.

\par \textbf{STRANGER}
\par   Fowling is the general term under which the hunting of all birds is included.

\par \textbf{THEAETETUS}
\par   True.

\par \textbf{STRANGER}
\par   The hunting of animals who live in the water has the general name of fishing.

\par \textbf{THEAETETUS}
\par   Yes.

\par \textbf{STRANGER}
\par   And this sort of hunting may be further divided also into two principal kinds?

\par \textbf{THEAETETUS}
\par   What are they?

\par \textbf{STRANGER}
\par   There is one kind which takes them in nets, another which takes them by a blow.

\par \textbf{THEAETETUS}
\par   What do you mean, and how do you distinguish them?

\par \textbf{STRANGER}
\par   As to the first kind—all that surrounds and encloses anything to prevent egress, may be rightly called an enclosure.

\par \textbf{THEAETETUS}
\par   Very true.

\par \textbf{STRANGER}
\par   For which reason twig baskets, casting-nets, nooses, creels, and the like may all be termed 'enclosures'?

\par \textbf{THEAETETUS}
\par   True.

\par \textbf{STRANGER}
\par   And therefore this first kind of capture may be called by us capture with enclosures, or something of that sort?

\par \textbf{THEAETETUS}
\par   Yes.

\par \textbf{STRANGER}
\par   The other kind, which is practised by a blow with hooks and three-pronged spears, when summed up under one name, may be called striking, unless you, Theaetetus, can find some better name?

\par \textbf{THEAETETUS}
\par   Never mind the name—what you suggest will do very well.

\par \textbf{STRANGER}
\par   There is one mode of striking, which is done at night, and by the light of a fire, and is by the hunters themselves called firing, or spearing by firelight.

\par \textbf{THEAETETUS}
\par   True.

\par \textbf{STRANGER}
\par   And the fishing by day is called by the general name of barbing, because the spears, too, are barbed at the point.

\par \textbf{THEAETETUS}
\par   Yes, that is the term.

\par \textbf{STRANGER}
\par   Of this barb-fishing, that which strikes the fish who is below from above is called spearing, because this is the way in which the three-pronged spears are mostly used.

\par \textbf{THEAETETUS}
\par   Yes, it is often called so.

\par \textbf{STRANGER}
\par   Then now there is only one kind remaining.

\par \textbf{THEAETETUS}
\par   What is that?

\par \textbf{STRANGER}
\par   When a hook is used, and the fish is not struck in any chance part of his body, as he is with the spear, but only about the head and mouth, and is then drawn out from below upwards with reeds and rods: —What is the right name of that mode of fishing, Theaetetus?

\par \textbf{THEAETETUS}
\par   I suspect that we have now discovered the object of our search.

\par \textbf{STRANGER}
\par   Then now you and I have come to an understanding not only about the name of the angler's art, but about the definition of the thing itself. One half of all art was acquisitive—half of the acquisitive art was conquest or taking by force, half of this was hunting, and half of hunting was hunting animals, half of this was hunting water animals—of this again, the under half was fishing, half of fishing was striking; a part of striking was fishing with a barb, and one half of this again, being the kind which strikes with a hook and draws the fish from below upwards, is the art which we have been seeking, and which from the nature of the operation is denoted angling or drawing up (aspalieutike, anaspasthai).

\par \textbf{THEAETETUS}
\par   The result has been quite satisfactorily brought out.

\par \textbf{STRANGER}
\par   And now, following this pattern, let us endeavour to find out what a Sophist is.

\par \textbf{THEAETETUS}
\par   By all means.

\par \textbf{STRANGER}
\par   The first question about the angler was, whether he was a skilled artist or unskilled?

\par \textbf{THEAETETUS}
\par   True.

\par \textbf{STRANGER}
\par   And shall we call our new friend unskilled, or a thorough master of his craft?

\par \textbf{THEAETETUS}
\par   Certainly not unskilled, for his name, as, indeed, you imply, must surely express his nature.

\par \textbf{STRANGER}
\par   Then he must be supposed to have some art.

\par \textbf{THEAETETUS}
\par   What art?

\par \textbf{STRANGER}
\par   By heaven, they are cousins! it never occurred to us.

\par \textbf{THEAETETUS}
\par   Who are cousins?

\par \textbf{STRANGER}
\par   The angler and the Sophist.

\par \textbf{THEAETETUS}
\par   In what way are they related?

\par \textbf{STRANGER}
\par   They both appear to me to be hunters.

\par \textbf{THEAETETUS}
\par   How the Sophist? Of the other we have spoken.

\par \textbf{STRANGER}
\par   You remember our division of hunting, into hunting after swimming animals and land animals?

\par \textbf{THEAETETUS}
\par   Yes.

\par \textbf{STRANGER}
\par   And you remember that we subdivided the swimming and left the land animals, saying that there were many kinds of them?

\par \textbf{THEAETETUS}
\par   Certainly.

\par \textbf{STRANGER}
\par   Thus far, then, the Sophist and the angler, starting from the art of acquiring, take the same road?

\par \textbf{THEAETETUS}
\par   So it would appear.

\par \textbf{STRANGER}
\par   Their paths diverge when they reach the art of animal hunting; the one going to the sea-shore, and to the rivers and to the lakes, and angling for the animals which are in them.

\par \textbf{THEAETETUS}
\par   Very true.

\par \textbf{STRANGER}
\par   While the other goes to land and water of another sort—rivers of wealth and broad meadow-lands of generous youth; and he also is intending to take the animals which are in them.

\par \textbf{THEAETETUS}
\par   What do you mean?

\par \textbf{STRANGER}
\par   Of hunting on land there are two principal divisions.

\par \textbf{THEAETETUS}
\par   What are they?

\par \textbf{STRANGER}
\par   One is the hunting of tame, and the other of wild animals.

\par \textbf{THEAETETUS}
\par   But are tame animals ever hunted?

\par \textbf{STRANGER}
\par   Yes, if you include man under tame animals. But if you like you may say that there are no tame animals, or that, if there are, man is not among them; or you may say that man is a tame animal but is not hunted—you shall decide which of these alternatives you prefer.

\par \textbf{THEAETETUS}
\par   I should say, Stranger, that man is a tame animal, and I admit that he is hunted.

\par \textbf{STRANGER}
\par   Then let us divide the hunting of tame animals into two parts.

\par \textbf{THEAETETUS}
\par   How shall we make the division?

\par \textbf{STRANGER}
\par   Let us define piracy, man-stealing, tyranny, the whole military art, by one name, as hunting with violence.

\par \textbf{THEAETETUS}
\par   Very good.

\par \textbf{STRANGER}
\par   But the art of the lawyer, of the popular orator, and the art of conversation may be called in one word the art of persuasion.

\par \textbf{THEAETETUS}
\par   True.

\par \textbf{STRANGER}
\par   And of persuasion, there may be said to be two kinds?

\par \textbf{THEAETETUS}
\par   What are they?

\par \textbf{STRANGER}
\par   One is private, and the other public.

\par \textbf{THEAETETUS}
\par   Yes; each of them forms a class.

\par \textbf{STRANGER}
\par   And of private hunting, one sort receives hire, and the other brings gifts.

\par \textbf{THEAETETUS}
\par   I do not understand you.

\par \textbf{STRANGER}
\par   You seem never to have observed the manner in which lovers hunt.

\par \textbf{THEAETETUS}
\par   To what do you refer?

\par \textbf{STRANGER}
\par   I mean that they lavish gifts on those whom they hunt in addition to other inducements.

\par \textbf{THEAETETUS}
\par   Most true.

\par \textbf{STRANGER}
\par   Let us admit this, then, to be the amatory art.

\par \textbf{THEAETETUS}
\par   Certainly.

\par \textbf{STRANGER}
\par   But that sort of hireling whose conversation is pleasing and who baits his hook only with pleasure and exacts nothing but his maintenance in return, we should all, if I am not mistaken, describe as possessing flattery or an art of making things pleasant.

\par \textbf{THEAETETUS}
\par   Certainly.

\par \textbf{STRANGER}
\par   And that sort, which professes to form acquaintances only for the sake of virtue, and demands a reward in the shape of money, may be fairly called by another name?

\par \textbf{THEAETETUS}
\par   To be sure.

\par \textbf{STRANGER}
\par   And what is the name? Will you tell me?

\par \textbf{THEAETETUS}
\par   It is obvious enough; for I believe that we have discovered the Sophist:  which is, as I conceive, the proper name for the class described.

\par \textbf{STRANGER}
\par   Then now, Theaetetus, his art may be traced as a branch of the appropriative, acquisitive family—which hunts animals,—living—land— tame animals; which hunts man,—privately—for hire,—taking money in exchange—having the semblance of education; and this is termed Sophistry, and is a hunt after young men of wealth and rank—such is the conclusion.

\par \textbf{THEAETETUS}
\par   Just so.

\par \textbf{STRANGER}
\par   Let us take another branch of his genealogy; for he is a professor of a great and many-sided art; and if we look back at what has preceded we see that he presents another aspect, besides that of which we are speaking.

\par \textbf{THEAETETUS}
\par   In what respect?

\par \textbf{STRANGER}
\par   There were two sorts of acquisitive art; the one concerned with hunting, the other with exchange.

\par \textbf{THEAETETUS}
\par   There were.

\par \textbf{STRANGER}
\par   And of the art of exchange there are two divisions, the one of giving, and the other of selling.

\par \textbf{THEAETETUS}
\par   Let us assume that.

\par \textbf{STRANGER}
\par   Next, we will suppose the art of selling to be divided into two parts.

\par \textbf{THEAETETUS}
\par   How?

\par \textbf{STRANGER}
\par   There is one part which is distinguished as the sale of a man's own productions; another, which is the exchange of the works of others.

\par \textbf{THEAETETUS}
\par   Certainly.

\par \textbf{STRANGER}
\par   And is not that part of exchange which takes place in the city, being about half of the whole, termed retailing?

\par \textbf{THEAETETUS}
\par   Yes.

\par \textbf{STRANGER}
\par   And that which exchanges the goods of one city for those of another by selling and buying is the exchange of the merchant?

\par \textbf{THEAETETUS}
\par   To be sure.

\par \textbf{STRANGER}
\par   And you are aware that this exchange of the merchant is of two kinds:  it is partly concerned with food for the use of the body, and partly with the food of the soul which is bartered and received in exchange for money.

\par \textbf{THEAETETUS}
\par   What do you mean?

\par \textbf{STRANGER}
\par   You want to know what is the meaning of food for the soul; the other kind you surely understand.

\par \textbf{THEAETETUS}
\par   Yes.

\par \textbf{STRANGER}
\par   Take music in general and painting and marionette playing and many other things, which are purchased in one city, and carried away and sold in another—wares of the soul which are hawked about either for the sake of instruction or amusement;—may not he who takes them about and sells them be quite as truly called a merchant as he who sells meats and drinks?

\par \textbf{THEAETETUS}
\par   To be sure he may.

\par \textbf{STRANGER}
\par   And would you not call by the same name him who buys up knowledge and goes about from city to city exchanging his wares for money?

\par \textbf{THEAETETUS}
\par   Certainly I should.

\par \textbf{STRANGER}
\par   Of this merchandise of the soul, may not one part be fairly termed the art of display? And there is another part which is certainly not less ridiculous, but being a trade in learning must be called by some name germane to the matter?

\par \textbf{THEAETETUS}
\par   Certainly.

\par \textbf{STRANGER}
\par   The latter should have two names,—one descriptive of the sale of the knowledge of virtue, and the other of the sale of other kinds of knowledge.

\par \textbf{THEAETETUS}
\par   Of course.

\par \textbf{STRANGER}
\par   The name of art-seller corresponds well enough to the latter; but you must try and tell me the name of the other.

\par \textbf{THEAETETUS}
\par   He must be the Sophist, whom we are seeking; no other name can possibly be right.

\par \textbf{STRANGER}
\par   No other; and so this trader in virtue again turns out to be our friend the Sophist, whose art may now be traced from the art of acquisition through exchange, trade, merchandise, to a merchandise of the soul which is concerned with speech and the knowledge of virtue.

\par \textbf{THEAETETUS}
\par   Quite true.

\par \textbf{STRANGER}
\par   And there may be a third reappearance of him;—for he may have settled down in a city, and may fabricate as well as buy these same wares, intending to live by selling them, and he would still be called a Sophist?

\par \textbf{THEAETETUS}
\par   Certainly.

\par \textbf{STRANGER}
\par   Then that part of the acquisitive art which exchanges, and of exchange which either sells a man's own productions or retails those of others, as the case may be, and in either way sells the knowledge of virtue, you would again term Sophistry?

\par \textbf{THEAETETUS}
\par   I must, if I am to keep pace with the argument.

\par \textbf{STRANGER}
\par   Let us consider once more whether there may not be yet another aspect of sophistry.

\par \textbf{THEAETETUS}
\par   What is it?

\par \textbf{STRANGER}
\par   In the acquisitive there was a subdivision of the combative or fighting art.

\par \textbf{THEAETETUS}
\par   There was.

\par \textbf{STRANGER}
\par   Perhaps we had better divide it.

\par \textbf{THEAETETUS}
\par   What shall be the divisions?

\par \textbf{STRANGER}
\par   There shall be one division of the competitive, and another of the pugnacious.

\par \textbf{THEAETETUS}
\par   Very good.

\par \textbf{STRANGER}
\par   That part of the pugnacious which is a contest of bodily strength may be properly called by some such name as violent.

\par \textbf{THEAETETUS}
\par   True.

\par \textbf{STRANGER}
\par   And when the war is one of words, it may be termed controversy?

\par \textbf{THEAETETUS}
\par   Yes.

\par \textbf{STRANGER}
\par   And controversy may be of two kinds.

\par \textbf{THEAETETUS}
\par   What are they?

\par \textbf{STRANGER}
\par   When long speeches are answered by long speeches, and there is public discussion about the just and unjust, that is forensic controversy.

\par \textbf{THEAETETUS}
\par   Yes.

\par \textbf{STRANGER}
\par   And there is a private sort of controversy, which is cut up into questions and answers, and this is commonly called disputation?

\par \textbf{THEAETETUS}
\par   Yes, that is the name.

\par \textbf{STRANGER}
\par   And of disputation, that sort which is only a discussion about contracts, and is carried on at random, and without rules of art, is recognized by the reasoning faculty to be a distinct class, but has hitherto had no distinctive name, and does not deserve to receive one from us.

\par \textbf{THEAETETUS}
\par   No; for the different sorts of it are too minute and heterogeneous.

\par \textbf{STRANGER}
\par   But that which proceeds by rules of art to dispute about justice and injustice in their own nature, and about things in general, we have been accustomed to call argumentation (Eristic)?

\par \textbf{THEAETETUS}
\par   Certainly.

\par \textbf{STRANGER}
\par   And of argumentation, one sort wastes money, and the other makes money.

\par \textbf{THEAETETUS}
\par   Very true.

\par \textbf{STRANGER}
\par   Suppose we try and give to each of these two classes a name.

\par \textbf{THEAETETUS}
\par   Let us do so.

\par \textbf{STRANGER}
\par   I should say that the habit which leads a man to neglect his own affairs for the pleasure of conversation, of which the style is far from being agreeable to the majority of his hearers, may be fairly termed loquacity:  such is my opinion.

\par \textbf{THEAETETUS}
\par   That is the common name for it.

\par \textbf{STRANGER}
\par   But now who the other is, who makes money out of private disputation, it is your turn to say.

\par \textbf{THEAETETUS}
\par   There is only one true answer:  he is the wonderful Sophist, of whom we are in pursuit, and who reappears again for the fourth time.

\par \textbf{STRANGER}
\par   Yes, and with a fresh pedigree, for he is the money-making species of the Eristic, disputatious, controversial, pugnacious, combative, acquisitive family, as the argument has already proven.

\par \textbf{THEAETETUS}
\par   Certainly.

\par \textbf{STRANGER}
\par   How true was the observation that he was a many-sided animal, and not to be caught with one hand, as they say!

\par \textbf{THEAETETUS}
\par   Then you must catch him with two.

\par \textbf{STRANGER}
\par   Yes, we must, if we can. And therefore let us try another track in our pursuit of him:  You are aware that there are certain menial occupations which have names among servants?

\par \textbf{THEAETETUS}
\par   Yes, there are many such; which of them do you mean?

\par \textbf{STRANGER}
\par   I mean such as sifting, straining, winnowing, threshing.

\par \textbf{THEAETETUS}
\par   Certainly.

\par \textbf{STRANGER}
\par   And besides these there are a great many more, such as carding, spinning, adjusting the warp and the woof; and thousands of similar expressions are used in the arts.

\par \textbf{THEAETETUS}
\par   Of what are they to be patterns, and what are we going to do with them all?

\par \textbf{STRANGER}
\par   I think that in all of these there is implied a notion of division.

\par \textbf{THEAETETUS}
\par   Yes.

\par \textbf{STRANGER}
\par   Then if, as I was saying, there is one art which includes all of them, ought not that art to have one name?

\par \textbf{THEAETETUS}
\par   And what is the name of the art?

\par \textbf{STRANGER}
\par   The art of discerning or discriminating.

\par \textbf{THEAETETUS}
\par   Very good.

\par \textbf{STRANGER}
\par   Think whether you cannot divide this.

\par \textbf{THEAETETUS}
\par   I should have to think a long while.

\par \textbf{STRANGER}
\par   In all the previously named processes either like has been separated from like or the better from the worse.

\par \textbf{THEAETETUS}
\par   I see now what you mean.

\par \textbf{STRANGER}
\par   There is no name for the first kind of separation; of the second, which throws away the worse and preserves the better, I do know a name.

\par \textbf{THEAETETUS}
\par   What is it?

\par \textbf{STRANGER}
\par   Every discernment or discrimination of that kind, as I have observed, is called a purification.

\par \textbf{THEAETETUS}
\par   Yes, that is the usual expression.

\par \textbf{STRANGER}
\par   And any one may see that purification is of two kinds.

\par \textbf{THEAETETUS}
\par   Perhaps so, if he were allowed time to think; but I do not see at this moment.

\par \textbf{STRANGER}
\par   There are many purifications of bodies which may with propriety be comprehended under a single name.

\par \textbf{THEAETETUS}
\par   What are they, and what is their name?

\par \textbf{STRANGER}
\par   There is the purification of living bodies in their inward and in their outward parts, of which the former is duly effected by medicine and gymnastic, the latter by the not very dignified art of the bath-man; and there is the purification of inanimate substances—to this the arts of fulling and of furbishing in general attend in a number of minute particulars, having a variety of names which are thought ridiculous.

\par \textbf{THEAETETUS}
\par   Very true.

\par \textbf{STRANGER}
\par   There can be no doubt that they are thought ridiculous, Theaetetus; but then the dialectical art never considers whether the benefit to be derived from the purge is greater or less than that to be derived from the sponge, and has not more interest in the one than in the other; her endeavour is to know what is and is not kindred in all arts, with a view to the acquisition of intelligence; and having this in view, she honours them all alike, and when she makes comparisons, she counts one of them not a whit more ridiculous than another; nor does she esteem him who adduces as his example of hunting, the general's art, at all more decorous than another who cites that of the vermin-destroyer, but only as the greater pretender of the two. And as to your question concerning the name which was to comprehend all these arts of purification, whether of animate or inanimate bodies, the art of dialectic is in no wise particular about fine words, if she may be only allowed to have a general name for all other purifications, binding them up together and separating them off from the purification of the soul or intellect. For this is the purification at which she wants to arrive, and this we should understand to be her aim.

\par \textbf{THEAETETUS}
\par   Yes, I understand; and I agree that there are two sorts of purification, and that one of them is concerned with the soul, and that there is another which is concerned with the body.

\par \textbf{STRANGER}
\par   Excellent; and now listen to what I am going to say, and try to divide further the first of the two.

\par \textbf{THEAETETUS}
\par   Whatever line of division you suggest, I will endeavour to assist you.

\par \textbf{STRANGER}
\par   Do we admit that virtue is distinct from vice in the soul?

\par \textbf{THEAETETUS}
\par   Certainly.

\par \textbf{STRANGER}
\par   And purification was to leave the good and to cast out whatever is bad?

\par \textbf{THEAETETUS}
\par   True.

\par \textbf{STRANGER}
\par   Then any taking away of evil from the soul may be properly called purification?

\par \textbf{THEAETETUS}
\par   Yes.

\par \textbf{STRANGER}
\par   And in the soul there are two kinds of evil.

\par \textbf{THEAETETUS}
\par   What are they?

\par \textbf{STRANGER}
\par   The one may be compared to disease in the body, the other to deformity.

\par \textbf{THEAETETUS}
\par   I do not understand.

\par \textbf{STRANGER}
\par   Perhaps you have never reflected that disease and discord are the same.

\par \textbf{THEAETETUS}
\par   To this, again, I know not what I should reply.

\par \textbf{STRANGER}
\par   Do you not conceive discord to be a dissolution of kindred elements, originating in some disagreement?

\par \textbf{THEAETETUS}
\par   Just that.

\par \textbf{STRANGER}
\par   And is deformity anything but the want of measure, which is always unsightly?

\par \textbf{THEAETETUS}
\par   Exactly.

\par \textbf{STRANGER}
\par   And do we not see that opinion is opposed to desire, pleasure to anger, reason to pain, and that all these elements are opposed to one another in the souls of bad men?

\par \textbf{THEAETETUS}
\par   Certainly.

\par \textbf{STRANGER}
\par   And yet they must all be akin?

\par \textbf{THEAETETUS}
\par   Of course.

\par \textbf{STRANGER}
\par   Then we shall be right in calling vice a discord and disease of the soul?

\par \textbf{THEAETETUS}
\par   Most true.

\par \textbf{STRANGER}
\par   And when things having motion, and aiming at an appointed mark, continually miss their aim and glance aside, shall we say that this is the effect of symmetry among them, or of the want of symmetry?

\par \textbf{THEAETETUS}
\par   Clearly of the want of symmetry.

\par \textbf{STRANGER}
\par   But surely we know that no soul is voluntarily ignorant of anything?

\par \textbf{THEAETETUS}
\par   Certainly not.

\par \textbf{STRANGER}
\par   And what is ignorance but the aberration of a mind which is bent on truth, and in which the process of understanding is perverted?

\par \textbf{THEAETETUS}
\par   True.

\par \textbf{STRANGER}
\par   Then we are to regard an unintelligent soul as deformed and devoid of symmetry?

\par \textbf{THEAETETUS}
\par   Very true.

\par \textbf{STRANGER}
\par   Then there are these two kinds of evil in the soul—the one which is generally called vice, and is obviously a disease of the soul...

\par \textbf{THEAETETUS}
\par   Yes.

\par \textbf{STRANGER}
\par   And there is the other, which they call ignorance, and which, because existing only in the soul, they will not allow to be vice.

\par \textbf{THEAETETUS}
\par   I certainly admit what I at first disputed—that there are two kinds of vice in the soul, and that we ought to consider cowardice, intemperance, and injustice to be alike forms of disease in the soul, and ignorance, of which there are all sorts of varieties, to be deformity.

\par \textbf{STRANGER}
\par   And in the case of the body are there not two arts which have to do with the two bodily states?

\par \textbf{THEAETETUS}
\par   What are they?

\par \textbf{STRANGER}
\par   There is gymnastic, which has to do with deformity, and medicine, which has to do with disease.

\par \textbf{THEAETETUS}
\par   True.

\par \textbf{STRANGER}
\par   And where there is insolence and injustice and cowardice, is not chastisement the art which is most required?

\par \textbf{THEAETETUS}
\par   That certainly appears to be the opinion of mankind.

\par \textbf{STRANGER}
\par   Again, of the various kinds of ignorance, may not instruction be rightly said to be the remedy?

\par \textbf{THEAETETUS}
\par   True.

\par \textbf{STRANGER}
\par   And of the art of instruction, shall we say that there is one or many kinds? At any rate there are two principal ones. Think.

\par \textbf{THEAETETUS}
\par   I will.

\par \textbf{STRANGER}
\par   I believe that I can see how we shall soonest arrive at the answer to this question.

\par \textbf{THEAETETUS}
\par   How?

\par \textbf{STRANGER}
\par   If we can discover a line which divides ignorance into two halves. For a division of ignorance into two parts will certainly imply that the art of instruction is also twofold, answering to the two divisions of ignorance.

\par \textbf{THEAETETUS}
\par   Well, and do you see what you are looking for?

\par \textbf{STRANGER}
\par   I do seem to myself to see one very large and bad sort of ignorance which is quite separate, and may be weighed in the scale against all other sorts of ignorance put together.

\par \textbf{THEAETETUS}
\par   What is it?

\par \textbf{STRANGER}
\par   When a person supposes that he knows, and does not know; this appears to be the great source of all the errors of the intellect.

\par \textbf{THEAETETUS}
\par   True.

\par \textbf{STRANGER}
\par   And this, if I am not mistaken, is the kind of ignorance which specially earns the title of stupidity.

\par \textbf{THEAETETUS}
\par   True.

\par \textbf{STRANGER}
\par   What name, then, shall be given to the sort of instruction which gets rid of this?

\par \textbf{THEAETETUS}
\par   The instruction which you mean, Stranger, is, I should imagine, not the teaching of handicraft arts, but what, thanks to us, has been termed education in this part the world.

\par \textbf{STRANGER}
\par   Yes, Theaetetus, and by nearly all Hellenes. But we have still to consider whether education admits of any further division.

\par \textbf{THEAETETUS}
\par   We have.

\par \textbf{STRANGER}
\par   I think that there is a point at which such a division is possible.

\par \textbf{THEAETETUS}
\par   Where?

\par \textbf{STRANGER}
\par   Of education, one method appears to be rougher, and another smoother.

\par \textbf{THEAETETUS}
\par   How are we to distinguish the two?

\par \textbf{STRANGER}
\par   There is the time-honoured mode which our fathers commonly practised towards their sons, and which is still adopted by many—either of roughly reproving their errors, or of gently advising them; which varieties may be correctly included under the general term of admonition.

\par \textbf{THEAETETUS}
\par   True.

\par \textbf{STRANGER}
\par   But whereas some appear to have arrived at the conclusion that all ignorance is involuntary, and that no one who thinks himself wise is willing to learn any of those things in which he is conscious of his own cleverness, and that the admonitory sort of instruction gives much trouble and does little good—

\par \textbf{THEAETETUS}
\par   There they are quite right.

\par \textbf{STRANGER}
\par   Accordingly, they set to work to eradicate the spirit of conceit in another way.

\par \textbf{THEAETETUS}
\par   In what way?

\par \textbf{STRANGER}
\par   They cross-examine a man's words, when he thinks that he is saying something and is really saying nothing, and easily convict him of inconsistencies in his opinions; these they then collect by the dialectical process, and placing them side by side, show that they contradict one another about the same things, in relation to the same things, and in the same respect. He, seeing this, is angry with himself, and grows gentle towards others, and thus is entirely delivered from great prejudices and harsh notions, in a way which is most amusing to the hearer, and produces the most lasting good effect on the person who is the subject of the operation. For as the physician considers that the body will receive no benefit from taking food until the internal obstacles have been removed, so the purifier of the soul is conscious that his patient will receive no benefit from the application of knowledge until he is refuted, and from refutation learns modesty; he must be purged of his prejudices first and made to think that he knows only what he knows, and no more.

\par \textbf{THEAETETUS}
\par   That is certainly the best and wisest state of mind.

\par \textbf{STRANGER}
\par   For all these reasons, Theaetetus, we must admit that refutation is the greatest and chiefest of purifications, and he who has not been refuted, though he be the Great King himself, is in an awful state of impurity; he is uninstructed and deformed in those things in which he who would be truly blessed ought to be fairest and purest.

\par \textbf{THEAETETUS}
\par   Very true.

\par \textbf{STRANGER}
\par   And who are the ministers of this art? I am afraid to say the Sophists.

\par \textbf{THEAETETUS}
\par   Why?

\par \textbf{STRANGER}
\par   Lest we should assign to them too high a prerogative.

\par \textbf{THEAETETUS}
\par   Yet the Sophist has a certain likeness to our minister of purification.

\par \textbf{STRANGER}
\par   Yes, the same sort of likeness which a wolf, who is the fiercest of animals, has to a dog, who is the gentlest. But he who would not be found tripping, ought to be very careful in this matter of comparisons, for they are most slippery things. Nevertheless, let us assume that the Sophists are the men. I say this provisionally, for I think that the line which divides them will be marked enough if proper care is taken.

\par \textbf{THEAETETUS}
\par   Likely enough.

\par \textbf{STRANGER}
\par   Let us grant, then, that from the discerning art comes purification, and from purification let there be separated off a part which is concerned with the soul; of this mental purification instruction is a portion, and of instruction education, and of education, that refutation of vain conceit which has been discovered in the present argument; and let this be called by you and me the nobly-descended art of Sophistry.

\par \textbf{THEAETETUS}
\par   Very well; and yet, considering the number of forms in which he has presented himself, I begin to doubt how I can with any truth or confidence describe the real nature of the Sophist.

\par \textbf{STRANGER}
\par   You naturally feel perplexed; and yet I think that he must be still more perplexed in his attempt to escape us, for as the proverb says, when every way is blocked, there is no escape; now, then, is the time of all others to set upon him.

\par \textbf{THEAETETUS}
\par   True.

\par \textbf{STRANGER}
\par   First let us wait a moment and recover breath, and while we are resting, we may reckon up in how many forms he has appeared. In the first place, he was discovered to be a paid hunter after wealth and youth.

\par \textbf{THEAETETUS}
\par   Yes.

\par \textbf{STRANGER}
\par   In the second place, he was a merchant in the goods of the soul.

\par \textbf{THEAETETUS}
\par   Certainly.

\par \textbf{STRANGER}
\par   In the third place, he has turned out to be a retailer of the same sort of wares.

\par \textbf{THEAETETUS}
\par   Yes; and in the fourth place, he himself manufactured the learned wares which he sold.

\par \textbf{STRANGER}
\par   Quite right; I will try and remember the fifth myself. He belonged to the fighting class, and was further distinguished as a hero of debate, who professed the eristic art.

\par \textbf{THEAETETUS}
\par   True.

\par \textbf{STRANGER}
\par   The sixth point was doubtful, and yet we at last agreed that he was a purger of souls, who cleared away notions obstructive to knowledge.

\par \textbf{THEAETETUS}
\par   Very true.

\par \textbf{STRANGER}
\par   Do you not see that when the professor of any art has one name and many kinds of knowledge, there must be something wrong? The multiplicity of names which is applied to him shows that the common principle to which all these branches of knowledge are tending, is not understood.

\par \textbf{THEAETETUS}
\par   I should imagine this to be the case.

\par \textbf{STRANGER}
\par   At any rate we will understand him, and no indolence shall prevent us. Let us begin again, then, and re-examine some of our statements concerning the Sophist; there was one thing which appeared to me especially characteristic of him.

\par \textbf{THEAETETUS}
\par   To what are you referring?

\par \textbf{STRANGER}
\par   We were saying of him, if I am not mistaken, that he was a disputer?

\par \textbf{THEAETETUS}
\par   We were.

\par \textbf{STRANGER}
\par   And does he not also teach others the art of disputation?

\par \textbf{THEAETETUS}
\par   Certainly he does.

\par \textbf{STRANGER}
\par   And about what does he profess that he teaches men to dispute? To begin at the beginning—Does he make them able to dispute about divine things, which are invisible to men in general?

\par \textbf{THEAETETUS}
\par   At any rate, he is said to do so.

\par \textbf{STRANGER}
\par   And what do you say of the visible things in heaven and earth, and the like?

\par \textbf{THEAETETUS}
\par   Certainly he disputes, and teaches to dispute about them.

\par \textbf{STRANGER}
\par   Then, again, in private conversation, when any universal assertion is made about generation and essence, we know that such persons are tremendous argufiers, and are able to impart their own skill to others.

\par \textbf{THEAETETUS}
\par   Undoubtedly.

\par \textbf{STRANGER}
\par   And do they not profess to make men able to dispute about law and about politics in general?

\par \textbf{THEAETETUS}
\par   Why, no one would have anything to say to them, if they did not make these professions.

\par \textbf{STRANGER}
\par   In all and every art, what the craftsman ought to say in answer to any question is written down in a popular form, and he who likes may learn.

\par \textbf{THEAETETUS}
\par   I suppose that you are referring to the precepts of Protagoras about wrestling and the other arts?

\par \textbf{STRANGER}
\par   Yes, my friend, and about a good many other things. In a word, is not the art of disputation a power of disputing about all things?

\par \textbf{THEAETETUS}
\par   Certainly; there does not seem to be much which is left out.

\par \textbf{STRANGER}
\par   But oh! my dear youth, do you suppose this possible? for perhaps your young eyes may see things which to our duller sight do not appear.

\par \textbf{THEAETETUS}
\par   To what are you alluding? I do not think that I understand your present question.

\par \textbf{STRANGER}
\par   I ask whether anybody can understand all things.

\par \textbf{THEAETETUS}
\par   Happy would mankind be if such a thing were possible!

\par \textbf{SOCRATES}
\par   But how can any one who is ignorant dispute in a rational manner against him who knows?

\par \textbf{THEAETETUS}
\par   He cannot.

\par \textbf{STRANGER}
\par   Then why has the sophistical art such a mysterious power?

\par \textbf{THEAETETUS}
\par   To what do you refer?

\par \textbf{STRANGER}
\par   How do the Sophists make young men believe in their supreme and universal wisdom? For if they neither disputed nor were thought to dispute rightly, or being thought to do so were deemed no wiser for their controversial skill, then, to quote your own observation, no one would give them money or be willing to learn their art.

\par \textbf{THEAETETUS}
\par   They certainly would not.

\par \textbf{STRANGER}
\par   But they are willing.

\par \textbf{THEAETETUS}
\par   Yes, they are.

\par \textbf{STRANGER}
\par   Yes, and the reason, as I should imagine, is that they are supposed to have knowledge of those things about which they dispute?

\par \textbf{THEAETETUS}
\par   Certainly.

\par \textbf{STRANGER}
\par   And they dispute about all things?

\par \textbf{THEAETETUS}
\par   True.

\par \textbf{STRANGER}
\par   And therefore, to their disciples, they appear to be all-wise?

\par \textbf{THEAETETUS}
\par   Certainly.

\par \textbf{STRANGER}
\par   But they are not; for that was shown to be impossible.

\par \textbf{THEAETETUS}
\par   Impossible, of course.

\par \textbf{STRANGER}
\par   Then the Sophist has been shown to have a sort of conjectural or apparent knowledge only of all things, which is not the truth?

\par \textbf{THEAETETUS}
\par   Exactly; no better description of him could be given.

\par \textbf{STRANGER}
\par   Let us now take an illustration, which will still more clearly explain his nature.

\par \textbf{THEAETETUS}
\par   What is it?

\par \textbf{STRANGER}
\par   I will tell you, and you shall answer me, giving your very closest attention. Suppose that a person were to profess, not that he could speak or dispute, but that he knew how to make and do all things, by a single art.

\par \textbf{THEAETETUS}
\par   All things?

\par \textbf{STRANGER}
\par   I see that you do not understand the first word that I utter, for you do not understand the meaning of 'all.'

\par \textbf{THEAETETUS}
\par   No, I do not.

\par \textbf{STRANGER}
\par   Under all things, I include you and me, and also animals and trees.

\par \textbf{THEAETETUS}
\par   What do you mean?

\par \textbf{STRANGER}
\par   Suppose a person to say that he will make you and me, and all creatures.

\par \textbf{THEAETETUS}
\par   What would he mean by 'making'? He cannot be a husbandman;—for you said that he is a maker of animals.

\par \textbf{STRANGER}
\par   Yes; and I say that he is also the maker of the sea, and the earth, and the heavens, and the gods, and of all other things; and, further, that he can make them in no time, and sell them for a few pence.

\par \textbf{THEAETETUS}
\par   That must be a jest.

\par \textbf{STRANGER}
\par   And when a man says that he knows all things, and can teach them to another at a small cost, and in a short time, is not that a jest?

\par \textbf{THEAETETUS}
\par   Certainly.

\par \textbf{STRANGER}
\par   And is there any more artistic or graceful form of jest than imitation?

\par \textbf{THEAETETUS}
\par   Certainly not; and imitation is a very comprehensive term, which includes under one class the most diverse sorts of things.

\par \textbf{STRANGER}
\par   We know, of course, that he who professes by one art to make all things is really a painter, and by the painter's art makes resemblances of real things which have the same name with them; and he can deceive the less intelligent sort of young children, to whom he shows his pictures at a distance, into the belief that he has the absolute power of making whatever he likes.

\par \textbf{THEAETETUS}
\par   Certainly.

\par \textbf{STRANGER}
\par   And may there not be supposed to be an imitative art of reasoning? Is it not possible to enchant the hearts of young men by words poured through their ears, when they are still at a distance from the truth of facts, by exhibiting to them fictitious arguments, and making them think that they are true, and that the speaker is the wisest of men in all things?

\par \textbf{THEAETETUS}
\par   Yes; why should there not be another such art?

\par \textbf{STRANGER}
\par   But as time goes on, and their hearers advance in years, and come into closer contact with realities, and have learnt by sad experience to see and feel the truth of things, are not the greater part of them compelled to change many opinions which they formerly entertained, so that the great appears small to them, and the easy difficult, and all their dreamy speculations are overturned by the facts of life?

\par \textbf{THEAETETUS}
\par   That is my view, as far as I can judge, although, at my age, I may be one of those who see things at a distance only.

\par \textbf{STRANGER}
\par   And the wish of all of us, who are your friends, is and always will be to bring you as near to the truth as we can without the sad reality. And now I should like you to tell me, whether the Sophist is not visibly a magician and imitator of true being; or are we still disposed to think that he may have a true knowledge of the various matters about which he disputes?

\par \textbf{THEAETETUS}
\par   But how can he, Stranger? Is there any doubt, after what has been said, that he is to be located in one of the divisions of children's play?

\par \textbf{STRANGER}
\par   Then we must place him in the class of magicians and mimics.

\par \textbf{THEAETETUS}
\par   Certainly we must.

\par \textbf{STRANGER}
\par   And now our business is not to let the animal out, for we have got him in a sort of dialectical net, and there is one thing which he decidedly will not escape.

\par \textbf{THEAETETUS}
\par   What is that?

\par \textbf{STRANGER}
\par   The inference that he is a juggler.

\par \textbf{THEAETETUS}
\par   Precisely my own opinion of him.

\par \textbf{STRANGER}
\par   Then, clearly, we ought as soon as possible to divide the image-making art, and go down into the net, and, if the Sophist does not run away from us, to seize him according to orders and deliver him over to reason, who is the lord of the hunt, and proclaim the capture of him; and if he creeps into the recesses of the imitative art, and secretes himself in one of them, to divide again and follow him up until in some sub-section of imitation he is caught. For our method of tackling each and all is one which neither he nor any other creature will ever escape in triumph.

\par \textbf{THEAETETUS}
\par   Well said; and let us do as you propose.

\par \textbf{STRANGER}
\par   Well, then, pursuing the same analytic method as before, I think that I can discern two divisions of the imitative art, but I am not as yet able to see in which of them the desired form is to be found.

\par \textbf{THEAETETUS}
\par   Will you tell me first what are the two divisions of which you are speaking?

\par \textbf{STRANGER}
\par   One is the art of likeness-making;—generally a likeness of anything is made by producing a copy which is executed according to the proportions of the original, similar in length and breadth and depth, each thing receiving also its appropriate colour.

\par \textbf{THEAETETUS}
\par   Is not this always the aim of imitation?

\par \textbf{STRANGER}
\par   Not always; in works either of sculpture or of painting, which are of any magnitude, there is a certain degree of deception; for artists were to give the true proportions of their fair works, the upper part, which is farther off, would appear to be out of proportion in comparison with the lower, which is nearer; and so they give up the truth in their images and make only the proportions which appear to be beautiful, disregarding the real ones.

\par \textbf{THEAETETUS}
\par   Quite true.

\par \textbf{STRANGER}
\par   And that which being other is also like, may we not fairly call a likeness or image?

\par \textbf{THEAETETUS}
\par   Yes.

\par \textbf{STRANGER}
\par   And may we not, as I did just now, call that part of the imitative art which is concerned with making such images the art of likeness-making?

\par \textbf{THEAETETUS}
\par   Let that be the name.

\par \textbf{STRANGER}
\par   And what shall we call those resemblances of the beautiful, which appear such owing to the unfavourable position of the spectator, whereas if a person had the power of getting a correct view of works of such magnitude, they would appear not even like that to which they profess to be like? May we not call these 'appearances,' since they appear only and are not really like?

\par \textbf{THEAETETUS}
\par   Certainly.

\par \textbf{STRANGER}
\par   There is a great deal of this kind of thing in painting, and in all imitation.

\par \textbf{THEAETETUS}
\par   Of course.

\par \textbf{STRANGER}
\par   And may we not fairly call the sort of art, which produces an appearance and not an image, phantastic art?

\par \textbf{THEAETETUS}
\par   Most fairly.

\par \textbf{STRANGER}
\par   These then are the two kinds of image-making—the art of making likenesses, and phantastic or the art of making appearances?

\par \textbf{THEAETETUS}
\par   True.

\par \textbf{STRANGER}
\par   I was doubtful before in which of them I should place the Sophist, nor am I even now able to see clearly; verily he is a wonderful and inscrutable creature. And now in the cleverest manner he has got into an impossible place.

\par \textbf{THEAETETUS}
\par   Yes, he has.

\par \textbf{STRANGER}
\par   Do you speak advisedly, or are you carried away at the moment by the habit of assenting into giving a hasty answer?

\par \textbf{THEAETETUS}
\par   May I ask to what you are referring?

\par \textbf{STRANGER}
\par   My dear friend, we are engaged in a very difficult speculation—there can be no doubt of that; for how a thing can appear and seem, and not be, or how a man can say a thing which is not true, has always been and still remains a very perplexing question. Can any one say or think that falsehood really exists, and avoid being caught in a contradiction? Indeed, Theaetetus, the task is a difficult one.

\par \textbf{THEAETETUS}
\par   Why?

\par \textbf{STRANGER}
\par   He who says that falsehood exists has the audacity to assert the being of not-being; for this is implied in the possibility of falsehood. But, my boy, in the days when I was a boy, the great Parmenides protested against this doctrine, and to the end of his life he continued to inculcate the same lesson—always repeating both in verse and out of verse:

\par  'Keep your mind from this way of enquiry, for never will you show that not-being is.'

\par  Such is his testimony, which is confirmed by the very expression when sifted a little. Would you object to begin with the consideration of the words themselves?

\par \textbf{THEAETETUS}
\par   Never mind about me; I am only desirous that you should carry on the argument in the best way, and that you should take me with you.

\par \textbf{STRANGER}
\par   Very good; and now say, do we venture to utter the forbidden word 'not-being'?

\par \textbf{THEAETETUS}
\par   Certainly we do.

\par \textbf{STRANGER}
\par   Let us be serious then, and consider the question neither in strife nor play:  suppose that one of the hearers of Parmenides was asked, 'To what is the term "not-being" to be applied? '—do you know what sort of object he would single out in reply, and what answer he would make to the enquirer?

\par \textbf{THEAETETUS}
\par   That is a difficult question, and one not to be answered at all by a person like myself.

\par \textbf{STRANGER}
\par   There is at any rate no difficulty in seeing that the predicate 'not-being' is not applicable to any being.

\par \textbf{THEAETETUS}
\par   None, certainly.

\par \textbf{STRANGER}
\par   And if not to being, then not to something.

\par \textbf{THEAETETUS}
\par   Of course not.

\par \textbf{STRANGER}
\par   It is also plain, that in speaking of something we speak of being, for to speak of an abstract something naked and isolated from all being is impossible.

\par \textbf{THEAETETUS}
\par   Impossible.

\par \textbf{STRANGER}
\par   You mean by assenting to imply that he who says something must say some one thing?

\par \textbf{THEAETETUS}
\par   Yes.

\par \textbf{STRANGER}
\par   Some in the singular (ti) you would say is the sign of one, some in the dual (tine) of two, some in the plural (tines) of many?

\par \textbf{THEAETETUS}
\par   Exactly.

\par \textbf{STRANGER}
\par   Then he who says 'not something' must say absolutely nothing.

\par \textbf{THEAETETUS}
\par   Most assuredly.

\par \textbf{STRANGER}
\par   And as we cannot admit that a man speaks and says nothing, he who says 'not-being' does not speak at all.

\par \textbf{THEAETETUS}
\par   The difficulty of the argument can no further go.

\par \textbf{STRANGER}
\par   Not yet, my friend, is the time for such a word; for there still remains of all perplexities the first and greatest, touching the very foundation of the matter.

\par \textbf{THEAETETUS}
\par   What do you mean? Do not be afraid to speak.

\par \textbf{STRANGER}
\par   To that which is, may be attributed some other thing which is?

\par \textbf{THEAETETUS}
\par   Certainly.

\par \textbf{STRANGER}
\par   But can anything which is, be attributed to that which is not?

\par \textbf{THEAETETUS}
\par   Impossible.

\par \textbf{STRANGER}
\par   And all number is to be reckoned among things which are?

\par \textbf{THEAETETUS}
\par   Yes, surely number, if anything, has a real existence.

\par \textbf{STRANGER}
\par   Then we must not attempt to attribute to not-being number either in the singular or plural?

\par \textbf{THEAETETUS}
\par   The argument implies that we should be wrong in doing so.

\par \textbf{STRANGER}
\par   But how can a man either express in words or even conceive in thought things which are not or a thing which is not without number?

\par \textbf{THEAETETUS}
\par   How indeed?

\par \textbf{STRANGER}
\par   When we speak of things which are not, are we not attributing plurality to not-being?

\par \textbf{THEAETETUS}
\par   Certainly.

\par \textbf{STRANGER}
\par   But, on the other hand, when we say 'what is not,' do we not attribute unity?

\par \textbf{THEAETETUS}
\par   Manifestly.

\par \textbf{STRANGER}
\par   Nevertheless, we maintain that you may not and ought not to attribute being to not-being?

\par \textbf{THEAETETUS}
\par   Most true.

\par \textbf{STRANGER}
\par   Do you see, then, that not-being in itself can neither be spoken, uttered, or thought, but that it is unthinkable, unutterable, unspeakable, indescribable?

\par \textbf{THEAETETUS}
\par   Quite true.

\par \textbf{STRANGER}
\par   But, if so, I was wrong in telling you just now that the difficulty which was coming is the greatest of all.

\par \textbf{THEAETETUS}
\par   What! is there a greater still behind?

\par \textbf{STRANGER}
\par   Well, I am surprised, after what has been said already, that you do not see the difficulty in which he who would refute the notion of not-being is involved. For he is compelled to contradict himself as soon as he makes the attempt.

\par \textbf{THEAETETUS}
\par   What do you mean? Speak more clearly.

\par \textbf{STRANGER}
\par   Do not expect clearness from me. For I, who maintain that not-being has no part either in the one or many, just now spoke and am still speaking of not-being as one; for I say 'not-being.' Do you understand?

\par \textbf{THEAETETUS}
\par   Yes.

\par \textbf{STRANGER}
\par   And a little while ago I said that not-being is unutterable, unspeakable, indescribable:  do you follow?

\par \textbf{THEAETETUS}
\par   I do after a fashion.

\par \textbf{STRANGER}
\par   When I introduced the word 'is,' did I not contradict what I said before?

\par \textbf{THEAETETUS}
\par   Clearly.

\par \textbf{STRANGER}
\par   And in using the singular verb, did I not speak of not-being as one?

\par \textbf{THEAETETUS}
\par   Yes.

\par \textbf{STRANGER}
\par   And when I spoke of not-being as indescribable and unspeakable and unutterable, in using each of these words in the singular, did I not refer to not-being as one?

\par \textbf{THEAETETUS}
\par   Certainly.

\par \textbf{STRANGER}
\par   And yet we say that, strictly speaking, it should not be defined as one or many, and should not even be called 'it,' for the use of the word 'it' would imply a form of unity.

\par \textbf{THEAETETUS}
\par   Quite true.

\par \textbf{STRANGER}
\par   How, then, can any one put any faith in me? For now, as always, I am unequal to the refutation of not-being. And therefore, as I was saying, do not look to me for the right way of speaking about not-being; but come, let us try the experiment with you.

\par \textbf{THEAETETUS}
\par   What do you mean?

\par \textbf{STRANGER}
\par   Make a noble effort, as becomes youth, and endeavour with all your might to speak of not-being in a right manner, without introducing into it either existence or unity or plurality.

\par \textbf{THEAETETUS}
\par   It would be a strange boldness in me which would attempt the task when I see you thus discomfited.

\par \textbf{STRANGER}
\par   Say no more of ourselves; but until we find some one or other who can speak of not-being without number, we must acknowledge that the Sophist is a clever rogue who will not be got out of his hole.

\par \textbf{THEAETETUS}
\par   Most true.

\par \textbf{STRANGER}
\par   And if we say to him that he professes an art of making appearances, he will grapple with us and retort our argument upon ourselves; and when we call him an image-maker he will say, 'Pray what do you mean at all by an image? '—and I should like to know, Theaetetus, how we can possibly answer the younker's question?

\par \textbf{THEAETETUS}
\par   We shall doubtless tell him of the images which are reflected in water or in mirrors; also of sculptures, pictures, and other duplicates.

\par \textbf{STRANGER}
\par   I see, Theaetetus, that you have never made the acquaintance of the Sophist.

\par \textbf{THEAETETUS}
\par   Why do you think so?

\par \textbf{STRANGER}
\par   He will make believe to have his eyes shut, or to have none.

\par \textbf{THEAETETUS}
\par   What do you mean?

\par \textbf{STRANGER}
\par   When you tell him of something existing in a mirror, or in sculpture, and address him as though he had eyes, he will laugh you to scorn, and will pretend that he knows nothing of mirrors and streams, or of sight at all; he will say that he is asking about an idea.

\par \textbf{THEAETETUS}
\par   What can he mean?

\par \textbf{STRANGER}
\par   The common notion pervading all these objects, which you speak of as many, and yet call by the single name of image, as though it were the unity under which they were all included. How will you maintain your ground against him?

\par \textbf{THEAETETUS}
\par   How, Stranger, can I describe an image except as something fashioned in the likeness of the true?

\par \textbf{STRANGER}
\par   And do you mean this something to be some other true thing, or what do you mean?

\par \textbf{THEAETETUS}
\par   Certainly not another true thing, but only a resemblance.

\par \textbf{STRANGER}
\par   And you mean by true that which really is?

\par \textbf{THEAETETUS}
\par   Yes.

\par \textbf{STRANGER}
\par   And the not true is that which is the opposite of the true?

\par \textbf{THEAETETUS}
\par   Exactly.

\par \textbf{STRANGER}
\par   A resemblance, then, is not really real, if, as you say, not true?

\par \textbf{THEAETETUS}
\par   Nay, but it is in a certain sense.

\par \textbf{STRANGER}
\par   You mean to say, not in a true sense?

\par \textbf{THEAETETUS}
\par   Yes; it is in reality only an image.

\par \textbf{STRANGER}
\par   Then what we call an image is in reality really unreal.

\par \textbf{THEAETETUS}
\par   In what a strange complication of being and not-being we are involved!

\par \textbf{STRANGER}
\par   Strange! I should think so. See how, by his reciprocation of opposites, the many-headed Sophist has compelled us, quite against our will, to admit the existence of not-being.

\par \textbf{THEAETETUS}
\par   Yes, indeed, I see.

\par \textbf{STRANGER}
\par   The difficulty is how to define his art without falling into a contradiction.

\par \textbf{THEAETETUS}
\par   How do you mean? And where does the danger lie?

\par \textbf{STRANGER}
\par   When we say that he deceives us with an illusion, and that his art is illusory, do we mean that our soul is led by his art to think falsely, or what do we mean?

\par \textbf{THEAETETUS}
\par   There is nothing else to be said.

\par \textbf{STRANGER}
\par   Again, false opinion is that form of opinion which thinks the opposite of the truth: —You would assent?

\par \textbf{THEAETETUS}
\par   Certainly.

\par \textbf{STRANGER}
\par   You mean to say that false opinion thinks what is not?

\par \textbf{THEAETETUS}
\par   Of course.

\par \textbf{STRANGER}
\par   Does false opinion think that things which are not are not, or that in a certain sense they are?

\par \textbf{THEAETETUS}
\par   Things that are not must be imagined to exist in a certain sense, if any degree of falsehood is to be possible.

\par \textbf{STRANGER}
\par   And does not false opinion also think that things which most certainly exist do not exist at all?

\par \textbf{THEAETETUS}
\par   Yes.

\par \textbf{STRANGER}
\par   And here, again, is falsehood?

\par \textbf{THEAETETUS}
\par   Falsehood—yes.

\par \textbf{STRANGER}
\par   And in like manner, a false proposition will be deemed to be one which asserts the non-existence of things which are, and the existence of things which are not.

\par \textbf{THEAETETUS}
\par   There is no other way in which a false proposition can arise.

\par \textbf{STRANGER}
\par   There is not; but the Sophist will deny these statements. And indeed how can any rational man assent to them, when the very expressions which we have just used were before acknowledged by us to be unutterable, unspeakable, indescribable, unthinkable? Do you see his point, Theaetetus?

\par \textbf{THEAETETUS}
\par   Of course he will say that we are contradicting ourselves when we hazard the assertion, that falsehood exists in opinion and in words; for in maintaining this, we are compelled over and over again to assert being of not-being, which we admitted just now to be an utter impossibility.

\par \textbf{STRANGER}
\par   How well you remember! And now it is high time to hold a consultation as to what we ought to do about the Sophist; for if we persist in looking for him in the class of false workers and magicians, you see that the handles for objection and the difficulties which will arise are very numerous and obvious.

\par \textbf{THEAETETUS}
\par   They are indeed.

\par \textbf{STRANGER}
\par   We have gone through but a very small portion of them, and they are really infinite.

\par \textbf{THEAETETUS}
\par   If that is the case, we cannot possibly catch the Sophist.

\par \textbf{STRANGER}
\par   Shall we then be so faint-hearted as to give him up?

\par \textbf{THEAETETUS}
\par   Certainly not, I should say, if we can get the slightest hold upon him.

\par \textbf{STRANGER}
\par   Will you then forgive me, and, as your words imply, not be altogether displeased if I flinch a little from the grasp of such a sturdy argument?

\par \textbf{THEAETETUS}
\par   To be sure I will.

\par \textbf{STRANGER}
\par   I have a yet more urgent request to make.

\par \textbf{THEAETETUS}
\par   Which is—?

\par \textbf{STRANGER}
\par   That you will promise not to regard me as a parricide.

\par \textbf{THEAETETUS}
\par   And why?

\par \textbf{STRANGER}
\par   Because, in self-defence, I must test the philosophy of my father Parmenides, and try to prove by main force that in a certain sense not-being is, and that being, on the other hand, is not.

\par \textbf{THEAETETUS}
\par   Some attempt of the kind is clearly needed.

\par \textbf{STRANGER}
\par   Yes, a blind man, as they say, might see that, and, unless these questions are decided in one way or another, no one when he speaks of false words, or false opinion, or idols, or images, or imitations, or appearances, or about the arts which are concerned with them; can avoid falling into ridiculous contradictions.

\par \textbf{THEAETETUS}
\par   Most true.

\par \textbf{STRANGER}
\par   And therefore I must venture to lay hands on my father's argument; for if I am to be over-scrupulous, I shall have to give the matter up.

\par \textbf{THEAETETUS}
\par   Nothing in the world should ever induce us to do so.

\par \textbf{STRANGER}
\par   I have a third little request which I wish to make.

\par \textbf{THEAETETUS}
\par   What is it?

\par \textbf{STRANGER}
\par   You heard me say what I have always felt and still feel—that I have no heart for this argument?

\par \textbf{THEAETETUS}
\par   I did.

\par \textbf{STRANGER}
\par   I tremble at the thought of what I have said, and expect that you will deem me mad, when you hear of my sudden changes and shiftings; let me therefore observe, that I am examining the question entirely out of regard for you.

\par \textbf{THEAETETUS}
\par   There is no reason for you to fear that I shall impute any impropriety to you, if you attempt this refutation and proof; take heart, therefore, and proceed.

\par \textbf{STRANGER}
\par   And where shall I begin the perilous enterprise? I think that the road which I must take is—

\par \textbf{THEAETETUS}
\par   Which?—Let me hear.

\par \textbf{STRANGER}
\par   I think that we had better, first of all, consider the points which at present are regarded as self-evident, lest we may have fallen into some confusion, and be too ready to assent to one another, fancying that we are quite clear about them.

\par \textbf{THEAETETUS}
\par   Say more distinctly what you mean.

\par \textbf{STRANGER}
\par   I think that Parmenides, and all ever yet undertook to determine the number and nature of existences, talked to us in rather a light and easy strain.

\par \textbf{THEAETETUS}
\par   How?

\par \textbf{STRANGER}
\par   As if we had been children, to whom they repeated each his own mythus or story;—one said that there were three principles, and that at one time there was war between certain of them; and then again there was peace, and they were married and begat children, and brought them up; and another spoke of two principles,—a moist and a dry, or a hot and a cold, and made them marry and cohabit. The Eleatics, however, in our part of the world, say that all things are many in name, but in nature one; this is their mythus, which goes back to Xenophanes, and is even older. Then there are Ionian, and in more recent times Sicilian muses, who have arrived at the conclusion that to unite the two principles is safer, and to say that being is one and many, and that these are held together by enmity and friendship, ever parting, ever meeting, as the severer Muses assert, while the gentler ones do not insist on the perpetual strife and peace, but admit a relaxation and alternation of them; peace and unity sometimes prevailing under the sway of Aphrodite, and then again plurality and war, by reason of a principle of strife. Whether any of them spoke the truth in all this is hard to determine; besides, antiquity and famous men should have reverence, and not be liable to accusations so serious. Yet one thing may be said of them without offence—

\par \textbf{THEAETETUS}
\par   What thing?

\par \textbf{STRANGER}
\par   That they went on their several ways disdaining to notice people like ourselves; they did not care whether they took us with them, or left us behind them.

\par \textbf{THEAETETUS}
\par   How do you mean?

\par \textbf{STRANGER}
\par   I mean to say, that when they talk of one, two, or more elements, which are or have become or are becoming, or again of heat mingling with cold, assuming in some other part of their works separations and mixtures,—tell me, Theaetetus, do you understand what they mean by these expressions? When I was a younger man, I used to fancy that I understood quite well what was meant by the term 'not-being,' which is our present subject of dispute; and now you see in what a fix we are about it.

\par \textbf{THEAETETUS}
\par   I see.

\par \textbf{STRANGER}
\par   And very likely we have been getting into the same perplexity about 'being,' and yet may fancy that when anybody utters the word, we understand him quite easily, although we do not know about not-being. But we may be; equally ignorant of both.

\par \textbf{THEAETETUS}
\par   I dare say.

\par \textbf{STRANGER}
\par   And the same may be said of all the terms just mentioned.

\par \textbf{THEAETETUS}
\par   True.

\par \textbf{STRANGER}
\par   The consideration of most of them may be deferred; but we had better now discuss the chief captain and leader of them.

\par \textbf{THEAETETUS}
\par   Of what are you speaking? You clearly think that we must first investigate what people mean by the word 'being.'

\par \textbf{STRANGER}
\par   You follow close at my heels, Theaetetus. For the right method, I conceive, will be to call into our presence the dualistic philosophers and to interrogate them. 'Come,' we will say, 'Ye, who affirm that hot and cold or any other two principles are the universe, what is this term which you apply to both of them, and what do you mean when you say that both and each of them "are"? How are we to understand the word "are"? Upon your view, are we to suppose that there is a third principle over and above the other two,—three in all, and not two? For clearly you cannot say that one of the two principles is being, and yet attribute being equally to both of them; for, if you did, whichever of the two is identified with being, will comprehend the other; and so they will be one and not two.'

\par \textbf{THEAETETUS}
\par   Very true.

\par \textbf{STRANGER}
\par   But perhaps you mean to give the name of 'being' to both of them together?

\par \textbf{THEAETETUS}
\par   Quite likely.

\par \textbf{STRANGER}
\par   'Then, friends,' we shall reply to them, 'the answer is plainly that the two will still be resolved into one.'

\par \textbf{THEAETETUS}
\par   Most true.

\par \textbf{STRANGER}
\par   'Since, then, we are in a difficulty, please to tell us what you mean, when you speak of being; for there can be no doubt that you always from the first understood your own meaning, whereas we once thought that we understood you, but now we are in a great strait. Please to begin by explaining this matter to us, and let us no longer fancy that we understand you, when we entirely misunderstand you.' There will be no impropriety in our demanding an answer to this question, either of the dualists or of the pluralists?

\par \textbf{THEAETETUS}
\par   Certainly not.

\par \textbf{STRANGER}
\par   And what about the assertors of the oneness of the all—must we not endeavour to ascertain from them what they mean by 'being'?

\par \textbf{THEAETETUS}
\par   By all means.

\par \textbf{STRANGER}
\par   Then let them answer this question:  One, you say, alone is? 'Yes,' they will reply.

\par \textbf{THEAETETUS}
\par   True.

\par \textbf{STRANGER}
\par   And there is something which you call 'being'?

\par \textbf{THEAETETUS}
\par   'Yes.'

\par \textbf{STRANGER}
\par   And is being the same as one, and do you apply two names to the same thing?

\par \textbf{THEAETETUS}
\par   What will be their answer, Stranger?

\par \textbf{STRANGER}
\par   It is clear, Theaetetus, that he who asserts the unity of being will find a difficulty in answering this or any other question.

\par \textbf{THEAETETUS}
\par   Why so?

\par \textbf{STRANGER}
\par   To admit of two names, and to affirm that there is nothing but unity, is surely ridiculous?

\par \textbf{THEAETETUS}
\par   Certainly.

\par \textbf{STRANGER}
\par   And equally irrational to admit that a name is anything?

\par \textbf{THEAETETUS}
\par   How so?

\par \textbf{STRANGER}
\par   To distinguish the name from the thing, implies duality.

\par \textbf{THEAETETUS}
\par   Yes.

\par \textbf{STRANGER}
\par   And yet he who identifies the name with the thing will be compelled to say that it is the name of nothing, or if he says that it is the name of something, even then the name will only be the name of a name, and of nothing else.

\par \textbf{THEAETETUS}
\par   True.

\par \textbf{STRANGER}
\par   And the one will turn out to be only one of one, and being absolute unity, will represent a mere name.

\par \textbf{THEAETETUS}
\par   Certainly.

\par \textbf{STRANGER}
\par   And would they say that the whole is other than the one that is, or the same with it?

\par \textbf{THEAETETUS}
\par   To be sure they would, and they actually say so.

\par \textbf{STRANGER}
\par   If being is a whole, as Parmenides sings,—

\par  'Every way like unto the fullness of a well-rounded sphere, Evenly balanced from the centre on every side, And must needs be neither greater nor less in any way, Neither on this side nor on that—'

\par  then being has a centre and extremes, and, having these, must also have parts.

\par \textbf{THEAETETUS}
\par   True.

\par \textbf{STRANGER}
\par   Yet that which has parts may have the attribute of unity in all the parts, and in this way being all and a whole, may be one?

\par \textbf{THEAETETUS}
\par   Certainly.

\par \textbf{STRANGER}
\par   But that of which this is the condition cannot be absolute unity?

\par \textbf{THEAETETUS}
\par   Why not?

\par \textbf{STRANGER}
\par   Because, according to right reason, that which is truly one must be affirmed to be absolutely indivisible.

\par \textbf{THEAETETUS}
\par   Certainly.

\par \textbf{STRANGER}
\par   But this indivisible, if made up of many parts, will contradict reason.

\par \textbf{THEAETETUS}
\par   I understand.

\par \textbf{STRANGER}
\par   Shall we say that being is one and a whole, because it has the attribute of unity? Or shall we say that being is not a whole at all?

\par \textbf{THEAETETUS}
\par   That is a hard alternative to offer.

\par \textbf{STRANGER}
\par   Most true; for being, having in a certain sense the attribute of one, is yet proved not to be the same as one, and the all is therefore more than one.

\par \textbf{THEAETETUS}
\par   Yes.

\par \textbf{STRANGER}
\par   And yet if being be not a whole, through having the attribute of unity, and there be such a thing as an absolute whole, being lacks something of its own nature?

\par \textbf{THEAETETUS}
\par   Certainly.

\par \textbf{STRANGER}
\par   Upon this view, again, being, having a defect of being, will become not-being?

\par \textbf{THEAETETUS}
\par   True.

\par \textbf{STRANGER}
\par   And, again, the all becomes more than one, for being and the whole will each have their separate nature.

\par \textbf{THEAETETUS}
\par   Yes.

\par \textbf{STRANGER}
\par   But if the whole does not exist at all, all the previous difficulties remain the same, and there will be the further difficulty, that besides having no being, being can never have come into being.

\par \textbf{THEAETETUS}
\par   Why so?

\par \textbf{STRANGER}
\par   Because that which comes into being always comes into being as a whole, so that he who does not give whole a place among beings, cannot speak either of essence or generation as existing.

\par \textbf{THEAETETUS}
\par   Yes, that certainly appears to be true.

\par \textbf{STRANGER}
\par   Again; how can that which is not a whole have any quantity? For that which is of a certain quantity must necessarily be the whole of that quantity.

\par \textbf{THEAETETUS}
\par   Exactly.

\par \textbf{STRANGER}
\par   And there will be innumerable other points, each of them causing infinite trouble to him who says that being is either one or two.

\par \textbf{THEAETETUS}
\par   The difficulties which are dawning upon us prove this; for one objection connects with another, and they are always involving what has preceded in a greater and worse perplexity.

\par \textbf{STRANGER}
\par   We are far from having exhausted the more exact thinkers who treat of being and not-being. But let us be content to leave them, and proceed to view those who speak less precisely; and we shall find as the result of all, that the nature of being is quite as difficult to comprehend as that of not-being.

\par \textbf{THEAETETUS}
\par   Then now we will go to the others.

\par \textbf{STRANGER}
\par   There appears to be a sort of war of Giants and Gods going on amongst them; they are fighting with one another about the nature of essence.

\par \textbf{THEAETETUS}
\par   How is that?

\par \textbf{STRANGER}
\par   Some of them are dragging down all things from heaven and from the unseen to earth, and they literally grasp in their hands rocks and oaks; of these they lay hold, and obstinately maintain, that the things only which can be touched or handled have being or essence, because they define being and body as one, and if any one else says that what is not a body exists they altogether despise him, and will hear of nothing but body.

\par \textbf{THEAETETUS}
\par   I have often met with such men, and terrible fellows they are.

\par \textbf{STRANGER}
\par   And that is the reason why their opponents cautiously defend themselves from above, out of an unseen world, mightily contending that true essence consists of certain intelligible and incorporeal ideas; the bodies of the materialists, which by them are maintained to be the very truth, they break up into little bits by their arguments, and affirm them to be, not essence, but generation and motion. Between the two armies, Theaetetus, there is always an endless conflict raging concerning these matters.

\par \textbf{THEAETETUS}
\par   True.

\par \textbf{STRANGER}
\par   Let us ask each party in turn, to give an account of that which they call essence.

\par \textbf{THEAETETUS}
\par   How shall we get it out of them?

\par \textbf{STRANGER}
\par   With those who make being to consist in ideas, there will be less difficulty, for they are civil people enough; but there will be very great difficulty, or rather an absolute impossibility, in getting an opinion out of those who drag everything down to matter. Shall I tell you what we must do?

\par \textbf{THEAETETUS}
\par   What?

\par \textbf{STRANGER}
\par   Let us, if we can, really improve them; but if this is not possible, let us imagine them to be better than they are, and more willing to answer in accordance with the rules of argument, and then their opinion will be more worth having; for that which better men acknowledge has more weight than that which is acknowledged by inferior men. Moreover we are no respecters of persons, but seekers after truth.

\par \textbf{THEAETETUS}
\par   Very good.

\par \textbf{STRANGER}
\par   Then now, on the supposition that they are improved, let us ask them to state their views, and do you interpret them.

\par \textbf{THEAETETUS}
\par   Agreed.

\par \textbf{STRANGER}
\par   Let them say whether they would admit that there is such a thing as a mortal animal.

\par \textbf{THEAETETUS}
\par   Of course they would.

\par \textbf{STRANGER}
\par   And do they not acknowledge this to be a body having a soul?

\par \textbf{THEAETETUS}
\par   Certainly they do.

\par \textbf{STRANGER}
\par   Meaning to say that the soul is something which exists?

\par \textbf{THEAETETUS}
\par   True.

\par \textbf{STRANGER}
\par   And do they not say that one soul is just, and another unjust, and that one soul is wise, and another foolish?

\par \textbf{THEAETETUS}
\par   Certainly.

\par \textbf{STRANGER}
\par   And that the just and wise soul becomes just and wise by the possession of justice and wisdom, and the opposite under opposite circumstances?

\par \textbf{THEAETETUS}
\par   Yes, they do.

\par \textbf{STRANGER}
\par   But surely that which may be present or may be absent will be admitted by them to exist?

\par \textbf{THEAETETUS}
\par   Certainly.

\par \textbf{STRANGER}
\par   And, allowing that justice, wisdom, the other virtues, and their opposites exist, as well as a soul in which they inhere, do they affirm any of them to be visible and tangible, or are they all invisible?

\par \textbf{THEAETETUS}
\par   They would say that hardly any of them are visible.

\par \textbf{STRANGER}
\par   And would they say that they are corporeal?

\par \textbf{THEAETETUS}
\par   They would distinguish:  the soul would be said by them to have a body; but as to the other qualities of justice, wisdom, and the like, about which you asked, they would not venture either to deny their existence, or to maintain that they were all corporeal.

\par \textbf{STRANGER}
\par   Verily, Theaetetus, I perceive a great improvement in them; the real aborigines, children of the dragon's teeth, would have been deterred by no shame at all, but would have obstinately asserted that nothing is which they are not able to squeeze in their hands.

\par \textbf{THEAETETUS}
\par   That is pretty much their notion.

\par \textbf{STRANGER}
\par   Let us push the question; for if they will admit that any, even the smallest particle of being, is incorporeal, it is enough; they must then say what that nature is which is common to both the corporeal and incorporeal, and which they have in their mind's eye when they say of both of them that they 'are.' Perhaps they may be in a difficulty; and if this is the case, there is a possibility that they may accept a notion of ours respecting the nature of being, having nothing of their own to offer.

\par \textbf{THEAETETUS}
\par   What is the notion? Tell me, and we shall soon see.

\par \textbf{STRANGER}
\par   My notion would be, that anything which possesses any sort of power to affect another, or to be affected by another, if only for a single moment, however trifling the cause and however slight the effect, has real existence; and I hold that the definition of being is simply power.

\par \textbf{THEAETETUS}
\par   They accept your suggestion, having nothing better of their own to offer.

\par \textbf{STRANGER}
\par   Very good; perhaps we, as well as they, may one day change our minds; but, for the present, this may be regarded as the understanding which is established with them.

\par \textbf{THEAETETUS}
\par   Agreed.

\par \textbf{STRANGER}
\par   Let us now go to the friends of ideas; of their opinions, too, you shall be the interpreter.

\par \textbf{THEAETETUS}
\par   I will.

\par \textbf{STRANGER}
\par   To them we say—You would distinguish essence from generation?

\par \textbf{THEAETETUS}
\par   'Yes,' they reply.

\par \textbf{STRANGER}
\par   And you would allow that we participate in generation with the body, and through perception, but we participate with the soul through thought in true essence; and essence you would affirm to be always the same and immutable, whereas generation or becoming varies?

\par \textbf{THEAETETUS}
\par   Yes; that is what we should affirm.

\par \textbf{STRANGER}
\par   Well, fair sirs, we say to them, what is this participation, which you assert of both? Do you agree with our recent definition?

\par \textbf{THEAETETUS}
\par   What definition?

\par \textbf{STRANGER}
\par   We said that being was an active or passive energy, arising out of a certain power which proceeds from elements meeting with one another. Perhaps your ears, Theaetetus, may fail to catch their answer, which I recognize because I have been accustomed to hear it.

\par \textbf{THEAETETUS}
\par   And what is their answer?

\par \textbf{STRANGER}
\par   They deny the truth of what we were just now saying to the aborigines about existence.

\par \textbf{THEAETETUS}
\par   What was that?

\par \textbf{STRANGER}
\par   Any power of doing or suffering in a degree however slight was held by us to be a sufficient definition of being?

\par \textbf{THEAETETUS}
\par   True.

\par \textbf{STRANGER}
\par   They deny this, and say that the power of doing or suffering is confined to becoming, and that neither power is applicable to being.

\par \textbf{THEAETETUS}
\par   And is there not some truth in what they say?

\par \textbf{STRANGER}
\par   Yes; but our reply will be, that we want to ascertain from them more distinctly, whether they further admit that the soul knows, and that being or essence is known.

\par \textbf{THEAETETUS}
\par   There can be no doubt that they say so.

\par \textbf{STRANGER}
\par   And is knowing and being known doing or suffering, or both, or is the one doing and the other suffering, or has neither any share in either?

\par \textbf{THEAETETUS}
\par   Clearly, neither has any share in either; for if they say anything else, they will contradict themselves.

\par \textbf{STRANGER}
\par   I understand; but they will allow that if to know is active, then, of course, to be known is passive. And on this view being, in so far as it is known, is acted upon by knowledge, and is therefore in motion; for that which is in a state of rest cannot be acted upon, as we affirm.

\par \textbf{THEAETETUS}
\par   True.

\par \textbf{STRANGER}
\par   And, O heavens, can we ever be made to believe that motion and life and soul and mind are not present with perfect being? Can we imagine that being is devoid of life and mind, and exists in awful unmeaningness an everlasting fixture?

\par \textbf{THEAETETUS}
\par   That would be a dreadful thing to admit, Stranger.

\par \textbf{STRANGER}
\par   But shall we say that has mind and not life?

\par \textbf{THEAETETUS}
\par   How is that possible?

\par \textbf{STRANGER}
\par   Or shall we say that both inhere in perfect being, but that it has no soul which contains them?

\par \textbf{THEAETETUS}
\par   And in what other way can it contain them?

\par \textbf{STRANGER}
\par   Or that being has mind and life and soul, but although endowed with soul remains absolutely unmoved?

\par \textbf{THEAETETUS}
\par   All three suppositions appear to me to be irrational.

\par \textbf{STRANGER}
\par   Under being, then, we must include motion, and that which is moved.

\par \textbf{THEAETETUS}
\par   Certainly.

\par \textbf{STRANGER}
\par   Then, Theaetetus, our inference is, that if there is no motion, neither is there any mind anywhere, or about anything or belonging to any one.

\par \textbf{THEAETETUS}
\par   Quite true.

\par \textbf{STRANGER}
\par   And yet this equally follows, if we grant that all things are in motion—upon this view too mind has no existence.

\par \textbf{THEAETETUS}
\par   How so?

\par \textbf{STRANGER}
\par   Do you think that sameness of condition and mode and subject could ever exist without a principle of rest?

\par \textbf{THEAETETUS}
\par   Certainly not.

\par \textbf{STRANGER}
\par   Can you see how without them mind could exist, or come into existence anywhere?

\par \textbf{THEAETETUS}
\par   No.

\par \textbf{STRANGER}
\par   And surely contend we must in every possible way against him who would annihilate knowledge and reason and mind, and yet ventures to speak confidently about anything.

\par \textbf{THEAETETUS}
\par   Yes, with all our might.

\par \textbf{STRANGER}
\par   Then the philosopher, who has the truest reverence for these qualities, cannot possibly accept the notion of those who say that the whole is at rest, either as unity or in many forms:  and he will be utterly deaf to those who assert universal motion. As children say entreatingly 'Give us both,' so he will include both the moveable and immoveable in his definition of being and all.

\par \textbf{THEAETETUS}
\par   Most true.

\par \textbf{STRANGER}
\par   And now, do we seem to have gained a fair notion of being?

\par \textbf{THEAETETUS}
\par   Yes truly.

\par \textbf{STRANGER}
\par   Alas, Theaetetus, methinks that we are now only beginning to see the real difficulty of the enquiry into the nature of it.

\par \textbf{THEAETETUS}
\par   What do you mean?

\par \textbf{STRANGER}
\par   O my friend, do you not see that nothing can exceed our ignorance, and yet we fancy that we are saying something good?

\par \textbf{THEAETETUS}
\par   I certainly thought that we were; and I do not at all understand how we never found out our desperate case.

\par \textbf{STRANGER}
\par   Reflect:  after having made these admissions, may we not be justly asked the same questions which we ourselves were asking of those who said that all was hot and cold?

\par \textbf{THEAETETUS}
\par   What were they? Will you recall them to my mind?

\par \textbf{STRANGER}
\par   To be sure I will, and I will remind you of them, by putting the same questions to you which I did to them, and then we shall get on.

\par \textbf{THEAETETUS}
\par   True.

\par \textbf{STRANGER}
\par   Would you not say that rest and motion are in the most entire opposition to one another?

\par \textbf{THEAETETUS}
\par   Of course.

\par \textbf{STRANGER}
\par   And yet you would say that both and either of them equally are?

\par \textbf{THEAETETUS}
\par   I should.

\par \textbf{STRANGER}
\par   And when you admit that both or either of them are, do you mean to say that both or either of them are in motion?

\par \textbf{THEAETETUS}
\par   Certainly not.

\par \textbf{STRANGER}
\par   Or do you wish to imply that they are both at rest, when you say that they are?

\par \textbf{THEAETETUS}
\par   Of course not.

\par \textbf{STRANGER}
\par   Then you conceive of being as some third and distinct nature, under which rest and motion are alike included; and, observing that they both participate in being, you declare that they are.

\par \textbf{THEAETETUS}
\par   Truly we seem to have an intimation that being is some third thing, when we say that rest and motion are.

\par \textbf{STRANGER}
\par   Then being is not the combination of rest and motion, but something different from them.

\par \textbf{THEAETETUS}
\par   So it would appear.

\par \textbf{STRANGER}
\par   Being, then, according to its own nature, is neither in motion nor at rest.

\par \textbf{THEAETETUS}
\par   That is very much the truth.

\par \textbf{STRANGER}
\par   Where, then, is a man to look for help who would have any clear or fixed notion of being in his mind?

\par \textbf{THEAETETUS}
\par   Where, indeed?

\par \textbf{STRANGER}
\par   I scarcely think that he can look anywhere; for that which is not in motion must be at rest, and again, that which is not at rest must be in motion; but being is placed outside of both these classes. Is this possible?

\par \textbf{THEAETETUS}
\par   Utterly impossible.

\par \textbf{STRANGER}
\par   Here, then, is another thing which we ought to bear in mind.

\par \textbf{THEAETETUS}
\par   What?

\par \textbf{STRANGER}
\par   When we were asked to what we were to assign the appellation of not-being, we were in the greatest difficulty: —do you remember?

\par \textbf{THEAETETUS}
\par   To be sure.

\par \textbf{STRANGER}
\par   And are we not now in as great a difficulty about being?

\par \textbf{THEAETETUS}
\par   I should say, Stranger, that we are in one which is, if possible, even greater.

\par \textbf{STRANGER}
\par   Then let us acknowledge the difficulty; and as being and not-being are involved in the same perplexity, there is hope that when the one appears more or less distinctly, the other will equally appear; and if we are able to see neither, there may still be a chance of steering our way in between them, without any great discredit.

\par \textbf{THEAETETUS}
\par   Very good.

\par \textbf{STRANGER}
\par   Let us enquire, then, how we come to predicate many names of the same thing.

\par \textbf{THEAETETUS}
\par   Give an example.

\par \textbf{STRANGER}
\par   I mean that we speak of man, for example, under many names—that we attribute to him colours and forms and magnitudes and virtues and vices, in all of which instances and in ten thousand others we not only speak of him as a man, but also as good, and having numberless other attributes, and in the same way anything else which we originally supposed to be one is described by us as many, and under many names.

\par \textbf{THEAETETUS}
\par   That is true.

\par \textbf{STRANGER}
\par   And thus we provide a rich feast for tyros, whether young or old; for there is nothing easier than to argue that the one cannot be many, or the many one; and great is their delight in denying that a man is good; for man, they insist, is man and good is good. I dare say that you have met with persons who take an interest in such matters—they are often elderly men, whose meagre sense is thrown into amazement by these discoveries of theirs, which they believe to be the height of wisdom.

\par \textbf{THEAETETUS}
\par   Certainly, I have.

\par \textbf{STRANGER}
\par   Then, not to exclude any one who has ever speculated at all upon the nature of being, let us put our questions to them as well as to our former friends.

\par \textbf{THEAETETUS}
\par   What questions?

\par \textbf{STRANGER}
\par   Shall we refuse to attribute being to motion and rest, or anything to anything, and assume that they do not mingle, and are incapable of participating in one another? Or shall we gather all into one class of things communicable with one another? Or are some things communicable and others not?—Which of these alternatives, Theaetetus, will they prefer?

\par \textbf{THEAETETUS}
\par   I have nothing to answer on their behalf. Suppose that you take all these hypotheses in turn, and see what are the consequences which follow from each of them.

\par \textbf{STRANGER}
\par   Very good, and first let us assume them to say that nothing is capable of participating in anything else in any respect; in that case rest and motion cannot participate in being at all.

\par \textbf{THEAETETUS}
\par   They cannot.

\par \textbf{STRANGER}
\par   But would either of them be if not participating in being?

\par \textbf{THEAETETUS}
\par   No.

\par \textbf{STRANGER}
\par   Then by this admission everything is instantly overturned, as well the doctrine of universal motion as of universal rest, and also the doctrine of those who distribute being into immutable and everlasting kinds; for all these add on a notion of being, some affirming that things 'are' truly in motion, and others that they 'are' truly at rest.

\par \textbf{THEAETETUS}
\par   Just so.

\par \textbf{STRANGER}
\par   Again, those who would at one time compound, and at another resolve all things, whether making them into one and out of one creating infinity, or dividing them into finite elements, and forming compounds out of these; whether they suppose the processes of creation to be successive or continuous, would be talking nonsense in all this if there were no admixture.

\par \textbf{THEAETETUS}
\par   True.

\par \textbf{STRANGER}
\par   Most ridiculous of all will the men themselves be who want to carry out the argument and yet forbid us to call anything, because participating in some affection from another, by the name of that other.

\par \textbf{THEAETETUS}
\par   Why so?

\par \textbf{STRANGER}
\par   Why, because they are compelled to use the words 'to be,' 'apart,' 'from others,' 'in itself,' and ten thousand more, which they cannot give up, but must make the connecting links of discourse; and therefore they do not require to be refuted by others, but their enemy, as the saying is, inhabits the same house with them; they are always carrying about with them an adversary, like the wonderful ventriloquist, Eurycles, who out of their own bellies audibly contradicts them.

\par \textbf{THEAETETUS}
\par   Precisely so; a very true and exact illustration.

\par \textbf{STRANGER}
\par   And now, if we suppose that all things have the power of communion with one another—what will follow?

\par \textbf{THEAETETUS}
\par   Even I can solve that riddle.

\par \textbf{STRANGER}
\par   How?

\par \textbf{THEAETETUS}
\par   Why, because motion itself would be at rest, and rest again in motion, if they could be attributed to one another.

\par \textbf{STRANGER}
\par   But this is utterly impossible.

\par \textbf{THEAETETUS}
\par   Of course.

\par \textbf{STRANGER}
\par   Then only the third hypothesis remains.

\par \textbf{THEAETETUS}
\par   True.

\par \textbf{STRANGER}
\par   For, surely, either all things have communion with all; or nothing with any other thing; or some things communicate with some things and others not.

\par \textbf{THEAETETUS}
\par   Certainly.

\par \textbf{STRANGER}
\par   And two out of these three suppositions have been found to be impossible.

\par \textbf{THEAETETUS}
\par   Yes.

\par \textbf{STRANGER}
\par   Every one then, who desires to answer truly, will adopt the third and remaining hypothesis of the communion of some with some.

\par \textbf{THEAETETUS}
\par   Quite true.

\par \textbf{STRANGER}
\par   This communion of some with some may be illustrated by the case of letters; for some letters do not fit each other, while others do.

\par \textbf{THEAETETUS}
\par   Of course.

\par \textbf{STRANGER}
\par   And the vowels, especially, are a sort of bond which pervades all the other letters, so that without a vowel one consonant cannot be joined to another.

\par \textbf{THEAETETUS}
\par   True.

\par \textbf{STRANGER}
\par   But does every one know what letters will unite with what? Or is art required in order to do so?

\par \textbf{THEAETETUS}
\par   Art is required.

\par \textbf{STRANGER}
\par   What art?

\par \textbf{THEAETETUS}
\par   The art of grammar.

\par \textbf{STRANGER}
\par   And is not this also true of sounds high and low?—Is not he who has the art to know what sounds mingle, a musician, and he who is ignorant, not a musician?

\par \textbf{THEAETETUS}
\par   Yes.

\par \textbf{STRANGER}
\par   And we shall find this to be generally true of art or the absence of art.

\par \textbf{THEAETETUS}
\par   Of course.

\par \textbf{STRANGER}
\par   And as classes are admitted by us in like manner to be some of them capable and others incapable of intermixture, must not he who would rightly show what kinds will unite and what will not, proceed by the help of science in the path of argument? And will he not ask if the connecting links are universal, and so capable of intermixture with all things; and again, in divisions, whether there are not other universal classes, which make them possible?

\par \textbf{THEAETETUS}
\par   To be sure he will require science, and, if I am not mistaken, the very greatest of all sciences.

\par \textbf{STRANGER}
\par   How are we to call it? By Zeus, have we not lighted unwittingly upon our free and noble science, and in looking for the Sophist have we not entertained the philosopher unawares?

\par \textbf{THEAETETUS}
\par   What do you mean?

\par \textbf{STRANGER}
\par   Should we not say that the division according to classes, which neither makes the same other, nor makes other the same, is the business of the dialectical science?

\par \textbf{THEAETETUS}
\par   That is what we should say.

\par \textbf{STRANGER}
\par   Then, surely, he who can divide rightly is able to see clearly one form pervading a scattered multitude, and many different forms contained under one higher form; and again, one form knit together into a single whole and pervading many such wholes, and many forms, existing only in separation and isolation. This is the knowledge of classes which determines where they can have communion with one another and where not.

\par \textbf{THEAETETUS}
\par   Quite true.

\par \textbf{STRANGER}
\par   And the art of dialectic would be attributed by you only to the philosopher pure and true?

\par \textbf{THEAETETUS}
\par   Who but he can be worthy?

\par \textbf{STRANGER}
\par   In this region we shall always discover the philosopher, if we look for him; like the Sophist, he is not easily discovered, but for a different reason.

\par \textbf{THEAETETUS}
\par   For what reason?

\par \textbf{STRANGER}
\par   Because the Sophist runs away into the darkness of not-being, in which he has learned by habit to feel about, and cannot be discovered because of the darkness of the place. Is not that true?

\par \textbf{THEAETETUS}
\par   It seems to be so.

\par \textbf{STRANGER}
\par   And the philosopher, always holding converse through reason with the idea of being, is also dark from excess of light; for the souls of the many have no eye which can endure the vision of the divine.

\par \textbf{THEAETETUS}
\par   Yes; that seems to be quite as true as the other.

\par \textbf{STRANGER}
\par   Well, the philosopher may hereafter be more fully considered by us, if we are disposed; but the Sophist must clearly not be allowed to escape until we have had a good look at him.

\par \textbf{THEAETETUS}
\par   Very good.

\par \textbf{STRANGER}
\par   Since, then, we are agreed that some classes have a communion with one another, and others not, and some have communion with a few and others with many, and that there is no reason why some should not have universal communion with all, let us now pursue the enquiry, as the argument suggests, not in relation to all ideas, lest the multitude of them should confuse us, but let us select a few of those which are reckoned to be the principal ones, and consider their several natures and their capacity of communion with one another, in order that if we are not able to apprehend with perfect clearness the notions of being and not-being, we may at least not fall short in the consideration of them, so far as they come within the scope of the present enquiry, if peradventure we may be allowed to assert the reality of not-being, and yet escape unscathed.

\par \textbf{THEAETETUS}
\par   We must do so.

\par \textbf{STRANGER}
\par   The most important of all the genera are those which we were just now mentioning—being and rest and motion.

\par \textbf{THEAETETUS}
\par   Yes, by far.

\par \textbf{STRANGER}
\par   And two of these are, as we affirm, incapable of communion with one another.

\par \textbf{THEAETETUS}
\par   Quite incapable.

\par \textbf{STRANGER}
\par   Whereas being surely has communion with both of them, for both of them are?

\par \textbf{THEAETETUS}
\par   Of course.

\par \textbf{STRANGER}
\par   That makes up three of them.

\par \textbf{THEAETETUS}
\par   To be sure.

\par \textbf{STRANGER}
\par   And each of them is other than the remaining two, but the same with itself.

\par \textbf{THEAETETUS}
\par   True.

\par \textbf{STRANGER}
\par   But then, what is the meaning of these two words, 'same' and 'other'? Are they two new kinds other than the three, and yet always of necessity intermingling with them, and are we to have five kinds instead of three; or when we speak of the same and other, are we unconsciously speaking of one of the three first kinds?

\par \textbf{THEAETETUS}
\par   Very likely we are.

\par \textbf{STRANGER}
\par   But, surely, motion and rest are neither the other nor the same.

\par \textbf{THEAETETUS}
\par   How is that?

\par \textbf{STRANGER}
\par   Whatever we attribute to motion and rest in common, cannot be either of them.

\par \textbf{THEAETETUS}
\par   Why not?

\par \textbf{STRANGER}
\par   Because motion would be at rest and rest in motion, for either of them, being predicated of both, will compel the other to change into the opposite of its own nature, because partaking of its opposite.

\par \textbf{THEAETETUS}
\par   Quite true.

\par \textbf{STRANGER}
\par   Yet they surely both partake of the same and of the other?

\par \textbf{THEAETETUS}
\par   Yes.

\par \textbf{STRANGER}
\par   Then we must not assert that motion, any more than rest, is either the same or the other.

\par \textbf{THEAETETUS}
\par   No; we must not.

\par \textbf{STRANGER}
\par   But are we to conceive that being and the same are identical?

\par \textbf{THEAETETUS}
\par   Possibly.

\par \textbf{STRANGER}
\par   But if they are identical, then again in saying that motion and rest have being, we should also be saying that they are the same.

\par \textbf{THEAETETUS}
\par   Which surely cannot be.

\par \textbf{STRANGER}
\par   Then being and the same cannot be one.

\par \textbf{THEAETETUS}
\par   Scarcely.

\par \textbf{STRANGER}
\par   Then we may suppose the same to be a fourth class, which is now to be added to the three others.

\par \textbf{THEAETETUS}
\par   Quite true.

\par \textbf{STRANGER}
\par   And shall we call the other a fifth class? Or should we consider being and other to be two names of the same class?

\par \textbf{THEAETETUS}
\par   Very likely.

\par \textbf{STRANGER}
\par   But you would agree, if I am not mistaken, that existences are relative as well as absolute?

\par \textbf{THEAETETUS}
\par   Certainly.

\par \textbf{STRANGER}
\par   And the other is always relative to other?

\par \textbf{THEAETETUS}
\par   True.

\par \textbf{STRANGER}
\par   But this would not be the case unless being and the other entirely differed; for, if the other, like being, were absolute as well as relative, then there would have been a kind of other which was not other than other. And now we find that what is other must of necessity be what it is in relation to some other.

\par \textbf{THEAETETUS}
\par   That is the true state of the case.

\par \textbf{STRANGER}
\par   Then we must admit the other as the fifth of our selected classes.

\par \textbf{THEAETETUS}
\par   Yes.

\par \textbf{STRANGER}
\par   And the fifth class pervades all classes, for they all differ from one another, not by reason of their own nature, but because they partake of the idea of the other.

\par \textbf{THEAETETUS}
\par   Quite true.

\par \textbf{STRANGER}
\par   Then let us now put the case with reference to each of the five.

\par \textbf{THEAETETUS}
\par   How?

\par \textbf{STRANGER}
\par   First there is motion, which we affirm to be absolutely 'other' than rest:  what else can we say?

\par \textbf{THEAETETUS}
\par   It is so.

\par \textbf{STRANGER}
\par   And therefore is not rest.

\par \textbf{THEAETETUS}
\par   Certainly not.

\par \textbf{STRANGER}
\par   And yet is, because partaking of being.

\par \textbf{THEAETETUS}
\par   True.

\par \textbf{STRANGER}
\par   Again, motion is other than the same?

\par \textbf{THEAETETUS}
\par   Just so.

\par \textbf{STRANGER}
\par   And is therefore not the same.

\par \textbf{THEAETETUS}
\par   It is not.

\par \textbf{STRANGER}
\par   Yet, surely, motion is the same, because all things partake of the same.

\par \textbf{THEAETETUS}
\par   Very true.

\par \textbf{STRANGER}
\par   Then we must admit, and not object to say, that motion is the same and is not the same, for we do not apply the terms 'same' and 'not the same,' in the same sense; but we call it the 'same,' in relation to itself, because partaking of the same; and not the same, because having communion with the other, it is thereby severed from the same, and has become not that but other, and is therefore rightly spoken of as 'not the same.'

\par \textbf{THEAETETUS}
\par   To be sure.

\par \textbf{STRANGER}
\par   And if absolute motion in any point of view partook of rest, there would be no absurdity in calling motion stationary.

\par \textbf{THEAETETUS}
\par   Quite right,—that is, on the supposition that some classes mingle with one another, and others not.

\par \textbf{STRANGER}
\par   That such a communion of kinds is according to nature, we had already proved before we arrived at this part of our discussion.

\par \textbf{THEAETETUS}
\par   Certainly.

\par \textbf{STRANGER}
\par   Let us proceed, then. May we not say that motion is other than the other, having been also proved by us to be other than the same and other than rest?

\par \textbf{THEAETETUS}
\par   That is certain.

\par \textbf{STRANGER}
\par   Then, according to this view, motion is other and also not other?

\par \textbf{THEAETETUS}
\par   True.

\par \textbf{STRANGER}
\par   What is the next step? Shall we say that motion is other than the three and not other than the fourth,—for we agreed that there are five classes about and in the sphere of which we proposed to make enquiry?

\par \textbf{THEAETETUS}
\par   Surely we cannot admit that the number is less than it appeared to be just now.

\par \textbf{STRANGER}
\par   Then we may without fear contend that motion is other than being?

\par \textbf{THEAETETUS}
\par   Without the least fear.

\par \textbf{STRANGER}
\par   The plain result is that motion, since it partakes of being, really is and also is not?

\par \textbf{THEAETETUS}
\par   Nothing can be plainer.

\par \textbf{STRANGER}
\par   Then not-being necessarily exists in the case of motion and of every class; for the nature of the other entering into them all, makes each of them other than being, and so non-existent; and therefore of all of them, in like manner, we may truly say that they are not; and again, inasmuch as they partake of being, that they are and are existent.

\par \textbf{THEAETETUS}
\par   So we may assume.

\par \textbf{STRANGER}
\par   Every class, then, has plurality of being and infinity of not-being.

\par \textbf{THEAETETUS}
\par   So we must infer.

\par \textbf{STRANGER}
\par   And being itself may be said to be other than the other kinds.

\par \textbf{THEAETETUS}
\par   Certainly.

\par \textbf{STRANGER}
\par   Then we may infer that being is not, in respect of as many other things as there are; for not-being these it is itself one, and is not the other things, which are infinite in number.

\par \textbf{THEAETETUS}
\par   That is not far from the truth.

\par \textbf{STRANGER}
\par   And we must not quarrel with this result, since it is of the nature of classes to have communion with one another; and if any one denies our present statement [viz., that being is not, etc. ], let him first argue with our former conclusion [i.e., respecting the communion of ideas], and then he may proceed to argue with what follows.

\par \textbf{THEAETETUS}
\par   Nothing can be fairer.

\par \textbf{STRANGER}
\par   Let me ask you to consider a further question.

\par \textbf{THEAETETUS}
\par   What question?

\par \textbf{STRANGER}
\par   When we speak of not-being, we speak, I suppose, not of something opposed to being, but only different.

\par \textbf{THEAETETUS}
\par   What do you mean?

\par \textbf{STRANGER}
\par   When we speak of something as not great, does the expression seem to you to imply what is little any more than what is equal?

\par \textbf{THEAETETUS}
\par   Certainly not.

\par \textbf{STRANGER}
\par   The negative particles, ou and me, when prefixed to words, do not imply opposition, but only difference from the words, or more correctly from the things represented by the words, which follow them.

\par \textbf{THEAETETUS}
\par   Quite true.

\par \textbf{STRANGER}
\par   There is another point to be considered, if you do not object.

\par \textbf{THEAETETUS}
\par   What is it?

\par \textbf{STRANGER}
\par   The nature of the other appears to me to be divided into fractions like knowledge.

\par \textbf{THEAETETUS}
\par   How so?

\par \textbf{STRANGER}
\par   Knowledge, like the other, is one; and yet the various parts of knowledge have each of them their own particular name, and hence there are many arts and kinds of knowledge.

\par \textbf{THEAETETUS}
\par   Quite true.

\par \textbf{STRANGER}
\par   And is not the case the same with the parts of the other, which is also one?

\par \textbf{THEAETETUS}
\par   Very likely; but will you tell me how?

\par \textbf{STRANGER}
\par   There is some part of the other which is opposed to the beautiful?

\par \textbf{THEAETETUS}
\par   There is.

\par \textbf{STRANGER}
\par   Shall we say that this has or has not a name?

\par \textbf{THEAETETUS}
\par   It has; for whatever we call not-beautiful is other than the beautiful, not than something else.

\par \textbf{STRANGER}
\par   And now tell me another thing.

\par \textbf{THEAETETUS}
\par   What?

\par \textbf{STRANGER}
\par   Is the not-beautiful anything but this—an existence parted off from a certain kind of existence, and again from another point of view opposed to an existing something?

\par \textbf{THEAETETUS}
\par   True.

\par \textbf{STRANGER}
\par   Then the not-beautiful turns out to be the opposition of being to being?

\par \textbf{THEAETETUS}
\par   Very true.

\par \textbf{STRANGER}
\par   But upon this view, is the beautiful a more real and the not-beautiful a less real existence?

\par \textbf{THEAETETUS}
\par   Not at all.

\par \textbf{STRANGER}
\par   And the not-great may be said to exist, equally with the great?

\par \textbf{THEAETETUS}
\par   Yes.

\par \textbf{STRANGER}
\par   And, in the same way, the just must be placed in the same category with the not-just—the one cannot be said to have any more existence than the other.

\par \textbf{THEAETETUS}
\par   True.

\par \textbf{STRANGER}
\par   The same may be said of other things; seeing that the nature of the other has a real existence, the parts of this nature must equally be supposed to exist.

\par \textbf{THEAETETUS}
\par   Of course.

\par \textbf{STRANGER}
\par   Then, as would appear, the opposition of a part of the other, and of a part of being, to one another, is, if I may venture to say so, as truly essence as being itself, and implies not the opposite of being, but only what is other than being.

\par \textbf{THEAETETUS}
\par   Beyond question.

\par \textbf{STRANGER}
\par   What then shall we call it?

\par \textbf{THEAETETUS}
\par   Clearly, not-being; and this is the very nature for which the Sophist compelled us to search.

\par \textbf{STRANGER}
\par   And has not this, as you were saying, as real an existence as any other class? May I not say with confidence that not-being has an assured existence, and a nature of its own? Just as the great was found to be great and the beautiful beautiful, and the not-great not-great, and the not-beautiful not-beautiful, in the same manner not-being has been found to be and is not-being, and is to be reckoned one among the many classes of being. Do you, Theaetetus, still feel any doubt of this?

\par \textbf{THEAETETUS}
\par   None whatever.

\par \textbf{STRANGER}
\par   Do you observe that our scepticism has carried us beyond the range of Parmenides' prohibition?

\par \textbf{THEAETETUS}
\par   In what?

\par \textbf{STRANGER}
\par   We have advanced to a further point, and shown him more than he forbad us to investigate.

\par \textbf{THEAETETUS}
\par   How is that?

\par \textbf{STRANGER}
\par   Why, because he says—

\par  'Not-being never is, and do thou keep thy thoughts from this way of enquiry.'

\par \textbf{THEAETETUS}
\par   Yes, he says so.

\par \textbf{STRANGER}
\par   Whereas, we have not only proved that things which are not are, but we have shown what form of being not-being is; for we have shown that the nature of the other is, and is distributed over all things in their relations to one another, and whatever part of the other is contrasted with being, this is precisely what we have ventured to call not-being.

\par \textbf{THEAETETUS}
\par   And surely, Stranger, we were quite right.

\par \textbf{STRANGER}
\par   Let not any one say, then, that while affirming the opposition of not-being to being, we still assert the being of not-being; for as to whether there is an opposite of being, to that enquiry we have long said good-bye—it may or may not be, and may or may not be capable of definition. But as touching our present account of not-being, let a man either convince us of error, or, so long as he cannot, he too must say, as we are saying, that there is a communion of classes, and that being, and difference or other, traverse all things and mutually interpenetrate, so that the other partakes of being, and by reason of this participation is, and yet is not that of which it partakes, but other, and being other than being, it is clearly a necessity that not-being should be. And again, being, through partaking of the other, becomes a class other than the remaining classes, and being other than all of them, is not each one of them, and is not all the rest, so that undoubtedly there are thousands upon thousands of cases in which being is not, and all other things, whether regarded individually or collectively, in many respects are, and in many respects are not.

\par \textbf{THEAETETUS}
\par   True.

\par \textbf{STRANGER}
\par   And he who is sceptical of this contradiction, must think how he can find something better to say; or if he sees a puzzle, and his pleasure is to drag words this way and that, the argument will prove to him, that he is not making a worthy use of his faculties; for there is no charm in such puzzles, and there is no difficulty in detecting them; but we can tell him of something else the pursuit of which is noble and also difficult.

\par \textbf{THEAETETUS}
\par   What is it?

\par \textbf{STRANGER}
\par   A thing of which I have already spoken;—letting alone these puzzles as involving no difficulty, he should be able to follow and criticize in detail every argument, and when a man says that the same is in a manner other, or that other is the same, to understand and refute him from his own point of view, and in the same respect in which he asserts either of these affections. But to show that somehow and in some sense the same is other, or the other same, or the great small, or the like unlike; and to delight in always bringing forward such contradictions, is no real refutation, but is clearly the new-born babe of some one who is only beginning to approach the problem of being.

\par \textbf{THEAETETUS}
\par   To be sure.

\par \textbf{STRANGER}
\par   For certainly, my friend, the attempt to separate all existences from one another is a barbarism and utterly unworthy of an educated or philosophical mind.

\par \textbf{THEAETETUS}
\par   Why so?

\par \textbf{STRANGER}
\par   The attempt at universal separation is the final annihilation of all reasoning; for only by the union of conceptions with one another do we attain to discourse of reason.

\par \textbf{THEAETETUS}
\par   True.

\par \textbf{STRANGER}
\par   And, observe that we were only just in time in making a resistance to such separatists, and compelling them to admit that one thing mingles with another.

\par \textbf{THEAETETUS}
\par   Why so?

\par \textbf{STRANGER}
\par   Why, that we might be able to assert discourse to be a kind of being; for if we could not, the worst of all consequences would follow; we should have no philosophy. Moreover, the necessity for determining the nature of discourse presses upon us at this moment; if utterly deprived of it, we could no more hold discourse; and deprived of it we should be if we admitted that there was no admixture of natures at all.

\par \textbf{THEAETETUS}
\par   Very true. But I do not understand why at this moment we must determine the nature of discourse.

\par \textbf{STRANGER}
\par   Perhaps you will see more clearly by the help of the following explanation.

\par \textbf{THEAETETUS}
\par   What explanation?

\par \textbf{STRANGER}
\par   Not-being has been acknowledged by us to be one among many classes diffused over all being.

\par \textbf{THEAETETUS}
\par   True.

\par \textbf{STRANGER}
\par   And thence arises the question, whether not-being mingles with opinion and language.

\par \textbf{THEAETETUS}
\par   How so?

\par \textbf{STRANGER}
\par   If not-being has no part in the proposition, then all things must be true; but if not-being has a part, then false opinion and false speech are possible, for to think or to say what is not—is falsehood, which thus arises in the region of thought and in speech.

\par \textbf{THEAETETUS}
\par   That is quite true.

\par \textbf{STRANGER}
\par   And where there is falsehood surely there must be deceit.

\par \textbf{THEAETETUS}
\par   Yes.

\par \textbf{STRANGER}
\par   And if there is deceit, then all things must be full of idols and images and fancies.

\par \textbf{THEAETETUS}
\par   To be sure.

\par \textbf{STRANGER}
\par   Into that region the Sophist, as we said, made his escape, and, when he had got there, denied the very possibility of falsehood; no one, he argued, either conceived or uttered falsehood, inasmuch as not-being did not in any way partake of being.

\par \textbf{THEAETETUS}
\par   True.

\par \textbf{STRANGER}
\par   And now, not-being has been shown to partake of being, and therefore he will not continue fighting in this direction, but he will probably say that some ideas partake of not-being, and some not, and that language and opinion are of the non-partaking class; and he will still fight to the death against the existence of the image-making and phantastic art, in which we have placed him, because, as he will say, opinion and language do not partake of not-being, and unless this participation exists, there can be no such thing as falsehood. And, with the view of meeting this evasion, we must begin by enquiring into the nature of language, opinion, and imagination, in order that when we find them we may find also that they have communion with not-being, and, having made out the connexion of them, may thus prove that falsehood exists; and therein we will imprison the Sophist, if he deserves it, or, if not, we will let him go again and look for him in another class.

\par \textbf{THEAETETUS}
\par   Certainly, Stranger, there appears to be truth in what was said about the Sophist at first, that he was of a class not easily caught, for he seems to have abundance of defences, which he throws up, and which must every one of them be stormed before we can reach the man himself. And even now, we have with difficulty got through his first defence, which is the not-being of not-being, and lo! here is another; for we have still to show that falsehood exists in the sphere of language and opinion, and there will be another and another line of defence without end.

\par \textbf{STRANGER}
\par   Any one, Theaetetus, who is able to advance even a little ought to be of good cheer, for what would he who is dispirited at a little progress do, if he were making none at all, or even undergoing a repulse? Such a faint heart, as the proverb says, will never take a city:  but now that we have succeeded thus far, the citadel is ours, and what remains is easier.

\par \textbf{THEAETETUS}
\par   Very true.

\par \textbf{STRANGER}
\par   Then, as I was saying, let us first of all obtain a conception of language and opinion, in order that we may have clearer grounds for determining, whether not-being has any concern with them, or whether they are both always true, and neither of them ever false.

\par \textbf{THEAETETUS}
\par   True.

\par \textbf{STRANGER}
\par   Then, now, let us speak of names, as before we were speaking of ideas and letters; for that is the direction in which the answer may be expected.

\par \textbf{THEAETETUS}
\par   And what is the question at issue about names?

\par \textbf{STRANGER}
\par   The question at issue is whether all names may be connected with one another, or none, or only some of them.

\par \textbf{THEAETETUS}
\par   Clearly the last is true.

\par \textbf{STRANGER}
\par   I understand you to say that words which have a meaning when in sequence may be connected, but that words which have no meaning when in sequence cannot be connected?

\par \textbf{THEAETETUS}
\par   What are you saying?

\par \textbf{STRANGER}
\par   What I thought that you intended when you gave your assent; for there are two sorts of intimation of being which are given by the voice.

\par \textbf{THEAETETUS}
\par   What are they?

\par \textbf{STRANGER}
\par   One of them is called nouns, and the other verbs.

\par \textbf{THEAETETUS}
\par   Describe them.

\par \textbf{STRANGER}
\par   That which denotes action we call a verb.

\par \textbf{THEAETETUS}
\par   True.

\par \textbf{STRANGER}
\par   And the other, which is an articulate mark set on those who do the actions, we call a noun.

\par \textbf{THEAETETUS}
\par   Quite true.

\par \textbf{STRANGER}
\par   A succession of nouns only is not a sentence, any more than of verbs without nouns.

\par \textbf{THEAETETUS}
\par   I do not understand you.

\par \textbf{STRANGER}
\par   I see that when you gave your assent you had something else in your mind. But what I intended to say was, that a mere succession of nouns or of verbs is not discourse.

\par \textbf{THEAETETUS}
\par   What do you mean?

\par \textbf{STRANGER}
\par   I mean that words like 'walks,' 'runs,' 'sleeps,' or any other words which denote action, however many of them you string together, do not make discourse.

\par \textbf{THEAETETUS}
\par   How can they?

\par \textbf{STRANGER}
\par   Or, again, when you say 'lion,' 'stag,' 'horse,' or any other words which denote agents—neither in this way of stringing words together do you attain to discourse; for there is no expression of action or inaction, or of the existence of existence or non-existence indicated by the sounds, until verbs are mingled with nouns; then the words fit, and the smallest combination of them forms language, and is the simplest and least form of discourse.

\par \textbf{THEAETETUS}
\par   Again I ask, What do you mean?

\par \textbf{STRANGER}
\par   When any one says 'A man learns,' should you not call this the simplest and least of sentences?

\par \textbf{THEAETETUS}
\par   Yes.

\par \textbf{STRANGER}
\par   Yes, for he now arrives at the point of giving an intimation about something which is, or is becoming, or has become, or will be. And he not only names, but he does something, by connecting verbs with nouns; and therefore we say that he discourses, and to this connexion of words we give the name of discourse.

\par \textbf{THEAETETUS}
\par   True.

\par \textbf{STRANGER}
\par   And as there are some things which fit one another, and other things which do not fit, so there are some vocal signs which do, and others which do not, combine and form discourse.

\par \textbf{THEAETETUS}
\par   Quite true.

\par \textbf{STRANGER}
\par   There is another small matter.

\par \textbf{THEAETETUS}
\par   What is it?

\par \textbf{STRANGER}
\par   A sentence must and cannot help having a subject.

\par \textbf{THEAETETUS}
\par   True.

\par \textbf{STRANGER}
\par   And must be of a certain quality.

\par \textbf{THEAETETUS}
\par   Certainly.

\par \textbf{STRANGER}
\par   And now let us mind what we are about.

\par \textbf{THEAETETUS}
\par   We must do so.

\par \textbf{STRANGER}
\par   I will repeat a sentence to you in which a thing and an action are combined, by the help of a noun and a verb; and you shall tell me of whom the sentence speaks.

\par \textbf{THEAETETUS}
\par   I will, to the best of my power.

\par \textbf{STRANGER}
\par   'Theaetetus sits'—not a very long sentence.

\par \textbf{THEAETETUS}
\par   Not very.

\par \textbf{STRANGER}
\par   Of whom does the sentence speak, and who is the subject? that is what you have to tell.

\par \textbf{THEAETETUS}
\par   Of me; I am the subject.

\par \textbf{STRANGER}
\par   Or this sentence, again—

\par \textbf{THEAETETUS}
\par   What sentence?

\par \textbf{STRANGER}
\par   'Theaetetus, with whom I am now speaking, is flying.'

\par \textbf{THEAETETUS}
\par   That also is a sentence which will be admitted by every one to speak of me, and to apply to me.

\par \textbf{STRANGER}
\par   We agreed that every sentence must necessarily have a certain quality.

\par \textbf{THEAETETUS}
\par   Yes.

\par \textbf{STRANGER}
\par   And what is the quality of each of these two sentences?

\par \textbf{THEAETETUS}
\par   The one, as I imagine, is false, and the other true.

\par \textbf{STRANGER}
\par   The true says what is true about you?

\par \textbf{THEAETETUS}
\par   Yes.

\par \textbf{STRANGER}
\par   And the false says what is other than true?

\par \textbf{THEAETETUS}
\par   Yes.

\par \textbf{STRANGER}
\par   And therefore speaks of things which are not as if they were?

\par \textbf{THEAETETUS}
\par   True.

\par \textbf{STRANGER}
\par   And say that things are real of you which are not; for, as we were saying, in regard to each thing or person, there is much that is and much that is not.

\par \textbf{THEAETETUS}
\par   Quite true.

\par \textbf{STRANGER}
\par   The second of the two sentences which related to you was first of all an example of the shortest form consistent with our definition.

\par \textbf{THEAETETUS}
\par   Yes, this was implied in recent admission.

\par \textbf{STRANGER}
\par   And, in the second place, it related to a subject?

\par \textbf{THEAETETUS}
\par   Yes.

\par \textbf{STRANGER}
\par   Who must be you, and can be nobody else?

\par \textbf{THEAETETUS}
\par   Unquestionably.

\par \textbf{STRANGER}
\par   And it would be no sentence at all if there were no subject, for, as we proved, a sentence which has no subject is impossible.

\par \textbf{THEAETETUS}
\par   Quite true.

\par \textbf{STRANGER}
\par   When other, then, is asserted of you as the same, and not-being as being, such a combination of nouns and verbs is really and truly false discourse.

\par \textbf{THEAETETUS}
\par   Most true.

\par \textbf{STRANGER}
\par   And therefore thought, opinion, and imagination are now proved to exist in our minds both as true and false.

\par \textbf{THEAETETUS}
\par   How so?

\par \textbf{STRANGER}
\par   You will know better if you first gain a knowledge of what they are, and in what they severally differ from one another.

\par \textbf{THEAETETUS}
\par   Give me the knowledge which you would wish me to gain.

\par \textbf{STRANGER}
\par   Are not thought and speech the same, with this exception, that what is called thought is the unuttered conversation of the soul with herself?

\par \textbf{THEAETETUS}
\par   Quite true.

\par \textbf{STRANGER}
\par   But the stream of thought which flows through the lips and is audible is called speech?

\par \textbf{THEAETETUS}
\par   True.

\par \textbf{STRANGER}
\par   And we know that there exists in speech...

\par \textbf{THEAETETUS}
\par   What exists?

\par \textbf{STRANGER}
\par   Affirmation.

\par \textbf{THEAETETUS}
\par   Yes, we know it.

\par \textbf{STRANGER}
\par   When the affirmation or denial takes Place in silence and in the mind only, have you any other name by which to call it but opinion?

\par \textbf{THEAETETUS}
\par   There can be no other name.

\par \textbf{STRANGER}
\par   And when opinion is presented, not simply, but in some form of sense, would you not call it imagination?

\par \textbf{THEAETETUS}
\par   Certainly.

\par \textbf{STRANGER}
\par   And seeing that language is true and false, and that thought is the conversation of the soul with herself, and opinion is the end of thinking, and imagination or phantasy is the union of sense and opinion, the inference is that some of them, since they are akin to language, should have an element of falsehood as well as of truth?

\par \textbf{THEAETETUS}
\par   Certainly.

\par \textbf{STRANGER}
\par   Do you perceive, then, that false opinion and speech have been discovered sooner than we expected?—For just now we seemed to be undertaking a task which would never be accomplished.

\par \textbf{THEAETETUS}
\par   I perceive.

\par \textbf{STRANGER}
\par   Then let us not be discouraged about the future; but now having made this discovery, let us go back to our previous classification.

\par \textbf{THEAETETUS}
\par   What classification?

\par \textbf{STRANGER}
\par   We divided image-making into two sorts; the one likeness-making, the other imaginative or phantastic.

\par \textbf{THEAETETUS}
\par   True.

\par \textbf{STRANGER}
\par   And we said that we were uncertain in which we should place the Sophist.

\par \textbf{THEAETETUS}
\par   We did say so.

\par \textbf{STRANGER}
\par   And our heads began to go round more and more when it was asserted that there is no such thing as an image or idol or appearance, because in no manner or time or place can there ever be such a thing as falsehood.

\par \textbf{THEAETETUS}
\par   True.

\par \textbf{STRANGER}
\par   And now, since there has been shown to be false speech and false opinion, there may be imitations of real existences, and out of this condition of the mind an art of deception may arise.

\par \textbf{THEAETETUS}
\par   Quite possible.

\par \textbf{STRANGER}
\par   And we have already admitted, in what preceded, that the Sophist was lurking in one of the divisions of the likeness-making art?

\par \textbf{THEAETETUS}
\par   Yes.

\par \textbf{STRANGER}
\par   Let us, then, renew the attempt, and in dividing any class, always take the part to the right, holding fast to that which holds the Sophist, until we have stripped him of all his common properties, and reached his difference or peculiar. Then we may exhibit him in his true nature, first to ourselves and then to kindred dialectical spirits.

\par \textbf{THEAETETUS}
\par   Very good.

\par \textbf{STRANGER}
\par   You may remember that all art was originally divided by us into creative and acquisitive.

\par \textbf{THEAETETUS}
\par   Yes.

\par \textbf{STRANGER}
\par   And the Sophist was flitting before us in the acquisitive class, in the subdivisions of hunting, contests, merchandize, and the like.

\par \textbf{THEAETETUS}
\par   Very true.

\par \textbf{STRANGER}
\par   But now that the imitative art has enclosed him, it is clear that we must begin by dividing the art of creation; for imitation is a kind of creation—of images, however, as we affirm, and not of real things.

\par \textbf{THEAETETUS}
\par   Quite true.

\par \textbf{STRANGER}
\par   In the first place, there are two kinds of creation.

\par \textbf{THEAETETUS}
\par   What are they?

\par \textbf{STRANGER}
\par   One of them is human and the other divine.

\par \textbf{THEAETETUS}
\par   I do not follow.

\par \textbf{STRANGER}
\par   Every power, as you may remember our saying originally, which causes things to exist, not previously existing, was defined by us as creative.

\par \textbf{THEAETETUS}
\par   I remember.

\par \textbf{STRANGER}
\par   Looking, now, at the world and all the animals and plants, at things which grow upon the earth from seeds and roots, as well as at inanimate substances which are formed within the earth, fusile or non-fusile, shall we say that they come into existence—not having existed previously—by the creation of God, or shall we agree with vulgar opinion about them?

\par \textbf{THEAETETUS}
\par   What is it?

\par \textbf{STRANGER}
\par   The opinion that nature brings them into being from some spontaneous and unintelligent cause. Or shall we say that they are created by a divine reason and a knowledge which comes from God?

\par \textbf{THEAETETUS}
\par   I dare say that, owing to my youth, I may often waver in my view, but now when I look at you and see that you incline to refer them to God, I defer to your authority.

\par \textbf{STRANGER}
\par   Nobly said, Theaetetus, and if I thought that you were one of those who would hereafter change your mind, I would have gently argued with you, and forced you to assent; but as I perceive that you will come of yourself and without any argument of mine, to that belief which, as you say, attracts you, I will not forestall the work of time. Let me suppose, then, that things which are said to be made by nature are the work of divine art, and that things which are made by man out of these are works of human art. And so there are two kinds of making and production, the one human and the other divine.

\par \textbf{THEAETETUS}
\par   True.

\par \textbf{STRANGER}
\par   Then, now, subdivide each of the two sections which we have already.

\par \textbf{THEAETETUS}
\par   How do you mean?

\par \textbf{STRANGER}
\par   I mean to say that you should make a vertical division of production or invention, as you have already made a lateral one.

\par \textbf{THEAETETUS}
\par   I have done so.

\par \textbf{STRANGER}
\par   Then, now, there are in all four parts or segments—two of them have reference to us and are human, and two of them have reference to the gods and are divine.

\par \textbf{THEAETETUS}
\par   True.

\par \textbf{STRANGER}
\par   And, again, in the division which was supposed to be made in the other way, one part in each subdivision is the making of the things themselves, but the two remaining parts may be called the making of likenesses; and so the productive art is again divided into two parts.

\par \textbf{THEAETETUS}
\par   Tell me the divisions once more.

\par \textbf{STRANGER}
\par   I suppose that we, and the other animals, and the elements out of which things are made—fire, water, and the like—are known by us to be each and all the creation and work of God.

\par \textbf{THEAETETUS}
\par   True.

\par \textbf{STRANGER}
\par   And there are images of them, which are not them, but which correspond to them; and these are also the creation of a wonderful skill.

\par \textbf{THEAETETUS}
\par   What are they?

\par \textbf{STRANGER}
\par   The appearances which spring up of themselves in sleep or by day, such as a shadow when darkness arises in a fire, or the reflection which is produced when the light in bright and smooth objects meets on their surface with an external light, and creates a perception the opposite of our ordinary sight.

\par \textbf{THEAETETUS}
\par   Yes; and the images as well as the creation are equally the work of a divine hand.

\par \textbf{STRANGER}
\par   And what shall we say of human art? Do we not make one house by the art of building, and another by the art of drawing, which is a sort of dream created by man for those who are awake?

\par \textbf{THEAETETUS}
\par   Quite true.

\par \textbf{STRANGER}
\par   And other products of human creation are also twofold and go in pairs; there is the thing, with which the art of making the thing is concerned, and the image, with which imitation is concerned.

\par \textbf{THEAETETUS}
\par   Now I begin to understand, and am ready to acknowledge that there are two kinds of production, and each of them twofold; in the lateral division there is both a divine and a human production; in the vertical there are realities and a creation of a kind of similitudes.

\par \textbf{STRANGER}
\par   And let us not forget that of the imitative class the one part was to have been likeness-making, and the other phantastic, if it could be shown that falsehood is a reality and belongs to the class of real being.

\par \textbf{THEAETETUS}
\par   Yes.

\par \textbf{STRANGER}
\par   And this appeared to be the case; and therefore now, without hesitation, we shall number the different kinds as two.

\par \textbf{THEAETETUS}
\par   True.

\par \textbf{STRANGER}
\par   Then, now, let us again divide the phantastic art.

\par \textbf{THEAETETUS}
\par   Where shall we make the division?

\par \textbf{STRANGER}
\par   There is one kind which is produced by an instrument, and another in which the creator of the appearance is himself the instrument.

\par \textbf{THEAETETUS}
\par   What do you mean?

\par \textbf{STRANGER}
\par   When any one makes himself appear like another in his figure or his voice, imitation is the name for this part of the phantastic art.

\par \textbf{THEAETETUS}
\par   Yes.

\par \textbf{STRANGER}
\par   Let this, then, be named the art of mimicry, and this the province assigned to it; as for the other division, we are weary and will give that up, leaving to some one else the duty of making the class and giving it a suitable name.

\par \textbf{THEAETETUS}
\par   Let us do as you say—assign a sphere to the one and leave the other.

\par \textbf{STRANGER}
\par   There is a further distinction, Theaetetus, which is worthy of our consideration, and for a reason which I will tell you.

\par \textbf{THEAETETUS}
\par   Let me hear.

\par \textbf{STRANGER}
\par   There are some who imitate, knowing what they imitate, and some who do not know. And what line of distinction can there possibly be greater than that which divides ignorance from knowledge?

\par \textbf{THEAETETUS}
\par   There can be no greater.

\par \textbf{STRANGER}
\par   Was not the sort of imitation of which we spoke just now the imitation of those who know? For he who would imitate you would surely know you and your figure?

\par \textbf{THEAETETUS}
\par   Naturally.

\par \textbf{STRANGER}
\par   And what would you say of the figure or form of justice or of virtue in general? Are we not well aware that many, having no knowledge of either, but only a sort of opinion, do their best to show that this opinion is really entertained by them, by expressing it, as far as they can, in word and deed?

\par \textbf{THEAETETUS}
\par   Yes, that is very common.

\par \textbf{STRANGER}
\par   And do they always fail in their attempt to be thought just, when they are not? Or is not the very opposite true?

\par \textbf{THEAETETUS}
\par   The very opposite.

\par \textbf{STRANGER}
\par   Such a one, then, should be described as an imitator—to be distinguished from the other, as he who is ignorant is distinguished from him who knows?

\par \textbf{THEAETETUS}
\par   True.

\par \textbf{STRANGER}
\par   Can we find a suitable name for each of them? This is clearly not an easy task; for among the ancients there was some confusion of ideas, which prevented them from attempting to divide genera into species; wherefore there is no great abundance of names. Yet, for the sake of distinctness, I will make bold to call the imitation which coexists with opinion, the imitation of appearance—that which coexists with science, a scientific or learned imitation.

\par \textbf{THEAETETUS}
\par   Granted.

\par \textbf{STRANGER}
\par   The former is our present concern, for the Sophist was classed with imitators indeed, but not among those who have knowledge.

\par \textbf{THEAETETUS}
\par   Very true.

\par \textbf{STRANGER}
\par   Let us, then, examine our imitator of appearance, and see whether he is sound, like a piece of iron, or whether there is still some crack in him.

\par \textbf{THEAETETUS}
\par   Let us examine him.

\par \textbf{STRANGER}
\par   Indeed there is a very considerable crack; for if you look, you find that one of the two classes of imitators is a simple creature, who thinks that he knows that which he only fancies; the other sort has knocked about among arguments, until he suspects and fears that he is ignorant of that which to the many he pretends to know.

\par \textbf{THEAETETUS}
\par   There are certainly the two kinds which you describe.

\par \textbf{STRANGER}
\par   Shall we regard one as the simple imitator—the other as the dissembling or ironical imitator?

\par \textbf{THEAETETUS}
\par   Very good.

\par \textbf{STRANGER}
\par   And shall we further speak of this latter class as having one or two divisions?

\par \textbf{THEAETETUS}
\par   Answer yourself.

\par \textbf{STRANGER}
\par   Upon consideration, then, there appear to me to be two; there is the dissembler, who harangues a multitude in public in a long speech, and the dissembler, who in private and in short speeches compels the person who is conversing with him to contradict himself.

\par \textbf{THEAETETUS}
\par   What you say is most true.

\par \textbf{STRANGER}
\par   And who is the maker of the longer speeches? Is he the statesman or the popular orator?

\par \textbf{THEAETETUS}
\par   The latter.

\par \textbf{STRANGER}
\par   And what shall we call the other? Is he the philosopher or the Sophist?

\par \textbf{THEAETETUS}
\par   The philosopher he cannot be, for upon our view he is ignorant; but since he is an imitator of the wise he will have a name which is formed by an adaptation of the word sophos. What shall we name him? I am pretty sure that I cannot be mistaken in terming him the true and very Sophist.

\par \textbf{STRANGER}
\par   Shall we bind up his name as we did before, making a chain from one end of his genealogy to the other?

\par \textbf{THEAETETUS}
\par   By all means.

\par \textbf{STRANGER}
\par   He, then, who traces the pedigree of his art as follows—who, belonging to the conscious or dissembling section of the art of causing self-contradiction, is an imitator of appearance, and is separated from the class of phantastic which is a branch of image-making into that further division of creation, the juggling of words, a creation human, and not divine—any one who affirms the real Sophist to be of this blood and lineage will say the very truth.

\par \textbf{THEAETETUS}
\par   Undoubtedly.

\par 
 
\end{document}